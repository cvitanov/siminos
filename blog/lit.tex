% siminos/blog/lit.tex
% $Author: predrag $ $Date: 2017-08-09 21:37:57 -0400 (Wed, 09 Aug 2017) $

\chapter{Literature}
\label{c:lit}

\begin{description}

\item[2012-01-10 Predrag] Collecting papers to read, and our notes on them in
this chapter.

\end{description}

\section{Heteroclinic connections literature}

Do literature review of algorithms to
compute \hec s.

\begin{description}

\item[2008-04-22 ES]
{\bf Locating \hec s}

I recently tried to locate \hec s in Lorenz equations. One
can easily see that the 2-dimensional unstable manifold of $\EQV{1}$ intersects the
$2-$dimensional stable manifold of $\EQV{0}$ and thus there should be such connections.
As I couldn't find Predrag's secret method documented somewhere I followed a simple
shooting approach.

Although a heteroclinic orbit is an infinite-time orbit it is sufficient to
pin down a finite time segment of the orbit originating at the linear unstable subspace $E_u^{(1)}$
of $\EQV{1}$ and ending at the linear stable subspace $E_s^{(0)}$ of $\EQV{0}$. Those
requirements can be expressed as the boundary value problem:
\bea
   x(0) & = & \EQV{1}+ \epsilon \Re(\mathbf{e}_1^{(1)}) \, \label{eq:shootHetIC} \\
   P_1^{(0)} (x(T)-\EQV{0}) & = &  0 \,. \label{eq:shootHetBC}
%% P_2^{(0)} (x(T)-\EQV{0}) & = &  d_2 \,. \label{eq:shootHetFixT} %% Not needed for Lorenz
\eea

Eq. \refeq{eq:shootHetIC} imposes the requirement that we start on the unstable subspace of $\EQV{1}$.
Alternatively we could have used as a search space a circle of radius $\epsilon$ on the plane defined
by orthonormalizing $\left(\Re(\mathbf{e}_1^{(1)}),\Im(\mathbf{e}_1^{(1)})\right)$, parametrized by some
angle $\theta$. Eq. \refeq{eq:shootHetBC} imposes the condition that the final point on the trajectory is
on the stable manifold of $\EQV{0}$. Here
\beq
   P_j^{(0)}= \prod_{i\neq j}^d \frac{\mathbf{A}(\EQV{0})-\lambda_i^{(0)} \mathbf{1}}{\lambda_j^{(0)}-\lambda_i^{(0)}}\,,
\eeq
is projection operator on the $j$'th eigendirection of the linear stability matrix $\mathbf{A}$ at $\EQV{0}$.
%Condition \refeq{eq:shootHetFixT} needs to be imposed due to the time-translational invariance of the
%equations. It corresponds to restricting the final point on a \PoincSec\ $\PoincS$ transverse to
%the least contracting eigendirection of $\mathbf{A}(\EQV{0})$. For Lorenz this would be a plane normal
%to the $z$-axis at distance $d_2$ from $\EQV{0}$.
\ES{According to Predrag's notation the left hand side
of equation \refeq{eq:shootHetBC} %%and \refeq{eq:shootHetFixT}
is a scalar.}

To solve the boundary value problem I have used Newton's method to refine a guess for $\epsilon_n$. %and $T_n$
%based on the linearization:
%\beq
%  f^{T_{n+1}}(x_{n+1}) \simeq f^{T_n}(x_n) + \mathbf{J}^{T_n}(x_n) \Re(\mathbf{e}_1^{(1)})\delta\epsilon\,, %+ v(f^{T_n}(x_n)) \delta T
%\eeq
%and imposing condition \refeq{eq:shootHetBC} % and \refeq{eq:shootHetFixT}
%on $f^{T_{n+1}}(x_{n+1})$.
Predrag suggests using the linear approximation of the flow to analytically continue the heteroclinic
orbits after period $T$ and impose the condition that this analytic solution ends up on the equilibrium.
I cannot see why this is necessary here. The condition \refeq{eq:shootHetBC} guarantees that the solution
is on the linear approximation of the stable manifold of $\EQV{0}$ and thus will end up on $\EQV{0}$.
Essentially this is the analytic part of the problem and no more needs to be done. The accuracy of the
method is limited by the approximation of the local stable manifold of $\EQV{0}$ and the local unstable
manifold of $\EQV{1}$ by the corresponding linear unstable subspaces, \ie, by planes in the case of Lorenz
equations. One can estimate the distance of minimum approach to $\EQV{0}$ by using the expressions for
the linearized flow. Disregarding the strongly contracting eigendirection $e_3^{(0)}$ I find:
\beq
   r_{min}^2= d_2^2\left(1-\frac{\lambda_2}{\lambda_1}\right)\left(\-\frac{d_1^2 \lambda_1}{d_2^2\lambda_2}\right)^{-\frac{\lambda_2}{\lambda_1-\lambda_2}}
\eeq
where $d_1=P_1^{(0)} (x(T)-\EQV{0})$ and $d_2=P_2^{(0)} (x(T)-\EQV{0})$ \ES{Predrag might want to compare
this against his secret notes.}. I use this to compare with the actual minimum distance from $\EQV{0}$ and
evaluate the validity of the linear approximation of the stable manifold.

In \refref{FriedmanDoedelConnections91} the authors present a
method for computation and continuation of \hec s
similar in spirit but in a more general setting.
They suggest that a condition breaking the time-translational
invariance should be used along with \refeq{eq:shootHetBC}.
Here, \refeq{eq:shootHetIC} is formulated in a way that
excludes variation in the initial conditions in the direction
of the flow and we need not impose an extra condition.

It works well for Lorenz equations but fails to converge for ZM.


\item[2015-02-16 Predrag]
Are
``\HREF{http://meetings.siam.org/sess/dsp\_programsess.cfm?SESSIONCODE=6204}
{Robust Heteroclinic Cycles}" relevant to our \hec s?

\item[2014-05-21 ]
A recent review of `freezing': {\em Stability and
    Computation of Dynamic Patterns in PDEs}, by Beyn, Otten and
    Rottmann-Matthes\rf{BeOtRo14}
    (\HREF{http://ChaosBook.org/library/BeOtRo14.pdf} {click here}).

\item[2016-07-27 Predrag] If interested in computation of stable/unstable
manifolds: H\"uls\rf{Huls16,BeyHuls14} {\em On the approximation of stable and
unstable fiber bundles of (non)autonomous ODEs -- A contour algorithm}
has some nice plots.


\item[2017-03-03 Predrag]
Pascal Chossat, Alexander Lohse and Olga Podvigina
{\em Pseudo-simple heteroclinic cycles in $R^4$},
\arXiv{1702.08731},
looks worth a look. They write: ``
  We study pseudo-simple heteroclinic cycles for a $\Gamma$-equivariant system
in $R^4$ with finite $\Gamma\subset O(4)$, and their nearby dynamics. In
particular, in a first step towards a full classification - analogous to that
which exists already for the class of simple cycles - we identify all finite
subgroups of $O(4)$ admitting pseudo-simple cycles. To this end we introduce a
constructive method to build equivariant dynamical systems possessing a robust
heteroclinic cycle. Extending a previous study we also investigate the
existence of periodic orbits close to a pseudo-simple cycle, which depends on
the symmetry groups of equilibria in the cycle. Moreover, we identify subgroups
$\Gamma\subset O(4)$, $\Gamma\not\subset SO(4)$, admitting fragmentarily
asymptotically stable pseudo-simple heteroclinic cycles. (Recall, that for
$\Gamma\subset SO(4)$ pseudo-simple cycles generically are completely
unstable.) Finally, we study a generalized heteroclinic cycle, which involves a
pseudo-simple cycle as a subset.
    '
\item[2017-08-09 Predrag]
Downloaded
Haro, Canadell, Luque, Mondelo, and Figueras\rf{HCFLM16}
{\em The Parameterization Method for Invariant Manifolds},
might come in handy.


\end{description}




\section{Symmetry literature}
\label{sec:symmLit}

\begin{description}

\item[2015-02-16 Predrag]
% Predrag 2013-04-20 moved to here from DOGS/saldana/blog.tex
% \section{Freezing solutions of equivariant evolution equations}
% \label{s:BeTh04}
Worth a read: Beyn and Th\"ummler\rf{BeTh04}. Their papers are
available under `Preprints' link on
\HREF{http://www.math.uni-bielefeld.de/~beyn/AG_Numerik/html/en/preprints/index.html}
{Beyn website}. Read also my notes in \refchap{c-excite}, and the
Chaosook   refrem {rem:SSLieGr} ``A brief history of relativity.''

\item[2010-06-05 Predrag]
I have read Beyn and Th\"ummler\rf{BeTh04} and believe ``freezing'' =
``slicing''. Siminos and I have Hermann and Gottwald\rf{HerGot10} on our
reading list, but have not studied it yet. Beyn and Th\"ummler impose a
slice by adding a new dimension (the phase parameter) and a condition (a
Lagrange multiplier), while we go one dimension down by restricting the
dynamics into the slice. Both approaches are standard in imposing
\PoincSec s; Rytis Paskauskas has written that up for Chapter 3 -
Discrete time dynamics (not yet on the public version), and Chapter 13 -
Fixed points, and how to get them, Section 13.4 Flows. My feeling is that
Beyn and Th\"ummler is better, but we have not implemented it.

\item[2009-12-23 Predrag]
    \PC{this entry is repeated around \refeq{parabolicPDE}, keep only one}
Went to Bielefeld, talked to
\HREF{http://www.math.uni-bielefeld.de/~beyn/AG_Numerik/html/en/people/}
{Wolf-J\"urgen Beyn}. Beyn
and his former PhD student Vera Th\"ummler\rf{BeTh04,Thum05} `freeze'
traveling waves. Their papers are mathematical, proving existence of spectra etc.,
but also applied to numerical examples. Scalar Nagumo equation is a good
testing ground, as one of the traveling wave solutions is known
analytically.

Beyn did his `freezing' at the same time as Rowley and
Marsden\rf{rowley_reduction_2003}, but they published first (a
year earlier?), so \refref{BeTh04} cites their `reconstruction
equation.' Beyn does not understand the `method of
connections,' either.

To understand why they call it `freezing' the figures are useful,
and especially the animations, like
\HREF{http://www.mathematik.uni-bielefeld.de/fgweb/Preprints/movies/fg03022_02.mpg}
{this one}. Of course, a turbulent state will not be 'frozen,'
but much of the hysterical fast drifting will be gone.

Beyn group\rf{BeTh04}  research focuses on the numerical
computation and stability of traveling wave solutions (and more
generally relative equilibria) of reaction-diffusion systems on
unbounded domains, \ie, of parabolic partial differential
equations (PDEs). On the real line they are of the form
\beq
u_t = A \, u_{xx} + f(u,u_x)
    \,,\quad
u: \reals \times [0,\infty) \to \reals
    \,,
A \in \reals^{d,d}
\,.
\ee{parabolicPDE1}
A traveling wave is a special type of
a relative equilibrium of equivariant evolution equations,
where the action is given by translation,
   % \PC{copied this to ChaosBook}
\beq
\LieEl(\velRel) \, u(x)
  = \hat{u}(x - \velRel t)\,,\quad  \velRel \in \reals^d
\,,
\ee{BeThTW1}
$\hat{u}(x)$ is the waveform and $\velRel$ the velocity. In the
comoving frame equations \refeq{parabolicPDE1} take form
\beq
\hat{u}_t
 = A\, \hat{u}_{xx} + \velRel \, \hat{u}_x + f(\hat{u},\hat{u}_x)
    \,,\quad
\hat{u}_x \in \reals
    \,,\;
t \ge 0
\,,
\ee{comovingPDE1}
with $\hat{u}(x)$ a stationary solution (the `freezing of the wave')
\beq
0= A \, \hat{u}{''} + \velRel \, \hat{u}{'} + f(\hat{u},\hat{u}{'})
\ee{BeThREQV1}
The pair $(\hat{u},\velRel$) is an equilibrium solution of the
partial differential algebraic equation (PDAE)
\refeq{BeThREQV1} which is constructed by inserting the co-moving
frame ansatz into PDE and adding an {\em additional phase
condition}. They say: ``By transforming the PDE into a
corresponding PDAE (partial differential algebraic equation)
via a freezing ansatz\rf{BeTh04} the relative equilibrium can
be analyzed as a stationary solution of the PDAE.''

Beyn finds Sandstede \etal\
articles\rf{FiSaScWu96,SaScWu97,SaScWu99a,FiTu98}
`remarkable.'
Sandstede \etal\rf{SaScWu97} use center manifold reduction
theory to study stability of \reqva. The main idea is to
transform the flow into a `skew product' form. One part is
orthogonal to the group orbit of the initial field $u(0)$, and
the other part acts within the group orbit and depends upon the
position in the orthogonal direction.

{\bf ES 2009-12-23} This decomposition is important in Krupa\rf{Krupa90} study
of bifurcations of relative equilibria, using a local slice. I think
he also introduces the tubular neighborhood. The book by Chossat
and Lauterbach\rf{ChossLaut00} gives a readable presentation
and I need it to compare their stability analysis for \reqva\ with
our Appendix, but I can't find it here.

%%%%%%%%%%%%%%%%%%%%%%%%%%%%%%%%%%%%%%%%%%%%%%%%%%
\SFIG{BeThEquiTraj}
{}{
Two trajectories
$\ssp(t)$, $\sspRed(t)$ are equivalent up to a group rotation
$\LieEl(t)$ as long as they belong to the same group orbit
$\pS_{\ssp(t)}$.
}
{fig:BeThEquiTraj}
%%%%%%%%%%%%%%%%%%%%%%%%%%%%%%%%%%%%%%%%%%%%%%%%%%
%
Beyn~\etal\rf{BeTh04} find it more convenient to make
use of the equivariance by extending the system rather than
reducing it as in bifurcation analysis, by adding an additional
parameter and a phase condition. They too write the solution as
${u}(t) = \LieEl(t)\,\hat{u}(t)$, see
\reffig{fig:BeThEquiTraj}, a composition of the action of a
time dependent group element $\LieEl(t)$ with a `frozen
solution' $\hat{u}(t)$ in a given Banach space, with
$\LieEl(t)$, $\hat{u}(t)$ to be determined. They keep the
`frozen solution' as constant as
possible%
\PC{`As constant as
possible' makes sense only for \reqva, not \rpo s which they do
not discuss.} by introducing a set of algebraic constraints
(phase conditions), $\psi(\hat{u}, \gSpace) = 0$, which fix the
extra degrees of freedom. Their number is given by the
dimension of the Lie group.


The freezing approach\rf{BeTh04} applies to traveling
waves and, more generally, to \reqva\rf{ChossLaut00,SaScWu99},
solutions which are
equilibria in an appropriately co-moving frame. They occur
in systems with underlying symmetry, such as
rotating waves on the real line and spiral waves in two space
dimensions. An example of a system with `rotating waves on the
real line' is the complex Ginzburg Landau equation, while
spirals are typical of the reaction-diffusion systems.

Consider an evolution equation in a Banach space $\pS$  of the form
\beq
u_t = \vel(u)
\ee{BeThEvEq1}
with an equivariant right hand side $\vel$, i.e.
$\LieEl\,\vel(u) = \vel(\LieEl\,u)$ where $\LieEl : \Group \to
GL(\pS), \gSpace \to \LieEl(\gSpace)$ denotes the action of a
Lie group $\Group$ on  $\pS$. Here --in the parlance of applied
mathematicians-- $\vel$ is defined on a dense subspace of some
Banach space (generally infinite dimensional) and is
equivariant with respect to the action of a finite dimensional
(not necessarily compact) Lie group. In other words, the
$\infty$-dimensional functional PDE \statesp\ is spanned by
nicely shaped waveforms $u(x)$, nothing too kinky.

The equation \refeq{BeThEvEq1}
can be transformed via the ansatz
$u(t) = \LieEl(t)\hat{u}(t)$ into the equivalent system
\beq
\hat{u}_t = \vel(\hat{u})
  - \LieEl(\gSpace)^{-1} \LieEl_\gSpace(\gSpace) \, \hat{u} \lambda
    \,,\qquad
\lambda = \gSpace_t
\,,
\ee{BeThReconstr1}
where subscripted quantities imply partial derivatives with
respect to the subscript. Beyn~\etal\rf{BeTh04} then `freeze'
the traveling wave by fixing a \slice, and using a dot product
w.r.t. the group tangent at the slice point. The evolution of
$\gSpace(t)$ describes the motion on the group manifold. They
denote by $T_\gSpace \Group$ the tangent space of $\Group$ at
$\gSpace$. Introducing Lagrange parameters $\lambda(t) =
\LieEl_t(t) \in T_\gSpace \Group$ they impose `freezing ansatz'
on the extended system of equations \refeq{BeThReconstr1}
\beq
\hat{u}_t = \vel(\hat{u})
  - \LieEl(\gSpace)^{-1} \LieEl_\gSpace(\gSpace) \, \hat{u} \lambda
    \,,\qquad
0 = \pi(\hat{u},\lambda)
\,,
\ee{BeThSlice1}
with a ``phase condition
$\pi : \pSRed \times T_\gSpace \Group \to \reals^N$,
$N= \mbox{dim }\Group$,
which has to satisfy some regularity conditions.''
They differentiate
$
h \circ \LieEl \, \hat{u}
$
with respect to $\LieEl$, define group tangent at unity
$\groupTan \in \times T_1 \Group$, $\lambda = d\gSpace_l(1)
\groupTan$ ($l$ for left multiplication), and with some algebra
I'm too bored to type arrive at the extra equation(s)
\[
 0= \psi(\hat{u},\groupTan) = \pi(\hat{u},d\gSpace_l(1) \groupTan)
\,,
\]
which - if I've decoded it right - is the ChaosBook slice condition.

Their `freezing ansatz'\rf{BeTh04} appears to be identical
to the Rowley and Marsden\rf{rowley_reduction_2003} and our
slicing, except that `freezing' is formulated as an
additional constraint, just as when we compute periodic
orbits of ODEs we add \PoincSec\ as an additional
constraint, \ie, increase the dimensionality of the problem
by 1 for every continuous symmetry. They prefer it this
way, as they are taking derivatives. They know that the
slice is local and things can diverge.
                                    \toCB

They illustrate `freezing' by numerical computations for the
quintic complex Ginzburg Landau equation (QCGL), which is
equivariant w.r.t. the action of the group $(\LieEl_r,\LieEl_t)
\in \Group = S^1 \times \reals$ on $u(x) \in \reals^2$. The
action is given by translation in the domain and rotation in
the image, i.e.
    \PC{copy to continuous.tex - an example?}
\bea
\LieEl \, u(x) &=& R_{\LieEl_r^{-1}} u(x - \LieEl_t)
    \,,\qquad
R_{\LieEl_r^{-1}} =
   \left(\barr{cc}
   \cos\theta  &  \sin\theta   \\
  -\sin\theta  &  \cos\theta
   \earr\right)
 \,,
 \label{QCGLrotation}
\eea
where subscripts are now just subscripts,
$r$ implying rotation and $t$ implying translation.

They refer to Chap.~8 of Govaerts\rf{Govaerts00} for a review
of numerical methods that employ equivariance with respect to
compact, and mostly discrete groups.
    \PC{copy to discrete.tex}                          \toCB
% Predrag 2013-04-20: end of text moved to here from DOGS/saldana/blog.tex

\item[2013-12-02 Predrag] notes on Shapere \& Wilczek\rf{ShWi06},
microswimmers, \etc,
are in \refchap{c:gauge}.


\item[2012-02-25 Predrag]
\HREF{http://www.massey.ac.nz/~rmclachl/}
{McLachlan} \etal\rf{McLPerlQui03} write in
{\em Lie group foliations: dynamical systems and integrators}
(I added it to Zotero): ``
Foliate systems are those which preserve some (possibly
singular) foliation of phase space, such as systems with integrals,
systems with continuous symmetries, and skew product systems. We study
numerical integrators which also preserve the foliation.
\toCB

\begin{quote}
The case in
which the foliation is given by the orbits of an action of a Lie group
has a particularly nice structure, which we study in detail, giving
conditions under which all foliate vector fields can be written as the
sum of a vector field tangent to the orbits and a vector field invariant
under the group action.
\end{quote}

This allows the application of many techniques of
geometric integration, including splitting methods and Lie group
integrators.
''

\item[2012-03-26 Predrag]
In \emph{Heteroclinic orbits in a spherically invariant system}
Armbruster and Chossat\rf{ArCho91} look at some bifurcations with
\On{3}-symmetry, might be a model to try slicing on in a non-abelian
setting.

\item[2012-06-13 Predrag]
Fran�ois Gay-Balmaz and Darryl D. Holm
{\em Parameterizing interaction of disparate scales: Selective decay by
Casimir dissipation in fluids}, \arXiv{1206.2607} say: ``
The problem of parameterizing the interactions of disparate scales in
fluid flows is addressed by considering a property of two-dimensional
incompressible turbulence. The property we consider is selective decay,
in which a Casimir of the ideal formulation (enstrophy in 2D flows)
decays in time, while the energy stays essentially constant. This paper
introduces a mechanism that produces selective decay by enforcing Casimir
dissipation in fluid dynamics. [...] a general theory of selective decay
is developed that uses the Lie-Poisson structure of the ideal theory.
[...] may be useful in turbulent geophysical flows where it is
computationally prohibitive to rely on the slower, indirect effects of a
realistic viscosity, such as in large-scale, coherent, oceanic flows
interacting with much smaller eddies.

[...] From the viewpoint of Noether's theorem, energy is conserved in
ideal fluid dynamics because of the time-translation symmetry of the
Lagrangian in Hamilton's principle for ideal fluid motion. A second type
of fluid conservation law arises via Noether's theorem because of
relabelling symmetry of the Lagrangian. Relabelling symmetries smoothly
transform the labels of the fluid parcels without changing the Eulerian
quantities on which Hamilton's principle for ideal fluids depends. The
conservation laws associated with relabelling symmetries are called
Casimirs, because in the Hamiltonian formulation of ideal fluid dynamics
in the Eulerian representation their Lie-Poisson brackets with any other
functionals vanish identically. Thus, the Casimirs arise from a geometric
symmetry of the Eulerian representation of ideal fluid dynamics. This
relabelling symmetry is also responsible for Kelvin's circulation theorem
in ideal fluid dynamics, which immediately leads to the conservation of
the Casimirs for 2D ideal incompressible flow.

The Kelvin circulation theorem and the associated Casimir conservation
laws are kinematic, because they hold for any choice of Hamiltonian in
the Eulerian representation. Energy conservation is dynamic.

[...] The two types of ideal fluid constants of motion, energy and
Casimirs, typically have quite different dependencies on spatial
gradients of the solutions. Consequently, the interplay between them can
be interpreted as an interaction between larger and smaller scales (or
coherence lengths, or spectral wavenumbers).
''

It is all 2D Eulerian, and mathematics is heavy.

\item[2013-11-23  Predrag] Moved all matters Gribov  to here to
\\
\texttt{gitHub.com} \texttt{reducesymm/QFT/Gribov.tex};
if you have something to
repost, blog it there, please.

\item[2013-06-17 Predrag] Possibly of interest -
{\em Relative equilibria and relative periodic solutions in systems with
  time-delay and $S^{1}$ symmetry}, by Serhiy Yanchuk and Jan Sieber,
  \arXiv{1306.3327}:
``
 We study properties of basic solutions in systems with dime delays and
$S^1$-symmetry. Such basic solutions are relative equilibria (CW solutions) and
relative periodic solutions (MW solutions). It follows from the previous theory
that the number of CW solutions grows generically linearly with time delay
$\tau$. Here we show, in particular, that the number of relative periodic
solutions grows generically as $\tau^2$ when delay increases. Thus, in such
systems, the relative periodic solutions are more abundant than relative
equilibria. The results are directly applicable to, e.g., Lang-Kobayashi model
for the lasers with delayed feedback. We also study stability properties of the
solutions for large delays.
''
                                            \toCB
\Reqva\ they call \emph{continuous waves} (CW), and \rpo s they call
\emph{Modulated waves} (MW).

\item[2013-06-17 Predrag] A bit of surrealistic reading. Our new
graduate student, Kimberly Short, likes going to libraries, so I
showed the Georgia Tech library.
Big mistake - checked out a pile of book, including
the {1995} {\em Nonlinear Mechanics, Groups and Symmetry}, by
Mitropolsky and Lopatin\rf{MitLop95} from Ukraine. Sounds like what
we do, right? They have a huge reference list mostly (post) Soviet -
only author I recognize is Marsden, mentioned at the very end of the book:
``In conclusion of this survey (375 pp. is ``short?''), ...''


I think it is all about symmetry in perturbation problems, for which
the unperturbed problem is linear. In Sect.~1.4.1 they define
                                \toCB
`invariant system'
\beq
\dot{\ssp'} = \vel{\ssp'}
\ee{MitLop95-1.80}
to be a system invariant under 1-parameter subgroup of transformation
of group $\Group$, of form
$\ssp = g(\ssp')= \exp(s\Lg)\ssp'$. The vector field associated with
time evolution \refeq{MitLop95-1.80}, group transformation are
respectively
\[
\mathbf{X} = \sum_i\vel_i\frac{\pde~}{\pde x_i}
\,\qquad
\mathbf{U} = \sum_i\Lg_i\frac{\pde~}{\pde x_i}
\,.
\]
Their Theorem 1.9 states that system is invariant if the two vector
fields commute,
\(
[\mathbf{X},\mathbf{U}]
\,.
\)
The do not use 'Lie derivative' or other such western terms, though
they do refer to Campbell-Hausdorff formula as `Campbell-Hausdorff'. The
theory relies heavily on Campbell-Hausdorff expansions, whose
use seems to be due to
\HREF{https://en.wikipedia.org/wiki/Nikolay_Bogolyubov} {N. N.
Bogoliubov}\rf{KryBog47,BogMit61}, the same one we know from quantum
field theory. Krylov subspaces, however, refer to another
\HREF{http://en.wikipedia.org/wiki/Alexei_Krylov} {Krylov}.

\item[2013-09-17 Predrag]
check out
J\k{e}drzej \'Sniatycki\rf{Sniatycki13},
 \emph{Differential Geometry of Singular Spaces and Reduction of Symmetry}.
 The blurb says:
``illustrates the power of the theory of
subcartesian differential spaces for investigating spaces with
singularities. Part I gives a detailed and comprehensive presentation of
the theory of differential spaces, including integration of distributions
on subcartesian spaces and the structure of stratified spaces. Part II
presents an effective approach to the reduction of symmetries. Concrete
applications covered in the text include reduction of symmetries of
Hamiltonian systems, non-holonomically constrained systems, Dirac
structures, and the commutation of quantization with reduction for a
proper action of the symmetry group.''

\item[2014-01-11 Predrag]
I always wonder whether we should be reducing symmetries
by averaging over group orbits (method of characters, used in
my derivation of the {\Fd} in presence of continuous symmetries).
Churchill, Kummer and Rod\rf{ChKuRo83} write in
{\em On averaging, reduction, and symmetry in {Hamiltonian} systems}:
The existence of periodic orbits for Hamiltonian systems at
low positive energies can be deduced from the existence of nondegenerate
critical points of an averaged Hamiltonian on an associated ``reduced
space.'' The paper exploits discrete symmetries, including reversing
diffeomorphisms, that occur in a given system. The symmetries are used to
locate the periodic orbits in the averaged Hamiltonian, and thence in the
original Hamiltonian when the periodic orbits are continued under
perturbations admitting the same symmetries."

\item[2014-04-18 Predrag]
Dresselhaus \etal\ textbook\rf{Dresselhaus07}
(\HREF{http://chaosbook.org/library/Dresselhaus07.pdf}{click here})
is good on discrete
and space (but not continuous) groups.
The MIT~course~6.734
\HREF{http://stuff.mit.edu/afs/athena/course/6/6.734j/www/group-full02.pdf}
{online version} contains much of the same material.

Chapter {\em 9. Space Groups in Real Space} is quite clear on matrix
representation of space groups. The translation group $T$ is a normal
subgroup of \Group\ and defines the Bravais lattice. The cosets by
translation $T$ (set all all group elements obtained by all translations)
form a factor group $\Group/T$, isomorphic to the point group $g$
(rotations). All irreducible representations of \Group\ can be compounded
from irreducible representations of $g$ and $T$.

Section {\em 9.3 Two-Dimensional Space Groups}: In the international
crystallographic notation, our hexagonal lattice \#17 is called $p6mm$,
with point group $6mm$.
\[
g = \{
E, C_6^+, C_6^-, C_3^+, C_3^-, C_2,
\sigma_{d1}, \sigma_{d2}, \sigma_{d3},
\sigma_{v1},\sigma_{v2}, \sigma_{v3}
\}
\]
Prefix $p$ indicates that the unit cell is primitive (not centered). This
is a simple or {\em symmorphic} group, which makes calculations easier.
The Bravais lattice is two equilateral triangles, not sure how to relate
it to our hexagonal `elementary cell'? A Brilloun zone? Bravais `unit cell'
is illustrated in Fig.~E.2. ChaosBook `Fundamental
domain' makes an appearance in Fig.~10.2.

The main trick in quantum-mechanical calculations is to go to the
\emph{reciprocal} space (see Fig.~E.2), in our case with the full
$\Gamma$ point, $k=0$, wave vector symmetry (see Table~10.1), and `Large
Representations'. This is something we have not tried in deriving the
trace formula for deterministic diffusion.

Sect. {\em 10.5 Characters for the Equivalence Representation} look
like those for the point group, sort of. We should probably work
out problems 10.1 and 10.2.

\item[2014-04-20 Predrag]
Mansfield\rf{Mansfield10}
{\em A practical guide to the invariant calculus},
(\HREF{http://ChaosBook.org/library/Mansfield10.pdf} {click here})
looks like a friendly introduction to moving frames. She discusses
special Euclidean group $SE(2)$ in some detail.

\item[2014-10-03 Francesco Fedele]
\HREF{http://www.math.umn.edu/~olver/} {Peter J. Olver} has a 2012 version of
\HREF{http://www.math.umn.edu/~olver/sm_/mflc.pdf} {{\em Lectures on
Moving Frames}}. And a boat called ``Aftermath''.

These are a part of his {\em Symmetry and Moving Frames}
\HREF{http://www.math.umn.edu/~olver/sm.html} {{\em Lecture Notes}}.
The {\em Moving Frames in Applications}
\HREF{http://www.math.umn.edu/~olver/sm_/mfslides.pdf} {slides} seem
accessible and a particularly useful reading. Our problem has been the
usual - it looks beautiful, but when it comes to our problems, we have
not been able to figure out how to use what he describes in his books.

Another interesting tidbit is Olver's
\HREF{http://www.math.umn.edu/~olver/plag.html} {plagiarism case}. In
a sense plagiarism of ChaosBook.org by the Serbian couple in Australia
is worse, as the publisher is Springer.

\item[2015-03-17 Predrag] Check out
Yu Jiang, Hexi Baoyin, Xianyu Wang and Hengnian Li,
\emph{Periodic Orbits, Chaos and
    Manifolds near the Equilibrium Points in the Rotating
    Plane-Symmetric Potential Field},
\arXiv{1403.1967}

\item[2015-09-15 Predrag] Kevrekidis article for Burak and Xiong to read:
V. Achilleos, A. R. Bishop, S. Diamantidis, D. J. Frantzeskakis, T. P.
  Horikis, N. I. Karachalios and P. G. Kevrekidis,
\emph{The dynamical playground of a higher-order nonlinear Schr\"odinger
  equation: from orbital connections and limit cycles to invariant tori and the
  onset of chaos}, \arXiv{1509.03828}. They write: ``
  The dynamical behavior of a higher-order nonlinear Schr\"odinger equation is
found to include a very wide range of scenarios due to the interplay of
higher-order physically relevant terms. The dynamics extends from
Poincar\'e-Bendixson--type scenarios, in the sense that bounded solutions may
converge either to distinct equilibria via orbital connections, or space-time
periodic solutions, to the emergence of almost periodic and chaotic behavior.
Suitable low-dimensional phase space diagnostics are developed and are used to
illustrate the different possibilities and to identify their respective
parametric intervals.''

\item[2015-09-29 Predrag]
Dhooge, and Govaerts and Kuznetsov\rf{DhGoKu03} {\em {MATCONT: A MATLAB}
package for numerical bifurcation analysis of {ODEs}} looks like a nice
piece of software for tracking bifurcations. I did not see from the
article how high can one realistically go in the \statesp\ dimension.
Available \HREF{http://sourceforge.net/projects/matcont/} {here}.
See also \refrefs{DWRGK13,Govaerts00,Kuzn04,homcont,auto,KuKuSa05}
Abstract:
``MATCONT is a graphical MATLAB software package for the interactive
numerical study of dynamical systems. It allows one to compute curves of
equilibria, limit points, Hopf points, limit cycles, period doubling
bifurcation points of limit cycles, and fold bifurcation points of limit
cycles. All curves are computed by the same function that implements a
prediction-correction continuation algorithm based on the Moore-Penrose
matrix pseudo-inverse. The continuation of bifurcation points of
equilibria and limit cycles is based on bordering methods and minimally
extended systems. Hence no additional unknowns such as singular vectors
and eigenvectors are used and no artificial sparsity in the systems is
created. The sparsity of the discretized systems for the computation of
limit cycles and their bifurcation points is exploited by using the
standard Matlab sparse matrix methods. The MATLAB environment makes the
standard MATLAB Ordinary Differential Equations (ODE) Suite interactively
available and provides computational and visualization tools; it also
eliminates the compilation stage and so makes installation
straightforward. Compared to other packages such as AUTO and CONTENT,
adding a new type of curves is easy in the MATLAB environment.''

\item[2015-09-29 Burak]
I used MATCONT initially for tracking bifurcations in \KS\ (see
\texttt{ksRecycled/tex/}), it has a graphical interface, which is user
friendly, but there were a lot of bugs and crashes, so I ended up coding
everything in python. I could have probably used command line interface of
MATCONT but I find working with ipython notebooks much more convenient.
In terms of dimensions, MATCONT does not have an upper limit and providing an
explicit stability matrix or higher order derivatives is optional. However, I
don't think it can handle dimensions, at which one would need Newton-Krylov
methods.

\item[2013-02-16 Predrag]
\HREF{https://www.math.uni-bielefeld.de/~beyn/AG_Numerik/html/en/preprints/sfb_13_020.html}
{This} might be of interest\rf{BeOtRo14}: ``Stability and
Computation of Dynamic Patterns in PDEs'', by Wolf-J\"urgen Beyn and
Denny Otten and Jens Rottmann-Matthes.

\item[2014-05-21 Predrag]
A recent review of `freezing': {\em Stability and
    Computation of Dynamic Patterns in PDEs}, by Beyn, Otten and
    Rottmann-Matthes\rf{BeOtRo14}
    (\HREF{http://ChaosBook.org/library/BeOtRo14.pdf} {click here}).

\item[2016-07-27 Roman]
I had a couple of useful discussions with Beyn during the conference in
Berlin. He has an alternative approach to local symmetry reduction (based
on partition of unity). There are some advantages and some
disadvantages to it, which might be worth pointing out. Either way, it
would be useful to do some comparison of his approach with tiling in
Marcotte thesis.

\end{description}
