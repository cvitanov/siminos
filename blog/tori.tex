% siminos/blog/tori.tex
% $Author: predrag $ $Date: 2016-06-01 04:44:20 -0400 (Wed, 01 Jun 2016) $

\chapter{Tori}
\label{c-tori}
% Predrag 2015-09-24

%\begin{description}
%
%\item[2012-11-12 Predrag]
%moved to here from \texttt{pipes/blog/tori.tex}.
%
%\end{description}

\section{Tori blog}

\bigskip\bigskip
\noindent
{\color{red} The latest blog entry at the end of this chapter}
\bigskip\bigskip

\begin{description}


    \PCpost{2015-09-15}{
(reposted here from \texttt{elton/blog/AdamBlog.tex})

Reading about robustness of invariant tori. Farazmand likes to refer to
Fenichel\rf{Feniche71} (to read it,
\HREF{http://chaosbook.org/library/Feniche71.pdf}{click here}). My
understanding is that if a stability exponent is purely imaginary, it
can destroy a torus at an rational resonance; but if it has a real part
(hyperbolic case), there is no way for the Floquet exponent to approach
the purely imaginary winding number of the torus, there can be no
resonance, and the torus remains smooth and robust for an open interval
of the system parameter values.

Figueras and Haro\rf{FigHar12} write:
``it has been known for a long time that persistence of invariant
manifolds is closely related to the concept of normal
hyperbolicity\rf{Feniche71}. %21
%34 M. W. Hirsch, C. C. Pugh, and M. Shub, Invariant manifolds,
%    Lecture Notes in Math. 583, Springer-Verlag, Berlin, 1977.
%50 R. Ma\~{n}\'e, Persistent manifolds are normally hyperbolic,
%   Trans. Amer. Math. Soc., 246 (1978), pp. 261-283.
%57 R. J. Sacker, A new approach to the perturbation theory of
%    invariant surfaces, Comm. Pure Appl. Math., 18 (1965), pp. 717-732.
We consider the analogous
concept, tailored for skew products over rotations. Roughly speaking, an
invariant torus is fiberwise hyperbolic if the linearized dynamics on the
normal bundle is exponentially dichotomous, that is, the normal bundle
splits into stable and unstable bundles on which the dynamics is
uniformly contracting and expanding, respectively. Notice that the
tangent dynamics is dominated by the normal dynamics, since the former
presents zero Lyapunov exponents. This implies that fiberwise hyperbolic
invariant tori are robust and are as smooth as the system [ 28 ].
% A. Haro and R. de la Llave, A parameterization method for the
% computation of invariant tori and their whiskers in quasi-periodic maps:
% Rigorous results, J. Differential Equations, 228 (2006), pp. 530-579.
''

Figueras' thesis might be an easier read:
\HREF{http://www2.math.uu.se/~figueras/preprints/files/phd_thesis/Jordi_LLuis_Figueras_Romero_PHD.pdf}
{click here}.
    }



\end{description}
