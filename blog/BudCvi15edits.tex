% siminos/blog/BudCvi15edits.tex
% $Author: predrag $ $Date: 2016-04-08 14:53:32 -0400 (Fri, 08 Apr 2016) $
% Predrag created Sep 15 2015

\section{BudCvi15 torus papers edits}

This section contains referee notes and material which has not been
included in \emph{Torus breakdown in the symmetry-reduced state space of
the Kuramoto-Sivashinsky system}\rf{BudCvi15}, {\tt siminos/ksTorus/} in
this repository,  and/or Budanur Ph.D. thesis. Please clean up whatever
seems extraneous.

\begin{description}

% this moved here from
% siminos/ksRecycled/tex/flotsam.tex    master file: ksRecycled.tex
\item[2014-01-11 Burak] Moved this paragraph here for possible later use:

Following \refref{SCD07}, we truncate the Fourier expansion at $N = 15$,
which yields a $30$ dimensional \statesp . We integrate \KSe\ numerically
using exponential time differencing 4th-order Runge-Kutta method (ETDRK4)
\rf{cox02jcomp,ks05com}. Details of the numerical integration in this \statesp\
representation can be found in \refref{SCD07}.

\item[2014-01-20 Burak] Moved this here for possible later use:

Our method for reducing the \SOn{2} symmetry is based on \mslices ,
where we define a \slicePlane
\beq
    (\sspRed (\zeit) - \slicep) \cdot \sliceTan{} = 0 \, ,
    \label{e-SliceCond}
\eeq
where \slicep\ is the \emph{slice template}, $\sliceTan{}$ is the \emph
{template tangent}, and $\sspRed(\zeit)$ is the symmetry reduced solution,
which satisfies the \emph{slice condition} \refeq{e-SliceCond}. Template
tangent $\sliceTan{} = \Lg \slicep$ is obtained by acting on the slice
template by the generator of infinitesimal \SOn{2} rotations, which for the
matrix representation \refeq{e-DSO2} can be written as
\beq
    \Lg = \mathrm{diag}\left[\,{}
        \Lg_1,\, \Lg_2,\, \ldots,\,
        \Lg_N \,\right]\,,
    \label{e-Lg}
\eeq
where
\beq
    T_k =  \begin{pmatrix}
        0 & -k   \\
        k & 0
            \end{pmatrix} \, .
\eeq
It is straightforward to check $\matrixRep (\LieEl (\theta)) = \exp \Lg
\theta$.

In general, the \slicePlane\ defined by \refeq{e-SliceCond} gives a local
symmetry-reduced description, valid as long as the group tangent $\groupTan
= \Lg \sspRed$ has a non-zero component, normal to the \slicePlane .
Budanur \etal\rf{BudCvi14} have recently made a key observation that if
one defines the \slicePlane\ with the template
\beq
    \slicep = (1, 0, 0, \ldots , 0) \, ,
    \label{e-slicep}
\eeq
then as long as the solutions has a non-zero first Fourier mode component,
they have a representative point on the slice; thus, one can obtain a global
symmetry reduced description.

% this moved here from
% siminos/ksRecycled/tex/flotsam.tex    master file: ksRecycled.tex
\item[2015-03-05 Xiong] removed the details of multishooting method:

Next, we apply Newton-Raphson algorithm to each piece
of this orbit to obtain better approximation of the PPO/RPO. This is called
multishooting method. More specifically, expanding \eqref{eq:multi1} and
\eqref{eq:multi2} at the true orbit points $a^*_i = a_i + \delta a_i$
to the first order gives
\[
J^{\Delta T}(a_i) \delta a_i - \delta a_{i+1} + v_{i+1}\delta \Delta T +
f^{\Delta T}(a_i) - a_{i+1} = 0
\]
and
\begin{equation}
\left\{
  \begin{array}{l l}
    g(\theta)J^{\Delta T}(a_{M-1}) + g(\theta)v_{0}\delta \Delta T \\
    +\mathbf{T}g(\theta)f^{\Delta T}(a_{M-1})\delta \theta - \delta a_0 \\
    + g(\theta)f^{\Delta T}(a_{M-1}) - a_0 & = 0 \quad \text{for RPO} \\
    \\
    RJ^{\Delta T}(a_{M-1}) + Rv_{0}\delta \Delta T - \delta a_0  \\
    + Rf^{\Delta T}(a_{M-1}) - a_0
    & = 0 \quad \text{for PPO}
  \end{array}
\right.
\label{eq:multi2}
\end{equation}

\item[2015-09-19 Predrag] This snippet from the NSF Outcomes15 was edited
away, a short version incorporated. Maybe of use somewhere else:

The prevailing custom is to visualize turbulent fluid by 3-dimensional
snapshots, such as Fig. 1. However, the function space of allowable
Navier-Stokes velocity fields is an infinite-dimensional (in numerical
work performed here, ~100,000-dimensional) state space , with the
instantaneous state of the fluid a point in this space. Here we deploy
novel, dynamical systems visualisations of moderate Reynolds number
turbulent flows, complementary to the traditional 3D spatio-temporal
visualizations. They are projections on dynamically intrinsic coordinate
frames, see Fig. 2. Even though the state-space is infinite-dimensional,
numerically exact solutions such as equilibria and periodic orbits are
easily visualized as points or closed loops in any state-space
projection. The understanding of chaotic dynamics in high-dimensional
state space offers a promising dynamical framework to study turbulence.
Here turbulence is viewed as a walk through a forest of numerically exact
unstable `coherent state' solutions of the governing equations (no
modelling). This emerging view complements and extends the more
traditional statistical view of the phenomenon. Recent progress was made
possible only by collaborative effort by physicists, engineers and
applied mathematicians, and it required every tool in the fluid
dynamicist's toolbox: numerical searches for recurrent flows, computation
of their stability, their symmetry classification, and theory of
dynamical zeta functions to calculate statistical averages over the
turbulent flow from these solutions. The research has lead to a
fundamentally new understanding of the geometry of Navier-Stokes as
dynamical systems, viewed in their infinite-dimensional state spaces.

\item[2015-09-19 Predrag] Removed this:

\bea
    b_2 &=& \pm \sqrt{\frac{p_2 + \sqrt{p_2^2 + 4 p_3^2}}{2}} \continue
    c_3 &=& p_3 / b_2 \continue
    b_4 &=& p_4 / c_3 \continue
    c_5 &=& p_5 / b_4 \\
    \vdots \nonumber
\eea

\item[2015-09-20 Predrag] \PCedit{EXPLAIN ``coherent solutions''}

\item[2015-09-20 Predrag] reuse this:

After quotienting the reflection symmetry, either type of a solution maps
into a single solution in the fundamental domain.

In this subspace there are two kinds of solutions:
reflection-related pairs, and  orbits self-dual under reflection, which
--if periodic-- are $r=2$ repeats (closing after two repeats) of an \rpo.


\item[2015-09-20 Predrag]  must remember to explain that partially
hyperbolic tori are NOT resonant, and thus expected to remain smooth
until they collide with something

\item[2015-09-21 Burak+Predrag]

\begin{itemize}
  \item
Q:      the main new thing is O(2) symmetry reduction, not described
anywhere else, right?
\\
A: I think the way I visualized the unstable manifolds of the periodic
orbits also is new, I don't know any other paper with 2D unstable
manifold of a periodic orbit. Even if someone have done something similar
on a say, 2D map, the proper projection of the Floquet vectors onto the
\PoincSec\ would still have been new in this particular context,
because nothing looks as pretty if one does not account for that.
  \item
Q:      that helps (but is not necessary) for the study of a torus
bifurcation, right?
\\
A: Correct. If the reflection symmetry wasn't reduced, then one would
have to integrate everything twice as long, and I'm not sure how would
period 3 orbits have looked like in that case. The 2D unstable manifold
is a challenging calculation in terms of data volume, thus it pays off to
reduce that as much as possible.

\end{itemize}

\item[2015-09-23 Predrag]
Replaced ``a pseudo-spectral formulation of \refeq{e-ODE} and its
gradient system" by ``a pseudo-spectral formulation of \refeq{e-ODE}''

Dropped "The $k < 0$ part of the Fourier spectrum is accounted for As $u
(\conf, \zeit)$ is real, $\Fu_{-k} = \Fu^*_k$. ''

\item[2015-09-23 Predrag]
I know what you want to say, but this definition might lead to more
confusion than it is worth, so I removed it, and other $\SymmRed (\ssp )$
that follow:

``
\emph{Symmetry reduction} is a coordinate transformation
\(
\sspRefRed = \SymmRed (\ssp )
\)
such that
\bea
\SymmRed (\ssp ) &=& \SymmRed ( \ssp' ) \quad
\mbox{if} \quad \ssp' = \matrixRep(g) \ssp \,,
\quad g \in \{\LieEl (\theta), \sigma \} \continue
\SymmRed (\ssp ) &\neq& \SymmRed ( \ssp' ) \quad
\mbox{if} \quad \ssp' \neq \matrixRep(g) \ssp \,,
\quad g \in \{\LieEl (\theta), \sigma \}
\,,
\label{e-SymmRed}
\eea
\ie, reduced coordinates $\sspRefRed$ are symmetry invariant.
''

Invariant polynomials are presumably not ``coordinate transformations'',
as in general they are higher-dimensional than the original coordinates
and have to be supplanted by syzygies. And in slicing, for each
continuous symmetry parameter one of the coordinates is the phase
`fiber', which is not invariant under the group action. No need to get
into any of that here.

\item[2015-09-23 Burak] I don't have any objections to these points.
I replaced the partial derivatives in the Jacobian notation with total
derivative in the appendix, because according to
\HREF{https://en.wikipedia.org/wiki/Jacobian_matrix_and_determinant}
{wikipedia}
common notations for Jacobian of
$f(x)$ are $df/dx, Df, J_f,
\partial (f_1, \ldots, f_m) / \partial (x_1, \ldots, x_n)$. Partial derivatives
are used for component-wise equations. Either way, I think it's clear from the
context, which Jacobian we are talking about.
\\
{\bf [2015-09-23 Predrag]} OK

\item[2015-09-21 Burak] I removed

"(\rpo s, \ie, invariant tori in the full \statesp)"

because all orbits in the paper are pre-periodic. They have continuously
many copies in the full \statesp\ but they themselves don't wind around
invariant tori.

\item[2015-09-23 Predrag] I cannot use more time on the paper now, so
please take over - I have not gone through the numerical results section.
You probably need to study Peckham\rf{peckham90} and similar papers,
harmonize what you see here with them. If you can compute rotation rates, at
least Ruelle will be happy - he likes them.

\item[2015-09-23 Burak] Thank you. I'll study Peckham and at least include a
discussion of what generic category does the bifurcation we see falls into
before submitting it. I computed the rotation rate for this orbit in the slice
for the torsion paper, but it might be tricky to do it in the \On{2}-reduced
setting but I'll think about it.

\item[2015-09-26 Burak] Took out the following:

``Halving the \SOn{2} continuous symmetry-reduced \statesp\ might not seem
like much, but it pays off quickly. Prior to the discrete symmetry
reduction, the pre-\po s of period 3 that we shall study in the example
below are \po s of period 9, and dealing with such orbits would make our
analysis not only conceptually much more cumbersome, but also numerically
more unwieldy.''

Because, while all periods double in the full \statesp , their
ratios stay the same, so I'm guessing the topological lengths
would still be the same. The analysis, to the extent we present in
the current paper, could have probably been done with the \SOn{2}
reduction only, may be even without it. Unfortunately I couldn't
find any ppo-rpo relations, in which case the \On{2} reduction
would be absolutely crucial. However, I'm saying all these things
in retrospect, I already know that the parent orbit is
pre-periodic, and so are the period-3 orbits. Having just enough
amount of data, an invariant representation, from the beginning is
what makes the entire analysis way much easier.

\end{description}
