% siminos/blog/norms.tex
% $Author: predrag $ $Date: 2016-04-13 20:36:35 -0400 (Wed, 13 Apr 2016) $

\chapter{Norms, distances}
\label{c-norms}
% Predrag 2012-05-04

\begin{description}

\item[2012-11-12 Predrag]
moved to here from \texttt{pipes/blog/norms.tex}.

\end{description}

\section{Norms blog}

\bigskip\bigskip
\noindent
{\color{red} The latest blog entry at the end of this chapter}
\bigskip\bigskip

\begin{description}


\label{sec:proxeq}

\item[2011-11-16 Predrag]
I think for \KS\ it would be the safest to use
\beq
  \Norm{\ssp-\ssp'}^2  = \braket{\ssp-\ssp'}{\ssp-\ssp'} =
\Lint{\pSpace} ({u}-{u}')^2
\,.
\label{KSnormFr} \eeq
The virtue of energy norm is that it is representation independent.

If
one goes to siminos/ksReduced/ invariant basis, the $2N-1$
$\SOn{2}$-invariant and functionally independent variables are
\bseq\label{eq:SO2cheb}
  \begin{align}
    \overline{b}_k &=
		    b_k\, \chebT_k\left(b_1/r_1\right)+
		    c_k\,\frac{c_1}{r_1} \chebU_{k-1}\left(b_1/r_1\right)\,, \label{eq:SO2cheb1}\\
    \overline{c}_k &=
		    b_k\, \frac{c_1}{r_1} \chebU_{k-1}\left(b_1/r_1\right)+
		    c_k\,\chebT_k\left(b_1/r_1\right)\,,  \label{eq:SO2cheb2}
  \end{align}
\eseq
for $k=1,\ldots N$, where $r_k\equiv\sqrt{b_i^2+c_i^2}$ and
$\chebT_k,\,\chebU_k$ are Chebyshev polynomials of the first and second
type, respectively. The Chebyshev polynomials $\chebT_k,\,\chebU_k$ are
degree $k$, so they tend to significantly distort the state space layout
of the attractor.

Evangelos, please confirm that you are using distance function
\refeq{KSnormFr}, not the norm
\beq
\Norm{u}  =   {\textstyle\frac{1}{2}} \sum_{k=0}^{\infty}
   (\overline{b}_k^2+\overline{c}_k^2)
\,,
\label{normNotGood} \eeq

\item[2011-12-1 Evangelos] I do not understand norm \refeq{KSnormFr}.
Take for instance a pure sin equilibrium solution of KSe and compare it
to itself under \refeq{KSnormFr} - you get zero. Now translate it by
$l<L$, and compare to the original: you get non-zero distance. Thus this
norm is not translation invariant.

I use $2\,\Norm{u}$ of \refeq{normNotGood}, since I see reduced KS as a
dynamical system defined in terms of invariant variables
$\overline{b}_k,\,\overline{c}_k$.

Please observe that although the Chebyshev polynomials
$\chebT_k,\,\chebU_k$ are degree $k$, they are functions of $b_1/r_1$ in
\refeq{eq:SO2cheb}. Therefore $\overline{b}_k, \overline{c}_k $ are not
really invariant polynomials in $b_k,\,c_k$. In fact, \refeq{eq:SO2cheb}
is equivalent to reducing to a slice with $c_1=0$. Therefore, even though
transformations \refeq{eq:SO2cheb} are nonlinear, they preserve
$r_k\equiv\sqrt{b_i^2+c_i^2}$. To see this, recall that they can be
written in the polar form
\bseq\label{eq:SO2polar0}
  \begin{align}
    \overline{b}_k &=
		    r_k\, \cos(\phi_k-k\,\phi_1)\,, \label{eq:SO2polar1}\\
    \overline{c}_k &=
		    r_k\, \sin(\phi_k-k\,\phi_1)\,.\label{eq:SO2polar2}
  \end{align}
\eseq
So any deformation of state space is a result of symmetry reduction
(identification of points along group orbits) and the discontinuity at
$r_1=0$.

If we were to follow Tuckerman's recipe and multiply \refeq{eq:SO2polar}
by $r_1^k$ we would indeed have invariant polynomials of order $k+1$ in
the $b_i,\,c_i$. In siminos/ksReduced, we try to keep state-space
distortion to a minimum, therefore multiplying by $r_1$ (or $r_1^2$).



\end{description}
