% siminos/blog/dailyBlog.tex
% $Author: predrag $ $Date: 2021-01-09 00:07:49 -0500 (Sat, 09 Jan 2021) $

\chapter{Daily blog}
\label{c-dailyBlog}

%\section{2016-04-12 Blogging on}
%\label{sect:2016-04-12blog}


\begin{description}

\item[2015-12-08 Burak]
Section 4 of Sartori\rf{Sartori91} can be a general treatment
of our reflection reduction problem.

\item[2015-12-08 Predrag] Oh, I see - I had downloaded
Sartori\rf{Sartori91} 7 years ago (it is in
\texttt{ChaosBook.org/library} but have not studied it, see above in this
blog:

\item[2008-09-01 PC] (rescued this from the siminos/blog/flotsam.tex
``have not read this ever'' list).
Following might be of interest (have not checked myself):
Sartori\rf{Sartori91},
{\em Geometric invariant theory: a model-independent approach
       to spontaneous symmetry and/or supersymmetry breaking}.
120 pages and the total of 8 citations in the last quarter century.

Check also Muriel K{\oe}nig\rf{Koenig97} \emph{Linearization of vector fields on
the orbit space of the action of a compact Lie group}

\item[2009-01-07 Predrag] track down
Sartori\rf{Sartori91},
Chossat\rf{Choss93},
K{\oe}ning\rf{Koenig97}, and
Leis\rf{Leis97}

\item[2015-12-08 Predrag]
Sartori and Valente\rf{SarVal05}, {\em Constructive axiomatic approach
to the determination of the orbit spaces of coregular compact linear
groups} (article cited 2 times, by the authors) write: ``
We review the proposal of a constructive axiomatic approach to the
determination of the orbit spaces of all the real compact linear groups,
obtained through the computation of a metric matrix $P^{(p)}$, which is
defined only in terms of the scalar products between the gradients
 of the elements of a minimal integrity basis (MIB) for
the ring of G-invariant polynomials. The domain of semi-positivity
of is known to realize the orbit space of G as a
semi-algebraic variety.

The matrices $P^{(p)}$ can be obtained from the solutions of a universal
differential equation (master equation), which satisfy convenient initial
conditions. This approach tries to bypass the actual impossibility of
explicitly determining a set of basic polynomial invariants for each
group.

Our results may be relevant in physical contexts where the study of
covariant or invariant functions is important, like in the determination
of patterns of spontaneous symmetry breaking in quantum field theory, in
the analysis of phase spaces and structural phase transitions (Landau's
theory), in covariant bifurcation theory, in crystal field theory and so
on.
''

Sartori and Valente\rf{SarVal05a, SarVal05b}, {\em The radial problem in
gauge field theory models}, write: ``
The study of spontaneous symmetry breaking patterns in theories in which
the ground state is determined by the minima of a potential invariant
under the symmetry group of the system may be traced back to the solution
of two classes of problems, the angular and radial problem, respectively.
Whilst the former problem, i.e., the determination of the isotropy-type
stratification, has been extensively treated both in condensed matter
physics and in particle physics, the radial problem, in particular the
construction of the phenomenological potential allowing the realization
of all the symmetry allowed symmetry phases, has up to now substantially
been disregarded in gauge field theory, because renormalizability limits
to four the degree of the Higgs potential and it is widely thought that
spontaneous radiative mass generation can anyway fix the issue. Through a
rigorous analysis in the framework of geometric invariant theory
($P^{(p)}$-matrix approach) we review these facts, focussing our
attention on the role of radiative corrections. Then, we propose a way of
reconciling renormalizability requirement and tree-level observability of
all the phases allowed by the symmetry. The idea will be illustrated in
simple extensions of two-Higgs-doublet SM, with additional scalar
singlets and discrete symmetries. This will allow us to explain the
rationale behind all the extensions of the Higgs sectors so far proposed.
''

Sartori and Valente\rf{SarVal06}, {\em Orbit spaces of reflection groups with
         2, 3 and 4 basic polynomial invariants}, write: ``
Functions which are covariant or invariant under the transformations of a
compact linear group can be expressed advantageously in terms of
functions defined in the orbit space of the group, i.e. as functions of a
finite set of basic invariant polynomials. The equalities and
inequalities defining the orbit spaces of all finite coregular real
linear groups (most of which are crystallographic groups) with at most
four independent basic invariants are determined. For each group G acting
in the Euclidean space the results are obtained through the computation
of a metric matrix , which is defined only in terms of the scalar
products between the gradients of a set of basic polynomial invariants of
G; the semi-positivity conditions are known to determine all the
equalities and inequalities defining the orbit space of G as a
semi-algebraic variety in the space  spanned by the variables. Our
results can be easily exploited, in many physical contexts where the
study of covariant or invariant functions is important, for instance in
the determination of patterns of spontaneous symmetry breaking, in the
analysis of phase spaces and structural phase transitions (Landau's
theory), in covariant bifurcation theory, in crystal field theory and in
most areas of solid-state theory where use is made of symmetry adapted
functions.''

This has been picked up by Talamini\rf{SarTal98,Talamini06}
{\em P-matrices in orbit spaces and invariant theory}, who reviews the
$P^{(p)}$-matrix approach and how it can be used to determine the
equations and inequalities defining the orbit space and its strata. When
the integrity basis is only partially known, calculating the
$P^{(p)}$-matrix elements one is able to determine the integrity basis
completely. Then he goes on\rf{Talamini10,Talamini11} to construct the
bases of invariant polynomials and $P^{(p)}$-matrices of exceptional
groups $E_7$ and $E_8$ (!), maybe making contact with my own useless
book\rf{PCgr}...


\item[2012-02-27 Predrag]
A reference I have not checked: Gatermann and Guyard\rf{GatGuy99},
\emph{Gr\"obner bases, invariant theory and equivariant dynamics}. He writes:``
This paper is about algorithmic invariant theory as it is required within
equivariant dynamical systems. The question of generic bifurcation
equations (arbitrary equivariant polynomial vector) requires the
knowledge of fundamental invariants and equivariants. We discuss
computations which are related to this for finite groups and semi-simple
Lie groups. We consider questions such as the completeness of invariants
and equivariants. Efficient computations are gained by the Hilbert series
driven Buchberger algorithm because computation of elimination ideals is
heavily required. Applications such as orbit space reduction are
presented.
''

\item[2015-12-08 Predrag]
But now we also might have to read Gaeta\rf{Gaeta08,Gaeta92,Gaeta95},
{\em Breaking of linear symmetries and Michel's theory: Grassmann
manifolds, and invariant subspaces}.

Gaeta and M. A. Rodr\'iguez\rf{GaeRod96,GaeRod96a}
{\em Discrete symmetries of differential equations} write: ``
The determination of the continuous symmetries of differential equations
follows a well known algorithm, and is reduced to solution of a set of
linear equations; this is based on considering infinitesimal generators
of the symmetries, so that the method does not extend to discrete
symmetries. We present a method to determine discrete symmetries in a
certain class by means of a linear system. We also consider the inverse
(and simpler) problem of determining the most general equation admitting
a given discrete symmetry. In the last part, we consider a number of
examples, dealing in particular with symmetries of relevance to physics.

Sturmfels and White\rf{StuWhi89}
write in {\em Gr\"obner bases and invariant theory}: ``
we study the relationship between Buchberger's Gr{\"o}bner basis method and
the straightening algorithm in the bracket algebra. These methods will be
introduced in a self-contained overview on the relevant areas from
computational algebraic geometry and invariant theory. We prove that a
certain class of van der Waerden syzygies forms a Gr{\"o}bner basis for the
syzygy ideal in the bracket ring. We also give a description of a reduced
Gr{\"o}bner basis in terms of standard and non-standard tableaux. Some
possible applications of straightening for symbolic computations in
projective geometry are indicated."

Bayer and Morrison\rf{BayMor88} {\em Standard bases and geometric
invariant theory {I. Initial} ideals and state polytopes} write: ``
characterise the set of monomial ideals that can occur as the initial
ideals of a given homogeneous ideal in a polynomial ring, as one varies
the monomial order used within a fixed coordinate system. This set is
canonically bijective to the set of vertices of a convex polytope arising
in the geometric invariant theory of the Hilbert point of Q."

Bayer and Stillman\rf{BaySti92} {\em Computation of {Hilbert}
functions}: ``
We present an algorithm along with implementation details and timing data
for computing the Hilbert function of a monomial ideal. Our algorithm is
often substantially faster in practice than existing algorithms, and
executes in linear time when applied to an initial monomial ideal in
generic coordinates. The algorithm generalizes to compute multi-graded
Hilbert functions.
''

Check also Crockett\rf{Crockett10} {\em Orbit Space Reduction for
Symmetric Dynamical Systems with an Application to Laser Dynamics}

Dhont and Zhilinski\'i\rf{DhoZhi13,DhoPatZhi15}
{\em The action of the orthogonal group on planar vectors:
         invariants, covariants and syzygies} write: ``
The construction of invariant and covariant polynomials from the x , y
components of n planar vectors under the \SOn{2} and \On{2} orthogonal groups
is addressed. Molien functions determined under the \SOn{2} symmetry group
are used as a guide to propose integrity bases for the algebra of
invariants and the modules of covariants. We propose a new representation
of the Molien function. This non-standard form is symbolically
interpreted in terms of a generalized integrity basis. Integrity bases
are explicitly given for n = 2, 3, 4 planar vectors and m ranging from 0
to 5. The integrity bases obtained under the \SOn{2} symmetry group are
subsequently extended to the \On{2} group.
''

These $n$ planar vectors are $(x_1,y_1,x_2,y_2,\cdots,x_n,y_n)$. Perhaps
it is OK to think of them as the $n$ complex Fourier coefficients? The
integrity bases for two planar vectors under $\SOn{2}$ (with one syzygy)
look like what we would write down (eq.~(16) in \refref{DhoZhi13}). They
credit Weyl\rf{Weyl39} for this integrity basis, as well as for the three
planar vectors integrity basis, and its 9 syzygies. In the 4-vector
case there are 38  linearly dependent syzygies of degree 4 in variables
$(x_i, y_i)$ between the generators of all the invariants.
The  $\On{2}$ integrity bases follow from  $\SOn{2}$ bases, but now
there is also one $A_2$ representation, and the rest sits in 2\dmn\ $E_m$
representations (this is explained in Xiong's blog, {\em inter alia}).
It's a holy mess, and I see no way of using this in high dimensions.

\item[2015-12-08 Burak] Most of these
papers have techniques that are way beyond my level, so I'd say we
tone our response down, and defend our approach for its simplicity.

I want to note one thing that I think is important for a broader
reason:

Chossat\rf{Choss93} shows that a ``quasi-periodic'' solution is born
out of the bifurcation of the homoclinic cycle in the presence of
\SOn{2} symmetry.
(He studies a forced system such that reflectional symmetry is broken and
what remains is \SOn{2}. But I still think that \rpo s are related to
this homoclinic cycle, \refref{AGHO288} or \refref{AGHks89} might have
this story.)
The homoclinic cycle he refers to is the
\hec\ from $E_2$ of the \KS\ system to its quarter
domain-shift, which you visualize in \refref{SCD07}.
I suspect that this bifurcation is the point where the \rpo s of the
\KS\ system are born. Because for parameter values, lower than what
I looked at in \refref{BudCvi15}, this ``homoclinic cycle'' is stable
(it is also stable at $L=22$ if we impose reflection-invariance).
But since I had no idea of how to follow bifurcations of a homoclinic
cycle, I didn't attempt to attack this issue. Moreover, since $E_2$
lies on the border of \fFslice, there are also technical issues of
slice discontinuities.

Another question I had in mind (I remember now) was this:
This homoclinic cycle shows up as soon as $E_2$ looses stability, and
is very robust. It also is a closed orbit in the symmetry-reduced
space (not in \fFslice\ but that is \fFslice's fault). So I think it
should contribute to the trace formulas. But the question is how?
This really bothers me and I suspect this might be the reason that
our periodic orbit theory attempts have failed so far.

\item[2015-12-08 Predrag]
Their notion of ``homoclinic cycle'' does not make sense (but we
make sure it is not Poincar\'e who named it that first:). Yes, it starts
from a time-invariant solution and it returns to the same time invariant
solution (perhaps shifted along its group orbit), but it takes infinite
time and it does not repeat, so what kind of `cyclic' behavior is that?
One should then call a heteroclynic connection from an time-invariant
solution to another a ``cycle'' as well, and that surely makes no sense at
all...

\item[2015-12-09 Burak]
(I'm sticking with heteroclinic cycle in what follows, even though
the object I'm talking about falls into the homoclinic cycle of
definition of Chossat.)
It's not the \hec\ they call ``heteroclinic cycle''
but the infinitely many of orbits that lie nearby.
\HREF{http://www.scholarpedia.org/article/Heteroclinic_cycles}{
Scholarpedia article} is reader-friendly. For instance, if you ran
your trajectories long enough in unstable manifold figure of
\rf{SCD07}, you would see that the orbit that seemed to connect $E_2$
and $\LieEl(\pi/4) E_2$ would jump back towards $E_2$ and repeat that
a few other times until it diverges away. The opposite happens if the
\hec\ is stable, then heteroclinic cycle repeat it
again and again, spending more time near \eqva\ at each return.

If we consider the stable case, then it I think is obvious that the
physical observables in the system would converge to their values
averaged over the \hec\ in long term. Then if we
were to predict these using periodic orbit theory, then we would need
a contribution to the trace from the \hec. So I
still think that this is worth to think about.

\item[2015-12-08 Predrag]
Perhaps an interesting curiosity about 2- and 3-modes systems:
Graham D Hesketh writes in
{\em General complex envelope solutions of coupled-mode optics with
quadratic or cubic nonlinearity},
\arXiv{1512.03092}, J. Opt. Soc. Am. B 32, 2391-2399 (2015):
``
The analytic solutions for the complex field envelopes are
derived using Weierstrass elliptic functions for two and three mode
systems of differential equations coupled via quadratic $\chi_2$ type
nonlinearity as well as two mode systems coupled via cubic $\chi_3$ type
nonlinearity. For the first time, a compact form of the solutions is
given involving simple ratios of Weierstrass sigma functions (or
equivalently Jacobi theta functions). A Fourier series is also given. The
models describe sum and difference frequency generation, polarization
dynamics, parity-time dynamics and optical processing applications.''

\item[2015-12-14 Burak] I found a recent review article
Guckenheimer \etal\
{\em Invariant manifolds and global bifurcations} \rf{GKOS15}
on how invariant manifolds shape the \statesp\ of chaotic systems and
their numerical computation.
Examples considered in the article are low dimensional except the
traveling waves of FitzHugh-Naguma PDE.
However, in the discussion, they list computation of higher dimensional
invariant manifolds and doing these in systems with symmetries are listed
as important research directions. A nice coincidence is that
one of the co-authors of the paper is Bj\"{o}rn Sandstede, so I think he
will appreciate our response to the rejection of our paper by the editor.

Another paper cited in \rf{GKOS15} might also be of interest:
Schilder \etal\
{\em Continuation of Quasi-periodic Invariant Tori} \rf{SOHV05}
discusses a method
for continuation of invariant tori. I haven't read the article, so
I don't know whether the method is suitable for high dimensional
settings or not.

    \PCpost{2015-12-22}{
Maybe it is time to revisit {\cLf}? We gave up on it, as we did not know
what to do with a 5\dmn\ system with one zeroth Fourier mode ($z$ axis)
and a pair of $k=1$ first Fourier modes, so we focused on systems like
\twomode\ system and \KS, where each mode contributes only once.

Invariant polynomial bases\rf{gatermannHab,ChossLaut00} are natural,
mathematically sound\rf{Hilbert93,Noether15} and aesthetically appealing
to us: they yield global symmetry reduction (no artificial {\sliceBord
s}!). Unfortunately at a price. They work very well in low
dimensions\rf{SiCvi10,BuBoCvSi14} but quickly become unmanageable. As an
example, consider Dhont and Zhilinski\'i\rf{DhoZhi13,DhoPatZhi15}
explicit construction of Weyl \SOn{2} and \On{2} integrity basis from
$(x_1,y_1,x_2,y_2,\cdots,x_n,y_n)$ components of $n$ planar vectors
(these might already be listed in the classic Weyl book\rf{Weyl39}, I
have not checked). I believe that by a `planar vector' they mean the
$k=1$ first Fourier mode representation. For two planar vectors there is
one syzygy, for three vectors there are 9, and in the 4-vector case there
are already 38 linearly dependent syzygies of degree 4 in variables
$(x_i, y_i)$.

In the spirit of Burak's $\On{2}$ hybridization of slices and integrity
bases, maybe we should try Dhont and Zhilinski\'i\rf{DhoZhi13} integrity
basis on the \SOn{2}-invariant {\cLf} - it would be only one syzygy. In
the spirit of teutonic completeness.
    }

    \PCpost{2015-12-23}{
On further reading, I do not see material that Evangelos has not covered
already.
The Sect.~4. {\em Integrity bases for two planar vectors under $\SOn{2}$
} of \refref{DhoZhi13}, for which they credit Weyl\rf{Weyl39},
and Michel-Zhilinski\'i\rf{MicZhi01}, is already covered in
Sect.~2.3.1 {\em Hilbert basis approach} of Siminos
thesis\rf{SiminosThesis}, following Gilmore and Letellier\rf{GL-Gil07b}:
\bea
	u_1 &=& x_1^2+x_2^2 \,,\qquad
	u_2 = y_1^2+y_2^2 \cont
	u_3 &=& x_1 y_2-x_2 y_1\,,\qquad
	u_4 = x_1 y_1+x_2 y_2\cont
	u_5 &=& z\,,
	\label{eq:ipLaser}
\eea
with syzygy
\beq
 	u_1u_2 -u_3^2-u_4^2 =0\,.
	\label{eq:syzLaser}
\eeq
Curiously, ``planar vectors'' are never defined, but I guess the notion
is obvious. Maybe worth a look is
Michel-Zhilinski\'i\rf{MicZhi01}. They write: ``
Elementary concepts of group actions: orbits and their
stabilizers, orbit types and their strata are introduced and illustrated
by simple examples. We give the unified description of these notions
which are often used in the different domains of physics under different
names. We also explain some basic facts about rings of invariant
functions and their module structure. This leads to a geometrical study
of the orbit space and of the level surfaces of invariant functions (e.g.
energy levels of Hamiltonians). Combining these tools with Morse theory
we study the extrema of invariant functions.''
    }

    \PCpost{2016-01-11}{
    Probably worth a quick read-through: Marques \etal\rf{MarLopBla04}
{\em Bifurcations in systems with {Z2} spatio-temporal
     and {\On{2}} spatial symmetry}.
     }

\item[2016-01-12 Burak] Read the first 3 sections carefully and skimmed
through the rest. The paper is about periodically driven systems, with
an fluid dynamics example, where there is a rectangular duct with
a periodic direction $z$, confined in $x$ and $y$ by walls, and the
wall at $x=0$ periodically oscillates in $y$ direction.
This introduces the
``spatio-temporal symmetry'' to the problem, that is if you reflect in
$y$ direction and shift time by half of the oscillation period that is
a symmetry of the system. In addition to this, there is usual \On{2} in
periodic $z$ direction. The following analysis looks at general forced
oscillator problem with symmetry and relates the global $t=t_0$
\PoincSec\ of the extended \statesp\ to the action of the
spatio-temporal symmetry. With these techniques, they study \On{2}
symmetry breaking bifurcations in the \PoincSec\ with normal
forms etc. I'm not sure if the particular results are directly relevant
to the things we do.

    \PCpost{2016-01-17}{
Another, similar paper by Blackburn, Marques and Lopez\rf{BlMaLo05}, not
sure whether it is any more helpful to us, or needs to be cited. They
note in {\em Symmetry breaking of two-dimensional time-periodic wakes}
that it pays to quotient shift-and-reflect: ``
A number of two-dimensional time-periodic flows, for example the Karman
street wake of a symmetrical bluff body such as a circular cylinder,
possess a spatio-temporal symmetry: a combination of evolution by half a
period in time and a spatial reflection leaves the flow invariant.
Floquet analyses for the stability of these flows to three-dimensional
perturbations have in the past been based on the Poincar\'e map, without
attempting to exploit the spatio-temporal symmetry. Here, Floquet
analysis based on the half-period-flip map provides a comprehensive
interpretation of the symmetry-breaking bifurcations.
        }

    \PCpost{2016-01-17}{
Note that SIAM has a new modernized set of
\HREF{http://connect.siam.org/manuscripts-enhanced-due-to-modernized-siam-macros/}
{LaTeX2e Macros} for use with its print (not only online) journals.
   }

    \PCpost{2016-03-065}{
Found yet another obscure article in a non-existent Ukrainian journal on
\cLe: Kiselev\rf{Kiselev98} {\em Symmetry breaking and bifurcations in
complex {Lorenz} model}. If he knew about slicing, this paper would have
been better, me thinks.
    }


\item[2016-04-12 Predrag]
  Edit and improve \refsect{sect:LorenzD1} as you see fit: if you have
  comments, enter them here.

\item[2016-04-23 Predrag] I thought it might pay off to understand irreps
of \On{n}, not just \On{2}, so my colleague Andreas Wirzba, master of
group theory as used in the standard model told me to study the discussion
of irreps of \On{4} in the PhD
thesis of J. Bsaisou\rf{Bsaisou14}, {\em Electric Dipole Moments of
Light Nuclei in Chiral Effective Field Theory}, Append.~B. Wow! It's
double cover is $\SUn{2}_L \otimes \SUn{2}_R$ and nothing is easy about it:
unit quaternions, the Cayley-Klein 2-1 isomorphism, etc., etc.

\item[2016-04-23 Predrag] Andreas Wirzba says that a way to deal with the
orbitold singularities (embeddings of invariant, ``fixed point''
subspaces) might be to enlarge the symmetry - that unfolds the
singularity. I am not convinced, as this introduces arbitrariness in
choice of the larger group, and the way one approaches the
singularities...

\item[2016-05-04 Predrag]
A few general comments before I return to \refref{BudCvi15}, currently
entitled
{\em Unstable manifolds of relative periodic orbits in the
symmetry-reduced state space of the Kuramoto-Sivashinsky system},
for a final push:

I have tried to go back to \On{n} representation theory and reformulate
our \KS\ symmetry reduction using the fact that $k\neq0$ irreps of \On{2}
are 2\dmn. In a sense we already use that when going from
$\cdots,-2,-1,0,1,2,\cdots$ to $k>0$ formulation of equations, so maybe
there is nothing new to be mined here, except a more precise remark about
irreps.

Now, while it is psychologically more comforting to have a continuous
flow visualization in $\pS/\On{2}$, I think we should not insist on it,
and make too much out of it. I am worried that Burak's
algorithm\rf{BudCvi15} is so specific to {\fFslice}, and very worried
about the 4th order polynomials that arise in Xiong's {\sFslice} \refeq{XDD4reduc}: they surely distort the {\reducedsp} very
much.

Why insist on the continuity of the flow in $\pS/\Group$? In ChaosBook I
go into great detail explaining the $\pS/\Dn{3}$ reduction for the 3 disk
billiard, emphasizing that all 2- and 3\dmn\ visualization of the phase
space flow are useless, and that only the symmetry-reduced \PoincSec\
makes sense and is beautiful. For billiards the phase-space flow is
explicitly discontinuous - momentum coordinates jumps from $p_\perp$ to
$- p_\perp$, but not only it does not matter, for billiards this
discontinuity defines a natural \PoincSec.

For \KS\ I would be happier if we used no nonlinear transformations to
invariant polynomial bases, but instead kept everything linear, and
reflected the trajectory back into the fundamental domain whenever it
exits it, marking --let's say-- the point where the trajectory exits by a
red dot, and where it re-enters by a blue dot. That makes trajectories
discontinuous, but much less so than for the 3-disk pinball. Then one
chooses a convenient \PoincSec\ in the fundamental domain, and studies
the unstable manifolds and return maps in it - that is the only thing we
need, any continuous flow visualization provides a bit of  comfort in
understanding which partition maps into which partition, but is otherwise
inessential.

That's what I think we should be doing. But for now I throw the towel into
the ring, and return to finalizing \refref{BudCvi15} as is.

\item[2016-06-03 Predrag]
van Veen and G. Kawahara and M. Atsush\rf{VaKaAt11}
{\em On matrix-free computation of {2D} unstable manifolds} write:
``Recently, a flexible and stable algorithm was introduced for the
computation of two-dimensional (2D) unstable manifolds of periodic
solutions to systems of ordinary differential equations. The main idea of
this approach is to represent orbits in this manifold as the solutions of
an appropriate boundary value problem (BVP). The BVP is underdetermined,
and a one-parameter family of solutions can be found by means of
arclength continuation. This family of orbits covers a piece of the
manifold.
[...]
we describe an implementation of the orbit continuation algorithm which
relies on multiple shooting and Newton-Krylov continuation.
[...]
We demonstrate our algorithm with two test systems: a low-order model of
shear flow and a well-resolved simulation of turbulent plane Couette
flow.

\item[2016-06-07 Burak] \refRef{VaKaAt11} is concerned with \po s with
$1$ unstable Floquet multiplier. They don't use Poincar\'e sections, so
they call the unstable manifold two-dimensional, however, the second
dimension is the orbit itself. It is elegant
and expensive, but I don't see an immediate advantage of it against what
we do in \refref{BudCvi15}, which we can extend to $2$ dimensions ($4$
if we count time-forward and translation symmetries). The application
paper \rf{VaKa11} that follows up is very interesting though; it
demonstrates that the unstable manifold of a \po\ in the edge of chaos
is homoclinic to itself and its dynamics resembles that of turbulent
bursting.

\item[2016-11-11 Predrag] Accidentally discovered MacKay\rf{MacKay84}
{\em Equivariant universality classes}. Robert (who is a smart cookie)
looks at interplay between symmetries and bifurcation sequences, such as
the period doubling.

``Call this a \Zn{2} symmetry. Such symmetries can lead to
anomalous behaviour. In particular, anomalies have
been observed in bifurcations of periodic orbits
and in the asymptotic structure near invariant circles.
[...]
a period-3 behaviour is observed for their residues
instead of the usual period-1 behaviour.''

This might explain why Burak sees period 3 in \refref{BudCvi15}?
I have not studied it in detail, but I sure like the conclusion:
\emph{The moral is that for systems possessing a symmetry
group, except when looking near fixed points of the
symmetries, you must divide out by the group. This
gives one a system more likely to exhibit generic behaviour.}

\item[2016-12-10 Predrag] {\bf to Xiong and Burak}
Please study Crane, Davidchack and Gorban\rf{CrDaGo16} {\em Minimal cover of
high-dimensional chaotic attractors by embedded coherent structures}, see how
their approach fits into our Poincar\'e section organization of the $L=22$
\KS\ attractor\rf{BuDiCv15}.

\item[2016-12-10 Xiong]
I just read Daniel \etal's paper\rf{CrDaGo16}. It is very enjoyable.
But it is quite different from cycle expansion theory.
They use Fourier mode projection (only their
magnitudes) to find a minimal cover of the attractor.
Also, the ideology of their algorithm relies on long orbits not
short orbits to represent the attractor. Short orbits will be pruned
at the second stage of the algorithm.

Look at their Fig.2. Only 4 Fourier modes (4 real + 4 imaginary)
will give the same information about the minimal cover.




\item[2016-12-13 Burak]
I agree with Xiong that \refref{CrDaGo16} departs from the spirit of
cycle expansions once it discards the short orbits in favor of longer ones.
However, I think with some modifications, this method can be useful for
us to have a starting point. If we just repeat the first stage of the
algorithm to find set of shortest orbits that ``cover'' the attractor,
not ``minimal'' in \refref{CrDaGo16} sense, then we would have finite number
of neighborhoods to workout local Poincar\'e sections. I'm pretty sure that
one of these neighborhoods is the one we studied in \refref{BudCvi15}.

\item[2017-02-23 Xiong]
I am rereading the chapter of discrete factorization and still
thinking about the how symmetries enter this factorization.
We have inserted
\[
  \intM{\ssp_e}
  \prpgtr{\ssp_e - f^t(\ssp_s)} = I
\]
in derivation of evolution operator.
We are sure that this identity is valid because for any
starting state
$\ssp_s$, there is a unique state $\ssp_e$ at which $\ssp_s$ will end
after time $t$.
Now suppose this system has a discrete symmetry $G=\{e, g_2,\cdots, g_n\}$.
Then if we are restricted in the fundamental domain, we have identity
\[
  \sum_{g\in G} \int_{\hat{\mathcal{M}}}{\delta \hat{\ssp}_e}
  \prpgtr{D(g)\hat{\ssp}_e - f^t(\hat{\ssp}_s)} = I
\]
because for any starting state $\ssp_s$ we can find an end state
$\hat{\ssp}_e$ and a unique group element such that
$D(g)\hat{\ssp}_e = f^t(\hat{\ssp}_s)$. So maybe we can define the evolution operator
as
\beq
  \Lop^t = \sum_{g\in G}
  \prpgtr{D(g)\hat{\ssp}_e - f^t(\hat{\ssp}_s)} e^{\beta A^t}
\ee{XiongWrongAgain}
But then in this case, we actually get a the symmetric irreducible representation.
I am confused how other irreducible representations enter this picture. How can
the characters show up ?

For continuous symmetries, we have identity.
\[
  \int_{\theta}d\theta\intM{\ssp_e}
  \prpgtr{g(\theta)\ssp_e - f^t(\ssp_s)} = I
\]

\item[2017-02-24 Predrag]
Now I understand - I explained the discrete symmetry factorization
in the famed lecture of
\HREF{http://chaosbook.org/course1/Course1w4.html}{28 Feb 2015(?)}
which Xiong watched in rapt admiration, but decided not to record for
posterity. And now he does not remember.

Just kidding, sort of.

For $C_3$, see \HREF{https://www.youtube.com/embed/Hjetytkb79E} {Week
13}~{\em Trace of evolution operator}. If you understand that, the rest
should be easier.

I do not think your guess \refeq{XiongWrongAgain} is correct, there is no sum
in the definition of the fundamental domain evolution operator, see ChaosBook
sect.~\HREF{http://chaosbook.org/chapters/symm.pdf} {\em 25.3 Dynamics in the
fundamental domain}

\item[2017-02-27 Xiong]
Now I understand why the formula is
\[
  \Lop^t =
  \prpgtr{D(h)\hat{\ssp}_e - f^t(\hat{\ssp}_s)} e^{\beta A^t} D^{reg}(h)
  \,.
\]
Here $h$ is the symmetry of the orbit. And assume the order (period)
of $h$ is $m$, \ie, $h^m = e$. Then we see that $\tr \Lop^t$ does
not contribute but $\tr \Lop^{mt}$ does. This makes sense for it
is equivalent to considering the \po\ in the full \statesp.
However, what if this orbit has two symmetries $h_1$ and $h_2$ whose
orders are different $m_1 \neq m_2$? There seems no simple trick to
add some $D^{reg}(?)$ to transform the evolution operator into
the fundamental domain.

\item[2015-06-01, 2017-03-03,2021-01-08 Predrag]
I still believe the deep way to understand the inertial manifold
dimension will rely on going beyond linearization,
\ie, differential geometry. Maybe this book does the trick:

Vargas\rf{Vargas14} {\em {Differential Geometry For Physicists And
Mathematicians: Moving Frames And Differential Forms: From Euclid Past
Riemann}}. I failed to download it via Georgia Tech library, linking to
\HREF{https://www.ebscohost.com/}{https://www.ebscohost.com/}, but
% that is very clunky - I only could download pages
% \HREF{http://ChaosBook.org/library/Vargas14-1.pdf} {i-80} and
% \HREF{http://ChaosBook.org/library/Vargas14-2.pdf} {81-180}, and that's it.
\HREF{https://book4you.org/}{book4you.org} works.

\item[2017-05-05 Predrag]
Balanov \etal\rf{BKKR17} {\em Hopf bifurcation of relative periodic
solutions: {Case} study of a ring of passively mode-locked lasers}
is a very detailed study of equivariant Hopf bifurcation of relative periodic
orbits from relative equilibria in systems of functional differential
equations respecting $\Gamma \times S^1$-spatial symmetries, applied to a
model of coupled identical passively mode-locked semiconductor lasers with
the dihedral symmetry group $\Dn{8} \times S^1$.


\end{description}
