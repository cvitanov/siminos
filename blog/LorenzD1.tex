% siminos/blog/LorenzD1.tex
% $Author: predrag $ $Date: 2016-04-27 17:14:34 -0400 (Wed, 27 Apr 2016) $


\section{Desymmetrization of Lorenz flow}
\label{sect:LorenzD1}
% Predrag 2016-04-11

                        \toCB

%%%%%%%%%%%%%%%%%%%%%%%%%%%%%%%%%%%%%%%%%%%%%%%%%%%%%%%%%%%%%%%%%%%%%%%
% extracted from ChaosBook
% \example{Equivariance of the Lorenz flow.}{\label{exam:3dinvDscr}
\index{Lorenz flow!symmetry}            \toCB
%(continued from \refexam{exam:3dCoordDscr})~~
The Lorenz equations\rf{lorenz}
\beq
    \left[
        \begin{array}{c}
\dot{x} \\ \dot{y} \\ \dot{z}
    \end{array}
    \right]
    =
    \left[
        \begin{array}{c}
\sigma (y-x) \\
\rho x - y -xz \\
xy -bz
    \end{array}
    \right]
    =
  \left(\barr{ccc}
    -\sigma  & \sigma &  0 \\
    \rho     &   -1   &  0 \\
       0     &    0   & -b
    \earr\right)
    \left[
        \begin{array}{c}
{x} \\ {y} \\ {z}
    \end{array}
    \right]
+
    \left[
        \begin{array}{c}
{0} \\ {-xz} \\ {xy}
    \end{array}
    \right]
\ee{LorenzA}
are equivariant under the action of the
order-2 group cyclic group $\Ztwo =
\{e,\Rot{1/2}\}$,
where $\Rot{1/2}$ is
$[x,y]$-plane, half-cycle rotation by $\pi$ about the $z$-axis,
%
%       } %end \example{Equivariance of the Lorenz flow
%%%%%%%%%%%%%%%%%%%%%%%%%%%%%%%%%%%%%%%%%%%%%%%%%%%%%%%%%%%%%%%%%%
% extracted from ChaosBook
% \example{Desymmetrization of Lorenz flow:}{ \label{exam:LorenzD1}
% Predrag                           04apr2008
% Predrag                           19jan2008
% moved to here from halcrow/blog/TEX/lorenz.tex
% \index{Lorenz flow}
%\PC{ChaosBook: start the discrete.tex example here, link to the intro example in
%    the flow.tex chapter}
%(continuation of  \refexam{exam:3dinvDscr})~~
%
\beq
(x,y,z) \to \Rot{1/2}(x,y,z) = (-x,-y,z) \,.
\ee{LorenzR}
$(\Rot{1/2})^2=1$ condition decomposes the \statesp\ into two linearly
irreducible subspaces $\pS = \pS^+ \oplus \pS^-$,
% the $z$-axis $\pS^+$ and the $[x,y]$ plane $\pS^-$,
with projection operators
\(
\PP^+ = \frac{1}{2}(1 + \Rot{1/2})\,,
\)
\(
 \PP^- = \frac{1}{2}(1 - \Rot{1/2})
 \,,
\)
onto the two subspaces given by
%(see \refsect{s:ProjOper})
 \beq
 \PP^+ % = \frac{1}{2}(1 + \Rot{1/2})
 =   \left(\barr{ccc}
    0  &  0 & 0  \\
    0  &  0 & 0 \\
    0  &  0 & 1
    \earr\right)
     ,\quad
 \PP^- % = \frac{1}{2}(1 - \Rot{1/2})
  =   \left(\barr{ccc}
    1  &  0 & 0  \\
    0  &  1 & 0 \\
    0  &  0 & 0
    \earr\right)
\,.
 \label{projOp:sig}
 \eeq
%
%\bea
%    \left(
%        \begin{array}{c}
%0 \\ 0 \\ \dot{z}
%    \end{array}
%    \right)
%&=&
%  \left(\barr{ccc}
%       0     &    0   &  0 \\
%       0     &    0   &  0 \\
%       0     &    0   & -b
%    \earr\right)
%    \left(
%        \begin{array}{c}
%{0} \\ {0} \\ {z}
%    \end{array}
%    \right)
%+
%    \left(
%        \begin{array}{c}
%{0} \\ {0} \\ \frac{1}{4}(x_+ + x_-)(y_+ + y_-)
%    \end{array}
%    \right)
%\continue
%    \left(
%        \begin{array}{c}
%\dot{x}_- \\ \dot{y}_- \\ 0
%    \end{array}
%    \right)
%&=&
%  \left(\barr{ccc}
%    -\sigma  & \sigma &  0 \\
%    \rho     &   -1   &  0 \\
%       0     &    0   &  0
%    \earr\right)
%    \left(
%        \begin{array}{c}
%{x_-} \\ {y_-} \\ 0
%    \end{array}
%    \right)
%+
%    \left(
%        \begin{array}{c}
%{0} \\ {-z\,x_- } \\ 0
%    \end{array}
%    \right)
%\label{Lorenz-+}
%\eea
%or
%
In the symmetry-reduced coordinates, the coupled ODEs \refeq{LorenzA}
decompose into a 2\dmn\ antisymmetric irrep and a 1\dmn\ symmetric
irrep.
\bea
    \left(
        \begin{array}{c}
\dot{x}_- \\ \dot{y}_-
    \end{array}
    \right)
&=&
  \left(\barr{cc}
    -\sigma  & \sigma \\
    \rho     &   -1
    \earr\right)
    \left(
        \begin{array}{c}
{x_-} \\ {y_-}
    \end{array}
    \right)
+
    \left(
        \begin{array}{c}
{0} \\ {-z\,x_- }
    \end{array}
    \right)
\continue
\dot{z}_+ \quad &=& -b\, {z}_+ + \frac{1}{4}x_-y_-
\,,
\label{LorenzDesym}
\eea
where $z_+ = z$ and $z_- = 0$.
As $(x,y)$ are odd under rotation by $\pi$, $({x}_+, {y}_+)=(0,0)$.
This example is too trivial to be instructive, all I have done is add
subscripts to the variables, but they are unchanged.
Oh, well.

We define the fundamental domain by the (arbitrary) condition $\hat{x}_-
\geq 0$, and whenever the trajectory $\ssp(\zeit)$ exits the domain, we
replace its functional dependence by the corresponding fundamental domain
coordinates,
\[(x_-, y_-) = \Rot{1/2}(\hat{x}_-, \hat{y}_-) = (-\hat{x}_-, -\hat{y}_-)
\quad \mbox{if} \quad {x_-}< 0
\,,
\]
and record that we have applied $\Rot{1/2}$ (that is the `reconstruction
equation' in the case of a discrete symmetry). When we integrate
\refeq{LorenzDesym}, the trajectory coordinates
$(\hat{x}_-(\zeit), \hat{y}_-(\zeit))$
are discontinuous whenever the trajectory crosses the fundamental domain
border. That, however, we do not care about - the only thing we need are
the \PoincSec\ points and the \Poincare\ return map in the fundamental
domain.


There are two kinds of compact (finite-time) orbits. \Po s
$\ssp(\period{p})=\ssp(0)$ are either self dual under rotation \Rot{1/2}
(a repeat of a \rpo), or appear in pairs related by \Rot{1/2}; in the
fundamental domain there is only one copy
$\sspRed(\period{p})=\sspRed(0)$. \Rpo s (or `\ppo s')
$\sspRed(\period{p})=\Rot{1/2}\sspRed(0)$ are \po s in the fundamental
domain.


\subsection{\PoincSec s}

\PoincSec\ hypersurface can be specified implicitly
by a single condition, through a function $\PoincC(\ssp)$ that is zero
whenever a point $\ssp$ is on the \PoincSec,
  \beq
\sspRed \in \PoincS \quad \mbox{iff}
\quad \PoincC(\sspRed) = 0 \,.
  \ee{PoincULor}
In order that there is only one copy of the section in the fundamental
domain, this condition has to be invariant,
$\PoincC(\LieEl\sspRed) = \PoincC(\sspRed)$
for $\LieEl \in \Group$, or, equivalently, the normal to it has to be
equivariant
\beq
\pde_j \PoincC( \LieEl \sspRed) =  \LieEl \pde_j \PoincC(\sspRed)
\quad \mbox{for} \quad \LieEl \in \Group
\,.
\ee{PoincSctEqvLor}

In case at hand, a choice of \PoincSec\  is rather easy.
By taking as a \PoincSec\ any  $\Rot{1/2}$-equivariant,
non-self-inter\-sect\-ing surface that contains the $z$ axis, the
\statesp\ is divided into a half-space fundamental domain
$\tilde{\pS}=\pS/\Ztwo$ and its $180^o$ rotation $\Rot{1/2}\tilde{\pS}$.
An example is afforded by the $\PoincS$ plane section of the Lorenz flow
given in ChaosBook.
% in \reffig{fig:LorenzSect}.
Take the  fundamental domain $\tilde{\pS}$ to
be the half-space between the viewer and $\PoincS$. Then the full Lorenz
flow is captured by re-injecting back into $\tilde{\pS}$ any trajectory
that exits it, by a rotation of $\pi$ around the $z$ axis.

As any such $\Rot{1/2}$-invariant section does the job, a choice of a
`fundamental domain' is here largely mater of taste. For purposes of
visualization it is convenient to make the double-cover nature of the
full \statesp\ by $\tilde{\pS}$ explicit, through any \statesp\
redefinition that maps a pair of points related by symmetry into a single
point.

\subsection{Invariant subspace}

As the flow is $\Ztwo$-invariant, so is its linearization $\dot{\ssp} =
\Mvar \ssp$. Evaluated at $\EQV{0}$, $\Mvar$ commutes with  $\Rot{1/2}$,
and
%, as we have already seen in  \refexam{exam:LorenzStab},
the $\EQV{0}$ {\stabmat} $\Mvar $ decomposes into $[x,y]$ and $z$ blocks.
    \PC{create ChaosBook example in  refsect{s:StabRpo} from the last
    sentence}

The 1\dmn\ $\pS^+$ subspace contains the fixed-point subspace, with the
$z$-axis points left \emph{point-wise invariant} under the group action
\beq
\Fix{\Ztwo} =
   \{ \ssp \in \pS \mid \LieEl  \, \ssp = \ssp \mbox{ for } g \in \{e,\Rot{1/2}\} \}
%\,.
\ee{dscr:LorFPsubsp}
(here $\ssp = (x,y,z)$ is a 3\dmn\ vector, not the coordinate $x$). A
\Ztwo-fixed point $\ssp(t)$ in $\Fix{\Ztwo}$ moves with time, but
% according to \refeq{flowInv}
remains within it, $\ssp(t) \in \Fix{\Ztwo}$, for
all times; the  subspace $\Fix{\Ztwo}$ is {\em flow invariant}.
In the case at hand, for $(x,y,z)=(0,0,z)$ the Lorenz equation
\refeq{LorenzDesym} is reduced to the exponential contraction to the
$\EQV{0}$ \eqv,
% \PC{pointer to turbulence chapter here}
\beq
\dot{z} = -b \, z
\,.
\ee{LorenzZaxis}
For higher-dim\-ens\-ion\-al flows the flow-invariant subspaces
can be high-dim\-ens\-ion\-al, with interesting dynamics of their own.
Even in this simple case this subspace plays an important role as a
topological obstruction: the orbits can neither enter it nor exit it, so
the number of windings of a trajectory around it provides a natural,
topological symbolic dynamics.

The $\pS^-/\EQV{0}$ subspace is, however, {\em not} flow-invariant, as the
nonlinear terms $\dot{z}=xy - bz$ in the Lorenz equation \refeq{LorenzA}
send all initial conditions within $\pS^-=(x_-(0),y_-(0),0)$ into the full
% , $z(t) \neq 0$
\statesp\  $\pS^+\cup\pS^-$.

\bigskip

The exercises that follow are about the basic
ideas of how one goes from finite groups to the continuous ones.
The main idea comes from discrete groups, the discrete Fourier transform
interpreted as the
{\em projection operators for cyclic group \Zn{N}}. The cyclic group
$C_N$ is generated by the powers of the rotation by $2\pi/N$, and in
general, in the $N\to\infty$ limit one only needs to understand the
algebra of $T_\ell$, generators of infinitesimal transformations,
$D(\theta) = 1+i \sum_\ell \theta_\ell T_\ell$. They turn out to be
derivatives.

The $N\to\infty$ limit of \Zn{N} gets you to the continuous Fourier
transform as a representation of $\Un{1}\simeq\SOn{2}$, but from then on
this way of thinking about continuous symmetries gets to be increasingly
awkward. So we need a fresh restart; that is afforded by matrix groups,
and in particular the unitary group $\Un{n}=\Un{1}\otimes\SUn{n}$, which
contains all other compact groups, finite or continuous, as subgroups.

    Reading: Chen, Ping and Wang\rf{ChPiWa89}
    {\em Group Representation Theory for Physicists},
    \HREF{http://chaosbook.org/library/Chen5-2.pdf}
    {Sect 5.2} {\em Definition of a Lie group, with examples}.

    Reading:
\HREF{http://ckw.phys.ncku.edu.tw/} {C. K. Wong} {\em Group Theory} notes,
\HREF{http://ckw.phys.ncku.edu.tw/public/pub/Notes/Mathematics/GroupTheory/Tung/Powerpoint/6._1DContinuousGroups.ppt}
{Chap 6 {\em 1D continuous groups}}, % (power point notes)
Sects. 6.1-6.3 {Irreps of $\SOn{2}$}. In particular, note that while
geometrically intuitive representation is the set of rotation $[2\!\times\!2]$
matrices, they split into pairs of 1\dmn\ irreps.
Also, not covered in the lectures, but worth a read: Sect. 6.6 completes
    discussion of Fourier analysis as continuum limit of cyclic groups
    $C_n$, compares $\SOn{2}$, % $\On{2}$,
    discrete translations group, and continuous translations group.


Going through these exercises might be helpful:

%%%%%%%%%%%%%%%%%%%%%%%%%%%%%%%%%%%%%%%%%%%%%%%%%%%%%%%%%%%%%%%%%
%\Problems{\On{2} reduction}{2016-04-12}
    \begin{exercises}
    \begin{enumerate}

\item

%%%%%%%%%%%%%%%%%%%%%%%%%%%%%%%%%%%%%%%%%%%%%%%%%%%%%%%%%%%%%%%%%
\exercise{Irreps of \SOn{2}.}{  \label{exer:SO2irrep}
Matrix
\beq
\Lg = \MatrixII{0}{-i}{i}{0}
\ee{SO2gen}
is the generator of rotations in a plane.
\begin{enumerate}
  \item Use the method of projection operators to show that for rotations
in the $k$th Fourier mode plane, the
irreducible $1D$ subspaces orthonormal basis vectors are
\[
\jEigvec[\pm k] = \frac{1}{\sqrt{2}}
            \left(\pm {\bf e}_1^{(k)} -i\,{\bf e}_2^{(k)}\right)
\,.
\]
How does $\Lg$ act on $\jEigvec[\pm k]$?
  \item
What is the action of the $[2\!\times\!2]$ rotation matrix
\[
 D^{(k)}(\theta)= \begin{pmatrix}
                \cos k\theta & -\sin k\theta\\
                \sin k\theta   & \cos k\theta
                  \end{pmatrix}
\,,\qquad k =1,2,\cdots
\]
on the $(\pm k)$th subspace \jEigvec[\pm k]?
  \item
What are the irreducible representations characters of \SOn{2}?
\end{enumerate}
} %end \label{exer:SO2irrep}
%%%%%%%%%%%%%%%%%%%%%%%%%%%%%%%%%%%%%%%%%%%%%%%%%%%%%%%%%%%%%%%%%

%%%%%%%%%%%%%%%%%%%%%%%%%%%%%%%%%%%%%%%%%%%%%%%%%%%%%%%%%%%%%%%%%
\exercise{Reduction of a product of two \SOn{2} irreps.}{ \label{exer:SO2Clebsch}
%    \item[Vvedensky exer. PS10.7]
%% http://www.cmth.ph.ic.ac.uk/people/d.vvedensky/groups/PS10.pdf
Determine the Clebsch-Gordan series for \SOn{2}. Hint: Abelian
group has 1\dmn\ characters. Or, you are just multiplying terms in
Fourier series.
      } %end \exercise{.}{ \label{exer:XXX}
%%%%%%%%%%%%%%%%%%%%%%%%%%%%%%%%%%%%%%%%%%%%%%%%%%%%%%%%%%%%%%%%%

%%%%%%%%%%%%%%%%%%%%%%%%%%%%%%%%%%%%%%%%%%%%%%%%%%%%%%%%%%%%%%%%%
\exercise{Irreps of \On{2}.}{  \label{exer:O2irrep}
\On{2} is a group, but not a Lie group, as in addition to continuous
transformations generated by \refeq{SO2gen} it has, as a group element, a
parity operation
\[
\sigma = \MatrixII{1}{0}{0}{-1}
\]
which cannot be reached by continuous transformations.
\begin{enumerate}
  \item Is this group Abelian, \ie, does  $\Lg$ commute with $R(k\theta)$?
Hint: evaluate first the
 $[\Lg,\sigma]$ commutator and/or show that
 \(
 \sigma D^{(k)}(\theta) \sigma^{-1} = D^{(k)}(-\theta) \,.
 \)
  \item
What are the equivalence classes of this group?
  \item
What are irreps of \On{2}? What are their dimensions?

Hint: \On{2} is the $n\to\infty$ limit of $D_n$, worked out in
\refexer{exer:irrepDn}
{\em Irreducible representations of dihedral group $\Dn{n}$}.
Parity $\sigma$ maps an \SOn{2} eigenvector into another eigenvector,
rendering eigenvalues of any $\On{2}$ commuting operator degenerate.
Or, if you really want to do it right, apply Schur's first lemma to
improper rotations
\[
 R^{'}(\theta)= \begin{pmatrix}
                \cos k\theta & -\sin k\theta\\
                \sin k\theta &  \cos k\theta
                  \end{pmatrix}
                \sigma
              = \begin{pmatrix}
                \cos k\theta &  \sin k\theta\\
                \sin k\theta & -\cos k\theta
                  \end{pmatrix}
\]
to prove irreducibility for $k\neq 0$.
  \item
What are irreducible characters of \On{2}?
  \item
Sketch a fundamental domain for \On{2}.
\end{enumerate}
} %end \label{exer:SO2irrep}
%%%%%%%%%%%%%%%%%%%%%%%%%%%%%%%%%%%%%%%%%%%%%%%%%%%%%%%%%%%%%%%%%

%%%%%%%%%%%%%%%%%%%%%%%%%%%%%%%%%%%%%%%%%%%%%%%%%%%%%%%%%%%%%%%%%
\exercise{Reduction of a product of two \On{2} irreps.}{ \label{exer:O2Clebsch}
%    \item[Vvedensky exer. PS10.7]
%% http://www.cmth.ph.ic.ac.uk/people/d.vvedensky/groups/PS10.pdf
Determine the Clebsch-Gordan series for \On{2}, \ie,
reduce the Kronecker product
\(
D^{(k)} \bigotimes D^{(\ell)}
\,.
\)
      } %end \exercise{.}{ \label{exer:XXX}
%%%%%%%%%%%%%%%%%%%%%%%%%%%%%%%%%%%%%%%%%%%%%%%%%%%%%%%%%%%%%%%%%

%%%%%%%%%%%%%%%%%%%%%%%%%%%%%%%%%%%%%%%%%%%%%%%%%%%%%%%%%%%%%%%%%
\exercise{A fluttering flame front.}{  \label{exer:O2KS}
\begin{enumerate}
  \item
Consider a linear partial differential equation for a real-valued field
$u=u(x,t)$ defined on a periodic domain $u(x,t) = u(x+L,t)$:
\beq
  u_t + u_{xx} + \nu u_{xxxx}=0
    \,,\qquad   x \in [0,L]
    \,.
\ee{ksLin}
In this equation $t \geq 0$ is the time and
$x$ is the spatial coordinate.
The subscripts $x$ and $t$ denote partial derivatives with respect to
$x$ and $t$:
$u_t = \partial u/d\partial $, $u_{xxxx}$ stands for the 4th spatial
derivative of
$u=u(x,t)$ at position $x$ and time $t$.
Consider the form of equations under coordinate shifts $x \to  x+\ell$
and reflection $x \to -x$. What is the symmetry group of \refeq{ksLin}?
  \item
Expand $u(x,t)$ in terms of its \SOn{2} irreducible components
(hint: Fourier expansion) and rewrite \refeq{ksLin} as
a set of linear ODEs for the expansion coefficients.
What are the eigenvalues of the time evolution operator? What is
their degeneracy?
  \item
Expand $u(x,t)$ in terms of its \On{2} irreducible components
(hint: Fourier expansion) and rewrite \refeq{ksLin} as
a set of linear ODEs.
What are the eigenvalues of the time evolution operator? What is
their degeneracy?
  \item
Interpret $u=u(x,t)$ as a `flame front velocity' and add a quadratic
nonlinearity to \refeq{ksLin},
\beq
  u_t + {\textstyle\frac{1}{2}}(u^2)_x + u_{xx} + \nu u_{xxxx}=0
    \,,\qquad   x \in [0,L]
    \,.
\ee{ks}
This nonlinear equation is known as the \KSe, a baby cousin of \NS.
What is the symmetry group of \refeq{ks}?
  \item
Expand $u(x,t)$ in terms of its \On{2} irreducible components
(see \refexer{exer:O2irrep}) and rewrite \refeq{ks} as
an infinite tower of coupled nonlinear ODEs.
  \item What are the degeneracies of the spectrum
of the eigenvalues of the time evolution operator?
\end{enumerate}

} %end {.}{  \label{exer:XXX}
%%%%%%%%%%%%%%%%%%%%%%%%%%%%%%%%%%%%%%%%%%%%%%%%%%%%%%%%%%%%%%%%%

%%%%%%%%%%%%%%%%%%%%%%%%%%%%%%%%%%%%%%%%%%%%%%%%%%%%%%%%%%%%%%%%%
\exercise{\On{2} fundamental domain.}{ \label{exer:O2fund}
You have $\mathrm{C}_{2}$ discrete symmetry generated by flip $\sigma$,
which tiles the space by two tiles.
\begin{itemize}
  \item Is there a subspace invariant under this $\mathrm{C}_{2}$? What
form does the tower of ODEs take in this subspace?
  \item How would you restrict the flow (the integration of the tower of
coupled ODEs) to a fundamental domain?
\end{itemize}
      } %end \exercise{.}{ \label{exer:XXX}
%%%%%%%%%%%%%%%%%%%%%%%%%%%%%%%%%%%%%%%%%%%%%%%%%%%%%%%%%%%%%%%%%

%\ProblemsEnd
\end{enumerate}
    \end{exercises}
