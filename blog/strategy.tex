% siminos/blog/strategy.tex
% $Author: predrag $ $Date: 2019-03-20 17:42:41 -0400 (Wed, 20 Mar 2019) $

\chapter{Strategy, to write up}

\begin{bartlett}{
Someone who makes the same mistake twice is not a wise man.
}
\bauthor{
An ancient Greek saying
    }
\end{bartlett}




\section{How to read me}

Throughout:  {\textdollar} on the margin
{\steady}
indicates that the text has been transferred to
articles and thesis siminos/*/,  or to ChaosBook.org desymmetrization
chapters, such as
\HREF{http://ChaosBook.org/continuous.pdf}
{continuous.pdf}.
%
For those whose Freud needs brushing up:
`Desymmetrization and its discontents' is a pun
on \HREF{http://en.wikipedia.org/wiki/Civilization_and_Its_Discontents}
{Civilization and its discontents}.

Here is a novice's guide to desymmetrization bloggery:
\begin{itemize}
  \item
How to read the running blog: go first to the latest blog post, end
of \refchap{c-DailyBlog}.
  \item
If you are reading an article of common interest (which does not fit into
one of the specialized topics), enter your notes into \refchap{c:lit}.
  \item
\refChap{c-freeze} \emph{Deep freeze} is the blog for the putative
\texttt{siminos/ksReduced/}
article, spearheaded by Evangelos
  \item
Comments to ChaosBook.org go into \refchap{chap:ChaosBook} blog.
  \item
Periodic orbit theory comments belong to \refchap{chap:UPO} {\em
Periodic orbit theory}.
  \item
If Hamiltonian dynamics is your obsession, that's in
\refchap{sect:LiePolice} {\em Lie police}.
  \item
Slicing all things laser should be confined to
\refchap{chap:lasers}{Laser physics: The lingo}
  \item
Symmetry reduction in fluid dynamics is in \refchap{chap:fluids}{Fluids}
  \item
Guys counting the number of degrees of freedom that capture the physics
of a `turbulent' PDE have gotten a divorce. All Kazzmania now resides in
\texttt{siminos/lyapunov/}.
  \item
Ditto for geophysicists. They reside in
\texttt{siminos/baroclinic/BrCv12.tex}.
  \item
Guys writing the ultimate guide to slicing for the woman on the street,
\texttt{siminos/atlas/}, blog in \refchap{chap:atlas}{\em Atlas}.
  \item
Plumbers who ponder how to slice experimental data blog in
\refchap{c-exp} {\em Symmetry reduction of experimental data}
  \item
Enter your ponderings on all things norm into \refsect{sect:norms}
\emph{Norms, distances}, though some of that is also in \refchap{c:lit}
(for experimental data) and \refchap{sect:LiePolice} (for symplectic
distances).
  \item
Cardiologists (mere electricians, really) have gotten a divorce, too. That
blog's gone to \texttt{DOGS/saldana/excite.tex}.
  \item
All things `{geometric phase}' are in \refchap{c-geometric} {\em
Geometric phase}.

\end{itemize}






\section{Email list}

create email list of our cohort:  Bristol, etc. participants

\subsection{Advertise arXiv rpo paper}

crosslink the paper with nonlin dynamics

email individually arXiv paper link to colleagues who might comment
    on the paper

\section{Classification, keywords}

						\noindent
elsevier (first number is Elsevier only? or? The 2nd is \textbf{PACS})
10.020: 02.20.-a Group theory	\\
10.050: 02.50.-r Probability theory, stochastic processes, and statistics 	\\
        02.70.Bf Finite-difference methods \\
10.150: 05.40.Ca Noise	\\
10.180: 05.45.-a Nonlinear dynamics and chaos	\\
10.190: 05.45.Ac Low-dimensional chaos	\\
10.210: 05.45.Gg Control of chaos, applications of chaos	\\
10.220: 05.45.Jn High-dimensional chaos	\\
10.230: 05.45.Mt Quantum chaos - semiclassical methods	\\
10.240: 05.45.Pq Numerical simulations of chaotic systems	\\
10.305: 05.10.Gg Stochastic analysis methods	\\
10.390: 05.70.Ln Nonequilibrium and irreversible thermodynamics	\\
20.030: 05.10.Gg Stochastic analysis methods	\\
20.080: 05.45.-a Nonlinear dynamics and nonlinear dynamical systems	\\
20.090: 05.45.Mt Semiclassical chaos (quantum chaos)	\\
        42.65.Sf Dynamics of nonlinear optical systems; optical instabilities,
                 optical chaos and complexity, and optical spatio-temporal dynamics \\
60.090: 46.70.-p Application of continuum mechanics to structures	\\
        47.10.Fg 	Dynamical systems methods (in Fluid Mechanics)	\\
        47.27.ed 	Dynamical systems approaches (turbulent flows)	\\
70.050: 47.27.-i Turbulent flows, convection, and heat transfer	\\
70.110: 47.52.+j Chaos (in fluid dynamics)	\\
70.130: 47.54.+r Pattern selection; pattern formation	\\
70.150: 47.60.+i Flows in ducts, channels, nozzles, and conduits	\\
70.160: 47.62.+q Flow control	\\
        83.60.Wc Flow instabilities \\
        95.10.Fh Chaotic dynamics



						\noindent
\textbf{keywords}	\\
symmetry reduction,	\\
equivariant dynamics,	\\
relative equilibria,	\\
relative periodic orbits,	\\
return maps,	\\
slices,	\\
moving frames,	\\
Hilbert polynomial bases,	\\
invariant polynomials,	\\
Lie groups	\\


\section{Papers to write}

\subsection{\emph{Physica D} ``Continuous symmetry...''}

\begin{description}

\item[2010-05-25 Vaggelis]
Will we follow editor's suggestion and resubmit with minor modifications only?
Deadline for receipt of the {\bf final} manuscripts is July 1st 2010.
Will we go for an arXiv version?

\item[2010-06-07 Predrag]
OK, now we have resubmitted, with minor edits. I think one should always submit
any article that is worth publishing also to arXiv;
it is open to anyone, rich or poor, and it is
more likely to reach the intended audience than only a publication through
any single journal.

\end{description}

\subsection{Reduced trace formulas?}

\begin{description}
 \item[2010-06-17 Vaggelis]
Since I have all rpo's up to level 7 for CLE I think I should try
to apply ``Continuous symmetry reduced trace formulas'' so that I get an incentive
to understand the paper. After all this group theory, it should be easier now.
 \item[2010-06-18 Predrag]
Would be nice if you did - both to understand the group theory better, and
also because I am not sure I have not missed some important detail about
invariant subspaces when I wrote the paper. Would be great to recycle KS
next, if CLE works.
\end{description}


\subsection{\emph{SIAM J. Appl. Dyn. Syst.}}

\begin{description}

\item[2010-06-07 Predrag] Next, the
``Continuous symmetry reduction of Kuramoto-Sivashinsky ...'' paper:
40,000 \rpo s and no place to go?
Can include movies and more graphics

\end{description}

\subsection{PRL on recycling energy}

\section{Write next NSF proposal }

\section{Spruce up personal websites}

\begin{description}

\item[2013-08-09 Predrag] Perhaps a way to move  blog to
webpages: Open Science with
\HREF{http://www.youtube.com/watch?v=D8pwr-Uizf4} {WorkingWiki}.

\item[2019-03-19 Predrag] But first consult the WorkingWiki
\HREF{https://www.mediawiki.org/wiki/Extension:WorkingWiki} {warning!}

\item[2013-11-24 Predrag]
Not only does
\HREF{http://scholar.google.com}
{scholar.google.com} work better than anything
for finding papers and their \texttt{.bib} citations, but now they
have launched
\HREF{http://scholar.google.com/scholar?scilib=1&hl=en&as_sdt=0,5}
{My library} which not only collects all of
\HREF{http://scholar.google.com/citations?user=utIBWvQAAAAJ&hl=en}
{my papers}, all citations
and all papers I've cited, but also enables me to save any paper and
assign group of papers labels. Google is taking us over.

\item[2013-08-09 Predrag] Asked my friend Bent-Erik Rasmussen
<ras@icmm.net> how he designs his homepages.

\item[2013-07-29 Bent-Erik]

\end{description}

I use \HREF{http://www.xara.com/designer-pro} {xara designer pro x}, by
\HREF{http://www.magix.com/us/} {Magix} - which makes graphics, photos
and audio - a German company that is very solid - a little old-fashioned,
not too fancy.

The good thing about this program is that it is based on character
features, but includes the necessary tools for image optimization and
sizing - can make 3D drawings and flash animations in a very simple way
(even if it's on the way out because OS systems do not support flash).

You could also use WordPress or Web editor at \texttt{One.com} host (that
I also use) - in general is the core course that websites today are
visually based to a much greater extent than they are text-based - and
it's too slow and laboriously typing in html - when there are other
solutions.

I also use Dreamweaver - it is too backward looking for today and not
very intuitive to use.

\subsection{ChaosBook goes eBook}
\begin{description}

                                        \toCB
\item[2013-07-29 Bent-Erik]
I have a gift for you: your ChaosBook as an active eBook,
\HREF{http://www.streamsound.dk/book1/chaos/chaos.html}
{www.streamsound.dk/book1/chaos/chaos.html},
created with \HREF{http://flippingbook.com/} {FlippingBook Publisher}

\item[2013-08-09 Predrag to Bent-Erik] this is now all in English, so my
collaborators can read it (aside to collaborators: Bent-Erik runs
\HREF{http://streamsound.dk} {http://streamsound.dk}, and if you are fond
of Schubert, he's \HREF{http://streamsound.dk/schubertselskabet/} {your
man} :)

\item[2013-08-12 Predrag]
\HREF{http://www.stitz-zeager.com/latex-source-code.html}
{www.stitz-zeager.com} has LaTeX source code for their math textbook;
might be useful in improving ChaosBook.org formatting? In particular,
batch processing file \texttt{genAlgTrig.bat} runs the whole version
for the webbook, and all their figures seem to be created by
\texttt{MetaPost}.

\item[2013-07-29 Bent-Erik]

\end{description}

If you have an Ipad - you can use safari and add the link to the
collection of icons using the arrow pointer, which is a kind of smart
bookmark feature that makes an icon. (We could also make an app - it's a
little more complicated, but I write my own apps).

I run small book publishing - and have for some time been working on
solutions for eBook formats - which are optimal in many ways.

Here I have converted your pdf to FlippingBook, which works well on most
desktop (all links are function - both in contents and in the
text). It has both a flash version (where the sides act like a real book),
a slightly  simpler html version (if the flash does not work), and
then at the same time a mobile version, which works fine on
Ipads - which I think is the right book format. It is also available in a
Kindle - but in general I think it's Ipad or tablet format that is
optimal - because it is the format people use, practical, that one
can have with everywhere - even to read in bed. One can both keep the
online version or download the offline version. One should use it a
bit to find out how the pages turn (use arrows or slide function with
mouse or finger) - how to zoom (double click on the page) - etc..

It provides a table of contents as a direct jump to the selected page
(table of contents on the left edge can be always seen) - and you can
create your own bookmarks (they will be remembered, as long as you open
them in the same browser), download the whole publication or the
individual pages - mark and paste text to clipboard, etc - it's actually
a really good solution.

For example, here is a link to
\HREF{http://www.streamsound.dk/book1/chaos/chaos.html\#284} {p.~267}.

I can also publish it (with ISBN number) as print
on demand - so there is no real investment (must be divided into two
volumes - max of a book is 700 pages) - and the company I work with does
also a eBook version and puts physical and eBook for sale on a lot
booksites so it actually is available.

The process is so flexible that you can easily maintain your distinction
between `stable' and later versions. That is probably important since
there are constantly new updates.

\begin{description}

\item[2013-08-12 Predrag to Bent-Erik]
I have attempted several times going to ``e-paper'' through
\HREF{shttp://www.scribd.com/birdtracks} {Scribd.com/birdtracks},
but your
\\
\HREF{http://www.streamsound.dk/book1/chaos/chaos.html}
{www.streamsound.dk/book1/chaos/chaos.html} is vastly superior.
Totally amazing, really!

A few questions:

\begin{enumerate}
    \item
The default size of the book is a bit too small on the screen.
Looking through \HREF{http://flippingbook.com/demos/}
{FlippingBook.com/demos} I note that the book itself, for example
\HREF{http://flippingbook.publ.com/college-algebra-textbook/}
{FlippingBook.publ.com/college-algebra-textbook}, is larger (screen
margins, also for the 'Full Screen' version, are much smaller).
  \item
The demo books are popped up (with little (x) in the upper right corner),
easier to close.
  \item
Next time you make the flipping book, can you call it
\\
\texttt{www.streamsound.dk/chaosbook/version14/index.html} ?
Current version is 14.4.2, one after your 14.4.1
  \item
In Denmark \HREF{http://www.visiolink.com/COMPANY/references.htm} {visiolink.com}
provides e-paper for many newspapers, such as
\HREF{http://www.e-pages.dk/ku/714/} {Uni Avisen}
  \item
German company \HREF{http://www.flipviewer.com/} {Flipviewer.com} seems to offer
the same, but is perhaps better in some ways - one can download the book in the
flip format, has help tips, it zooms as one turns the
mouse wheel, it uses all of the screen.
  \item
There is \HREF{http://www.flipsnack.com/flipsnackedu/fh3wx0ys} {Flipsnack.com},
but that is more basic. There seems to be much other software.

\end{enumerate}

\item[2013-08-11 Bent-Erik]
I just tried the eBook now and found that page 3 and 5 were unclear - so
maybe there is some testing after all.

\item[2013-08-12 Predrag] I fear that much more than
pages 3 and 5 you will find `unclear' :)

\item[2013-08-11 Bent-Erik]
We need two more pages: a
\HREF{http://www.uky.edu/Libraries/klp/gallery/art1977/colophon.jpeg}
{colophon page} (to place between the title page and your drawing) and a
final page (the back side of the book) with a short introduction.

\item[2013-08-12 Predrag] A colophon page I'll have to work on - I have
to get the right copyright statement... I do have a draft of
\HREF{http://chaosbook.org/backblurb.html} {ChaosBook.org/backblurb.html},
can turn it into the back side of the book pretty quickly...

\end{description}



\section{Zoteromania}

\begin{description}

\item[2008-07-18 Predrag] about webtools for generating BibTeX:
www.zotero.org
        will pick up most books from Amazon, etc; but
        better to find a book first on
\HREF{http://www.worldcat.org}{www.worldcat.org}
          or
\HREF{http://scholar.google.com}{scholar.google.com}, then zotero it
          in a collection, and export in BibTeX format

\item[2009-12-22 Evangelos]
setting up a cns group at zotero.org

\item[2011-08-16 Predrag] moved the instructions to siminos/bibtex/zotero.txt

\item[2013-09-10 Predrag]
What is the difference between \verb+\eqref{...} vs \ref{...}+?
\HREF{http://tex.stackexchange.com/questions/107422/what-is-the-difference-between-eqref-and-ref}
{Click here}. \verb+\eqref+ is defined in amsTeX.

\end{description}
