% siminos/blog/atlasRev.tex
% $Author: predrag $ $Date: 2013-11-15 15:54:52 -0500 (Fri, 15 Nov 2013) $

\section{Atlas paper revisions}
\label{sect:atlasRev}

{\color{red} enter here the internal discussion of referee comments}

\begin{description}

\item[2012-07-01 Predrag] Mein Gott. Will proud ignorance of fellow
plumbers ever end?

\end{description}

\subsection{Response 1.1}
\label{sect:Response1.1}

\begin{description}

\item[2012-07-01 Predrag] Not sure what to do with this report -
there are different paths to wisdom, and we cannot
write a different paper. Our point is that symmetry reduction is
necessary, so downplaying it is not what we want to do...

\bigskip

Referee: ``This paper is mainly a review of some techniques for symmetry reduction
that are useful for high-dimensional systems. In particular, the authors
present a method of constraining to a subspace called a "slice," or a set
of such subspaces. The method itself is not new, having been used in the
authors' references 3-6, but this paper presents a nice overview and
several interesting examples.''

\item[2012-07-16 Evangelos]
	Regarding novelty, I think that the idea of bringing together
	charts into an atlas has been proposed many times (including our
	own papers) but has not been implemented elsewhere. Should we
	emphasize this in the text?

{\bf [2012-07-16 Predrag]} to Evangelos: I have answered this question
several times in the blog above, but again, we presumably know the
literature, so
(1) can you give me one or more external references in which the
algorithm of constructing an atlas for the symmetry-reduced \statesp\ is
described?
(2) can you give me one or more external references in which the
$(d\!-\!2)$\dmn\ \emph{\poincBord} ${\cal S} \subset \PoincS$ is defined,
discussed, computed, explained?
(3) can you give me one or more references authored by us (or anyone
else) in which the \emph{ridge} is defined and constructed? I.e., a pair
of \KCedit{$(d\!-\!N)$\dmn\ } local \slice\ hyperplanes intersects in a
{ridge}, a \KCedit{$(d\!-\!N-\!1)$\dmn\ } hyperplane {\PoincS} of points
$\sspRed^*$ shared by a pair of charts and thus satisfying the \slice\
condition \refeq{PCsectQ0}.
{\bf [2012-07-17 Evangelos]} No, I can't and
that's exactly my point. This is all new but we do not emphasize it
properly, so it gives the impression of it all being review material.
I've slightly rephrased a sentence in the introduction to reflect novelty,
but I think we need to do more than this - emphasize it in the abstract,
conclusions and maybe introductory paragraph.
{\bf [2012-07-18 Predrag]} Now inserted bold novelty claims into the paper
and the response to the referees.

{\bf [2012-07-20 Evangelos]} Changed algorithm to prescription. As with
Poincar\'e sections, finding good templates cannot be done algoritmically,
it needs thinking.

Also, I would remove, if possible, credit to Chaosbook.org for figures -
people think that if this is already in a book, then it has to be review
material.
 {\bf [2012-07-17 Predrag]} Done.


\item[(0)  x ] 2012-07-01 Predrag:

\item[(3) |?|] 2012-07-17 Predrag:

\item[(4) |?|] 2012-07-01 Predrag:

\item[(4) |x|] 2012-07-01 Predrag:

\item[minor (1) |x|] {\bf [2012-07-18 Daniel]:} Removed all references to
61,506 dimensions and replaced it with D dimensions, specifying that D is
large or just saying ``high-dimensional'' wherever necessary. Also took
d and D which were used in the text for dimensionality and replaced them
all with D's. Wrote comment outlining this change in \texttt{response1.1.txt}.
{\bf [2012-07-18 Predrag]}
ChaosBook.org convention is to use $d$ for \statesp\ dimension, $D$ for
configuration space dimension; Navier-Stokes pipe, etc. Reverted to $d$
throughout.

\item[minor (3) |x|] {\bf [2012-07-17 Daniel]:} Modified the text to try
and clarify ``turbulence breaks all symmetries'' and explain that
solutions calculated in invariant subspaces have different dynamics and
may have nothing to do with turbulence. Included review of changes in
\texttt{response1.1.txt}.

\item[minor 6. p9 |x|] 2012-07-01 Predrag:
 referee does not understand the difference between the `method of
 co-moving frames' and the `method of connections',
 \reffig{fig:BeThMconnect}. Shouldn't we weave Evangelos clarification
 into the text proper? {\bf [2012-07-17 Evangelos]} Done.

\item[minor (9) |?|] 2012-07-01 Predrag: can you weave this into the text?: ``
\edit{
Blah, blah}
''

\item[minor (11) |?|] 2012-07-01 Predrag:


\item[2012-07-19 Evangelos] Referee says: I described the manuscript as
self-focussed because about one third of the references are to the
authors' work.

We respond:
We feel referee's pain, having recently refereed a paper with 12
self-references out of 44 total, many of them deliriously misplacing the
credit.
*   Now there are 9 self-references (from the list of 44) that authors
    deem essential to this paper. By comparison, our internal
    symmetry-reduction blog cites 306 references.

ES: ``having recently refereed a paper'' means that we (\ie\ Predrag)
refereed one such paper or that the referee refereed our paper?

\item[2012-07-19 Evangelos] The ref. says:

I do not understand the remarks about sections and strobing, at the end
of section II: isn't it obvious that a Poincare section is not a
projection? and is it not obvious that strobing is a Poincare section
with the section being defined by time taking certain values? I.e. in an
extended state space the two are equivalent, aren't they?

RESPONSE: Not sure what an `extended' state space is.
(1) Our students often internally visualize a (d-1)-dimensional Poincare
section as one views a 3 dimensional world projected onto a 2-dimensional
photo (from one photo one cannot reconstruct the 3 dimensions), so that
remark was meant for them.
(2) suppose you strobe a periodic orbit of period $\pi$ every second.
({\bf ES:} What does every second mean here?)
What you get is a set of points that erotically fills out a
1-dimensional loop, whereas *stationary* Poincare section is
0-dimensional.

ES: Regarding remark (2), I think the referee has in mind systems with
explicit time dependence where one usually considers time as an extra
depended variable and defines a stroboscopic Poincar\'e section.

Or maybe what he has in mind is that you could think of the stroboscopic
condition as defining implicitly a manifold in phase space, which you
then call a Poincar\'e section. But then one really needs to carefully
choose the strobing period in order to get something meaningful in the
case of dissipative chaotic or turbulent system (as Predrag's example
shows).

\item[2012-07-19 Predrag]  I now realize that `strobing' remarks confuse
rather than aid the reader, and have removed them from the text.

\item[2012-07-19 Evangelos] Removed: ``the bit rushed submission that
reviewers have received (graduate students firmly believe that papers are
written one day before the deadline)'' from response. Do you really want
to trust an anonymous reviewer with behind the scenes details?

\end{description}


\subsection{Response 1.2}
\label{sect:Response1.2}

\begin{description}

\item[2012-07-14 Predrag] In response to referee's plaint:
``I described the manuscript as self-focussed because about one third of
the references are to the authors work.'' I removed a bunch of
Cvitanovi\'ciania, including Froehlich and Cvitanovi\'c (2011) paper, with
a heavy heart - Stefan did this work as undergraduate...

{\bf [2012-07-16 Evangelos]} I think you should not remove this. This is were
		slicing is formulated as an extremum condition. Also,
		I think the reviewer is pissed off about not citing (his)
		fluid dynamics papers - he does not care about this one.

{\bf [2012-07-16 Predrag]} thanks - Froelich is back.


\end{description}

\subsection{Proofs}

\begin{description}

\item[2012-10-05 Predrag to Daniel, Evangelos] Apparently we have 48
hours starting Fri, Oct 5, 2012 at 8:21am. I'll submit the annotated; can
you guys go through

\texttt{siminos/atlas/Chaos-v2/proofCopy.pdf}

\noindent
and enter here (not the pdf file - I have a special `active' copy) in
this format
\\
(line \#) correction

I can see references are screwed up, go through them carefully

\end{description}


From cha@aip.org Fri, Oct 5, 2012

proofs fare now available for your review.

The publisher will take no further action until you review page proofs
and notify the production staff of any necessary changes.
\\

\noindent
Article Title: CARTOGRAPHY OF HIGH-DIMENSIONAL FLOWS:
               A VISUAL GUIDE TO SECTIONS AND SLICES\\
Manuscript Code \#: 12259CATR\\
AIP ID: 009295CHA\\
Corresponding Author: PREDRAG CVITANOVIC\\
Journal: Chaos\\
Section: FOCUS ISSUE: FIFTY YEARS OF CHAOS: APPLIED AND THEORETICAL\\

http://aipprod.aip.org/ap2/proof?id=009295CHA-26458

All proof corrections must be indicated, by you, in English, as
annotations in the PDF file. Follow these steps:

\begin{itemize}
      \item Open the PDF file, then choose
        \\
        Tools $\to$ Comment \& Markup $\to$ Callout Tool.
        \\
        If you are using Acrobat Reader X, use the Sticky Note tool.
      \item You should now be able to place your cursor anywhere within the
        PDF file and note any changes needed.
      \item Read your proofs thoroughly and check all text, illustrations,
        captions, equations, and tables.
      \item Indicate all corrections as annotations within the PDF file. Make
        sure you include all of your changes on the annotated proof that
        you return to us. Do not send us a revised manuscript, in lieu of
        an annotated PDF.
      \item Multiple rounds of proof corrections will not be accepted; the
        production staff cannot accommodate corrections communicated by you
        separately.
      \item The extent to which you have altered your original article will
        be evaluated. If it is determined that your corrections are
        extensive (new material, not copy editing errors), your article
        will be returned to the Editorial Office and this will delay
        publication.
      \item Do not use TeX formatting in your annotations on the PDF proofs.

      \item Please check each illustration and provide any feedback as
        annotations within the PDF file.
      \item The appearance of illustrations in the proofs directly
        reflect how they will appear in the final published paper online.
        Note that graphics may have been down-sampled to produce a more
        manageable file size for downloading by online users of the
        journal.  Actual viewing and output results will vary depending
        upon the reader's screen or output device resolution.  (Note that
        high-resolution versions of the graphics will be used to produce
        the printed version of the journal.)
      \item If illustrations appear in color in these proofs, then they will
        appear in color in the online journal.
      \item If, and only if, you have agreed to the color-figure charges,
        color illustrations will appear in color in the printed journal.
      \item If you expected one or more of your figures to appear in color,
        but they appear only in grayscale in these PDF proofs, please add a
        note to the PDF.

      \item Save the annotated PDF file.
\end{itemize}

RETURNING CORRECTIONS

      AIP's eProof\\
      website:  approve without corrections, or upload a marked PDF file,
      and upload new figure files as necessary.
      \\
      \\
      URL: http://pubpartner.aip.org/aipportal/proof\_login.jsp \\
      PIN -$\to$ 009295CHA-26458

\begin{description}
\item[2012-10-05 Daniel]
AIP COMMENT: Please provide the definition for �PDE.�

RESPONSE: Replace ``the construction of invariant, PDE discretization
independent state space coordinates,5'' with ``the construction of
invariant state space coordinates (independent of the discretization
scheme chosen to solve the relevant partial differential equation)'' or
just ``the construction of invariant state space coordinates''. PDE
doesn't appear anywhere else in the paper, so neglecting the independent
of PDE discretization part probably doesn't hurt the paper too much and
is in Ref. 5.
\\
\PCedit{PC: {[2012-10-08 Predrag]} I just wrote ``partial differential equation (PDE)'',
that's all they wanted.}
\\
\\
AIP COMMENT: Please update Ref. 10 with volume number, page number, and year if published. DB 10-8-2012

RESPONSE: Not sure where R. R. Kerswell, ``Misunderstanding the
turbulence in a pipe,'' (unpublished) came from. Is this some document
that somebody has actually seen? It's not a preprint that I can find
online. Somebody update it if possible. I (DB) emailed Rich about it, just in
case...
\\
{[2012-10-11 Predrag]} \PCedit{PC: A perfect solution: replaced \\
10 R. R. Kerswell (private communication) \\
by\\
10 Pringle, C.C.T., Duguet, Y. and Kerswell, R.R. ``Highly symmetric travelling waves in pipe flow''
Phil. Trans. Roy. Soc. A 367, 457-472, 2009.
}
\\
\\
AIP COMMENT: Please update Ref. 11 with volume number, page number, and year if published. DB 10-08-2012

RESPONSE: Has A. P. Willis, P. Cvitanovic, and M. Avila, ``Revealing the state space of turbulent pipe flow by symmetry reduction,'' J. Fluid Mech. (submitted); 734 e-print arXiv:1203.3701. been accepted yet? Don't see anything in this regard on arxiv.org nor the JFM website. Predrag, please update the reference if you have any inside information. Contacted Ashley and Marc in case they know something but haven't told Predrag.
\\
{[2012-10-11 Predrag]} it's to appear (we still have to return the revised version)\\
\\
AIP COMMENT: Please provide the publisher detail in Ref. 14. DB 10-08-2012

RESPONSE: I'm not sure what the official JFM citation format for websites
is but assuming that it is something like the IEEE format (Authors.
``Title.'' Internet: www.chaosbook.org/tutorials, date published [date
accessed]), the reference for chaosbook/tutorials is J. F. Gibson and P.
Cvitanovi{\'c}. ``Movies of Plane Couette Flow.'' Internet:
www.chaosbook.org/tutorials, 2012 [Oct. 5, 2012].
\\
\PCedit{{[2012-10-11 Predrag]} Just because {\em Chaos J} typesetting is brain damaged please
do not screw up references for all other sensible publications. I reverted
\refref{GibsonMovies} to what it has always been.}
%@ELECTRONIC{GibsonMovies,
%  author = {J. F. Gibson and P. Cvitanovi{\'c}},
%  year = {2012},
%  title = {Movies of plane {Couette}},
%  note = {{\wwwcb{/tutorials}}},
%  url = {http://www.chaosbook.org/tutorials/},
%  institution = {Georgia Inst. of Technology},
%  publisher = {Center for Nonlinear Science}
%}
\\
\\
AIP COMMENT: Please update Ref. 43 with volume number, page number, journal title, and year if published. DB 10-08-2012

RESPONSE: N. Vandersickel and D. Zwanziger, ``The Gribov problem and QCD
dynamics,'' \emph{Physics Reports} or \emph{Phys. Rep.},  in press;
e-print
http://www.sciencedirect.com/science/article/pii/S0370157312002074 or
maybe leave original arxiv link for eprint.
\\
\\
FURTHER COMPLAINTS: \\
(1) Fig. 4a got miniturized (in a pretty bad way!)! DB 10-08-2012\\
(2) Fig. 7a got miniturized (in a pretty bad way!)! DB 10-08-2012 \\
(3) List of steps in slicing method of section V. CHARTING THE SLICE got indented weird or at least different from submitted version. I think they look better the way we submitted them originally. DB 10-08-2012\\
(4) Figures 6, 8, 9, 10, and 11 got blown up (not horribly) but they looked better the way we submitted them. DB 10-08-2012 \\
(5) Figure 12 is slightly smaller than submitted. DB 10-08-2012\\
(6) Captions for Figures 9 and 10 have weird spacing between lines when a line has a symbol with a superscript. DB 10-08-2012\\
\\
There's probably more but it's 3 AM. I'll go through it with a fine toothed comb tomorrow morning.

Evangelos COMPLAINTS:

(line 41) space s $\rightarrow$ spaces DB 10-08-2012\\
(line 115) space s $\rightarrow$spaces DB 10-08-2012\\
(line 245) by Siminos${}^{17}$, $\rightarrow$ in Ref. 17 DB 10-08-2012\\
(line 301) solutions $\rightarrow$ solution DB 10-08-2012\\
(line 396) $z=-b z$ $\rightarrow$ $\dot{z}=-bz$ DB 10-08-2012\\
(line 427) Eq. (1) $\rightarrow$ (1) DB 10-08-2012\\
(line 485) prime should be superscripted DB 10-08-2012\\
(line 525) space s $\rightarrow$ spaces DB 10-08-2012\\
(caption of Fig. 6, second line) primes should be  superscripted DB 10-08-2012\\

\item[2012-10-06 Daniel]
\DBedit{(7) Lines 3-5: Should Evangelos's current affiliation with MPI be
noted in the author list or is CNS ok?}\\
\ESedit{ES: I've added MPI as second affiliation. If it is not a moral
thing to do Predrag please remove.}
\PCedit{PC: It is highly moral.}\\
(8) Line 34: Missing Oxford comma (that's what we use throughout). Should be ``...equilibriaCOMMA and infinite hierarchies...'' DB 10-08-2012\\
(9) Lines 41, 115, and 525: ``state spaces'' macro rendered as ``state space s''. Needs to be fixed. DB 10-08-2012\\
(10) Lines 75 and 366: We are using $\tau$ for time. $f^t(x)$ should be $f^\tau (x)$. DB 10-08-2012\\
(11) Lines 367: Again $t$ should be $\tau$. $\hat{f}^t$ is kind of rendered weird, almost like the $t$ is a superscript of the hat. DB 10-08-2012\\
(12) Line 95: ``...and thus are...'' should be ``...and, thus, are...''. Add commas. DB 10-08-2012\\
(13) Line 97: ``...turbulence, and capture...'' should be ``...turbulence and capture...''. Remove comma. DB 10-08-2012\\
(14) Line 104: ``...the only, or the most natural...'' should be ``...the only or the most natural...'' Remove comma. DB 10-08-2012\\
(15) Caption for Figure 1: ``...flow $x(\tau)$, and the N group...'' should be ``...flow $x(\tau)$ and the N group...'', no comma. DB 10-08-2012\\
(16) Line 125: ``revisited'' should be ``visited''. DB 10-08-2012\\
(17) Line 127: ``sections, with'' should be ``sections with''. No comma. DB 10-08-2012\\
(18) Line 245: ``Siminos�${}^{17}$'' should be ``Siminos,${}^{17}$'' comma instead of apostrophe. DB 10-08-2012\\
(19) Line 246: ``$\sigma = 10, e = 1/10.$'' should be ``$\sigma = 10$,
     and $e = 1/10.$'' DB 10-08-2012\\
(20) Line 254: Oxford comma missing ``...symbolic dynamics and in-depth...'' should be ``...symbolic dynamics, and in-depth...'' DB 10-08-2012\\
(21) Line 266: Might be improved by changing ``the group orbit'' to ``its own group orbit'' DB 10-08-2012\\
(22) Line 276: Remove comma after ``pose'' DB 10-08-2012\\
\DBedit{(23) Lines 337, 342, 346, 364, 387, Figure 7 caption, 413, 439, 442, Figure 10 caption, 448, 494, and 508: $t'$ looks weird like the t and the prime are not spaced correctly (as in for example in Fig. 5d).}\\
\ESedit{ES: I had to really look closely in order to see the difference. I think it's OK}\\
(24) Figure 6 caption: The primes are not superscripted in ``$t'_1...t'_N$'' DB 10-08-2012\\
(25) Figure 9 caption: There is no room between the primes and the parenthesis in $\hat{x}'^(n)$. It looks funny. Should be like in the figure. DB 10-08-2012\\
(26) Lines 404-407: The line spacing in this paragraph is weird. Seems like the superscripts are way too high and you get a bunch of white space between lines. DB 10-08-2012\\
(27) Line 416: Remove extra parenthesis after ``Fig. 10(a)'' DB 10-08-2012\\
\DBedit{(28)~THROUGHOUT: The hat over the script M for the reduced state space or charts is not centered over the M as it would be if you layed it out in LaTex. It is off to the left and looks weird.}\\
\ESedit{ES: Probably that was the best they could do, let it be.}\\
(29) Line 425: Replace the awkward ``but this seems not possible'' with ``but this does not seem possible''DB 10-08-2012\\
(30) Line 426: Replace ``hydrodynamics turbulence'' with ``hydrodynamic turbulence''. No s. DB 10-08-2012\\
(31) Figure 11 caption: primes after $x$s are not superscripted. DB 10-08-2012\\
(32) Line 471: DB Replace ``slices transversally a group orbit''
with ``slices a group orbit transversally''. Hmmm actually... is
tranversally even a word. Should it be transversely?\\
\ESedit{ES: It should be transversely.}\\
\PCedit{PC: why change it?} it is a standard adverb in
\HREF{http://mathworld.wolfram.com/TransversalIntersection.html}
     {\underline{ma-the-matics}}.
Now, if you were serious, you should have changed it everywhere, not only one occurrence.
\PCedit{Fortunately, you did not.}\\
(33) Lines 483-491: Line spacing is weird when there are superscripts in this paragraph. DB 10-08-2012\\
(34) Line 485: prime after $x$ is not superscripted. DB 10-08-2012\\
(35) Lines 508 and 511: The prime on $\hat{t}$ is superscripted on the instead of $\hat{t}$ DB 10-08-2012\\
(36) Lines 533-534: Don't split the number $60,000$ between lines. DB 10-08-2012\\
(37) Line 555: Replace ``flow, there'' with ``flow. There'' DB 10-08-2012\\

\item[2012-10-08 Daniel] Marked up proof and uploaded it to Dropbox.
There are a couple of issues that are still open which I left as red
DBedit's in the entries above. I am not totally sure what to do with
them. Could somebody take a look and fix these?
\item[2012-10-09 Evangelos] Went through Daniel's list. See ESedits.
\item[2012-10-11 Predrag] Did last cursory check, changed a few things.
See PCedits. Anyway, sent \texttt{proofEdits.pdf} back to the journal,
and that is final. Now we keep fingers crossed.
\item[2012-10-11 Predrag to Daniel] You might want to enter your new
edits into \texttt{atlas12.tex}; that will be on our home pages, and will
always look better than the mangled Chaos J. version...
\item[2012-10-23 Daniel] Went through the second proof, double checked that they
fixed what we asked them to from the first proof, added further comments and uploaded 
it to Drop Box as \texttt{proof2.pdf}. Basically only saw little minor cosmetic things. 
Also added all changes to \texttt{atlas.tex}. Do we want to do an update to 
\texttt{arxiv.org}?
\end{description}
