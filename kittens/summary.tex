% siminos/kittens/summary.tex      pdflatex CL18
% $Author: predrag $ $Date: 2020-09-20 16:02:24 -0400 (Sun, 20 Sep 2020) $

\section{Summary and discussion}
\label{s:summary}

In this paper we have analyzed the {\catlatt}  \refeq{dDCatsT}
linear symbolic dynamics. We now summarize our main findings.

%%%%%%%%%%%%%%%%%%%%%%%%%%%%%%%%%%%%%%%%%%%%%%%%%%%%%%%%%%%%%%%%%%%%%%%%
    \PC{2016-11-08} {
Say: THE BIG DEAL is

for $d$\dmn\ field theory, symbolic dynamics is not one temporal sequence
with a huge alphabet, but $d$\dmn\ {\spt} tiling by a finite alphabet

corresponding dynamical zeta functions
should be sums over $d$\dmn\ {\twots}, rather than $1$\dmn\ \po s
    }
%%%%%%%%%%%%%%%%%%%%%%%%%%%%%%%%%%%%%%%%%%%%%%%%%%%%%%%%%%%%%%%%%%%%%%%%

%%%%%%%%%%%%%%%%%%%%%%%%%%%%%%%%%%%%%%%%%%%%%%%%%%%%%%%%%%%%%%%%%%%%%%%%
    \PC{2016-11-10} {Curb you enthusiasm
{\bf How to think about matters {\spt}?}
%\label{s:introSpTemp}
text currently purged from the introduction:
\\

Laws of motion drive a spatially extended system (clouds, say) through a
repertoire of unstable patterns, each defined over a finite  {\spt}
region.

But in dynamics, we have no fear of the infinite extent in time. That is \po\
theory's\rf{DasBuch} raison d'\^{e}tre; the dynamics itself describes the
infinite time strange sets by a hierarchical succession of \po s, of longer
and longer, but always finite periods (with no artificial external
periodicity imposed along the time axis). And, since 1996 we know how to deal
with both spatially and temporally infinite regions by tiling them with
finite {\spt}ly periodic tiles\rf{Christiansen97,GHCW07}. More
precisely: a time periodic orbit is computed in a finite time, with period
\period{}, but its repeats ``tile'' the time axis for all times. Similarly, a
{\spt}ly periodic ``tile'' or ``\twot'' is computed on a finite
spatial region $L$, for a finite period \period{}, but its repeats in both
time and space directions tile the infinite spacetime.

Taken together, these open a path to determining exact solutions on
\emph{spatially infinite} regions.
This is important, as many turbulent flows of physical interest come equipped
with $D$ continuous spatial symmetries. For example, in a pipe flow at
transitional Reynolds number, the azimuthal and radial directions (measured
in viscosity length units) are compact, while the pipe length is infinite.
If the theory is recast as a $d$\dmn\ space-time theory,
\(d= D +1\,,\)
{\spt}ly translational invariant recurrent solutions are \dtors\
(and \emph{not} the $1$\dmn\ \po s of the traditional periodic orbit theory),
and the symbolic dynamics is likewise $d$\dmn\ (rather than what is
today taken for given, a single 1\dmn\ temporal string of symbols).

This changes everything. Instead of studying time evolution of a chaotic
system, one now studies the repertoire of {\spt} patterns allowed by
a given PDE.
To put it more provocatively: junk your old equations and look for guidance
in clouds' repeating patterns.
There is no more \emph{time} in this vision of nonlinear \emph{dynamics}!
Instead, there is the space of all {\spt} patterns, and the
likelihood that a given finite {\spt}ly pattern can appear, like the
mischievous grin of Cheshire cat, anywhere in the turbulent evolution of a flow.
A bold proposal, but how does it work?
\\

and thus a $d$\dmn\ {\spt} pattern is
mapped one-to-one onto a $d$\dmn\ discrete lattice state, symbolic
dynamics labelled configuration - a configuration very much like that of an
Ising model of statistical mechanics.
    } %end censored \PC{2016-11-10 Curb you enthusiasm
%%%%%%%%%%%%%%%%%%%%%%%%%%%%%%%%%%%%%%%%%%%%%%%%%%%%%%%%%%%%%%%%%%%%%%%%



\subsection{Discussion.}

\PC{2020-05-31} {
Politi and Torcini\rf{PolTor92} note that
a problem in reconstructing the statistical properties of an
{\spt\ H{\'e}non} attractor
is ensuring that all \twots\  used are embedded into the inertial manifold.
For instance, in the single H{\'e}non map, one
of the two fixed points is isolated and it does not belong to the strange
attractor.
    }

\PC{2019-06-26}{
Mramor and Rink\rf{MraRin12}
{\bf $d$-dimensional Frenkel-Kontorova lattice:}
Here, the goal is to find a
$d$-dimensional ``lattice configuration'' $x:\integers^d\to \reals$ that satisfies
\beq\label{RR}
V'(\ssp_i) - (\Delta \ssp)_i = 0 \  \ \mbox{for all} \ i\in \mathbb{Z}^d
\,.
\eeq
The smooth function $V: \mathbb{R} \to \mathbb{R}$
satisfies $V(\xi+1)=V(\xi)$ for all $\xi\in\reals$. It has the interpretation
of a periodic onsite potential.

I like their definition of the discrete Laplace operator
$\Delta:\mathbb{R}^{\mathbb{Z}^d}\to\mathbb{R}^{\mathbb{Z}^d}$, defined as
\beq\label{Lap}
(\Delta x)_i := \frac{1}{2d} \sum_{||j-i||=1} \!\! (\ssp_j - \ssp_i)
\ \mbox{for all} \ i \in \integers^d
\,.
\eeq
where $||i||:=\sum_{k=1}^{d}|i_k|$.
Thus, $(\Delta x)_i$ is the average of the quantity $\ssp_j-\ssp_i$ computed
over the lattice points that are nearest to that with index $i$, \ie, the
graph Laplacian\rf{Pollicott01,Cimasoni12} \refeq{gaphLapl} for the case
of hypercubic lattice, or the ``central difference operator''\rf{PerViv}.
  }

\PC{2019-06-26}{
Mramor and Rink\rf{MraRin12}: ``
Eq.~(\ref{RR}) is relevant for statistical mechanics, because it is
related to the Frenkel-Kontorova Hamiltonian lattice differential
equation
\beq \label{FKHam}
\frac{d^2 \ssp_i}{dt^2} + V'(\ssp_i) - (\Delta \ssp)_i = 0 \ \mbox{for all} \ i\in\mathbb{Z}^d.
\eeq
This differential equation describes the motion of particles under the
competing influence of an onsite periodic potential field and nearest
neighbor attraction. Eq.~(\ref{RR}) describes its
stationary solutions.
  }

While the setting is classical,
such classical field-theory advances offer new semi-classical
approaches to quantum field theory and many-body problems.
