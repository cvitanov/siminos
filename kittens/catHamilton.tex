% siminos/kittens/catHamilton.tex                   pdflatex CL18
% $Author: predrag $ $Date: 2020-12-19 00:52:16 -0500 (Sat, 19 Dec 2020) $

\section{\AW\ partition of the cat map \statesp}
\label{s:catMapHam}

    \PC{2020-12-17}{
Remove from here, relink, once incorporated in ChaosBook.
    }
As explained in the companion paper\rf{GHJSC16},
the deep problem with the \PV\ code prescription is that it does not
yield a generating partition; the borders (\ie, $\ssp_0$, $\ssp_1$ axes)
of their unit-square partition
$(\ssp_{\zeit-1},\ssp_{\zeit})\in(0,1]\times(0,1]$
do not map onto themselves, resulting in the infinity of, to us unknown,
grammar rules for {\inadmissible} symbol sequences.

This problem was resolved in 1967 by Adler and
Weiss\rf{AdWei67,ArnAve,AdWei70} who utilized the stable/\-unstable
manifolds of the fixed point at the origin to cover a unit area torus by
a two-rectangles generating partition; for the \PV\ cat map
\refeq{eq:StateSpCatMap}, such partition\rf{DasBuch} is drawn in
 reffig{fig:PVAdlerWeiss}. Following Bowen\rf{Bowen70}, one refers to
such parallelograms as `rectangles'; for details
see Devaney\rf{deva87}, Robinson\rf{Robinson12}, or
ChaosBook\rf{DasBuch}. Siemaszko and Wojtkowski\rf{SieWoj11} refer to
such partitions as the `Berg partitions', and Creagh\rf{Creagh94} studies
their generalization to weakly nonlinear mappings.

While Percival and Vivaldi were well aware of \AW\ partitions, they felt
that their ``coding is less efficient in requiring more symbols, but it
has the advantage of linearity.'' Our construction demonstrates that one
can have both:  an \AW\ generating cat map partition, and a linear code.
The only difference from the \PV\ formulation\rf{PerViv} is that one
trades the single unit-square cover of the torus of
\refeq{eq:StateSpCatMap} for the dynamically intrinsic, two-rectangles
cover, but the effect is magic - now every
infinite walk on the {\markGraph}
corresponds to a unique {\admissible} orbit $\{\ssp_{\zeit}\}$, and the
{\markGraph} generates all {\admissible} itineraries $\{\Ssym{\zeit}\}$.

To summarize:
an explicit \AW\ generating partition completely solves the Hamiltonian cat map
problem, in the sense that it generates all {\admissible} orbits.
Rational and irrational initial states generate periodic and ergodic
orbits, respectively\rf{PerViv87b,Keating91}, with every \statesp\ orbit
uniquely labeled by an {\admissible} bi-infinite itinerary of symbols
from alphabet \A.

    \PC{2020-02-08}{
Note:
$N_2=\Det\jMorb=({s}-2)({s}+2)$,
$N_3 %   = {s}^3-3{s}-2
    = ({s}-2)({s}+1)^2$,
$N_4 = ({s}-2)({s}+1)\,{s}^2$,
$N_5 = ({s}-2)(s^2+ s-1 )^2$.
I think the factorization is true for all $\cl{}$, as the $s=2$ Laplacian
has a zero mode (constant $\ssp_i$, I think).

an sequence of non-negative integers counting the orbits of
a map; the sequence of periodic points for that map.
    }

This derivation was based on the \AW\ generating partition, a clever
explicit visualization of the cat map dynamics, whose generalization to
several coupled maps (let alone spatially infinite coupled cat
maps lattice) is far from obvious: one would have to construct covers of
high-dimensional {\fundPip}s by sets of sub-volumes.
However, as Keating\rf{Keating91} explains, no such explicit generating
partition is needed to count cat map \po s.
