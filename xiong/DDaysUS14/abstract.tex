

\documentclass[12pt]{article}
\usepackage{amsfonts}
\usepackage{amssymb}
\usepackage{authblk}
\usepackage{amsmath}

\newcommand{\KS}{Kuramoto-Siva\-shin\-sky}
\newcommand{\KSe}{Kuramoto-Siva\-shin\-sky equation}

\renewcommand\Authfont{\scshape\small}
\renewcommand\Affilfont{\itshape\small}
\setlength{\affilsep}{1em}

\newcommand{\smalllineskip}{\baselineskip=5pt}
\renewenvironment{abstract}[0]{\small\rm
        \begin{center}ABSTRACT
        \\ \vspace{2pt}
        \begin{minipage}{6.0in}\smalllineskip
        \hspace{1pc}}{\end{minipage}\end{center}\vspace{-1pt}}
\newcommand{\emailaddress}[1]{\newline{\sf#1}}

\let\LaTeXtitle\title
\renewcommand{\title}[1]{\LaTeXtitle{\large\textsf{\textbf{#1}}}}

%%%TITLE
\title{Lyapunov exponents, Floquet exponents and covariant vectors of \KSe}
\date{}

%%AFFILIATIONS
\author[1]{Xiong Ding}
\author[2]{Daniel Crane}
\author[1]{Predrag Cvitanovi\'{c}}
\author[2]{Ruslan L. Davidchack}
\author[3]{Kazumasa A. Takeuchi}
\affil[1]{School of Physics, Georgia Inst. of Technology,\emailaddress{xding@gatech.edu}}
\affil[2]{Dept. of Mathematics, Univ. of Leicester,
          Leicester LE1 7RH, UK}
\affil[3]{Dept. of Physics, Univ. of Tokyo, Japan}

%%DOCUMENT
\begin{document}
\maketitle

%%PLEASE PUT YOUR ABSTRACT HERE
\begin{abstract}
The largest positive Lyapunov exponent has long been used as an
indicator of chaotic dynamics, and recently sets of `covariant Lyapunov vectors'
have been shown to determine the `physical dimension' of a strange
attractor~\cite{GiChLiPo12} of a spatially extended flow,
for ex. the \KS\ on a finite periodic
domain. Such system can have a large range of Lyapunov
exponents, some
 expanding rapidly along the few unstable eigen-directions,
 while others contract dramatically along the many stable eigen-directions.
It
is a challenge to calculate those exponents accurately; the
\textit{QR iteration}~\cite{Trefethen97} provides a practical way to
evaluate them.
\\\\

The evolution of the tangent space of a dynamic system is governed by
the Jacobian matrix $\delta x(x_0, t)=J^t(x_0)\,\delta x(x_0, 0)$.
For a periodic orbit, the eigenvalues of the Jacobian matrix are called the
Floquet multipliers, and the associated Floquet eigenvectors
define the invariant directions
of the tangent space. The eigen-equation cannot be solved in
the traditional way, as the magnitude of matrix elements in such
Jacobian matrix can easily range over 100 orders of magnitude.
We show that the \textit{periodic Schur
decomposition}~\cite{Bojanczyk92theperiodic} enables us to compute
{\em all} eigenvalues to a high precision, provided the dimension
(the number of spatial Fourier modes) is not too high.



\end{abstract}
%%THE END OF ABSTRACT

\begin{thebibliography}{99}
\small
\bibitem{GiChLiPo12}
    F. Ginelli, H. Chat\'{e}, R. Livi and A. Politi,
  \textit{Covariant {Lyapunov} vectors}, J. Phys. A 46,
 2013, 254005.

\bibitem{Trefethen97} L. N. Trefethen and D. Bau, \textit{Numerical Linear
    Algebra}, SIAM, Philadelphia, 1997.

\bibitem{Bojanczyk92theperiodic} A. Bojanczyk, G. Golub and P. Van Dooren,
    \textit{The periodic Schur decomposition. Algorithms and
    applications}, In Proc. SPIE Conference, 1992, 31-42.

\end{thebibliography}
\end{document}
