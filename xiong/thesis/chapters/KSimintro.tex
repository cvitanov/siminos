% siminos/xiong/thesis/chapters/KSimintro.tex
% $Author: xiong $ $Date: 2017-03-03 18:13:05 -0500 (Fri, 03 Mar 2017) $


As stated in \refsect{sect:diss},
dynamics in chaotic dissipative systems is expected to land, after a
transient period of evolution, on the inertial
manifold\rf{constantin_integral_1989,infdymnon,temam90,Foias1988a,Robinson1995},
which is a finite\dmn\ object in \statesp. 
This is true even for infinite\dmn\ systems described by partial
differential equations, and offers hope that their asymptotic dynamics
may be represented by a finite set of ordinary differential equations.

The existence of a finite\dmn\
inertial manifold has been established for systems such as the
\KS, the \cGL, and some reaction-diffusion
systems\rf{infdymnon}. For the \NS\ flows its
existence remains an open problem\rf{temam90},
but dynamical studies, such as the determination of sets of \po s embedded in
turbulent flows\rf{GHCW07,WiShCv15}, strengthen the case for a geometrical
description of turbulence.

In this chapter, we discuss the existence and the dimension
of the global attractor and
the inertial manifold for a particular one\dmn\ \KS\ system,
defined on a ``minimal domain''.
Our discussion has two parts. In \refsect{sect:ksrb},
we review the mathematical proof of the existence of a global attractor and
the rigorous bounds for the dimension of the inertial manifold.
In \refsect{sect:ksdim}, we determine the dimension of the inertial manifold
by the numerical study of \Fv s along pre/relative \po s for this particular system.
