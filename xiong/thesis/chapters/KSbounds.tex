\section{The existence of an inertial manifold}
\label{sect:ksrb}


As discussed in \refsect{subsec:IM}, an inertial manifold is the graph of a map
$\Phi : P_n\pS \mapsto Q_n\pS$, where $P_n$ is the projection to the eigenspace spanned
by the eigenvectors of $\partial_{xx} + \partial_{xxxx}$,
corresponding to its smallest $n$ eigenvalues. $\partial_{xx} + \partial_{xxxx}$
has eigenvalues $-q_k^2+q_k^4$ with
Fourier modes as eigenvectors. Thus the existence of an inertial manifold indicates that
a finite number of \lowFmode s can describe the asymptotic behavior
of this system. Any trajectory will be attracted to this manifold exponentially fast.
This expectation is also reflected in \refeq{eq:ksfourier}, where the velocity field
of a \highFmode, \ie, $a_k$ with large $k$, is dominated by the linear
part $( q_k^2 - q_k^4 )\, a_k$. Thus, \highFmode s are almost decoupled from other modes.

An inertial manifold exists in a system which possesses the \emph{strong squeezing property}
\eqref{eq:ssp1}--\eqref{eq:ssp4}. As shown in Theorem \ref{them:spectGap}, as long as
the spectrum of $\partial_{xx} + \partial_{xxxx}$ has a sufficiently large gap for some $n$,
then an inertial manifold exists. The $k$th eigenvalue of $\partial_{xx} + \partial_{xxxx}$
is $\lambda_k = -q_k^2+q_k^4$. so $\lambda_{n+1} - \lambda_n$ is unboundedly increasing
with respect to $n$. In one\dmn\ \KSe\ \refeq{eq:ks}, the nonlinear part of the velocity
field is $-uu_x$. If we can determine a Lipschitz constant $C_1$ for $-uu_x$, then surely
there is an $n$ such that $\lambda_{n+1} - \lambda_n > 4C_1$. As a result, an inertial
manifold exists.

\emph{Strong squeezing property} has been shown to hold for one\dmn\ \KSe\ (Theorems 3.2 and corollary 3.7 in \rf{FNSTks88}, Theorem 4 in \rf{Robinson-PLA1994}).
Moreover, Robinson (Corollary 5 in \rf{Robinson-PLA1994}) states that if there is an
absorbing ball with radius $O(L^{\alpha})$
\footnote{
  The big-O notation:
  $f(x) = O(g(x))$ if and only if there exists a positive real number $M$ and
  a real number $x_0$ such that $|f(x)| \le M |g(x)|$ for all $x \ge x_0$.
}
in the one\dmn\ \KSe, then an inertial manifold exists with dimension bounded above
by $O(L^{3\alpha/5 + 3/2})$. Also, Theorem \ref{them:attractor} says that a global attractor
exists if an absorbing set exists in the \statesp. Therefore,
both a global attractor and an inertial manifold exist, provided the existence
of an absorbing set in the \statesp, which is a Hilbert space with
the $L^2$ norm
\[
  \norm{u}_{2} = \left(\int_{0}^L u^2 dx \right)^{1/2}
  \,.
\]


\subsection{Rigorous upper bounds}

In 1985, Nicolaenko, Scheurer and Temam\rf{NSTks85} gave the first asymptotic
boundedness of the $L^2$ norm of $u(x,t)$ in the antisymmetric subspace,
showing the existence of  an absorbing ball
$S = \{u\,:\, \norm{u}_{2} \le C L^{5/2}\}$
for antisymmetric solutions. They also show that the Hausdorff dimension of the
global attractor is bounded above by $O(L^{13/8})$.
This antisymmetric assumption was later removed by
Goodman\rf{Good94}.
The estimate was improved by Collet \etal\rf{CEEksgl93}
who extended it to the whole \statesp\ and improved the
exponent from $5/2$ to $8/5$.
In 2006, Bronski and Gammbill\rf{bronski2005} gave a better upper bound
\begin{equation}
  \label{eq:ksbound1}
  \limsup_{t\to\infty}\norm{u}_2 = O(L^{\frac{3}{2}})
  \,.
\end{equation}
All these results were obtained through the Lyapunov function approach,
the main idea of which is to find an appropriate gauge function $\Phi(x)$:
\begin{equation}
  \label{eq:gauge}
  u(x, t) = v(x,t) + \Phi(x)
  \,
\end{equation}
such that the transformed field $\norm{v(x, t)}_{2}$ is bounded.
Upper bound \refeq{eq:ksbound1} is claimed to be
the best result one can get by the Lyapunov function approach.
On the other hand, by treating \KSe\ as a perturbation of Burgers' equation,
Giacomelli and Otto\rf{GiacoOtto05} proved that
\footnote{
  The little-O notation :
  $f(x) = o(g(x))$ if and only if there exists a
  a real number $x_0$ such that $|f(x)| \le M |g(x)|$ for all $x \ge x_0$ and
  for all positive real number $M$.
}
\begin{equation}
  \label{eq:ksbound2}
  \limsup_{t\to\infty}|\!|u|\!|_2 = o(L^{\frac{3}{2}})
  \,.
\end{equation}
Later on, Otto\rf{Otto09} shows that
\[
  \limsup_{t\to\infty} \frac{1}{T}\int_0^T dt \norm{|\partial_x|^\alpha u}^2
  = O(L\cdot\ln^{10/3} L)
  \,,\qquad 1/3 < \alpha \le 2
  \,,
\]
and claims that it is the optimal bound for the one\dmn\ \KSe.
The norm of the
time-averaged fractional derivatives of $u(x,t)$ is almost
proportional to $L^{1/2}$. Bronski and Gammbill\rf{bronski2005}
also claim that $1/2$ is believed to be the best possible exponent.

Based on an upper bound of size of the attracting set, we can estimate
the dimension of the inertial manifold.
In 1988, Foias, Nicolaenko, Sell and Temam\rf{FNSTks88} gave an upper bound
$O(L^{7/2})$ for the dimension of the inertial manifold.
Better estimate $O(L^{2.46})$ is given in \refref{FNSTks85, jolly_evaluating_2000}.
Based on bound \refeq{eq:ksbound2}, the upper bound can be further
improved to $o(L^{12/5})$\rf{GiacoOtto05}.

In the remaining part of this section, we show the existence of an absorbing
ball through the Lyapunov function approach. The main result comes from
the work of Collet \etal\rf{CEEksgl93}. Though the estimate is not
optimal, it demonstrates the general process of obtaining the upper bound
by choosing an appropriate gauge function in \refeq{eq:gauge}.

\subsection{Existence of an absorbing ball}

Due to Galilean invariance of the \KSe, we impose zero mean velocity
$\int_0^L u(x,t)dx = 0$ in the \statesp. Then we state that
\begin{theorem}
  There is an absorbing ball $S$ with radius
  $C L^{8/5}$ in the one\dmn\ \KSe\ defined on a periodic domain
  $[0, L]$. Here, $C$ is a constant.
  % Namely,
  % \begin{equation}
  %   \limsup_{t\to\infty} ||u(x, t)||_{L^2} \le C L^{8/5}
  %   \,.
  %   \label{eq:ks_absorbing_radius}
  % \end{equation}
  \label{them:ks_readius}
\end{theorem}
Here, for simplicity, we only provide the proof of
theorem \ref{them:ks_readius} for the antisymmetric case.
For the full proof, please refer\rf{CEEksgl93, jolly_evaluating_2000}.
That is , we impose that both $v(x,t)$ and $\Phi(x)$ in \refeq{eq:gauge}
have period $L$ and are antisymmetric:
$v(-x, t) = -v(x, t)$, $\Phi(-x) = -\Phi(x)$.
Then rewriting \KSe\ \refeq{eq:ks} in terms of $v(x,t)$, we have
\[
  v_t  = (-\partial_x^2 - \partial_x^4)(v+\Phi) - vv_x - v\Phi_x -\Phi v_x
  -\Phi \Phi_x
  \,.
\]
Multiplying on both sides of the above equation with $v$
and integrating over the whole domain,
we get
\begin{align}
  \frac{1}{2} \partial_t \int v^2
  & = \int v(-\partial_x^2 - \partial_x^4)(v+\Phi)
    - \int v^2v_x - \int  v^2\Phi_x - \int  v\Phi v_x - \int  v\Phi \Phi_x
    \nonumber \\
  & = \int v(-\partial_x^2 - \partial_x^4)(v+\Phi)
    - 0 - \int v^2\Phi_x + \frac{1}{2}\int v^2\Phi_x - \int v\Phi \Phi_x
    \nonumber \\
  & = \int v(-\partial_x^2 - \partial_x^4)(v+\Phi)
    - \frac{1}{2}\int v^2\Phi_x - \int v\Phi \Phi_x  \,.
    \label{eq:ks_v2}
\end{align}
In the above, $\int v^2 v_x$ vanishes because of the periodic boundary condition.
From \refeq{eq:ks_v2}, we see that the evolution of $\norm{v(x,t)}_{2}^2$ depends
on the choice of the gauge function. In order to bound $\norm{v(x,t)}_{2}$, we need
to bound the right side of \refeq{eq:ks_v2} for some carefully
chosen gauge function $\Phi(x)$. Before that, we define notation
\begin{equation}
  \label{eq:ks_inner}
  (v_1, v_2)_{\gamma \Phi} = \int v_1(\partial_x^2 + \partial_x^4 + \gamma \Phi_x)v_2
  \,.
\end{equation}
Therefore, \refeq{eq:ks_v2} becomes
\begin{equation}
  \label{eq:ks_v2_new}
  \frac{1}{2} \partial_t \int v^2  = -(v, v)_{\Phi/2} -(v, \Phi)_{\Phi}
  \,.
\end{equation}
To bound the right side of \refeq{eq:ks_v2_new}, we bring up two propositions.
\begin{proposition}
  There is a constant $K$
  and an antisymmetric gauge function $\Phi(x)$ such that for
  all $\gamma\in[\frac{1}{4}, 1]$ and all antisymmetric
  $v(x, t)$ one have inequalities
  \begin{equation}
    \label{eq:ks_prop_1}
    (v, v)_{\gamma\Phi} \ge \frac{1}{4} \int(v_{xx}^2 + v^2)
    \,\quad \text{and }
  \end{equation}
  \begin{equation}
    \label{eq:ks_prop_2}
    (\Phi, \Phi)_{\gamma\Phi} \le KL^{16/5}
    \,.
  \end{equation}
  %\label{prop:ks_1} % lower letter does not show up in chapter title
  \label{PROP:KS1}
\end{proposition}
\begin{proposition}
  Definition \refeq{eq:ks_inner} defines an inner product in the $L^2$ space.
  \label{prop:ks_2}
\end{proposition}
With these two propositions, we then continue from \refeq{eq:ks_v2_new},
\begin{align}
  \frac{1}{2} \partial_t \int v^2
  & \le -(v, v)_{\Phi/2} + (v, v)_{\Phi}^{1/2} (\Phi, \Phi)_{\Phi}^{1/2}
    \label{eq:ks_v2_step1} \\
  & = -(v, v)_{\Phi/2} + \left(\epsilon (v, v)_{\Phi}\right)^{1/2}
    \left(\frac{1}{\epsilon}(\Phi, \Phi)_{\Phi}\right)^{1/2}
    \nonumber \\
  & \le -(v, v)_{\Phi/2} + \frac{\epsilon}{2}(v, v)_{\Phi} +
    \frac{1}{2\epsilon}(\Phi, \Phi)_{\Phi}
    \nonumber \\
  & = -\int v\left((1 - \frac{\epsilon}{2})(\partial_x^2 + \partial_x^4)
    + (\frac{1}{2}-\frac{\epsilon}{2})\gamma \Phi_x \right)v +
    \frac{1}{2\epsilon}(\Phi, \Phi)_{\Phi} \nonumber \\
  & = -(1 - \frac{\epsilon}{2})
    (v, v)_{\Phi(\frac{1}{2}-\frac{\epsilon}{2})/(1 - \frac{\epsilon}{2})} +
    \frac{1}{2\epsilon}(\Phi, \Phi)_{\Phi} \nonumber \\
  & \le -(1 - \frac{\epsilon}{2}) \frac{1}{4} \int(v_{xx}^2 + v^2) +
    \frac{1}{2\epsilon}(\Phi, \Phi)_{\Phi}
    \label{eq:ks_v2_step2} \\
  & \le -(1 - \frac{\epsilon}{2}) \frac{1}{4} \int v^2 +
    \frac{1}{2\epsilon}(\Phi, \Phi)_{\Phi} \nonumber
    \,.
\end{align}
We set $\epsilon = 2/3$ to make
$(\frac{1}{2}-\frac{\epsilon}{2})/(1 - \frac{\epsilon}{2}) = 1/4$, so we get
\begin{equation}
  \label{eq:ks_v2_bound}
  \partial_t \int v^2 \le -\frac{1}{3} \int v^2 + \frac{3}{2}(\Phi, \Phi)_{\Phi}
  \,.
\end{equation}
Step \refeq{eq:ks_v2_step1} has used
proposition \ref{prop:ks_2} such that Cauchy-Schwarz inequality can be
applied : $(v, \Phi)^2 \le (v, v)(\Phi, \Phi)$. Step \refeq{eq:ks_v2_step2}
has used \refeq{eq:ks_prop_1} in proposition \ref{PROP:KS1}. Derivative
of $\int v^2$ is bounded as shown in \refeq{eq:ks_v2_bound} and note that
term $\frac{3}{4}(\Phi, \Phi)_{\Phi}$ does not depend on time, so applying
lemma \ref{lem:Gronwall}, we obtain
\begin{equation}
  \label{eq:eq:ks_v2_bound2}
  \int v(x, t)^2 dx \le e^{-t/3}\int v(x, 0)^2 dx + \frac{9}{2}(1-e^{t/3})
  (\Phi, \Phi)_{\Phi}
  \,.
\end{equation}
Since $u(x, t) = v(x, t)+ \Phi(x)$, we therefore obtained an absorbing ball in the \KSe\
centered at $\Phi(x)$
with radius $\rho > \rho_0$, where
\begin{equation}
  \label{eq:ks_radius}
  \rho_0 = 3\sqrt{\frac{(\Phi, \Phi)_{\Phi}}{2}}  = O (L^{8/5})
  \,.
\end{equation}
Here, we have used relation \refeq{eq:ks_prop_2}.
Actually, Jolly, Rosa,
and Temam\rf{jolly_evaluating_2000} constructed a gauge function
such that the $L^2$ norm of $\Phi$ is bounded
$\norm{\Phi}_2 = O(L^{3/2})$. Therefore, the center of the
absorbing ball can be moved to the origin. Thus
\[
  \limsup_{t\to\infty} \norm{u(x, t)}_2 =   O(L^{8/5}) +  O(L^{3/2}) = O(L^{8/5})
  \,.
\]
As shown in \refeq{eq:ks_radius},
the radius of the absorbing ball depends on the choice of the gauge function
$\Phi(x)$. To make the radius as small as possible,
we try to minimize the exponent in \refeq{eq:ks_prop_2} and at the same time
to meet the requirement of  \refeq{eq:ks_prop_1}. This requires us to
choose the gauge function with the least norm which also supports
\refeq{eq:ks_prop_2} and proposition \ref{prop:ks_2}.
Actually, proposition  \ref{prop:ks_2} is a natural corollary of proposition
\ref{PROP:KS1}. The reason is as follows.
Apparently, the bilinear form \refeq{eq:ks_inner} satisfies
relations
$(\lambda v_1 + \mu v_2, v_3)_{\gamma \Phi} = \lambda (v_1, v_3)_{\gamma \Phi}
+ \mu (v_2, v_3)_{\gamma \Phi}$ and
$(v_1, v_2)_{\gamma \Phi} = (v_2, v_1)_{\gamma \Phi}$. In order to
make it an inner product, we only need to
demonstrate that $(v, v)_{\gamma \Phi} \ge 0$
with equality if and only if $v = 0$. Equation \refeq{eq:ks_prop_1}
provides this relation. Therefore, we only
need to prove proposition \ref{PROP:KS1}, which is given in
appendix \ref{chap:ksproof}.
