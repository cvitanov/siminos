% siminos/xiong/thesis/chapters/overview.tex
% $Author: xiong $ $Date: 2017-03-05 17:40:12 -0500 (Sun, 05 Mar 2017) $


\section{Overview of the thesis and its results}

% \PC{2017-03-01 This is my first draft. Rewrite Burak text.
%     Make this into a concise overview, like we do for our papers. State
%     clearly what is NEW that you did, in a separate paragraph
%     }
This thesis is organized as follows.
In \refchap{chap:intro}, we recall some basic facts about dissipative
dynamical systems relevant to this thesis, the traditional types of fractal
dimensions, and introduce the concept of the inertial manifold.
We review the literature on estimating
the dimension of inertial manifolds by covariant
(Lyapunov) vectors, and related algorithms.
\refchap{chap:Symmetry} is devoted to the discussion of symmetries in
dynamical systems. A reader who is familiar with the group representation
theory can skip \refsect{sect:group}.
\refsect{sec:symReduce} discusses the slicing technique we use to reduce
continuous symmetries of dynamical systems, and the tangent dynamics in the
slice.
Methods of \refchap{chap:Symmetry} are a prerequisite to the calculations of
\refchap{chap:ks}, where we study invariant structures in the
symmetry-reduced \statesp\ of the one\dmn\ \KSe.
\refChap{chap:im} contains the main result of this thesis. We investigate the
dimension of the inertial manifold in the one\dmn\ \KSe\ using the
information about the linear stability of pre/relative \po s of the system.
\refChap{chap:ped} introduces our \ped\ algorithm, essential tool for
computation of the Floquet multipliers and \Fv s reported in
\refchap{chap:im}. Readers not interested in the implementation
details can go directly to \refsect{sect:applic}, where the
performance of the algorithm is reported.
We summarize our results and outline some future directions in
\refchap{chap:conc}.

The original contributions of this thesis are mainly contained in
\refchap{chap:im} and \refchap{chap:ped}. The \ped\ introduced
in \refchap{chap:ped} is capable of resolving Floquet
multipliers as small as $10^{-27\,067}$.
In \refchap{chap:im}, we estimate the dimension of an
inertial manifold from the \po s embedded in it, and verify that our results
are consistent with earlier work based on averaging \cLvs\ over ergodic
trajectories.
This calculation is the first of its kind on this subject that is not an ergodic
average, but it actually pins down the geometry of inertial manifold's
embedding in the \statesp. It opens a door to tiling inertial manifolds by
invariant structures of system's dynamics.
