% siminos/xiong/thesis/chapters/symGroup.tex
% $Author: predrag $ $Date: 2021-06-19 11:30:00 -0400 (Sat, 19 Jun 2021) $

\section{Group theory and symmetries: a review}
\label{sect:group}

In quantum mechanics, whenever a system exhibits some symmetry, the
corresponding symmetry group commutes with the Hamiltonian of this
system, namely, $[U(\LieEl), H] = U(\LieEl)H - HU(\LieEl) = 0$. Here
$U(\LieEl)$ denotes the operation corresponding to symmetry $g$ whose
meaning will be explained soon. The set of eigenstates with degeneracy
$\ell$, $\{\phi_1, \phi_2, \cdots, \phi_\ell\}$, corresponding to the same
system energy $H\psi_i = E_n\psi_i$, is invariant under the symmetry
since $U(\LieEl)\psi_i$ are also eigenvectors for the same energy.
This information helps us understand the spectrum of a Hamiltonian and
the quantum mechanical selection rules. We now apply the same idea to
the classical {\evOper} $\Lop^t(\ssp_e, \ssp_s)$
for a system $\flow{t}{\ssp}$ equivariant under a discrete symmetry group
$\Group=\{e, \LieEl_2, \LieEl_3,\cdots, \LieEl_{|\Group|}\}$ of order
$|\Group|$:
\begin{equation}
  \label{eq:equiva}
  \flow{t}{{D}\LieEl}\ssp)={D}(\LieEl)\,\flow{t}{\ssp} \quad \text{for}
  \quad \forall
  \LieEl\in\Group
  \,.
\end{equation}
We start with a review of some basic facts of
the group representation theory. Some examples of good references
on this topic are \refref{Hamermesh62, Tinkham}.

Suppose group $\Group$ acts on a linear space $V$ and function
$\rho(\ssp)$ is defined on this space $\ssp\in V$. Each element
$\LieEl\in\Group$ will transform point $\ssp$ to ${D}(\LieEl)\ssp$. At
the same time, $\rho(\ssp)$ is transformed to $\rho'(\ssp)$. The value
$\rho(\ssp)$ is unchanged after state point $\ssp$ is transformed to
${D}(\LieEl)\ssp$, so $\rho'({D}(\LieEl)\ssp) = \rho(\ssp)$. Denote
$U(\LieEl)\rho(\ssp)=\rho'(\ssp)$, so we have
\begin{equation}
  \label{eq:ogfx}
  U(\LieEl)\rho(\ssp) = \rho({D}(\LieEl)^{-1}\ssp)
  \,.
\end{equation}
This is how functions are transformed by group operations. Note, $D(\LieEl)$
is the representation of $G$ in the form of space transformation matrices.
The
operator $U(\LieEl)$, which acts on the function space, is not the same as
group operation ${D}(\LieEl)$, so \refeq{eq:ogfx} does not mean that
$\rho(\ssp)$ is invariant under $\Group$. \refExam{exam:C3matrRep} gives
the space transformation matrices of $\Zn{3}$.

%%%%%%%%%%%%%%%%%%%%%%%%%%%%%%%%%%%%%%%%%%%%%%%%%%%%%%%%%%%%%%%%%%%%%%
\exampl{A matrix representation of cyclic group $\Zn{3}$.}{
  \label{exam:C3matrRep}                                    \inCB
  A 3\dmn\ matrix representation of the 3-element cyclic group
  $\Zn{3}=\{e,C^{1/3},C^{2/3}\}$ is given by the three rotations by
  $2\pi/3$ around the $z$-axis in a 3\dmn\ \statesp,
  \bea
  {D}(e) &=&
  \begin{bmatrix}
    1 & & \\
    & 1 & \\
    & & 1
  \end{bmatrix}
  \,,\quad
  {D}(C^{1/3}) =
  \begin{bmatrix}
    \cos\frac{2\pi}{3}  & -\sin\frac{2\pi}{3} & \\
    \sin\frac{2\pi}{3}  & ~\cos\frac{2\pi}{3}  & \\
    & & 1
  \end{bmatrix}
  \,,
  \continue
  {D}(C^{2/3}) &=&
  \begin{bmatrix}
    \cos\frac{4\pi}{3}  & -\sin\frac{4\pi}{3}  & \\
    \sin\frac{4\pi}{3}  & ~\cos\frac{4\pi}{3}  & \\
    & & 1
  \end{bmatrix}
  \,.
  \nnu
  \eea
  %\index{matrix rep!cyclic group}
  ~~(continued in \refexam{exam:C3regularRep})
}
%%%%%%%%%%%%%%%%%%%%%%%%%%%%%%%%%%%%%%%%%%%%%%%%%%%%%%%%%%%%%%%%%%%%%%

\subsection{Regular representation}
An operator $U(\LieEl)$ which acts on an infinite\dmn\ function space
is too abstract to analyze.
We would like to represent it in a more familiar way.
Suppose there is a function $\rho(\ssp)$ with symmetry $\Group$ defined in full
\statesp\ $\pS$, then full \statesp\ can be decomposed as a union
of $|\Group|$ tiles each of which is obtained by transforming the fundamental
domain,
\begin{equation}
  \label{eq:domain}
  \pS =  \bigcup_{g\in \Group}g\pSRed
  \,,
\end{equation}
where $\pSRed$ is the chosen fundamental domain.
So $\rho(\ssp)$ takes $|G|$ different forms by \refeq{eq:ogfx} in each sub-domain
in \refeq{eq:domain}. Now, we obtained a natural choice of a set of bases in this
function space called the \emph{regular bases},
\beq
\label{eq:RegBasis}
\{ \rho_1^{reg}(\sspRed), \rho_2^{reg}(\sspRed), %\rho_3^{reg}(\sspRed),
\cdots, \rho_{|\Group|}^{reg}(\sspRed)\}
=
\{
\rho(\sspRed), \rho(\LieEl_2\sspRed), % \rho(\LieEl_3\sspRed),
\cdots, \rho(\LieEl_{|\Group|}\sspRed) \}
\,.
\eeq
Here, for notation simplicity we use
$\rho(\LieEl_i\sspRed)$ to represent $\rho(D(\LieEl_i\sspRed))$ without
ambiguity.
These bases are
constructed by applying $U(\LieEl^{-1})$ to $\rho(\sspRed)$ for each
$\LieEl\in\Group$, with $\sspRed$ a point in the fundamental domain.
The [$|G|\!\times\!|G|$] matrix representation of the
action of $U(\LieEl)$ in bases \refeq{eq:RegBasis} is called the \emph{(left)
regular representation} $D^{reg}(\LieEl)$. Relation \refeq{eq:ogfx} says that
$D^{reg}(\LieEl)$ is a permutation matrix, so each row or column has only one nonzero
element.

We have a simple trick to obtain the regular representation quickly.
Suppose the element at the $i$th row and
the $j$th column of $D^{reg}(\LieEl)$
is $1$. It means
$\rho(\LieEl_i\sspRed) = U(\LieEl) \rho(\LieEl_j\sspRed)$, which
is $g_i=\LieEl^{-1}g_j \implies g^{-1} = g_i g_j^{-1}$. Namely,
\beq
D^{reg}(\LieEl)_{ij} = \delta_{\LieEl^{-1},\, g_i g_j^{-1}}
\,.
\ee{eq:RegRep}
So if we arrange the
columns of the multiplication table by the inverse of the group elements,
then setting positions with $\LieEl^{-1}$ to 1 defines the regular
representation $D^{reg}(\LieEl)$. Note, the above relation can
be further simplified to $g = g_jg_i^{-1}$, but it exchanges the rows and
columns of the multiplication table, so $g = g_jg_i^{-1}$
should not be used to get $D^{reg}(\LieEl)$.
On the other hand, it is easy to see
that the regular representation of group element $e$ is always the identity matrix.

%%%%%%%%%%%%%%%%%%%%%%%%%%%%%%%%%%%%%%%%%%%%%%%%%%%%%%%%%%%%%%%%%%%%%%%
\begin{table}[h]
  \caption[ The multiplication tables of the $\Zn{2}$ and $\Zn{3}$]{
    The multiplication tables of the (a) group $\Zn{2}$ and (b) $\Zn{3}$.
  }
  \label{tab:C3MultTab}
  \begin{center}
    \centering
    (a)
    \begin{tabular}{c | c c}
      $\Zn{2}$         & $e$        & $\sigma^{-1}$  \\ \hline
      $e$              & $e$        & $\sigma$   \\
      $\sigma$ & $\sigma$  & $e$         \\
    \end{tabular}
    \quad
    (b)
    \begin{tabular}{c | c c c}
      $\Zn{3}$         & $e$        & $(C^{1/3})^{-1}$ & $(C^{2/3})^{-1}$ \\ \hline
      $e$              & $e$        & $C^{2/3}$ & $C^{1/3}$  \\
      $C^{1/3}$ & $C^{1/3}$  & $e$       & $C^{2/3}$   \\
      $C^{2/3}$ & $C^{2/3}$  & $C^{1/3}$ & $e$
    \end{tabular}
  \end{center}
\end{table}
%%%%%%%%%%%%%%%%%%%%%%%%%%%%%%%%%%%%%%%%%%%%%%%%%%%%%%%%%%%%%%%%%%%%%%%


%%%%%%%%%%%%%%%%%%%%%%%%%%%%%%%%%%%%%%%%%%%%%%%%%%%%%%%%%%%%%%%%%%%%%%
\exampl{The regular representation of cyclic group $\Zn{3}$.}{
  \label{exam:C3regularRep}                                    \inCB
  (continued from \refexam{exam:C3matrRep})~~
  Take an arbitrary function $\rho(\ssp)$ over the \statesp\ $\ssp \in \pS$, and
  define a fundamental domain $\pSRed$ as a 1/3 wedge, with axis $z$ as its
  (symmetry invariant) edge. The \statesp\ is tiled with three copies of the wedge,
  \[
    \pS =  \pSRed_1\cup\pSRed_2\cup\pSRed_3
    =  \pSRed\cup C^{1/3}\pSRed\cup C^{2/3}\pSRed
    \,.
  \]
  Function $\rho(\ssp)$ can be written as the 3\dmn\ vector of functions
  over the fundamental domain $\sspRed \in \pSRed$,
  \beq
  (\rho_1^{reg}(\sspRed),\rho_2^{reg}(\sspRed),\rho_3^{reg}(\sspRed))
  = (\rho(\sspRed),\rho(C^{1/3}\sspRed),\rho(C^{2/3}\sspRed))
  \,.
  \ee{eq:C3RegRep}
  The multiplication table of $\Zn{3}$ is given in \reftab{tab:C3MultTab}.
  By \refeq{eq:RegRep}, the regular representation matrices
  $D^{reg}(\LieEl)$ have `1' at the location of $\LieEl^{-1}$ in the
  multiplication table, `0' elsewhere. The actions of the operator
  $U(\LieEl)$ are now represented by permutations matrices (blank entries
  are zeros):
  \beq
  D^{reg}(e) = %&=&
  \begin{bmatrix}
    1 & & \\
    & 1 & \\
    & & 1 \\
  \end{bmatrix} \,,  \quad
  % \continue
  D^{reg}(C^{1/3}) = %&=&
  \begin{bmatrix}
    ~ & 1 & ~\\
    ~ & ~ & 1\\
    1 & ~ & ~ \\
  \end{bmatrix}\,,  \quad
  D^{reg}(C^{2/3}) =
  \begin{bmatrix}
    ~ & ~ & 1\\
    1 & ~ & ~\\
    ~ & 1 & ~ \\
  \end{bmatrix}
  \,.
  \label{eq:C3reg}
  \eeq
  %\index{regular rep!cyclic group}
} %end \exampl{A regular representation of cyclic group $\Zn{3}$.}{
%%%%%%%%%%%%%%%%%%%%%%%%%%%%%%%%%%%%%%%%%%%%%%%%%%%%%%%%%%%%%%%%%%%%%%

%%%%%%%%%%%%%%%%%%%%%%%%%%%%%%%%%%%%%%%%%%%%%%%%%%%%%%%%%%%%%%%%%%%%%%%
\begin{table}[h]
  \caption[The multiplication table of  $\Dn{3}$]{
    The multiplication table of  $\Dn{3}$, the group of symmetries of an equilateral
    triangle.
  }
  \label{tab:D3MultTab}
  \begin{center}
    \centering
    \begin{tabular}{c | c c c c c c}
      $D_3$ & $e$ & $(\sigma_{12})^{-1}$ & $(\sigma_{23})^{-1}$ & $(\sigma_{31})^{-1}$
      & $(C^{1/3})^{-1}$ & $(C^{2/3})^{-1}$ \\ \hline
      $e$ & $e$ & $\sigma_{12}$ & $\sigma_{23}$ & $\sigma_{31}$  & $C^{2/3}$ & $C^{1/3}$\\
      $\sigma_{12}$ & $\sigma_{12}$ & $e$ & $C^{1/3}$ & $C^{2/3}$  & $\sigma_{31}$ & $\sigma_{23}$\\
      $\sigma_{23}$ & $\sigma_{23}$ & $C^{2/3}$ & $e$ & $C^{1/3}$  & $\sigma_{12}$ & $\sigma_{31}$\\
      $\sigma_{31}$ & $\sigma_{31}$ & $C^{1/3}$ & $C^{2/3}$ & $e$  & $\sigma_{23}$ & $\sigma_{12}$\\
      $C^{1/3}$ & $C^{1/3}$ & $\sigma_{31}$ & $\sigma_{12}$ & $\sigma_{23}$ & $e$ & $C^{2/3}$   \\
      $C^{2/3}$ & $C^{2/3}$ & $\sigma_{23}$ & $\sigma_{31}$ & $\sigma_{12}$ & $C^{1/3}$ & $e$
    \end{tabular}
  \end{center}
\end{table}
%%%%%%%%%%%%%%%%%%%%%%%%%%%%%%%%%%%%%%%%%%%%%%%%%%%%%%%%%%%%%%%%%%%%%%%

%%%%%%%%%%%%%%%%%%%%%%%%%%%%%%%%%%%%%%%%%%%%%%%%%%%%%%%%%%%%%%%%%%%%%%
\exampl{The regular representation of dihedral group $\Dn{3}$.}{
  \label{exam:D3regularRep}                                    \inCB
  $\Dn{3} = \{ e, \sigma_{12}, \sigma_{23}, \sigma_{31}, C^{1/3}, C^{2/3}\}$
  represents the symmetries of a triangle with equal sides.
  $C^{1/3}$ and  $C^{2/3}$ are rotations by $2\pi/3$ and $4\pi/3$ respectively.
  $\sigma_{12}, \sigma_{23}$ and $\sigma_{31}$ are 3 reflections.
  The regular bases in this case are
  \[
    \left(\rho(\sspRed),\,\rho(\sigma_{12}\sspRed) ,\, \rho(\sigma_{23}\sspRed) ,\,
      \rho(\sigma_{31}\sspRed) ,\, \rho(C^{1/3}\sspRed) ,\, \rho(C^{2/3}\sspRed)\right)
    \,.
  \]
  It helps us obtain the multiplication table quickly by the following relations
  \begin{equation}
    \label{eq:C3relations}
    \sigma_{31} = C^{1/3}\sigma_{12} \,,\quad
    \sigma_{23} = C^{2/3}\sigma_{12}\,,\quad
    C^{1/3}\sigma_{12} = \sigma_{12}C^{2/3} \,,\quad
    C^{2/3}\sigma_{12} = \sigma_{12}C^{1/3}
    \,.
  \end{equation}
  The multiplication table of
  $\Dn{3}$ % = \{ e, \sigma_{12}, \sigma_{23}, \sigma_{31}, C^{1/3}, C^{2/3}\}$
  is given in \reftab{tab:D3MultTab}.
  By \refeq{eq:RegRep}, the 6 regular representation matrices
  $D^{reg}(\LieEl)$ have `1' at the location of $\LieEl^{-1}$
  in the multiplication table, `0' elsewhere.
  For example, the regular representation of the action of
  operators $U(\sigma_{23})$ and
  $U(C^{2/3})$ are, respectively:
  \[
    D^{reg}(\sigma_{23}) =
    \begin{bmatrix}
      0 & 0 & 1 & 0 & 0 & 0 \\
      0 & 0 & 0 & 0 & 0 & 1 \\
      1 & 0 & 0 & 0 & 0 & 0 \\
      0 & 0 & 0 & 0 & 1 & 0 \\
      0 & 0 & 0 & 1 & 0 & 0 \\
      0 & 1 & 0 & 0 & 0 & 0
    \end{bmatrix}
    \,,\quad
    D^{reg}(C^{1/3}) =
    \begin{bmatrix}
      0 & 0 & 0 & 0 & 1 & 0 \\
      0 & 0 & 0 & 1 & 0 & 0 \\
      0 & 1 & 0 & 0 & 0 & 0 \\
      0 & 0 & 1 & 0 & 0 & 0 \\
      0 & 0 & 0 & 0 & 0 & 1 \\
      1 & 0 & 0 & 0 & 0 & 0
    \end{bmatrix}
    \,.
  \]
  %%%%%%%%%%%%%%%%%%%%%%%%%%%%%%%%%%%%%%%%%%%%%%%%%%%%%%%%%%%%%%%%%%%%%%
} %end \exampl{The regular representation of dihedral group $\Dn{3}$.}{

\subsection{Irreducible representations}

$U(\LieEl)$ is a linear operator under the regular bases.
Any linearly independent combination of the regular bases can be used as
new bases, and then the representation of $U(\LieEl)$ changes respectively.
So we ask a question: can we find a new set of bases
\begin{equation}
  \rho^{irr}_i=\sum_j S_{ij}\rho^{reg}_j
  \label{eq:trans}
\end{equation}
such that the new representation $D^{irr}(\LieEl) = SD^{reg}(\LieEl)S^{-1}$ is block-diagonal
for any $\LieEl\in\Group$ ?
\begin{equation}
  D^{irr}(\LieEl) =
  \begin{bmatrix}
    D^{(1)}(\LieEl) & & \\
    & D^{(2)}(\LieEl) & \\
    & & \ddots \\
  \end{bmatrix}
  = \bigoplus_{\mu=1}^{r} d_\mu D^{(\mu)}(\LieEl)
  \,.
  \label{eq:irre}
\end{equation}
In such a block-diagonal representation, the
subspace corresponding to each diagonal block is invariant under $\Group$
and the action of $U(\LieEl)$ can be analyzed subspace by subspace.
It can be easily checked that for each $\mu$, $D^{(\mu)}(\LieEl)$ for all
$\LieEl\in\Group$ form another representation (\emph{irreducible
  representation}, or \emph{irrep}) of group $\Group$.
Here, $r$ denotes the total number of
irreps of $\Group$. The same irrep may show up more than once in the decomposition
\refeq{eq:irre}, so the coefficient $d_{\mu}$ denotes the number of its
copies.  Moreover, it is proved\rf{Hamermesh62} that $d_{\mu}$ is also equal to the dimension
of $D^{(\mu)}(\LieEl)$ in \refeq{eq:irre}.
Therefore, we have a relation
\[
  \sum_{\mu=1}^r d_\mu^2 = |G|
  \,.
\]

%%%%%%%%%%%%%%%%%%%%%%%%%%%%%%%%%%%%%%%%%%%%%%%%%%%%%%%%%%%%%%%%%%%%%%
\exampl{Irreps of cyclic group $\Zn{3}$.}{
  \label{exam:C3irReps}                                     \inCB
  (continued from \refexam{exam:C3regularRep})~~
  For $\Zn{2}$ whose multiplication table is in \reftab{tab:C3MultTab}, we can form
  the symmetric base $\rho(\sspRed) + \rho(\sigma\sspRed)$
  and the antisymmetric base $\rho(\sspRed) - \rho(\sigma\sspRed)$. You can verify
  that under these new bases, $\Zn{2}$ is block-diagonalized.
  We would like to generalize this symmetric-antisymmetric
  decomposition to the order 3 group $\Zn{3}$. Symmetrization
  can be carried out on any number of functions, but there is no obvious
  anti-symmetrization. We draw instead inspiration from the Fourier
  transformation for a finite periodic lattice, and construct from the
  regular bases \refeq{eq:C3RegRep} a new set of bases
  \bea
  \rho^{irr}_0(\sspRed) &=& \frac{1}{3}\left[
    \rho(\sspRed) ~+~ \rho(C^{1/3} \sspRed) ~+~ \rho(C^{2/3} \sspRed)
  \right]
  \label{eq:c3f1}\\
  \rho_1^{irr}(\sspRed) &=& \frac{1}{3}\left[
    \rho(\sspRed) + \omega\,\rho(C^{1/3} \sspRed) + \omega^2 \rho(C^{2/3} \sspRed)
  \right]
  \label{eq:c3f2}\\
  \rho_2^{irr}(\sspRed) &=& \frac{1}{3}\left[
    \rho(\sspRed) + \omega^2 \rho(C^{1/3} \sspRed) + \omega\,\rho(C^{2/3} \sspRed)
  \right]
  \label{eq:c3f2}
  \,.
  \eea
  Here $\omega = e^{2i\pi/3}$.
  The representation of group $\Zn{3}$ in this new bases is block-diagonal
  by inspection:
  \begin{equation}
    D^{irr}(e) =
    \begin{bmatrix}
      1 & & \\
      & 1 & \\
      & & 1 \\
    \end{bmatrix} \,,\quad
    D^{irr}(C^{1/3}) =
    \begin{bmatrix}
      1 & 0 & 0\\
      0 & \omega & 0\\
      0 & 0 & \omega^2 \\
    \end{bmatrix}  \,,\quad
    D^{irr}(C^{2/3}) =
    \begin{bmatrix}
      1 & 0 & 0\\
      0 & \omega^2 & 0\\
      0 & 0 & \omega \\
    \end{bmatrix}
    \,.
    \label{eq:c3irr}
  \end{equation}
  So $\Zn{3}$ has three 1\dmn\ irreps. Generalization to any
  $\Zn{n}$ is immediate: this is just a finite lattice Fourier transform.
}% end \exampl{Irreps of cyclic group $\Zn{3}$
%%%%%%%%%%%%%%%%%%%%%%%%%%%%%%%%%%%%%%%%%%%%%%%%%%%%%%%%%%%%%%%%%%%%%%

\paragraph{Character tables.}
Finding a transformation $S$ which simultaneously block-diagonalizes the
regular representation of each group element sounds difficult.
However, suppose it can be achieved and we obtain a set of irreps $D^{(\mu)}(\LieEl)$,
then according to Schur's lemmas\rf{Hamermesh62}, $D^{(\mu)}(\LieEl)$ must satisfy a set of
orthogonality relations:
\begin{equation}
  \label{eq:ortho}
  \frac{d_\mu}{|G|} \sum_g D_{il}^{(\mu)}(\LieEl) D_{mj}^{(\nu)}(g^{-1}) = \delta_{\mu \nu}
  \delta_{ij} \delta_{lm}
  \,.
\end{equation}
Denote the trace of irrep $D^{(\mu)}$ as $\chi^{(\mu)}$, which is referred to as
the \emph{character} of $D^{(\mu)}$. Properties of irreps can be derived from
\refeq{eq:ortho}, and we list them as follows:
\begin{enumerate}
\item The number of irreps is the same as the number of
  classes.
\item Dimensions of irreps satisfy
  $\sum_{\mu=1}^{r} d^2_\mu = |G| $
\item Orthonormal relation I :
  $\sum_{i=1}^{r} |K_i| \chi_i^{(\mu)} \chi_i^{(\nu)*} = |G|\delta_{\mu \nu} $. \\
  Here, the summation goes through all classes of this group, and $|K_i|$ is
  the number of elements in class $i$. This weight comes from the fact that
  elements in the same class have the same character. Symbol $*$ means
  the complex conjugate.
\item Orthonormal relation II :
  $\sum_{\mu=1}^{r} \chi_i^{(\mu)} \chi_j^{(\mu)*} = \frac{|G|}{|K_i|}\delta_{ij} $. \\
\end{enumerate}
The characters for all classes and irreps of a finite group are
conventionally arranged into a \emph{character table}, a square matrix
whose rows represent different
classes and columns represent different irreps.
Rules 1 and 2 help determine the number of irreps and
their dimensions. As the matrix representation of class $\{e\}$ is always the
identity matrix, the first row is always the dimension of the
corresponding representation. All entries of the first column are always 1,
because the symmetric irrep is always one\dmn. To compute the
remaining entries, we should use properties 3, 4 and the class multiplication
tables.   Spectroscopists conventions use labels $A$ and $B$ for
symmetric, respectively antisymmetric nondegenerate irreps, and
$E$, $T$, $G$, $H$ for doubly, triply, quadruply, quintuply degenerate irreps.

%%%%%%%%%%%%%%%%%%%%%%%%%%%%%%%%%%%%%%%%%%%%%%%%%%%%%%%%%%%%%%%%%%%%%%
\begin{table}[h]
  \caption[Character tables of $\Zn{2}$, $\Zn{3}$ and $\Dn{3}$]{
    Character tables of $\Zn{2}$, $\Zn{3}$ and $\Dn{3}$.
    The classes
    $\{\sigma_{12},\sigma_{13},\sigma_{14}\}$, $\{ C^{1/3}, C^{2/3} \}$
    are denoted $3\sigma$, $2C$, respectively.
  }
  \label{tab:D3charac}
  \centering
  \begin{tabular}{c|ccc}
    $\Zn{2}$ & $A$ & $B$ \\
    \hline
    $e$  & 1 & 1  \\
    $\sigma$ & 1 & -1
  \end{tabular}
  \qquad
  \begin{tabular}{c|ccc}
    $\Zn{3}$ & $A$ & \multicolumn{2}{c}{$E$}  \\
    \hline
    $e$ & 1 & 1  & 1 \\
    $C^{1/3}$ & 1 & $\omega$ & $\omega^2$ \\
    $C^{2/3}$ & 1 & $\omega^2$  & $\omega$
  \end{tabular}
  \qquad
  \begin{tabular}{c|ccc}
    $\Dn{3}$ & $A$ & $B$ & $E$ \\
    \hline
    $e$ & 1 & 1  & 2 \\
    $3\sigma$ & 1 & -1 & 0 \\
    $2C$   & 1 & 1  & -1
  \end{tabular}
\end{table}
%%%%%%%%%%%%%%%%%%%%%%%%%%%%%%%%%%%%%%%%%%%%%%%%%%%%%%%%%%%%%%%%%%%%%%

%%%%%%%%%%%%%%%%%%%%%%%%%%%%%%%%%%%%%%%%%%%%%%%%%%%%%%%%%%%%%%%%%%%%%%
\exampl{Character table of $\Dn{3}$.}{\label{exam:D3charTab}         \inCB
  (continued from \refexam{exam:D3regularRep})~~
  Let us construct \reftab{tab:D3charac}.
  one\dmn\ representations are denoted by $A$ and $B$, depending on
  whether the basis function is symmetric or antisymmetric with respect to
  transpositions $\sigma_{ij}$. $E$ denotes the two\dmn\ representation.
  As
  $\Dn{3}$ has 3 classes, the dimension sum rule $d_1^2+d_2^2+d_3^2 = 6$
  has only one solution $d_1=d_2=1$, $d_3=2$. Hence there are two one\dmn\
  irreps and one two\dmn\ irrep. The first row is $1,1,2$, and the first
  column is $1,1,1$ corresponding to the one\dmn\ symmetric representation. We take
  two approaches to figure out the remaining 4 entries. First, since $B$
  is an antisymmetric one\dmn\ representation, so the characters should be $\pm 1$.
  We anticipate $\chi^{B}(\sigma) = -1$ and can quickly figure out the
  remaining 3 positions. Then we check that the obtained table satisfies the
  orthonormal relations. Second, denote $\chi^{B}(\sigma)=x$ and
  $\chi^E(\sigma)=y$, then from the orthonormal relation of the second
  column with the first column and itself, we obtain $1+x+2y=0$ and
  $1+x^2+y^2=6/3$. Then we get two sets of solutions, one of which is
  incompatible with other orthonormal relations, so we are left with
  $x=-1$, $y=0$.
  Similarly, we can get the other two characters.
} %end \exampl{Character table of $\Dn{3}$.}{\label{exam:D3charTab}
%%%%%%%%%%%%%%%%%%%%%%%%%%%%%%%%%%%%%%%%%%%%%%%%%%%%%%%%%%%%%%%%%%%%%%

\subsection{Projection operator}
We have listed the properties of irreps and the
techniques of constructing a character table, but we still do not know how to
construct the similarity transformation $S$ which takes a regular representation into a
block-diagonal form. Think of it in another way,
each irrep is associated with an invariant subspace, so by
projecting an arbitrary function $\rho(\ssp)$ into its invariant subspaces, we
find the transformation \refeq{eq:trans}.
One of these invariant subspaces is $\sum_g \rho(g\sspRed)$, which is the basis of
the one\dmn\ symmetric irrep $A$. For $\Zn{3}$, it is \refeq{eq:c3f1}.
But how to get the others? We resort to the projection operator:
\begin{equation}
  \label{eq:projecIrre}
  P^{(\mu)}_{i} = \frac{d_\mu}{|G|}\sum_g \left(D^{(\mu)}_{ii} (\LieEl)\right)^* U(g)
  \,.
\end{equation}
It projects an arbitrary function into the $i$th basis of irrep
$D^{(\mu)}$ provided the diagonal elements of this representation
$D^{(\mu)}_{ii}$ are known. $ P^{(\mu)}_{i} \rho(\ssp) = \rho^{(\mu)}_i$.
Here, symbol $*$ means the complex conjugate. For unitary groups
$\left(D^{(\mu)}_{ii} (\LieEl)\right)^* = D^{(\mu)}_{ii} (\LieEl^{-1})$.
Summing $i$ in \refeq{eq:projecIrre} gives
\begin{equation}
  \label{eq:projectSum}
  P^{(\mu)} = \frac{d_\mu}{|G|}\sum_g \left(\chi^{(\mu)}(\LieEl)\right)^*U(g)
  \,.
\end{equation}
This is also a projection operator which projects an arbitrary function onto
the sum of the bases of irrep $D^{(\mu)}$.

Note, for one\dmn\ representations, \refeq{eq:projectSum}
is equivalent to \refeq{eq:projecIrre}. The projection operator is
known after we obtain
the character table, since the character of an one\dmn\ matrix is the matrix itself.
However, for two\dmn\ or higher\dmn\ representations, we need to know the
diagonal elements $D^{(\mu)}_{ii}$ in order to get the bases of invariant subspaces. That is to say,
\refeq{eq:projecIrre} should be used instead of \refeq{eq:projectSum} in this case.
\refExam{exam:D3irrepBases} illustrates this point. The two one\dmn\ irreps are obtained
by \refeq{eq:projectSum}, but the other four two\dmn\ irreps are obtained by
\refeq{eq:projecIrre}.

%%%%%%%%%%%%%%%%%%%%%%%%%%%%%%%%%%%%%%%%%%%%%%%%%%%%%%%%%%%%%%%%%%%%%%
\exampl{Bases for irreps of $\Dn{3}$.}{
  \label{exam:D3irrepBases}                                       \inCB
  (continued from \refexam{exam:D3regularRep})~~
  We use projection operator \refeq{eq:projectSum} to obtain the
  bases of irreps of $\Dn{3}$.
  From \reftab{tab:D3charac}, we have
  \begin{align}
    P^{A}\rho(\sspRed)
    & = \frac{1}{6}
      \left[
      \rho(\sspRed) + \rho(\sigma_{12}\sspRed) + \rho(\sigma_{23}\sspRed)
      + \rho(\sigma_{31}\sspRed) + \rho(C^{1/3} \sspRed) + \rho(C^{2/3} \sspRed)
      \right] \\
    P^{B}\rho(\sspRed)
    & = \frac{1}{6}
      \left[
      \rho(\sspRed) - \rho(\sigma_{12}\sspRed) - \rho(\sigma_{23}\sspRed)
      - \rho(\sigma_{31}\sspRed) + \rho(C^{1/3} \sspRed) + \rho(C^{2/3} \sspRed)
      \right]
      \,.
  \end{align}
  For projection into irrep E, we need to figure out the explicit
  matrix representation first. Obviously, the following 2 by 2 matrices are E irreps.
  \begin{equation}
    D^E(e) =
    \begin{bmatrix}
      1 & 0\\
      0 & 1 \\
    \end{bmatrix} \,,\quad
    D^E(C^{1/3}) =
    \begin{bmatrix}
      \omega & 0 \\
      0 & \omega^2 \\
    \end{bmatrix}  \,,\quad
    D^E(C^{2/3}) =
    \begin{bmatrix}
      \omega^2 & 0 \\
      0 & \omega \\
    \end{bmatrix}  \,
    \label{eq:c3E_1}
  \end{equation}
  \begin{equation}
    D^E(\sigma_{12}) =
    \begin{bmatrix}
      0 & 1 \\
      1 & 0\\
    \end{bmatrix} \,,\quad
    D^E(\sigma_{23}) =
    \begin{bmatrix}
      0 & \omega^2 \\
      \omega & 0 \\
    \end{bmatrix}  \,,\quad
    D^E(\sigma_{31}) =
    \begin{bmatrix}
      0 & \omega  \\
      \omega^2 & 0 \\
    \end{bmatrix}  \,.
    \label{eq:c3E_2}
  \end{equation}
  So apply projection operator \refeq{eq:projecIrre} on $\rho(\sspRed)$ and
  $\rho(\sigma_{12}\sspRed)$, we get
  \begin{align}
    P^E_1\rho(\sspRed)
    & = \frac{1}{6}
      \left[
      \rho(\sspRed) + \omega \rho(C^{1/3} \sspRed) + \omega^2 \rho(C^{2/3} \sspRed)
      \right] \\
    P^E_2\rho(\sspRed)
    & = \frac{1}{6}
      \left[
      \rho(\sspRed) + \omega^2 \rho(C^{1/3} \sspRed) + \omega \rho(C^{2/3} \sspRed)
      \right]  \\
    P^E_1\rho(\sigma_{12}\sspRed)
    & = \frac{1}{6}
      \left[
      \rho(\sigma_{12}\sspRed) + \omega \rho(\sigma_{31} \sspRed) + \omega^2 \rho(\sigma_{23} \sspRed)
      \right] \\
    P^E_2\rho(\sigma_{12}\sspRed)
    & = \frac{1}{6}
      \left[
      \rho(\sigma_{12}\sspRed) + \omega^2 \rho(\sigma_{31} \sspRed) + \omega \rho(\sigma_{23} \sspRed)
      \,.
      \right]
  \end{align}
  The above derivation has used formulas \refeq{eq:C3relations}.
  Under the invariant bases
  \[
    \left\{
      P^A\rho(\sspRed), P^B\rho(\sspRed), P^E_1\rho(\sspRed), P^E_2\rho(\sigma_{12}\sspRed),
      P^E_1\rho(\sigma_{12}\sspRed),  P^E_2\rho(\sspRed)
    \right\}
    \,,
  \]
  we have
  \[
    D^{irr}(\sigma_{23}) =
    \begin{bmatrix}
      1 & 0 & 0 & 0 & 0 & 0 \\
      0 & -1 & 0 & 0 & 0 & 0 \\
      0 & 0 & 0 & \omega^2 & 0 & 0 \\
      0 & 0 & \omega & 0 & 0 & 0 \\
      0 & 0 & 0 & 0 & 0 & \omega^2 \\
      0 & 0 & 0 & 0 & \omega & 0 \\
    \end{bmatrix}
    \quad
    D^{irr}(C^{1/3}) =
    \begin{bmatrix}
      1 & 0 & 0 & 0 & 0 & 0 \\
      0 & 1 & 0 & 0 & 0 & 0 \\
      0 & 0 & \omega & 0 & 0 & 0 \\
      0 & 0 & 0 & \omega^2 & 0 & 0 \\
      0 & 0 & 0 & 0 & \omega & 0 \\
      0 & 0 & 0 & 0 & 0 & \omega^2 \\
    \end{bmatrix}
    \,.
  \]
} % \exampl{Basis for irreps of $\Dn{3}$.}{\label{exam:D3irrepBases}
%%%%%%%%%%%%%%%%%%%%%%%%%%%%%%%%%%%%%%%%%%%%%%%%%%%%%%%%%%%%%%%%%%%%%%

The $C_3$ and $D_3$ examples used in this section can be generalized
to any $C_n$ and $D_n$. For references, \refExam{exam:CnChars}, \refexam{exam:DnOddChars}
and \refexam{exam:DnEvenChars} give the character tables of $C_n$ and $D_n$.

%%%%%%%%%%%%%%%%%%%%%%%%%%%%%%%%%%%%%%%%%%%%%%%%%%%%%%%%%%%%%%%%%%%%%%
\exampl{Character table of cyclic group \Zn{n}.}{\label{exam:CnChars}
                                                                     \inCB
  The symmetry under a discrete rotation by angle $2\pi/n$ gives birth to a
  cyclic group $\Zn{n}=\{e,C_n,C_n^2,\cdots,C_n^{n-1}\}$.
  Since $\Zn{n}$ is Abelian, each element forms a separate class, and thus
  $\Zn{n}$ has $n$ one\dmn\ irreducible representations. The
  characters multiply as group elements:
  \(\chi_\alpha (C_n^i)\chi_\alpha
  (C_n^j)=\chi_{\alpha} (C_n^{i+j})
  \, \mod n
  \,.
  \)
  Therefore, we get \reftab{tab:CnChars}.
}% end \exampl{Character table of \Zn{n}.}{\label{exam:DnChars}
%%%%%%%%%%%%%%%%%%%%%%%%%%%%%%%%%%%%%%%%%%%%%%%%%%%%%%%%%%%%%%%%%%%%%%

%%%%%%%%%%%%%%%%%%%%%%%%%%%%%%%%%%%%%%%%%%%%%%%%%%%%%%%%%%%%%%%%%%%%%%
\begin{table}[h]
  \caption[Character table of cyclic group \Zn{n}]{
    Character table of cyclic group \Zn{n}. Here $k,j=1,2,\cdots,n-1$.
  }
  \label{tab:CnChars}
  \begin{center}
    \begin{tabular}{c|cr}
      $\Zn{n}$       & $A$& $\Gamma_{j}$ \\
      \hline
      $  e  $        &   1  &   1  \\
      $C^{k}_{n} $        &   1  &   $\exp(\frac{i2\pi kj}{n})$   \\
    \end{tabular}
  \end{center}
\end{table}
%%%%%%%%%%%%%%%%%%%%%%%%%%%%%%%%%%%%%%%%%%%%%%%%%%%%%%%%%%%%%%%%%%%%%%%

%%%%%%%%%%%%%%%%%%%%%%%%%%%%%%%%%%%%%%%%%%%%%%%%%%%%%%%%%%%%%%%%%%%%%%
\exampl{Character table of dihedral group $\Dn{n}=C_{nv}$, $n$ odd.}{
  \label{exam:DnOddChars}                                           \inCB
  The  \Dn{n} group
  \[
    \Dn{n}=\{e,C_n,C_n^2,\cdots,C_n^{n-1},\sigma,C_n\sigma,\cdots,
    \Zn{n}^{n-1}\sigma\}
  \]
  has
  $n$ rotation elements and $n$ reflections.
  Group elements satisfies $C_n^i\cdot C_n^j\sigma=C_n^j\sigma\cdot C_n^{n-i}$,
  so $C_n^i$ and $C_n^{n-i}$ form a class. Also,
  $C_n^{n-i}\cdot C_n^{2i+j}\sigma=C_n^j\sigma\cdot C_n^{n-i}$ implies that
  $C_n^j\sigma$ and $C_n^{2i+j}\sigma$ are in the same class. Therefore,
  there are only three different types of classes:
  $\{e\}$, $\{C_n^k,C_n^{n-k}\}$ and
  $\{\sigma,C_n\sigma,\cdots, \Zn{n}^{n-1}\sigma\}$. The total number of
  classes is $(n+3)/2$. In this case, there are 2 one\dmn\
  irreducible representations (symmetric $A_1$ and anti-symmetric $A_2$ )
  and $(n-1)/2$ two\dmn\ irreducible representations. In the $j$th
  two\dmn\ irreducible representation, class $\{e\}$ has form
  $\bigl(\begin{smallmatrix}
    1&0\\ 0&1
  \end{smallmatrix} \bigr)$,
  class $\{C_n^k,C_n^{n-k}\}$ has form
  $\bigl(\begin{smallmatrix}
    \exp(\frac{i2\pi kj}{n}) &0 \\ 0& \exp(-\frac{i2\pi kj}{n})
  \end{smallmatrix} \bigr)$,
  and class $\{\sigma,C_n\sigma,\cdots, \Zn{n}^{n-1}\sigma\}$ has form
  $\bigl(\begin{smallmatrix}
    0&1\\ 1&0
  \end{smallmatrix} \bigr)$.
  We get \reftab{tab:DnOdd}.
}% end \exampl{Character table of \Dn{n}, $n$ odd
%%%%%%%%%%%%%%%%%%%%%%%%%%%%%%%%%%%%%%%%%%%%%%%%%%%%%%%%%%%%%%%%%%%%%%

%%%%%%%%%%%%%%%%%%%%%%%%%%%%%%%%%%%%%%%%%%%%%%%%%%%%%%%%%%%%%%%%%%%%%%
\begin{table}[h]
  \caption[Character table of dihedral group $\Dn{n}=C_{nv}$, $n$ odd.]{
    Character table of dihedral group $\Dn{n}=C_{nv}$, $n$ odd.
  }
  \label{tab:DnOdd}
  \begin{center}
    \begin{tabular}{c|ccc}
      $\Dn{n}$ ($n$ odd)  & $A_1$& $A_2$& $E_{j}$ \\
      \hline
      $  e  $        &   1  &   1  &  2 \\
      $C_n^k,C_n^{n-k} $ &   1  &   1  &  $2\cos(\frac{2\pi kj}{n})$ \\
      $\sigma,\sigma C_n^1,\cdots,\sigma C_n^{n-1}$  &   1  &   -1  &  0  \\
    \end{tabular}
  \end{center}
\end{table}
%%%%%%%%%%%%%%%%%%%%%%%%%%%%%%%%%%%%%%%%%%%%%%%%%%%%%%%%%%%%%%%%%%%%%%%


%%%%%%%%%%%%%%%%%%%%%%%%%%%%%%%%%%%%%%%%%%%%%%%%%%%%%%%%%%%%%%%%%%%%%%
\exampl{Character table of dihedral group $\Dn{n}=C_{nv}$, $n$ even.}{
  \label{exam:DnEvenChars}                                     \inCB
  In this case, there are $(n+6)/2$ classes:
  $\{e\}$,
  $\{\trDiscr{}{1/2}\}$,
  $\{\trDiscr{}{k/n},\trDiscr{}{(n-k)/n}\}$,
  $\{\sigma,\sigma\trDiscr{}{2/n},\cdots,\sigma\trDiscr{}{(n-2)/n}\}$ and
  $\{\sigma\trDiscr{}{1/n},\sigma\trDiscr{}{3/n},\cdots,\sigma\trDiscr{}{(n-1)/n}\}$.
  There are four different one\dmn\ irreducible representations,
  whose characters are $\pm 1$ under reflection $\sigma$ and shift-reflect
  operation $\sigma\trDiscr{}{1/n}$. We get \reftab{tab:DnEven}.
}% end \exampl{Character table of \Dn{n}, $n$ even
%%%%%%%%%%%%%%%%%%%%%%%%%%%%%%%%%%%%%%%%%%%%%%%%%%%%%%%%%%%%%%%%%%%%%%

%%%%%%%%%%%%%%%%%%%%%%%%%%%%%%%%%%%%%%%%%%%%%%%%%%%%%%%%%%%%%%%%%%%%%%
\begin{table}[h]
  \caption[Character table of dihedral group $\Dn{n}=C_{nv}$, $n$ even.]{
    Character table of dihedral group $\Dn{n}=C_{nv}$, $n$ even.
    Here $k,j=1,2,\cdots,n-1$.
  }
  \label{tab:DnEven}
  \begin{center}
    \begin{tabular}{c|ccccc}
      $\Dn{n}$ ($n$ even)  & $A_1$& $A_2$& $B_1$& $B_2$& $E_j$ \\
      \hline
      $  e  $        &   1  &   1  &  1   &  1   &  2  \\
      $\trDiscr{}{1/2}$
                           &   1  &   1  &  $(-1)^{n/2}$   & $(-1)^{n/2}$  & $2(-1)^{j}$ \\
      $\trDiscr{}{k/n},\trDiscr{}{(n-k)/n}$ ($k$ odd)
                           &   1  &   1  & -1   & -1
                                                       & $2\cos(\frac{2\pi kj}{n})$  \\
      $\trDiscr{}{k/n},\trDiscr{}{(n-k)/n}$ ($k$ even)
                           &   1  &   1  & 1   & 1
                                                       &  $2\cos(\frac{2\pi kj}{n})$  \\
      $\sigma,\sigma\trDiscr{}{2/n},\cdots,\sigma\trDiscr{}{(n-2)/n}$
                           &   1  &  -1  &  1   & -1   &  0  \\
      $\sigma\trDiscr{}{1/n},\sigma\trDiscr{}{3/n},\cdots,\sigma\trDiscr{}{(n-1)/n}$
                           &   1  &  -1  &  -1   & 1   &  0  \\
    \end{tabular}
  \end{center}
\end{table}
%%%%%%%%%%%%%%%%%%%%%%%%%%%%%%%%%%%%%%%%%%%%%%%%%%%%%%%%%%%%%%%%%%%%%%%
