The \cGLe\rf{cross93, AKcgl02} is one the most studied nonlinear equations in physics 
and applied mathematics community. 
In the context of pattern formation\rf{cross93}, 
it can be derived as a general amplitude equation near bifurcation point, and 
has applications in various other areas of physics
ranging from nonlinear optics\rf{AkhSotTow01} to
superconductivity\rf{Ginzburg04}. Due to symmetry constraint, \cGLe\ can only 
have odd-order nonlinear terms. Its cubic form
is frequently used to study turbulence and intermittent traveling waves. 
Recently, however, \cGLe\ with a quintic term has attracted attention for its peculiar 
dissipative soliton solutions  in one\dmn\ and two\dmn\
cases\rf{Chang07, AkhSotTow01, SoAkAn00, SoAkCh01, CaCiDeBr12, Cisternas2016}. 
Here we focus on the
one\dmn\ version defined on a periodic domain
\begin{equation}
  \label{eq:cqcgl1d}
  A_t  = \mu A + (D_r + iD_i) A_{xx} + (\beta_r + i\beta_i)|A|^2A 
         + (\gamma_r + i\gamma_i)|A|^4A 
  \,,\quad x\in[0,L]
  \,.
\end{equation}
Here, $A(x,t)$ is a complex field. All parameters including domain size
$L$ are real. This form is taken from the field of pattern formation. 
While in the community of nonlinear optics, people tend to use form
\begin{equation}
  i \psi_z + \frac{D}{2} \psi_{tt} + |\psi|^2\psi + \nu  |\psi|^4\psi
  = i \delta \psi +  i\beta \psi_{tt}  + i\epsilon |\psi|^2\psi 
    +  i \mu  |\psi|^4\psi 
      \,,\quad t\in[0,L]
      \,,
  \label{eq:cqcgl1d_optics}
\end{equation}
where $\psi(t, z)$ is the envelope of the field, $z$ and $t$ are the propagation distance
and the retarded time respectively, $D$ is the group velocity dispersion coefficient,
$\beta$ stands for spectral filtering, 
$\delta$ and $\epsilon$ are the linear and nonlinear gain coefficients respectively.
$\mu $ and $\nu$, if negative, are the saturation coefficients of nonlinear gain and
nonlinear saturation index. The correspondence of the two sets of parameters in
\eqref{eq:cqcgl1d} and \eqref{eq:cqcgl1d_optics} can be 
derived easily\rf{Descalzi10}. Through out this paper, we use equation form 
\eqref{eq:cqcgl1d} and set 
$\mu = -0.1$, $D_r = 0.125$, $D_i = 0.5$, $\beta_r = 1$, $\beta_i = 0.8$, 
$\gamma_r = -0.1$ and $\gamma_i = -0.6$ to
observe symmetric and asymmetric explosions. Also, we set $L=50$ which is large 
enough to hold explosions.