% siminos/xiong/thesis/chapters/symIntro.tex
% $Author: predrag $ $Date: 2021-06-19 11:30:00 -0400 (Sat, 19 Jun 2021) $

Symmetries play an important role in physics. In the study of
pattern formation\rf{cross93},
patterns with different symmetries form under different boundary conditions or
initial conditions. By considering symmetries only, quite a few
prototype equations such as \cGLe\rf{AKcgl02}
are proposed and have abundant applications in many fields.
So, in general,
symmetries help create a wonderful physical world for us.
However, in the analysis of chaotic systems, symmetries introduce
drifts of orbits along the symmetry directions and thus make the geometrical
structure of the global attractor more complicated than it really is.
In this case,
symmetries should be reduced before we conducting any analysis. In this chapter,
we review the basic notions of group theory, symmetry reduction
methods, and establish the relation between dynamics in the full \statesp\
and that in the symmetry-reduced \statesp.
% Finally, we factorize the
% evolution operator with respect to symmetries of the system so as to
% simplify the cycle averaging formula.
