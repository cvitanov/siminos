\chapter{Integration of the 1d \KSe\ on the 1st mode slice}
\label{append:ksint}

\renewcommand{\Im}{\mathtt{Im}}
\renewcommand{\Re}{\mathtt{Re}}

In the full \statesp\ \refeq{eq:fourierspace}, the evolution of a state
vector and a vector in the linearized tangent space are governed
by, respectively
\begin{align*}
  \dot{a}_k & = \vel_k(\ssp) =
              ( q_k^2 - q_k^4 )\, a_k
              - i \frac{q_k}{2} \sum_{m=-\infty}^{+\infty} a_m a_{k-m} \\
  \dot{y}_k & = (Ay)_k =
              ( q_k^2 - q_k^4 )\, y_k
              - i q_k \sum_{m=-\infty}^{+\infty} y_m a_{k-m} \\
\end{align*}
For a reduced \statesp\ point \refeq{eq:KSspred}, which is 
a $(N-3)$-element vector, the corresponding group tangent in the full
\statesp\ is 
$\groupTan(\sspRed) = (0, -\hat{b}_{1}, 2\hat{c}_{2}, -2\hat{b}_{2}, \cdots, 
(N/2-1)\hat{c}_{N/2-1}, -(N/2-1)\hat{b}_{N/2-1})^\top$. 
Choosing the template point $\slicep=(1,0,\cdots,0)$;
then $\sliceTan{} = (0, -1, 0, \cdots, 0)$.
From \refeq{MFdtheta}
we get the dynamics in the slice 
\[
  \velRed(\sspRed) = \vel(\sspRed)
  \,-\, \dot{\gSpace}(\sspRed) \groupTan(\sspRed)
  \,,\quad
  \dot{\gSpace}(\sspRed) = \frac{-\Im[\vel_1(\sspRed)] }{ \hat{b}_{1} }
  \,.
\]
The appearance of $\hat{b}_{1}$ in the denominator causes rapid jumps close the 
slice border. In order to make the trajectories in the slice smoother, we 
rescale the time step by $dt=\hat{b}_1 d\tau$. 
Thus the time rescaled dynamics in the slice is
\[
  \frac{d \sspRed}{d\tau} = \hat{b}_1 v(\sspRed) + \mathtt{Im}[v_1(\sspRed)] t(\sspRed)
  \,,
\]
Moreover, for the time rescaled
in-slice dynamics, we can also analyze its stability but with more effort.
In complex format, the $k$th element of group tangent is given by 
$t_k(\sspRed) = -ik \hat{a}_k = -ik(\hat{b}_k + i\hat{c}_k)$, so 
the elements of the {\stabmat} are
\[
  \left\{
    \begin{aligned}
      \frac{\partial \hat{v}_k}{\partial \hat{b}_j} & =
      \delta_{j1}v_k \;\;+ & \hat{b}_1 \frac{\partial v_k}{\partial  \hat{b}_j} - ik
      \frac{\partial \Im[v_1]}{\partial \hat{b}_j} \hat{a}_k - ik \Im[v_1]\delta_{kj} \\
      \frac{\partial \hat{v}_k}{\partial \hat{c}_j} & =
      & \hat{b}_1 \frac{\partial v_k}{\partial \hat{c}_j} - ik
      \frac{\partial \Im[v_1]}{\partial \hat{c}_j}] \hat{a}_k + k\Im[v_1] \delta_{kj} \\
    \end{aligned}
  \right.\,.
\]
Dynamics in the tangent space is given as
\begin{align*}
  (\frac{dy}{d\tau})_k
  = & \sum \frac{\partial \hat{v}_k}{\partial b_j} \Re[y_{j}]
      + \sum \frac{\partial \hat{v}_k}{\partial c_j} \Im[y_{j}] \\
  = & y_1 v_k + a_1 (Ay)_k - ik\mathtt{Im}[(Ay)_1]a_k - ik\mathtt{Im}[v_1]y_k
      \,.
\end{align*}
where $A=A(\sspRed)$ is the {\stabmat} in the full state space
evaluated at $\sspRed$. Realizing that
\[
\mathtt{Im}[v_1] =  - \frac{q_k}{2} \mathtt{Re}[\sum a_m a_{k-m}]
\;,\quad
\mathtt{Im}[v_1] = -  q_k \mathtt{Re}[\sum y_m a_{k-m}]
\]
we finally get the evolution equation in the slice and the associated
tangent space evolution:
\begin{equation}
\left\{
\begin{aligned}
  \frac{d\sspRed}{d\tau} = & \hat{b}_1 \left(( q_k^2 - q_k^4 )\, \hat{a}_k
  - i \frac{q_k}{2} \sum \hat{a}_m \hat{a}_{k-m} \right) +
  i \frac{q_k}{2} \hat{a}_k \mathtt{Re}[\sum \hat{a}_m \hat{a}_{1-m}]\\
  \\
  \frac{d\hat{y}_k}{d\tau} = &
  \hat{b}_1 \left(( q_k^2 - q_k^4 )\, \hat{y}_k
  - i q_k \sum \hat{a}_{k-j} \hat{y}_j \right)
  + i \frac{q_k}{2} \hat{y}_k \mathtt{Re}[\sum \hat{a}_m \hat{a}_{1-m} ] \\
  &
  \hat{y}_1 \left(( q_k^2 - q_k^4 )\, \hat{a}_k
  - i \frac{q_k}{2} \sum \hat{a}_{m} \hat{y}_{k-m} \right)
  + i q_k \hat{a}_k \mathtt{Re}[\sum \hat{a}_m \hat{y}_{1-m} ] \\
\end{aligned}
\right.\,.
\label{eq:ksfjacoM1}
\end{equation}
Equation \eqref{eq:ksfjacoM1} is the formula implemented in the code with
initial condition for $\hat{y}_k$:
\[
  \hat{y}(0) =
  \begin{pmatrix}
    0 & 0 & 0 & 0 & 0 & \cdots & 0 \\
    1 & 0 & 0 & 0 & 0 & \cdots & 0 \\
    0 & 1 & i & 0 & 0 & \cdots & 0 \\
    0 & 0 & 0 & 1 & i & \cdots & 0 \\
    &&& \cdots & & & \\
    0 &   &   &   & \cdots & 1 & i \\
    0 &   &   &   &   & \cdots & 0 \\
  \end{pmatrix}
\]

I follow Burak's trick to extract linear term
$( q_k^2 - q_k^4 )\hat{a}_k$ out of the velocity field and
leave the nonlinear term to be stiff as it is, then implement ETDRK4 to
integrate the trajectory and the Jabobian together. I don't know why
this trick works but I found that now smaller time step is required
to resolve all the Floquet exponent compared to the case in the unreduced
full space.

\KSe\ also has reflection symmetry, so what is the reflection rule after
quotienting out the SO(2) symmetry? Reflection and SO(2) satisfy relation
\[
Rg(\gSpace) = g(-\gSpace)R \,\quad \text{and}\quad R\mathbf{T}=-R\mathbf{T}
\,,
\]
where $\mathbf{T}$ is the generator of SO(2). Suppose two points are related by reflection
in the unreduced space $x(0)=Rx(T_p)$; after projected onto 1st mode slice, they are
$\sspRed(0)=g(\gSpace)x(0)$, $\sspRed(T_p)=g(\pi-\gSpace)x(T_p)$. Here the rotation angle
for the second point is $\pi-\gSpace$ because reflection matrix in the 1st mode subspace is
$R_1=diag(-1,1)$ which only flips the sign of $b_1$. Note, for 1st mode slice,
\[
g(\pi-\gSpace) = Sg(-\gSpace)
\]
with $S=diag(-1,-1,1,1,\cdots,-1,-1)$ the sign flip matrix. If different mode is chosen as
template point, then the explicit form $S$ will change accordingly. Now
\begin{align*}
  R\sspRed(T_p) & = RSg(-\gSpace)x(T_p) \\
  & = S Rg(-\gSpace)x(T_p)  \\
  & = Sg(\gSpace)Rx(T_p) \\
  & = Sg(\gSpace)x(0) \\
  & = S\sspRed(0)
  \,,
\end{align*}
which is $\sspRed(0)=SR\sspRed(T_p)$. This is the reformed reflection rule in the 1st mode
slice. For a preperiodic orbit, initial point $x(0)$ returns to $SRx(0)$ after one prime period;
while for a relative periodic orbit, if $\sspRed(0)$ is the initial condition on the slice, then
$SR\sspRed(0)$ is the initial condition for its dual orbit on the slice.

To justify that this approach faithfully recover the dynamics on the slice,
we can compare the Floquet exponents and Floquet vectors of the same
relative periodic orbit \cycle{rpo1} in the unreduced space and on the
slice.

\paragraph{Floquet exponents}
Table \ref{tab:floquetM1} compares the Floquet exponents of the relative periodic orbit
\cycle{rpo1} for SO(2) unreduced full space and the 1st mode slice. Note that after
quotient out SO(2), the number of marginal exponents decreases by one. The difference
of two groups of exponents has relative error of order $10^{-4}$.

\begin{table}[ht]
  \centering
  \begin{tabular}{l l l |l l l}
    \multicolumn{3}{c}{unreduced} & \multicolumn{3}{c}{1st mode reduced}\\
    $i$ & ~~~~~$\eigRe[i]$  & $\gSpace_{i}$  & $i$ & ~~~~~$\eigRe[i]$ & $\gSpace_{i}$  \\
    \hline
    1  &     0.3279115945    &               & 1  &   0.3279115925      &                \\
    2  &     5.035248171e-09 &               & 2  &   6.260337858e-13   &                \\
    3  &    -1.239861439e-08 &               &    &                                      \\
    4  &    -0.1321427982    &  -1           & 3  &   -0.1321425658     &  -1            \\
    5,6&    -0.2859700678    &  2.772432014  & 4,5&   -0.2859702314     &  2.772428566   \\
    7  &    -0.3624170728    &               & 6  &   -0.3624170063     &                \\
    8  &    -0.3282144959    &  -1           & 7  &   -0.3282144430     &  -1            \\
    9,10&   -1.961696302     &  2.241071099  & 8,9&   -1.961695814      &  2.241069955   \\
    11,12&  -5.601557908     &  1.366329853  &10,11&  -5.601555218      &  1.366329998   \\
    13,14&  -11.92077356     &  0.5548983577 &12,13&  -11.92076874      &  0.5548953759  \\
         &   $\cdots$        &               &    &  $\cdots$           &                \\
    27,28&  -239.4071347     &  0.8815898981 & 27 &   -239.3090112      &                \\
    29   &  -313.9806680     &               & 28 &   -312.9057132      &                \\
    30   &  -323.4121827     &               & 29 &   -315.0643644      &                \\
  \end{tabular}
  \caption{comparision of Floquet exponents of \cycle{rpo1} for SO(2) unreduced space and
    the 1st mode reduced space. }
  \label{tab:floquetM1}
\end{table}