\section{Symmetries}
\label{sect:kssym}

The one\dmn\ \KSe\ has three different
symmetries. Suppose $u(x, t)$ is an orbit in this system, then we have
\begin{itemize}
\item \emph{Galilean invariance}: $u(x-ct,t)+c$ is also a
  valid orbit, where c is a constant
  number. These two orbits have different mean velocity
  $\int dx\, u$.
\item \emph{Reflection invariance}:   $-u(-x,t)$ is also a valid orbit.
  In the Fourier mode space, reflection takes form $a_k \to -a_k^{*}$.
\item \emph{Translation invariance}: $u(x+\ell,t)$ is another valid orbit.
  In the Fourier mode space, translation takes form $a_k \to e^{iq_k\gSpace} a_k$
  with $\gSpace = 2\pi \ell/L$.
\end{itemize}
The zeroth Fourier mode $a_{0}$ represents the
mean velocity of $u(x, t)$. by setting $a_{0}=0$ in the integrator,
we eliminate the Galilean symmetry.
Therefore, we only need to account for the \On{2} symmetry of this system.
Reflection in \statesp\ \eqref{eq:fourierspace} takes the form
\[
  R=\diag(-1,\, 1, \, -1, \, 1,\cdots)
  \,.
\]
The translation symmetry corresponds to an one-parameter \SOn{2}
group in the \statesp,
\[
  \LieEl(\gSpace)=\diag(r_{1},r_{2},\cdots,r_{N/2-1})
\]
with
\[
  r_{k}=
  \begin{pmatrix}
    \cos k\gSpace & \sin k\gSpace \\
    -\sin k\gSpace & \cos k\gSpace
  \end{pmatrix}
  %,\quad k=1,2,\cdots,N/2-1
  \,.
\]
The corresponding Lie group generator is
\[
  \Lg= \diag(t_{1},t_{2},\cdots,t_{N/2-1}),\quad
  t_{k}=
  \begin{pmatrix}
    0 & k \\
    -k & 0
  \end{pmatrix}
  \,.
\]
Based on the consideration of these symmetries,
there are three types of invariant orbits in \KS\ system: \po s in the
$b_k=0$ invariant antisymmetric subspace, pre\po s which are self-dual
under reflection,
and \rpo s with a shift along group orbit after one
period. As claimed in \refref{SCD07}, the first type is absent for a domain
as small as $L=22$, and thus we focus on the last two types of orbits.
\begin{itemize}
\item
  For pre\po s $\cssp(0)=R\cssp(\period{p})$ , we only need to evolve
  the system for a prime period $\period{p}$ which is half of the whole
  period. The Floquet matrix is
  $\jMps_{p}(\cssp)=R\jMps^{\period{p}}(\cssp)$.
\item
  A \rpo,
  $\cssp(0)=\LieEl_p\cssp(\period{p})$, returns after one period
  $\period{p}$ to the initial state upon the group transform
  $\LieEl_p=\LieEl(\gSpace_p)$, so the corresponding Floquet matrix is
  $\jMps_p(\cssp)=\LieEl_p\jMps^{\period{p}}(\cssp)$.
\end{itemize}
In later sections, we calculate the stability of both pre\po s
and \rpo s. We anticipate that there are two marginal directions
for both types of orbits. One marginal direction corresponds to the
velocity field and the other one is the group tangent, which is
proved in \refexam{exam:KSmarginal}.
%%%%% example start %%%%%
\exampl{
  $\vel(\cssp)$ and $\groupTan(\cssp)$ are the two marginal directions
  of both pre\po s and \rpo s}{
  \label{exam:KSmarginal}
  The \JacobianM\ transports both velocity field and group tangent along the
  flow $\jMps^{\period{p}}\vel(\cssp(0)) = \vel(\cssp(\period{p}))$,
  $\jMps^{\period{p}}\groupTan(\cssp(0)) = \groupTan(\cssp(\period{p}))$.
  Therefore, for pre\po s, we have
  $\jMps_p\vel(\cssp(0)) = R\vel(\cssp(\period{p}))
  =R\vel(R\cssp(0))$.
  Here, we have used the definition of a pre\po\ and the form of
  its Floquet matrix. By use of the equivariance relation of
  the velocity field
  under reflection $\vel(R\cssp(0)) = R\vel(\cssp(0))$, we get
  \[
    \jMps_p\vel(\cssp(0)) = R \cdot R\vel(\cssp(0)) = \vel(\cssp(0))
    \,.
  \]
  So, we see that the velocity field is one marginal direction of pre\po s
  with Floquet multiplier $1$. Similarly, for the group tangent we have
  $\jMps_p\groupTan(\cssp(0)) = R\groupTan(R \cssp(0)) =
  R \cdot \Lg \cdot R \cssp(0) $ following definition \refeq{eq:gTan}.
  Since reflection anti-commutes with rotation $R\Lg + \Lg R = 0$, then
  we have
  \[
    \jMps_p\groupTan(\cssp(0)) = -\Lg \cdot R \cdot R \cssp(0) =
    -\groupTan(\cssp(0))
    \,.
  \]
  Therefore, the group tangent is also a marginal direction of pre\po s
  but with Floquet multiplier $-1$.  A group tangent reverses direction
  after one period for pre\po s.


  For \rpo s, by a similar process we have
  \[
    \jMps_p\vel(\cssp(0)) = \LieEl_p \vel( \LieEl_p^{-1} \cssp(0)) = \vel(\cssp(0))
  \]
  and
  \[
    \jMps_p\groupTan(\cssp(0)) = \LieEl_p \groupTan( \LieEl_p^{-1} \cssp(0))
    = \groupTan(\cssp(0))
    \,.
  \]
  So, the velocity field $\vel(\cssp)$ and the group tangent  $\groupTan(\cssp)$
  are two degenerate \Fv s for \rpo s, but not degenerate for
  pre\po s.
}
%%%%% example end %%%%%
In order to reduce \On{2} symmetry, we can choose to
reduce reflection symmetry first and then translation
symmetry, or vice versa.
Note that reflection does not commute with translation
$R\LieEl(\gSpace) = \LieEl(-\gSpace)R$,
so the result of symmetry reduction depends on the order
we choose.
In this section, we elect to quotient out the \SOn{2} symmetry
by the technique described in \refsect{sec:symReduce},
more precisely, by the 1st mode slice\rf{BudCvi14}
defined by
\begin{equation}
  \label{eq:ksslice}
  c_1 = 0 ,\, b_1 >0
  \,.
\end{equation}
This corresponds to choosing
$\slicep = (1, 0,\cdots, 0)$ as the template point
in \refeq{eq:slice}.
The reduced \statesp\ is denoted as
\begin{equation}
  \label{eq:KSspred}
  \sspRed=(\hat{b}_{1}, \hat{b}_{2}, \hat{c}_{2},\cdots, \hat{b}_{N/2-1}, \hat{c}_{N/2-1})^\top
  \,.
\end{equation}
Here, $\hat{c}_{1} = 0$ is omitted explicitly.
We can rotate orbits in the full \statesp\ to
the \SOn{2}-reduced \statesp\ by transformation
$a_k \to e^{-ik\gSpace_1}a_k$ where $\gSpace_1$ is the phase of the first Fourier mode.
Alternatively, we can choose to integrate the system directly in the slice.
For a reduced \statesp\ point \refeq{eq:KSspred}, which is
a $(N-3)$-element vector, the corresponding group tangent in the full
\statesp\ is
$\groupTan(\sspRed) = (0, -\hat{b}_{1}, 2\hat{c}_{2}, -2\hat{b}_{2}, \cdots,
(N/2-1)\hat{c}_{N/2-1}, -(N/2-1)\hat{b}_{N/2-1})^\top$.
The template point is $\slicep=(1,0,\cdots,0)$;
then the corresponding group tangent is $\sliceTan{} = (0, -1, 0, \cdots, 0)$.
From \refeq{MFdtheta}
we get the dynamics in the slice
\[
  \velRed(\sspRed) = \vel(\sspRed)
  \,-\, \dot{\gSpace}(\sspRed) \groupTan(\sspRed)
  \,,\quad
  \dot{\gSpace}(\sspRed) = \frac{-\Im[\vel_1(\sspRed)] }{ \hat{b}_{1} }
  \,.
\]
When an in-slice orbit gets close to the slice border $\hat{b}_1 = 0$,
the trajectory can attain arbitrarily high speed.
To alleviate this numerical difficulty,
we rescale the time step by $dt=\hat{b}_1 d\tau$.
Thus the time-rescaled dynamics in the slice is
\begin{equation}
  \label{eq:ksRescale}
  \frac{d \sspRed}{d\tau} = \hat{b}_1 v(\sspRed) + \mathtt{Im}[v_1(\sspRed)] t(\sspRed)
  \,.
\end{equation}
