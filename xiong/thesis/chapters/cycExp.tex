\section{Dynamics averaged over \po s}
\label{sect:cycExp}

Statistical properties and the geometrical structure of the global attractor
are among the major
questions in the study of chaotic nonlinear dissipative systems.
Generally,
such a system will get trapped to the global attractor after a transient
period, and we are only interested
in the dynamics on the attractor. The intrinsic instability of
orbits on the attractor make the longtime simulation unreliable, which is
also time-consuming. Fortunately, ergodic theorem\rf{sinai76} indicates that
longtime average converges to the same answer as a spatial average over the attractor,
provided that a natural measure exists on the attractor.
\begin{align}
  \overline{\expct{a}}
  & = \lim_{t\to \infty} {1\over |\pS|}
  \intM{\ssp_0}
  {1\over t} \int_0^t d\tau \,
  a(\ssp(\tau)) \label{eq:spa_ave1}\\
  & =  {1\over |\pS_{\msr}|}
  \intM{\ssp} \msr(\ssp) \, a(\ssp) \,. \label{eq:spa_ave2}
\end{align}
Here, $a(\ssp(t))$ is an observation, namely, a temporal
physical quantity such as average diffusion rate, energy dissipation rate,
Lyapunov exponents and so on. $\overline{\expct{a}}$ refers to its \spt\ average
on the attractor.
$\ssp(t)$ defines a dynamical system described by \refeq{eq:flow}.
$\pS$ is the \statesp\ of this system.
$\msr(\ssp)$ is the natural measure. Normalization quantities are
\begin{equation}
  \label{eq:ave_norm}
  |\pS| = \intM{\ssp} \,,\quad
  |\pS_{\msr}| = \intM{\ssp\,\msr(\ssp)}
\end{equation}
We make a distinction between the notation for \spt\ average
$\overline{\expct{a}}$ and that for spatial average
\[
  \expct{a} = {1\over |\pS|}  \intM{\ssp} \, a(\ssp)
  \,.
\]
So if
we define the integrated observable
\begin{equation}
  \label{eq:ave}
  A^t(\ssp_0) =
  \int_0^t d\tau \, a(\ssp(\tau))
  \,,
\end{equation}
then
\begin{equation}
  \label{eq:ave2}
  \overline{\expct{a}} = \lim_{t\to\infty} \frac{1}{t} \expct{A^t}
  = \lim_{t\to\infty} \frac{1}{t} {1\over |\pS|}
  \intM{\ssp_0} A^t(\ssp_0)
  \,.
\end{equation}
Formula \refeq{eq:spa_ave2} provides a nice way to calculate \spt\
average while avoiding longtime integration.
However, as a strange attractor usually has a fractal structure
and the natural measure $\msr(\ssp)$ could be arbitrarily complicated
and non-smooth,
computation by \refeq{eq:spa_ave2}
is not numerically feasible. This is where the
\emph{cycle averaging theory}\rf{DasBuch} enters.
In this section, we illustrate the process of obtaining the \spt\
average \refeq{eq:spa_ave1} by the weighted contributions from a set of \po s.

%======================================================================
\subsection{The evolution operator}
\label{sec:eo}

Towards the goal of calculating \spt\ averages, it does not suffice to
follow a single orbit. Instead, we study a swarm of orbits and see how
they evolve as a whole.
Equation \refeq{eq:ave2} deploys this idea exactly. We take all
points in the \statesp\ and evolve them for a certain time, after which
we study their overall asymptotic behavior.
By formula \refeq{eq:ave2}, the \spt\ average of observable
$a(\ssp(t))$ is given by the asymptotic behavior of the corresponding
integrated observable $A^t(\ssp)$.
However, instead of calculating $\expct{A^t}$, we turn to
\begin{equation}
  \label{eq:ave3}
    \expct{  e^{\beta A^t} } =
        {1 \over {|\pS|}} \intM{\ssp_s}  e^{\beta A^t(\ssp_s)}
    \,.
\end{equation}
Here we use $\ssp_s$ instead of
$\ssp_0$ as in \refeq{eq:ave2} to denote the starting point of the
trajectory. $\beta$ is an auxiliary variable. The motivation of studying
$e^{\beta A^t}$ instead of $A^t$ will be manifest later. Actually, this
form resembles the partition function in statistical mechanics where
$\beta = -1/kT$. And we will find several analogous formulas in this
section with those in the canonical ensemble.
Equation \refeq{eq:ave3} can be transformed as follows.
\begin{align}
    \expct{  e^{\beta A^t} }
  & = {1 \over {|\pS|}} \intM{\ssp_s} \left(
    \intM{\ssp_e}
    \prpgtr{\ssp_e - \flow{t}{\ssp_s}} \right)e^{\beta A^t(\ssp_s)} \label{eq:ave4} \\
  & = {1 \over {|\pS|}} \intM{\ssp_e}
    \intM{\ssp_s}
    \prpgtr{\ssp_e - \flow{t}{\ssp_s}} e^{\beta A^t(\ssp_s)} \label{eq:ave5} \\
  & = {1 \over {|\pS|}}
    \sum_{\substack{\text{all trajectories} \\ \text{of length } t}} e^{\beta A^t(\ssp_s)}
  \label{eq:ave6}
  \,.
\end{align}
From \refeq{eq:ave3} to \refeq{eq:ave4}, we insert an identity in the integral,
and from \refeq{eq:ave4} to \refeq{eq:ave5}, we change the integral order.
Here, $\ssp_s$ and $\ssp_e$ denote respectively the starting state
and end state of a trajectory. Basically, \refeq{eq:ave6} says
that whenever there is a path from $\ssp_s$ to $\ssp_e$, we should count its
contribution to the \spt\ average.
This idea is inspired  by the path integral in quantum mechanics.
Feynman interprets the \emph{propagator} (transition probability)
$\braket{\psi(x',t')}{\psi(x, t)}$  as a summation over all
possible paths connecting the starting and end states, where
the classical path is picked out
when $i/\hbar \to \infty$. Here, in \refeq{eq:ave5} we are on a better
standing because the Dirac delta function picks out paths that obey the flow equation
exactly.
Actually, the transition from a procedural law \refeq{eq:spa_ave1} to a high-level
principle \refeq{eq:ave6} is a tendency in physics, similar to the transition from
Lagrangian mechanics to the principle of least action, or the transition from
Schr\"odinger equation to the path integral.

The kernel of the integral in \refeq{eq:ave5} is called the \emph{evolution operator}
\begin{equation}
    \Lop^t(\ssp_e, \ssp_s)\,=\,
    \prpgtr{\ssp_e- \flow{t}{\ssp_s}}\, e^{\beta A^t(\ssp_s)}
    \, .
    \label{eq:evooper}
\end{equation}
The evolution operator shares a lot of similarities with the propagator
in quantum mechanics. For example,
the evolution operator also forms a semigroup:
\begin{equation}
  \label{eq:lsemi}
  \Lop^{t_1+t_2}(\ssp_e, \ssp_s)\,=\,
  \intM{\ssp'}  \Lop^{t_2}(\ssp_e, \ssp')  \Lop^{t_1}(\ssp', \ssp_s)
\end{equation}
We define the action of the evolution operator on a function as
\begin{equation}
  \label{eq:evoact}
  \Lop^t \circ \phi \,=\,
  \intM{\ssp_s} \Lop^t(\ssp_e, \ssp_s) \phi(\ssp_s)
\end{equation}
which is a function of the end state $\ssp_e$.
A function $\phi(\ssp)$ is said to be the eigenfunction of $\Lop^t$ if
$\Lop^t \circ \phi = \lambda(t)\, \phi$. Here $\lambda(t)$ is the eigenvalue.
We make it explicit that it depends on time.
Note, $\Lop^t$ acts on a function space, so in principle,
$\Lop^t$ is an infinite\dmn\ operator.
However, in some cases such as piece-wise maps, if the observable is defined
uniformly in each piece of the domain, then $\Lop^t$ can effectively
be expressed as a finite\dmn\ matrix.
% Anyway, the dimensionality of $\Lop^t$ does not stop us from investigating its
% eigenspectrum.
Here we give two examples of the eigenfunctions of $\Lop^t$.

%%%%%%%%%%%%%%%%%%%%%%%%%%%%%%%%%%%%%%%%%%%%%%%%%%%%%%%%%%%%%%%%%%%%%%
\exampl{The invariant measure of an equilibrium is an eigenfunction of $\Lop^t$}{
  \label{exam:imeq}                                                 \toCB
  The invariant measure of an equilibrium $\ssp_q$ is given by
  \[
    \phi(\ssp) = \delta(\ssp - \ssp_q)
    \,.
  \]
  Then
  \begin{align*}
    \Lop^t \circ \phi
    & = \intM{\ssp_s} \Lop^t(\ssp_e, \ssp_s) \delta(\ssp_s - \ssp_q) \\
    & = \intM{\ssp_s} \prpgtr{\ssp_e- \flow{t}{\ssp_s}}\, e^{\beta A^t(\ssp_s)}
      \delta(\ssp_s - \ssp_q) \\
    & = \prpgtr{\ssp_e- \flow{t}{\ssp_q}}\, e^{\beta A^t(\ssp_q)} \\
    & = \prpgtr{\ssp_e- \ssp_q}\, e^{t\beta a(\ssp_q)} \\
    & = e^{t\beta a(\ssp_q)} \phi(\ssp_e)  \,.
  \end{align*}
  Therefore, the invariant measure of an equilibrium is an eigenfunction of
  the evolution operator with eigenvalue  $e^{t\beta a(\ssp_q)}$.
}
%%%%%%%%%%%%%%%%%%%%%%%%%%%%%%%%%%%%%%%%%%%%%%%%%%%%%%%%%%%%%%%%%%%%%%

%%%%%%%%%%%%%%%%%%%%%%%%%%%%%%%%%%%%%%%%%%%%%%%%%%%%%%%%%%%%%%%%%%%%%%
\exampl{The invariant measure of a \po\ is an eigenfunction
  of $\Lop^{nT}$}{
  \label{exam:impo}                                             \toCB
  The invariant measure of a \po\ is given by
  \begin{equation}
    \label{eq:poinv}
    \phi(\ssp) = \frac{1}{T} \int_{0}^T \delta(\ssp - \flow{t}{\ssp_0} dt
    \,.
  \end{equation}
  Here, $T$ is the period of this orbit. $\ssp_0$ is an arbitrarily chosen
  point on this orbit. Then
  \begin{align*}
    \Lop^{nT} \circ \phi
    & = \frac{1}{T} \int_{0}^T
      \intM{\ssp_s} \prpgtr{\ssp_e- \flow{nT}{\ssp_s}}\, e^{\beta A^{nT}(\ssp_s)}
      \delta(\ssp_s - \flow{nT}{\ssp_0} dt \\
    & = \frac{1}{T} \int_{0}^T
      \intM{\ssp_s} \prpgtr{\ssp_e- \flow{nT+t}{\ssp_s}}\, e^{\beta A^{nT}(\flow{t}{\ssp_0}}
      \delta(\ssp_s - \flow{t}{\ssp_0} dt  \\
    & = \frac{1}{T} \int_{0}^T  \prpgtr{\ssp_e- \flow{t}{\ssp_0}}
      e^{n\beta A^{T}(\ssp_0)} dt
      \intM{\ssp_s}
      \delta(\ssp_s - \flow{t}{\ssp_0}  \\
    & = \frac{1}{T} \int_{0}^T  \prpgtr{\ssp_e- \flow{t}{\ssp_0}}\,
      e^{n\beta A^T(\ssp_0)} dt \\
    & = e^{n\beta A^T(\ssp_0)} \phi(\ssp_e) \,.
  \end{align*}
  In the above derivation, we have used the identity
  $A^{nT}(\flow{t}{\ssp_0} = n A^{T}(\ssp_0)$ which is manifest
  because $\ssp_0$ is a periodic point.
  In general, \refeq{eq:poinv} is not an eigenfunction
  of $\Lop^{t}$, but it is for $t=nT$. In the above formula, $A^T(\ssp_0)$
  actually does not depend on the choice of the starting point $\ssp_0$
  as long as it is one point on the \po.
}
%%%%%%%%%%%%%%%%%%%%%%%%%%%%%%%%%%%%%%%%%%%%%%%%%%%%%%%%%%%%%%%%%%%%%%

Let us now turn to the original
problem of how to calculate \refeq{eq:ave2} and why we choose the exponential
form in \refeq{eq:ave3}. Combine \refeq{eq:ave2}, \refeq{eq:ave3} and
\refeq{eq:ave5}, we have
\begin{align}
  \overline{\expct{a}}
  & = \lim_{t\to\infty} \frac{1}{t}
    \left. \frac{\partial}{\partial \beta} \expct{  e^{\beta A^t} } \right|_{\beta=0}
  \label{eq:aveDif1}\\
  & = \left. \frac{\partial}{\partial \beta}
     \lim_{t\to\infty} \frac{1}{t} {1 \over {|\pS|}} \intM{\ssp_e}
    \intM{\ssp_s} \Lop^t(\ssp_e, \ssp_s) \right|_{\beta=0} \,.
    \label{eq:aveDif2}
\end{align}
Here, we have used a trick to obtain the spatial average by using an auxiliary variable
$\beta$, which is similar to what we do in
the canonical ensemble. $\overline{\expct{a}}$
is given by the longtime average of the evolution operator.
From example \ref{exam:imeq} and \ref{exam:impo},
we see that the eigenvalues of $\Lop^t$ go to $\infty$ when
$t\to\infty$. Therefore, asymptotically, the leading eigenvalue of
$\Lop^t$ will dominate the \spt\ average in \refeq{eq:aveDif2}.
By the semi-Lie group property \refeq{eq:lsemi}, we define
$\Lop^t = e^{t\Aop}$ with $\Aop$ defined as the generator of the evolution operator.
Then \refeq{eq:aveDif2} is simplified as follows,
\begin{align}
  \overline{\expct{a}}
  & = \left. \lim_{t\to\infty}  {1 \over {|\pS|}} \intM{\ssp_e}
    \intM{\ssp_s} \Lop^t(\ssp_e, \ssp_s) \frac{\partial}{\partial \beta}
    \Aop\right|_{\beta=0}
    \label{eq:aveDif3} \\
  & = \left. \lim_{t\to\infty}  {1 \over {|\pS|}} \intM{\ssp_e}
    \intM{\ssp_s} \prpgtr{\ssp_e- \flow{t}{\ssp_s}} \frac{\partial}{\partial \beta}
    \Aop\right|_{\beta=0}
    \label{eq:aveDif4} \\
  & = \left. \lim_{t\to\infty}
    \expct {\frac{\partial}{\partial \beta}
    \Aop}\right|_{\beta=0} \,.
\end{align}
As we said, the time limit above will converge to the leading eigenvalue of $\Aop$.
By letting
\begin{equation}
  \label{eq:s}
  s_0(\beta) := \text{the largest eigenvalue of } \Aop \text{ when } t\to\infty \,,
\end{equation}
we ultimately reach  the formula for \spt\ averages,
\begin{equation}
  \label{eq:spav}
  \overline{\expct{a}} =  \left. \frac{\partial s_0}{\partial \beta} \right|_{\beta=0}
  \,.
\end{equation}
Equation \refeq{eq:spav} connects the \spt\ averages with the
largest eigenvalue of the generator of the evolution operator. It is
one of the most important formulas in the cycle averaging theory.
We need to study $\Lop^t$ for $t\to \infty$ to obtain \refeq{eq:s}.
To make calculations easier,
we turn to the \emph{resolvent} of $\Lop^t$, i.e., the Laplace transform
of $\Lop^t$.
\begin{equation}
  \label{eq:resolv}
  \int_0^\infty dt \, e^{-st} \Lop^t
        = {1 \over s - \Aop} \,,\quad\quad   \Re s > \eigenvL_0
        \,.
\end{equation}
So, the leading eigenvalue of $\Aop$ is the pole of the resolvent of the evolution
operator. In the next subsection, we will obtain the expression for the resolvent
of $\Lop^t$ by a set of \po s.

%======================================================================
\subsection{\FD s}
\label{sec:det}

The discussion in \refsect{sec:eo} motivates us to calculate the leading
eigenvalue of the generator $\Aop$. Put it in another way, we need to
solve equation
\begin{equation}
  \label{eq:Adet}
  \det(s - \Aop) = 0
\end{equation}
whose answer gives the full spectrum of $\Aop$.
We claim that \spt\ average can be calculated by \po s in
this system. Still, there isn't any hint
how \refeq{eq:Adet} is related to \po s.
On one hand, we see that \po s are related to the trace of the evolution operator
by the definition \refeq{eq:evooper}.
\begin{equation}
  \label{eq:trL}
  \tr\Lop^t \,=\,
  \intM{\ssp}\Lop^t(\ssp, \ssp)\,=\,
  \intM{\ssp} \prpgtr{\ssp- \flow{t}{\ssp}}\, e^{\beta A^t(\ssp)}
  \,.
\end{equation}
On the other hand, matrix identity
\begin{equation}
  \label{eq:Miden}
  \ln\det M = \tr \ln M
\end{equation}
relates the determinant of a matrix $M$ on the left-hand side
in \refeq{eq:Miden} with its
trace on the right-hand side. With these two pieces of information,
we can express \refeq{eq:Adet} in terms of $\tr\Lop^t$.
\begin{equation}
  \label{eq:trace}
  \ln \det (s  - \Aop)
  = \tr \ln (s - \Aop) = \int_{} \tr  \frac{1}{s - \Aop}  \, ds
  \,.
\end{equation}
Also by the definition of resolvent \refeq{eq:resolv}, we have
\begin{equation}
  \label{eq:Adet2}
  \det (s - \Aop) = \exp
  \left(\int ds \int_0^\infty dt \, e^{-st} \tr\Lop^t \right)
  \,.
\end{equation}
The remaining part of this subsection is devoted to calculating
$\int_0^\infty dt \, e^{-st} \tr\Lop^t$. For a given \po\ with period
$T_p$, we decompose the trace \refeq{eq:trL} in two directions: one
is parallel to the velocity field $\ssp_{\parallel}$ and the other in the
transverse direction $\ssp_{\perp}$,
\[
  \int_0^\infty dt \, e^{-st} \tr\Lop^t =
  \int_0^\infty dt \, e^{-st}
  \intM{\ssp_{\perp}} d\ssp_{\parallel} \prpgtr{\ssp_{\perp}- f^t_{\perp}(\ssp)}
  \prpgtr{\ssp_{\parallel}- f^t_{\parallel}(\ssp)}
  \, e^{\beta A^t(\ssp)}
  \,.
\]

We first calculate the integration in the parallel direction. This is a
one\dmn\ spatial integration. Due to the periodicity
$\ssp_{\parallel}= f^{rT}_{\parallel}(\ssp)$ for $r=1, 2,\cdots$, we
split the integration
in this parallel direction into infinitely many periods:
\begin{align}
 \int_{0}^\infty dt \, e^{-s t} \, & \oint_p d\ssp_{\parallel}
  \prpgtr{\ssp_\parallel -  f^t_{\parallel}(\ssp)}
  =  \int_{0}^\infty dt \, e^{-s t} \, \int_0^{T_p} d\tau \,||\vel(\tau)|| \,
  \prpgtr{\ssp_\parallel(\tau) -  \ssp_{\parallel}(\tau + t)}
    \label{eq:detpara1}   \\
  & = \sum_{r=1}^{\infty}  e^{-\eigenvL \period{p}r} \,
    \int_0^{T_p} d\tau \, ||\vel(\tau)||\,  \int_{-\epsilon}^{\epsilon} dt \, e^{-\eigenvL t} \,
    \prpgtr{\ssp_\parallel(\tau) -   \ssp_{\parallel}(\tau + rT_p + t)}
    \label{eq:detpara2}
    \,.
\end{align}
The integrand in \refeq{eq:detpara2} is defined in a small time window
$[-\epsilon, \epsilon]$. Within this window
$\ssp_\parallel(\tau) -   \ssp_{\parallel}(\tau + rT + t) =
\ssp_\parallel(\tau) -   \ssp_{\parallel}(\tau + t) \simeq -\vel(\tau)t$.
So if we take $\epsilon \to 0$,
\[
  \int_{-\epsilon}^{\epsilon} dt \, e^{-\eigenvL t} \,
  \prpgtr{\ssp_\parallel(\tau) -   \ssp_{\parallel}(\tau + rT_p + t)}
  = \frac{1}{\norm{\vel(\tau)}}
  \,.
\]
Therefore, we obtain the integration in the parallel direction,
\begin{equation}
  \label{eq:detpara3}
  \int_{0}^\infty dt \, e^{-s t} \, \oint_p d\ssp_{\parallel}
  \prpgtr{\ssp_\parallel -  f^t_{\parallel}(\ssp)}
  =  T_p \sum_{r=1}^{\infty}  e^{-\eigenvL \period{p}r}
  \,.
\end{equation}

Now we calculate the trace integration in the transverse direction.
In this case,
we are actually integrating on a Poincar\'e section transverse to this
\po. So $f^t_\perp(\ssp)$ is the projected evolution function
in this section, which has codimension one with the full \statesp.
Therefore,
\begin{equation}
  \label{eq:dettrans}
  \int_{\PoincS} d\ssp_{\perp} \prpgtr{\ssp_{\perp}- f^{rT_p}_{\perp}(\ssp)}
  = {1 \over \oneMinJ{r} }
  \,.
\end{equation}
Here $\monodromy_p$ is the Floquet matrix projected on the Poincar\'e section
of this \po.

Combine \refeq{eq:detpara3} and \refeq{eq:dettrans} and consider all \po s inside
this system, we obtain the
\emph{trace formula}
\begin{equation}
  \label{eq:tr}
  \int_0^\infty dt \, e^{-st} \tr\Lop^t =
  \sum_p \period{p} \sum_{r=1}^\infty
  { e^{r (\beta A_p -\eigenvL\period{p})}
    \over  \oneMinJ{r} }
  \,.
\end{equation}
$A_p$ is the integrated observable along the orbit for one period.
Note, summation $\sum_p$ counts all \po s inside this system.
Substitute \refeq{eq:tr} into \refeq{eq:Adet2}, we obtain the
\emph{\Fd}
\begin{equation}
  \det(s - \Aop)  \defeq \exp \left(
    - {
      \sum_{p} \sum_{r=1}^\infty \frac{1}{r}
      \frac{   e^{r (\beta A_p -\eigenvL\period{p}) }}
      {\oneMinJ{r}}
    }\right)
  \label{eq:sd}
\end{equation}
of the flow.

%======================================================================
\subsection{\Dzeta s}
\label{sect:zeta}

In the \Fd\ \refeq{eq:sd}, $\oneMinJ{r}$
is approximately equal to the product of all the expanding eigenvalues
of $\monodromy_p$. That is, $\oneMinJ{r} \simeq |\ExpaEig_p|^r$. Here,
$ \ExpaEig_p\,=\,\prod_e \ExpaEig_{p,e}$ is
the product of the expanding eigenvalues of the
Floquet matrix $\monodromy_p$.
The accuracy of this approximation improves as $r\to\infty$.
Substitute it into the \Fd\ \refeq{eq:sd},
we have
\[
  \exp \left(
    -\sum_{p} \sum_{r=1}^\infty \frac{1}{r}
    \frac{   e^{r (\beta A_p -\eigenvL\period{p}) }}
    {|\ExpaEig_p|^r}
  \right)
  = \exp \left(
    \sum_{p} \ln \left( 1 -
      \frac{   e^{\beta A_p -\eigenvL\period{p} }}{|\ExpaEig_p|}
    \right)
  \right)
  = \prod_{p} \left(
    1 -
    \frac{   e^{\beta A_p -\eigenvL\period{p} }}{|\ExpaEig_p|}
  \right)
  \,,
\]
where we have used the Taylor expansion
$\ln(1 - x) = -\sum_{n=1}^\infty \frac{x^n}{n}$.
Then we obtain the {\em \dzeta},
\begin{equation}
  1/\zeta
  \,=\,\prod_{p}
  \left( 1 - t_p \right)
  \,,\qquad \text{with} \qquad
  t_p = \frac{   e^{\beta A_p -\eigenvL\period{p} }}{|\ExpaEig_p|}
  \,.
  \label{eq:zeta}
\end{equation}

Formulas \refeq{eq:spav}, \refeq{eq:sd} and \refeq{eq:zeta}
are the ultimate goal of the discussion
in this section. They tell us that the \spt\ average is determined
by the leading eigenvalue of the generator of the evolution operator $\Aop$, and
the eigenspectrum of $\Aop$ can be obtained by the whole set of \po s inside
this system. Formula \refeq{eq:sd} precisely describes our perspective on chaotic
deterministic flows. The flow on the global attractor can be
visualized as a walk chaperoned by a hierarchy of unstable invariant
solutions (\eqva, \po s) embedded in the attractor. An
ergodic trajectory shadows one such invariant solution for a while, is
expelled along its unstable manifold, settles into the neighborhood of
another invariant solution for a while, and repeats this process forever.
Together, the infinite set of these unstable invariant solutions forms
the skeleton of the strange attractor,
and in fact, \spt\ averages can be
accurately calculated as a summation taken over contributions from
\po s weighted by their stabilities\rf{Christiansen97, DasBuch}.

In practice, we truncate \refeq{eq:sd} or \refeq{eq:zeta}
according to the topological length of \po s,
which is primarily established by symbolic dynamics, or if not available, by
the stability of \po s. This technique is called \emph{cycle expansion},
whose effectiveness has been demonstrated in a
few one\dmn\ maps\rf{AACI, AACII} and
ergodic flows\rf{BuBoCvSi14, Christiansen97, lanCvit07}. See \rf{DasBuch}
for more details.
