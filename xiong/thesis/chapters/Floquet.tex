A linear ordinary differential equation with periodic
coefficients is an old topic. Solutions of such type of
equations can be decomposed into the product of 
a time-exponential part
and a bounded periodic part. The rigorous statement is given by
\begin{theorem}\cite{Floquet1883}
  \label{them:ft}
  If $\Phi(t)$ is a fundamental matrix solution of linear system
  $\dot{x} = A(t)x$ with periodic coefficients $A(t+T)=A(t)$,
  $T$ being the period, then $\Phi(t)$ can be decomposed as
  \[
    \Phi(t) = P(t)e^{tB} \quad \text{with} \quad P(t) = P(t+T)
    \,.
  \]
  Here $P(t)$ is called monodromy matrix, and $B$ is a constant
  matrix. Eigenvalues of $B$ are called Floquet exponents. The
  eigenvectors are called \Fv s.
\end{theorem}
The proof of this theorem is very simple. First, it is easy to
see that if $\Phi(t)$ is solution matrix, so is $\Phi(t+T)$.
Their relation is given by $\Phi(t+T)=\Phi(t)C(t, T)$, where
$C(t, T) = \Phi^{-1}(t)\Phi(t+T)$. Second, $C(t, T)$ is independent
of time which can be shown by taking time derivative
at both sides of $\Phi(t)C(t, T) = \Phi(t+T)$, 
\begin{align*}
  & \dot{\Phi}(t)C(t, T) + \Phi(t)\dot{C}(t, T) = \dot{\Phi}(t+T) \\
  \implies 
  &  A(t) \Phi(t) C(t, T) + \Phi(t)\dot{C}(t, T) = A(t + T)\Phi(t+T) \\
  \implies
  & \dot{C}(t, T) = 0 \,.
\end{align*}
So matrix $C(T)$
is only parametrized by the period. Furthermore,
\[
  C(nT) = \prod_{k=0}^{n-1}\left( \Phi^{-1}(t+kT) \, \Phi(t+(k+1)T) \right) = C(T)^n
  \,;
\]
thus $C(T)$ has form $e^{TB}$, where $B$ is a constant matrix.
So $\Phi(t+T)=\Phi(t)e^{TB}$. Last,
If we define a new matrix $P(t) = \Phi(t)e^{-tB}$, then
\[
  P(t+T) = \Phi(t+T)e^{-TB-tB} = \Phi(t)e^{-tB} = P(t) 
  \,.
\] 
In this way,
we decompose the fundamental matrix solution into a periodic part
$P(t)$
and a time exponential part $e^{tB}$,
even though we do not know the explicit
form of $P(t)$.

Floquet theorem in condensed matter physics is under another 
name: \textbf{Bloch theorem}, which states that wave function in a periodic
potential can be written as
$\psi(\vec{r}) = e^{i\vec{k}\cdot\vec{r}}u(\vec{r})$.
Here $u(\vec{r})$ is a periodic function with the same potential period.
Eigenstates are classified into different bands by different profiles of
$u(\vec{r})$.

For our interest, dynamics in the tangent space
in a nonlinear dynamical system is governed by $\dot{J} = A J$.
Especially, for \po s, $A$ is periodic, and if we start
with $J(t_0) = I$, then Jacobian corresponding to one period $J_p$ is
just the exponential part $e^{TB}$ in theorem \ref{them:ft}.
This is the theoretical basis of periodic eigendecomposition algorithm,
which tries to calculate the eigenvalues and eigenvectors of $J_p$.
On the other hand, the periodic part $P(t)$ in theorem \ref{them:ft}
evolves \Fv s along the \po, and returns to
the initial value after one period. This information is revealed
in the reordering stage of periodic eigendecomposition algorithm.
