% siminos/xiong/thesis/proposal/tex/abstract.tex
% $Author: xiong $ $Date: 2016-12-19 10:02:10 -0500 (Mon, 19 Dec 2016) $

Invariant subsets inside a chaotic flow, such as equilibria, periodic
orbits and invariant tori, play the key role in shaping the geometry of
the global attractor (or inertial manifold). The goal of my research is
to understand how chaotic attractor is organized by the set of (relative)
periodic orbits and sort them by their weighted contributions to the
spatio-temporal averages of observables measured on the attractor. In
order to achieve this goal, technical advances on two directions need to be
surmounted. First, the accuracy of stability analysis of (relative)
periodic orbits is needed to be brought to a level much higher than the
standard methods available in the literature, to establish whether the
Floquet vectors evaluated along (relative) periodic orbits are a tool for
estimating the local dimension of system's inertial manifold. Second,
while it is already known that for systems with discrete and/or
continuous symmetries, dynamics can be substantially simplified if
symmetries are quotiented, one still needs to establish that the
dynamical zeta functions and spectral determinants 
are decomposed into irreducible representations of the continuous
symmetries of the flow. This thesis proposal introduces the background
for conducting research in this area, summarizes the work I have done so
far, and outlines the plan for my future work.
