% siminos/xiong/thesis/proposal/tex/proposal.tex
% $Author: xiong $ $Date: 2015-09-03 13:21:45 -0400 (Thu, 03 Sep 2015) $

\section{Plan for future work}
\label{sec:future}

\subsection{Dimension of inertial manifold}
We have only tested the method described in sec.~\ref{subsec:LDIM}
to investigated the dimension of inertial manifold in
1-dimensional \KSe, but we think this method can be applied to
other turbulent systems as well, for example \cGLe, 2-d or 3-d
Navier-Stokes equation. The point we conduct such research to other
systems in the next step
 is that
usually the mathematical proof of existence of inertial manifold
sheds
limited information about its exact dimension. Moreover,
for systems like
3-d Navier-Stokes, the existence of inertial manifold is still
unknown, so the experiments investigating decoupling in the
tangent space by \Fv s can serve as a valuable tool to provide
accurate enough information about the local geometry of inertial
manifold ahead of strict mathematical statements.
Also, this information can help engineers to use the appropriate
resolution in the numerical turbulence simulation.

\subsection{Cycle expansions for \cqcGLe}
We have found quite a few \reqv s in 1-d \cqcGLe\ and converted
the dynamics into the symmetry-reduced space. The next step
is trying to find \po s or \rpo s inside this system and
hopefully build the symbolic dynamics according to the
geometry of this hierarchy of \po s.

with this set of \po s, cycle expansion can be applied to \Fd\
\eqref{Z(s)} with interesting physical quantity such as diffusion rate as
the observable. Therefore, our ultimate goal of studying \cqcGLe\ is to
give exact prediction about the long time behavior of soliton solution in
it. Also such method can be applied to the 2-d \cqcGL and analyze the
diffusion constant of the random walk reported in\rf{CaCiDeBr12}.
