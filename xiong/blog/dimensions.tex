% siminos/xiong/thesis/chapters/dimensions.tex
% $Author: predrag $ $Date: 2017-03-09 17:25:05 -0500 (Thu, 09 Mar 2017) $

% this file is not called by Xiong's thesis
\section{Dimensions}

                                            \toCB
{\bf [2017-03-08 Predrag]} Included parts not in the thesis.

\begin{description}

    \item[2015-10-12 Predrag]
Barreira\rf{Barreira15} {\em Dimension theory of flows: {A} survey}
might be of scholarly interest - mathematical literature on the notion
of dimension. Or other articles in the same issue of the journal, see
\HREF{http://aimsciences.org/journals/contentsListnew.jsp?pubID=808}
{here}. Three of them are meant to be reviews.

\item[2015-10-12 Xiong to Predrag] Thanks for pointing these review paper
    to me. They are difficult. I need to read the books in their
    reference first.

    \end{description}


%\subsection{A short survey of various dimensions}
%\label{sect:varDims}

%    \PC{2014-09-22 please merge \texttt{fractDim.tex} into this file,
%        then remove fractDim.tex from \texttt{thesis/chapters}, and
%        link to the thesis draft.}
    \subsection{Kaplan-Yorke}
      \label{s-KapYorke}

                                            \toCB
{\bf [2017-03-08 Predrag]} merge with the thesis befor moving to ChaosBook.

%        \item[2014-07-21 Predrag]
In a brief glance at the Kaplan and Yorke\rf{KapYor79}
\HREF{http://chaosbook.org/library/KapYor79.pdf} {paper}
I do not recognize the formula for the Kaplan-Yorke dimension
\beq
    D_{KY}= k + \frac{1}{|\Lyap^{(k+1)}|} \sum_{i=1}^k \Lyap^{(i)}
\,,
\ee{KapYorDim}
where $k$ is the largest integer such that the Lyapunov exponents sum
$\sum_{i=1}^k \Lyap^{(i)}>0$, \ie\ such that the volume of the parallelepiped
subtended by the leading $k$ covariant vectors (for us, the leading
Floquet vectors $\jEigvec[j]$) is expanding.
I see it in \refref{FKYY83} where Yorke calls it `Liapunov
dimension' (careful, here the symbol for multipliers are $\lambda_i$).

I think we want to
list this dimension for \rpo s and \reqva\ in terms of Floquet /
stability exponents \eigRe[i],
\beq
    D_{KY}= k + \frac{1}{|\eigRe[k+1]|} \sum_{i=1}^k \eigRe[i]
\,.
\ee{KapYorFloq}


Let $\lambda_1 \ge \lambda_2, \ge \cdots, \ge \lambda_n$
are the Lyapunov spectral
of a n dimensional chaotic system($\lambda_1 > 0$), now we try to
determine how many cubes needed to cover the neighborhood of a template
point $x(0)$ as system evolves. Suppose the neighborhood is a
n\dmn\ parallelogram with each side oriented in the covariant
direction at $x(0)$ initially, and the number of $\epsilon$-cubes needed
to cover this parallelogram is $N(\epsilon)$; then after an infinitesimal
time $\delta t$, the neighborhood moves to $x(\delta t)$ the parallelogram
gets stretched/contracted along each side. For some $j+1$ which
$\lambda_{j+1} < 0$, we use a smaller cube with length
$e^{\lambda_{j+1} \delta t}\epsilon$ to cover the new neighborhood, then
\begin{equation}
N(e^{\lambda_{j+1} \delta t}\epsilon) =
\left\{ \prod_{i=1}^{j} e^{(\lambda_i - \lambda_{j+1})\delta t}\right\}
N(\epsilon)
\label{eq:ky_relation}
\end{equation}
Let's explain the coefficient above.
For side number $i<j+1$, it has been stretched $e^{\lambda_i \delta t}$, and the
new cube side length is $e^{\lambda_{j+1} \delta t}\epsilon$, so it needs
$\exp((\lambda_i - \lambda_{j+1})\delta t)$ more cubes along this direction.
For $i > j+1$, the original number of cubes along this side is enough to
cover it, which means the above formula is actually
over-counting in this direction. Now suppose the exponential law is valid
when $\epsilon \ll 1$ as the box-counting dimension
$N(\epsilon) \propto \epsilon^{-d}$. Then \eqref{eq:ky_relation} reduces to
$(e^{\lambda_{j+1} \delta t}\epsilon)^{-d}  =
\prod_{i=1}^{j} e^{(\lambda_i - \lambda_{j+1})\delta t} \epsilon^{-d}$. We get
\[
d_j = j - \frac{\sum_{i=1}^{j} \lambda_i}{\lambda_{j+1}}
\,.
\]
Just as stated above, this is just an uper bound of the dimension,
$D_{KY} = \min\{d_j | \lambda_{j+1} < 0\}$.
\begin{align*}
  d_{j+1} - d_j &= 1 -\frac{\sum_{i=1}^{j+1} \lambda_i}{\lambda_{j+2}}
  + \frac{\sum_{i=1}^{j} \lambda_i}{\lambda_{j+1}} \\
  & = \frac{(\lambda_{j+2} - \lambda_{j+1})(\lambda_1 + \cdots + \lambda_{j+1})}
  {\lambda_{j+2}\lambda_{j+1}}
\end{align*}
Let $\lambda_1 + \cdots + \lambda_k \ge 0$ and
$\lambda_1 + \cdots + \lambda_{k+1} < 0$, then $d_{k+1} > d_{k}$ and
$d_k < d_{k-1}$, therefore
\begin{equation}
  D_{KY} = k + \frac{\sum_{i=1}^{k} \lambda_i}{|\lambda_{k+1}|}
\end{equation}
with $k$ the largest number making $\lambda_1 + \cdots + \lambda_k$
non-negative.

\paragraph{Summary}
                                            \toCB
The fractal dimension is important in a geometric point of view; however,
when dealing with system average, dimension which takes the probability
distribution into account is more informative. On the other hand, these
dimensions are in general difficult to calculate numerically for high
dimensional systems (except the KY dimension), so that is why cycle
expansion is more powerful when dealing with ergodic averages.
