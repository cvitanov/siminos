\ifsvnmulti
 \svnkwsave{$RepoFile: lyapunov/xiong/blog/CovVecs.tex $}
 \svnidlong {$HeadURL: svn://zero.physics.gatech.edu/siminos/xiong/blog/CovVecs.tex $}
 {$LastChangedDate: 2016-03-01 08:25:36 -0500 (Tue, 01 Mar 2016) $}
 {$LastChangedRevision: 4663 $} {$LastChangedBy: predrag $}
 \svnid{$Id: CovVecs.tex 4663 2016-03-01 13:25:36Z predrag $}
\fi

\section{\CLvs}
\label{sect:CovVecs}

\begin{description}

\item[2013-06-27 Predrag]
I rewrote ChaosBook.org chapters
\HREF{http://chaosbook.org/paper.shtml\#stability} {``Linear stability''},
\HREF{http://chaosbook.org/paper.shtml\#invariants} {``Cycle
stability''}, and
\HREF{http://chaosbook.org/paper.shtml\#Lyapunov} {``Lyapunov
exponents''}, to help graduate students. Any suggestion for improving
these chapters is greatly appreciated.

                              \inCB
{Finite-time} {Lyapunov} or \emph{characteristic} exponents and
associated \emph{principal axes} in the theory of dynamical systems
are discussed in chapter
\HREF{http://chaosbook.org/paper.shtml\#Lyapunov} {``Lyapunov
exponents''}. Oseledec \emph{Lyapunov exponents}\rf{lyaos} are the
$\zeit\to\infty$ limit of these. \emph{Floquet multipliers} and
\emph{eigen-vectors} are property of finite-time, compact invariant
solutions, such as \po s and \rpo s; they are explained in chapter
\HREF{http://chaosbook.org/paper.shtml\#invariants} {``Cycle
stability''}. \emph{Stability exponents}\rf{GoSuOr87} are the
corresponding long-time limit, and --in one of those frustrating
historical accidents-- the corresponding
\emph{stability} eigenvectors are misnamed in recent papers
\emph{Lyapunov vectors}, even though \emph{they are not} the
eigenvectors that correspond to the Lyapunov exponents. Sorry, not my
fault. The prefix `covariant' is meant to distinguish the two kinds of
eigenvectors. That's just confusing, for no good reason - Lyapunov
has nothing to do with linear stability described by the \jacobianM\
$\jMps$, as far as I know his paper\rf{Lyap1892}
is about $(\transp{\jMps}\!\jMps)$ and the associated principal axes.

Oseledec proofs are important in mathematics, but not
sensible for computational work in dynamical systems. For me the
Goldhirsch, Sulem and Orszag\rf{GoSuOr87} exposition is the clearest
(read it \HREF{http://chaosbook.org/library/GoSuOr87.pdf}{here}).
They correctly distinguish \emph{Lyapunov} eigenvalues and eigenvectors
from the \emph{stability} eigenvalues and eigenvectors.

At that time Goldhirsch \etal\ had no proof that the long-time Lyapunov
exponents converge to the stability exponents (Kazumasa, do you have
some more recent paper that you prefer to Goldhirsch \etal?).

Trevisan and Pancotti\rf{TrePan98} \emph{Periodic orbits, Lyapunov
vectors, and singular vectors in the Lorenz system} (see above, read it
\HREF{http://chaosbook.org/library/TrePan98.pdf}{here})) apparently need
to be cited for the observation that {\cLvs} reduce to Floquet
eigenvectors in the particular case of a {\po}. Seems so obvious it is in
ChaosBook without attribution, and Ruelle and
Eckmann\rf{EckmannRuelle1985} surely say the same (though I have not
checked). Most importantly for our project, they say:

``The leading Lyapunov vectors, as defined here, as well as the
asymptotic final singular vectors, are tangent to the attractor,
while the leading initial singular vectors, in general, point away
from it. Perturbations that are on the attractor can be found in the
subspace of the leading Lyapunov vectors.''

They have very nice pictures for Lorenz unstable orbits illustrating that.

\item[2013-06-27 Predrag]
Graduate students are brave people who immediately jump into the deep
end of the pool, without any testing. Here are two 2\dmn\ maps, the
Lozi map (for $a=1.85, b=0.3$)
% \PC{need figure of Lozi strange attractor}
\bea
   x_{n+1} &=& 1-a |x_{n}|+ b y_n  \continue
   y_{n+1} &=& x_{n}
\,,
\label{e_lozi_def}
\eea
and the H\'enon map (for $a=1.4, b=0.3$; described at length in
\refchap{c-Henon})
\bea
    x_{n+1}&=&1-ax^2_n+b y_n
        \continue
    y_{n+1}&=& x_n
\,,
\label{eq2.1d}
\eea
for which fixed points are available analytically\rf{DasBuch} (for
the Lozi map all periodic points are available analytically). So see
what Floquet exponents
your program returns for \po s of these humble maps, and then for
ergodic trajectories, before running the programs on ergodic
trajectories in zillion dimensions.

If your programs only apply to continuous time flows (ODEs rather
than maps), ChaosBook\rf{DasBuch} has \eqva\ and \po s for 2\dmn\
Lorenz and R\"ossler flows, and Evangelos has lots of \rpo s for the
\cLe; make sure your programs work on small systems first.

\end{description}
