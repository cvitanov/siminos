\ifsvnmulti
 \svnkwsave{$RepoFile: xiong/blog/CLVs.tex $}
 \svnidlong {$HeadURL: svn://zero.physics.gatech.edu/siminos/xiong/blog/CLVs.tex $}
 {$LastChangedDate: 2014-10-01 10:41:30 -0400 (Wed, 01 Oct 2014) $}
 {$LastChangedRevision: 3707 $} {$LastChangedBy: xiong $}
 \svnid{$Id: CLVs.tex 3707 2014-10-01 14:41:30Z xiong $}
\fi

% --------------------------------------------------------
\subsection{Oseledec splitting}
\label{sect:CLVs}

\begin{description}
  
\item[2013-09-04 \XD] Notes on the Ginelli \etal\
  method\rf{GiChLiPo12}, \arXiv{1212.3961}.

\item[2013-09-18 \XD]
  I didn't read Oseledec\rf{lyaos}. I try to summarize the basic ideas
  of \refref{GiChLiPo12}, which are also abstract to me.

\item[2014-09-30 \XD]
  I tried to study the Oseledec splitting again, and find that my 
  previous understanding is wrong, but the notes here may be helpful
  for me in future. I will rewrite my understanding in the next 
  subsection.

\end{description}


 Let $f^{k}(x) : \pS \mapsto \pS$ be a measure preserving map in \statesp\ \pS. Every point
 in this $\pS$ is a vector $x_{t}=(x_{t}^{(1)},x_{t}^{(2)},\dots,x_{t}^{(N)})\in \mathbb{R}^N$,
 where the subscript $t$ denotes instant time. If time is discretized
 $t=k\Delta t,\quad k=0,1,2...$ , then we can define Jacobina
 matrix of this system as
 $M_{k,n}=M^{k\Delta t}(x_n)=\frac{\partial{f(x_{(n+k)\Delta t})}}{\partial{x_{n\Delta t}}}$.
 Therefore, $M_{k,n}$ evolves the tangent space forward $k$ time steps from a point in tangent
 space $\delta x_n$.

    \PC{2013-09-15 Define dimension of \statesp\ \pS, you use it later.\\
    Xiong-2013-09-18: You are
    right. I define the space to be N-dimensional.}
    \PC{2013-09-15 What measure? Why measure preserving?\\
    Xiong-2013-09-18: For this one, I don't
    know the answer. I will update my opinion later.}
    \PC{2013-09-15 Index $n$ in $M_{k,n}$ means what? A pair of matrix indices,
    action on a vector $\ssp \in \pS$? To me it looks like the initial time is $n$,
    the \jacobianM\ is evaluated $k$ time steps later. Otherwise it would be stupid
    to have that index ${}_n$ on everything that follows.\\
    Xiong-2013-09-18: Yes, you are
    right. It is my mistake.}


 If the mapping is invertible, then we have several statements.

 \begin{itemize}
  \item forward/backward Oseledec matrix:
    \[
     \Xi^{\pm}= \lim_{k\to \pm \infty} \frac{1}{2k} \ln [M_{k,n}^{\dagger}M_{k,n}]
    \]
     exits.

  \item $\Xi^{\pm}$ share the same eigenvalues
  $\lambda_{n}^{(1)}> \lambda_{n}^{(2)}> \dots \lambda_{n}^{(m)}$.
      \PC{2013-09-15 If I reverse dynamics, and go $k\to - \infty$, shouldn't all these
      exponents change sign?\\
      Xiong-2013-09-18: Yes. If you reverse dynamics, the exponents of Jacobian
      matrix will change sign. But
      the definition of forward Oseledec matrix is the
      product of two Jacobian matrices. $M_{k,n}^{\dagger}$ evolves the system backward
      and $M_{k,n}$ evolves system forward, so forward Oseledec matrix will evolve system
      forward first and then backward, which consists two processes. Similarly, backward Oseledec
      martrix evolves the system backward first and then forward. For me, the definition
      of forward/backward Oseledec matrix is time invertible. I am not sure about this point since
      I don't know
      how mathmaticians proved this theorem.}

  The corresponding eigenspaces are
  $(U_{n}^{(1)})^{\pm}, (U_{n}^{(2)})^{\pm}, \dots (U_{n}^{(m)})^{\pm} $,
  with degeneracies
  $m_{n}^{(i)}=dim (U_{n}^{(i)})^{\pm}$. Also, we define Oseledec subspaces as
    \PC{2013-09-15 Are you saying there are $i$ positive $\lambda_{n}^{(\ell)}$,
    rest are negative? Have you assumed $\lambda_{n}^{(\ell)}\neq 0$?
    Why is $(U_{n}^{(i)})^{+}$ shared? Why is in $(T_{n}^{(i)})^{-}$?\\
    Xiong-2013-09-18: I am not assuming that there are $i$ positive exponents, nor
    they cannot be zero. I just want to construct a subspace $(T_{n}^{(i)})^{+}$
    which is spaned by the last $m-i+1$ eigenspaces of the forward Oseledec matrix, and
    a subspace $(T_{n}^{(i)})^{-} $ which is spaned by the first $i$ eigenspaces of the
    backward Oseledec matrix. So
    for each vector $u\in (T_{n}^{(i)})^{\pm}\backslash (T_{n}^{(i\pm 1)})^{\pm}$:
    \[
     \lim_{k\to \pm \infty} \frac{1}{k} \ln \frac{\| M_{k,n}u \|}{\| u \|} =\lambda_{n}^{(i)}
    \]
     by the way,
     \[
     \lim_{k\to \pm \infty} \frac{1}{|k|} \ln \frac{\| M_{k,n}u \|}{\| u \|} =\pm \lambda_{n}^{(i)}
    \]
    $(U_{n}^{(i)})^{+}$ is not included in the definition of $(T_{n}^{(i)})^{-}$. Sorry for my typo.
    }
  \begin{align}
   (T_{n}^{(i)})^{+} &= (U_{n}^{(i)})^{+} \oplus \dots \oplus (U_{n}^{(m)})^{+}\\
   (T_{n}^{(i)})^{-} &= (U_{n}^{(1)})^{-} \oplus \dots \oplus (U_{n}^{(i)})^{-}
   \label{eq::xiong_back}
  \end{align}
  The forward Oseledec subspace covers the space
  spanned by the eigenvectors corresponding
  to \PCedit{negative} eigenvalues; while, the backward Oseledec subspace
  covers the space spanned by the eigenvectors corresponding to
  \PCedit{positive} eigenvalues.

  Then, we have the following identity:
  \begin{align}
   &\lim_{k\to \pm \infty} \frac{1}{k} \ln \frac{\| M_{k,n}u \|}{\| u \|}  \\
   &=\lim_{k\to \pm \infty} \frac{1}{2k} \ln \frac{u^{\dagger}[M_{k,n}^{\dagger}M_{k,n}]u}{u^{\dagger} u}\\
   &=\frac{u^{\dagger}\Xi^{\pm}u}{u^{\dagger} u}\\
   &=\lambda_{n}^{(i)}
  \end{align}
  for $u\in (T_{n}^{(i)})^{\pm} \backslash (T_{n}^{(i\pm 1)})^{\pm}$

  This is the definition of Covariant Lyapunov Exponents. For
  ergodic system, these exponents are flow invariant,
  % which means that they don't depend on there positions.
  \PCedit{\ie, they almost always do not depend on the initial point $\ssp$
  at time $n$}.
    \PC{2013-09-15 not true if $\ssp$ is a fixed or periodic point.}
  \item Oseledec splitting is defined as

  \begin{equation}
   \Omega_{n}^{(i)}=(T_{n}^{(i)})^{+}\cup (T_{n}^{(i)})^{-}
   \label{equa::xiong_splitting}
  \end{equation}
   It is the overlap of the \texttt{ith} forward Oseledec subspace
   and the \texttt{i}th backward Oseledec subspace, so
   $M_{k,n}\Omega_{n}^{(i)}=\Omega_{n+k}^{(i)}$ for both $k>0$ and
   $k<0$ (invariant under time reversal). The vectors spanning
   ${\Omega_{n}^{(i)}}$ are called Covariant Lyapunov Vectors (CLV),
   and they are obtained by overlapping of forward Oseledec
   subspaces and backward Oseledec subspaces. Therefore, for a CLV
   $V_{n}^{(i)} \in \Omega_{n}^{(i)}$, we have
   $M_{k,n}V_{n}^{(i)}=d_{n,k}^{(i)}V_{n+k}^{(i)}$. That is
   \begin{equation}
    M_{k,n}V_{n}=V_{n+k}D_{k,n}
    \,,
    \label{eq::xiong_clv_evolve}
   \end{equation}
   where, $D_{k,n}$ is a diagonal matrix. In this sense, the CLVs are
   just eigenvectors of Jacobian matrix.

  \end{itemize}


% --------------------------------------------------------------------
\subsection{The Multiplicative Ergodic Theorem}

\subsubsection{Mathematical preparation}


% --------------------------------------------------------------------
\subsection{Gram-Schmidt vectors}

   We evolve the vectors in the tangent space as
   \begin{equation}
    \tilde{G}_{k+n}=M_{k,n}G_{n}
   \end{equation}
   where, $G_{n}=(g^{1}_{n},g^{2}_{n},\dots ,g^{N}_{n})$ is a orthogonal matrix
   whose columns are vectors in tangent space. In order to keep the
   columns of $\tilde{G}_{k+n}$ from expanding or contracting desperately,
   we conduct QR decomposition,
    \begin{equation}
    M_{k,n}G_{n}=\tilde{G}_{k+n}=G_{k+n}R_{k,n}
    \,,
    \label{eq::xiong_qr}
   \end{equation}
where, $G_{k+n}$ is a unitary matrix and $R_{k,n}$ is an upper-triangular matrix, whose diagonal
    elements represent the local expanding rate of CLVs.
    So, for an initial condition $G_{0}$, we have
    \begin{align*}
     G_{mk} &=M_{mk,0}G_{0} \\
     & =M_{k,(m-1)k}M_{k,(m-2)k}\dots (M_{k,0}G_{0}) \\
     & =M_{k,(m-1)k}M_{(k,(m-2)k}\dots (M_{k,k}G_{k})R_{0,k} \\
     & \dots  \\
     & =G_{mk}R_{(m-1)k,k}R_{(m-2)k,k}\dots R_{0,k} \\
    \end{align*}

    It provides us a way to calculate the Jacobian matrix. The benefit of QR decomposition is manifest:
    the product of two upper-triangular matrices is still an upper-triangular matrix, and the diagonal
    elements of the new matrix are just the product of diagonal elements of old matrices.

    \begin{itemize}
     \item GS vectors are not invariant under time-reversal.
     Let $\delta x_{k+n}=M_{k,n} \delta x_{n}$; then
     $\delta x_{n}^{\dagger}[M_{k,n}]^{\dagger}\delta x_{k+n}=\delta x_{k+n}^{\dagger} \delta x_{k+n}$
     the right side of this equation is just a number and set it be $C_{k+n}$, so we get
     $[M_{k,n}]^{\dagger}\delta x_{k+n}=(C_{k+n}/\delta x_{n}^{(i)},
     \dots C_{k+n}/ \delta x_{n}^{(i)}))^{\dagger}$, so the backward evolution can not take $\delta x_{k+n}$ back to
     $\delta x_{n}$.

     \item GS vectors are norm-dependent because the process to conduct QR decomposition will use the norm of
     vectors.

     \item The GS vectors will converge to the eigenvectors of the backward Oseledec matrix.
      \begin{equation}
       \lim_{k\to \infty} \| g^{(i)}_{n+k}-(d_{n+k}^{(i)})^{-} \| =0
      \end{equation}

      This information is very important for us to write the algorithm of calculating CLVs. The \texttt{ith}
      CLV belongs to the  \texttt{ith} Oseledec splitting space $\Omega_{n}^{(i)}$, and from
      \eqref{equa::xiong_splitting} and \eqref{eq::xiong_back}, it can seen that the \texttt{ith}
      CLV is in the space spanned by the first \texttt{i} eigenvectors of the backward Oseledec matrix.
      Therefore, if we evolve the system long enough to make sure that the GS vectors ${g^{(i)}_{n}}$ converge to
      the eigenvectors of the backward Oseledec matrix, then we can express the  \texttt{ith}
      CLV as $v_{n}^{(i)}=\sum_{j=1}^{i}g^{(j)}_{n}c_{n}^{j,i}$, which is just
      \begin{equation}
       V_{n}=G_{n}C_{n}
       \label{eq::xiong_clv}
      \end{equation}

      The coefficient matrix $C_n$ is an upper-triangular matrix.

      From \eqref{eq::xiong_clv_evolve}, \eqref{eq::xiong_qr} and \eqref{eq::xiong_clv}, we get
      \begin{equation}
       G_{n+k}C_{n+k}D_{k,n}=M_{k,n}G_{n}C_{n}=G_{n+k}R_{k,n}C_{n}
      \end{equation}
      So
      \begin{equation}
       C_{n}=R_{k,n}^{-1}C_{n+k}D_{k,n}
       \label{eq::xiong_coef}
      \end{equation}

      This is the backward evolution equation for coefficient matrix $C_n$. Ginelli \etal\
      have demonstrated that
      any non-singular upper triangular matrix will converge to the coefficient matrix $C_n$ under
      backward evolution \eqref{eq::xiong_coef} for sufficient long time.

    \end{itemize}


\subsection{Algorithm to calculate CLVs for \po s}

    \begin{itemize}
     \item \texttt{Forward transient ($ t:0\to T \cdot N_{temp}$):}
     Evolve the system for $N_{temp}$ periods. Choose $N_{temp}$ large enough to make
     sure the GS vectors converge to the eigenvectors of
     the backward Oseledec matrix, so equation
     \eqref{eq::xiong_clv} holds. The trajectory will repeat the orbit for $N_{temp}$
     times.

     If we evolve the {\statesp} for long time, the trajectory will depart from the
     orbit due to the amplification of noise.
     In order to keep the trajectory trapped in the orbit, every time the state point
     comes back to the initial point, it is reset to the initial point. Or, you can
     just evolve the system in {\statesp} for just one period, and record the state
     for each point. Reuse these states for $N_{temp}$ times when you evolve the system
     in tangent space.

     In this stage, we only store the last GS vectors $G_{T \cdot N_{temp}}$.

     \item \texttt{Forward dynamics ($t:T \cdot N_{temp}\to T \cdot (N_{temp}+1)$):}
     After the \texttt{forward transient} step, the GS vectors are converged to the
     eigenvectors of the
     backward Oseledec matrix. Now we continue to evolve the {\statesp} and tangent
     space for just one period. The reason is that if you choose to evolve GS vectors
     for several periods,
     you can only get a replica of GS vectors and upper triangular matrix $R_{n}$ for
     very orbit since GS vectors are already converged to the eigenvectors of the
     backward Oseledec matrix. But we should be aware that the upper triangular matrix
     $R_{n}$ will be reused for several times when you conduct \texttt{backward transient}.

     In this stage, we store GS vectors and upper triangular matrix
     $R_{n}$ for one period.

      \item \texttt{Backward transient ($t:T \cdot (N_{temp}+1)\to T \cdot (N_{temp}+1-N_{temp2})$):}
      We need use backward evolution equation for coefficient matrix $C_{n}$ by equation
      \eqref{eq::xiong_coef} in this step.  Since any none-singular upper
      triangular matrix will converge to the coefficient matrix $C_n$ under
      backward evolution, an identity matrix can be selected as this initial
      matrix. In \eqref{eq::xiong_coef}, $R_{k,n}$ is the upper triangular
      matrix we stored in step \texttt{Forward dynamics}, and $D_{k,n}$ is a diagonal
      matrix, which will only change the norm of each columns of $C_{n}$ in
      \eqref{eq::xiong_coef}. Therefore we don't need to care about this diagonal matrix
      if we normalize every column of $C_{n}$ every step.

       \begin{align}
       C_{n} & =R_{k,n}^{-1}C_{n+k} \label{eq::xiong_coef_new2}\\
       C_{n}^{(i)} &=\frac{C_{n}^{(i)}}{\| C_{n}^{(i)} \|}\quad \text{for every column}
       \label{eq::xiong_coef_new}
      \end{align}

      \Xiongedit{So I don't understand why Daniel calculates the
      matrix $D_{k,n}$ in this step.}

      Backward evolve tangent space according to method \eqref{eq::xiong_coef_new} and
      \eqref{eq::xiong_coef_new2} for
      $N_{temp2}$ periods, and the upper triangular
      matrix $R_{k,n}$ from step \texttt{Forward dynamics} will be reused for $N_{temp}2$
      times. Now, the initial arbitrary none-singular upper
      triangular matrix converges to the coefficient matrix $C_n$.

      In this step, only the final $C_n$ is stored.

      \item \texttt{Backward dynamics ($t:T \cdot (N_{temp}+1-N_{temp}2)\to T \cdot (N_{temp}-N_{temp}2)$):}
      We continue backward the tangent space by method \eqref{eq::xiong_coef_new} and
      \eqref{eq::xiong_coef_new2} for one period,
      but this time we need to store $C_n$ for each step, so we get the CLVs
      in each point in the orbit by:

      \begin{equation}
       V_{n}=G_{n}C_{n}
      \end{equation}

      Where, the GS matrix $G_{n}$ comes from step \texttt{Forward dynamics}.


    \end{itemize}
