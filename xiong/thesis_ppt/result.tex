\begin{frame}%[allowframebreaks]
  \frametitle{Main contribution}

  \htbc{
    (Our simulations use 62 degrees of freedom.)
  }
  
  \vfill

  \htrc{
    For one-dimensional Kuramoto-Sivashinsky equation defined on a periodic
    domain of size L = 22, the dimension of the inertial manifold is 8.
  }
  
  \vfill

  \setbeamercolor{block title}{fg=white, bg=green!75!black}
  \begin{block}{}
    \textrm{
      \small
      X. Ding  and P. Cvitanovi\'c,
      ``Periodic eigendecomposition and its application in
      Kuramoto-Sivashinsky system,''
      {\color{red} \emph{SIAM J. Appl. Dyn. Syst.}} {\bf 15}, 1434--1454 (2016)
    }
  \end{block}

  \setbeamercolor{block title}{fg=white, bg=green!75!black}
  \begin{block}{}
    \textrm{
      \small
      Ding, X., Chat\'e, H., Cvitanovi\'c, P., Siminos, E., and Takeuchi, K. A.,
      ``Estimating the dimension of the inertial manifold from unstable
      periodic orbits,''
      {\color{red}\emph{ Phys. Rev. Lett.}} {\bf 117}, 024101 (2016)
    }
  \end{block}
\end{frame}
