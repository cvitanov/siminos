% siminos/inputs/defsKSred.tex    macros for ksReduced.tex
% $Author: predrag $ $Date: 2015-11-03 12:10:38 -0500 (Tue, 03 Nov 2015) $

% ES ver. 2.0 reboot: re-started writing 			2011-11-03
%   after being interrupted by job applications,
%   childbirth, moving to Dresden, real plumbing, etc.
% ES started writing and thinking sometime in		March 2011

%                \newif\ifdraft \drafttrue
                \ifdraft
%% Date your comment. Once settled, please move intersting comments
%% to siminos/blog/freeze.tex (append at the end)
\newcommand{\PCedit}[1]{{\color{magenta}#1}}
\newcommand{\ESedit}[1]{{\color{blue}#1}}
\newcommand{\RLDedit}[1]{{\color{red}#1}}
\newcommand{\ES}[1]{\ESedit{\small [ES: #1]}}
\newcommand{\PC}[1]{\PCedit{\small[PC: #1]}}
\newcommand{\RLD}[1]{\RLDedit{\small [RLD: #1]}}
\newcommand{\toCB}{\marginpar{\footnotesize 2CB}}  % to compare with ChaosBook
\newcommand{\inCB}{\marginpar{\footnotesize now in CB}} % entered in ChaosBook
                \else
\newcommand{\PCedit}[1]{{#1}}
\newcommand{\ESedit}[1]{{#1}}
\newcommand{\RLDedit}[1]{{#1}}
\newcommand{\ES}[1]{}
\newcommand{\PC}[1]{}
\newcommand{\RLD}[1]{}
\newcommand{\toCB}{}
\newcommand{\inCB}{}
                \fi %end of internal draft switch

%%%%%%% Setup hyperlinks
\definecolor{hreflinkcolor}{rgb}{0.13,0.17,0.83}
\hypersetup{colorlinks=true,urlcolor=hreflinkcolor,
		linkcolor=hreflinkcolor,citecolor=hreflinkcolor}
% Graphics files are in directory
\graphicspath{{../figs/}}
\bibliographystyle{plain}

%%%%%%%%% Macros
%% ES: Please only add macros that will allow to easilly change journal.
%% Do not add macros to keep terminology, notation etc flexible.
%% Internal edits/comments macros are OK, as are some math macros already added.
%% PC: I feel personally persecuted. OK, OK, I surrender - you're the Boss.

\newcommand{\beq}{\begin{equation}}
\newcommand{\eeq}{\end{equation}}
\newcommand{\bseq}{\begin{subequations}}
\newcommand{\eseq}{\end{subequations}}
\newcommand{\barr}{\begin{array}}
\newcommand{\earr}{\end{array}}

%% this block follows APS style:
\newcommand{\rf}     [1] {\cite{#1}}
\newcommand{\refref} [1] {Ref.~\onlinecite{#1}}
\newcommand{\refrefs}[1] {Refs.~\onlinecite{#1}}
\newcommand{\refeq}  [1] {Eq.~(\ref{#1})}                   %APS style
\newcommand{\refeqs} [2] {Eqs.(\ref{#1}--\ref{#2})}         %APS style
\newcommand{\reffig} [1] {FIG.~\ref{#1}}                    %APS style
\newcommand{\reffigs} [2] {FIGS.~\ref{#1} and~\ref{#2}}     %APS style
\newcommand{\refFig} [1] {FIG.~\ref{#1}}                    %APS style
\newcommand{\refFigs} [2] {FIGS.~\ref{#1} and~\ref{#2}}     %APS style
\newcommand{\refsect}[1] {Section~\ref{#1}}                 %APS style
\newcommand{\refappe}[1] {Appendix~\ref{#1}}                %APS style
\newcommand{\reftab} [1] {TABLE~\ref{#1}}                   %APS style
%% end of APS style specific

\newcommand{\etc}{{etc.}}
\newcommand{\etal}{{\em et al.}}
\newcommand{\ie}{{i.e.}}
\newcommand{\cf}{{\em cf.\ }}
\newcommand{\eg}{{e.g.\ }}

\newcommand{\KS}{Kuramoto-Siva\-shin\-sky}
\newcommand{\KSe}{Kuramoto-Siva\-shin\-sky equation}
\newcommand{\pCf}{plane Couette flow}
\newcommand{\PCf}{Plane Couette flow}

\newcommand{\Rls}[1]{\ensuremath{\mathbb{R}^{#1}}}
\newcommand{\Clx}[1]{\ensuremath{\mathbb{C}^{#1}}}
\newcommand{\Un}[1]{\ensuremath{\textrm{U}(#1)}}         % in DasBuch
\newcommand{\On}[1]{\ensuremath{\textrm{O}(#1)}}
\newcommand{\SOn}[1]{\ensuremath{\textrm{SO}(#1)}}         % in DasBuch
\newcommand{\Dn}[1]{\ensuremath{\textrm{D}_{#1}}}              % in DasBuch
\newcommand{\Zn}[1]{\ensuremath{\textrm{C}_{#1}}}              % in DasBuch
\newcommand{\Ztwo}{\ensuremath{\textrm{C}_2}}                % in DasBuch
\newcommand{\Refl}{\ensuremath{\sigma}}
\newcommand{\bbU}{\mathbb{U}}

\newcommand{\chebT}{\mathrm{T}}
\newcommand{\chebU}{\mathrm{U}}

\newcommand{\Nd}{\ensuremath{d}} % state-space dimension
\newcommand{\Nc}{\ensuremath{N}} % complex state-space dimension

\newcommand{\ii}{\ensuremath{\mathrm{i}}} % \sqrt{-1}

%%%%%%% From chaosbook def.tex (only if absolutely necessary)
\newcommand{\wwwcb}[1]{       % keep homepage flexible:
                  {\tt \href{http://ChaosBook.org#1}
              {ChaosBook.org#1}}}
\newcommand{\arXiv}[1]{
              {\tt \href{http://arXiv.org/abs/#1}{arXiv:#1}}}
\newcommand{\weblink}[1]{{\tt \href{http://#1}{#1}}}
\newcommand{\Poincare}{Poincar\'e}
\newcommand{\jEigvec}[1][]{\ensuremath{{\bf e}^{(#1)}}} % right jacobiam eigenvector
\newcommand{\ExpaEig}{\ensuremath{\Lambda}}
\newcommand{\eigExp}[1][]{
\ifthenelse{\equal{#1}{}}{\ensuremath{\lambda}}{\ensuremath{\lambda^{(#1)}}}
                        }
\newcommand{\eigRe}[1][]{
\ifthenelse{\equal{#1}{}}{\ensuremath{\mu}}{\ensuremath{\mu^{(#1)}}}
                        }
\newcommand{\eigIm}[1][]{
  \ifthenelse{\equal{#1}{}}{\ensuremath{\omega}}{\ensuremath{\omega^{(#1)}}}
            }
\newcommand{\eqv}{equilib\-rium}
\newcommand{\Eqv}{Equilib\-rium}
\newcommand{\eqva}{equilib\-ria}
\newcommand{\Eqva}{Equilib\-ria}
\newcommand{\reqv}{rela\-ti\-ve equilib\-rium}
\newcommand{\Reqv}{Rela\-ti\-ve equilib\-rium}
\newcommand{\reqva}{rela\-ti\-ve equilib\-ria}
\newcommand{\Reqva}{Rela\-ti\-ve equilib\-ria}
\newcommand{\po}{periodic orbit}
\newcommand{\Po}{Periodic orbit}
\newcommand{\rpo}{rela\-ti\-ve periodic orbit}
\newcommand{\Rpo}{Rela\-ti\-ve periodic orbit}
\newcommand{\period}[1]{{\ensuremath{T_{#1}}}}
\newcommand{\Shift}{\ensuremath{\tau}}
\newcommand{\shift}{\ensuremath{d}}
\newcommand{\EQV}[1]{\ensuremath{E_{#1}}}
\newcommand{\REQV}[2]{\ensuremath{TW_{#1#2}}} % #1 is + or -
\newcommand{\PO}[1]{\ensuremath{PO_{#1}}}
\newcommand{\RPO}[1]{\ensuremath{\mathrm{RPO}_{#1}}}
% \newcommand{\RPO}[1]{\ensuremath{\text{RP$#1$}}}
% RPO_{period to 2-4 significant digits} - relative PO.  We use ^{+,-}
% to distinguish between members of a reflection-symmetric pair.
% ES: it seems we need 4 significant digits to distinguish rpo's.

\newcommand{\pS}{\ensuremath{{\cal M}}}          % symbol for state space
\newcommand{\ssp}{a}            % state space point
\newcommand{\vel}{\ensuremath{v}}   % state space velocity
\newcommand{\pSRed}{\ensuremath{\bar{\cal M}}} % reduced state space
\newcommand{\sspRed}{\ensuremath{\bar{\ssp}}}    % reduced state space point, experiment
\newcommand{\velRed}{\ensuremath{\bar{\vel}}}    % ES reduced state space velocity
\newcommand{\velRel}{\ensuremath{c}}    % relative state or phase velocity
\newcommand{\slicep}{\ensuremath{\bar{\ssp}'}}   % slice-fixing point, experimental
\newcommand{\expct}    [1]{\left\langle {#1} \right\rangle}
\newcommand{\timeAver} [1]{\overline{#1}}
\newcommand{\expctE}{\ensuremath{E}}    % E space averaged
\newcommand{\Norm}[1]{\|{#1}\|}
\newcommand{\braket}[2]
		   {\langle{#1}\vphantom{#2}|\vphantom{#1}{#2}\rangle}
\newcommand{\bra}[1]{\langle{#1}\vphantom{ }|}
\newcommand{\ket}[1]{|\vphantom{}{#1}\rangle}
\newcommand{\Lint}[1]{\frac{1}{L}\!\oint d#1\,}
\newcommand{\pSpace}{x}       % Hamiltonian phase space x=(q,p) coordinate
\newcommand{\LieElrep}{\ensuremath{\mathbf{G}}}
\newcommand{\Sset}{Inflection hyperplane}
\newcommand{\sset}{inflection hyperplane} 	% {singularity hyperplane}
											% {singular set}
\newcommand{\sspSing}{\ensuremath{\ssp^*}} 	% inflection point
\newcommand{\sspRSing}{\ensuremath{\sspRed^*}} 	% inflection point, reduced space
