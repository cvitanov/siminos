\ifsvnmulti
 \svnkwsave{$RepoFile: lyapunov/Lyapunov.tex $}
 \svnidlong {$HeadURL: svn://zero.physics.gatech.edu/siminos/lyapunov/Lyapunov.tex $}
 {$LastChangedDate: 2020-07-02 12:19:27 -0400 (Thu, 02 Jul 2020) $}
 {$LastChangedRevision: 7451 $} {$LastChangedBy: predrag $}
 \svnid{$Id: Lyapunov.tex 7451 2020-07-02 16:19:27Z predrag $}
\fi

\chapter{Lyapunov vectors}
\label{s:LyapunovVec}

This part of the blog focuses on the use of `{\cLvs}' (CLV) to identify the
number of degrees of freedom that capture the physics of a `turbulent' PDE on
a spatial domain.

As material is written up, parts of it will migrate from its
current placement into coherent sections, suitable for
inclusion into ChaosBook.org, or, God Forbid, actual {\em
publications}.

\begin{description}

\item[2009-09-04 Predrag]
Talked to  Hugues Chat\'{e}, got very
enthused about their results about splitting the KS
eigenvectors into the `{\entangled}', inertial manifold ones,
and the `hyperbolically isolated', transient ones. We should
reexamine eigenvectors of our solutions, might find the split
already on our very small system sizes (they work at $L=96$).

Here is my email to Chat\'{e}:

\end{description}

Iggy (as Anglos spell it),
%
now that I have become your fan - I just love all those `unphysical,'
lonely, isolated, alienated eigenvectors banished into their neurotic
transitory world off the inertial manifold -  I'm entering following
references to your papers:
\refrefs{PoGiYaMa06,ginelli-2007-99,YaTaGiChRa08,TaGiCh09}
in the final final edits of to be published in SIADS\rf{SCD07} (part I;
part II is still being penned - 60,000 relative {\po s} and no
place to go). Let me know if these are the right ones, add/correct what
needs to be edited, corrected.

I have no clue why you guys call linearized stability
eigenvectors `Lyapunov' (in My Book Lyapunov has to do with
the $J^T J$ symmetric matrix, with only real parts of Floquet
multipliers captured, and eigenvectors that do not seem to
mean anything), but you must have your reasons. We have 10-30
leading eigenvectors for our `narrow cell' with $L=22$, for
each one of the 60,000 invariant solutions that clever but
mindless computation has handed us, and nothing interesting
we had found to do with them, but they should presumably fit
into your bigger L eigenvectors as into a glove. We see
`unphysical' eigenvalues drop like a ton of rocks - only 4
leading eigenvalues really matter, 2 of them marginal at
$L=22$.

PS - for something more `physical,' check out
\HREF{http://www.warwick.ac.uk/~masax/}{Dwight Barkley}'s long pipe calculations with
\HREF{http://adsabs.harvard.edu/abs/2008APS..DFD.BD008B}{David Moxley}.
You'll find them inspirational. Probably the same story as KS, but explaining the
phase transition between frozen puffs and turbulent puffs would make plumbers happy.

\section{Reading assignments}

\begin{description}
\item[2014-10-25 Predrag] Moved most of the literature discussions to
\\
\texttt{siminos/xiong/thesis/chapters/cLvsLit.tex}

\end{description}


\section{QR decomposition}
\label{QRdecomp}

\begin{description}
\item[2009-10-10 Sara and Predrag] Understood
    \refref{ginelli-2007-99} (might want to check
    \refref{PoGiYaMa06} as well). Still need to copy
    relevant definitions from ChaosBook.org here.
\end{description}

Consider a $d$-dimensional discrete-time dynamical system. Let
$\ssp_{n}\in \pS \subset \reals^d$ denote the \statesp\ point
at time $t_{n}$ and let $\{{\bf g}_{n}^{(\ell)}\}$, $\ell=1,
\ldots d$, be a set of $d$ orthogonal vectors, ${\bf
g}_{n}^{(\ell)} \cdot {\bf g}_{n}^{(k)} = \delta_{\ell k}$,
that span the tangent space $T\pS$ of the flow at time $t_{n}$.
Collect these into the $n$th \emph{GS basis} (or GS fundamental
matrix), written as the orthogonal matrix
\[
{\bf G}_n = ({\bf g}_n^{(1)} , \ldots ,  {\bf g}_n^{(d)})
\]
whose columns are the basis vectors $\{{\bf g}_{n}^{(\ell)}\}$.
Start now at $t_{n-1}$. Iterating the evolution equations once
maps the ${\bf G}_{n-1}$ orthogonal basis into a set of
non-orthogonal, non-unit basis vectors $\overline{\bf G}_n =
(\overline {\bf g}_n^{(1)} , \ldots , \overline {\bf
g}_n^{(d)})$,
\[
\overline {\bf g}_n^{(\ell)} =\jMps_{n-1}
            \, {\bf g}_{n-1}^{(\ell)}
\,,
\]
where $\jMps_{n-1}$ is the \jacobianM\ of the
transformation evaluated at time $t_{n-1}$.
The $n$th GS basis is now obtained by applying the
Gram-Schmidt orthogonalization to the vectors
$\overline {\bf g}_n^{(\ell)}$. Pick
$\overline{\bf g}_n^{(1)}$ as the direction of the first
Gram-Schmidt vector,
${\bf g}_n^{(1)} = \overline{\bf g}_n^{(1)}/|\overline{\bf g}_n^{(1)}|$.
Pick the component of
$\overline{\bf g}_n^{(2)}$ normal to ${\bf g}_n^{(1)}$
as the direction of the second
Gram-Schmidt vector, normalize to the unit eigenvector,
\bea
{\bf g}_n^{(2)} &=& \frac{1}{|\cdots|}
            \left(\overline{\bf g}_n^{(2)}
                  - (\overline{\bf g}_n^{(2)} \cdot {\bf g}_n^{(1)})
                      \, {\bf g}_n^{(1)}
            \right) \,,
\continue
{\bf g}_n^{(3)} &=& \frac{1}{|\cdots|}
            \left(\overline{\bf g}_n^{(3)}
                  - (\overline{\bf g}_n^{(3)} \cdot {\bf g}_n^{(2)})
                      \, {\bf g}_n^{(2)}
                  - (\overline{\bf g}_n^{(3)} \cdot {\bf g}_n^{(1)})
                      \, {\bf g}_n^{(1)}
            \right) \,,
\nnu
\eea
and so on.
Thus the Gram-Schmidt orthogonalization  amounts to the $\overline{\bf G} =
{\bf Q}{\bf R}$ decomposition, where
\[
\overline{\bf G}_n =
 {\bf G}_n ({\bf G}_n^T \overline{\bf G}_n) =
 {\bf Q}_n {\bf R}_n \,.
\]
The columns of the orthogonal matrix ${\bf Q}_n = {\bf G}_n$
are the elements of the $n$th GS basis, while ${\bf R}_n= {\bf
G}_n^T \overline{\bf G}_n$ is an upper-triangular matrix whose
nonzero elements are obtained by projecting each vector
$\overline {\bf g}_n^{(j)}$ onto the subspace spanned by $\{
{\bf g}_n^{(k)}\}$ with $k \leq j$.
        \PC{${\bf R}_n= {\bf
G}_n^T \overline{\bf G}_n$ formula is reminiscent of the
expression for the \jacobianM\ in terms of fundamental ones,
$\jMps_n= {\bf G}_n {\bf G}_{n-1}^{-1}$.}
        \PC{the reminder of this entry is clippings from \refref{PoGiYaMa06},
        still needs to be edited; there is more QR algebra to fill in. Apologies.
        }

%
%%%%%%%%%%%%%%%%%%%%%%%%%%%%%%%%%%%%%%%%%%%%%%%%%%
\SFIG{GinelliFrames} {}{ \JacobianM\ transports initial
orthogonal frame into a non-orthogonal one. This frame is then
QR decomposed into an $R$-matrix and the next Gram-Schmidt
frame, which is then transported by the next \jacobianM, and so
on. A vector that starts within the subspace spanned by such
finite Gram-Schmidt basis stays within the subspace spanned by
the successive orthogonal frames.
}{fig:GinelliFrames}
%%%%%%%%%%%%%%%%%%%%%%%%%%%%%%%%%%%%%%%%%%%%%%%%%%
%

What we have is a succession of co-moving re-orthogonalized
frames, \reffig{fig:GinelliFrames},
\[
{\bf G}_{n-1} \to \overline{\bf G}_n =
 {\bf G}_n {\bf R}_n
 \to
{\bf G}_{n+1} {\bf R}_{n+1}{\bf R}_n
\]
where ${\bf G}_n$ is an orthogonal frame spanning the tangent
space at $\ssp_n$, and
 \[
{\bf R}_n{\bf R}_{n-1} {\bf R}_{n-1}\cdots {\bf R}_1
 \]
 is the \jacobianM\ computed within the
co-moving frame. It keeps track of stretching along
eigen-directions, but not rotations of the frame, that
information is contained in the imaginary parts of
eigen-exponents of the linearized flow. At each iteration
the linearized flow stretches and squashes the hypercube
whose edges are the $n$th GS basis into a parallelepiped
whose edges are the vectors forming the columns of
$\overline{\bf G}_{n+1}$. Projections of these
vectors onto the $(n\!+\!)$th GS hypercube form
the upper-triangular co-moving frame representation
of the \jacobianM\ $R_{n+1}$.

Ershov and Potapov have shown\rf{ErshPot98} that, by repeating
the above procedure up to a time $t_m$ for $m$ much larger than
$n$,  the GS basis converges to an orthogonal set of vectors
$\{{\bf e}_m^{(k)}\}$, $k=1,\ldots, d$ - the $m$th Gram-Schmidt
vectors - which solely depend on the \statesp\ point
$\ssp_{m}$. Thus the {\cLvs} are independent
of where the backward evolution is started along a given
trajectory, provided that it is sufficiently far in the future.
    \PC{For \po s, we need no Ershov and
    Potapov\rf{ErshPot98}. Oseledec arguments are only
    needed for a generic long-time chaotic trajectory}

The Lyapunov exponents $\lambda_1 \geq \lambda_2 \geq \ldots
\geq \lambda_d$ are the time-averaged values of the logarithms
of the diagonal elements of ${\bf R}_n$.
        \PC{Ginelli's talk shows that that diagonal elements of ${\bf R}_n$
            are the eigenvalues of the \jacobianM\ $\jMps_n$.}

Assume that a set of Gram-Schmidt vectors at $\ssp_m$ has been
generated by iterating the generic initial condition $\ssp_0$.
Let ${\bf u}_m^{[j]}$ be a generic vector inside the subspace
$\pS^{[j]}$ spanned by $\{{\bf g}_m^{(k)}\}$, $k=1,\ldots, j$,
\ie, by the first $j$ Gram-Schmidt vectors at time $t_m$. This
vector can be iterated backward in time by inverting the
upper-triangular matrix ${\bf R}_m$: if the $c_m^{i}=({\bf
g}_m^{(i)} \cdot {\bf u}_m^{[j]} )$ are the coefficients
expressing ${\bf u}_m^{[j]}$ in terms of the Gram-Schmidt
vectors at $\ssp_m$, then $c_{m-1}^{i}= \sum_k [{\bf
R}_m^{-1}]_{ik} c_m^{k}$. Since ${\bf R}_m$ is
upper-triangular, it is easy to verify that ${\bf u}_n^{[j]}
\in T\pS^{[j]}_n$ at all times $t_n$.


Iterating ${\bf u}_m^{[j]}$ backward for a sufficiently large
number $(m-n)$ of times, it eventually aligns with the
(backward) most expanding direction within $T\pS^{[j]}$. This
defines  ${\bf e}_n^{(j)}$, our intrinsic $j$-th (forward)
expanding direction at the \statesp\ point $\ssp_n$.

In order to verify that ${\bf e}_n^{(j)}$ is covariant, define
the matrix $[{\bf C}_m]_{ij} = c_m^{ij}$,
\[
{\bf C}_m = {\bf R}_m {\bf C}_{m-1}
\,.
\]
By multiplying both sides by ${\bf Q}_m$ and substituting
$\overline {\bf G}_m$ for its QR decomposition on the resulting
right hand side, one is left with ${\bf
e}_{m}^{(j)}=\jMps_{m-1} {\bf e}_{m-1}^{(j)}$ for $j=1,\ldots,
d$.

{\CLvs} $\{{\bf v}_{m}^k\}$ constitute an
intrinsic, covariant basis defining expanding/contracting
directions in \statesp.
In the presence of degeneracies {\cLvs}
have to be grouped according to the multiplicity of the
corresponding Lyapunov exponent.

The $i$th Lyapunov exponent is the average of the growth rate
of the $i$th the {\cLv}. For 2\dmn\ maps and
3\dmn\ flows this has been shown by B. Eckhardt, D. Yao,
Physica D {\bf 65}, 100 (1993); G. Froyland, K. Judd and A. I.
Mees, Phys. Rev. E {\bf 51}, 2844 (1995); and A. Politi
\etal\rf{PoGiYaMa06}.

Evidence of the validity of the approach was given in
\refref{PoToLe98}, where {\cLvs} were
introduced to characterize time {\po s} in a 1D
lattice of coupled maps. There, it was found that the number
of nodes (changes of sign) in a {\cLvs} is
directly connected to the position of the corresponding
Lyapunov exponent within the Lyapunov spectrum.


The {\cLvs} are also well defined for
non-invertible dynamics, since it is necessary and sufficient
to follow backward a trajectory previously generated forward in
time. The determination of the {\cLvs} is
very efficient. The main computational bottleneck is the memory
required to store the matrices ${\bf R}_n$ and the $n$-time
Gram-Schmidt vectors during the forward integration. The memory
required can be substantially reduced by occasionally storing
the instantaneous configuration in real and tangent space and
re-generating the rest when needed.


\section{{\CLvs}, an algorithm}

\begin{description}
\item[2009-10-10 Sara and Predrag] Understood
\refref{ginelli-2007-99} (might want to
check \refref{PoGiYaMa06} as well).
\end{description}

\paragraph{
A general method to
determine {\cLvs} in both discrete- and
continuous-time dynamical systems.
            }

The Benettin \etal\rf{bene80a} algorithm to calculate the
Lyapunov exponents relies on construction of orthogonal sets of
Gram-Schmidt vectors. They  are not covariant, \ie, the
Gram-Schmidt vectors at a given \statesp\ point are not mapped
by the linearized dynamics into the Gram-Schmidt vectors of the
forward images of this point. In contrast, the \jacobianM\
$\jMps_n$ eigenvectors \jEigvec[i] also span the
$d$-dimensional tangent space, but are generically not normal.
In numerical work, frequent Gram-Schmidt re-orthogonalization
of the tangent coordinate frame is necessitated by the
exponential growth-rate separation along $\jMps_n$
eigendirections as $n$ increases; for a long orbit direct
computation of eigenvalues of $\jMps_n$ yields a set of
exponentially separated leading multipliers
$\{\ExpaEig_1,\ExpaEig_2,\cdots\}$, subset of which can
resolved within a given machine precision.

The method of \refref{ginelli-2007-99} enables computation of
{\em all} eigenvectors of $\jMps_n$. The two key ides are:
\begin{enumerate}
  \item
Due to the upper-triangular structure a Gram-Schmidt
re-orthogonalization matrix ${\bf R}_n$, a vector $\deltaX$
that starts within the tangent subspace $\deltaX \in
T\pS_{\ssp}^{[j]}$ spanned by the first $j$ Gram-Schmidt basis
vectors, {\em stays} within that subspace spanned by a comoving
frame under subsequent evolution and re-orthogonalizations.
\item Once a set of $\{{\bf R}_1,{\bf R}_2,\cdots,{\bf R}_n\}$
along trajectory $\{\ssp_1,\ssp_2,\cdots,\ssp_n\}$ has been
computed and stored in the memory, it can be used to describe
the action of the the linearized flow \jacobianM\ $\jMps_n$
both \emph{forward} and \emph{backward} in time.

So far, this is true of any QR decomposition. Now to
hyperbolic dynamics; order the \jacobianM\ $\jMps_n$
eigenvectors \jEigvec[\ell] by the real parts of their
eigen-exponents, and let $T\pS^{[j]}$ be the tangent
subspace spanned by the leading $j$ eigenvectors.

Strongly contracting $\ExpaEig^{(j)}$ multiplier forward in
time, becomes the leading $1/\ExpaEig^{(j)}$ multiplier
backward in time. Matrix power method then pulls out this
eigenvalue as the leading one within the subspace
$T\pS^{[j]}$. \emph{Presto:} Increase the dimension of the
subspace by one, and you get the next $\ExpaEig^{(j+1)}$.
Repeat, and you get all eigenvalues and eigenvectors, even
those insanely contacting ones, like $\ExpaEig^{(j)}
\approx 10^{-137}$.

\end{enumerate}
\emph{Bonus points:}
\begin{enumerate}
  \item For {\po s} we need to evaluate ${\bf R}_n$ for
  only one traversal of the cycle: the method then yields {\em
  all} Floquet vectors and Floquet exponents (no need to mess
  with convergence of `Lyapunov' eigenvectors). As unstable \po
  s fill the ergodic component of the \statesp\ hierarchically,
  their linearized stable / unstable manifolds tessellate the
  stable / unstable foliation with exponentially increasing
  accuracy. No need to explore the \statesp\ by mindless
  numerics, starting with a random initial point.
	\PC{for
  ChaosBook; define Floquet vector to be eigenvector, so change
  all `Floquet eigenvectors' to `Floquet vectors.'}

  \item \emph{Unexpected bonus:} Near-orthogonality of
      these eigenvectors appears to separate the women from
      girls. According to Ginelli
      \etal\rf{ginelli-2007-99}, it gives us apparently a
      sharp criterion to count the number of `{\entangled}'
      degrees of freedom for a PDE solved on a compact
      domain.
\end{enumerate}

A coordinate-independent, local decomposition of \statesp\
into covariant Lyapunov directions is called Oseledec
splitting\rf{ruelle79,EckmannRuelle1985}.
Its numerical implementation in \refref{ginelli-2007-99}
is based on forward/backward
iterations of the tangent dynamics and determination
a set of dynamics covariant directions,
called `{\cLvs} (CLV).'

\begin{description}
  \item[Ruslan 2009-10-14] I'm sorry to ruin your optimism
      here, but I don't think this procedure is numerically
      stable: The round-off errors will spill outside
      $T\pS^{[j]}$ and then $1/\ExpaEig^{(k)}$ multipliers
      for $k > j$ will stretch these errors.

\item[Predrag 2009-10-15]
You are right that as it stands, we have not explained the
algorithm. The answer seems to be in Wolfe and
Samelson\rf{WoSa07} (above and in ChaosBook.org/library, link
below), perhaps the most important article to understand.
If Ginelli \etal\rf{ginelli-2007-99} say the key thing (how
to compute covariant eigenvectors, actually) they hide it
well.

  \item[Ruslan 2009-10-14]
Another thing that you have noticed is that they don't say
anything about complex eigenvalues.  This is because in the
construction of the algorithm they assume that all Lyapunov
exponents are distinct, which implies that all Floquet
multipliers of {\po s} are real and distinct as well
(Even though they write $\lambda_1 \geq \lambda_2 \cdots$
instead of $>$, this is mentioned in one of the papers you
discussed above, but I don't remember which one).  I just
hope that, if we do have a pair of complex conjugate
eigenvalues, the corresponding two Lyapunov vectors will span
the same subspace as the real and imaginary parts of the
corresponding eigenvector.  But this needs to be verified.

\item[Predrag 2009-10-15]
You are right that Wolfe and Samelson\rf{WoSa07} cannot prove
that their method converges for degenerate sets of eigenvalues
(complex pairs being a particular case - real parts are
equal, and the method maps orthogonal frame to orthogonal
frame, codes stretching/shrinking into $R_{ij}$ matrix,
ignores frame rotations. Ginelli \etal\rf{ginelli-2007-99}
deal with complex eigenvalues (see the plots I copied to the
blog above), so I guess it works. I might ask Takeuchi how
they really do it.

  \item[Ruslan 2009-10-14]
 This may be the best
algorithm we have for determining Lyapunov vectors along a
typical chaotic orbit, but if we want to determine them for a
PO or a RPO, then I think that the best way to do this is to
find out how to solve the eigen-problem for a very badly
conditioned matrix $J$ which can be represented as a product
$J = J_{n}J_{n-1}\cdots J_1$ of not-so-badly conditioned
matrices $J_j$.  You seem to have something along these lines
in the extract `Transport of vector fields', \refappe{sect:transport}, but I
would prefer to see this construction done for maps, since
\emph{computers don't do flows}.


\item[Predrag 2009-10-15]
It's usually easier for maps than flows, and as this is all
formulated for finite times, I think it works for discrete
and continuous finite times equally well. Computers do flows
pretty well, that's how we compute \FloquetM\ \monodromy.
Wolfe and Samelson\rf{WoSa07} did not do it for Floquet
eigenvectors. Trevisan and Pancotti (see above, and
\wwwcb{/library}) apparently need to be cited for the
observation that {\cLvs} reduce to Floquet
eigenvectors in the particular case of a {\po}.
Seems so obvious it is in ChaosBook without attribution, and
Ruelle and Eckmann\rf{EckmannRuelle1985} surely say that.

\item[Ruslan 2009-10-15]
Sure, all I'm saying is that a flow in a computer is actually
a map, since it's represented by a numerical integration
scheme.  Anyway, I'll read Wolfe and Samelson\rf{WoSa07}
next.  By the way, cannot download Trevisan and Pancotti from
the library, please check the link.  Also, regarding the
relationship between  {\cLvs} and Floquet
eigenvectors: I can see that it will work for real distinct
eigenvalues (aka Floquet multipliers), but again it is not
clear how they are related in the case of complex or repeated
eigenvalues.

\end{description}


\section{Separate the women from girls:
		 Hyperbolicity}


\begin{description}
\item[2009-10-15 Predrag]
Added references \refref{WoSa07,KaCoCa02,TrePan98,PaSzLoRo09}
(discussed above) to
\\
\wwwcb{/library}. As far as I can tell, the important one is
\refref{WoSa07}.

\item[2009-10-15 Predrag to Ruslan]
I have lost interest in quotienting $\SOn{2}$? Nothing could
be further from the actual state of affairs. Ashley Willis
has made a slice work for a down-stream traveling wave in
pipe (downstream we have pure $\SOn{2}$, and for
angular-traveling waves we have $\On{2}$), and we might soon have a
neighborhood spiral-out. His plots of $\dot{\theta}$ already
show hiccups with velocity going large every so often.

But computing Lyapunov vectors seems important as well. I've
put all the papers you did not yet read onto the
\wwwcb{/library}.

\item[2009-10-15 Ruslan]
I believe slices will work for traveling waves and other
types of relative steady (or nearly steady) states, but not
for generic \rpo s and other 'moving' states.  There we will
probably have to learn how to live with a discontinuous
$\theta(t)$.

\item[2009-10-29 Predrag] Tend to agree.

%\RLDedit{
%\item[2009-10-15 Ruslan]
%I cannot download Trevisan and Pancotti from
%the library, please check the link.
%}
%
%\item[2009-10-15 Predrag]
%Sorry, forgot the *.pdf suffix - now Trevisan and Pancotti is
%downloadable from the library.
%
\end{description}

\section{``Lyapunov analysis'' workshop}
\label{iscpif}

\begin{description}

\item[2009-10-27 Predrag] Evangelos (physicist formerly known
as Vaggelis) and I are at the ``Lyapunov analysis from theory
to geophysical applications,''
\HREF{http://iscpif.fr/LTG09}{www.iscpif.fr/LTG09} meeting with
all the Lyapunov heavies. I have put the Ginelli, Pikovsky and
Talagrand talks into \wwwcb{/library}; all talks are eventually
meant to be on their website.
\end{description}

\begin{itemize}
  \item F. Ginelli, {\em Characterizing dynamics with
      covariant Lyapunov vectors.} Please read this, it
      addresses directly the questions in the blog above,
      concerning details of the QR algorithm. Francesco
      credits Abarbanel \etal\rf{AbBrKe91,AbBrKe91a} and
      Politi \etal\rf{PoToLe98} for earlier Lyapunov
      vectors. He says that the distinction from the Wolfe
      and Samelson\rf{WoSa07} `intersection algorithm' is
      that their key formula \beq
   D^{(n)}  y^{(n)} = 0 \ee{eq:WoSa07} is unstable as it
  needs inversion of $D^{(n)}$, whereas the dynamical
  algorithm requires only matrix multiplication and is thus
  more reliable.

  \item G. Radons, {\em Lyapunov modes in extended
      dynamical systems.} See Predrag's 2009-10-29 notes below.
  \item V. Lucarini, {\em Parametric smoothness and
      self-scaling of the statistical properties of a
      minimal climate model} is closely related to
      Siminos thesis, with 5-dimensional \cLe\ as a
      special case, and a 10-dimensional variant of
      Lorenz that has also 2nd Fourier modes, but (do
      not understand why) those can be rotated
      independently, so the symmetry is
      $\SOn{2}\times\SOn{2}$. He keeps the term that
      accounts for heat generation by viscosity, and
      this seems to break the symmetry.
\begin{description}
\item  Evangelos: Talked to Lucarini. The			
symmetry is $\SOn{2}\times\SOn{2}$ since
the model is a truncation of Boussinesq equation
in two spatial dimensions. Lucarini claims that
the viscosity term does not break the symmetry in
Boussinesq equation but does break it in its truncation.
It sounds more like he does not truncate correctly. Without
the viscosity term his truncation behaves like \cLe\ in the
degenerate case of coexisting, symmetry related attractors.
One should be able to also get a version analogous to the generic
situation in \cLe\ with dynamics along the direction of group action.

\item[2009-12-02 Evangelos] Read part of Lucarini and Saltzman papers. It
appears that Lucarini is wrong, the system cannot have
$\SOn{2}\times\SOn{2}$. Boundary conditions in both papers are periodic
in $x$ direction and free in upper and lower boundary (along the $z$
direction), which implies a vanishing stream function and vorticity,
while we also have $\theta=0$ along the vertical boundaries. Lucarini
uses a complex Fourier expansion in both directions therefore arguing he
has $\SOn{2}\times\SOn{2}$ symmetry, but fails to notice he cannot
satisfy boundary conditions. Saltzman (and of course Lorenz) appears to
get the right truncation by fixing certain terms in his Fourier series
(but I have not checked the details of his calculation.)

\end{description}
  \item K. A. Takeuchi, {\em Lyapunov analysis captures
      the collective dynamics of large chaotic
      systems.} Kazumasa AKA Kazz is our best hope for
      computing Floquet eigenvectors for our KS \rpo s,
      unless we do it ourselves.

  \item A. Pikovsky, {\em An introduction to Lyapunov
      analysis
        }
  \end{itemize}
The remaining talks were by the weathermen and women. They were
very fascinating, as there is amazing amount of atmospheric and
ocean data available.
  \begin{itemize}
  \item A. Trevisan and O. Talagrand, {\em On the
      existence of an optimal subspace dimension for
        4DVar }
  \item M. Ghil, {\em Data assimilation and predictability
      in climate models} explained Kalman filters in detail
      that might be useful for including into Lippolis
      thesis. While I am at it; we had heard
      Abarbanel\rf{ACFK09} talk about that, it might be a
      better reference.
  \item G. Lapeyre, {\em Nonlinear singular vectors and
      atmospheric predictability }
  \end{itemize}
The next bunch showed very pretty signatures of
Lyapunov-exponent colored 2-dimensional ocean dynamics and
high atmosphere (Antarctic ozon hole) dynamics. Legras has
interesting formulas for diffusion enhanced by stretching
of unstable manifold fronts.
  \begin{itemize}
  \item F. d'Ovidio, {\em Lyapunov analysis for ocean
      front detection }
  \item B. Legras, {\em Transport and Mixing}
  \item E. Shuckburgh, {\em Particle dispersion and
      effective diffusivity }
  \item J. B. Sallee, {\em Observed Lagrangian
      dispersion of surface drifters in the Southern
      ocean }
  \item E. Hernandez-Garcia, {\em Stretching structures
      from finite-size Lyapunov exponents: their impact
      across all biological scales.}
 \item J. Verron and O. Titaud, {\em On the
     assimilation of submesoscale observations into
     ocean models }
  \end{itemize}

\begin{description}

\item[2009-10-29 G. Radons talk]  Gunther says `orthogonal'
(rather than Gram-Schmidt) vectors. This seems more
descriptive, but I am not sure we should adopt this usage, as
the use of Gram-Schmidt sequential orthogonalizations is
crucial in imposing the upper-triangular structure on $R$'s in
the $QR$ decomposition.

In his monograph\rf{Gas97} Gaspard talks of `Mather
decomposition' (rather than Oseledec); recheck Gaspard's
discussion.
\[
J^t = {\bf f}^{(j)}(t){}^T \Gamma_j {\bf e}^{(j)}(0)
\]
where ${\bf e}^{(j)}(0)$ are the initial {\cLvs},
${\bf f}^{(j)}(t){}^T$ the final {\cLvs}, and $\Gamma$ encodes
`stretching factors.'

\item[2009-10-29 Predrag on G. Radons talk]
CVLs he calls  `Lyapunov modes'. That is
motivated by thinking of these eigenvectors as a generalization
of `normal modes' of mechanical systems.						\inCB

Radons \etal\ central result: Lyapunov modes split in (a)
infinitely many `trivial,' hyperbolically decaying modes that
are isolated and do not mix and (b) finite number of
`{\entangled},' entangled modes, in the tangent space of the
attractor, whose number is proportional to the size $L^D$ of
the $D$-dimensional PDE system. Schematically, the `{\entangled}'
ones are pointing along the `entangled,' {\entangled} tangent
directions.

In his talk, Radons shows examples of Lyapunov modes evolving
in time with $\ssp(t)$. Physical ones seem localized,
hyperbolic ones seem very Fourier-mode like. Spatial Fourier
spectrum averaged of the time peaks with increase in $k$, peak
at $k$. Physical ones are flat.

Probability density of angles between adjacent {\entangled}
eigenvectors computed over long time is flat, not peaked at
$90^o$, see \reffig{fig:lyapSpecCLG} above. In this way the
`{\entangled}' tangent space is transported across the whole curved
strange manifold ergodically, and nonhyperbolicity of the
attractor is used as a test that trajectory initiated along a
given direction stays within the attractor. The `trivial,'
hyperbolically decaying eigen-directions that are isolated
exhibit no such small inter-angles anywhere along the chaotic
trajectory.

The very dramatic drop between decaying and entangled modes is
seen in the Domination of Oseledec Splitting (DOS), a figure in
Yang \etal\rf{YaTaGiChRa08} that I did not include in
\reffig{fig:lyapSpecCLG}. The idea is this: there is always a
long finite time $\tau$ Lyapunov exponent time for which
eigenvalues are correctly ordered. For short a time $\tau$
`{\entangled}' ones violate the ordering -\ie, nearby finite time
Lyapunov exponent values cross as $\tau$ increases. They plot a
fraction of the time this happens along a long trajectory, for
a $\tau=0.2$, and for $\tau=2$ - both very sharp. Now,
$\tau=0.2$ seems very short on the scale where shortest
periodic solutions are of order $T=20$. Predrag thinks that DOS
might be more important than angles, as separation should also
work for Smale horseshoes repellers embedded in high
dimensions. Chat\'e says the information in angles is equally
good. {\bf A. Pikovsky} is unconvinced about the angles
approach, and thinks one can construct counter-examples where
it would fail.

While it is very impressive for KS, with {\entangled} modes sharply
separated into a low-$j$ rectangle, DOS is not a rectangle for
CGL equation, but more of a sausage along the diagonal, and it
might reflect different kinds of turbulence (`phase' turbulence
vs. `amplitude' turbulence). Predrag thinks quotienting the
symmetry might help, see blog entry of 2009-10-30 below.



\end{description}

\section{Hydrodynamic Lyapunov modes}
\label{sect:HDLmodes}

\begin{description}

\item[2013-04-06 Predrag] I admit to not having yet developed any
appreciation of `hydrodynamic Lyapunov modes' or `Lyapunov modes'.
But they must mean something, as Eckmann,
Posch\rf{PoHi00,HooPoFoDeZh02}, \HREF{http://williamhoover.info/}
{Hoover}, and Morriss study them.

Eckmann\rf{EckGat00} writes:
``Hydrodynamic behavior in phase space is present in every
extended system with a continuous symmetry. In the models considered
in \refref{DePoHoo96} the symmetries in question are translation and
Galilei invariance, precisely those which enable the hydrodynamic
description of a fluid in terms of the Navier-Stokes equations.
However, it is for the first time that a similar phenomenon is
observed in tangent space.''

\item[2012-02-12 Predrag]
{\em Covariant hydrodynamic Lyapunov modes and strong stochasticity threshold
 in Hamiltonian lattices}
by M. Romero-Bastida, Diego Paz\'o, and Juan M. L\'opez\rf{romero-bastida2012},
\arXiv{1202.3476}. They scrutinize
``
the reliability of covariant and Gram-Schmidt Lyapunov vectors for
capturing hydrodynamic Lyapunov modes (HLMs) in one-dimensional
Hamiltonian lattices and show that, in contrast with previous claims,
HLMs do exist for any energy density, so that strong chaos is not
essential for the appearance of genuine (covariant) HLMs. In
contrast, Gram-Schmidt Lyapunov vectors lead to misleading results
concerning the existence of HLMs in the case of weak chaos.
''
\item[2013-04-07 Evangelos] This is a helpful paper.
They clearly demonstrate the perils of using OLVs instead of {\cLvs}, by varying the
scalar product used in Gram-Schmidt orthogonalization. It answers some of my questions
for now (except for physical motivation - the argument is the usual, they study
it because it has recently received much attention).

\item[2013-04-07 Predrag] Eckmann\rf{EckGat00} might be right,
but it is not the obvious that random matrices are enlightening
in this context. They do some numerics, and show a counter-example
where Lyapunov (or GS) vectors do not work.

\item[Comparison of covariant] and orthogonal {Lyapunov
vectors} by Yang and Radons\rf{YaRa10}. They compare
covariant and orthogonal Lyapunov vectors (CLVs/OLVs), with
respect to the hydrodynamic Lyapunov modes (HLMs). Both
Hamiltonian and dissipative HLMs formerly detected via OLVs
survive if {\cLvs} are used instead. Moreover the previous
classification of two universality classes works for {\cLvs} as
well, i.e. the dispersion relation is linear for Hamiltonian
systems and quadratic for dissipative systems respectively.
The significance of HLMs changes in different ways for
Hamiltonian and dissipative systems with the replacement of
OLVs by {\cLvs}. For general dissipative systems with
nonhyperbolic dynamics the long wave length structure in
Lyapunov vectors corresponding to near-zero Lyapunov
exponents is strongly reduced if {\cLvs} are used instead,
whereas for highly hyperbolic dissipative systems the
significance of HLMs is nearly identical for {\cLvs} and OLVs.
In contrast the HLM significance of Hamiltonian systems is
always comparable for {\cLvs} and OLVs irrespective of
hyperbolicity.

Appendices are potentially useful: Transformation properties
of Lyapunov exponents, {\cLvs} and hyperbolicity under changes
of coordinate are discussed.

\item[2013-03-31 Evangelos] Could somebody please explain a few points about HLMs studied
in the above Yang and Radons paper?

a) Why are HLMs worth studying. The argument seems to be that since they correspond to
almost zero Lyapunov exponents they are ``slow modes'' of the system and thus they could be
useful in some multiple scale analysis. Moreover, since they correspond to positive Lyapunov
exponents they should participate in the dynamics (and thus they are not like the trivial
modes one finds in e.g. KSe). However, I cannot see the value of such modes, when there are
a few hundred modes corresponding to larger positive Lyapunov exponents (as seems to be the typical
case). How could scale separation work, and how could any role be played by these HLMs when there
exist multiple more ``unstable'' modes of various time-scales.

b) I cannot see why they say that OLVs and {\cLvs} give the same results. A look at their
figure 3 suggests that OLVs basically give nonsense. Isn't that what one would expect?
Aren't OLVs (especially the ones corresponding to near-zero Lyapunov exponents) just a
byproduct of the Gram-Schmidt method? Aren't the {\cLvs} valuable exactly because they are the
only vectors that have some invariant meaning?

c) I do not understand what the acceptable standards are in statistical physics in order to
establish a universality class are. For instance in Fig. 2 of \refref{YaRa06}, it seems that
points for small $k$ (exactly where universal behavior is supposed to occur) span an order of
magnitude for the XY model, while for the rest of the systems studied, one would have to be
very optimistic in order to read a single slope.

d) What do we learn from such universality classes?

e) They say that these HLMs appear because of continuous symmetries. As they study
coupled map lattices, they say that the relevant symmetry is a shift in the dependent
variable (in all lattice points). However, it seems to me that the relevant symmetry is
a discrete lattice translation symmetry, which in the continuous limit (number of points
on the lattice going to infinity) becomes a continuous translation symmetry. This would
be the only common (non-trivial) symmetry of these systems with KSe for which the same authors
also find HLMs in \refref{YaRa06}.

\item[2013-04-06 Predrag] Can you have a look at the `Lyapunov modes'
source literature: Eckmann, Posch\rf{PoHi00,HooPoFoDeZh02},
\HREF{http://williamhoover.info/} {Hoover}? Not sure what is the best
review (I have saved the articles in ChaosBook.org/library, as
usual.) It might work better than trying to decode Yang... Blog as
you progress/regress...


\item[2013-04-08 Kazumasa]
I agree with Predrag that we should not worry so much
 about what Yang and Radons write on the hydrodynamic Lyapunov modes
 (because they have to sell HLMs at any rate)
 and that we do not expect HLMs in the KS equation.
At least for the KS equation,
 they just mistook the {\transient} modes (as in \refref{YaTaGiChRa08}) for HLMs
 (because of the sinusoidal structure).
Concerning the dispersion relation, we showed
 in \refref{YaTaGiChRa08} that
 $\lambda$ for large $k$ is given by the highest-order term in $k$,
 i.e., $\lambda \sim -k^4$ for KS,
 while $\lambda$ for small $k$ plotted by Yang and Radons are
 modes near the {\entangled}-{\transient} threshold (at least for KS),
 where everything is mixed up and hard to extract meaningful information...

\item[2013-04-08 Kazumasa]
Moreover, I and Hugues doubt that HLMs are well-defined
 even in Hamiltonian systems,
 based on preliminary (but time- and memory-consuming) data.
We tried to reproduce Yang and Radons' results
 in \rf{Yang.Radons-PRL2008}
 for their symplectic coupled map lattice and found that
 one cannot set any threshold between ordinary and hydrodynamic Lyapunov modes
 on the basis of the domination of Oseledec splitting,
 as opposed to their claims
 (e.g., blue curve in their Fig.1b seems to suggest a threshold,
 but in logarithmic scales
 it decreases monotonously over the full index range).
Second, when we increase the system size,
 the apparent fraction of the modes with sinusoidal structure
 (which is the sign of HLMs here) decreases (slowly and irregularly).
This suggests that HLMs, if exist, might not occupy
 a finite fraction of the spectrum in the infinite-size limit...
I don't hate the idea of HLMs, rather I like it and hope to see them
 in a clear way, but I don't have chance to do so yet.

\item[2013-04-08 Kazumasa]                      \inCB
By the way, Hugues and I do not omit ``hydrodynamic'' when calling HLMs,
 in order to reserve the term ``Lyapunov mode'' to mean simply a set
 of a Lyapunov exponent and its associated vector.

\end{description}



\section{2009-12-14 Getting fussy}

\begin{description}
\item[Strict and fussy modes splitting in the tangent space]
of the Ginzburg-Landau equation
by Pavel V. Kuptsov, Ulrich Parlitz (G\"{o}ttingen University),
\HREF{http://arxiv.org/abs/0912.2261v1}{arXiv:0912.2261}. They say:
``
In the tangent space of some spatially extended dissipative
systems one can observe `{\entangled}' modes which are highly
involved in the dynamics and are decoupled from the remaining set
of hyperbolically `isolated' degrees of freedom representing
strongly decaying perturbations. This modes splitting is studied
for the Ginzburg--Landau equation at different strength of the
spatial coupling. We observe that isolated modes coincide with
eigenmodes of the homogeneous steady state of the system; that
there is a local basis where the number of non-zero components of
the state vector coincides with the number of `{\entangled}' modes;
that in a system with finite number of degrees of freedom the
strict modes splitting disappears at finite value of coupling;
that above this value a fussy modes splitting is observed.
''

The big mystery is that I have met them and wrote ton of notes into
this blog (I believe) but have never found these notes again. Well...
Briefly: I'm impressed with Pavel's computations of {\cLvs}.
In his way of computing their relative angles one sees: (1) the separation
of positive exponents from negative ones (that's what {\po} theory likes)
(2) a range of vectors up to Kaplan-Yorke (3) a different range of vectors
up to Ginelli \etal\ cliff.

\item[2010-11-12 Predrag to
\HREF{http://www.coas.oregonstate.edu/faculty/samelson.html}{Roger M. Samelson}:]
We still do not have a working code, but we will eventually need to get
something running if we hope to count the number of '{\entangled}' degrees of
freedom for our plane Couette and pipe transitional boundary shear
turbulence simulations with Gibson and assorted gentlemen.

\item[2010-11-24 Samelson writes to Predrag:]
Many thanks for the interest in the paper and the
approach.  It's great to hear that Christopher's insight and method is
gaining some adherents.

Juan Lopez (Cantabria, Spain) has been using it also, and invited me (us;
only I could go) to a workshop on this topic in Dresden last January
\HREF{http://www.pks.mpg.de/~ecodyc10/}
{www.pks.mpg.de/$\sim$ecodyc10},
where we had a mild competition of views with Ginelli.   Juan has also
invited us (this time Christopher Wolfe <clwolfe@ucsd.edu> will
represent) to a symposium on this topic at the SIAM Snowbird meeting May
2011.  If you end up publishing something that uses the approach, we'd
definitely be interested to hear.

Re the codes, Christopher may still have something or be able to assist,
though he's been working on other topics during his postdoc at SIO with
Paola Cessi.  I'm copying this reply to him, so that he can comment.
Juan seemed a generous fellow and perhaps may also be willing to share
something from his group.

\item[2010-12-30 Bounds on Floquet multipliers]
from Random Matrix Theory,
by Alejandro Gabriel Monastra <monastra@tandar.cnea.gov.ar>,
\\
\HREF{http://arxiv.org/abs/1012.5952}{arXiv:1012.5952}. Predrag
wrote him an email:

(1)
this is a bit of nitpicking, but I have to get it right, so let me know what you
think (and what authoritative references we are supposed to follow). In
Chaobook.org chapters
4 Local stability
and
5 Cycle stability
I define the real part $i$th Floquet exponent of a prime {\po} $p$
in units of time,

$ \lambda_{p,i} = (1/T_p) \ln |\ExpaEig_{p,i}| ,$
$\ExpaEig_{p,i}$ the Floquet multiplier,

otherwise I cannot identify the mean long time average expansion rate
as a Lyapunov exponent: Lyapunovs are measured in time units, right?
What you call `Lyapunov' is a dimensionless number,
the logarithm of the Floquet multiplier?

(2)
Next remark I have not thought through, so I'm probably wrong - your bound
on Floquet multipliers is  $|\ExpaEig_{p,i}| > \exp(0.850738) =2.34137...$. As
multipliers of unstable orbits grow exponentially, for a chaotic flow this
constrains only a few shortest {\po s}, unless there are sequences of
orbits of nearly marginal stability (but then the flow probably has a mixed,
non-hyperbolic phase space).

For short {\po s} anything larger than 1 goes. For example, in
\HREF{http://www.cns.gatech.edu/~predrag/papers/preprints.html\#akp}
{``Periodic orbit quantization} of the anisotropic Kepler problem''
(with F. Christiansen), CHAOS 2, 61 (1992),\rf{CC92} I see numbers like
$|\ExpaEig_{p,i}| = 1.167...$, and could probably dig up lots of small
instability {\po s} in other systems. Of course, anisotropic Kepler
is not a nice hyperbolic system at all, might work for something nicer.

\item[2011-01-17 Characteristic {Lyapunov} vectors]
in chaotic time-delayed systems
by Paz{\'o} and L{\'o}pez\rf{PaLo10},
\HREF{http://arxiv.org/abs/1101.2779}{arXiv:1101.2779}. They say:
``
We compute Lyapunov vectors (LVs) corresponding to the largest Lyapunov
exponents in delay-differential equations with large time delay. We find
that characteristic LVs, and backward (Gram-Schmidt) LVs, exhibit
long-range correlations, identical to those already observed in
dissipative extended systems. In addition we give numerical and
theoretical support to the hypothesis that the main LV belongs, under a
suitable transformation, to the universality class of the
Kardar-Parisi-Zhang equation. These facts indicate that in the large
delay limit (an important class of) delayed equations behave exactly as
dissipative systems with spatiotemporal chaos.
''

\item[2011-03-04 PC]
\emph{Probing the local dynamics of periodic orbits by
      the generalized alignment index (GALI) method}
by T. Manos, Ch. Skokos and  Ch. Antonopoulos\rf{MaSkAn11},
\HREF{http://arxiv.org/abs/1103.0700}{arXiv:1103.0700}. They say:
``
We extend GALI realm of applicability to the
local dynamics of periodic orbits. We show theoretically and verify
numerically that for stable periodic orbits the GALIs tend to zero
following particular power laws for Hamiltonian flows, while they
fluctuate around non-zero values for symplectic maps. The
GALIs of unstable periodic orbits tend exponentially to zero, both for
flows and maps. We also apply the GALIs for investigating the dynamics in
the neighborhood of periodic orbits, and show that for chaotic solutions
influenced by the homoclinic tangle of unstable periodic orbits, the
GALIs can exhibit a remarkable oscillatory behavior during which their
amplitudes change by many orders of magnitude. Finally, we use the GALI
method to elucidate further the connection between the dynamics of
Hamiltonian flows and symplectic maps. In particular, we show that, using
for the computation of GALIs the components of deviation vectors
orthogonal to the direction of motion, the indices behave for flows as
they do for maps.
''

\item[2011-03-05 PC] I keep returning to this article,
\emph{Geometrical constraints on finite-time
    		{Lyapunov} exponents in two and three dimensions},  by
Thiffeault and Boozer\rf{ThiBoo99}. They say ``
Constraints are found on the spatial variation of finite-time Lyapunov
exponents of two and three-dimensional systems of ordinary differential
equations. In a chaotic system, finite-time Lyapunov exponents describe
the average rate of separation, along characteristic directions, of
neighboring trajectories. The solution of the equations is a coordinate
transformation that takes initial conditions (the Lagrangian coordinates)
to the state of the system at a later time (the Eulerian coordinates).
This coordinate transformation naturally defines a metric tensor, from
which the Lyapunov exponents and characteristic directions are obtained.
By requiring that the Riemann curvature tensor vanish for the metric
tensor (a basic result of differential geometry in a flat space),
differential constraints relating the finite-time Lyapunov exponents to
the characteristic directions are derived. These constraints are realized
with exponential accuracy in time. A consequence of the relations is that
the finite-time Lyapunov exponents are locally small in regions where the
curvature of the stable manifold is large, which has implications for the
efficiency of chaotic mixing in the advection-diffusion equation. The
constraints also modify previous estimates of the asymptotic growth rates
of quantities in the dynamo problem, such as the magnitude of the induced
current.
''

Or perhaps this is more useful:
Thiffeault\rf{Thiffeault2001},
\emph{Derivatives and constraints in chaotic flows:
asymptotic behaviour and a numerical method.}
He says: ``
In a smooth flow, the leading-order response of trajectories to
infinitesimal perturbations in their initial conditions is described by
the finite-time Lyapunov exponents and associated characteristic
directions of stretching. We give a description of the second-order
response to perturbations in terms of Lagrangian derivatives of the
exponents and characteristic directions. These derivatives are related to
generalised Lyapunov exponents, which describe deformations of phase
space elements beyond ellipsoidal. When the flow is chaotic, care must be
taken in evaluating the derivatives because of the exponential
discrepancy in scale along the different characteristic directions. Two
matrix decomposition methods are used to isolate the directions of
stretching, the first appropriate in finding the asymptotic behaviour of
the derivatives analytically, the second better suited to numerical
evaluation. The derivatives are shown to satisfy differential constraints
that are realised with exponential accuracy in time. With a suitable
reinterpretation, the results of the paper are shown to apply to the
Eulerian framework as well.
''

See \refappe{sect:stabComoving} for further discussion.

Also \emph{Matrix exponential-based closures for 		the turbulent
subgrid-scale stress tensor} by Chevillard, Li, Eyink and  Meneveau, is
of potential interest, because they express the stress tensor as the
product of the matrix exponential of the resolved velocity gradient
tensor multiplied by its transpose - of import here, and perhaps also for
Domenico's noisy cigars. They say ``
Two approaches for closing the turbulence subgrid-scale stress tensor in
terms of matrix exponentials are introduced and compared. The first
approach is based on a formal solution of the stress transport equation
in which the production terms can be integrated exactly in terms of
matrix exponentials. This formal solution of the subgrid-scale stress
transport equation is shown to be useful to explore special cases, such
as the response to constant velocity gradient, but neglecting
pressure-strain correlations and diffusion effects. The second approach
is based on an Eulerian-Lagrangian change of variables, combined with the
assumption of isotropy for the conditionally averaged Lagrangian velocity
gradient tensor and with the `Recent Fluid Deformation' (RFD)
approximation. It is shown that both approaches lead to the same basic
closure in which the stress tensor is expressed as the product of the
matrix exponential of the resolved velocity gradient tensor multiplied by
its transpose. Short-time expansions of the matrix exponentials are shown
to provide an eddy-viscosity term and particular quadratic terms, and
thus allow a reinterpretation of traditional eddy-viscosity and nonlinear
stress closures. The basic feasibility of the matrix-exponential closure
is illustrated by implementing it successfully in Large Eddy Simulation
of forced isotropic turbulence. The matrix-exponential closure employs
the drastic approximation of entirely omitting the pressure-strain
correlation and other `nonlinear scrambling' terms.
''

\item[2011-03-24 PC] \arXiv{1103.4536},
\emph{Extensive and sub-extensive chaos in globally-coupled dynamical systems},
by Takeuchi, Chat\'e, Ginelli, Politi, and Torcini. They say:

``
Using a combination of analytical and numerical techniques, we show that
chaos in globally-coupled identical dynamical systems, be they
dissipative or Hamiltonian, is both extensive and sub-extensive: their
spectrum of Lyapunov exponents is asymptotically flat (thus extensive) at
the value $\lambda_0$ given by a single unit forced by the mean-field,
but sandwiched between sub-extensive bands containing typically
$\mathcal{O}(\log N)$ exponents whose values vary as $\lambda \simeq
\lambda_\infty + c/\log N$ with $\lambda_\infty \neq \lambda_0$.
''

\item[2009-10-27 Predrag]
J. Kurchan, {\em Studying unusually regular or unusually chaotic
trajectories } uses a bunch of cloned trajectories, with probability of
cloning largest for the extremal values of finite-time Lyapunov exponent,
in order to detect fine, structures, such as minuscule elliptic islands
within a chaotic sea. Seems to work impressively well, and mindlessly
detect features that otherwise would require real thinking.
      	[search for Kurchan in channelflow.org]

\item[2011-06-30 Predrag] Above observation is in agreement with
\emph{Manifold structures of unstable periodic orbits and the appearance of periodic
windows in chaotic systems},
by Miki U. Kobayashi and Yoshitaka Saiki\rf{KoSa11}.
They say
``
Manifold structures of the Lorenz system and the Kuramoto-Sivashinsky
system are investigated in terms of unstable periodic orbits embedded in
the attractors. Especially, the change of manifold structures are focused
on when some parameter values are varied. It is found that the angle
between a stable manifold and an unstable manifold (manifold angle) at
sample points on an unstable periodic orbit, which is measured by using
the {\cLvs}, characterizes the parameter at which the
periodic window corresponding to the unstable periodic orbit vanishes,
that is a saddle- node bifurcation point. In particular, when the minimum
value of the manifold angle along an unstable periodic orbit at a
parameter is small (large), the corresponding periodic window exists near
(away from) the parameter. It is concluded that the window sequence in a
parameter space can be predicted from the manifold angles of unstable
periodic orbits at some parameter. This approach helps us and periodic
windows including quite small ones.
''

They go off on a tangent (it's pretty obvious that tangencies have to do
with small angles, nein?), but such is life.
Lorenz system is hyperbolic at $r = 28$ (as proven by Tucker\rf{tucker1-2}),
so they study larger $r$ when it starts developing folds.

I have the paper and can put it into our Zotero library, if there is
anyone out there that is actually interested in the research discussed in
this blog.

\item[2011-07-02 Predrag]
Masanobu Inubushi, Miki U. Kobayashi, Shin-ichi Takehiro,
and Michio Yamada,
\emph{Covariant {Lyapunov} analysis of chaotic {Kolmogorov} flows
and time-correlation functions}

``
We study a hyperbolic/non-hyperbolic transition of the flows on
two-dimensional torus governed by the incompressible Navier-Stokes
equation (Kolmogorov flows) using the method of covariant Lyapunov
analysis developed by Ginelli et al.(2007). As the Reynolds number is
increased, chaotic Kolmogorov flows become non-hyperbolic at a certain
Reynolds number, where some new physical property is expected to appear
in the long-time statistics of the fluid motion. Here we focus our
attention on behaviors of the time-correlation function of vorticity
across the transition point, and find that the long-time asymptotic form
of the correlation function changes at the Reynolds number close to that
of the hyperbolic/non-hyperbolic transition, which suggests that the
time-correlation function reflects the transition to non-hyperbolicity.
''

I have the talk (110702Inubush.pdf) and can put it into our Zotero
library, if there is anyone out there that is actually interested in the
research discussed in this blog.

\item[2013-03-05 Predrag]
Now we should also read \refref{InKoTaYa12,InKoTaYa12a}, and there is also a 122
pages report \emph{``Covariant Lyapunov Analysis of Navier-Stokes
Turbulence''} that might be Inubushi's thesis\rf{Inubushi13}.

\item[2011-07-07 Predrag] Overheard Skokos apologizing to Miki Kobayashi
for not citing the original Japanese paper that introduced the method for
computation of Lyapunov {\cLvs}. Have to find that reference...

\item[2011-07-08 PC]                                        \toCB
Gave major PR to Kazzmania: check my
\HREF{http://chaosbook.org/version13/Maribor11.shtml}{Maribor lectures},
\HREF{http://chaosbook.org/overheads/dimension/dimension.pdf}{piece
\#5}: ``Dynamics in infinitely many dimensions.''


\item[2011-07-09 Predrag]
Gerlach, Eggl, and {Skokos}\rf{GeOgSk11},
``Efficient integration of the variational equations of
    multi-dimensional {Hamiltonian} systems: {Application} to the
    {Fermi-Pasta-Ulam} lattice'' say:

``
According to Tangent Map method, a symplectic integrator is used to
approximate the solution of the Hamilton's equations of motion by the
repeated action of a symplectic map S, while the corresponding tangent
map TS, is used for the integration of the variational equations.
''

``
We study the problem of efficient integration of variational equations in
multi-dimensional Hamiltonian systems. For this purpose, we consider a
Runge-Kutta-type integrator, a Taylor series expansion method and the
so-called `Tangent Map' (TM) technique based on symplectic integration
schemes, and apply them to the Fermi-Pasta-Ulam $\beta$ (FPU-$\beta$)
lattice of $N$ nonlinearly coupled oscillators, with $N$ ranging from 4
to 20. The fast and accurate reproduction of well-known behaviors of the
Generalized Alignment Index (GALI) chaos detection technique is used as
an indicator for the efficiency of the tested integration schemes.
Implementing the TM technique--which shows the best performance among the
tested algorithms--and exploiting the advantages of the GALI method, we
successfully trace the location of low-dimensional tori.
''

\item[2011-07-21 Kazz] \arXiv{1107.2567},
\emph{Hyperbolic decoupling of tangent space
      and effective dimension of dissipative systems},
by Takeuchi, Yang, Ginelli, Radons, and Chat\'e\rf{TaGiCh11}.
They say:

``
We show, using {\cLvs}, that the tangent space of
spatially-extended dissipative systems is split into two hyperbolically
decoupled subspaces: one comprising a finite number of frequently
entangled {\entangled} modes, which carry the physically relevant
information of the trajectory, and a residual set of strongly decaying
`{\transient}' modes. The decoupling of the {\entangled} and {\transient} subspaces
is defined by the absence of tangencies between them and found to take
place generally; we find evidence in partial differential equations in
one and two spatial dimensions and even in lattices of coupled maps or
oscillators. We conjecture that the {\entangled} modes may constitute a local
linear description of the inertial manifold at any point in the global
attractor.
''

\item[2011-09-30 Kazz] \arXiv{1109.6452},
\emph{Chaos in the Hamiltonian mean field model},
by  Ginelli, Takeuchi, Chat\'e, Politi and Torcini\rf{GiKaChPoAl11}.
They say:
``
 We study the dynamical properties of the canonical ordered phase of the
Hamiltonian mean-field (HMF) model, in which $N$ particles, globally-coupled
via pairwise attractive interactions, form a rotating cluster. Using a
combination of numerical and analytical arguments, we first show that the
largest Lyapunov exponent remains strictly positive in the infinite-size limit,
converging to its asymptotic value with $1/\ln N$ corrections. We then
elucidate the scaling laws ruling the behavior of this asymptotic value in the
critical region separating the ordered, clustered phase and the disordered
phase present at high energy densities. We also show that the full spectrum of
Lyapunov exponents consists of a bulk component converging to the (zero) value
taken by a test oscillator forced by the mean field, plus subextensive bands of
${\cal O}(\ln N)$ exponents taking finite values. We finally investigate the
robustness of these results by studying a "2D" extension of the HMF model where
each particle is endowed with 4 degrees of freedom, thus allowing the emergence
of chaos at the level of single particle. Altogether, these results illustrate
the subtle effects of global (or long-range) coupling, and the importance of
the order in which the infinite-time and infinite-size limits are taken: for an
infinite-size HMF system represented by the Vlasov equation, no chaos is
present, while chaos exists and subsists for any finite system size.
''


\item[2011-09-10 Predrag 2 Dan Goldman: Templates in locomotion]
You might find this paper,
\emph{Dimension reduction near periodic orbits of hybrid systems}"
by Burden, Revzen and Shankar Sastry\rf{BuReSh11}, \arXiv{1109.1780},
(and the earlier ones on 'templates')
interesting. The funny thing is that we think of the dimensional
reduction and control in fluids and cardiac dynamics in a similar way,
and even use the word `template' for important periodic solutions.
They say
``
When the {\Poincare} map associated with a periodic orbit of a hybrid
dynamical system has constant-rank iterates, we demonstrate the existence of a
constant-dimensional invariant subsystem near the orbit which attracts all
nearby trajectories in finite time. This result shows that the long-term
behavior of a hybrid model with a large number of degrees-of-freedom may be
governed by a low-dimensional smooth dynamical system. The appearance of such
simplified models enables the translation of analytical tools from smooth
systems -such as Floquet theory- to the hybrid setting and provides a bridge
between the efforts of biologists and engineers studying legged locomotion.
''

``[...]
the main result of this paper: when
iterates of the \Poincare map associated with a periodic orbit
of a hybrid dynamical system have constant rank, trajectories
starting near the orbit converge in finite time to an embedded
smooth dynamical system.
''

``
[...]
the large number of DOF available to an animal are collapsed
during regular motion to a low-dimensional dynamical attractor
that may be captured by a template model embedded
within a higher-dimensional model anchored to the animal's
morphology.
[...]
hybrid
dynamical systems are comprised of differential equations
written over disparate domains together with rules for switching
between the domains. Of particular interest to us are
periodic orbits of such systems. From a modeling viewpoint,
a stable hybrid periodic orbit provides a natural abstraction
for the dynamics of steady-state legged locomotion.
[...]
In certain cases, it has
been possible to formally embed a low-dimensional abstraction
in a higher-dimensional physically-realistic model.
[...]
Under the condition that iterates of the \Poincare map associated with a periodic orbit are constant rank,
we demonstrate the existence of a constant-dimensional
invariant subsystem which attracts all nearby trajectories in
finite time.
[...]
the results of this paper
imply that hybrid dynamical systems may exhibit dimension
reduction near periodic orbits solely due to the interaction
of the switching dynamics with the smooth flow.
[...]
The hybrid systems we consider are constructed using
switching maps defined between boundaries of smooth dynamical
systems. The behavior of such systems can be
studied by alternately applying flows and maps.
''

                                                \toCB
They discuss covering a smooth $n$-dimensional manifold $M$ by a
collection of smooth coordinate charts. They use hyperplanes to define
`slice coordinates' for the embedded submanifolds. They refer to
trajectories as `integral curves'. The Jacobian matrix is a `linear map
called the pushforward'. The glue two smooth dynamical systems
(time-to-impact maps) together along their boundaries to obtain a new
smooth system.


``[...]
the class of hybrid systems considered in this paper
[...]
has not been defined before.
[...]
it encompasses many closed-loop models of legged locomotion.
[...]
This definition supports powerful geometric analysis of the dynamics near
a hybrid periodic orbit.
[...]
A \emph{periodic orbit} of a hybrid dynamical system is a hybrid
dynamical embedding  of the dynamical system onto $S_1$, the unit circle.
                                                \toCB
They alternately refer to {periodic orbit} as a \emph{periodic trajectory}.
[...]
The Floquet multipliers are the eigenvalues of the linearized
\Poincare map associated with a periodic orbit.
[...]
It may be easier to evaluate the rank of the \Poincare map in some
domains than others.
''

{\bf Predrag:} I am worried that the transients following a transition from one
`time-to-impact map' to the next explore much higher dimensional space
than their hybrid models? There is too much math blah blah in this paper...

I do not understand how this `constant rank' is realistic, or
how is it enforced in modeling - seems by fiat. There are two examples,
the first too abstract, and the second, `model for forced
vertical hopping,' does not seem to have any dimension reduction:
``
[...]
\Poincare map has full rank equal to 2 near its fixed point.
Therefore
[...] the system's dynamics collapse
to a smooth 3-dimensional subsystem after one hop.''


\HREF{http://www.eecs.berkeley.edu/~sburden/res.html}{Sam Burden} is a
recovering Matlab user, tired of the sluggish performance, the lackluster
feature set, and the vicious unending fee escalation. In Python he found
a simple, cross-platform computational environment with straightforward
capabilities to wrap fast C or Fortran code and display results with ease
in plots and GUIs. The simulation code can be downloaded from
\HREF{http://www.eecs.berkeley.edu/~sburden/cdc2011/}{here}.

\item[2011-09-10 Goldman] I was with Bob Full as a postdoc and he and Dan
Koditschek were the first in this field to push the templates for
locomotion idea. Revzen was a grad student with Full back then, and Sam
is a grad student with Sastry now, so I know these guys, and the general
idea, pretty well.

I've been tasked by the head of a grant I'm on with Sastry to try to
interpret the Burden stuff, and I had been hoping to engage the 5th floor
in such  pursuits. My feeling is that they (Burden et al)  love some kind
of math. I would be VERY interested to chat with you about this.

I've been collaborating with another robot guy, Howie Choset, recently,
and he and his student have used their Wilczek inspired geometric
mechanics to predict optimal 3-link swimming in sand, which we've tested
in experiment. Choset is a smart guy and will give a RIM seminar I think
around Oct 18. You might like to talk with him?

\item[2011-10-05 Predrag 2 Dan Goldman]
I asked Louis to apply for PURA with me - that is Friday;
and come talk to you about the intersection of our interests.
He seems smart, might be able to get a start on something.

That's our only chance - no professor is ever going to do any
actual work.

\item[2011-10-05 Goldman]
I'll be happy to chat with him. If you can forward him the Full and
Koditschek paper that I sent before, I've attached a longer review\rf{KoHoGu06} that
might be of interest. It should give him an idea.

\item[2011-10-05 Predrag]
Holmes, Full, Koditschek and Guckenheimer\rf{KoHoGu06} is
an oldie - but perhaps papers that follow it are worth a look:

\emph{Lateral stability of the spring-mass hopper suggests a two-step control strategy for running}
    Sean G. Carver, Noah J. Cowan, and John M. Guckenheimer, Chaos 19, 026106 (2009)

\emph{A model for insect locomotion in the horizontal plane: Feedforward activation of fast muscles, stability, and robustness}
    Raghavendra P. Kukillaya et al., Journal of Theoretical Biology 261, 210 (2009)

\emph{Steering by transient destabilization in piecewise-holonomic models of legged locomotion}
    J. Proctor et al., Regular and Chaotic Dynamics 13, 267 (2008)

\emph{A hexapedal jointed-leg model for insect locomotion in the horizontal plane}
    Raghavendra P. Kukillaya and Philip J. Holmes, Biological Cybernetics 97, 379 (2007)

\emph{Molecular spiders in one dimension}
    Tibor Antal, P. L. Krapivsky and Kirone Mallick,
    Journal of Statistical Mechanics Theory and Experiment 2007, P08027 (2007)

    \emph{A Twenty-First Century Guidebook for Applied Dynamical Systems}
    A. R. Champneys, J. Comput. Nonlinear Dynam. 1, 279 (2006). He says~rf{Champ06}
``The impact of the book
Nonlinear Oscillations, Dynamical Systems and Bifurcations of Vector
Fields by John Guckenheimer and Philip Holmes
(Springer-Verlag, Berlin, 1983): If
one were to write a similar book for the 21st century, which topics
should be contained and what form should the book take in order to have a
similar impact on the modern generation of young researchers in applied
dynamical systems? ''


    Harry Dankowicz and Oliver M. O' Reilly~rf{Holmes06},
J. Comput. Nonlinear Dynam. 1, 271 (2006) have 167 Holmes related
references.

    Double impact periodic orbits for an inverted pendulum
    Jun Shen et al., International Journal of Non-Linear Mechanics 46, 1177 (2011)

    Neuromechanical models for insect locomotion: Stability, maneuverability, and proprioceptive feedback
    R. Kukillaya et al., Chaos 19, 026107 (2009)

    Improving horizontal plane locomotion via leg angle control
    A. Wickramasuriya et al., Journal of Theoretical Biology 256, 414 (2009)

\emph{On Synchronization and Traveling Waves in Chains of Relaxation
Oscillators with an Application to Lamprey CPG}
    Peter L. Varkonyi et al., SIAM J. Appl. Dyn. Syst. 7, 766 (2008)

    R. Ronsse et al., IEEE Transactions on Robotics 23, 790 (2007)

\item[2011-10-07 Hong-liu Yang] sent us a PRL submission
Hong-liu Yang and G\"unter Radons\rf{YaRa11},
\emph{Local geometry of inertial manifold probed via Lyapunov projection of recurrent
states},
(click \HREF{http://ChaosBook.org/library/YaRa11.pdf}{here}),
and Yang's Oldenburg DDays2011 talk
(click \HREF{http://ChaosBook.org/library/YangDDs11.pdf}{here}). He says:
``
[...] our paper about the relation between
the phase space structure of PDEs and the hyperbolicity of their tangent
space dynamics. It is along the line of our previous PRL on that topic.
Instead of the local linear stability (Lyapunov) analysis used
previously, we focus here on the finite length difference vectors between
a reference state and the recurrent states. It fits quite well to your
picture of ``turbulence'' figured out at Maribor.
''

\item[2011-10-21 Predrag] some papers we should probably have a look at
again: Check out Paz\'o \refrefs{szendro-2007,PaSzLoRo09} (and references
therein) on uses of Lyapunov vectors in spatiotemporal chaos.

``Lyapunov modes'' in Hamiltonian setting are in discussed in
\emph{Lyapunov Mode Dynamics in Hard-Disk Systems}, by D. J. Robinson, G.
P. Morriss\rf{RoMo08}. It is all about Hamiltonian dynamics. It has nice
formulas for tangent space evolution in $N$-hard balls systems, and
physical interpretation of `hydro-dynamic' modes.

\item[2011-10-21 Predrag] Remember to mention that UDV, or the unstable
dimension variability\rf{kostelich97} is an indicator of
non-hyperbolicity of the chaotic attractor. The classical papers are, for
example, R. Abraham and S. Smale\rf{AbSm70} and Dawson, Grebogi, Sauer,
and Yorke\rf{DaGrSaYo94}.

\item[Fast numerical test of hyperbolic chaos]
by Pavel V. Kuptsov,
\HREF{http://arxiv.org/abs/1111.4828}{arXiv:1111.4828}, says:
``
The effective numerical method is developed performing the test of the
hyperbolicity of chaotic dynamics. The method employs ideas of algorithms for
{\cLvs} but avoids their explicit computation. The outcome
is a distribution of a characteristic value which is bounded within the unit
interval and whose zero indicate the presence of tangency between expanding and
contracting subspaces. To perform the test one needs to solve several copies of
equations for infinitesimal perturbations whose amount is equal to the sum of
numbers of positive and zero Lyapunov exponents. Since for high-dimensional
system this amount is normally much less then the full phase space dimension,
this method provide the fast and memory saving way for numerical hyperbolicity
test of such systems.''

\item[2012-04-18 Predrag]
Froyland \etal\rf{FrHuMoWa12},
\emph{Computing {\cLvs}, {Lyapunov} vectors, {Oseledets}
        vectors, and dichotomy projectors: {A} comparative numerical study},
\arXiv{1204.0871} says: ``
Covariant vectors, Lyapunov vectors, or Oseledets vectors are
increasingly being used for a variety of model analyses in areas such as
partial differential equations, nonautonomous differentiable dynamical
systems, and random dynamical systems. These vectors identify spatially
varying directions of specific asymptotic growth rates and obey
equivariance principles. In recent years new computational methods for
approximating Oseledets vectors have been developed, motivated by
increasing model complexity and greater demands for accuracy. In this
numerical study we introduce two new approaches based on singular value
decomposition and exponential dichotomies and comparatively review and
improve two recent popular approaches of Ginelli et al. (2007) and Wolfe
and Samelson (2007). We compare the performance of the four approaches
via three case studies with very different dynamics in terms of symmetry,
spectral separation, and dimension. We also investigate which methods
perform well with limited data.
''

\item[2013-02-15 Predrag to Kazumasa] Is Froyland
\etal\rf{FrHuMoWa12}, \emph{Computing {\cLvs}, {Lyapunov}
vectors, {Oseledets} vectors, and dichotomy projectors: {A}
comparative numerical study}, \arXiv{1204.0871}, a paper one should
study? They say that they ``review and improve two recent popular
approaches of Ginelli et al. (2007)\rf{ginelli-2007-99} and Wolfe and
Samelson (2007)\rf{WoSa07}.''

\item[2013-02-21 Kazumasa]
Concerning Froyland \etal\ paper\rf{FrHuMoWa12}, while it may
complement some aspects that are not discussed in other reviews, I
don't think it has priority over them. In particular, their
``improvement" on Ginelli \etal's method is quite minor and in my view
can be even harmful, losing speed (and simplicity of the code) for
somewhat decreasing a risk of numerical instability; however, this
instability can be overcome easily in Ginelli \etal's original
algorithm, by changing the time step and the period between
consecutive Gram-Schmidt orthogonalizations. Anyhow, strangely
enough, they do not compare their improved algorithms with the
original ones when evaluating the performance of the different
schemes. They seem to like so much the singular value decomposition
which is a numerically heavy task to do, and do not discuss practical
issues like memory storage and numerical complexity.
For reviews, I recommend Ginelli \etal, \arXiv{1212.3961},
and Kuptsov and Parlitz\rf{KuPa12},
 \HREF{http://dx.doi.org/10.1007/s00332-012-9126-5}{J. Nonlinear Sci. (2012)}.

\item[2011-05-27, 2013-02-22 Predrag] in
\emph{Theory and computation of covariant Lyapunov vectors},
\arXiv{1105.5228}, Kuptsov and Parlitz\rf{KuPa12} write: `` Covariant
(or characteristic) Lyapunov vectors indicate growth rates of
perturbations applied to a trajectory of a dynamical system in
different state space directions. They became practically computable
only recently due to algorithms of Ginelli et al. [Phys. Rev. Lett.
99, 2007, 130601] and by Wolfe and Samelson [Tellus 59A, 2007, 355].
We summarize the available information related to Lyapunov vectors
and provide a detailed explanation of both the theoretical basics and
numerical algorithms. We introduce the notion of adjoint covariant
Lyapunov vectors. The angles between these vectors and the original
{\cLvs} are norm-independent and can be considered as
characteristic numbers. Moreover, we present and study in detail an
improved approach for computing {\cLvs}s. Also we
describe how one can test for hyperbolicity of chaotic dynamics
without explicitly computing {\cLvs}. ''

\item[2012-08-11 Predrag]
Takeuchi and Chat\'{e}\rf{TaCh12},
\emph{Collective {Lyapunov} modes},
\arXiv{1207.5571} say: ``
We show, using covariant Lyapunov vectors in addition to standard
Lyapunov analysis, that there exists a set of collective Lyapunov modes
in large chaotic systems exhibiting collective dynamics. Associated with
delocalized Lyapunov vectors, they act collectively on the trajectory and
hence characterize the instability of its collective dynamics. We further
develop, for globally-coupled systems, a connection between these
collective modes and the Lyapunov modes in the corresponding
Perron-Frobenius equation. We thereby address the fundamental question of
the effective dimension of collective dynamics and discuss the
extensivity of chaos in presence of collective dynamics.
,,

\item[2012-09-15 Predrag]
Hugues\rf{ChaMann91,CLMM96} gets the terrible name of 1991 prize for
introducing NTCB = `non-trivial collective behavior'. Twenty one years later
we are stuck with it...

                                                    \inCB
{\CLvs} (CLVs) are the vectors spanning the subspaces
of the Oseledec decomposition of tangent space. Each Lyapunov exponent
\eigExp[j] has its associated normalized {\cLv} \jEigvec[j] at any point on
the trajectory $\ssp(t)$, which provides the intrinsic direction of
perturbation growing at the rate \eigExp[j] (Predrag note: this has to be
fine tuned for complex eigenvector pairs).

                                                    \toCB
The {\FP} equation describes the evolution of the instantaneous
distribution function of the dynamical variables.

They also study tangent space dynamics of their nonlinear {\FP}
equations. They find themselves compelled to use the finite support,
twice differentiable the Kumaraswamy noise distribution (instead of the
unbounded Gaussian noise).

They give a definition of the collective and microscopic modes, or the
delocalized and localized modes by the mean value of the inverse
participation ratio (IPR), \ie, the type of fluctuation of $\sum_j^d
|\jEigvec[j]|^4$ averaged along an chaotic trajectory, as a function of the
system size, and which also serves as a criterion for distinguishing
them. They also showed, for globally-coupled maps with additive noise,
that the collective modes in the standard Lyapunov analysis also arise as
Lyapunov modes for the corresponding Perron-Frobenius dynamics, further
justifying their definition of the collective modes.

Having read Takeuchi and Chat\'{e}\rf{TaCh12},
\emph{Collective {Lyapunov} modes} paper,
I am impressed by the amount of work and analysis that went into their
simulations. The paper is an important contribution to the study of
non-trivial collective behavior in coupled lattices, initiated in
1988-1991 by Chate and Manneville.
I still have not developed appreciation for the importance of
collective Lyapunov modes, but hopefully that day will come too...

\item[2013-03-21 Predrag]
Reading old literature on Lyapunovs for ChaosBook.org LyapunovCB.tex
chapter. Here is a paper by Shimada\rf{Shimada79} on {\em Gibbsian
distribution on the {Lorenz} attractor} of kind that worries me: ``A
representation of the Lorentz attractor by a 1-dimensional Ising
system is constructed. Gibbsian distribution function which describes
the irregular motion of this dissipative dynamical system is
investigated, as well as the relation between measure-theoretical
entropy and positive Lyapunov characteristic exponent. The following
conclusions are obtained: (1) The statistical properties of the
Lorenz system can be reduced to those of 1-dimensional Ising system
with short-range interaction, in other words, the time correlation
function of the Lorenz system shows no singular long-time behaviour.
(2) The positive Lyapunov characteristic exponent of the Lorenz
system is almost equal to its measure-theoretical entropy.'' I doubt
this is right...

\item[2013-04-08 Kazumasa]
Concerning the importance of the collective Lyapunov modes,
 what we usually argue are, for example, that (i) their existence indicates
 that the extensivity of chaos does not hold as it is
 in the presence of collective behavior, and that (ii) since
 collective modes characterize macroscopic instability, one can know
 the effective dimension in collective dynamics, use collective modes
 for controlling collective dynamics, etc.
In addition, my interest is that collective modes may have
 very rich bifurcation structure.
Not only may their exponents change the sign as usual,
 but they can also be created and annihilated,
 because the number of collective modes is not determined \textit{a priori}.
I'm wondering if one can consider a kind of normal forms
 for bifurcation of these collective modes
 and its relation to phase transitions in such systems,
 but the current numerical power does not allow us to study them right now...

\item[2013-04-08 Kazumasa]
About Shimada's paper, first I'd like to note that,
 since he coarse-grained the Lorentz attractor into two states
 (two wings of the butterfly correspond to up and down spins),
 his description cannot capture chaotic dynamics in finer scales.
Second, the effective spin-spin interaction he found
 is not like that of the Ising model,
 but rather goes beyond the nearest neighbors
 (with interaction strength decaying exponentially in distance).
Given this, his conclusion that the time correlation decays exponentially
 seems to me a sound result in such long time/length scales.
On the other hand, since he was not able to give an explicit expression
 for the spin-spin interaction, I think his argument does not strictly rule out
 the possibility of a singularity...

\item[2013-04-29 Predrag]                       \toCB
Reading old literature on Lyapunovs for ChaosBook.org LyapunovCB.tex
chapter. Here is a paper by Voth, Haller and Gollub\rf{voth02} which says
``To define stretching, consider an infinitesimal circular fluid element.
After time $\Delta \zeit$, the flow has stretched it into an ellipse, and
the amount of stretching is defined as the major diameter divided by its
initial diameter. To measure the stretching, we first determine the flow
map, a function that specifies the destination vector at time $\zeit+
\Delta \zeit$ of any fluid particle starting from $x$ at time $\zeit_0$.
(For $\Delta \zeit$ equal to one period,  the flow map becomes the
\Poincare map of the flow.) The stretching experienced by a fluid
element is determined by the gradients of the flow map. In particular,
the stretching is the square root of the largest eigenvalue of the right
Cauchy-Green strain tensor. The largest finite-time Lyapunov exponent is
given by the logarithm of the stretching after division by $\Delta
\zeit$.'' The paper then shows very beautiful snapshots of Lagrangian
mixing.

I wonder, however, whether the finite-time Lyapunov exponents are what
describes the mixing. I would think that streamlines along the unstable
manifolds would be the right thing...

\item[2013-06-29 Predrag] Because of the special issue of \emph{J. Physics A},
there is a deluge of new Lyapunov articles to have a look at:

Overview of the issue is given by
M. Cencini and F. Ginelli\rf{CenGin13},
{\em Lyapunov analysis: from dynamical systems theory to applications}.

L.-S. Young writes in {\em Mathematical theory of {Lyapunov}
exponents}\rf{Young13} review: ``
[...] discuss relations between Lyapunov exponents
and several dynamical quantities of interest, including entropy, fractal
dimension and rates of escape. The second half of this review focuses on
observable chaos , characterized by positive Lyapunov exponents  [...].
Paradoxical as it may seem, given a concrete system, it is generally
impossible to determine with mathematical certainty if it has observable
chaos  [...] noisy or stochastically perturbed
systems, for which we present a dynamical picture simpler than that for
purely deterministic systems. ''

M. Cencini and A. Vulpiani\rf{CenVul13}, {\em Finite size {Lyapunov}
exponent: review on applications}: ``[...] finite amplitude perturbations
 are ruled by nonlinear
dynamics out of tangent space, and thus cannot be captured by the
standard Lyapunov exponents. We review the application of the finite
size Lyapunov exponent (FSLE) for the characterization of
non-infinitesimal perturbations in a variety of systems. In particular,
we illustrate their usage in the context of predictability of systems
with multiple spatio-temporal scales of geophysical relevance, in the
characterization of nonlinear instabilities, and in some aspects of
transport in fluid flows. We also discuss the application of the FSLE
to more general aspects such as chaos-noise detection and
coarse-grained descriptions of signals.
''

Laffargue, Nguyen Thu Lam, Kurchan  and Tailleur\rf{LaNgKuTa13},
{\em Large deviations of {Lyapunov} exponents} write: ``
[...] A generic system also has trajectories with exceptional
  values of the exponents, corresponding to unusually stable or chaotic
  situations [...] large deviations of
  Lyapunov exponents characterize phase-space topological structures such
  as separatrices, homoclinic trajectories and degenerate tori.
  Numerically sampling such large deviations using the Lyapunov Weighted
  Dynamics allows one to pinpoint, for example, stable configurations in
  celestial mechanics or collections of interacting chaotic `breathers'
  in nonlinear media. Furthermore, we show that this numerical method
  allows one to compute the topological pressure of extended dynamical
  systems, a central quantity in the thermodynamic of trajectories of
  Ruelle.
''

Palmer and L. Zanna\rf{PalZan13}, write in the review article
{\em Singular vectors, predictability and
         ensemble forecasting for weather and climate}: ``
[...] The leading singular vectors are perturbations with the
  greatest linear growth and are therefore key in assessing the system's
  predictability [...] singular vectors for the
  predictability of weather and climate and ensemble forecasting is
  discussed [...] the error growth rate in numerical models of the atmosphere is
  given. [...] Singular vectors for the ocean and coupled
  ocean-atmosphere system in order to understand the predictability of
  climate phenomena such as ENSO and meridional overturning circulation
  are reviewed [...] As stochastic parameterizations are being
  implemented, some speculations are made about the future of singular
  vectors for the predictability of weather and climate for theoretical
  applications and at the operational level.
''

Paz\'o, L\'opez and Rodr\'iguez\rf{PaLoRo13},
{\em On the angle between the first and second
          {Lyapunov} vectors in spatio-temporal chaos},
          write: ``
[...] the first Lyapunov vector (LV) is
  associated with the largest Lyapunov exponent and indicates
  the direction of maximal growth in tangent
  space. The LV corresponding to the second largest Lyapunov exponent
  generally points in a different direction, but tangencies between both
  vectors can in principle occur. [...] the probability
  density function of the angle spanned by the first and second
  LVs should be approximately symmetric around $\pi/4$ and
  peak at 0 and $\pi/2$ . [...] for small angles we uncover a scaling law
  for the PDF [...] we justify the
  scaling form and why it should be universal for spatio-temporal chaos in one spatial
  dimension.
''

Paz\'o\etal\rf{PaLoPo13},
{\em Universal scaling of {Lyapunov}-exponent fluctuations in space-time chaos},
write: ``
Finite-time Lyapunov exponents of generic chaotic dynamical systems fluctuate in time. These fluctuations are due to the different degree of stability across the accessible phase-space. A recent numerical study of spatially-extended systems has revealed that the diffusion coefficient D of the Lyapunov exponents (LEs) exhibits a non-trivial scaling behavior, $D(L) ~ L^{-\gamma}$, with the system size L. Here, we show that the wandering exponent $\gamma$ can be expressed in terms of the roughening exponents associated with the corresponding "Lyapunov-surface".
''


Kuptsov\rf{Kuptsov13}, {\em Violation of hyperbolicity via unstable
dimension variability in a chain with local hyperbolic chaotic
attractors}, writes: `` consider a chain of oscillators with hyperbolic
chaos coupled via diffusion. When the coupling is strong, the chain is
synchronized and demonstrates hyperbolic chaos so that there is one
positive Lyapunov exponent. With the decay of the coupling, the second
and the third Lyapunov exponents approach zero simultaneously. The second
one becomes positive, while the third one remains close to zero. Its
finite-time numerical approximation fluctuates changing the sign within a
wide range of the coupling parameter. These fluctuations arise due to the
unstable dimension variability which is known to be the source for
non-hyperbolicity. ''
	
\item[2013-07-03 Kazz]
In addition to what Predrag suggested, I would like to recommend
 the Special Issue paper by Kantz \textit{et al.}\rf{KaRaYa13}
 on the method to measure Lyapunov exponents from time series analysis,
 which is a follow-up paper to their PRL \rf{YaRaKa12}.
This is another interesting and important application of the {\cLvs},
 I think.
When one wants to measure Lyapunov exponents in an experiment, one usually uses
 embedding constructed with Takens' time delay coordinates.
Because the embedding space has usually larger dimension
 than the original phase space (at least twice the attractor dimension),
 one then finds, in addition to the ``true'' exponents of the original system,
 `{\transient} exponents' corresponding
 to the extra dimension of the embedding space.
There had been no practical method to distinguish true from {\transient} exponents
 before, until Kantz \textit{et al.} found that the angle between
 a {\cLv} and attractor tangent space provides a clear criterion
 to do this (though I think there remains some ambiguity in how to construct
 this attractor tangent space).
Apart from the clear results presented in the paper,
 I think it is a good review on this problematic (except that
 they underestimate the work by Sano and Sawada,
 which was actually submitted before the famous review by Eckmann and Ruelle).


\item[2013-07-22 Yohann Duguet] duguet@limsi.fr writes: ``
I got involved with my student into computing the leading Lyapunov exponent for what we believe are strange nonchaotic attractors. We encountered so many difficulties on the way that my understanding of the concept of Lyapunov exponents is constantly evolving. Anyway I will send you my notes this week, I just updated them. My main point (related to my own experience) is related to the leading exponent, for which there are already enough open questions, especially for flows (how do you choose the time horizon non-ambiguously? maps are way easier) and even more in the case where you have strong non-normality. I discussed with Pikovsky at this very good Russian conference. His point about the Lyapunov exponents is about `scalability'. He claims that this is one of the few concepts in nonlinear dynamics for which you can extend the definition to any arbitrary level of complexity (space dimension, chaos or not etc...).

I read less about {\cLvs} these last months, but I received (as you probably) \HREF{http://iopscience.iop.org/1751-8121/46/25} {this link} to \emph{J. Phys A} with a special issue on the subject. It looks like a gold mine but I admit I did not find time to get into it.
''

\item[2014-02-22 Predrag] in
\emph{Predictable nonwandering localization of covariant Lyapunov vectors for
  scale-free networks of chaotic maps},
\arXiv{1402.4971}, Pavel V. Kuptsov and Anna V. Kuptsova write: ``
For scale-free networks of Henon maps we show that the first covariant
Lyapunov vectors demonstrate high nonwandering localization. The nodes of
localization are not synchronized with others, and the distributions of square
deviations of dynamical variables from their neighborhood have identical power
law shapes for all of such nodes. The revealed features of the localization
nodes allow to find them without computing of the vectors.''

\item[2014-11-13 Predrag]
Takayuki Yamaguchi and Makoto Iima,
{\em Numerical analysis of transient orbits by the pullback method for
covariant Lyapunov vector}, \arXiv{1411.2794}, write:
``In order to analyze structure of tangent spaces of a transient orbit,
we propose a new algorithm which pulls back vectors in tangent spaces
along the orbit by using a calculation method of covariant Lyapunov
vectors. As an example, the calculation algorithm has been applied to a
transient orbit converging to an equilibrium in a three-dimensional
ordinary differential equations. We obtain vectors in tangent spaces that
converge to eigenvectors of the linearized system at the equilibrium.
Further, we demonstrate that an appropriate perturbation calculated by
the vectors can lead an orbit going in the direction of an eigenvector of
the linearized system at the equilibrium.''

\item[2015-02-25 Predrag] Study
\emph{Oseledets' Splitting of Standard-like Maps} by
Matteo Sala and Roberto Artuso\rf{SalArt15}, \arXiv{1203.4187}. They write:
``
    For the class of differentiable maps of the plane and, in
    particular, for standard-like maps (McMillan form), a simple
    relation is shown between the directions of the local invariant
    manifolds of a generic point and its contribution to the
    finite-time Lyapunov exponents (FTLE) of the associated orbit. By
    computing also the point-wise curvature of the manifolds, we
    produce a comparative study between local Lyapunov exponent,
    manifold's curvature and splitting angle between stable/unstable
    manifolds. Interestingly, the analysis of the Chirikov-Taylor
    standard map suggests that the positive contributions to the FTLE
    average mostly come from points of the orbit where the structure of
    the manifolds is locally hyperbolic: where the manifolds are flat
    and transversal, the one-step exponent is predominantly positive
    and large; this behaviour is intended in a purely statistical
    sense, since it exhibits large deviations. Such phenomenon can be
    understood by analytic arguments which, as a by-product, also
    suggest an explicit way to point-wise approximate the splitting.
''

\item[2015-10-02 Predrag]
\emph{Statistical and Dynamical Properties of Covariant Lyapunov Vectors in a
  Coupled Atmosphere-Ocean Model - Multiscale Effects, Geometric Degeneracy,
  and Error Dynamics}
by Vannitsem and Lucarini,
\arXiv{1510.00298}: ``
We study a simplified coupled atmosphere-ocean model using the formalism of
covariant Lyapunov vectors (CLVs), which link physically-based directions of
perturbations to growth/decay rates. The model is obtained via a severe
truncation of quasi-geostrophic equations for the two fluids, and includes a
simple yet physically meaningful representation of their
dynamical/thermodynamical coupling. The model has 36 degrees of freedom, and
the parameters are chosen so that a chaotic behaviour is observed. One finds
two positive Lyapunov exponents (LEs), sixteen negative LEs, and eighteen
near-zero LEs. The presence of many near-zero LEs results from the vast
time-scale separation between the characteristic time scales of the two fluids,
and leads to nontrivial error growth properties in the tangent space spanned by
the corresponding CLVs, which are geometrically very degenerate. Such CLVs
correspond to two different classes of ocean/atmosphere coupled modes. The
tangent space spanned by the CLVs corresponding to the positive and negative
LEs has, instead, a non-pathological behaviour, and one can construct robust
large deviations laws for the finite time LEs, thus providing a universal model
for assessing predictability on long to ultra-long scales along such
directions. It is somewhat surprising to find that the tangent space of the
unstable manifold has strong projection on both atmospheric and oceanic
components. Our results underline the difficulties in using hyperbolicity as a
conceptual framework for multiscale chaotic dynamical systems, whereas the
framework of partial hyperbolicity seems better suited, possibly indicating an
alternative definition for the chaotic hypothesis. Our results suggest the need
for accurate analysis of error dynamics on different time scales and domains
and for a careful set-up of assimilation schemes when looking at coupled
atmosphere-ocean models.
''

\item[2016-05-07 Predrag]
Gritsun and Lucarini\rf{LucGri16}, \arXiv{1604.04386},
{\em Fluctuations, response, and resonances in a simple atmospheric model},
looks worth a detailed read.

\item[2009-10-27 Predrag]
Harold Posch -always totally reliable- says that his group
has switched from their older Lyapunov spectra codes to
Ginelli \etal\ algorithm, and it works much better.

\item[2013-04-06 Predrag]                    \inCB
{Hoovers}\rf{HooHoo12} write: ``Our treatment is intended to be
pedagogical.'' And indeed, their Sect.~2~\emph{Lyapunov spectrum
using Lagrange multipliers} is something we should study for
Lyapunov.tex chapter of ChaosBook (but is no help in understanding
`modes').

\item[2013-04-07 Predrag] Posch\etal\rf{HooPoFoDeZh02} write:
``some of
the Lyapunov exponents, those describing relatively-weak instabilities with
near-zero growth rates, correspond to wavelike eigenvectors, both longitudinal
and transverse. The phase relations in these eigenvectors differ from
those of acoustic waves. The coordinate and momentum displacements in
the Lyapunov `modes' are `in phase' reflecting the exponential time
dependence $\exp(\zeit\lambda)$ of Lyapunov instability.''

The language sounds Hamiltonian, but I know that thermostats that
they work with are not. Also, seems that the evidence for these modes
is unclear: ``We have to conclude that the modes have no special
hydrodynamic significance, since their very existence seems to hinge
on the detailed nature of the interparticle forces and, perhaps, on
system size.'' Which disagrees with McNamara and Mareschal\rf{McNMar01},
{\em Origin of the hydrodynamic {Lyapunov modes}}.

Maybe that is the article to read; but they credit
\refref{DePoHoo96}, and write: ``we present an explanation for the
`hydrodynamic' Lyapunov modes first observed by Posch and
Hirschl\rf{PoHi00} [PC: an article in a book, have not found it
alone]. We consider the components of the Lyapunov vector associated
with each particle to be quantities carried by that particle. We then
look at how these quantities change during a collision, and we find
that certain combinations of these quantities are conserved. The
hydrodynamic Lyapunov modes arise from these new conservation laws in
the same way the hydrodynamic modes arise from the conservation of
mass, momentum, and energy. [...] We are able to explain a number of
properties of the exponents: the existence of two `shear'  modes and
four `sound' modes for each wave number, and an estimate of the
Lyapunov exponent associated with the shearing mode. [...]
combination of success and failure leads us to the conclusion that
the link between the conserved quantities and the hydrodynamic
Lyapunov modes is correct, but it must be elaborated in a way
different than the kinetic theory approach presented here.''

It all seems very Hamiltonian and Boltzmann to me, not sure why we
should ponder this when working with \KS....

Wm. G. Hoover and Carol G. Hoover, \arXiv{1106.2367},
``Local Gram-Schmidt and Covariant Lyapunov Vectors and Exponents for
 Three Harmonic Oscillator Problems.'' They say

``
We compare the Gram-Schmidt and covariant phase-space-basis-vector
descriptions for three time-reversible harmonic oscillator problems, in two,
three, and four phase-space dimensions respectively. The two-dimensional
problem can be solved analytically. The three-dimensional and four-dimensional
problems studied here are simultaneously chaotic, time-reversible, and
dissipative. Our treatment is intended to be pedagogical, for use in an updated
version of our book on Time Reversibility, Computer Simulation, and Chaos.
,,

\item[2013-06-29 Predrag]
Posch\rf{Posch13},
{\em Symmetry properties of orthogonal and
   covariant {Lyapunov} vectors and their exponents}, writes:
   ``Lyapunov exponents [...] are most naturally defined in terms of
the time evolution of a set of so-called {\cLvs}, co-moving
with the linearized flow in tangent space. Taking a simple spring
pendulum and the H\'enon-Heiles system as examples, we demonstrate the
consequences of symplectic symmetry and of time-reversal invariance for
such vectors, and study the transformation between different
parameterizations of the flow.''

\item[2015-10-19 Predrag]
Hadrien Bosetti and Harald A. Posch\rf{BosPos14} write in
{\em What does dynamical systems theory teach us about fluids?}:
``We use molecular dynamics simulations to compute the Lyapunov spectra
of many-particle systems resembling simple fluids in thermal equilibrium
and in non-equilibrium stationary states. Here we review some of the most
interesting results and point to open questions.''

The truth is, there is much of interest to read by Posch,
\refref{DePoHoo96,PoHi00,HooPoFoDeZh02,Bosetti2010,BoPo10,Posch13,BosPos14}.

\item[2015-10-20 Evangelos] I should be preparing my move to Sweden but
reading Bosetti and Posch~\cite{BosPos14} is much more fun. I did not get
very far yet, but they define the local Lyapunov exponents in the same
way as Xiong Ding (see their equation 15). They also go one step further
and write it in terms of what Predrag calls (maybe not anymore?) the
matrix of variations (as suggested by Predrag in the last webex meeting).
This shows explicitly that the local Lyapunov exponent only depends on
the given point on the trajectory.

\item[2015-10-20 Predrag]                 \toCB
I meant `stability matrix', used for example by Tabor\rf{Tabor89} but
what it is is not obvious from the name; or the very explicit,
descriptive ``matrix of velocity gradients".

\HREF{https://en.wikipedia.org/wiki/Strain_rate_tensor}
{Strain rate tensor wiki} says:

\begin{quote}
In continuum mechanics, the strain rate tensor is a physical quantity
that describes the rate of change of the deformation of a material in the
neighborhood of a certain point, at a certain moment of time. It can be
defined as the derivative of the strain tensor with respect to time, or
as the symmetric component of the gradient (derivative with respect to
position) of the flow velocity, a standard notion in fluid dynamics.
\end{quote}
It is my curse that next they come up with `Jacobian matrix', meaning
stability matrix:
\begin{quote}

$\nabla v$ is the Jacobian matrix of the field,
\[
    \nabla v = J =
    \begin{bmatrix} \displaystyle{\partial_1 v_1}
                     & \displaystyle{\partial_2 v_1}
                       & \displaystyle{\partial_3 v_1}\\
                    \displaystyle{\partial_1 v_2}
                      & \displaystyle{\partial_2 v_2}
                        & \displaystyle{\partial_3 v_2}\\
                    \displaystyle{\partial_1 v_3}
                      & \displaystyle{\partial_2 v_3}
                         & \displaystyle{\partial_3 v_3}
    \end{bmatrix}.
\]
[...]  decomposed into the sum of a symmetric matrix and an antisymmetric
matrix [...]
The antisymmetric part represents a \emph{rigid-like rotation} of the
fluid about the point \ssp. Rigid rotation does not change the relative
positions of the fluid elements, so the antisymmetric part of the
velocity gradient, or the \emph{rate of rotation tensor},  does not
contribute to the rate of change of the deformation. The actual strain
rate is therefore described by the symmetric part, the \emph{strain rate
tensor}
\beq
D = \frac{1}{2}\left(\nabla \cdot v + \transp{\nabla \cdot v}\right)
\,.
\ee{StrainRateTens}

The symmetric part of velocity gradient (the rate-of-strain tensor) can
be broken down further as the sum of a scalar times the unit tensor, that
represents an isotropic expansion or contraction (changes the volume);
and a traceless symmetric tensor which represents a shearing deformation,
with no change in volume (a deviatoric component which tends to distort
the infinitesimal body element).

The \emph{expansion rate}  is $1/d$ of the divergence of the velocity field:
\[
    \nabla \cdot v = \partial_1 v_1 + \partial_2 v_2 + \partial_3 v_3;
\]
which is the rate at which the volume of a fixed amount of `fluid'
changes at that point.

Some random Googling:

The velocity gradient tensor has been used in vortex detection (swirling
strength) in turbulent flows. See
\emph{Analysis and interpretation of instantaneous turbulent velocity fields},
 Adrian, Christensen, Liu, \emph{Experiments in fluids} 29 (2000) 275-290.
 See also  Chong et al. (JFM 357), Ooi et al. (JFM 381) and other papers
 by Cantwell in J. of Fluid Mech.

The gradient of the velocity vector produces (in 3D) a 3x3 matrix of
derivatives. These matrix is not a tensor in its own right because is not
``objective" or ``frame indifferent"... In fluid mechanics, we usually
refer to the symmetric part of the velocity gradient, which is objective
and frame indifferent.

``shear strain rate" is used as the measure of gradient of velocity

\emph{Continuum Mechanics: Concise theory and problems}
by Peter Chadwick is worth looking at.
\end{quote}

The velocity gradient $\nabla v$ is a 2-index matrix. Is it a (an
invariant or `equivariant') tensor? A tensor is defined only with respect
to a symmetry group. If we assume that there is an invariant $L^2$ norm
(length) in $d$ dimensions, its Euclidean group contains the special
orthogonal group \SOn{d}, with group elements $R$. Under coordinate
rotation $\nabla v$ transforms as
\[
\nabla v' = R\,\nabla v \transp{R} + \dot{R}\transp{R}
\,,
\]
so the velocity gradient matrix is not a tensor,
\(
\nabla v' \neq R\,\nabla v \transp{R}
\,.
\)
The {strain rate tensor} \refeq{StrainRateTens} is, however, a tensor.

\item[2014-10-16 Predrag]
N. V. Kuznetsov, T. A. Alexeeva and G. A. Leonov write in \emph{Invariance
of Lyapunov characteristic exponents, Lyapunov exponents, and Lyapunov
dimension for regular and non-regular linearizations},
\arXiv{1410.2016},
(Keywords: chaos, Lyapunov exponent, Lyapunov characteristic exponent,
Lyapunov dimension of attractor, time-varying linearization, regular and
non regular linearization, time reparametrization, change of coordinates,
Perron effect of Lyapunov exponent sign reversal):
``
The question of invariance of Lyapunov exponents for regular and
non-regular linearizations under the change of coordinates and time
reparametrization is considered. The relation between Lyapunov exponents and
Lyapunov characteristic exponents is discussed. Definition of Lyapunov
dimension is generalized for non-regular linearization. The invariance of
Lyapunov dimension under time rescaling, which preserves direction, and
under diffeomorphism of the phase space is demonstrated.
''

\item[2015-08-29 Predrag]
Leonov, Kuznetsov, Korzhemanova and Kusakin write in {\em The Lyapunov
dimension formula for the global attractor of the Lorenz system},
\arXiv{1508.07498} (text overlap with \arXiv{1505.04729}):
``
  The exact Lyapunov dimension formula for the Lorenz system has been
analytically obtained by G.A.Leonov in 2002 under certain
restrictions on parameters, permitting classical values. He used the
construction technique of special Lyapunov-type functions developed by
him in 1991 year. Later it was shown that the consideration of larger
class of Lyapunov-type functions permits proving the validity of this
formula for all parameters of the system such that all the equilibria
of the system are hyperbolically unstable. In the present work it is
proved the validity of the formula for Lyapunov dimension for a wider
variety of parameters values, which include all parameters satisfying
the classical physical limitations. One of the motivation of this work
is the possibility of computing a chaotic attractor in the Lorenz
system in the case of one unstable and two stable equilibria.
''

\item[2015-09-25 Predrag]
Should read Kuznetsov, Mokaev and ,Vasilyev\rf{KuMoVa14}.
Leonov, Kuznetsov and Mokaev, \emph{The Lyapunov dimension formula of
self-excited and hidden attractors in the Glukhovsky-Dolzhansky system},
\arXiv{1509.09161}, write: ``
  In the past two decades Lyapunov functions are used for the estimation of
attractor dimensions. By means of these functions the upper estimate of
Lyapunov dimension for R\"{o}ssler attractor and the exact formulas of Lyapunov
dimension for H\'{e}non, Chirikov, and Lorenz attractors are obtained.
  In this report the simplest model, suggested by Glukhovsky and Dolzhansky,
which describes a convection process in rotating fluid, is considered. A system
of differential equations for this model is a generalization of Lorenz system.
For the Lyapunov dimension of attractor of the model, the upper estimate is
obtained.''

\item[2015-10-14 Farizmand]
Leonov and Kuznetsov\rf{LeoKuz15} {\em A short survey on {Lyapunov}
dimension for finite dimensional dynamical systems in {Euclidean} space},
\\
\arXiv{1510.03835} may be of  interest. They refer to at least 25 of their own
papers on the subject...

\item[2016-04-13 Predrag]
Pavel V. Kuptsov and Sergey P. Kuznetsov,
\arXiv{1604.03521},
{\em Numerical test for hyperbolicity of chaotic dynamics in time-delay
  systems}
write: ``
  We develop the numerical test of hyperbolicity of chaotic dynamics in
time-delay systems. The test is based on the angle criterion and includes the
computation of angle distribution between expanding, contracting and neutral
manifolds of trajectories on the attractor. Three examples are tested. For two
of them previously predicted hyperbolicity is confirmed. The third one provides
an example of time-delay system with non-hyperbolic chaos.''

\item[2016-07-27 Predrag] Yet another article about computing Lyapunov
exponents: following up on Aston and Dellnitz\rf{AstDel99} {\em The
computation of {Lyapunov} exponents via spatial integration with
application to blowout bifurcations}, Beyn and Lust\rf{BeyLus09,BeyLus13}
{\em A hybrid method for computing Lyapunov exponents}, compute all ``all
Lyapunov coefficients (?)''.

\item[2016-08-11]
Should (re)read Gozzi and Reuter\rf{GozReu94} {\em Lyapunov exponents,
path-integrals and forms}. There must be a local measure (perhaps a
2-form?) of the physical dimension, rather than our cockeyed
norm-dependent close passages\rf{DCTSCD14}.

\item[2016-08-13 Xiong]
Paper \rf{GozReu94} is hard for me. It seems about stochastic Hamiltonian
system.


\item[2016-08-07 Predrag]                                       \toCB
Two papers to study:
Franzosi, Poggi and Cerruti-Sola\rf{franzosi-2004}
{\em {Lyapunov} exponents from unstable periodic orbits in the {FPU}-beta model},

  and
Franzosi, Poggi and Cerruti-Sola\rf{FrPoCe05}
{\em Lyapunov exponents from unstable periodic orbits},

also, \HREF{https://scholar.google.com/citations?hl=en&user=pHUTWi8AAAAJ}
{Franzosi} has  several Lyapunov exponent papers of possible interest

{\em Analytic {Lyapunov} exponents in a classical nonlinear field equation}

{\em Lyapunov exponents from geodesic spread in configuration space}

{\em Riemannian geometry of Hamiltonian chaos: Hints for a general theory}



\end{description}

\renewcommand{\ssp}{a}
