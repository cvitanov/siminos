% siminos/lyapunov/DCTSCD14.tex
% $Author: predrag $ $Date: 2016-06-05 19:53:29 -0400 (Sun, 05 Jun 2016) $

\section{Edits, upodim DCTSCD14}

\begin{description}

\item[2016-02-05 PC] to Xiong - Is Eckmann \&
    Ruelle\rf{EckmannRuelle1985} really the place to be credited for
    ``Covariant Lyapunov vectors span the Oseledec subspaces"? Not
    Oseledec\rf{lyaos}?

\item[2016-02-15 XD] \refRef{EckmannRuelle1985} is a review article.
    Its chapter 3A discusses multiplicative ergodic theorem of
     Oseledec, with \refref{lyaos} the original article for this
     topic. We also have other excellent proof paper like
     \refref{ruelle79}. Which one is better ?

\item[2016-02-05 PC] to Xiong - I thought that people who developed the
new codes for computing \cLvs\rf{YaTaGiChRa08,TaGiCh11} did not use
\refRef{ruelle79,EckmannRuelle1985}. But it is OK with me to use Eckmann
\& Ruelle\rf{EckmannRuelle1985}.

\item[2016-02-04 ES] I changed: ``but all are unable to describe it
    constructively'' to ``a constructive description of inertial
    manifolds remains a challenge.'' Someone might complaint that this
    is not true and that papers like \refref{jolly1990} achieve that.

\item[2016-02-22 PC] Implemented what Ruslan wrote ``I don't think I've
made, or will be able to make, a sufficiently meaningful contribution to
this paper in order to justify my co-authorship.  So, I would be more
comfortable with a line in 'Acknowledgements', rather than as an
author.''

\item[2014-05-01 XD]
Should I just skip the discussion of degeneracy? I am not 100
percent sure about this case.

\item[2016-03-03 KT] What do you mean, Xiong, by the discussion of degeneracy?

Assuming Xiong's possible concern on the degeneracy is not troublesome, I
believe all issues are now resolved. I suggest we make a final check of
the manuscript.

%\item[2016-02-15 PC] Do we really need these 3 footnotes, rather than
%having that text as (...) in the text proper? Footnotes do not work well
%in Phys Rev format - they do not show up at the bottom of the page (see
%the manual: ``Avoid custom footnotes using \texttt{footnotemark} and
%\texttt{footnotetext}'' and `` However, be advised that, if your article
%is accepted for publication, footnotes may be incorporated into text
%during the production process.'').
%Instead, they are numbered as references (something one ignores while
%reading a paper) and show up in the reference list at the end of the paper.
%Just sayin'...
%
%\item[2016-03-02 KT]
%Although I understand Predrag's antipathy against PhysRev footnotes, here
%I do consider it's better to use a footnote. The reason is that (1) this
%note must be recalled when our estimate of the inertial manifold
%dimensionality is compared with previous results, and (2) this note only
%concerns those who try to make this comparison, in which case they
%certainly look at the bibliography. Similarly, I use footnotes for
%remarks that are only concerned with references.
%
%Concerning the footnotes,  I think they are justified to be in the
%endnote for this particular case. About the supplementary, if I remember
%correctly, APS requires citing it in the endnote.

\item[2016-03-04 PC to KT] You are right. I checked the latest issue of
Phys. Rev. Letters, and your number of footnotes is pretty typical.

\item[2016-02-22 PC] wrote:
``The Floquet exponents $\lambda_j$ (we shall only consider their real
parts) are related to multipliers by
$\lambda_j=\ln|\Lambda_j|/\period{p}$.''

\item[2016-03-02 KT] edit:
``The Floquet exponents $\lambda_j$ are related to multipliers by
$\lambda_j=\ln|\Lambda_j|/\period{p}$.''

I dropped ``(we shall only consider their real parts),'' because the
exponents are by definition real numbers, even for complex Floquet
multipliers.

\item[2016-03-04 PC to KT and Burak]
Wow, I did not know that. Can you give me the reference that explains
this fact about \emph{Floquet} (not Lyapunov) exponents, \ie, that a
logarithm of a complex number is real? I better correct that in
ChaosBook. Also Burak's paper\rf{Bud15} (\texttt{pipes/torsion/}
repository) will have to rewritten.

Kidding aside, check Eq.~(3.29) in
\HREF{http://www.emba.uvm.edu/~jxyang/teaching/Floquet_theory_Ward.pdf}
{these lecture notes}, as a random example.

% \item[2016-03-05 PC] I reverted `real parts'.

%\item[2016-03-04 Xiong ]
%1. In the abstract, ``we determine the dimension of the inertial manifold
%for a Kuramoto-Sivashinsky system''. should we delete 'a' in front of
%'Kuramoto-Sivashinsky system' ? \\
%{\bf 2016-03-04 Predrag} done

%4. At page 4, reflection $u(x,t)\to{}\equiv\sigma\,u(x,t)$. Two symbols are
%put together. I want to
%confirm my understanding that it means
%that it is a transformation also a definition.\\
%{\bf 2016-03-04 Kazz}
%Thanks Xiong for finding this important typo!

\item[2016-03-09 PC]
removed ``with $A(\ssp)\equiv\frac{\partial{}J^\tau}{\partial\tau}(\ssp)|_{\tau=0}$.'',
replaced with the standard definition of the \stabmat\ (matrix of velocity gradients)

\item[2016-03-14 PC on Einstein's birthday]
I interpret this text in \reffig{fig1} to mean that Ginelli
\etal\rf{ginelli-2007-99} code is accurate to $10^{-3}$ (by contrast,
Xiong's code is accurate to $10^{-11}$), not that that there is something
funny about what should have been a marginal mode:

\begin{quote}
    Note that the number of the vanishing exponents is unchanged
    (the 4th exponent of the chaotic trajectory is slightly negative, -0.003).
\end{quote}

so I have removed it from the caption.
\\{\bf 2016-03-28 KT}
this is NOT the reason I made this remark. The inset looks like the
chaotic trajectory has three neutral exponents instead of two. The remark
is to clarify it is not true, just the largest negative exponent is
unfortunately close to zero. The precision of the algorithm is better
than $10^{-3}$. Anyhow, I put back the deleted phrase into the caption.

\item[2016-03-14 PC]
Symmetry reduction of nonlinear chaotic flows is anything but
``trivial'', so I removed

``excluding trivial contributions from spatial translation''


\item[2016-02-22 Ruslan]
    {There are no
    random variables here: see my comment below on the use of
    'pdf($\mathrm{\phi}$)' in the axes labels in Fig. 2.}

    {I understand what you mean by the 'distribution of angle $\phi$',
    i.e. the fraction of time this function takes on a specific value in
    the infinitesimal interval $d\phi$ (right?) but I would refrain from
    using 'pdf' in the axis label, since I don't believe it is correct to
    call it 'probability distribution'.  After all, $\phi(t)$ is a
    deterministic function, not a random process.}

    {I don't get this.  If tangency cannot be reached for any specific
    orbit, then how can it be reached for a collection?  Isn't the
    distribution of the angle for the collection just a (weighted)
    average of distributions for individual orbits?}
\item[2016-03-03 Kazz]
    {See the dailyblog discussions for the answer to your questions.
    About the use of "pdf", it's true it's a bit strange. I renamed it to
    "density".}

\item[2016-02-04 Evangelos]
    I think that I understand what you mean but maybe the wording needs
    to change? Aren't tangencies along individual orbits the origin of
    the small theta part of the distributions?
    \\
    {\bf 2016-03-14 PC} is day's rewrite better?
    {\bf 2016-03-29 ES} I think it's fine now.

\item[2016-03-28 Kazz]
I made the following changes:\\
1. put back the deleted phrase to the caption of Fig.1. See my comment above to {\bf 2016-03-14 PC on Einstein's birthday}.\\
2. rephrased the last sentence of the caption of Fig.2.\\
3. removed the repeated definition of $q_k$ and $a_k$ in p.8 (they were already defined in p.4).\\
4. the first sentence of the paragraph "This strategy cannot..." (p.6) is slightly changed.\\
Otherwise I'm okay about the new manuscript.

I remind you that we also need to prepare a cover letter, as well as a 50-words summary of the significance of the work, to submit to PRL. Hugues and I had already prepared a draft for them (see \verb|coverletter.txt|).

\item[2016-03-28 Predrag]
Sorry, Xiong and I misunderstood the remark about $\lambda_4= -0.003$.

\item[2016-03-29 Evangelos]
The manuscript is fine I think, except that it might be indeed too long.
I find that the most convenient way to measure the length is to actually
start the submission process and get to the page where you can check
the length of the manuscript. Then you can remove some material and
replace the tex file until you are below 3750 words.


\item[2016-06-02 Xiong]
I made some changes to the dimension draft. Here is a summary.
\begin{itemize}

\item The text highlighted in red color, which mainly concerns the 2 points
  proposed by referee B.
  In caption of Fig.~1, I do not give the values of Floquet
  exponents explicitly but only the number of positive ones to save space.

\item  I comment out paragraph 5:
  ``In principle there are infinitely many unstable orbits...''.
  This paragraph states that the organization of \po s is
  unknown and we have enough \po s. I do not think this
  information is too important. Moreover, this paragraph
  serves as a summary of the entire paper. We
  already have 2 large paragraphs in the conclusion part.
  Also, paragraph 4 indicates
  clearly what we are about to do. So for balance, I guess
  this paragraph is not indispensable.
  \\
  {\bf 2016-06-05 Predrag} reinstated the said paragraph

\item In fig.~1(b), the y-label should get rid of superscript `loc' to
make it consistent with the text. I tried to open the source fig files,
but found that there are not the final version used in the draft.
Probably, Kazumasa forgot to update them.
  \\
  {\bf 2016-06-05 Predrag} removed loc in the label

\item Chat\'e suggests collating the 2 panels in Fig.2 to
  effectively get rid of the x-axis labels for the top panel,
  and also changing the style of the x-axis labels left
  from $\frac{1}{n}\pi$ to $\pi/n$ to save space.
  Similar modification could be done in Fig3 almost
  collating panels (c) and (e), and (d) and (f).
  I think it deserves a try. Also, I find if we remove
  paragraph 5 and the arXiv info in the reference, then
  the draft can be put in 5 pages.
  \\
  {\bf 2016-06-05 Predrag} lets consider such edits in a full-length
  paper, or if PRLett editors ask us to save space. Until then, the plots
  look very nice as they are.

\end{itemize}
Anyway, I wait for Kazumasa to change the figures or upload the
current figures.
  \\
  {\bf 2016-06-05 Predrag} That would be nice. But for the PRLett
  resubmission having removed loc in the fig.~1(b) axis label is
  sufficient.


\item [2016-06-05 Evangelos]
  \begin{itemize}
  \item I think that the reshuffles of text are fine and should satisfy the referee.

  \item In Fig. 1 I think it's not enough to state the number of positive
  exponents since the referee does not seem convinced unless he can read
  the information. It would be much better to list the values of a few
  exponents. If you implement Hugues' suggestions about the figures you
  should be able to save enough space to do that.
  \\
  {\bf 2016-06-05 Predrag} listed 3 exponents

  \item I would prefer not to delete that introductory paragraph... it
  nicely and compactly explains the important points.
  \\
  {\bf 2016-06-05 Predrag} reinstated the said paragraph

  \item You don't gain something by removing the arxiv link from the
  references, since references do not count in the word count.
    \\
  {\bf 2016-06-05 Predrag} what's done is done. Does not hurt.

  \item There is no need to stay below 5 pages. I think that if you
  don't exceed the length of the original submission manuscript you
  should be fine. Right now the length is less than that of the original
  manuscript by about 20 lines. This means that you do not need to delete
  that paragraph.
  \\
  {\bf 2016-06-05 Predrag} reinstated the said paragraph

  \end{itemize}


\item [2016-6-05 Xiong to Evangelos]
I have almost no control on the figures.


\end{description}
