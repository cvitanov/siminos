\ifsvnmulti
	\svnkwsave{$RepoFile: siminos/lyapunov/products.tex $}
	\svnidlong {$HeadURL: svn://zero.physics.gatech.edu/siminos/lyapunov/products.tex $}
	{$LastChangedDate: 2016-02-05 16:57:28 -0500 (Fri, 05 Feb 2016) $}
	{$LastChangedRevision: 4619 $} {$LastChangedBy: predrag $}
	\svnid{$Id: products.tex 4619 2016-02-05 21:57:28Z predrag $}
\fi

\section{Inner products}
\label{def:innerProduct}
% Predrag extracted from siminos/froehlich/symm.tex 2011-03-10

One final piece of notation must be introduced before we can discuss
the \mslices. The notion of an inner product on the {\statesp} is
necessary to define a linear slice of the {\statesp}. We will
therefore assume from now on that the {\statesp} has an inner product.

As we shall use here several inner products:
over group manifolds, over real and complex finite-dimensional
coordinates, and over function spaces, it is convenient to introduce a
compact notation that subsumes them all as special cases.
    \PC{We need to define these properly. When we get to fluids, it's not
    trivial - we will use `energy norm.' }

Here the dot product $\gSpace \cdot \Lg$ shall refer to the sum over
the Lie algebra generators of an $N$-dimensional Lie group \Group,
\beq
\gSpace \cdot \Lg = \sum_{a=1}^N \gSpace_a \Lg_a
%    \,,\qquad a = 1,2,\cdots,N
\,.
\ee{dotGroup}

The inner product of two \statesp\ vectors $x, y \in \pS$ will be denoted
by $\braket{x}{y}$. If the \statesp\ is $\mathbb{R}^d$, then by the inner
product we usually mean the Euclidian product of two vectors $x,y$,
\beq
\braket{x}{y} = \sum_i^d {x}_i y_i
    \,,\qquad \pS \subset \reals
\,.
\ee{innerR}
All norms will refer to the Euclidean norm for which we use the notation
$\|\cdot\|$ instead of  $\|\cdot\|_2$. All vectors are assumed to be
column vectors. The superscript notation $(~)^T$ will denote the
transpose of vectors/matrices.

If the \statesp\ is finite dimensional and complex,
\beq
\braket{x}{y} = \sum_i^d \dual{x}_i y_i
    \,,\qquad \pS \subset \complex
\,,
\ee{innerC}
where $\dual{x}$ is the complex conjugate transpose of vector $x$.
We found the discussion of norms in Stone and Goldbart\rf{StGo09}
illuminating.
Here we follow the notation of Trefethen and Bau\rf{Trefethen97}.
    \PC{add a paragraph on the length as integral along a
        geodesic on a curved surface with a metric tensor}

In an inner product, a matrix $M$ acts as
    \PC{This should probably be extended to non-selfadjoint
        actions, explain the adjoint {\jacobianM} $\dual{\jMps}$.
    Had: ",
or, more generally, the hermitian conjugate $\dual{M}$ of matrix $M$.
    " Should discus metrics here,
    insert kernel/weight into \refeq{innerL2},
    perhaps like Stone and Goldbart\rf{StGo09}
    }
\beq
\braket{x}{M\,y} =
  \braket{\dual{M}\,x}{y}
\,.
\ee{adjointG}
where $\dual{M}$ is the hermitian conjugate of $M$.

If the \statesp\ is a normed function space (Banach, Hilbert, Sobolev, ...),
the inner product is given by the integral
\beq
\braket{g}{f} = \int dx \, \dual{g}(x) f(x)
\,.
\ee{innerL2}
The associated $L^2$ norm is
$|\ssp|^2 = \braket{\ssp}{\ssp} \neq 0$, unless $\ssp = 0$.

In computations the functions are expressed in terms of
complete orthonormal basis sets $\{u_n\}$,
\bea
f(x) &=& \sum_{n=0}^{\infty} a_n u_n(x)
    \continue
\braket{u_n}{u_m} &=& \delta_{nm}
\,.
\label{basisL2}
\eea

Unitary and orthogonal groups (as well as their subgroups) are defined as
groups that preserve these `length' norms, $\braket{\LieEl x}{\LieEl x} =
\braket{x}{x}$, and infinitesimally their generators
% \refeq{eq:infinitesimal}
induce no change in the norm,
\(
\braket{ \Lg_a\ssp}{\ssp}
  +\braket{\ssp}{\Lg_a\ssp} =0
\,,
\)
hence the Lie algebra generators $\Lg$ are antisymmetric for orthogonal
groups, and antihermitian for unitary ones,
\beq
\dual{\Lg} = - \Lg
\,.
\ee{antiHerm}
This antisymmetry of generators implies that the action of the group on
vector $\ssp$ is locally normal to it,
\beq
\braket{\ssp}{\groupTan_{a}(\ssp)} =0
\,.
\ee{TtimesX}

A group tangent
% \refeq{PC:groupTan}
is a vector both in the group tangent space and in the \statesp.
We shall indicate by $\braket{\groupTan_{a}(x)}{\groupTan_{b}(y)}$
the sum over \statesp\ inner product only, and by
\beq
\braket{\groupTan(x)}{\groupTan(y)} =
    \sum_{a=1}^N \braket{\groupTan_{a}(x)}{\groupTan_{a}(y)} =
  \braket{x}{\dual{\Lg} \cdot {\Lg}\,y}
\ee{innerGdot}
the sum over both group and spatial dimensions.

Any representation of a compact group $\Group$ is fully reducible, and
for a Lie group the invariant tensors constructed by contractions of
$\Lg_a$ are useful for identifying irreducible representations. The
simplest such invariant is
\beq
\dual{\Lg} \cdot \Lg = \sum_\alpha C_2^{(\alpha)} \, \id^{(\alpha)}
\,,
\ee{QuadCasimir}
where $C_2^{(\alpha)}$ is the quadratic Casimir for irreducible
representation labeled $\alpha$, and $\id^{(\alpha)}$ is the identity on
the $\alpha$-irreducible subspace, 0 elsewhere. $ C_2^{(\alpha)} =0$ if
$\alpha$ is an invariant subspace. The dot product of two tangent fields
\refeq{innerGdot} is thus a sum of inner products weighted by Casimirs,
\beq
\braket{\groupTan(\sspRed)}{\groupTan(\slicep)}
   = \sum_\alpha C_2^{(\alpha)} \dual{\sspRed}_i\, \delta_{ij}^{(\alpha)} \slicep_j
\,.
\ee{braket}
An example is the Fourier series.
% \refeq{tangL2norm}.
For compact groups $C_2^{(\alpha)}$ are strictly nonnegative by
the antihermiticity \refeq{antiHerm} of Lie algebra generators.
