\ifsvnmulti
 \svnkwsave{$RepoFile: lyapunov/sensitive.tex $}
 \svnidlong {$HeadURL: svn://zero.physics.gatech.edu/siminos/lyapunov/sensitive.tex $}
 {$LastChangedDate: 2015-04-05 09:53:25 -0400 (Sun, 05 Apr 2015) $}
 {$LastChangedRevision: 4018 $} {$LastChangedBy: predrag $}
 \svnid{$Id: sensitive.tex 4018 2015-04-05 13:53:25Z predrag $}
\fi

\chapter{Sensitivity}
\label{c-sensitive}

\begin{description}

\item[2013-06-15 Nick Trefethen to Divakar] % <trefethen@maths.ox.ac.uk>
We had a visit from Qiqi Wang of MIT.  [...] His problem is (modulo
details) to compute the average of a quantity driven by a chaotic
system.  For example, what's the average $<z(t)>$ of the $z$
component of a Lorenz equation solution, averaged over infinite time?
[...] If anybody has a slick algorithm for this, it's probably you.
Any pointers?


\item[2013-06-17 Divakar to Nick Trefethen]
Computing averages of chaotic sets is the focus of periodic orbit
theory as developed by Predrag Cvitanovi\'c and others\rf{DasBuch}.
The formulas of periodic orbit theory are a sophisticated form of
extrapolation. For a few examples, they give an amazing level of
accuracy.

In one of my Lorenz papers\rf{DV04}, I applied periodic orbit theory
to the Lorenz system. The average quantities I looked at were the
Hausdorff dimension and the Lyapunov exponent. Periodic orbit theory
showed good convergence but not the super-exponential convergence it
gives for some other examples. In my paper the treatment of these
formulas is very brief, but Predrag's webbook \wwwcb{} has all the
details. It is not difficult to apply the formulas to computation of the
average of $z(t)$.

The reason for the somewhat slower convergence for Lorenz is the
fixed point at the origin which belongs to the closure of the Lorenz
attractor. Predrag and others have worked on accelerating convergence
in the presence of such intermittency.

\item[2013-06-23 Predrag] The short answer to Nick is the usual: I
did it first (see a draft of \emph{Periodic orbit formulation of
linear response}
\HREF{http://www.cns.gatech.edu/~predrag/papers/unfinished.html}
{here}), and it is wrong anyway.

The student got away, and no other student was interested in
completing the project, but I believe that for the Lorenz model \po\
theory would have convergence that Divakar\rf{DV04} obtains (and a
theory of convergence of those numbers), rather than Qiqi
Wang\rf{Wang13} 2-3 digits.

Qiqi Wang\rf{Wang13,BloWan13} takes the '{\cLvs}' path of
Ginelli \etal\rf{ginelli-2007-99}. While I like that work, it is
purely numerical, and we have had a project for several years of
incorporating their algorithm into \po\ theory. Again, all students I
started on this got away, so far, so cannot make strong claims...

The reason why it is ``wrong'' is that I am no fan of
Ruelle Linear Response Theory\rf{Ruelle96,Ruelle09}. As chaotic sets
are in generally \emph{not} structurally stable, the expectation
values of averages computed on them are fractal curves, in general
not smooth and not differentiable with respect to small system
parameter variations.
A very persuasive example is graph of the deterministic diffusion
constant as a function of a system parameter, see ChaosBook
Figure 25.5. (Eq. nos. refer here to
\HREF{http://chaosbook.org/version14/pdf.shtml} {version 14} of Dec
31 2012).

\item[2013-06-22 Predrag]
However, as my plumber colleagues are all enamored of adjoint
\evOper s, and I am notoriously insensitive,
in what follows I'll
read Qiqi Wang\rf{Wang13} in some detail (his blog is
\HREF{http://engineer-chaos.blogspot.com/} {here}), in hope of understanding
what this sensitivity is, and
getting infatuated myself... My problem is that adjoint is defined
only if we define a norm, and that in computing dynamical averages
usually a norm is not needed, and often not natural. (I rant about that
in the preliminary chapter on
\HREF{http://chaosbook.org/paper.shtml\#Lyapunov} {Lyapunov exponents},
Remark 6.1.)

                                        \toCB
P. Moin \etal\ seem to refer to chaotic trajectories as `unsteady'?
Or are limit cycles also `unsteady'? Also, while this literature has
a flavor of `Koopman modes', `Dynamic mode decomposition', Santa
Barbara and Stockholm groups are never cited.

The problem: Compute sensitivity of a chaotic
model with respect to parameter changes
\begin{equation}
\mbox{Given } \frac{du}{dt} = \vel(u, \xi), \quad
\timeAver{\obser} = \lim_{T\rightarrow\infty} \frac1T \int_0^T \obser(u,\xi)
dt,\quad
\mbox{Compute } \frac{\partial \timeAver{\obser}}{\partial \xi}
\label{e:gen_problem}
\end{equation}
``sensitivity of long-time averaged quantities is not equal to the long-time average sensitivities of chaotic systems\rf{LeAllHa00} (to me \refref{LeAllHa00}
seems naive, if not downright wrong).''


``This paper outlines a computational method for efficiently
estimating the sensitivity derivative of time averaged statistical
quantities, relying on a single trajectory over a small time interval.''
\\
{\bf 2013-06-22 Predrag} Depend how small is ``small.'' If it is
shorter than the shortest recurrent time of the flow, it
\emph{cannot} probe enough of the \statesp\ to give a reliable
expectation value for an observable. He writes ``Considering that the
oscillation period of the Lorenz attractor is around $1$, the
combined trajectory length of $20$ is a reasonable time integration
length for most simulations of chaotic dynamical systems,'' so
numerics should be reasonable.

``In simulations of chaotic dynamical systems, such as turbulent flows and
the climate system, many output quantities of interest are
                                \toCB
`statistical
averages'.  Denote the state of the dynamical system as $x(t)$;
for a function of the state $\obser(x)$,
the corresponding statistical quantity $\timeAver{\obser}$ is defined
as an average of $\obser(x(t))$ over an infinitely long time interval:''
\begin{equation} \label{Wang13:stats}
\timeAver{\obser} = \lim_{T\rightarrow\infty}
\frac1T \int_0^T \obser(x(t))\,dt\;,
\end{equation}
\\
{\bf 2013-06-22 Predrag} cool, the same as ChaosBook 'time average of
the observable' eq.~(17.4).

``The differentiability of these statistical averages to parameters
of interest as been established through the recent developments in
the Linear Response Theory for dissipative
chaos\rf{Ruelle96,Ruelle09}. A class of chaotic dynamical systems,
known as `quasi-hyperbolic' systems, has been proven to have
statistical quantities that are differentiable with respect to small
perturbations.  These systems include the Lorenz attractor, and
possibly many systems of engineering interest, such as turbulent
flows.''

``The key idea of our method, inversion of the `shadow' operator, is
already used as a tool for proving structural stability of strange
attractors\rf{Ruelle96}. The key strategy of our method, divide and
conquer of the shadow operator, is inspired by recent advances in
numerical computation of the Lyapunov {\cLvs}\rf{ginelli-2007-99,WoSa07}.''
\\
{\bf 2013-06-22 Predrag} That's cool, I am fan of `{\cLvs}'.

Might have to read Krakos\etal\
\emph{Sensitivity analysis of limit cycle oscillations},
and similar \refref{KrWaHaDa12,EyHaLe04}.

                                    \toCB
``For a trajectory $x(t)$ satisfying $\frac{dx}{dt} = f(x)$ and a
scalar or vector field $a(x)$ in the state space, we often use
$\frac{da}{dt}$ to denote $\frac{da(x(t))}{dt}$.  The chain rule
$\frac{da}{dt} = \frac{da}{dx}\cdot \frac{dx}{dt} =
\frac{da}{dx}\cdot f$ is often used without explanation.''

``Now consider a trajectory $x(t)$ satisfying an ordinary differential
equation''
\begin{equation}\label{Wang13:ode}
\dot{x} = \vel(x) \;,
\end{equation}
``with a smooth vector field $\vel(x)$ as a function of $x$.''

A small change in parameters of the system will result in a slightly
different vector field; the equation for a `shadow' trajectory $x'(t)$
\begin{equation} \label{Wang13:wrap0}
x'(x) = x + \epsilon\, \delta x(x) \;,
\end{equation}
$\epsilon$ a small real number,
is
\begin{equation}\label{Wang13:dfode}
\dot{x'} = \vel(x') + \epsilon\, \delta \vel(x') + O(\epsilon^2)
\end{equation}
where the perturbation $\delta \vel$ is
\begin{equation}\label{Wang13:Sf}
\begin{split}
\delta \vel(x) &= -\frac{\partial \vel}{\partial x} \cdot \delta x(x)
             + \frac{\partial \delta x}{\partial x} \cdot \vel(x)  \\
            &= -\frac{\partial \vel}{\partial x} \cdot \delta x(x)
             + \frac{d \delta x}{dt}\\
           :&= (S_\vel \delta x)(x)
\end{split}
\end{equation}
He calls $S_\vel$ the ``{\em Shadow Operator}'' of $\vel$. ${\partial
\vel}/{\partial x}$ is the usual matrix of velocity gradients; the
new part is the $x(t)$ dependence of $\delta x(x)$ that presumably
accounts for the flow being perturbed. Integral of an initial
perturbation $\epsilon\,\delta x$ alone would yield the \JacobianM\
$\jMps$, but now we also integrate over deformation of $\vel$ due to
system parameter changes.

``Given an ergodic dynamical system $\dot{x} = \vel(x)$, and a pair
$(\delta x, \delta \vel)$ that satisfies $\delta \vel =
S_\vel \delta x$, $\delta x$ determines the {\em sensitivity of
statistical quantities} of the dynamical system to an infinitesimal
perturbation $\epsilon\delta \vel$.''

The notation is not fortunate. Here $\delta x$ is not a small vector
(that is taken care of by taking $\epsilon$ small), it is a
generalization of the tangent space of the orbit to the case where
flow is linearized on the deformed vector space.

Qiqi Wang next derives:
\begin{equation} \label{Wang13:sens}
\frac{d \timeAver{\obser}}{d\epsilon}
= \lim_{T\rightarrow\infty}
\frac1T \int_0^T \frac{\partial \obser}{\partial x} \cdot \delta x\:dt
= \timeAver{
\frac{\partial \obser}{\partial x} \cdot \delta x } \;.
\end{equation}
``Equation (\ref{Wang13:sens}) is the \emph{sensitivity derivative} of
a statistical quantity $\timeAver{\obser}$ to the size of a
perturbation $\epsilon\delta \vel$.''

At this juncture I abandon the paper (if someone wants me to, I'll chew
through the rest) - I think we can understand  the essence of the
paper at this point already.

As we are usually given the deformation of dynamics $\delta
\vel(x)$ in terms of the system parameters, we need to compute from
it $\delta x$; that requires the inversion of the  shadow operator in
\refeq{Wang13:Sf}. That is what the paper is about. Natural thing is to
evaluate 'sensitivity' in the covariant frame, spanned by
the eigenvectors of the \JacobianM.
If one does
it on an ergodic trajectory, it is a non-trivial
Ginelli \etal\rf{ginelli-2007-99} kind of undertaking, no wander Nick
is looking for an easier method. I think \po s fit the bill - there
is no problem in computing Floquet vectors for short \po s,
especially for low-dimensional flows such as Lorenz (in real life,
for Navier-Stokes we are computing these in 100.000 dimensions).

\item[2013-06-17 Divakar to Nick Trefethen]
Zai-Qiao Bai (Beijing Normal University) has adapted periodic orbit
theory to random recurrences\rf{Bai07,Bai09,Bai11}. I believe he has
computed the random Fibonacci constant with about 20 digits of
accuracy.

\item[2013-06-22 Predrag]
Zai-Qiao Bai\rf{Bai07} is a continuation of Ronnie
Mainieri's work\rf{Mainieri92a,Mainieri92b}. Our ambition at that
time was that we could transfer the results on super-exponential
convergence of our dynamical \Fd s back to stat mech - did not
get much traction on that project.

McLellan\rf{McLellan12} writes ``Viswanath\rf{DV00} has shown that almost all
random Fibonacci sequences grow exponentially at the rate
1.13198824.... He was only able to find 8 decimal places of this
constant through the use of random matrix theory and a fractal
measure, although Bai\rf{Bai07} has extended the Viswanath's constant by 5
decimal places.'' I have also learned about the Vibonacci numbers :)

Doing literature searches these days is scary - it's impossible to keep up,
no matter how specialized the problem is... Should probably check these:
Lan\rf{Lan11,ZhaLan13},
Forrester\rf{Forrester13}.



\end{description}
