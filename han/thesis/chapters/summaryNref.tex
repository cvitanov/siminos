% siminos/xiong/thesis/chapters/summaryNref.tex
% $Author: predrag $ $Date: 2021-06-22 07:16:06 -0400 (Tue, 22 Jun 2021) $


High- and infinite\dmn\ nonlinear dynamical systems often exhibit
complicated flow (\spt\ chaos or turbulence) in their \statesp\
(phase space).
Sets invariant under time evolution, such as equilibria, periodic orbits,
invariant tori and unstable manifolds, play a key role in shaping the
geometry of such system's longtime dynamics. These invariant solutions
form the backbone of the
global attractor, and their linear stability controls the nearby dynamics.

In this thesis we study the geometrical structure of inertial manifolds of
nonlinear dissipative systems. As an exponentially attracting subset of the
\statesp, inertial manifold serves as a tool to reduce the study of an
infinite\dmn\ system to the study of a finite set of determining
modes.
We determine the dimension of the inertial manifold for the one\dmn\
\KSe\ using the information about the linear stability of system's unstable
\po s.
In order to attain the numerical precision required to study the exponentially
unstable \po s, we
formulate and implement ``\ped'', a new algorithm that enables us
to calculate all Floquet multipliers and vectors of a given \po, for a given
discretization of system's partial differential equations (PDEs).
It turns out that the \On{2} symmetry of \KSe\ significantly complicates
the geometrical structure of the global attractor, so a symmetry reduction
is required in order that the geometry of the flow can be clearly
visualized. We reduce the continuous symmetry using so-called slicing
technique.
The main result of the thesis is that for one\dmn\ \KSe\ defined on a
periodic domain of size $L=22$, the dimension of the inertial manifold is 8,
a number considerably smaller that the number of Fourier modes, 62, used in our simulations.

Based on our results, we believe that inertial manifolds can, in general,
be approximately constructed by using sufficiently dense sets of \po s and
their linearized neighborhoods.
With the advances in numerical algorithms for finding \po s in
chaotic/turbulent flows, we hope that methods developed in this thesis for a
one\dmn\ nonlinear PDE, \ie, using \po s to
determine the dimension of an inertial manifold, can be ported to
higher\dmn\ physical nonlinear dissipative systems, such as \NSe.
