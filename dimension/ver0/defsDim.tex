% siminos/dimension/defsDim.tex
% $Author: xiong $ $Date: 2015-10-01 14:57:09 -0400 (Thu, 01 Oct 2015) $
% copied from schur manuscript

    \usepackage{ifthen}
    \usepackage{color}

        \ifboyscout %%%%%%% DISPLAY COMMENTS IN THE TEXT %%%%%%%%%%%%%%%
   \newcommand{\PCedit}[1]{{\color{blue}#1}}
   \newcommand{\PC}[2]{\begin{quote}\PCedit{[#1 Predrag] #2}\end{quote}}
   \newcommand{\RLDedit}[1]{{\color{red}#1}}
   \newcommand{\RLD}[2]{\begin{quote}\RLDedit{[#1 Ruslan] #2}\end{quote}}
   \newcommand{\Xiongedit}[1]{{\color{green}#1}}
   \newcommand{\Xiong}[2]{\begin{quote}\ESedit{[#1 Xiong] #2}\end{quote}}
   \newcommand{\HC}[2]{\begin{quote}[#1 Hugues] #2\end{quote}}
        \else  % drop comments
   \newcommand{\PC}[2]{}{}
   \newcommand{\PCedit}[1]{#1}
   \newcommand{\RLD}[2]{}{}
   \newcommand{\RLDedit}[1]{#1}
   \newcommand{\Xiong}[2]{}{}
   \newcommand{\Xiongedit}[1]{#1}
   \newcommand{\HC}[2]{}{}
        \fi %%%%%%%%%%%% END OF ON/OFF COMMENTS SWITCH %%%%%%%%%

\ifpreparepdf % hyperlinked pdf, keep homepage flexible:
    \newcommand{\wwwcb}[1]{
                  {\tt \href{http://ChaosBook.org#1}
              {ChaosBook.org#1}}}
    \newcommand{\weblink}[1]{{\tt \href{http://#1}{#1}}}
    \newcommand{\HREF}[2]{{\href{#1}{#2}}}
    \newcommand{\arXiv}[1]{
              {\tt \href{http://arXiv.org/abs/#1}{arXiv:#1}}}
\else  %% prepare for postscript printing:
    \newcommand{\wwwcb}[1]{{\tt ChaosBook.org#1}}
    \newcommand{\weblink}[1]{{\tt #1}}
    \newcommand{\HREF}[2]{{#2}}
    \newcommand{\arXiv}[1]{ {\tt arXiv:#1}}
\fi

%%%%%%%%%%%% MACROS, Xiong Ding specific %%%%%%%%%%
\newcommand{\xDft}{Discrete Fourier Transform }
%\newcommand{\cLv} {covariant vector}  % ChaosBook.org
\newcommand{\cLv} {covariant Lyapunov vector}  % ChaosBook.org
%\newcommand{\cLv} {covariant Lyapunov vector} % Ginelli et al
\newcommand{\CLv} {Covariant vector}  % ChaosBook.org
%\newcommand{\CLv} {Covariant Lyapunov vector} % Ginelli et al
\newcommand{\cLvs} {covariant vectors}  % ChaosBook.org
%\newcommand{\cLvs} {covariant Lyapunov vectors} % Ginelli et al
\newcommand{\CLvs} {Covariant vectors}  % ChaosBook.org
%\newcommand{\CLvs} {Covariant Lyapunov vectors} % Ginelli et al
\newcommand{\Fm}{Floquet multiplier}
\newcommand{\Fe}{Floquet exponent}
\newcommand{\Fv}{Floquet vector}
\newcommand{\entangled} {entangled}
\newcommand{\transient} {spurious}
\newcommand{\psd}{periodic Schur decomposition}
\newcommand{\Psd}{Periodic Schur decomposition}
\newcommand{\prsf}{periodic real Schur form}
\newcommand{\LMa}{Levenberg-Marquardt algorithm}
\newtheorem{per_schur}{Theorem}
\newcommand{\pse}{periodic Sylvester equation}
\newcommand{\Pse}{Periodic Sylvester equation}
\newcommand{\psm}{periodic Sylvester matrix}
\newcommand{\pqr}{periodic QR algorithm}
\newcommand{\Pqr}{Periodic QR algorithm}
\newcommand{\ped}{periodic eigendecomposition}
\newcommand{\Ped}{Periodic eigendecomposition}
\newcommand{\ps}[2]{\mathbf{#1}^{(#2)}}
\newcommand{\Rve}[1]{v_{#1}} % eigenvector of upper-triangular matrix R.
\newcommand{\Jve}[2][0]{\ensuremath{{\bf e}_{#2}^{(#1)}}} % right jacobiam eigenvector
\newcommand{\inm}{inertial manifold}
\newcommand{\Inm}{Inertial manifold}
\newcommand{\InM}{Inertial Manifold}
%%%%%%%%%%%%%% ChaosBook macros %%%%%%%%%%%%%%%%%%%%%%%%%%%%
\newcommand{\refeq}  [1] {(\ref{#1})}
            % in amstex, \eqref is predefined and better than \refeq
\newcommand{\refeqs} [2]{(\ref{#1}--\ref{#2})}
%\newcommand{\href}[2]{{#2}}  % no hyperref
%\newcommand{\HREF}[2]{{#2}}
\newcommand{\rf}     [1] {~\cite{#1}}
\newcommand{\refref} [1] {ref.~\cite{#1}}
\newcommand{\refRef} [1] {Ref.~\cite{#1}}
\newcommand{\refrefs}[1] {refs.~\cite{#1}}
\newcommand{\refRefs}[1] {Refs.~\cite{#1}}
\newcommand{\refsect}[1] {sect.~\ref{#1}}
\newcommand{\refsects}[2] {sects.~\ref{#1} and \ref{#2}}
\newcommand{\refSect}[1] {Sect.~\ref{#1}}
\newcommand{\refSects}[2] {Sects.~\ref{#1} and \ref{#2}}
\newcommand{\reffig} [1] {figure~\ref{#1}}
\newcommand{\reffigs} [2] {figures~\ref{#1} and~\ref{#2}}
\newcommand{\refFig} [1] {Figure~\ref{#1}}
\newcommand{\refFigs} [2] {Figures~\ref{#1} and~\ref{#2}}
\newcommand{\reftab} [1] {table~\ref{#1}}
\newcommand{\refTab} [1] {Table~\ref{#1}}
\newcommand{\reftabs}[2] {tables~\ref{#1} and~\ref{#2}}

%%%%%%%%%%%%%%% EQUATIONS %%%%%%%%%%%%%%%%%%%%%%%%%%%%%%%
\newcommand{\beq}{\begin{equation}}
\newcommand{\continue}{\nonumber \\ }
\newcommand{\nnu}{\nonumber}
\newcommand{\eeq}{\end{equation}}
\newcommand{\ee}[1] {\label{#1} \end{equation}}
\newcommand{\bea}{\begin{eqnarray}}
\newcommand{\ceq}{\nonumber \\ & & }
\newcommand{\eea}{\end{eqnarray}}
\newcommand{\barr}{\begin{array}}
\newcommand{\earr}{\end{array}}

\newcommand{\norm}[1]{\left\Arrowvert \, #1 \, \right\Arrowvert}
\newcommand{\KS}{Kuramoto-Siva\-shin\-sky}
\newcommand{\KSe}{Kuramoto-Siva\-shin\-sky equation}
\newcommand{\cGL}{complex Ginzburg-Landau}
\newcommand{\CGL}{Complex Ginzburg-Landau}
\newcommand{\cGLe}{complex Ginzburg-Landau equation}
\newcommand{\CGLe}{Complex Ginzburg-Landau equation}
\newcommand{\cqcGL}{cubic quintic complex Ginzburg-Landau}
\newcommand{\CqcGL}{Cubic quintic complex Ginzburg-Landau}
\newcommand{\cqcGLe}{cubic quintic complex Ginzburg-Landau equation}
\newcommand{\CqcGLe}{Cubic quintic complex Ginzburg-Landau equation}
\newcommand{\statesp}{state space}
\newcommand{\Statesp}{State space}
% \newcommand{\jacobianM}{fundamental matrix} % no known standard name?
% \newcommand{\jacobianMs}{fundamental matrices}  %
% \newcommand{\JacobianM}{Fundamental matrix} %
% \newcommand{\JacobianMs}{Fundamental matrices}  %
\newcommand{\jacobianM}{Jacobian matrix}  % back to Predrag's name 20oct2009
\newcommand{\jacobianMs}{Jacobian matrices}   % matrices
\newcommand{\JacobianM}{Jacobian matrix} %
\newcommand{\JacobianMs}{Jacobian matrices}  %
\newcommand{\FloquetM}{Floquet matrix} % specialized to periodic orb
\newcommand{\FloquetMs}{Floquet matrices}  %
\newcommand{\stabmat}{stability matrix}     % stability matrix, velocity gradients
\newcommand{\Stabmat}{Stability matrix}     % Stability matrix
\newcommand{\stabmats}{stability matrices}
\newcommand{\monodromyM}{monodromy matrix} % monodromy matrix, Poincare cut
\newcommand{\MonodromyM}{Monodromy matrix} % monodromy matrix, Poincare cut
\newcommand\flow[2]{{f^{#1}(#2)}}
\newcommand{\vel}{\ensuremath{v}}   % state space velocity
\newcommand\velField[1]{{v(#1)}}    % ODE velocity field

\newcommand{\po}{periodic orbit}
\newcommand{\Po}{Periodic orbit}
\newcommand{\rpo}{rela\-ti\-ve periodic orbit}
%   \newcommand{\rpo}{equi\-vari\-ant periodic orbit}
\newcommand{\Rpo}{Rela\-ti\-ve periodic orbit}
%   \newcommand{\Rpo}{Equi\-vari\-ant periodic orbit}
\newcommand{\eqv}{equi\-lib\-rium}
\newcommand{\Eqv}{Equi\-lib\-rium}
\newcommand{\eqva}{equi\-lib\-ria}
\newcommand{\Eqva}{Equi\-lib\-ria}
\newcommand{\reqv}{rela\-ti\-ve equi\-lib\-rium}
%   \newcommand{\reqv}{equi\-vari\-ant equilibrium}
%   \newcommand{\reqv}{travelling wave}
\newcommand{\Reqv}{Rela\-ti\-ve equi\-lib\-rium}
%   \newcommand{\Reqv}{Equi\-variant equi\-librium}
%   \newcommand{\Reqv}{travelling wave}
\newcommand{\reqva}{rela\-ti\-ve equi\-lib\-ria}
%   \newcommand{\reqva}{equivariant equilibria}
\newcommand{\Reqva}{Rela\-ti\-ve equi\-lib\-ria}
%   \newcommand{\Reqva}{Equivariant equilibria}

\newcommand{\reals}{\mathbb{R}}
\newcommand{\complex}{\mathbb{C}}
\newcommand{\sign}[1]{\sigma_{#1}}
%\newcommand{\sign}[1]{{\rm sign}(#1)}
\newcommand{\dual}[1]{{#1}^\ast}
% \newcommand{\transp}[1]{\bar{#1}}
% \newcommand{\transp}[1]{{#1}{}^T}
\newcommand{\transp}[1]{{#1}{}^\top}
\newcommand{\pde}{\partial}
\renewcommand\Im{\ensuremath{{\rm Im}\,}}
\renewcommand\Re{\ensuremath{{\rm Re}\,}}
%\newcommand\stagn{*}        %equilibrium/stagnation point suffix
\newcommand\stagn{q}      %equilibrium/stagnation point suffix

\newcommand{\pS}{\ensuremath{{\cal M}}}          % symbol for state space
\newcommand{\ssp}{\ensuremath{x}}                % state space point
\newcommand\xInit{{\ssp_0}}        %initial x
\newcommand{\zeit}{\ensuremath{t}}  %time variable Ashley
\newcommand{\Mvar}{\ensuremath{A}}  % stability matrix
\newcommand{\jMps}{\ensuremath{J}}   % jacobian matrix, phase space/state space
\newcommand{\monodromy}{\ensuremath{M}}   % monodromy matrix, full Poincare cut
\newcommand{\derf}[2]{\ensuremath{{J}^{#1}(#2)}}    % Predrag fundamental matrix
\newcommand\period[1]{{\ensuremath{T_{#1}}}}         %continuous cycle period
\newcommand{\cl}[1]{{\ensuremath{n_{#1}}}}   % discrete length of a cycle, Predrag
\newcommand{\deltaX}{\ensuremath{{\delta x}}}       %trajectory displacement
\newcommand{\unitVec}{\ensuremath{\hat{n}}}     %unit vector
\newcommand{\cycle}[1]{{\ensuremath{\overline{#1}}}}
\newcommand{\matId}{\ensuremath{{\bf 1}}}      % matrix identity
%\newcommand{\jEigvec}[1]{\ensuremath{{\bf e}^{(#1)}}}  % right jacobiam eigenvector
%\newcommand{\jEigvecT}[1]{\ensuremath{{\bf e}_{(#1)}}}  % left jacobiam eigenvector
%\newcommand{\jEigvec}[1][]{\ensuremath{{\bf e}^{(#1)}}} % right jacobiam eigenvector
%\newcommand{\jEigvecT}[1][]{\ensuremath{{\bf e}_{(#1)}}} % left jacobiam eigenvector
\newcommand{\jEigvec}[1][]{\ensuremath{{\bf e}_{#1}}} % right jacobiam eigenvector
\newcommand{\jEigvecT}[1][]{\ensuremath{{\bf e}_{#1}^\top}} % left jacobiam eigenvector
\newcommand{\ExpaEig}{\ensuremath{\Lambda}}
\newcommand{\Lyap}{\ensuremath{\lambda}}            %Lyapunov exponent
\newcommand{\eigExp}[1][]{
     \ifthenelse{\equal{#1}{}}{\ensuremath{\lambda}}{\ensuremath{\lambda^{(#1)}}}}
\newcommand{\eigRe}[1][]{
     \ifthenelse{\equal{#1}{}}{\ensuremath{\mu}}{\ensuremath{\mu^{(#1)}}}}
\newcommand{\eigIm}[1][]{
     \ifthenelse{\equal{#1}{}}{\ensuremath{\omega}}{\ensuremath{\omega^{(#1)}}}}

%%%%%%%%%%%%%%  Abbreviations %%%%%%%%%%%%%%%%%%%%%%%%%%%%%%%%%%%%%%%%
%%% APS (American Physiology Society, it seems) style:
%%%     Latin or foreign words or phrases should be roman, not italic.
%%%     Insert a `hard' space after full points
%%%                                         that do not end sentences.

\newcommand{\etc}{{etc.}}       % APS
\newcommand{\etal}{{\em et al.}}    % etal in italics, APS too
\newcommand{\ie}{{i.e.}}        % APS
\newcommand{\cf}{{\em cf.\ }}     % APS
\newcommand{\eg}{{e.g.\ }}        % APS, OUP, hard space '\eg\ NextWord'
% \newcommand{\etc}{{\em etc.}}     % etcetera in italics
% \newcommand{\ie}{{that is}}       % use Latin or English?  Decide later.
% \newcommand{\cf}{{cf.}}
% \newcommand{\eg}{{\it e.g.,\ }}   % Wirzba 2sep2001

%%%%%%%%%%%%%%% macros for this paper %%%%%%%%%%%%%%%%%%%%%%
\newcommand{\tph}{TPH}
\newcommand{\op}{\mathbf{x}}    % orbit point