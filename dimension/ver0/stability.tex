% siminos/dimension/stability.tex
% $Author: xiong $ $Date: 2015-09-29 21:24:00 -0400 (Tue, 29 Sep 2015) $

\section{Linear stability and symmetry reduction}
\label{sec:stability}

Vectors in the tangent space of an autonomous flow $\dot{x} =v(x)$ are 
transformed by its first order expansion:
$\dot{\delta x} = A(x) \delta x$ with $A(x)=\partial v / \partial x$.
The solution of $\dot{J}(x,t) = AJ(x,t)$ with $J(x_0,0) = I$ is called
the Jacobian and it transform vectors in tangent space
along the trajectory : $\delta x(t) = J(x_0,0) \delta x_0$.
For a periodic orbit, the eigenvalues and eigenvectors 
of $J_p(x) = J(x, T)$ are called \Fm s and \Fv s. 
Here $T$ is the period 
of this orbit.

A flow transports the
displacement $\deltaX (\zeit)$ along the trajectory
$\ssp(\xInit,\zeit) = \flow{\zeit}{\xInit}$
\beq
{d \over dt} \deltaX(\xInit,t) =
{\Mvar}(\ssp) \,  \deltaX(\xInit,t)
	\,,\qquad \ssp=\ssp(\xInit,t)
\,,
\label{lin_odes}
\eeq
where
\beq
{\Mvar}_{ij}(\ssp) ={\pde \vel_i(\ssp)\over \pde \ssp_j  }
\ee{DerMatrix}
is the \stabmat\ or
velocity gradients matrix.

The finite time $\zeit$
deformation of an infinitesimal neighborhood, with initial frame at $\xInit$,
into the co-moving, non-orthogonal frame at $\ssp(\zeit)$
is described by the eigen\-vectors and eigen\-values
\beq
\jMps^\zeit\, \jEigvec[k] = \ExpaEig_{k} \,\jEigvec[k]
\,,\qquad
j = 1,2, \cdots,d
\,.
\ee{cplxExpaEig1}
of the {\jacobianM} $\jMps^\zeit$
of the linearized flow
\beq
    \deltaX(t) = \jMps^\zeit(\xInit) \, \deltaX_0
    \,, \qquad
\jMps^\zeit_{ij}(\xInit)
  =  \frac{\pde \ssp(\zeit)_i}{\pde \ssp(0)_j}
    \,, \quad
\jMps^0(\xInit) = \matId
\, .
\label{hOdes}
\eeq
%
%\beq
% \jMps^t(\ssp_\stagn) = e^{{\Mvar}_\stagn t}
%    \,,\qquad
% {\Mvar}_\stagn={\Mvar}(\ssp_\stagn)
%\,.
%\ee{eqPointStab}
The symbol $\ExpaEig_k$ denotes the $k$th {\em eigen\-value}
({\em stability multiplier}) of the finite time {\jacobianM} $\jMps^\zeit$.
Symbol $\eigExp[k]$ will be reserved for the $k$th \emph{stability exponent},
with real part $\eigRe[k]$ and phase $\eigIm[k]$:
\beq
\ExpaEig_k = e^{\zeit \eigExp[k]}
    \,\qquad
\eigExp[k] = \eigRe[k] +i \eigIm[k]
\,.
\ee{stabExpon}
Stability multipliers are either real or
come in complex pairs,
\[
\{\ExpaEig_{k},\ExpaEig_{k+1}\}
= \{e^{\cl{}(\eigRe[k] + i\eigIm[k])}, e^{\cl{}(\eigRe[k] - i\eigIm[k])}\}
\,,
\]
with magnitude $|\ExpaEig_k|$ and phase $\eigIm[k]$.
    \PC{2015-05-12}{wrong, fix!}
The phase describes the
rotation velocity in the plane defined by the corresponding pair of
real orthogonal eigen\-vectors, $\{\Re\jEigvec[k],\Im\jEigvec[k]\}$,
with one period of rotation given by
\( %beq
    \period{} = 2\pi/\eigIm[k]
\,.
\) %ee{RotPeriod}
$\derf{t}{\xInit}$ depends on the initial point $\xInit$ and the
elapsed time $t$. For notational brevity we omitted this dependence,
but in general both the stability
multiplier and the eigenvectors,
\(
\ExpaEig_j = \ExpaEig_j(\xInit,t)
    \,,\; \cdots \,,\;
\jEigvec[j] = \jEigvec[j](\xInit,t)
\,,
\)
also depend on the trajectory traversed and the choice of
coordinates. We shall refer to the eigenvectors $\jEigvec[j]$ as
\emph{{\cLvs}}.

an important property of \jacobianMs\ is that they
are multiplicative along the flow,
\beq
\jMps^{\zeit+\zeit'}\!(\ssp)
    = \jMps^{\zeit'}\!(\ssp')\, \jMps^\zeit(\ssp), \qquad \mbox{where} \,\;
\ssp'=\flow{\zeit}{\xInit}
\,.
\ee{Jmultiplic}


Nearby trajectories separate along the {\em unstable directions},
approach each other along the {\em stable directions}, and change
their distance along the {\em marginal directions} at a rate slower
than exponential, corresponding to the eigen\-values of  the
{\jacobianM} with magnitude larger than, smaller than, or equal to 1.
One of the preferred directions is the
direction of the flow itself: $\jMps^{\zeit} (\xInit)$ transports
the velocity vector at $\xInit$ to the velocity vector at
$\ssp(\zeit)$:
\beq
\velField{\ssp(\zeit)} = \jMps^{\zeit} (\xInit) \,\velField{\xInit} \, .
\ee{JacobVeloc}

% 2013-10-18 Predrag from \Chapter{invariants}{17mar2013}{Cycle stability}
The eigenvectors are also the eigenvectors of the \jacobianM,
\( %beq
\jMps^\zeit_\stagn \, \jEigvec[j]
   = \exp({\zeit\eigExp[j]_\stagn })\,\jEigvec[j]
\,.
\) %ee{equilJacob}

$\jMps_{p}(\ssp) = \jMps^\period{}(\ssp)$ is the \jacobianM\
for a single traversal of the prime cycle $p$, $\ssp \in \pS_p$ is
any point on the cycle, and $f^{r\period{}}(\ssp)=\ssp$ as
$f^{t}(\ssp)$ returns to $\ssp$ every multiple of the period
$\period{}$.

The \jacobianM\ $\jMps_p(\ssp)$ depends upon $\ssp$ (the
`starting' point of the periodic orbit),
but its eigenvalues do not, so we
write the eigenvalue equation as
\beq
\jMps_{p}(\ssp)\, \jEigvec[j](\ssp)
   = \ExpaEig_{j} \,\jEigvec[j] (\ssp)
\,,
\ee{cplxExpaEig}
where $\ExpaEig_{j}$ are independent of $\ssp$, and we refer to
eigen\-vectors $\jEigvec[j]$ as `{\cLvs}', or, for \po s,
as `Floquet vectors'.

The time-dependent $\period{}$-periodic vector fields, such as the
flow linearized around a \po, are described by Floquet theory. Hence
we refer to a \jacobianM\ evaluated on a periodic
orbit $p$ either as a [$d\!\times\!d$] {\em \FloquetM} $\jMps_p$ or a
 [$(d\!-\!1)\times(d\!-\!1)$] {\em \monodromyM} $\monodromy_p$, to its
eigenvalues $\ExpaEig_{j}$ as \emph{Floquet multipliers}
\refeq{cplxExpaEig}, and to $\eigExp[j]_p = \eigRe[j]_p+i\eigIm[j]_p$
as \emph{Floquet exponents}.
The stretching/contraction rates per unit time
are given by the real parts of {Floquet exponents}
\beq
\eigRe[j]_p = {1 \over \period{p}} \ln \left|\ExpaEig_{p,j}\right|
\,.
\ee{stabExps}
A Floquet exponent is \emph{not} a Lyapunov
exponent evaluated on one period the prime cycle $p$.
When $\ExpaEig_{j}$ is real, we do care about
\(\sign{}^{(j)}
     =   \ExpaEig_{j}/|\ExpaEig_{j}|
     \in \{+1,-1\}
\,,
\)
the sign of the $j$th Floquet multiplier.
If $\sign{}^{(j)}  = -1$ and $|\ExpaEig_{j}| \neq 1$, the corresponding
eigen-direction is said to be {\em inverse hyperbolic}.

The expanding directions, $|\ExpaEig_e| > 1$, have to be
taken care of first, while the contracting directions $|\ExpaEig_c| <
1$ tend to take care of themselves, hence one orders multipliers
$\ExpaEig_k$ in order of decreasing magnitude
\( %beq
| \ExpaEig_1 | \geq | \ExpaEig_2 |
    \geq \ldots \geq | \ExpaEig_d |
\,.
\) %ee{LambdaOrd}
Since $| \ExpaEig_j | = e^{t \eigRe[j]}$, this is the same as
ordering by
\( %beq
\eigRe[1] \geq \eigRe[2]
    \geq \ldots \geq \eigRe[d]
\,.
\) %ee{lambdaOrd}
We sort the {Floquet multipliers} $\{\ExpaEig_{p,1}$,
$\ExpaEig_{p,2}$, $\dots$, $\ExpaEig_{p,d}\}$ of the
\FloquetM\  evaluated on the $p$-cycle into three sets
$\{e,m,c\}$
\bea
\mbox{expanding:}
        &\quad \{ \ExpaEig \}_e &
        =\, \{ \ExpaEig_{p,j}: \left|\ExpaEig_{p,j}\right| > 1 \}
                \continue
        &\quad \{ \eigExp \}_e &
        =\, \{ \eigExp[j]_p:~\eigRe[j]_p > 0 \}
                \continue
\mbox{marginal:}
        &\quad \{ \ExpaEig \}_m &
        =\, \{ \ExpaEig_{p,j}: \left|\ExpaEig_{p,j}\right| = 1 \}
                \label{EigSorted}\\
        &\quad \{ \eigExp \}_m &
        =\, \{ \eigExp[j]_p:~\eigRe[j]_p = 0 \}
                \continue
\mbox{contracting:}
        &\quad \{ \ExpaEig \}_c &
        =\, \{ \ExpaEig_{p,j}: \left|\ExpaEig_{p,j}\right| < 1 \}
                \continue
        &\quad \{ \eigExp \}_c &
        =\, \{ \eigExp[j]_p:~\eigRe[j]_p < 0 \}
\,.
\nnu
\eea
The volume of expanding manifold plays an
important role. We denote by $\ExpaEig_{p}$ (no $j$th eigenvalue
index) the product of {\em expanding} Floquet multipliers
\beq
\ExpaEig_p=\prod_e \ExpaEig_{p,e}
\,.
\ee{expVol}

If {\em all} Floquet exponents (other than the vanishing longitudinal
exponent) of {\em all} periodic orbits of a flow are strictly bounded
away from zero, the flow is said to be {\em hyperbolic}. Otherwise the
flow is said to be {\em nonhyperbolic}. A confined smooth flow or map is
generically nonhyperbolic, with partial {ellipticity}  or {marginality}
expected only in presence of continuous symmetries, or for bifurcation
parameter values. In presence
of continuous symmetries \eqva\ and \po s are not likely solutions, and
their role is played by higher-dimensional, toroidal, \reqva\ and \rpo s.
