% siminos/dimension/intro.tex
% $Author: xiong $ $Date: 2015-09-29 21:24:00 -0400 (Tue, 29 Sep 2015) $

Dynamics in dissipative systems usually land on an \inm\ [ref] after a 
transient period of evolution. This exponential attracting, 
forward invariant manifold contains the global attractor and 
simplifies the study of the asymptotic dynamics since it is finite
dimensional. The existence of \inm\ has been proved for a lot 
of systems such as \KSe, \cGLe, reaction-diffusion equations and 
so on[ref]. On one hand, mathematicians are continuing to produce better
upper bound on the dimension of \inm\ for some dynamical systems, 
for example, see [ref] for a series of result for one dimensional
\KSe\ [ref]; at the same time,  numerical experiments conducted on
\cLv s[ref] demonstrate that the tangent bundle of ergodic trajectories 
is decomposed into two dynamically decoupled invariant sub-bundles, one 
of which is strictly contracting and the other contains the whole
set of expanding directions and a subset of contracting directions. The 
numerical results suggests that the dimension of \inm\ 
should be larger than Kaplan-Yorke dimension but much smaller than the 
mathematical upper bound.

On the other hand, 
as shown by Predrag \etc, the effective dynamics in chaotic
systems can be visualized as a walk chaperoned
by a hierarchy of unstable invariant solutions (equilibria, periodic orbits
and invariant tori) embedded in the attractor. Moreover, 
if the symbolic dynamics coding this walk is known, then asymptotic 
properties is totally determined by a subset of short periodic orbits.
From this geometrical
point of view, \Fv s associated with \po s should suffice to give 
information about the dimension of \inm\ as \cLv s along ergodic 
trajectories do, and this is the main purpose of this article.
Specifically,
we focus on the 1D \KSe\
\begin{equation}
  u_t+\frac{1}{2}(u^2)_x+u_{xx}+u_{xxxx}=0\,,\; x\in [0,L]
  \label{eq:ks}
\end{equation}
on a periodic domain of size $L=22$, large enough to 
exhibit complex spatiotemporal chaotic dynamics. It is
invariant under Galilean transformation $u(x,t)\to u(x-ct)+c$,
reflection $u(x,t)\to -u(-x,t)$ and spatial stranslation
$u(x,t) \to u(x+\ell,t)$. The Galilean invariance is 
enforced in the integrator, and we will reduce 
the translational symmetry in the experiments.
