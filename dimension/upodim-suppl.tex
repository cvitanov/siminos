\documentclass[pre,twocolumn,twoside,showpacs,superscriptaddress]{revtex4-1}
%\input{epsf,graphics}
\usepackage{graphicx,amssymb,amsmath,epsf,bm,here}
\epsfclipon
\bibliographystyle{apsrev4-1}

%define new commands
\renewcommand{\eqref}[1]{Eq.~(\ref{#1})}
\newcommand{\eqsref}[1]{Eqs.~(\ref{#1})}
\newcommand{\pref}[1]{(\ref{#1})}
\newcommand{\figref}[1]{Fig.~\ref{#1}}
\newcommand{\refcite}[1]{Ref.~\cite{#1}}
\newcommand{\ve}{\bm{e}}
\newcommand{\op}{\mathrm{\mathbf{x}}}
\newcommand{\opa}{\mathrm{\mathbf{a}}}
\DeclareMathOperator{\im}{Im}
\DeclareMathOperator{\re}{Re}
\newcommand{\bra}[1]{\left\langle{#1}\right|}
\newcommand{\ket}[1]{\left|{#1}\right\rangle}
\newcommand{\bracket}[2]{\langle{#1}|{#2}\rangle}

\renewcommand{\thefigure}{S\arabic{figure}}
\renewcommand{\theequation}{S\arabic{equation}}

\begin{document}

\title{Supplementary Material for\\ ``Estimating dimension of inertial manifold from unstable periodic orbits''}

\author{X. Ding}
\affiliation{
                Center for Nonlinear Science, School of Physics,
                Georgia Institute of Technology,
                Atlanta, GA 30332-0430, USA
               }
\author{H. Chat\'e}
\affiliation{
Service de Physique de l'Etat Condens\'e, CEA, CNRS,
Universit\'e Paris-Saclay, CEA-Saclay, 91191 Gif-sur-Yvette, France}%
\affiliation{
Beijing Computational Science Research Center, Beijing 100094, China
}%
\author{P. Cvitanovi\'c}
\affiliation{
                Center for Nonlinear Science, School of Physics,
                Georgia Institute of Technology,
                Atlanta, GA 30332-0430, USA
               }
\author{E. Siminos}
\affiliation{Max-Planck Institute for the Physics of Complex Systems,
	  N\"{o}thnitzer Str. 38, D-01187 Dresden, Germany}%
\affiliation{Department of Physics, Chalmers University of Technology,
             Gothenburg, Sweden}
\author{K. A. Takeuchi}
\email{kat@kaztake.org}
\affiliation{Department of Physics, Tokyo Institute of Technology, 2-12-1 Ookayama, Meguro-ku, Tokyo 152-8551, Japan}%

\date{\today}

\maketitle

\section{In-slice Floquet Vectors}

This section describes
 how the in-slice Floquet vectors $\ket{\hat\ve_j(\hat\op)}$
 are defined and obtained from the original Floquet vectors $\ket{\ve_j(\op)}$
 (bra-ket notation is used).
As in the Letter,
 we consider an autonomous flow $\ket{\op(t)}=f^t\ket{\op(0)}$
 in $d$-dimensional phase space.
%\begin{equation}
% \ket{\op(t)}=f^t\ket{\op(0)}.  \label{eq:Flow}
%\end{equation}
The Kuramoto-Sivashinsky equation formally corresponds to $d = \infty$,
 but in practice numerical integration is performed
 in a finite-dimensional phase space, whose dimensionality $d$ is set
 by the cutoff wavenumber chosen in the pseudospectral method.
In any case, $\ket{\op(t)}$ represents the field $u(x,t)$,
 which is also described by its Fourier components
\begin{equation}
 a_k(t) \equiv \frac{1}{L}\int_0^L u(x,t) e^{-iq_k x},
% \quad u(x,t) = \sum_k a_k e^{iq_k x},
  \label{eq:Fourier}
\end{equation}
 with $q_k=2\pi{}k/L$ and $a_0(t)=0$.
Within this formalism, a periodic orbit $\ket{\op(0)}$
 is a fixed point of the map $f^{T_p}$, with $T_p$ the period of the orbit.
Its Floquet multipliers $\Lambda_j$ and Floquet vectors $\ket{\ve_j(\op)}$
 are the eigenvalues and eigenvectors, respectively,
 of the corresponding Jacobian $J^{T_p}$.
Pre-periodic orbits and relative periodic orbits
 can also be dealt with straightforwardly,
 by replacing $f^{T_p}$ with $Rf^{T_p}$ and $g(\theta_p)f^{T_p}$,
 respectively, with reflection $\sigma$ and spatial translation $g(\theta_p)$
 as defined in the Letter.
Therefore, in the following, we describe the case of periodic orbits
 for the sake of simplicity.

Here we focus on the invariance
 under spatial translation $u(x,t)\to{}u(x+\ell,t)$,
 which amounts to a rotation $a_k(t)\to{}e^{iq_k\ell}a_k(t)$ in Fourier space, 
 described by the operator $g(2\pi\ell/L)$.
To reduce the marginal dimension due to this symmetry,
 we send all trajectories and orbits to the $(d-1)$-dimensional hyperplane
\begin{equation}
 \im(a_1(t))=0, \quad \re(a_1(t))>0,  \label{eq:Slice}
\end{equation}
 which is called the first Fourier-mode slice \cite{Budanur.etal-PRL2015}.
This is realized by transformation
\begin{equation}
  \ket{\hat\op(t)} \equiv g(-\theta(\op(t)))\ket{\op(t)}  \label{eq:SliceTrans}
\end{equation}
 with $\theta(\op(t))=\arg{}a_1(t)$.

First we consider infinitesimal perturbations to $\ket{\op(t)}$,
 denoted here by $\ket{\delta\op(t)}$.
Using \eqref{eq:SliceTrans}, one can show that
 $\ket{\delta\op(t)}$ is transformed to
\begin{equation}
 \ket{\delta\hat\op(t)}' \equiv
 h(\op(t))g(-\theta(\op(t)))\ket{\delta\op(t)},  \label{eq:SliceTransVec1}
\end{equation}
 with
\begin{equation}
 h(\op) \equiv 1-\frac{\ket{T\op}\bra{T\op_0}}{\bracket{T\op_0}{T\op}}.  \label{eq:MatH}
\end{equation}
Here, $T$ is the generator defined by $g(\theta)\equiv{}e^{T\theta}$,
 the inner product is
\begin{equation}
 \bracket{\op}{\op'} \equiv \frac{1}{L}\int_0^L u^*(x)u'(x) dx = \sum_k a_k^*a'_k,  \label{eq:InnerProd}
\end{equation}
 and $\op_0$ is a reference point on the slice \cite{Budanur.etal-PRL2015},
 which therefore satisfies $\bracket{\delta\hat\op(t)}{T\op_0}=0$.
Since we use the first Fourier-mode slice \pref{eq:Slice},
 we can take $\op_0 = (1,0,\cdots,0)$ in the Fourier representation.
Note that, since the dimensionality of the slice is one less than
 that of the phase space,
 so is the dimensionality of the in-slice perturbations.
This is reflected by the fact that $\bra{T\op_0}h(\op)=0$,
 or $\sum_it'_ih_i(\op)=0$,
 where $h_i(\op)$ is the $i$th row vector of $h(\op)$
 and $t'_i$ the $i$th component of $\bra{T\op_0}$.
Therefore, by rank factorization, one obtains
\newpage
\begin{widetext}
\begin{equation}
 h(\op)
 = \begin{bmatrix} h_1 \\ \vdots \\ h_{i_0-1} \\ h_{i_0} \\ h_{i_0+1} \\ \vdots \\ h_d \end{bmatrix}
 = \begin{bmatrix}
 1 & & & & & \\ & \ddots & & & O & \\ & & 1 & & & \\
 -\frac{t'_1}{t'_{i_0}} & \cdots & -\frac{t'_{i_0-1}}{t'_{i_0}}
 & -\frac{t'_{i_0+1}}{t'_{i_0}} & \cdots & -\frac{t'_d}{t'_{i_0}} \\
 & & & 1 & & \\ & O & & & \ddots & \\ & & & & & 1
 \end{bmatrix} \begin{bmatrix}
  h_1 \\ \vdots \\ h_{i_0-1} \\ h_{i_0+1} \\ \vdots \\ h_d
 \end{bmatrix}
 \equiv P \hat{h}(\op),  \label{eq:SliceTransH}
\end{equation}
\end{widetext}
 with $d \times (d-1)$ matrix $P$, $(d-1) \times d$ matrix $\hat{h}(\op)$,
 and $i_0$ chosen such that $t'_{i_0} \neq 0$.
For the first Fourier-mode slice adopted here,
 all the components of $\bra{T\op_0}$ except $t'_2$ are zero
 \cite{Budanur.etal-PRL2015}, so that $i_0=2$.

Now let us define the $(d-1)$-dimensional
 in-slice perturbation $\ket{\delta\hat\op(t)}$
% that corresponds to $\ket{\delta\op(t)}$
 by
\begin{equation}
 \ket{\delta\hat\op(t)} \equiv \hat{h}(\op(t))g(-\theta(\op(t)))\ket{\delta\op(t)}.  \label{eq:SliceTransVec2}
\end{equation}
By construction, there is one-to-one correspondence between
 $\ket{\delta\hat\op(t)}'$ and $\ket{\delta\hat\op(t)}$
 through the operator $P$
 [see \eqsref{eq:SliceTransVec1}, \pref{eq:SliceTransH},
 \pref{eq:SliceTransVec2}],
 unless $\ket{\delta\hat\op(t)}'$ is tangent to the spatial translation,
 $\ket{\delta\hat\op(t)}' \propto \ket{T\op}$,
 which results in $\ket{\delta\hat\op(t)}=0$ [see \eqref{eq:MatH}].
Therefore, $\ket{\delta\hat\op(t)}$ indeed describes
 the perturbation within the slice.
Moreover, if we define the evolution operator $\hat{J}^t(\op)$ by
\begin{equation}
 \ket{\delta\hat\op(t)} = \hat{J}^t(\op(0)) \ket{\delta\hat\op(0)},  \label{eq:InsliceJacobian}
\end{equation}
 similarly to 
\begin{equation}
 \ket{\delta\op(t)} = J^t(\op(0)) \ket{\delta\op(0)},  \label{eq:Jacobian}
\end{equation}
 with $J^t(\op)$ being the Jacobian of $f^t(\op)$, we obtain
\begin{align}
 \ket{\delta\hat\op(t)}
 &= \hat{h}(\op(t))g(-\theta(\op(t)))J^t(\op(0)) \ket{\delta\op(0)} \notag \\
 &= \hat{J}^t(\op(0)) \hat{h}(\op(0))g(-\theta(\op(0)))\ket{\delta\op(0)}.  \label{eq:InsliceJacobian2}
\end{align}
For a periodic orbit of period $T_p$ (hence $\op(T_p) = \op(0)$),
 we have $J^{T_p}(0)\ket{\ve_j(\op(0))} = \Lambda_j \ket{\ve_j(\op(0))}$
 with its Floquet multipliers $\Lambda_j$ and vectors $\ket{\ve_j(\op)}$.
Therefore, defining
\begin{equation}
 \ket{\hat\ve_j(\op)} \equiv \hat{h}(\op)g(-\theta(\op))\ket{\ve_j(\op)},  \label{eq:InsliceFloquet}
\end{equation}
 we find that they are indeed eigenvectors of $\hat{J}^t(\op)$,
 associated with the eigenvalues $\Lambda_j$ unchanged by the transformation.
This justifies calling $\ket{\hat\ve_j(\op)}$ the in-slice Floquet vectors,
 associated with the Floquet multipliers $\Lambda_j$ or exponents $\lambda_j$
 of the orbit.
Similarly to $\ket{\delta\hat\op(t)}$,
 all information of the $j$th Floquet mode
 is retained in the corresponding in-slice Floquet mode,
 except that the marginal mode due to the spatial translation
 is excluded in the in-slice descriptions.
Therefore, the number of the marginal modes,
 as well as the number of the entangled Floquet modes, are one less
 than those in the full-space descriptions.

\begin{thebibliography}{99}
\bibitem[S1]{Budanur.etal-PRL2015}%
  N.~B. Budanur, P. Cvitanovi\'c, R. L. Davidchack, and E. Siminos, Phys. Rev. Lett. \textbf{114}, 084102 (2015).
\end{thebibliography}
%\bibliography{WEBinKPZ}



\end{document}

