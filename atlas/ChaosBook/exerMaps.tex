%\Problems{exerMaps}{29jan2012}
% $Author: predrag $ $Date: 2012-02-23 20:52:45 -0500 (Thu, 23 Feb 2012) $

% Predrag                   29jan2012
%   added {e:RosslTransv} on border of transversality
% Mason                     28aug2003 25sep2003
% Rytis                     10sep2003
% Predrag                   10aug2003
% Dorte                     21sep2001
% Predrag                   19apr2001

\Exercise{Poincar\'e sections of the R\"ossler flow.}{\label{e:RosslMap}
\index{Rossler@R\"ossler!flow}

(continuation of \refexer{e:RosslSys})
Calculate numerically a Poincar\'e section (or several Poincar\'e
sections) of the R\"ossler flow. As  the  R\"ossler flow {\statesp} is
3$D$, the flow maps onto a 2$D$ Poincar\'e section. Do you see that in
your numerical results? How good an approximation would a replacement of
the return map for this section by a 1\dmn\ map be? More precisely,
estimate the thickness of the strange attractor.
~(continued as \refexer{e:RosslEqStab})

\authorRP
    }% end \Exercise{R\"ossler flow (Rytis - 09/10/2003).}{

\exercise{A return Poincar\'e map for the R\"ossler flow.}{
\label{e:RosslRetMap}
% \index{Rossler@R\"ossler!flow}
(continuation of \refexer{e:RosslMap})
That Poincar\'e return maps of \reffig{FigRosslRetMap} appear multimodal
and non-invertible is an artifact of projections of a 2\dmn\ return map
$(R_n,z_{n}) \to (R_{n+1},z_{n+1})$ onto a 1\dmn\ subspace $R_n \to
R_{n+1}$.

Construct a genuine $s_{n+1} = f(s_{n})$ return map by parameterizing
points on a Poincar\'e section of the attractor \reffig{f:RosslSect} by a
Euclidean length $s$ computed curvilinearly along the attractor section.

This is best done (using methods to be developed in what follows) by a
continuation of the unstable manifold of the 1-cycle embedded in the
strange attractor,  \reffig{FigRosslCycles}\,(b).

\authorPC
    }% end \Exercise{A return Poincar\'e map

\exercise{Arbitrary Poincar\'e sections.}{\label{exPS1}
We will generalize the construction of Poincar\'e sections so that
they can have any shape, as specified by the equation $\PoincC(x)=0$.
\begin{enumerate}
\item
Start by modifying your integrator so that you can change the
coordinates once you get near the Poincar\'e section.  You can do this
easily by writing the equations as
\begin{equation}
    \frac{dx_{k}}{ds} = \kappa f_{k}
    \,,
\end{equation}
with $dt/ds = \kappa$, and choosing $\kappa$ to be $1$ or $1/f_{1}$.
This allows one to switch between $t$ and $x_{1}$ as the integration
'time.'
\item
Introduce an extra dimension $x_{n+1}$ into your system and set
\begin{equation}
    x_{n+1} = \PoincC(x)
    \,.
    \label{eqPSEx1}
\end{equation}
How can this be used to find a Poincar\'e section?
\end{enumerate}
%\authorRM
}

\Exercise{Classical  collinear helium dynamics.}{
\index{helium!collinear}
\label{e-helium-PoincSec}
} %end \exercise{Classical  collinear helium dynamics.}{

\Exercise{H\'enon map fixed points.}{\label{e-Henon-fixed}
% Predrag moved this from exerCycles        Sept 6 2001
\index{Henon@H\'enon map!fixed points}
Show that the two fixed points $(x_0,x_0)$,  $(x_1,x_1)$ of the H\'enon map
\refeq{eq2.1a} are given by
\bea
      x_0  &=&  { -(1-b)-\sqrt{(1-b)^2+4a} \over 2a} \,, \quad
                        \continue
      x_1  &=&  { -(1-b)+\sqrt{(1-b)^2+4a} \over 2a}
\,.
\label{exer:eq2}
\eea
} %end \exercise{H\'enon map fixed points.}{

\exercise{Fixed points of maps.}{\label{e:FixedPointsMaps}
% Predrag moved this from exerCycles        Sept 6 2001
\index{fixed point!maps}
\index{map!fixed point}
A continuous function $F$ is a contraction of the unit interval if it
maps the interval inside itself.
\begin{enumerate}
\item
Use the continuity of $F$ to show that a 1\dmn\ contraction $F$ of the
interval $[0,1]$ has at least one fixed point.
\item
In a uniform (hyperbolic) contraction the slope of $F$ is always smaller
than one, $|F'| < 1$.  Is the composition of uniform contractions a
contraction?  Is it  uniform?
\end{enumerate}
%\authorRM
}

\exercise{Border of transversality for R\"ossler Poincar\'e sections.}
{\label{e:RosslTransv}
\index{Rossler@R\"ossler!flow}
\index{transversality!border of}\index{border of transversality}
(continuation of \refexer{e:RosslMap})
Determine numerically borders of transversality \refeq{eq:sspRSing} for
several R\"ossler flow Poincar\'e sections of \refexer{e:RosslMap}
and \reffig{f:RosslSect}, at least for angles
\begin{itemize}
\item[(a)] $-60^o$
, (b) $0^o$, and
\item[(c)] A Poincar\'e section hyperplane that goes through both \eqva,
see \refeq{RossEqP} and \reffig{f:RossEquil}. Two points only fix a line:
think of a criterion for a good orientation of the section hyperplane,
perhaps by demanding that the contracting eigenvector of the 'inner'
\eqv\ $\ssp_{-}$ lies in it.
\item[(d)] (Optional) Hand- or computer-draw a visualization of the
border of transversality as 3\dmn\ fluid flow which either crosses,
is tangent to, or fails to cross a sheet of light cutting across the
flow.
\end{itemize}
As the \statesp\ is 3\dmn, the borders of transversality are 1\dmn, and
it should be easy to outline the border by plotting the color-coded
magnitude of $\vel_\perp(\sspRed)$, component of the $\vel(\sspRed)$
normal to the section, for a fine grid of 2\dmn\ Poincar\'e section plane
points. For sections that go through the $z$-axis, the normal velocity
$\vel_\perp(\sspRed)$ is tangent to the circle through $\sspRed$, and
vanishes for $\dot{\theta}$ in the polar coordinates \refeq{eq2.1b}, but
that is not true for other Poincar\'e sections, such as the case (c).

\authorPC
    }% end \Exercise{Border of transversality


%    \ProblemsEnd
