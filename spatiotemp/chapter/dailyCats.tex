\svnkwsave{$RepoFile: siminos/spatiotemp/chapter/dailyCats.tex $}
\svnidlong {$HeadURL: svn://zero.physics.gatech.edu/siminos/spatiotemp/chapter/dailyCats.tex $}
{$LastChangedDate: 2021-12-08 00:39:15 -0500 (Wed, 08 Dec 2021) $}
{$LastChangedRevision: 7383 $} {$LastChangedBy: predrag $}
\svnid{$Id: dailyCats.tex 7383 2020-04-19 05:28:36Z predrag $}

\chapter{\catLatt, blogged}
\label{chap:dailyCats}
% Predrag                                           12 January 2016

\begin{bartlett}{
% \HREF{http://www.jango.com/music/The+Verve} {Space And Time}
\HREF{https://soundcloud.com/redwantingblue/spacetime}
     {I'm a space and time continuum }
                }\bauthor{
% The Verve on Urban Hymns
\HREF{http://music.redwantingblue.com/track/space-time}
{Red Wanting Blue}
                }

\end{bartlett}

\bigskip

   % *********************************************************************
\hfill   {\color{red} The latest entry at the bottom for this blog}

\bigskip


\begin{description}

\item[2016-01-12 PC]  Literature
related to Klaus Richter - Engl \etal\rf{EDASRU14}
{\em  Coherent backscattering in {Fock} space: {A} signature of quantum
many-body interference in interacting bosonic systems};
Engl \etal\rf{EnUrRi15}
{\em Boson sampling as canonical transformation:
{A} semiclassical approach in {Fock} space}

Some of the stuff might already be in Richter\rf{Richter10}
\HREF{http://www.physik.uni-regensburg.de/forschung/richter/richter/pages/research/springer-tracts-161.pdf}
{book} {\em Semiclassical Theory of Mesoscopic Quantum Systems}.

    \PCpost{2016-05-16}{ I am putting all stuff about cat maps known to us
into \refsect{sect:CatMap}.
    }

    \PCpost{2016-05-17}{
I have added for the time being \refchap{c-appendStatM} {\em Statistical
mechanics applications} from ChaosBook to this blog.
%Sorry about
%\begin{quote}
%... $\{4\}$ Rana's blog$\}\{23\}\{$chapter.$4\}$
%\end{quote}
%error. I'm too lazy to fix it - just press return.
Note that there are yet
more references to read in the Commentary to the
\refchap{c-appendStatM}.
    }

%    \PCpost{2016-03-02}{meeting with Li Han:
%He has not done anything yet (Boris?) and read no
%literature from the list above (this is a research project, not
%something predigested. One has to read the literature on one's own,
%nobody can do that for you) so
%I have asked him to understand Dong and
%Lan\rf{DoLa14} {\em Organization of spatially periodic solutions of the
%steady {Kuramoto–Sivashinsky} equation}.
%% in detail and soon.
%    }

    \BGpost{2016-05-17}{
There has been some progress in the cat maps direction. Li did a simple
simulation for single cat map which has  confirmed my guesses so far.
General question is about 2D symbolic dynamics which we used in our paper
with Vladimir\rf{GutOsi15}. We encoded trajectories by   winding numbers so the
alphabet is small. This radically   differs  from standard symbolic
dynamics - Markov partitions, etc. which you normally  find in the
literature.

Question: Given  rectangular domain of symbols - what is
the probability  to  find it within a generic N-particle {\catlatt}
trajectory of duration T (N and T are large)? To my surprise  an answer
for this question might be within reach  (at least for most of the
sequences, small rectangles, etc.) This symbolic dynamics is much nicer
then what I thought it would be. If true, this will be a nice project for
students. I am writing some notes, will send this  crude staff at night
today - hopefully it will be  digestible.
    }

%    \PCpost{2016-05-17}{
%Li Han, can you explain this to us? We are all in Howey until Friday.
%    }

    \BGpost{2016-05-18}{
To speed up everything I send my notes
\HREF{160521Gutkin.pdf} {(click here)} on recent progress on cat
map symbolic dynamics. Very crude, limited edition (probably barely
digestible :). This what I managed to put in latex tonight. More staff
and explanations  will come on Friday . So far only few  things (and only
for single cat) were checked numerically.
    }

    \BGpost{2016-05-21}{
The current version of my notes: \HREF{160521Gutkin.pdf} {(click here)},
with more staff,  more or less the same level of  disorder. The main news
- this symbolic dynamics (for single cat) was investigated  a  time ago
by Percival and Vivaldi\rf{PerViv}. At least some of their
results are  of relevance for us and maybe overlap with my results. Need
some time to digest their paper.
    }

    \PCpost{2016-05-21}{
For simple linear maps with integer coefficient it might be possible to
write explicitly all period points in terms of rational numbers.
See \toChaosBook{exmple.14.10} example 14.10
{\em Periodic points of the full tent map}.
%\HREF{http://www.streamsound.dk/book1/chaos/chaos.html\#248/z} {ChaosBook}.
    }

    \PCpost{2016-06-01}{
The downside the \PV\rf{PerViv} `linear code' is that the Markov/generating
partition is infinite, meaning that for longer and longer orbits there are
more and more new pruning (inadmissible {\brick s}) rules, ad infinitum.

As far as I can tell, for $N$ coupled maps this gets harder to describe,
as in general there is an coupling strength parameter. Perhaps for rational
values of it some miracles might happen.
    }


    \BGpost{2016-07-10}{
The current ver.~5020 of my notes: \HREF{160521Gutkin.pdf} {(click here)},

Main atractions:
\begin{enumerate}
  \item Single cat:
Frequencies  of sequences which are composed of internal symbols
eq.~(2.23) - analytic answer.
  \item Infinity of cats, pp.~9-11:
Results for single symbol frequency.  Some results for  [2$\times$2] squares
- Rana eqs, internal symbols, forbidden sequences. Looks like extendable
to general rectangles (Future looks bright
\HREF{https://www.youtube.com/watch?v=8qrriKcwvlY}{gotta wear shades}).

\end{enumerate}
To Rana, Adrien and Li - we Skype Wednesday 5pm.
    }

\PCpost{2016-08-15}{:
Before I start sounding critical: Rana, Adrien and Li are all good
students / postdoc, and the work and what people learned this summer is
very good. Now, to my first impressions (Boris still has to chime in, and
we should all meet soon to discuss).

Boris was the primary advisor for Adrien and Rana, and he did not push
them to write; European style, more hands off than is the American
custom. Rana has written too little, not covering all the work that she
did (which was good, but only Boris knows what it was). Adrien also has
some work to do before the report can be read by anyone other than the
members of this team.

For me, what lacks in Rana, Adrien and Li's work is that they did not
(on their own initiative) read the literature, or if they did, they never
summarized what they learned in their blogs, or cited the original
sources in their reports (if it is not recorded, I assume it did not
happen). I did an exhaustive literature search for them (for coupled
maps, see above in this blog, \refchap{chap:dailyCats}; for cat maps see
\refsect{sect:ArnCatLit} and \refsect{c-appendStatM}), but they did not
pick it up, so I'll have to do that work myself.
}

    \PCpost{2016-10-03}{
Not quick or easy to explain, but I have a hunch that the {\spt}
zeta function should be something like the 2D Ising model zeta function
described by Aizenman, see \refchap{chap:Ising2D}.
    }


    \PCpost{2016-11-06}{ Early references on the {\spt}
invariants and invariant measures:

Grassberger\rf{Grassberger89}
{\em Information content and predictability of lumped and distributed
dynamical systems}
``
... we point out the difference between difficulty and possibility of
forecasting, illustrating it with quadratic maps. Next, we ask ourselves how
this should be generalized to distributed, spatially extended and homogeneous
systems. We point out that even the basic concepts of how information is
processed by such systems are unknown. Finally, we discuss some
intermittency-like effects in coupled maps and cellular automata.

Most interesting systems are spatially extended (``distributed'')
and have thus, in the infinite-volume limit, an infinite number
of degrees of freedom. Nevertheless, concepts developed for
dynamical system with few degrees of freedom can be
applied.

Let us first consider a system located in a finite space
volume V with fixed boundary conditions. For the moment
we assume that the field variables (local observables) are
continuous, while space and time are discrete.

It is very plausible that the number of excited degrees of freedom in such a
system is proportional to the volume, at least in typical situations. More
precisely, we expect that both the metric entropy h and the dimension D are
extensive quantities. This is supported by overwhelming
theoretical\rf{ruelext,FMTT83,Nicolaenko86} and numerical evidence that the
Lyapunov exponents (ordered such that 2, decreases with i) scale as
\beq
\Lyap_i \approx f( i/V )
\ee{Grassberger89(4.1)}
[...]
The limit
\(
\eta = \lim_{V\to\infty}(h/V)
\) is
called density of metric entropy.
[...]
more interesting is the
information needed to describe a finite part of an infinite
system than that needed to describe the system as a whole.
''
    }


    \BGpost{2016-11-06}{ Papers to read:
  	
Bardet and Keller\rf{BarKel06} {\em Phase transitions in a piecewise
expanding coupled map lattice with linear nearest neighbour coupling} write: ``
We construct a mixing continuous piecewise linear map on [-1,1] with the
property that a two-dimensional lattice made of these maps with a linear
north and east nearest neighbour coupling admits a phase transition. We also
provide a modification of this construction where the local map is an
expanding analytic circle map. The basic strategy is borrowed from
    Gielis and MacKay (2000 Nonlinearity 13 867-88 );
namely, we
compare the dynamics of the CML with those of a probabilistic cellular
automaton of Toom's type; see
    MacKay (2005) {\em Dynamics of coupled map lattices and of related
    spatially extended systems} (Lecture Notes in Physics vol 671)
ed J-R Chazottes and B Fernandez (Berlin: Springer) pp 65–94)
for a detailed discussion.
''

de Maere\rf{deMaere10}
{\em Phase transition and correlation decay in coupled map lattices} writes: ``
For a Coupled Map Lattice with a specific strong coupling emulating
Stavskaya's probabilistic cellular automata, we prove the existence of a
phase transition using a Peierls argument, and exponential convergence to the
invariant measures for a wide class of initial states using a technique of
decoupling originally developed for weak coupling. This implies the
exponential decay, in space and in time, of the correlation functions of the
invariant measures.
''

Schmitt\rf{Schmitt04}
{\em Spectral theory for nonanalytic coupled map lattices} write: ``
We consider weakly coupled strong mixing interval maps on the infinite
lattice with finite range couplings. We construct Banach spaces defined by bounded
variation conditions such that the transfer operator associated with the full
infinite system has a simple eigenvalue 1 (corresponding to a unique natural
invariant probability measure) and a spectral gap.
''
}

    \PCpost{2016-11-11}{
Bountis and Helleman\rf{Bount81}
{On the stability of periodic orbits of two-dimensional mappings}: ``
We apply our criterion and derive a sufficient stability condition for a
large class of periodic orbits of the widely studied ``standard mapping''
describing a periodically `kicked' free rotator.
''

I find this paper quite interesting, because the computation of Floquet
multipliers, \ie, linearization of periodically `kicked' free rotor, is full
of matrices that look like Laplacians + a diagonal term which varies along
the periodic orbit. For cat maps this term is constant, essentially the
stretching factor $s$. This might help with interpreting coupled `kicked'
rotor lattices.

This is presumably related to the block circulant stability
matrices\rf{AGGN91,GadAmr93} for spatially and temporally periodic orbits
in coupled map lattices.
    }

    \PCpost{2017-02-13}{
\renewcommand{\Ssym}[1]{{\ensuremath{m_{#1}}}}    % Boris

Consider admissible Lagrangian trajectories in the $[\ssp_0,\ssp_{\ell+1}]$
plane. The
1-step Lagrangian map is
\beq
\ssp_{2} = (s \, \ssp_{1} -\Ssym{1}) - \ssp_{0}
\,.
\ee{eq:CatMapLagr}
For $s=3$ have $\Ssym{j}\in\{-1,0,1,2\}$, and the partition
borders are slope -1 lines defined by $\ssp_{1} =  0 \mbox{ or } 1$.
\beq
-\Ssym{1} - \ssp_{0} \leq \ssp_{2} \leq s -\Ssym{1} - \ssp_{0}
\,.
\ee{eq:CatMapLagrEnds}
Anything outside $0 \leq \ssp_{2}  \leq 1$ is not a constraint.
\beq
1 - \ssp_{0} \leq \ssp_{2}^{(-1)} \leq 4 - \ssp_{0}
\,.
\ee{eq:CatMapLagr-1}
\beq
- \ssp_{0} \leq \ssp_{2}^{(0)} \leq 3 - \ssp_{0}
\,.
\ee{eq:CatMapLagr0}
\beq
-1- \ssp_{0} \leq \ssp_{2}^{(1)} \leq 2 - \ssp_{0}
\,.
\ee{eq:CatMapLagr1}
\beq
-2- \ssp_{0} \leq \ssp_{2}^{(2)} \leq 1 - \ssp_{0}
\,.
\ee{eq:CatMapLagr2a}
So, there is no pruning for $\ssp_{2}^{(0)}$ and $\ssp_{2}^{(1)}$;
the only pruning constraints are
\beq
1 - \ssp_{0} \leq \ssp_{2}^{(-1)}
\,,
\ee{eq:CatMapLagr-1p}
\beq
\ssp_{2}^{(2)} \leq 1 - \ssp_{0}
\,.
\ee{eq:CatMapLagr2b}



2-step Lagrangian map
\beq
\ssp_{3} = (s-1)(\ssp_{2}+\ssp_{1}) - \ssp_{0}
-\Ssym{2}-\Ssym{1}
\,,
\ee{eq:CatMapNewton2a}
Eliminate $\ssp_{2}$ by \refeq{eq:CatMapLagr}
\beq
\ssp_{2}+\ssp_{1} = ((s+1)\ssp_{1} -\Ssym{1}) - \ssp_{0}
\,.
\ee{eq:CatMapLagr2+1}
\beq
\ssp_{3} = (s^2-1)\ssp_{1} - s\,\ssp_{0}
-\Ssym{2}-s\,\Ssym{1}
\,,
\ee{eq:CatMapNewton2+1}
    }


\PCpost{2016-09-28}{
Learned much more from \HREF{https://www.math.gatech.edu/users/rll6}
{Rafael de la Llave} that I can possibly remember.

He says that fancy Smalians, like Pugh and Shub, who studied
dynamical systems with `multiple times'. The results are expressed in
terms of `Weyl chambers'. They are particularly simple if symmetries
commute. They call that the `Abelian case'.
    }

    \PCpost{2016-09-28}{                                        %\toCB
Volevich\rf{Volevich91} {\em Kinetics of coupled map lattices} writes: ``
A simple example of coupled map lattices generated by expanding maps of
the unit interval with some kind of diffusion coupling is considered. The
author proves that probability measures from some natural class weakly
converge to the unique invariant mixing measure under the actions of
dynamics. The main idea of the proof is the symbolic representation of
his system by two-dimensional lattice model of statistical mechanics.
''
    }

    \PCpost{2016-09-28}{                                        %\toCB
Giberti and Vernia\rf{GibVer94,BFGV97} {\em Normally attracting manifolds and
periodic behavior in one-dimensional and two-dimensional coupled map
lattices} study weakly coupled logistic maps in one- and two-dimensional
lattices. They show numerically for some classes of coupled map lattices that
the stability of a spatial structure is determined by the stability of its
pattern with the minimal (spatial) scale, i.e. by the tiniest detail of this
structure.

Continued in Livi, Mart{\'i}nez-Mekler and Ruffo\rf{BILMR89,LiMaRu90,BLMR92}
{\em Periodic orbits and long transients in coupled map lattices}.

Hopefully something we can ignore as long as we are interested into the
``high-temperature'' phase.
    }

    \PCpost{2016-10-03}{
In 1986 Zimmer outlined a general program directed at understanding
smooth actions of lattices in semisimple Lie groups on compact manifolds.
So hopefully this has nothing to do with CLMs.

de la Llave and Mireles James\rf{LlaMir16}
{\em Connecting orbits for compact infinite dimensional maps: {Computer}
assisted proofs of existence}
  is indigestible from get go.
    }

    \PCpost{2016-11-11}{
Boris A. Khesin
\HREF{http://www.math.toronto.edu/khesin/recent_grad_teaching.html}
{course notes} might be helpful if we ever get to Hamiltonian PDEs :)
    }

    \PCpost{2016-11-11}{
Belykh and Mosekilde\rf{BelMos96} {\em One-dimensional map lattices:
{Synchronization}, bifurcations, and chaotic structures} study
\bea
x_{n, t+1} &=& f(x_{n,t}) + \epsilon
            [
            x_{n+1,t}  - (1+\gamma) x_{n,t} + \gamma x_{n-1,t}
            ]
\continue
            &=&  f(x_{n,t})+ (1-\gamma)\epsilon\, (x_{n,t} - x_{n-1,t})
                           + 2\epsilon\,\Box x_{n,t}
\,,
\label{BelMos96-1}
\eea
asymmetrically drifting due to coupling by $(1-\gamma)$. With $g(x)=f(x)-x$,
and rescaling the nonlinearity, this is a discretization of
a dynamics of extended, non-equilibrium media PDE
\beq
\frac{\partial x}{\partial t}
 = g(x) + (1-\gamma)\frac{\partial x} {\partial s}
        +           \frac{\partial^2 x} {\partial s^2}
\,.
\ee{BelMos96-2}
Consider the diffusively coupled maps \refeq{BelMos96-1} for $\gamma=1$ and
$N=2m$ and assume periodic boundary conditions ($m\to\infty$ when the array
is unbounded). As this has a spatial Laplacian (the dynamics is now
reflection-symmetry invariant) they construct a pair of new variables
$x_{n,t}$, $y_{n,t}$ from $x_n$ with odd or even $n$, and obtain a 2D lattice
symmetric under $(x,y) \leftrightarrow (y,x)$. Then the symmetric and
asymmetric spaces behave differently. In particular, the second
period-doubling bifurcation for can be changed into a torus
bifurcation\rf{ReiMos95}.
    }

    \PCpost{2016-11-11}{
Belykh and Mosekilde\rf{BelMos96} is a descendent of the system previously
studied by Reick and Mosekilde\rf{ReiMos95} in {\em Emergence of
quasiperiodicity in symmetrically coupled, identical period-doubling
systems}: `` When two identical period-doubling systems are coupled
symmetrically, the period-doubling transition to chaos may be replaced by a
quasiperiodic transition. The reason for this is that at an early stage of
the period-doubling cascade, a Hopf bifurcation instead of a period-doubling
bifurcation occurs. Our main result is that the emergence of this Hopf
bifurcation is a generic phenomenon in symmetrically coupled, identica1
period-doubling systems. The whole phenomenon is stable against small
nonsymmetric perturbations. Our results cover maps and differential equations
of arbitrary dimension. As a consequence the Feigenbaum transition to chaos
in these coupled systems - which exists, but tends to be unstable - is
accompanied by an infinity of Hopf bifurcations.''

The seemingly first remark on this splitting is due to Ruelle\rf{Ruelle73NY}.
He noted that the symmetry of the coupling results in a splitting into a
symmetric solution and a nonsymmetric solution, where for the latter the two
subsystems are 180' out of phase. Later Neu\rf{Neu79,Neu79a} specialized the
problem to linear coupling and analyzed the stability of the symmetric and
nonsymmetric solutions by singular perturbation methods.

All of these might be of importance for Burak's torus birth and
destruction\rf{BudCvi15}.

As for the cats crowd, these seem to suggest that spacetime interchange
symmetry $t \leftrightarrow n$ of the {\catlatt}\rf{GHJSC16}
should lead to different dynamics (and symbolic dynamics! only positive
$(n,t)$ are needed after the decomposition) in symmetrized and
anti-symmetrized lattices, with a possible flow-invariant subspace currently
invisible. In Gibson, Halcrow and Cvitanovi{\'c}\rf{GHCW07} {\em Visualizing
the geometry of state-space in plane {Couette} flow} we also had a
$\Dn{1}\times\Dn{1}$ symmetry, with important consequences on types of
solutions, and invariant subspaces.
}

    \PCpost{2016-11-23}{
Having a variational principle (action \refeq{GutOsi15-3.1:model} in case at
hand) and a continuous symmetry means that the \textbf{Noether's theorem} applies

``To every one-parameter, continuous group of symmetries of a Lagrangian
dynamical system there corresponds a scalar, real-valued conserved quantity.''

Is there is a version of it for discrete translations? What is the conserved
quantity for a single cat map? What is it for the lattice? Internet says
many contradictory things:

``The fact that a Lagrangian is unchanged by a discrete transformation  is of
no significance. There is no conserved quantity associated with the
transformation.''

``For infinite symmetries like lattice translations the conserved quantity is
continuous, albeit a periodic one. So in such case momentum is conserved
modulo vectors in the reciprocal lattice. The conservation is local just as
in the case of continuous symmetries.''

Read about it
\HREF{http://physics.stackexchange.com/questions/8518/is-there-something-similar-to-noethers-theorem-for-discrete-symmetries}
{here}.

{Mansfield}\rf{Mansfield06} in
\HREF{https://www.kent.ac.uk/smsas/personal/elm2/liz/papers/focm.pdf}
{proceedings} and in her
\HREF{https://www.kent.ac.uk/smsas/personal/elm2/liz/papers/santander-land.pdf} {talk}
defines \emph{total difference} and says ``
Just as an integral of a total divergence depends only on the boundary data,
so does the sum over lattice domain of a total difference.''

She states the discrete {Noether's} Theorem, and in Example 1.3.7  she shows
that for a discretization of a standard mechanical Lagrangian, time
invariance3 yields ``energy'' as a the conserved quantity.

Hydon and Mansfield\rf{HydMan11}.

Capobianco and Toffoli\rf{CapTof11} {Can anything from {Noether's} Theorem be
salvaged for discrete dynamical systems?} is fun to read (but ultimately
unsatisfactory):

``
we take the Ising spin model with both
ferromagnetic and antiferromagnetic bonds. We show that-and why-
energy not only acts as a generator of the dynamics for this family of
systems, but is also conserved when the dynamics is time-invariant.''

The \emph{microcanonical Ising model} is strictly deterministic and
invertible: on a given step, a spin will flip (that is, reverse its
orientation) if and only if doing so will leave the sum of the potential
energies of the four surrounding bonds unchanged. The Ising dynamics is a
second-order recurrence relation. They define ``energy'' as the length of the
boundary between `up' and `down' domains. While the magnetization-number of
spins up minus number of spins down-may change with time, that length, and
thus the energy, remains constant. The total energy of a system may be
defined as
\begin{enumerate}
  \item A real-valued function of the system's state,
  \item that is additive,
  \item and is a generator of the dynamics.
\end{enumerate}
In a discrete Hamiltonian dynamics, a state is no longer a
``position/momentum'' pair $<q,p>$ as in the continuous case, but an ordered
pair of configurations $<q_0,q_1>$.

A second-order dynamical system
has an evolution rule of the form
\[
x_{t+1} = g(x_{t},x_{t-1})
\,.
\]
            }


    \PCpost{2016-12-10}{
Read Akila \etal\rf{AWGBG16} {\em Semiclassical identification of periodic
orbits in a quantum many-body system}
    }

    \PCpost{2016-12-12}{
{\bf \catLatt} claim retracted:

On the second thought, \catlatt\ is not radically different from the previous
examples of ($D$+1)\dmn\ {\spt} symbolic dynamics cited by Boris
(Pesin and Sinai\rf{PesSin88},
Bunimovich and Sinai\rf{BunSin88},
Pethel, Corron and
Bollt\rf{PetCorBol06,PetCorBol07}).
In all coupled maps literature, the starting point is time evolution of a
non-interacting particle at each site, with its 1\dmn\ temporal symbolic
dynamics, with spatial coupling to $D$ spatial neighbors subsequently tacked
on. While the \emph{coordinate} of site is ($D$+1)\dmn, its symbolic dynamics
is 1\dmn. In all models the I am aware of, with the exception of zeta
functions in $d=2$, the field being discretized (``particles'' at sites) is
\emph{scalar}, and so is the \catlatt\ field (I was wrong when I claimed in a
conversation with Boris that is a \emph{vector} field). In $d$ {\spt}
dimensions it is $d$\dmn, but its symbolic dynamics alphabet at each site is
$1$\dmn\ set.

The zeta functions in $d=2$, reviewed in \refchap{chap:zeta2D}, are not based
on any dynamics, and have problems of their own. In nutshell, for such zeta
functions grammars are awkward, while in the \catlatt\ the grammar rules are
automatically obeyed by any simulation, and can be explicitly computed as a
finite set of rules for any given \twot.
    }

    \PCpost{2018-12-05}{
\HREF{https://scholar.google.com/citations?hl=en&user=WhaV9vAAAAAJ&view_op=list_works&sortby=pubdate}
{Liverani}'s paper
Keller and Liverani\rf{KelLiv05} {\em Uniqueness of the {SRB} measure for
piecewise expanding weakly coupled map lattices in any dimension} is an example
of rigorous work on ergodicity of weakly couple chaotic maps. Cite, emphasize
weak-coupling is NOT what we do. Also, our Euclidian metric leads to
hyperbolicity, no attracting subsets and phase transitions in the evolution.
    }

\PCpost{2016-12-17}{ Read Qin\rf{Qin03} {\em Bifurcations of steady states
in a class of lattices of nonlinear discrete {Klein-Gordon} type with
double-quadratic on-site potential}
    }

    \PCpost{2017-02-17}{
For a single cat map linear code is simply awkward
\begin{enumerate}
  \item
depends on choice of $x_t$ origin: 4 letters for Boris, 5 for (symmetric) Vivaldi
  \item
linear code partition areas have no relation to \po\ weights
  \item
linear code pruning rules undercount \po\ pruning rules
  \item
\po s are intrinsic to the flow, so makes more sense to use AW Markov partition
\end{enumerate}
    }

\noindent{\bf Boris 2017-02-17}
Mostly agree, but linear code has its own charm: internal part of the code
has trivial grammar rules, it is very homogeneous, \ie, admissible sequences
of finite length have  volumes either close to zero or close to max (volumes
of internal sequences).

\noindent{\bf Boris 2017-02-17}
For $p$  iterations of the map situation rapidly improves - number of internal
symbols grows exponentially with $p$, number of external symbols is fixed (kind
of renormalization).
\\{\bf Predrag 2017-03-04} That's just a cheat.
For a piecewise linear expanding map of mod~1 variety, it is always the orbit
of the outermost critical point (in your language, exterior alphabet) that
defines the grammar, in the spirit of the kneading theory, with no grammar
rules for the interior alphabet, see ChaosBook chapter {\em Deterministic
diffusion}. If one looks at iterates of the map, of course the number of
interior alphabet sequences grows exponentially, but that does not mean that
they dominate the topological entropy - that ones is what it is, an
irrational number larger than $\ln N_{\Ai}$, the same number for the original
map or any of its iterates.


    \PCpost{2017-04-29}{
Tourigny\rf{Tourigny17}
{\em Networks of planar Hamiltonian systems}
writes things we probably disagree with:

``Hamiltonian systems famously do not admit attractors,  but rather families of
periodic orbits parameterised by level sets of the Hamiltonian.  There have
been surprisingly few attempts to study complex networks consisting of
coupled Hamiltonian systems.
Consequently, in this paper we initiate the study of complex networks whose
autonomous units are planar Hamiltonian systems.  We choose to focus on
planar Hamiltonian systems because already there exists a rich theory that
may be identified with the geometry of planar algebraic curves.

Define a weighted graph $\mathcal{G}$ by assigning
a vertex to each $i \in V$ and an edge with ``weight'' $W_{ij} = W(i,j)$
joining vertex $i$ to $j$. We assume this graph $\mathcal{G}$ is {\em
undirected} (i.e., $W_{ij} = W_{ji}$) and {\em connected} meaning that there
always exists a path from any vertex $i$ to any other vertex $j$. The network
is then defined to be the dynamical system on $\mathbb{R}^{m \times n}$ whose
time evolution is governed by the equations
\[ \dot{z}_i = F_i(z_i) + \sum_{j=1}^n W_{ij} U(z_i,z_j) \ . \]
We will actually only be concerned with
networks where the autonomous units are planar ($m = 2$) and the coupling
function is {\em diffusive}, implying $U(z_i,z_j) = U(z_i - z_j)$.
We say that a network is an {\em oscillator network} if
each of the underlying systems $\dot{z_i} = F_i(z_i)$ admits at least one
periodic solution.
''

This work is inspired by Winfree, Kuramoto and Turing (rather than coupled
maps lattices). If the dynamics on a node is the same everywhere, the network
(graph) is ``homogenous'', otherwise it is ``heterogenous.'' The diffusive
coupling $W_{ij}=W_{ji}$ couples all nodes, in his model with the same
strength, but the nodes are assumed heterogenous. After further reading it
seems that the networks are not Hamiltonian (coupling destroys that, and that
we can safely avoid this literature.

Rest in pacem.
    }

    \BGpost{2017-10-10}{
Our recent\rf{AGBWG18} {\em Semiclassical prediction of large
spectral fluctuations in interacting kicked spin chains}, \arXiv{1709.03601},
hides a feline creature in the form of \refeq{eq:po:deltaQint}:

\noindent
``{\bf Integrable Case}:
%\label{sec:po:int}
For  \(b^x\! = \! 0\) all rotations are around the \(z\)-axis and therefore
commute with the Hamiltonian making the system integrable. As a result the
dynamics of the kicked system for arbitrary times is equivalent to one at fixed
time, \eg \(T\! = \! 1\) with rescaled system parameters \(J\to JT\) and \(b^z\to
b^z T\). Moreover, the flow induced by the Hamiltonian $ \hat{H}_I+ \hat{H}_K$  for time
$T$ is identical to the evolution of the kicked system for \(T\) time steps
with the same parameters.

%Therefore, the momenta form the action variables of the integrable system
%restricting the angles \(\vec{q}\) to a \(L\) dimensional subspace.
For the classical trajectories $p_n=\text{const.}$ holds for each $n$ and
periodic orbits form \(L\) dimensional manifolds.
% All trajectories in this subspace are equivalent with respect to their
% dynamic properties. If for certain values of \(\vec{p}\) a periodic orbit
% exists it is therefore a \(L\) dimensional manifold.
To close a trajectory in phase-space after \(T\) iterations it is sufficient
that the total change in angles \(\Delta q_n\) is a multiple of \(2\pi\),
\beq
 \Delta q_n=4 T\left(J (p_{n-1}+p_{n+1})+ 2V p_n \right)+2b^z T=2\pi \hat{m}_n\,.
\ee{eq:po:deltaQint}
The \(\hat{m}_n\in\mathds{Z}\) is a local winding number for spin \(n\). Since the
momenta are bounded, \(|p_n|\leq 1\),   \(\chi_n\! = \! p_{n-1}+p_{n+1}\) resides
within the interval $[-2,+2]$. Therefore, this equation  has no solution if,
for instance, \(b^z>4(J\!+\!V )\) and \(4(J+V )+b^z<\pi/T\). In such cases
the system does not posses any classical periodic orbits of period \(T\) or
shorter. If all parameters (times \(T\)) are sufficiently small, the first
accessible winding number is necessarily zero. With increasing time $T$ the
number of possible \(\hat{m}_n\) grows linearly, and with it the number of possible
distinct periodic orbits grows algebraically.
% In contrast to the other cases time in the integrable case can be seen as a
% scaling factor of the system parameters, making arbitrary real times
% accessible.
With respect to \(L\) the number of periodic orbits is determined by all
admissible combinations of the winding numbers. If there is more than one
allowed \(m_n\) the growth is thus exponential in \(L\). This exponential
growth also holds for non-integrable parameter choices.
''

Boris continues:
This is the (dual) equation for ``spatially'' periodic orbits of an  integrable
spin chain.  For $4JT=2\pi$ it is the cat map
\beq
p_{n+1} - s\,p_n + p_{n-1}=-{m}_n
\,,
\ee{eq:po:deltaQint1}
with the cat map ``stretching'' parameter $s=-{2TV}/{\pi}$, and the winding number
$m_n= {b^z T}/{\pi} - \hat{m}_n \in \integers$.
For other times T, the map is non-symplectic but still linear. The equation
says, loudly, that  temporally integrable system can be ``spatially" chaotic
(e.g., number of spatial \po s grows exponentially with the spatial period $L$
(in case at hand, the number of ``particles'').

On the quantum side the corresponding ``dual" evolution is  exactly the quantum cat
map.
    }

    \PCpost{2017-10-10}{
Thanks for pointing it out, I would have never noticed that the very feline
\refeq{eq:po:deltaQint1} is hiding within \refeq{eq:po:deltaQint}. So, whenever
one has the nearest neighbor couplings, spatially, spatial translation
invariance, a spatial reflection symmetry, and a linear dynamics on the site,
one gets a spatial cat map. Your $\period{}=1$ example is the discrete time
analog of the Michelson\rf{Mks86} $\period{}=0$ study of \eqva\ on the \ks\
spatial $L\to\pm\infty$ (described in the parallel \texttt{blog.tex}, in this
subversion repo directory). This system is not integrable, but, needless to
say, the number of \eqva\ does not grow with increasing \period{} (it is
constant).

This continues with {\bf 2020-01-11 Predrag} post below.

    }

    \PCpost{2017-10-10}{
Currently open ends:
\begin{enumerate}
  \item
Finish the Pythagorean tiling of \reffig{fig:CatMapEigVecs} in terms of two
rectangles. In these coordinates the cat map should again have a linear
encoding, but with no pruning rules (the two rectangles are a generating
partition).
  \item
Do the same for $d>1$.
  \item
Redo everything with doubly-periodic boundary conditions, where
all Green's functions trivial.
  \item
I have started writing up the {\spt} Jacobian and the $\det(1-J)$ for
$d$-tori in what is currently called Sect.~1.3.2 {\em Spatiotemporal stability}
in the parallel \texttt{blog.tex}. In the discrete spacetime case, that should
agree with the determinants that Boris has computed.
\end{enumerate}
    }

    \PCpost{2018-01-20}{
Akila, Waltner, Gutkin, Braun and Guhr\rf{AWGBG17}
{\em Collectivity and periodic orbits in a chain of interacting, kicked spins}
is presumably closely related to
\refref{AGBWG18} {\em Semiclassical prediction of large
spectral fluctuations in interacting kicked spin chains},
see {\bf 2017-10-10 Boris} post above.
    }

    \PCpost{2018-01-20}{
Catnipping is taking a delirious turn.
Axenides, Floratos and Nicolis\rf{AxFlNi17}
{\em The quantum cat map on the modular discretization of extremal black hole horizons}
write: ``
We present a toy model for the chaotic unitary evolution of infalling
black hole horizon single
particle wave packets. We construct explicitly the eigenstates and eigenvalues
for the single particle dynamics for an observer falling into the BH horizon,
with time evolution operator the quantum Arnol'd cat map (QACM). Using these
results we investigate the validity of the eigenstate thermalization hypothesis
(ETH), as well as that of the fast scrambling time bound (STB). We find that
the QACM, while possessing a linear spectrum, has eigenstates which are random
and satisfy the assumptions of the ETH. We also find that the thermalization of
infalling wave packets in this particular model is exponentially fast, thereby
saturating the STB, under the constraint that the finite dimension of the
single--particle Hilbert space takes values in the set of Fibonacci integers.
''
}

\PCpost{2018-03-28}{Plasma physicists
Xiao \etal\rf{XQSLZ17} {\em A lattice {Maxwell} system with discrete
space-time symmetry and local energy-momentum conservation},
\arXiv{1709.09593} might of interest to us. By `Maxwell' they mean EM on a
discrete spacetime lattice. Their system has gauge symmetry, symplectic
structure and discrete space-time symmetry. They generalized Noether's
theorem  to discrete symmetries for the lattice Maxwell system, and the
system is shown to admit a discrete local energy-\-momentum conservation law
corresponding to the discrete space-\-time symmetry. These conservation laws
make the discrete system an effective algorithm for numerically solving the
governing differential equations on continuous space-time.

There is also Stern \etal\rf{STDM15} {\em Geometric computational
electrodynamics with variational integrators and discrete differential
forms},
They write: ``
We develop a structure-preserving discretization of the Lagrangian framework
for electrodynamics, combining the techniques of variational integrators and
discrete differential forms. This leads to a general family of variational,
multisymplectic numerical methods for solving Maxwell's equations that
automatically preserve key symmetries and invariants. We generalize the Yee
scheme to unstructured meshes in 4-dimensional spacetime, which relaxes the
need to take uniform time steps or even to have a preferred time coordinate.
We introduce a new asynchronous variational integrator (AVI) for solving
Maxwell's equations. These results are illustrated with some prototype
simulations that show excellent numerical behavior and absence of spurious
modes.
''
}

\BGpost{2018-04-05}{
You show that your \AW\ (Markov) partition can be used to
get position of the points in the phase space by linear transformation.
So your partition is good on both fronts = simple grammar rules + easy walk  from
phase space to symbolic representation and back (as opposed to the linear
code of \PV, %/GHJSC,
where only the second part is true).

Q: Was it known already (I mean simple connection between phase space and
symbolic dynamics for \AW\ partitions)?
}

\PCpost{2018-04-08}{
The generating partition of \reffig{fig:PVAdlerWeiss2HL} is new.

Newer still is the generating partition of
\reffig{fig:HL7-rectanglePartition}\,(a) which incorporates the invariance of
the cat map dynamics \refeq{eq:CatMapNewton5} under spatial reflection flip
across the $\ssp_1=-\ssp_0$ anti-diagonal, together with $\Ssym{n} \to
-\Ssym{n}$ time reversal symmetry, so the full symmetry is
$\Dn{1}\times\Dn{1}$. This should lead to very simple symbolic dynamics, but
\refsect{sect:fundDomHL}~{\em Reduction to the fundamental domain} has still
to be completed.
}

\BGpost{2018-04-05}{
Start  from e.g., 5 letter alphabet, then translate admissible sequences
(from the transition  graph) into tree letter (-1,0,1) LINEAR alphabet
\refeq{catAlphabetPVAW},
and use Green's function  to get to the phase space points. Right?
}

\PCpost{2018-04-08}{
As explained in ChaosBook.org, in symmetry reduction the \emph{links} of
such graphs are not labelled arbitrarily, they have a precise
group-theoretic meaning.

\refFig{fig:PVAdlerWeiss2HL}\,(d) indeed has 5 links (I prefer not use an
arbitrary `alphabet' to label them), but the important thing is that the
links correspond to the symmetry group elements; they enable you to
reconstruct from a walk on the torus the corresponding walk in the space
tiled by the copies of the torus. In the symmetry reduction of dynamical
systems we call that ``reconstruction equations''. It is these group
elements \refeq{catAlph-s-PVAW}, \refeq{fiveLett}, \etc, that are the
\emph{alphabet}, as explained in %ChaosBook
\wwwcb{paper.shtml\#diffusion}  {diffusion chapter},
and the preceding chapters that deal with discrete symmetry reduction
(like the 3-disk billiard).

To repeat: we are reducing translational symmetry. Hence the alphabet
are the group elements that reduce global motions to the elementary cell,
in this case our 2-rectangle partition.
}

\BGpost{2018-04-05}{
If linear refers to linear connection between symbol sequences and
phase space points (my definition) then  your three letter code is indeed
linear, but (probably) not the original 5-letter one.
}

\PCpost{2018-04-08}{
What code? There is no ``original 5-letter'' code... Maybe you mean link
labels in the \refeq{2rectDetTransfMatr} calculation? That just looks
like clumsy notation to me. Writing down a \markGraph\ with 5 nodes
corresponding to such `alphabet' would be a mess, I did not do it.
}

\BGpost{2018-04-05}{
I would say that under any \AW\ looms a linear code
}

\PCpost{2018-04-08}{I think so too.
While the canonical Thom-Arnol'd cat map \refeq{ArnoldCat} and the
\PV\rf{PerViv} two-configuration representation
\refeq{PerViv:2confRepMat} are related by the anti-symplectic
transformation \refeq{HLsimilarityC}, the Thom-Arnol'd cat map seems
clumsier, as it seems to require 2D translations in \reffig{fig:Lect13p8}
to return subpartitions back into the 2-rectangle partition (have not
bothered to work that out). Still a linear code.
}

\BGpost{2018-04-05}{
\AW\ code is not linear on its own.  Agree?
}

\PCpost{2018-04-08}{
I agree - I have never seen it recast in the group-theoretic symbolic
dynamics. \PV\rf{PerViv} do discuss \AW, and do
write down the 1D (damped) {\sPe} and 1D Green's functions, but they never
saw that \AW\ could be combined with their linear code. Would sure
have saved a ton of needless work in the subsequent literature.
}

\BGpost{2018-04-05}{
The biggest question to me what are you doing/going to do in $d=2$ case. For
me, \AW\ partitions can be (probably) turned to 1D linear code
(with a huge alphabet), but not to the 2D linear code. Right, Wrong?}

\PCpost{2018-04-08}{Wrong.
Why would one turn $d=2$ \AW\ into 1D linear code? For finite
spatial periodicity $L$ that would be stupid, and for $L\to\infty$ (the
subject of these papers) impossible. No, for $d=2$ one has to partition
the 4D \twot\ partition hypercube. I'm optimistic about us being able to
accomplish that. Symmetries should help. The stupid
\PV\rf{PerViv} hypercube is already working (see Han's
examples \refeq{HL2DimensionOrbit2}, \etc), now we have to find a
generating partition.

Just get cracking :)
}

\PCpost{2017-12-18}{
More seriously, what is wrong with the argument so far?

I used the Hamiltonian, evolution-in-time thinking to generate the $d=1$
generating partition. That will not work in higher dimensions, so the
above argument has to be recast in the Lagrangian form.

In principle, it is just a discrete Legendre transform,
see \refsect{LiTom17b:GenFctn}~{\em Generating functions; action},
but I do not see it yet...
}

\BGpost{2018-04-12}{
By the ``original 5-letter" code I mean the  one which appears  in
\refeq{catAlphabetPVAW2HL}. %[2018-02-11 Predrag].
You can use it to label paths on the transition graph = all admissible
sequences/trajectories. The grammar rules for this encoding/ labeling (call it
whatever) are local - you just say which pairs of symbols are admissible. In
other words if I give you  an arbitrary  sequence of symbols (from this five
letter alphabet) you can easily say me whether it admissible or not. Right?
}
\PCpost{2018-04-19}{
If by ``local'' you mean ``full shift'', \ie, \markGraph\ of memory 1, wrong.
They are walks on the \markGraph\ \refeq{fig:PVAdlerWeissC}\,(e), so 2 cannot
be followed by 1 or 3, 5 cannot be followed by 4, etc. But any \markGraph\ of
\emph{finite} memory encodes a finite grammar (subshift of finite type)- no
need to recode it as a memory 1 code.
}

\BGpost{2018-04-12}{
But what about your 3 letter code \refeq{threeLett}? Given  a  sequence of symbols from
this 3-letter alphabet, can you say easily  whether it admissible or not
(without going to the phase space)?  In other words, whether the grammar rules
are local here?
}

\PCpost{2018-04-19}{
Don't you see how brilliant our solution is?
\begin{enumerate}
  \item
Any walk \Mm\ on the \markGraph\
\reffig{fig:PVAdlerWeiss2HL}\,(d) is admissible
  \item
By linearity of the (damped) {\sPe}, to each admissible \brick\ \Mm\
corresponds a unique admissible state \Xx.
\end{enumerate}
Done.
}

\PCpost{2018-05-14}{
There is some
\HREF{https://www.quantamagazine.org/a-chemist-shines-light-on-a-surprising-prime-number-pattern-20180514/}
{excitement} about new regularities of Bragg peaks in diffraction patterns of
1\dmn\ aperiodic crystal composed of primes. I find the limit periodic
structures, such as the spatial period-doubling\rf{BaaGri11} particularly
intriguing (see Fig.~6.1 in Torquato, Zhang and De Courcy-Ireland,
\arXiv{1804.06279}).
}

\PCpost{2018-10-06}{
Ruelle\rf{Ruelle99} {\em Zeros of graph-counting polynomials}, and
Ruelle\rf{Ruelle18} {\em Graph-counting polynomials for oriented graphs}
deal with things like counting subgraphs, motivated by studies of zeros of the
grand partition function, such the circle theorem of Lee and Yang. I truly do
not understand why this is done... I am not alone, as very few people cite
these articles.
    }

\PCpost{2018-11-05}{
\HREF{http://seb199.me.vt.edu/mpaul/}
{Mark Paul} and Jonathan Barbish are testing their \cLv\ codes on the
diffusive coupled map lattices (CML) \refeq{KanekoCML} of
Kaneko\rf{Kaneko83,Kaneko84} type on finite spatial periodic lattice of
periodi $\speriod{}$,
\beq
u^{n}_{j+1}=f(u^{n}_{j}) +
            [
            f(u^{n-1}_{j}) - 2f(u^{n}_{j}) + f(u^{n+1}_{j})
            ]
          = (1+\epsilon\,\Box) f(u^{n}_{t})
\,,
\ee{GriCro97(1)}
where the individual site dynamical system $f(x)$ is the 1D
quadratic map
\beq
f(x) = ax + b z(1-z), z = x \mbox{ mod }1
\,,
\ee{GriCro97(2)}
studied in
Grigoriev and Cross\rf{GriCro97}
{\em Dynamics of coupled maps with a conservation law}.

A differential or discrete equation of this form represents the
competition between: generation of chaotic perturbations by $f(x)$ and
their dissipation the diffusive coupling induced by the spatial
Laplacian.

For map \refeq{GriCro97(1)} the space average of the `velocity field' $u$
\beq
\spaceAver{u}
    =
\frac{1}{\speriod{}} \sum_{j} u^{n}_{j}
\ee{GriCro97(4)}
is conserved (Galilean invariance), so the system has, in addition to the
two parameters $a$ and $b$ of the local map $f$, as the additional
control parameter the conserved quantity $\spaceAver{u}$, which can be
defined by the initial condition. The give plots of possible phases in
the $[a,u]$ and $[b,u]$ parameter planes.

Their objectives is to determine the effect of this conservation law on
the dynamics of the extended chaotic system; it leads to the singularity
in Lyapunov spectrum at $\lambda_j =0$, shown in their fig.~10\,(b). They
break the conservation law in their eq.~(70). In \KS\ we do not seem to
get anything much out of it, as far as I remember, but they comment on
that in their eq.~(18).

In their eqs.~(10) and (13) they give analytically the two 2-cycles. As
is the case for parabola, it might be possible to give 3-cycles
analytically, but nothing beyond.

They state that there are non-linear waves (we call these \reqva\ or
travelling waves), propagating trough the laminar background with unit
velocity, their fig.~(3): ``All defects move with a constant velocity,
but while the majority of defects is moving with the maximal speed $v=\pm
1$, the rest have a smaller speed.'' They neither derive nor prove
existence of \reqva\ (travelling defects) with $|v|\leq1$, nor do they
study their stability.

If I were Jonathan,
I would first look for \eqva\ $u^n=u^{n+1}$ for small $\speriod{}$, then
for relative equilibria $u^n=u^{n+q}+p$. Show that phase velocity is
maximal for $v=1,\;(q=p)$. Determine their linear stability – perhaps
$v=1$ is the least unstable, that's why you see it all the time. The go
on finding unstable periodic and \reqva\ for small
$(\speriod{},\period{})$ tori (periodic tilings).
    }

\PCpost{2019-09-28}{
\HREF{https://www.hit.ac.il/en/faculty_staff/Boris_Gutkin} {Boris} gets a
\HREF{https://www.hit.ac.il/en/news/news-and-stories/Boris_Gutkin_grant}
{prestigious grant}.
}

\BGpost{2019-11-18}{
There are some exciting developments during the last two years regarding
quantum dual models, mostly by the group of Toma\v{s} Prosen, see
Bertini, Kos, and Prosen\rf{BeKoPr19-4} {\em Exact correlation functions
for dual-unitary lattice models in 1+1 dimensions}, \arXiv{1904.02140}.

Although they go by some funny names - dual-unitary  spin chains, dual
local quantum circuits etc., behind each of these model sits a
(quantized) {\catlatt}. Amazingly, we know how to calculate correlations
between local operators in such models resp. in quantum {\catlatt} (even
in perturbed one).

This  generates some funny questions regarding classical cats as well.
One should be able to do it also for the classical model as well. In
short, if you look for an interesting problem  -- calculating classical
correlators in {\catlatt}s  might be the hit.

In a  week I will be able to send you a Duisburg preprint (with setting
closer  to cats).
    }
\PCpost{2020-01-11}{
This continues the conversation that includes \refeq{eq:po:deltaQint}
above, but I do not understand the details - some of the quantum notation
obscures them for me.

No further peep from Boris, but this is presumably
the Duisburg preprint:\\
Boris Gutkin, Petr Braun, Maram Akila, Daniel Waltner and Thomas Guhr,
{\em Local correlations in dual-unitary kicked chains},
\arXiv{2001.01298}. \\
Turns out Boris knows how to upload to arXiv, but was too
busy to do it with our preprint :)
}

\item[2020-12-16 Predrag]
My 2000 {\em {Chaotic Field Theory}: {A} sketch}\rf{CFTsketch} gets cited
every so often. Most citations seems useless, with the exception of these
two:

Stam Nicolis \HREF{https://doi.org/10.1134/S1547477120050295}
{{\em Supersymmetry and Deterministic Chaos}} (2020):
We show that the fluctuations of the periodic orbits of deterministically
chaotic systems can be captured by supersymmetry, in the sense that they
are repackaged in the contribution of the absolute value of the
determinant of the noise fields, defined by the equations of motion.

[...] In a chaotic phase there are infinitely many periodic orbits and
there have been attempts to use them to construct the measure they
define, using perturbative field theoretic techniques\rf{CFTsketch} (that's me).
[... He]  discusses another way to address this issue, that does not rely
on perturbation theory,
following the approach of his earlier papers, he writes a \emph{lattice action}
and computes the identities that the correlation functions that the
noise fields would be expected to satisfy, were the system consistently
closed.

Bernd M\"umken                                              \toCB
\HREF{https://doi.org/10.1007/s00220-008-0448-y}
{{\em A Dynamical Zeta Function for Pseudo Riemannian Foliations}}
(2006):
``We investigate a generalization of geodesic random walks to pseudo
Riemannian foliations. The main application we have in mind is to
consider the logarithm of the associated zeta function as grand canonical
partition function in a theory unifying aspects of general relativity,
quantum mechanics and dynamical systems.''

It looks familiar in glimpses, but the math is killing me...



\PCpost{2020-01-13}{                                        \toCB
MacKay, Johnson and Sansom\rf{MaJoSa20}
{\em How directed is a directed network?},
\CBlibrary{MaJoSa20}:

We consider directed networks (also known as directed graphs or digraphs)
with set N of nodes (also known as vertices) and set E of directed edges
(also known as links). We suppose that there is at most one edge from a
node m to a node n, and denote the edge by mn. There can also be an edge
from n to m. Each edge carries a weight $w_{mn} > 0$. This can represent
the strength of the edge. We write wmn = 0 if there is no edge from m to
n and we assemble the wmn into a matrix W. The edge weights could be set
to 1, as is common in the literature, and the array W is then called the
adjacency matrix A of the network, but the ability to represent the
strength of the edge is a useful extension. If there were multiple edges
from m to n then we would amalgamate them into a single edge by adding
the weights. Self-edges mm (also called loops) are permitted.

The \emph{weight} of the node n is
\[
  u_n = w_n^{in}+ w_n^{out}
\]
and the \emph{imbalance} for node n by
\[
  v_n = w_n^{in}-w_n^{out}
\]
In matrix form the (weighted) graph-Laplacian operator $\Lambda$ on
functions $h:N\to\reals$ is defined by
\[
  \Lambda = \mbox{diag}(u)-W-\transp{W}
\]
They seek the solution h of the linear system of equations
\[
  \Lambda h = v
\]
This always has a solution but it is non-unique, because one can add an
arbitrary constant in each connected component of the network. Etc.

They seek levels to minimise the \emph{trophic confusion} :)

A directed network is said to be normal if its
weight matrix ${W}$ commutes with its transpose $\transp{W}$ :
\[
  W\transp{W} = \transp{W}W
\]
Note that $\transp{W}$ represents the same weighted network but
with all the edges reversed.
                                                \toCB
The term ``normal'' came
from people who spent their lives with self-adjoint operators
and unitary operators, both of which are normal, but people
working in stability of ordinary differential equations are fully
cognizant that most matrices are not normal.

For the unweighted case of an adjacency matrix A, normality
implies the imbalance vector v = 0. This is because
$(\transp{A}A)_{mn}$ is the number of sources in common to nodes m and
n, and $(A\transp{A})_{mn}$ is the number of sinks in common. In particular,
$(\transp{A}A)_{nn}= w_n^{in}$
and $(A\transp{A})_{nn}= w_n^{out}$, so $\transp{A}A=A\transp{A}$
implies that $w^{in} = w^{out}$ and v = 0.
When v = 0 we say a network is balanced.

Another special case of normality is symmetric networks $ \transp{W}=W$.

Normality of ${W}$ is equivalent to existence of a unitary
matrix U such that $U^\dagger{W}U$ is diagonal.

A cycle in a directed network is a closed walk in it. We allow repeated
edges and repeated nodes. In particular, we allow a cycle to be a
periodic repetition of a shorter cycle. The weight w of a cycle is the
product of the weights along its edges.
The total weight of cycles of length p is given by the
trace of the $p$th power of W: $\tr{W}^p$.

                                                \toCB
The {\orbit}s are those which are
not repetitions of a shorter cycle. We consider two {\orbit}s to
be the same if they differ only by a cyclic permutation. We denote
by ${\cal P}$ the set of {\orbit}s,
\[
  \zetatop(z) = \prod_p (1-z^\cl{p}w_p)
\]
This can be reduced to one in terms of \PCedit{``elementary cycles''},
those which do not repeat a node before closing. They are prime
and for a finite network there are only finitely many of them. The
formula is
\[
  \zetatop(z) = 1 + \sum_C\prod_{p\in C} (-z^\cl{p}w_p)
\]
where the sum is over non-empty collections C of disjoint elementary
cycles. Might want to study MacKay's proof to see if it is better than
what is in ChaosBook.
}

\item[2020-06-30 Moshe Rozali]
{\em Effective Field Theory for Chaotic CFTs}
``    \toVideo{youtu.be/zkOihsqBtGI}
Relations between chaos and hydrodynamics are one of the unique feature
of holographic CFTs. The early time Lyapunov regime can described by an
effective field theory of a single mode, which for maximally chaotic
systems is an hydrodynamic mode. We describe that effective field theory
for conformal field theories, both in two dimensions and in higher
dimensions, and show how it captures maximal chaos and pole skipping. We
discuss the relation of the theory to other formulations of CFTs and show
how it captures interesting objects such as conformal blocks and partial
waves. We speculate on what is needed to extend the discussion to non
maximal chaos.
''

\arXiv{1712.04963};
\arXiv{1808.02898}; %https://doi.org/10.1007/JHEP10(2018)118
\arXiv{1909.05847}

\arXiv{1612.06330};
\arXiv{1811.09641};
\arXiv{1812.10073};\\
G.J. Turiaci \arXiv{1901.04360};
\arXiv{1912.02810}

\item[2020-07-02 Ruairí Brett]
{\em From two to three-body systems in lattice QCD}
% link to the recording here: http://ctp.lns.mit.edu/latticecolloq/

Problem: finite box has no continuous spectum, scattering.
\arXiv{1911.09047} 2-body scattering: quantization condition is given by
the L\"uscher formula, stated as a determinant. \arXiv{1707.05817}
implements is with group theory, octahedral $O_h$ crystallographic irreps
for a cubic box which mix some the continuous limit $\On{4}$. They also
compute on elongated boxes, with different discrete symmetry. These
9elongated in the $z$, not the temporal $t$ direction) yields many more
states.

\arXiv{1901.00483}

relation to relativistic formulation
\arXiv{1905.12007}

\arXiv{1709.08222}
2-body scattering is a sub-calculation in the 3-body scattering.

``Wrap-around effects'' arise from finite temporal size of the lattice.

The cleanest example is $\pi^+\pi^+\pi^+$ ellastic scattering. Lattice
simulation data are surprisingly sharp. They are close to 3
non-interacting pions. The agreement with the determinant zeros (infinite
volume limit) using only 2-body scattering date, no fit are very sharp.
So the 3-body contact term will be small (they are working on that now)

The latest 3-pion formalism: \arXiv{2003.10974}

\item[2020-11-22 Uzy]
To dig out  the ``Levit-Smilansky theorem'' you had to be a real
archaeologist. A more profound representation is in the subsequent paper,
about the phase space representation of the path integral,
Levit and Smilansky\rf{LevSmi77b} {\em The {Hamiltonian} path integrals
and the uniform semiclassical approximations for the propagator}, (1977);
cited over 50 times.
% Muratore-Ginanneschi https://doi.org/10.1016/S0370-1573(03)00212-6
% A.G Basile and C.G Gray
% {\em A relaxation algorithm for classical paths as a function of end points:
% Application to the semiclassical propagator for far-from-caustic and
% near-caustic conditions} %   https://doi.org/10.1016/0021-9991(92)90044-Y
The stability operator turns out to be a Dirac like (and not
Schr\"odinger like) operator, with spectrum which covers the entire line.
Yet, we wrote a similar expression  for its regularized determinant and
the Maslov index turns out to be the excess of negative eigenvalues. The
phase space representation is of dimension (1+1)  (not in your sense) but
still it might be related by some miracle/good reason???

\item[2020-11-22 Predrag]
Scher, Smith and Baranger\rf{ScSmBa80} {\em Numerical calculations in
elementary quantum mechanics using {Feynman} path integrals} (1980): ``
We show that it is possible to do numerical calculations in elementary
quantum mechanics using Feynman path integrals. Our method involves
discretizing both time and space, and summing paths through matrix
multiplication. We give numerical results for various one-dimensional
potentials. The calculations of energy levels and wave functions take
approximately 100 times longer than with standard methods, but there are
other problems for which such an approach should be more efficient.
''

\item[2020-11-23 Predrag]
Could it be that in the large $[\speriod{}\times\period{}]$ limit
we should be looking at Floquet of `Brillouin' bands?

\item[2020-11-23 Predrag]
\tempLatt\ satisfies 1-d Klein-Gordon equation. Does it meant there is
a Gaussian solution on it? Probably not, as the usual situation has
a Gaussian spreading in time, but here time is taken to $\infty$...

\item[2020-12-16 Predrag] Not exactly `Gaussian', but maybe the outgoing
Klein–Gordon Green's function \refeq{GreenFunContinuesPC1} or the`Yukawa
potential' \refeq{GreenFunYukawa} maybe answer the above random thought.

    \PCpost{2017-09-16} {
We know the exact, {\AW} answer, in terms of a golden mean - state it.
The half-baked linear {encoding} is an arbitrary, infinite sequence of
approximations to the exact answer.
    }

    \PCpost{2021-02-01} {
\HREF{https://twitter.com/gabrielpeyre/status/1356120095030841347}
{Gabriel Peyr\'e}: The Fiedler vector of a graph is the second eigenvector
of the Laplacian. It is the lowest Fourier mode and is useful to
order the nodes (clustering, dimensionality reduction, etc).

\HREF{https://en.wikipedia.org/wiki/Algebraic_connectivity} {Wiki}.
    }

\item[2021-09-23 Martin Richter] <martin.richter@nottingham.ac.uk>

Can you give a presentation here in Nottingham? The subject of Fritz
Haake talk would be great. However, I would like to hear the second half
in only 1/2-Toma\v{z}-Prosen speed :-)

\item[2021-10-11 Predrag]

A half? That would kill you, my talk was at 1.0 deciProsen. You are
pleading for milliProsens, I assume. I make no promises in that
spatiotemporal direction.

I would like to give a technical talk, because you guys have a lot of
experience with Laplacians, and I'm puzzled.

\HREF{https://calendar.google.com/calendar/embed?src=cvitanov@gmail.com\&ctz=America/Chicago}
{My schedule}
is pretty flexible, and I do not mind getting up at 1:30am (for the
pleasure of being slapped around by Toma\v{z} Prosen, to give a random
example. Or Bogomolny).

For the people who would rather just be entertained, I would suggest
listening to one of the talks on
\HREF{http://ChaosBook.org/overheads/spatiotemporal/} {overheads/Spatiotemporal}.
Thanks to my graduate students (yet another level of
paper-writing reluctant) there are no papers to read :-(

\item[2021-09-23 Martin]
A question: Is there an intuitive reason why the Klein-Gordon equation
shows up? Why not something first order, for example? Either like the
eikonal equation or better something linear like Dirac? There is probably
a very simple answer to this but I was just wondering. Is there a
geometrical reason, for example? Nothing of urgency, otherwise I would
have shouted earlier.

\item[2021-10-11 Predrag]
That will precisely be the technical part: Klein-Gordon shows up because
of the nearest neighbor coupling, and it has to be second order because
of the reflection symmetry, so Laplacian. So far, we have only looked at
the Euclidean version, to keep things as simple as possible.  Yes, we can
take the "square root" of Laplacian and get two 1st order, time-asymmetric
equations and a factorized Hill determinant, but I am not sure what to
make of it. So far the field theory is only a scalar, single field
component per site. I'm very reluctant to get into Dirac spin-1/2 fields at
this time, because of lattice no-go-theorems, and the usual fermion
nonsense in lattice formulations…








    \PCpost{2020-02-02}{
\HREF{https://graphics.reuters.com/USA-ELECTION-IOWA/0100B5BX3CX/index.html}
{Cata at work}. Can we weave this into our presentations?
    }

    \PCpost{2017-09-30}{
\HREF{http://comms.iop.org/q/17ExylSkWoUvn9LVkqCiGb/wv} {Cats rule}.
    }

\end{description}


%%%%%%%%%%%%%%%%%%%%%%%%%%%%%%%%%%%%%%%%%%%%%%%%%%%%%%%%%%%%%%%%%%%%%%%
\printbibliography[heading=subbibintoc,title={References}]
