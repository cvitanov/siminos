% siminos/reversal/cat.tex      pdflatex LC21; bibtex LC21
% temporary: siminos/spatiotemp/chapter/LC21cat.tex
% $Author: predrag $ $Date: 2021-12-24 01:25:20 -0500 (Fri, 24 Dec 2021) $

\section{A kicked rotor}
\label{s:kickRot}

Temporal Bernoulli is perhaps the simplest chaotic lattice field
theory, but our goal here is to formulate a deterministic {\spt}ly chaotic
field theory, Hamiltonian and energy conserving, because (a) that is
physics, and (b) one cannot do quantum theory without it. We need a
system as simple as the Bernoulli map, but mechanical. So, we move on
from running in circles, to a mechanical rotor to kick.

The 1-degree of freedom maps that describe kicked rotors
subject to discrete time sequences of angle-dependent impulses
$V'(\coord_{\zeit})$, $\zeit\in\integers$,
\bea
\coord_{\zeit+1} - \coord_{\zeit} &=& p_{\zeit+1} \qquad  (\mbox{mod}\;1),
    \label{LC21PerViv2.1b}\\
p_{\zeit+1} - p_{\zeit} &=& V'(\coord_{\zeit})
\,,
    \label{LC21PerViv2.1a}
\eea
with $2\pi \coord$ the  angle of the rotor, $p$ the momentum conjugate to
the angular coordinate $\coord$, and the angular pulse
$V'(\coord)=V'(\coord+1)$ periodic with period $1$, play a key role in the
theory of deterministic and quantum chaos in  atomic physics, from the
Taylor, Chirikov and Greene  standard map\rf{Lichtenberg92,Chirikov79},
to the cat maps that we turn to now. The equations are of the
Hamiltonian form: eq.~\refeq{LC21PerViv2.1b} is $\dot{\coord}=p/m$ in terms of
discrete time derivative \refeq{lattTimeDer}, \ie, the configuration
trajectory starting at $\coord_{\zeit}$ reaches
$\coord_{\zeit+1}=\coord_{\zeit}+p_{\zeit+1}\Delta{\zeit}/m$
in one time step $\Delta{\zeit}$. Eq.~\refeq{LC21PerViv2.1a} is the
time-discretized $\dot{p}=-\partial V(\coord)/\partial \coord$: at each
kick the angular momentum $p_{\zeit}$ is accelerated to $p_{\zeit+1}$ by
the force pulse $V'(\coord_{\zeit})\Delta{\zeit}$, with the time step and
the rotor mass set to $\Delta{\zeit}=1$,  $m=1$.

In atomic physics applications, the values of the angle $\coord$
differing by integers are identified, but the momentum $p$ is unbounded.
In dynamical systems theory one compactifies the momentum as well, by
adding $(\mbox{mod}\;1)$ to \refeq{LC21PerViv2.1a}, as for the Bernoulli map
\refeq{n-tuplingMap}. This reduces the phase space to a square
$[0,1)\times [0,1)$ of unit area, with the opposite edges identified,
\ie, 2-torus.

%\section{Life of a single Hamiltonian cat}
\subsection{Cat map}
%    \fi
\label{s:catPV}

The simplest kicked rotor is subject to a force is proportional to
displacement, that is, Hooke's law force $P(\coord)=K\coord$ linear in
the angular displacement $\coord$.

When the force is proportional to displacement, that is, when Hooke's law
is obeyed, the spring is said to be linear, the potential is quadratic.

The $(\mbox{mod}\;1)$ added to
\refeq{LC21PerViv2.1a} makes the map a discontinuous `sawtooth,' unless $K$
is an integer. In the integer $K$ case, the map
(\ref{LC21PerViv2.1b},\ref{LC21PerViv2.1a}) is of form
 \beq
 \left(\begin{array}{c}
 \coord_{\zeit+1}  \\
   p_{\zeit+1}
  \end{array} \right )=
  \jMat \left(\begin{array}{c}
 \coord_{\zeit}  \\
   p_{\zeit}
  \end{array} \right )\quad (\mbox{mod}\;1)
    \,,  \qquad
 {\jMat} =\left(\begin{array}{cc}
 a & c \\
 d & b
  \end{array} \right)
\,,
\ee{catMap}
where $a,b,c,d$ are integers whose precise values do not matter, as long
as $\det \jMat=1$, \ie, the map is area-preserving. The map is then a
Continuous Automorphism of the Torus known as the Thom-Anosov
diffeomorphism, or Thom-Anosov-Arnol'd-Sinai {\em `cat
map'}\rf{ArnAve,deva87,StOtWt06}, extensively
studied as the simplest example of a chaotic Hamiltonian system.

{Cat map} is
a linear
area-preserving map on the unit 2-torus \statesp, with the field at the
temporal lattice site \zeit,
\(
\ssp_{\zeit} =(\coord_{\zeit},p_{\zeit}) \in  (0,1]\times(0,1]
\)
interpreted as the angular position and its conjugate momentum at time
instant $\zeit$.  For $|s|>2$ the one-time-step {\jacobianM} $\jMat$
{characteristic equation}
\beq
\ExpaEig^{2} - {s}\ExpaEig + 1 = 0
\,,
\ee{LC21:StabMtlpr}
has real roots
$(\ExpaEig\,,\;\ExpaEig^{-1})$  and a positive Lyapunov exponent
$\Lyap >0$,
\beq
\ExpaEig=e^{\Lyap} = \frac{1}{2}(s+\sqrt{(s-2)(s+2)})
\,,\qquad
s=\tr{\jMat}=\ExpaEig+\ExpaEig^{-1}
\,.
\ee{StabMtlpr}
A matrix $\jMat$ with no eigenvalue on the unit circle is called
hyperbolic.
The eigenvalues are functions of the stretching parameter $s$, and
for $|s| > 2$ the cat map \refeq{catMap} is a fully chaotic
Hamiltonian dynamical system. Cat maps with the same $s$ are equivalent
up to a similarity transformation, so it suffices to work out a single
convenient realization, as we shall do here for the
\PV\rf{PerViv} `two-configuration representation'
\refeq{eq:StateSpCatMap}.

\subsection{\tempLatt}
\label{s:catLagrange}
    % earlier names:
    % \section{Life of a single Lagrangian cat}
    % \section{Cat map in Lagrangian formulation}
\renewcommand{\period}[1]{{\ensuremath{n_{#1}}}}
    % discrete length of a cycle, Predrag

To motivate our formulation of a \spt\ chaotic field theory to be
developed in \refref{CL18}, we again recast the local initial
value, time-evolution cat \emph{map} \refeq{catMap} as a global
\emph{temporal lattice} condition that we shall refer to as the
{\em \templatt}.

Ignoring (mod~1) for a moment, we can use
\refeq{LC21PerViv2.1b} to eliminate
$p_{\zeit}$ from \refeq{LC21PerViv2.1a}
and rewrite the kicked rotor equation as the 3-term recurrence
\beq
  - \coord_{\zeit+1} + 2\,\coord_{\zeit} - \coord_{\zeit-1} + V'(\coord_{\zeit})
  = 0
  \,.
\ee{cattyMappo}

the discrete
Euler–Lagrange equations \refeq{LC21eqMotion} take form of 3-term,
second-order difference equations
% \refeq{LC21:1dTempFT}
\[
- \ssp_{\zeit+1} + V'(\ssp_{\zeit}) - \ssp_{\zeit-1}
    =
\Ssym{\zeit}
\,.  %\qquad  \ssp_{\zeit} \in [0,1)
\]
\refeq{LC21:1dTempFT}

The discrete time Hamilton's equations
(\ref{LC21PerViv2.1b},\ref{LC21PerViv2.1a}) induce forward-in-time evolution on a
2-torus  $(\coord_{\zeit},p_\zeit)$ {\em phase space}. For the problem at
hand, it pays to go from the Hamiltonian (configuration, momentum) phase
space formulation to the discrete Lagrangian
$(\ssp_{\zeit-1},\ssp_{\zeit})$ formulation. If the momentum is replaced
by the discrete time velocity,
\beq
(\coord_\zeit,p_\zeit) \to
\left(
    \ssp_{\zeit},\frac{\ssp_{\zeit} - \ssp_{\zeit-1}}{\Delta\zeit}
\right)
%\,,\qquad \Delta\zeit= 1
\,,
\ee{Ham2Lagr}
and the time step set to $\Delta\zeit=1$, a cat map can be brought to the
\PV\ `two-configuration representation'\rf{PerViv}
\beq
 \left(\begin{array}{c}
 \ssp_{\zeit}  \\
 \ssp_{\zeit+1}
 \end{array} \right )=
 \jMat_{PV} \left(\begin{array}{c}
 \ssp_{\zeit-1}  \\
 \ssp_{\zeit}
 \end{array} \right ) %\mbox{ mod } 1
 - \left(\begin{array}{c}
 0  \\
 \Ssym{\zeit}
 \end{array} \right )
 \,,  \qquad
 {\jMat_{PV}} =\left(\begin{array}{cc}
 0 & 1 \\
 -1 & s
 \end{array} \right ),
%\,.
\ee{eq:StateSpCatMap}
with matrix $\jMat_{PV}$ acting on the 2\dmn\ space of successive
configuration points $\transp{(\ssp_{\zeit-1},\ssp_{\zeit})}$. As was
case for the Bernoulli map \refeq{1stepDiffEq}, the cat map
$(\mbox{mod}\;1)$ condition \refeq{catMap} is enforced by integers
$\Ssym{\zeit}\in  \A$, where for a given integer stretching parameter $s$
the alphabet \A\ ranges over $|\A|={s}\!+\!1$ possible values for
$\Ssym{\zeit}$,
\beq
\A=\{\underline{1},0,\dots s\!-\!1\}
\,,
\ee{catAlphabet}
necessary  to keep $\ssp_{\zeit}$ for all times $t$ within the unit
interval $[0,1)$. We find it convenient to have symbol
$\underline{\Ssym{}}{}_{\zeit}$ denote $\Ssym{\zeit}$ with the negative
sign, \ie, `$\underline{1}$' stands for symbol `$-1$'.

As was the case for {temporal Bernoulli} \refeq{tempBern}, the condition
\refeq{eq:StateSpCatMap} is a linear relation: a given `code'
$\{\Ssym{\zeit}\}$ in terms of alphabet \refeq{catAlphabet} corresponds
to a unique temporal sequence $\{\ssp_\zeit\}$. That is why Percival and
Vivaldi\rf{PerViv} refer to such symbol \brick\ $\Mm$ as a {\em linear
code}. As for the Bernoulli system, $\Ssym{\zeit}$ can also be
interpreted as `winding numbers'\rf{Keating91}, or, as they shepherd
stray points back into the unit torus, as `stabilising
impulses'\rf{PerViv}. Here we shall use the field-theoretical parlance,
and refer to them  as `sources'.

Written out as a second-order difference equation, the \PV\ map
\refeq{eq:StateSpCatMap} takes a particularly elegant, {\em \templatt}
form \refeq{LC21:1dTemplatt}
\beq
-\ssp_{\zeit+1}  +  s \, \ssp_{\zeit} - \ssp_{\zeit-1}
    =
\Ssym{\zeit}
\,,
\ee{catMapNewt}
or,
in terms of a {{\lattstate}} $\Xx$, the corresponding {symbol \brick}
$\Mm$ \refeq{pathBern}, and the $[\cl{}\!\times\!\cl{}]$ {\shiftOp}
$\shift$ \refeq{hopMatrix},
\beq
(-\shift + s\id - \shift^{-1})\,\Xx =  \Mm
\,,
\ee{catTempLatt}
very much like the {temporal Bernoulli} condition \refeq{tempBern}.
`Temporal' again refers to the global {\lattstate} (field) $\Xx$, and
the winding numbers (sources) $\Mm$ taking their values on the lattice
sites of a 1\dmn\ \emph{temporal} lattice $\zeit\in\integers$.
%    \PC{2021-10-13}{Dropped:
%Nonlinearity of the \catlatt\ arises from restricting the admissible
%values of fields $\ssp_z$ to the unit interval.
%    }

                               \toCB

\subsection{\tempLatt\ as a continuous time dynamical system}
\label{s:tempCatODE}

Recall how the Bernoulli first-order difference equation could be viewed as
a time-discretization of the first-order linear ODE \refeq{1stepVecEq}. The
second-order difference equation \refeq{catMapNewt} can be interpreted as the
second order discrete time derivative ${d^2}/{dt^2}$, or the temporal
lattice Laplacian \refeq{LC21LaplTime},
\beq
\Box\,\ssp_\zeit \equiv
\ssp_{\zeit+1} - 2\ssp_{\zeit} + \ssp_{\zeit-1}
= (s-2)\ssp_{\zeit} -\Ssym{\zeit}
\,,
\ee{LC21PerViv2.2}
 with the time step set to $\Delta\zeit=1$.
But that is nothing but Newton's Second Law: ``acceleration equals
force,'' so Percival and Vivaldi\rf{PerViv} refer to this formulation as
`Newtonian', while we follow Allroth\rf{Allroth83}, Mackay, Meiss,
Percival, Kook \& Dullin\rf{MacMei83,meiss92,MKMP84,DulMei98,kooknewt},
and Li and Tomsovic\rf{LiTom17b} in referring  to it as the `Lagrangian'
formulation.

In other words, for a cat map the force pulse $P(\coord)=(s-2)\,\coord$
in \refeq{LC21PerViv2.1a} is linear in the angular displacement $\coord$, so
the temporal lattice equation takes form (see free action
\refeq{LC21freeAction})
\beq
(-\Box + {\mu}^2\id)\,\Xx = \Mm
%\,,  \qquad
%{\mu}^2={s}-2
% \Box\,\ssp_{\zeit}:= \ssp_{\zeit-1}-2\ssp_{\zeit} +\ssp_{\zeit+1}
\,,
\ee{OneCat}
where the Klein-Gordon mass ${\mu}$ is related to the cat-map
stretching parameter ${s}$ by ${\mu}^2={s}-2$, as already noted
in \refeq{CatMap2d}.

For small stretching parameter values, $s<2$, this discretized
Euler–\-Lagrange equation \refeq{LC21eqMotion} describes a set of coupled
penduli, with oscillatory solutions, known as the discrete Helmholtz
equation in applied math\rf{DiHaHu01,Lick89,FetWal03}, as  the
tight-binding model, the Harper's or Azbel-Hofstadter model in solid
state physics\rf{Peierls33,MiDuWh92,Cserti00,Economou06,CsSzDa11},  and
the critical almost Mathieu operator in mathematical physics\rf{Simon82},
with quadratic action \refeq{LC21freeAction} written as Hamiltonian
\[ %beq
H=\sum_\ell\ket{\ell}\epsilon_0\bra{\ell}
  + \sum_{\ell m}\ket{\ell}V_{\ell m}\bra{m}
\,,\quad
   V_{\ell m} = \left\{
     \begin{array}{ll}
         V & \mbox{if\ } \ell,m \mbox{ nearest neighbors}\\
         0 & \mbox{otherwise}
     \end{array}
             \right.
\] %ee{Economou06(5.7)}
with the stretching factor ${s}=-\epsilon_0/V$ in
\refeq{LC21PerViv2.2}.
%{OneCat}.
Here we study the strong stretching, $s>2$ case, known as the discrete
\sPe\rf{Dorr70,GoVanLo96,HuCon96,HuRyCo98,FetWal03,Pozrikidis14},
whose solutions are hyperbolic. We refer to the
Euler–\-Lagrange equation
\refeq{OneCat} as the `{\em \templatt}', both to distinguish it from
the forward-in-time Hamiltonian cat \emph{map} \refeq{catMap}, and in the
anticipation of the \catlatt\ to be discussed in the sequel
\refref{CL18}.

\bigskip

\noindent\textbf{\tempLatt, summarized.}
In the \spt\ formulation a \emph{global} {temporal {\lattstate}}
\beq
\transp{\Xx} % = \{\ssp_j\}
             = (\ssp_\zeit,\ssp_{\zeit+1},\cdots,\ssp_{\zeit+k})
\ee{path}
is not determined by a forward-in-time `cat map' evolution
\refeq{catMap}, but rather by the fixed point condition
\refeq{LC21eqMotion}
% tempCatFixPoint}
that the \emph{local}, 3-term discrete temporal
lattice condition \refeq{catMapNewt} is satisfied at every lattice point.
This temporal 1\dmn\ lattice reformulation is the bridge that takes us
from the single cat map \refeq{catMap} to the higher-\dmn\ coupled
``multi-cat'' \spt\ lattices of \refref{CL18}.

And, did you know that the cute Arnold cat is but the % very fundamental
{\sPe} in disguise? And that the lattice form \refeq{catMapNewt}
of the theory is so much more elegant than the
cat-map form \refeq{catMap}?
A cat is Hooke's wild, `anti-harmonic' sister.
For $s<2$ Hooke rules: restoring oscillations around the sleepy resting
state.
For $s>2$ cats rule: exponential runaway, wrapped globally around a
\statesp\ torus. {Cat} is to {chaos} what {harmonic oscillator} is to
{order}. There is no more fundamental example of chaos in mechanics.


\renewcommand{\period}[1]{{\ensuremath{T_{#1}}}}         %continuous cycle period
