% siminos/spatiotemp/chapter/powwows.tex
% $Author: predrag $ $Date: 2021-09-15 01:14:55 -0400 (Wed, 15 Sep 2021) $

% called by siminos/spatiotemp/blogCats.tex
% Predrag  created                          2020-12-27
\renewcommand{\ssp}{\ensuremath{\phi}}             % lattice site field
\renewcommand{\Ssym}[1]{{\ensuremath{m_{#1}}}}    % Boris


\chapter{Pow Wows}
\label{chap:powwow}

\noindent
Scribes Sidney, Han and Matt.

   % *********************************************************************
\hfill   {\color{red} The latest entry at the bottom for this chapter}

\section{Pow wow 2020-12-08}
\label{sect:pw201208}

Scribe Predrag
\medskip

One must distinguish \emph{coordinate system} symmetries (the floor cat
dances on) from \emph{dynamical} symmetries (the cat reflected through
its bisection plane).

\subsection{Hereby resolved:}
\begin{description}
\item[Task 1]
Define \templatt\ \statesp\ fundamental domain for
the \emph{dynamical} $\Dn{1}=\{e,\sigma\}$ symmetry
\\
\textbf{Predrag 2020-12-XX}: Done! See XXX %\refeq{CatMap/D1fundD}.

\item[Task 2]
Define \templatt\ \statesp\ fundamental domain symbolic dynamics.
\\
\textbf{XXX 2020-12-XX}: Done! See XXX

\item[Task 3]
Factorize \templatt\ zeta using the \emph{dynamical} $\Dn{1}=\{e,\sigma\}$
symmetry on the traces ({\lattstate}s count), then exponentiating.
\\
\textbf{XXX 2020-12-XX}: Done! See XXX

\item[Suggestion] Check the above guess factorization for the Bernoulli first.
It is more instructive to assume that the ${s}$ branch map is not
piecewise linear, and work out weights $t_p$ factorization, write
down cycle expansions in terms of {\orbit}s.
\\
\textbf{XXX 2020-12-XX}: Done! See XXX
\end{description}

\section{Pow wow 2020-12-28}
\label{sect:pw201228}

Scribe Predrag
\begin{description}
	\HLpost{2020-12-28}{
(See the post above including \refeq{HanInTime}.)
The fundamental domain should be
$\{\hat{\ssp}_{t-1},\hat{\ssp}_{t}\}\in[0,1)^2/\Dn{1}$, \ie, 1/2 of the
\statesp\ unit square area, not the qarter of the area $[0,1/2]^2$. Any
codimension-1 hyperplane going through the center of mass
$\ssp_\zeit=1/2$, all $\zeit\in\integers$ point can
serve as a boundary of the fundamental domain.
    }
	\PCpost{2020-12-XX}{{\bf done!}:
The natural fundamental domain is given by XXX %\refeq{CatMap/D1fundD}.
    }
\end{description}

\subsection{Hereby resolved:}
\begin{description}
\item[Task 1 Sidney]
Define \Henon\ \toChaosBook{equation.15.7.20}{(15.20)} \statesp\
fundamental domain for the \emph{coordinate} time reversal
$\Dn{1}=\{e,\sigma\}$ symmetry. To Predrag, the
\toChaosBook{figure.caption.291} {fig.~15.5}
and
Gutkin \etal\rf{GHJSC16} cat map
\HREF{http://chaosbook.org/overheads/spatiotemporal/GHJSC16.pdf\#figure.2}
{figure 2}
suggest that the fundamental domain border
is the diagonal, but Predrag is usually wrong.
\\
\textbf{Sidney 2021-01-XX}: Done! See  refeq{XXX}.

\item[Task 2 Sidney]
Define \Henon\ \statesp\ fundamental domain symbolic dynamics.
Predrag expects it to be like what is described in
\toChaosBook{section.25.5} {sect.~25.5} {\em $\Dn{1}$
factorization}.
\\
\textbf{Sidney 2021-01-XX}: Done! See XXX

\item[Task 3 Sidney]
Factorize \Henon\ zeta using the \emph{dynamical} $\Dn{1}=\{e,\sigma\}$
symmetry.
Predrag expects it to be like what is described in
\toChaosBook{section.25.5} {sect.~25.5} {\em $Dn{1}$
factorization}.
\\
\textbf{Sidney 2021-01-XX}: Done! See XXX
\end{description}

\section{Pow wow 2021-01-08}
\label{sect:pw20210108}

Scribe Predrag
\begin{description}
	\PCpost{2021-01-08}{
Try to understand irreps of \Dn{n}:
For worked out \Dn{2}, \Dn{4} and \Dn{n} examples,
see the text following \refeq{eq:D4}.
    }
\end{description}

\subsection{Hereby resolved:}
\begin{description}
\item[Task 1 Han]
\emph{Derive} \templatt\ factorization \refeq{AABHM99-46} and
\refeq{AABHM99-46a} into ${1}/{\zeta'(t)}$'s, either from the dynamical
$\Dn{1}$ symmetry, or from the lattice coordinate time-reversal $\Dn{1}$
symmetry.
\\
\textbf{Han 2021-01-XX}: Done!
The \Dn{1} irrep decomposition of the \templatt\  {\jacobianOrb}
$\jMorb$ is given by XXX %\refeq{CatMap/D1fundD}.

\item[Task 1 Predrag]
Incorporate into ChaosBook the \Dn{2}, \Dn{4} and \Dn{n} examples
from the text following \refeq{eq:D4}.
\\
\textbf{Predrag 2021-01-XX}: Done!

\end{description}

\section{Pow wow 2021-03-22}
\label{sect:pw2021-03-22}

Scribe Predrag
\medskip

Homework: verify whether
\begin{enumerate}
  \item
Hamiltonian \refeq{PC:HamReversal1} equations lead to the
golden (Fibonacci\rf{BaNeRo13}) cat map \refeq{goldenCat}?
  \item
is this consistent with the
Yukawa massive field mass ${\mu}$ relation \refeq{catlattMass}
to the \catlatt\
stretching parameter ${s}$ by ${\mu}^2=d(s-2)$. Note that
Grava \etal\rf{GKMM21} \refsect{sect:GKMM21} is full of such
square roots relations for parameters, such as $\tau_0=-\sqrt{\kappa_1}$.
  \item
is what they call ``$m$-physical vector and half-$m$-physical vector''
the same as our $\sqrt{...}$ or half-unit spacing lattice?
  \item
this leads to factorization \refeq{AABHM99-46}, \refeq{AABHM99-46b}?
  \item
this works (or not) for nonlinear systems?
I believe we have to verify it for the {\jacobianOrb} $\jMorb_{\pm}$,
for each solution, not on the {\lattstate}s themselves.
Sidney's \Henon\ \refeq{HenonLapl} first, than
  \item
this works any nonlinear map?
\end{enumerate}

    {\textcolor{red}{The above is a failed attempt again,
    it seems...}}

\section{Pow wow 2021-04-02}
\label{sect:pw2021-04-02}

Scribe Predrag
\medskip

\begin{enumerate}
  \item
Predrag believes that
Han's approach - for example \refeq{HLtimeRevers} and thereabouts is
the right way to go about it.
  \item
Sidney and Predrag should understand
Gallas\rf{gallas_counting} {\em Counting orbits in conjugacy classes of
the {{\Henon} Hamiltonian} repeller}, see \refeq{gallas_counting:tab1}
  \item
Han should derive \templatt\ formulas corresponding to Gallas \Henon\
$C_\cl{}$, $D_\cl{}$, $N_\cl{}$.
  \item
Han will presumably do it only for $s=3$, but Predrag would be happier to
see this done for arbitrary integer $\mu$...
  \item
Then figure out what this has to do with \Dn{\cl{}}
  \item
  Added
\begin{enumerate}
  \item
\refexam{exam:RevHenonMap}~{\em Hamiltonian {\HenonMap}, reversibility}
  \item
\refexam{exam:StandMapSymmLin}~{\em Symmetry lines of the standard map}
  \item
\refrem{rem:symmLines} {\em Symmetries of the symbol square}
  \item
\refsect{c-symbol-plane} {\em Symmetries of the \topp}
  \item
See eq.~\refeq{EGfactorSincExcl}
\end{enumerate}


\section{Pow wow 2021-04-09}
\label{sect:pw2021-04-09}

Scribe Predrag
\medskip

\begin{enumerate}
    \item
Han knows how to count the three types of time-reversal self-dual orbits,
and has explicit trigonometric products expressions for their {\HillDet}s.
    \item {\bf 2021-03-10 Predrag}
The discussion reminds me of Baake \etal\ 1997 paper\rf{BaHePl97} {\em
The torus para\-metri\-za\-tion of quasiperiodic {LI}-classes}
\CBlibrary{BaHePl97}  (enter what you learn from that paper into
\refsect{sect:BaHePl97}: is the only paper we found that associates a
1/2-shift unit length lattice with time reversal). But under my reading
the paper does not seem to be applicable to our 1D lattice with
time-inversion problem. Here is what we can adopt:
    \item
An infinite {\lattstate} can be self-dual under time-inversion about
an integer lattice point (with a single field $\ssp_0$ invariant under inversion),
or about 1/2 midway point between two integer lattice sites, with all
lattice fields paired as $\ssp_i=\ssp_{-i}$, with no invariant field.
For a finite period $\cl{}$ {\lattstate}, there are four pairs of
time-inversion points for self-dual {\lattstate}s
\beq
(0,0)\,\; ({1}/{2},0)\,\;  (0,{1}/{2})\,\;  ({1}/{2},{1}/{2})\,\;
\,,
\ee{BaHePl97(1)a}
that form $\Dn{1}\times\Dn{1}$ the discrete subgroup of ‘two-reflection
points’ of $\integers/\cl{}$. A {\lattstate} starts at a
reflection points, marches to the other reflection point, and then it
marches back, so any self-dual {\lattstate} has two reflections points,
no more and no less.
As reflection destroys cyclic invariance, all self-dual {\lattstate}s
are prime, they cannot be tiled by repeats of a shorter segment
{\color{red}(???)}.

$({1}/{2},0)$ and $(0,{1}/{2})$ are the same
by a half-period rotation, so there are three kinds of self-dual cycles
illustrated in \reffig{COMcycles}\,(a,b,c).
(compare with \refeq{EGfactorSincExcl}: Han has two diagonal classes
instead of Gallas one)
\begin{enumerate}
  \item
Both time-inversion invariant fields on the integer lattice
\refeq{HLreflectionSymSecondKind}
(Gallas\rf{gallas_counting} class $?D_1?$)
  \item
One time-inversion point on the integer lattice, the other
on the 1/2-integer lattice \refeq{HLreflectionSymFirstKind} (class $?D_2?$)
  \item
Both time-inversion points on the 1/2-integer lattice \refeq{HLreflectionSymOdd}
(class $?N?$)
\end{enumerate}
    \item
MacKay had the logistic map counts listed already in Table 1.2.3.5.1 of
his 1982 PhD thesis\rf{Bmack93} \CBlibrary{Bmack93}. Should be the same
as Gallas Table.~1, but I do not see the correspondence for all columns
- you might want
to check MacKay's construction, it is very clearly explained.
    \item
MacKay says that for self-dual cycles of odd period
there is precisely one periodic point on the symmetry line; or two or
none for even period cycles, so he correctly separates the two kinds
of `diagonal' cycles that Gallas lumps into one.
    \item                           \toCB
MacKay might be the first to note (p.~46) that period-6  is the smallest
period for which there are non-symmetric periodic orbits.
    \item
The \refeq{EGfactorSincExcl} hunch for time-reversal quotiented zeta function:
in the numerator all cycles appear in pairs, but that overcounts the boundary cycles,
which thus contribute $(D_1D_2N)$ to the denominator???
\end{enumerate}


\end{enumerate}

\section{Pow wow 2021-04-27}
\label{sect:pw2021-04-27}

Scribe, for some inscrutable reason, always Predrag
\medskip
\begin{description}
	\PCpost{2021-04-27}{{\bf in progress}:

\refexam{exam:HamHenonMap}~{\em {\Henlatt}}
\\
\refexam{exam:HamHenonJacob}~{\em {\Henlatt} stability}

Note the rescaled definition of the field $\field=a\,\ssp$
    }
\end{description}

\subsection{Hereby resolved:}
\begin{description}
\item[Task 1 Sidney]
    compute the {\jacobianOrb} $\jMorb_{p}$ for a set of shortest periods
    {\lattstate}s $p$
\item[Task 2 Sidney]
    compute the {\lattstate}s $p$ {\jacobianOrb} eigen-values, -vectors.
\item[Task 3 Sidney]
    Interpret these eigen-values, -vectors.
\item[Task 1 Sidney \& Han]
    compute analytically the {\jacobianOrb} $\det\jMorb_{p}$ for fixed
    points, period-2 and periods-3.
\item[Task 2 Sidney \& Han]
    verify Hill's formula for each
\item[Task 1 Han]
    time-symmetry reduce the {\jacobianOrb} $\det\jMorb_{p}$ for fixed
    points, period-2 and periods-3.
\item[Task 1 Sidney \& Han]
    verify the time-symmetry factorization of the Hill's formula for each
\end{description}

\section{Pow wow 2021-05-04}
\label{sect:pw2021-05-04}

Scribe, for some inscrutable reason, always Predrag
\medskip
\begin{description}
	\PCpost{2021-05-04}{{\bf in progress}:

\refexam{exam:HamHenonJacob}~{\em {\Henlatt} stability}
    }
\end{description}

\subsection{Hereby resolved:}
\begin{description}
\item[Task 1 Han]
    Figure out how to piece together period $r\cl{}$ repeats of
    period-$\cl{}$  {\jacobianOrb} $\jMorb_{p}$.
    \refeq{notRepeatOrbitJac} and \refeq{notRepeatOrbitJac1} are failed
    attempts.
\item[Task 2 Han]
    Figure out how to piece together period $\tilde{\cl{}}$  $\tilde{\jMorb_{p}}$
    with its time-reversed copy $\transp{\tilde{\jMorb_{p}}}$ by pulling out
    a matrix representation of the group element (in this case, a pair of the
    two types of reflections).
\item[Dream 1 Han]
    Replace the forward-in-time {\FPoper} by including the time-reversal
    into the operator definition. Perhaps by complexifying it, as in
    \toChaosBook{equation.28.8.19}{(28.10)} and
    \toChaosBook{equation.29.2.17}{(29.17)}?
    By replacing the monomial $x^k$ eigenvectors
    \toChaosBook{equation.28.8.18}{(28.18)}, with exponential
    multipliers, with a time-reversal invariant combination of
    eigenvalues, something like $\tanh$?
\end{description}

\section{Pow wow 2021-05-07}
\label{sect:pw2021-05-07}

Scribes Han, Predrag
\medskip

\begin{enumerate}
	\item
Using the reflection operator in the space of {\lattstate}, the
\jacobianOrb\ can be projected into symmetric and antisymmetric subspace
of time reflection. Han used the symmetric eigenvectors of the
\jacobianOrb\ to project the \jacobianOrb\ into symmetric subspace. A
better way is to use the time reflection operator to find the projection
operators as in \refeqs{timeRefl1}{timeP-Aj3}.

	\item
There are two kinds of reflection points, corresponding to two kinds of
reflection operators, denoted by a bar or a box, as in
\refeq{HLreflectionSymOdd}, and 3 (or 4) kinds of time-reversal self-dual
orbits
\refeq{COMdefN},
\refeq{EGfactor2D} and
\refeq{EGfactor1B}, or
\refeq{HLreflectionSymSecondKind},
\refeq{HLreflectionSymFirstKind},
and
\refeq{HLreflectionSymOdd}.

	\item
The eigenvalues of the antisymmetric eigenvectors \emph{probably} count
the number of antisymmetric {\lattstate}s:
$\{\Xx \in \mathbb{T}^\integers | R \Xx \mod 1\}$. %= 1 - \Xx\}$.

	\item
However, $\mod 1$ does not allow for $\field_\zeit<1$. Perhaps we should
replace the  $\mod 1$ condition by the $\field_\zeit\in[-1/2,1/2)$
condition. In that case, the antisymmetric eigenspace might be taking
care of what we call the `dynamical', site-wise symmetry under
$\field_\zeit\leftrightarrow-\field_\zeit$ $\Dn{1}$, specific to
\templatt.

	\item
We still don't know how to piece together \jacobianOrbs, other than by
doing the multiplication along temporal evolution, and then using
the Hill's formula.
The \jacobianOrbs\ for repeats of {\orbit}s do not multiply, as their
determinants do not multiply.

	\item
The $\sum 1/|\Det\jMorb_{j}|$ always satisfies the flow conservation sum
rule \toChaosBook{equation.23.4.18}{(23.18)}. Except that for \Henon\ we
are working with \toChaosBook{section.20.3} {sect.~20.3 {\em Open
systems}}, so the sum is related to the escape rate.
\end{enumerate}

\section{Pow wow 2021-05-14}
\label{sect:pw2021-05-14}

Scribe Predrag
    \begin{description}
	\HLpost{2021-05-14}{
There is no proposal for time-reversal quotiented
{\tzeta}, {\dzeta} or {\Fd}.
    }
	\PCpost{2021-05-14}{
We have:
the {\tzeta} for metal cat maps \refeq{metalZeta};
the $\mu=1$ golden (Fibonacci\rf{BaNeRo13}) cat zeta function \refeq{AABHM99-46a};
the time-reflection factorized (?)
${1}/{\tilde{\zeta}(t)}$ \refeq{AABHM99-46b};
the Hamiltonian factorized \refeq{qP:doub_polePC}, \refeq{qP:doub_pole},
\refeq{Hami-rat}, \refeq{doub_pole};
the unique Euler product \refeq{euler1};
the \Dn{1} \dzeta\  factorization \refeq{symm-isin};
something like \refeq{EGfactorSincExcl} computed on the fundamental domain;
\catlatt\  Z-transform (discrete Laplace transform) $\zeta$ function \refeq{zetaCatLattRSM};
Ihara zeta functions  \refeq{GuIsLa08Zeta}, \refeq{Rangarajan2}.
The last I looked.
    }
	\HLpost{2021-05-14}{
I'm 99\% sure that the metal \refeq{metalZeta};
tand the golden cat \refeq{AABHM99-46a} factorization have nothing to do
with time-reversal.
    }
	\PCpost{2021-05-14}{
NOTHING to do with time-reversal? You are willing to bet \$100 to \$1 that
they have NOTHING to do with time-reversal?
 }
    \end{description}

\subsection{Hereby resolved:}
    \begin{description}
\item[Task 1 Han]
\emph{Think} some more.
    \end{description}

%%%%%%%%%%%%% TEMPLATE %%%%%%%%%%%%%%%%%%%
%\section{Pow wow 2021-05-14}
%\label{sect:pw2021-05-14}
%
%Scribe Predrag
%    \begin{description}
%	\PCpost{2021-05-14}{
%XXX.
%    }
%    \end{description}
%
%\subsection{Hereby resolved:}
%    \begin{description}
%\item[Task 1 Han]
%\emph{Derive} \XXX.
%\\
%\textbf{Han 2021-05-XX}: Done!
%XXX.
%
%\item[Task 1 Predrag]
%XXX.
%\\
%\textbf{Predrag 2021-05-XX}: Done!
%    \end{description}
%%%%%%%%%%%%%%%%%%%%%%%%%%%%%%%%%%%%%%%%%%%%%%%%



\renewcommand{\ssp}{x}
\renewcommand{\Ssym}[1]{{\ensuremath{s_{#1}}}}  % ChaosBook
%%%%%%%%%%%%%%%%%%%%%%%%%%%%%%%%%%%%%%%%%%%%%%%%
\printbibliography[heading=subbibintoc,title={References}]

\ChapterEnd % formatted for ChaosBook.org
