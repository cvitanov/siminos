\svnkwsave{$RepoFile: siminos/spatiotemp/chapter/blogKITP.tex $}
\svnidlong {$HeadURL: svn://zero.physics.gatech.edu/siminos/spatiotemp/chapter/blogKITP.tex $}
{$LastChangedDate: 2021-03-20 11:53:00 -0400 (Sat, 20 Mar 2021) $}
{$LastChangedRevision: 6807 $} {$LastChangedBy: predrag $}
\svnid{$Id: blogKITP.tex 6807 2019-03-19 22:53:09Z predrag $}

\chapter{Matt's KITP blog}
\label{chap:blogKITP}
% Predrag                                           1 May 2018

\begin{description}
\MNGpost{2017-01-03}{Happy New Year!
\begin{description}
\item[KITP chats]
Chatted with Ismail Hameduddin about his project in laymen's terms about
defining non-zero mean velocity near the walls in pipe flow and how he
and his adviser's model predicts this. Chatted with Burak about puffs in
pipe flow, I believe he's going to run his presentation by me as practice
for himself and education for me.

Burak is also convinced beyond a reasonable doubt that the key to spatial
integration is symplectic integration, even though we have discussed at
length with Predrag that the formulation of the \KSe\ is \emph{not} of
the symplectic form (\ie, does not have the coordinate/momentum type
structure). That being said he wants me to look into this more.

\item[KITP poster]
Made arrangements for the poster to be printed on campus after some minor
edits.

\item[Numerical Linear Algebra]
Read through more of Trefethen and Bau\rf{Trefethen97} in preparation of
meeting with PC and JG. Arnoldi iteration, GMRES, Krylov subspaces were
main topics covered. Trying to work through some of the problems to gain
a better intuition rather than just reading.

\item[{\descent} antisymmetric subspace $\bbU^+$]
In response to PC's comments of December 24th. All of the figures
\emph{prior} to \reffig{fig:MNGVND6} were either ill-conditioned initial
conditions that did not converge to periodic orbits
(\reffig{fig:MNGVND4}) or completely numerically wrong code-wise
(\reffig{fig:MNGVND1}, \reffig{fig:MNGVND2}, \reffig{fig:MNGVND3}). I was
trying to show any progress I was making in terms of quantitative pieces
as opposed to me merely claiming ``I'm making progress". I suppose I
should have gone back for the record and stated ``avoid looking at this"
or ``this is complete nonsense" to prevent confusion, as there is no way
anyone could know otherwise. Equations such as \refeq{whyThisDeformation}
should be avoided like the plague.
\\
{\bf Predrag 2017-01-03} No need to revisit any of that, it's just the
log of your work as it progresses. It's my fault for having fallen so far
behind in reading your blog. Sorry...

\refFig{fig:MNGVND6}\,(b) and (d) is the periodic orbit $\cycle{01}$ as
defined in \refref{lanCvit07} (the {\descent} applied in
\reffig{fig:MNGVND6} is with $L = 38.5$).

\item[Spatial {\descent}]
Made small contributions to the code, but needs a quite a bit more work.
Still working through how to properly handle real and imaginary parts
separately in an efficient manner.
\end{description}
}


\MNGpost{2017-01-04}{:

\begin{description}
\item[Waleffe] is trying to answer the question of ``which solutions are
fundamental in shear flows". In investigation he looked at optimum
transport solutions on a reduced length scale (relative to solutions of
Boussinesq equation) and looked at the scaling of solutions in regards to
Nusselt and Rayleigh numbers.

Main take-away: claims that 2-D sheet scaling being close to 3-D data
scaling implies that the dynamics can be described by sheets alone(?)

\item[Wesfreid]
gave an experimenter's look into B\'enard -- von K\'arm\'an instabilities and use of vortex generators as actuators for flow control as well as
Couette-Poiseuille puffs/spots.

Main take-aways: Vortex generators can be used to reduce total drag in
expense of local regions of higher drag. In Couette-Poiseuille spots the
Couette flow was generally antiparallel to the Poiseuille flow.

\item[John Gibson] introduced me to Channelflow and we discussed some of
the methods that it uses, specifically Arnoldi iteration.
Discussed the general layout of Channelflow in terms of library and
scripts but ran out of time to do anything specific. I'm trying to get it
up and running (see Misc.) so that I can give the c++ a whirl.

\item[Reading]
Reviewed some material from Trefethen \& Bau\rf{Trefethen97}, began
picking out some Navier-Stokes papers to read\rf{CviGib10,GHCW07}. Gibson  \etal\ refer
to Aubry \etal\rf{Aubry88}.

\item[Spatial {\descent}]
In between all of the informal meetings and cookie breaks I maybe some more small edits to this code, but minimal progress from yesterday.

\item[Misc]
Installing a virtual machine image of Linux mint in hopes that I would be
able to get Channelflow up and running so I can toy around with it...
I'm convinced that I should just reinstall Linux mint to be the
main operating system of my computer immediately, I know I would have to
do this eventually anyway as performance is otherwise split between the
machine image and the Windows operating system. I'm going to try and do
this early tomorrow in hopes that I can look at channel flow before
talking to John again.
\end{description}
}

\MNGpost{2017-1-5}{:
\begin{description}
\item[Ati Sharma] tries to answer the question of why Koopman modes look
like resolvent modes... a question I ask myself daily (just
kidding... I'm hoping the tutorial to be quite extensive). Previewed
Koopman operators and their use in rewriting nonlinearities into a
different form such that one can expand about the mean flow solution. Use
of Singular Value Decomposition (SVD) acts as a filter iff there is
separation of the singular values $\sigma_i$. The main connection that
popped into my head was how Arnoldi iteration picks out dominant
eigenvalues... I looked into DMD and wikipedia mentioned both Arnoldi
iteration and SVD methods so maybe the similarity is not entirely in my
head.

\item[Dennice Gayme] investigates ``restricted nonlinear methods'' (RNL). The
main question posed was "does momentum redistribution lead to
structure?". I.e. does a streamwise description of \pCf\
with unstructured noise, lead to similar structure present in
\pCf. Dennice claims it does, although this was undertaken with
relatively low Reynolds number. I believe it was Prof. Caulfield who
suggested (and Prof. Gayme agreed completely) that the next step is to
undertake some sort of linear stability analysis in order to see if this
is somehow related to the transition of streaks/rolls to self-sustained
turbulence.

\item[Reading]
Tried to familiarize myself with dynamic mode decomposition methods in
preparation for the Koopman mode tutorial tomorrow. Borrowed
Pope\rf{popebook} from Ismail Hameduddin in order to prepare for the second
talk tomorrow per recommendation of Burak.

\item[Channelflow]
Going through a C++ tutorial recommended by Ismail to help with the transition towards Channelflow. Still trying to set it up properly such that I can run some example files before starting any of my own endeavors.

\item[Misc.]
\begin{itemize}
\item Discussed the Lagrangian formulation of \KSe\ with Burak and Predrag.
\item Briefly discussed machine learning to find inertial manifolds with Genta Kawahara.
\end{itemize}
\end{description}
}
\MNGpost{2017-1-7}{:
\begin{description}
\item[Koopman \`a la Mezi\'c]
\HREF{http://online.kitp.ucsb.edu/online/transturb17/mezic}{(video)}.
\item[Koopman \`a la Clarence Rowley]
\HREF{http://online.kitp.ucsb.edu/online/transturb17/rowley}{(video)}.
\item[Mean flow tutorial \`a la Laurette Tuckerman]
\HREF{http://online.kitp.ucsb.edu/online/transturb17/tuckerman}{(video)}.
\item[readings]
More Pope\rf{popebook} and Channelflow documentation
\item[spatial {\descent}]
Did a little bit more work towards completion of code; specifically deriving
matrix elements for stability matrices; I want to rederive everything a second
time before I upload it here as small mistakes held me back in antisymmetric
$\bbU^+$ of \KS.
I believe I will be reducing the number of modes in numerical implementation as
I have to keep track of 8 times the number of coefficients (factor of two from
real and imaginary parts of Fourier coefficients, factor of four from
system of four equations for each mode number).
The only tasks I have left are making sure the velocity function,
stability matrix elements, and differentiation matrix used to approximate
the loop tangent are defined correctly. I guess there is one additional factor
which is producing the initial conditions for searches

\item[Weekend Plans]
\begin{itemize}
\item Review literature posted to transturb wiki \HREF{http://transturb17.wikispaces.com/}{wikispace}
\item Familiarize myself with Channelflow by going through tutorial, finish c++ review \HREF{www.cplusplus.com}{c++ resource}
\item Finish spatial Newton code hopefully
\end{itemize}
\end{description}
}


\MNGpost{2017-01-09}{:
\begin{description}
\item[Channelflow]
Still learning how to manipulate things in c++ and linux. I'm still getting errors when trying to compile the simple example files, and it's obvious from the output that the libraries paths aren't defined correctly but I've tried a number of different ways to change this and they don't work. I'll be asking my office mates in hopes that they will be kind enough to elucidate the matter. Trying to keep the chin held high.
\item[reading]
Read some of the literature from transturb, I am trying to focus on the video feeds however as I find that these are better avenues for review.
\item[spatial {\descent}]
I suppose I was quite hopeful in trying to get this done over the weekend.
\end{description}
}

\MNGpost{2017-1-09}{:
%Regretfully did not type these up from the start. Need to transcribe and/or rewatch to get main points on the record.
\begin{description}
\item[R. Kerswell Talk]
\HREF{http://online.kitp.ucsb.edu/online/transturb-c17/kerswell/}{(video)}.
\item[G. Kawahara Talk]
\HREF{http://online.kitp.ucsb.edu/online/transturb-c17/kawahara/}{(video)}.
\item[R. Grigoriev Talk]
\HREF{http://online.kitp.ucsb.edu/online/transturb-c17/grigoriev/}{(video)}.
\item[C. Cossu Talk]
\HREF{http://online.kitp.ucsb.edu/online/transturb-c17/cossu/}{(video)}.
Near wall streak structures known since the sixties. Self-sustaining process is main source of turbulent kinetic energy at low to moderate $Re$. Increasing $Re$ contributes to an emerging energy peak in terms
of streamwise wavelength and wall distance. This emerging peak has to
do with structures scaling with outer units, and is believed by some
to be dominant at very large $Re$.

Looking at linearized equation with turbulent mean flow and include eddy-viscosity associated with turbulent mean flow and look at the most
amplified perturbation as a function of the spanwise scale. There are two peaks, one that does not change with $Re$ and the second peak which
does scale with $Re$.

Showed that linear instability can be a source of energy, but asks if
large scale structures can self sustain in the absence of small scales.

The opposite case, showing small-scale structures can self-sustain, was
done by using minimal flow units to remove any large active scales.

To do this with large scales there were two ideas, use a very coarse discretization, or filter the small scales and take into account dissipation such that there is no unphysical energy production peak at
the scale of the grid.

Using a local spatial filter, (spatial average around each point). This
yields and additional term which is the divergence of the Reynolds stress. Reynolds stress modeled with eddy viscosity and Smagorinsky.

To damp more small scales, increase the Smagorinsky constant. Although
this produces unphysical solutions, want to see if large-scale structures survive, which they do.

For intermediate sized structures, take a smaller sized box than that which was used for large scaled structures, and then kill out the small
scale structures using the method that was used for large scale structures.

The takeaway is that structures of all scales can self-sustain, contradicting the hypothesis of multiple hairpin vortices being advected
by the mean flow contribute to large scale structures. These larger structures extract energy from the mean flow via lift-up.

Using Newton's Method to find invariant solutions, then do continuation
in the Smagorinsky constant, to find upper branch solutions. The filtered steady solutions are connected to the branch of Navier-stokes
solutions found by many others.

\item[M. Avila Talk]
\HREF{http://online.kitp.ucsb.edu/online/transturb-c17/avila/}{(video)}.
\item[A. Duran Talk]
\HREF{http://online.kitp.ucsb.edu/online/transturb-c17/duran/}{(video)}.
\item[S. Tobias Talk]
\HREF{http://online.kitp.ucsb.edu/online/transturb-c17/tobias/}{(video)}.
\textit{The generalised quasilinear approximation: Application to jets,
HMRI, and rotating Couette flows}.
Quasilinear really means to keep only certain interaction terms. Divide the quantity of desired investigation and split it into a mean term, and fluctuation term about that mean.

The quasilinear approximation is to discard interaction terms which are contribute to fluctuations and are based on only fluctuation terms. I.e. mean and fluctuation interaction that effects the mean is kept, fluctuation interactions that affect the mean are kept, but fluctuation-fluctuation interactions that affect the fluctuations are discarded.

Effectiveness of this approximation in fluids is ofter measured by Kubo number, but you need to do the full calculation in order to find this number; it is not something one can know a priori.

In the Generalised quasilinear approximation (GQL) the terms are separated into short-scale and large-scale interactions.

GQL consists of keeping interactions of large scale mean-mean interactions (large scale determined by low wavenumber, for example), large scale mean and small scale fluctuation interactions, and small scale fluctuation-fluctuation interactions.

GQL with a wavenumber cutoff
keeping only three spatial Fourier modes outperforms the regular QL approach, as QL overemphasizes the role of waves. (Rossby wave can only reinforce Rossby wave). GQL allows for scattering into different wave numbers.
\end{description}
}

\MNGpost{2017-1-10}{:
\begin{description}
\item[Channelflow]
Practicing using Channelflow. Trying to learn all of the command line switches and codes through practice, went through
the tutorial.

\item[T. Mullin Talk]
\textit{The sensitivity of transition in a pipe}
\HREF{http://online.kitp.ucsb.edu/online/transturb-c17/mullin/}{(video)}. \\
Poiseuille flow, add perturbations. Measured the probability
of transition to turbulence versus the normalized amplitude of perturbation.
A way to introduce these perturbations is to introduce a jet with a small hole. At
$Re \approx 3,000$ the transition probability is still sharp (sigmoid like curve), but with $Re \approx
2000$, you get a small hump that proceeds the main transition.
Two different amplitudes of disturbance leads to
two different behaviors even though the centerline velocities behave in the same manner
showing similar patterns to Marc Avila's slug profiles.

Velocity defect and variance seem to settle in same place for puffs.
For the amplitudes between these two large transition
probability areas (the region where you have small probability to transition)
the correlation does not not follow this behavior.

\item[S. Rubenstein Talk]
\textit{Colliding vortex rings; Rapid breakup of coherent structures}
\HREF{http://online.kitp.ucsb.edu/online/transturb-c17/rubinstein/}{(video)}.
How do coherent structures break up, how large scales give energy to small scales. Basing
off of Lim and Nickels (1992). Small Reynolds number vortex ring collision produced a tiara of
(which looked self-similar) rings. Inviscid process of vortex reconnection
described by vortex filaments and sheets to explain energy cascade.

Rodolfo introduced the numerical side of the vortex ring collisions.
Navier-Stokes fluids look like Biot-Savart equations vortex collision. Fatter cores
in the numerics is able to reproduce the vortex reconnection.

\item[R. Poole Talk]
\textit{Low-drag turbulent states in Newtonian and non-Newtonian fluids}
\HREF{http://online.kitp.ucsb.edu/online/transturb-c17/poole/}{(video)}. \\
Drag reduction is something that you can measure in non-Newtonian fluids in any geometry
In non-Newtonian fluids you have to talk about different viscosities, extensional, biaxial, etc.
Separation of dynamics into active and hibernating turbulence,
where hibernating is defined by Virk limit (away from Blasius) characterized
with low Reynolds shear stresses. Stereoscopic PIV (2D3C) results show hibernating turbulence is not
laminar flow, and mainly consists of a spanwise motion of counter rotating
vortices.

\item[E. Wesfreid Talk]
\textit{New experiments on the subcritical transition to turbulence in Couette-Poiseuille flow}
\HREF{http://online.kitp.ucsb.edu/online/transturb-c17/wesfreid/}{(video)}
Two types of Couette-Poiseuille flow, depending on direction of pressure gradient.
Can measure the velocity profile to verify it is behaving correctly.
Centrifugal instability in plane-Couette flow is the Taylor instability while in plane Couette-Poiseuille flow
is the Taylor-Dean instability.

Introduce pertubation with hole in stationary wall in Couette-Poiseuille flow. Streamwise
perturbations of spot due to transient forcing of temporal extent "1". Can produce large scale
structures with inhomogeneous friction. Self-sustatining iff the amplitude of the perturbation
is sufficiently large.

\item[M. Schatz]
\textit{Forecasting turbulence using the geometry of invariant solutions: Experiments, theory, and numerics}
\HREF{http://online.kitp.ucsb.edu/online/transturb-c17/schatz/}{(video)}.
Talking about shadowing invariant solutions in {\statesp}.
Are there specific experiments that can elucidate the unstable/stable manifolds
of these invariant solutions?
Kolmogorov flow. Spatially sinusoidal forcing.
Can't really do 2-d flows experimentally so do quasi 2-d flows instead.
Separation of variables to introduce vertical velocity profile, results in modified Navier-Stokes
equation with a parameter $\beta$ on the advection term, and a linear friction (Rayleigh friction) term.
Inertial term (prefactor) introduced to delve into bifurcations of solutions in more
detail. Doubly periodic, singly-periodic + no slip, NPS boundary conditions.

Modulated flow; rotation symmetry. Imperfect bif. from laminar.
Non-periodic simulation (NPS). recurrence diagram.

Using norm of flow fields to find times of slow "{\statesp} speed" to initiate
Newton-Krylov searches for periodic orbits.

\item[C. Beaume]
\textit{Transitional shear flows: Computing exact coherent states in two dimensions}
\HREF{http://online.kitp.ucsb.edu/online/transturb-c17/beaume/}{(video)}.
Using asymptotic reductions as reduced order modeling of nonlinear flows.
Introduction to \pCf.
Introduction to Waleffe flow. Free-slip boundary conditions, sinusoidal forcing.
coexistance of laminar (attractive) and turbulent solutions. Constrain to the
separatrix between laminar basin of attraction and these unstable manifolds of
invariant solutions. Typically the fixed points that organize the dynamics are
upper branch solutions.

Lower branch solution Wang, Gibson, Waleffe (2007). Fourier decomposition for
steady-state ECS, shows the coefficients of decomposition scale with Re
number.

Wants to solve for fluctuations (fast time scale) while keeping averaged quantities
(slow-time scale)
fixed. This yields an eigenvalue equation for the fluctuations.

Using his specific algorithm, he wants to bring the "leading" eigenvalue (discarding
eigenvalues with larger real parts but also imaginary parts) to zero, as the marginality
is an indicator of a steady state.

Uses L. Tuckerman algorithm for preconditioning, semi-implicit Euler scheme to solve for linear term
explicitly and nonlinear term implicitly. Results in resolvent $(I - \omega L)^{-1}$.

Do not need same preconditioner for mean, fluctuation equations.
mean equation preconditioner $P = I-L$
fluctuation equation preconditioner $P = (I - Re^{-1} L)$

Messy Bifurcation curves for modulated solutions, strange behaviors. They make Rich Kerswell nervous. Increasing
resolution smoothes them, has to do with low-dimensional representation.

\item[B. Budanur]
\textit{Unstable manifolds of recurrent flows in pipe flow}
\HREF{http://online.kitp.ucsb.edu/online/transturb-c17/budanur/}{(video)}.
Unstable manifolds shape the separatrix of the chaotic attractor.
Unstable manifolds for Poiseuille flow/pressure driven flow
Continuous symmetries lead to relative invariant solutions.
Reducing the symmetry by removing the parameterization of translation, leads to
something much nicer topologically.
Introducing time evolution as mapping f. Listing symmetries. In group notation,
flow mapping f commutes with group element operation (group action).
Each solution has infinitely many symmetry copies on its group orbit.
Symmetry reduction is a coordinate transformation such that group
orbit is mapped to one orbit.

Intro to \SOn{2} reduction via {\fFslice}. Not geometry dependent, can define
coordinates as you see fit. Can use these projections to project neighbors onto slice.
Therefore one can project the tangent space onto the slice. which allows
for numerical integration that is completely contained in the symmetry
reduced space if you are careful.

Example, unstable manifolds of traveling waves. Create a small ball (circle for 2d unstable
manifold) and integrate to map out unstable manifold. Use a PIM triple/bisection to increase resolution
and capture more of the unstable manifold.

Using a recurrence based norm, can use small values of the norm to begin Newton searches for
invariant solutions. Then one can play the game of mapping this invariant solution's unstable manifold

Going over using higher energy solution as energy control that drives system towards laminarisation.
"too much turbulence kills turbulence".

Talking about localized relative periodic orbits, M. Avila's paper.

Can reduce dynamics/Floquet vectors of map of {\fFslice} with Poincar\'e section.

Calculating unstable manifolds of traveling waves and relative periodic orbits and generalizing
so long as symmetries are of the \SOn{2} class.

Puff formation in this subspace seems to be forming along heteroclinic connection between LB and UB
solutions.

\textbf{E. Knobloch Comments}
Bifurcations from a group orbit. Derived 25 years ago. Projections lose structure about the global structure of the
manifolds. Lose information of the vector field that is orthogonal to the group orbit.

\item[P. Cvitanovi\'c]
\textit{Turbulence: How fat is it?} %My favorite title so far.
\HREF{http://online.kitp.ucsb.edu/online/transturb-c17/cvitanovic/}{(video)}.
Strange attractor stuffed into a finite-dimensional body bag.
Computation of covariant Lyapunov vectors, covariant vectors are non-normal.
Algorithmic approach to compute these.
Physical dynamics is hyperbolically separated from the infinity of transient modes.

\KSe\ spectrum of eigenvalues (Lyapunov spectrum). The values of the Lyapunov exponents.
Sharp shoulder where you fall off the cliff.

Doubling the spatial length doubles the number of "physical" dimension. Increasing
modes in same length just increases the number of "transient" modes.

Transient and "entangled" = "physical" perturbations. Entanglements is between stable and unstable
manifolds, where stable can win for a short time until unstable takes back over. Inertial manifold
locally spanned by entangled covariant vectors.

Dimension of the inertial manifold.
Distribution of angles between eigenvectors. Need to take stable eigenvectors into
account due to nonlinearities. Kaplan-Yorke wrong by factor of two.
Transient eigenvectors cannot have angle of zero and in fact are likely to be orthogonal
to one another.

With these linearizations we are still embedded in higher dimensions so where the blip are we.

Cartography. Cover the inertial manifold with a set of flat charts/tiles. The tiles determine
when you need to switch your maps.

Computing Floquet vectors of periodic orbits to machine
precision.

From periodic Schur decomposition, 8 modes are entangled, the rest are transient.

Believed that this is only necessary to do for one periodic orbit, as this
computation done for 500 periodic orbits all provided the same number of entangled
modes.

The more honest answer to Laurette. Take the first n eigenvectors, defines linear space,
find complement to that space, find the angle between those two spaces.

When you do not have enough eigenvectors to span the neighborhood of periodic orbit
then the angles of ergodic trajectories will not be small as you will be "poking" out.

inertial manifold spanned locally by entangled covariant vectors, tangent to unstable/stable
manifolds. The rest is transient.
\end{description}
}

\MNGpost{2017-1-11}{:
\begin{description}
\item[coding]
Small amount of work put into spatial {\descent} code but most of the evening was used chatting
with people over dinner at KITP residence.
\item[K. Julien Talk]
\textit{Quasigeostrophic investigations of non-hydrostatic, stably-stratified and rapidly rotating flows}
\HREF{http://online.kitp.ucsb.edu/online/transturb-c17/julien/}{(video)}.
Motivations are strong system rotation and stable density stratification characterize many geophysical flows.
Rossby and Froude numbers much less than unity.
Large spatial size leads to normal mode analysis, eddy dynamics on different
scale than gravity waves and coriolis terms(Related to Froude and Rossby number respectively).
Classical QG equations.
Scale of the forcing leads to either inverse energy cascade or direct enstrophy cascade.

With Froude order unity get columnal waves. Try to get reduced equations as these things are hard
to resolve due to aspect ratio.

This leads to non-dimensional Boussinesq Equations, which in turn with assumptions about
scale lead to non-hydrostatic quasi-geostrophic equations.
Results simulation results in large scale barotropic modes. To explain the layering
of the solutions to some of the simulations.

\item[N. Constantinou Talk]
\textit{Understanding self-organization in turbulent flows by studying the statistical state dynamics}
\HREF{http://online.kitp.ucsb.edu/online/transturb-c17/constantinou/}{(video)}.
Long streaky structures in boundary layer turbulence of the Earth.
Claims.
"Underlying dynamics of structure formation lies in the interaction of turbulent eddies with
mean flows"
"Often structure formation only has analytic expressions in statistical state dynamics"
"Because of the first claim, a second-order closure of SSD is often adequate"

Introduction to SSD.
\begin{enumerate}
\item Split into mean + eddy terms
\item Form hierarchy of same-time statistical moments/cumulants
\item Find how each of the moments/cumulants evolve
\item Throw away everything above second order (S3T)
\end{enumerate}

"By studying the dynamics of the statistics, new phenomena arise that are not present or
obscured in single flow realizations."

How does a state with no mean flow become unstable? Perturbations produce Reynolds
stresses that reinforces the perturbation. Applies to 2d as well as 3d.

Find eigenvalues and eigenfunctions and then plug into dynamics.

\item[K. Deguchi]
\textit{The high-Reynolds number limit of self-sustained magnetohydrodynamic dynamos}
\HREF{http://online.kitp.ucsb.edu/online/transturb-c17/deguchi/}{(video)}.
MHD dynamo introduction, exchange kinetic and magnetic energy.
Applying nonlinear analysis of subcritical shear flows to MHD dynamos. Zero
net flux implies purely driven by shear.
Asymptotic expansion in inverse Reynolds number. Stretching coordinate and matching.

Self-consistent asymptotic analysis
Many different ways to handle the order the leading powers of asymptotic expansion.

Vortex-wave interaction, self-sustaining process theories for the driving mechanism
of streamwise vortices. Explanation of these processes.

Using MHD equations with Hall term.
Derive a wave equation, produce Alfven wave.

Stress jump in roll component, Resonant Absorption Theory.

Lots of really long equations, with the effect of system rotation and linear instability following as analysis.
\item[J. Klewicki Talk]
\textit{Invariant representation of mean inertia provides a theoretical basis for the log law in turbulent boundary layers}
\HREF{http://online.kitp.ucsb.edu/online/transturb-c17/klewicki/}{(video)}.
High Reynolds number boundary layers. Logarithmic mean velocity profile. Von K\`arm\`an constant.

Open question for logarithmic dependence in boundary layers not
well understood.

Rationalizing the log law. Prandtl
derivation (distance from the wall scaling hypothesis).

Looking for invariant formulation (coordinate independent) version of equation that we can solve to get log law.

Rescaling equations when viscous, pressure gradient, turbulent inertial forces are all of same order leads to invariant representation in this layer.

Taken equations of motion, used similarity for closure, rescaled them in a way such that you get a nonlinear equation that you can integrate to get a log-law.

Apply similar transformation for boundary layers after finding that one of the terms is constant.
\end{description}
}

\MNGpost{2017-1-12}{:
\begin{description}
\item[spatial {\descent}]
Worked on code a slight bit more in between talks.
\item[B. Mckeon Talk]
\textit{Self-similarity in the resolvent model: Linear response and nonlinear interactions}
\HREF{http://online.kitp.ucsb.edu/online/transturb-c17/mckeon/}{(video)}.
Summary of resolvent analysis, expansions around mean such that nonlinearity all
contained in one term.
Resolvent operator transports, singular values.
Randomized scheme because resolvent operator is very low rank in regime of turbulence,
i.e. only few singular values from decomposition are required. Low rank
related to critical layers.

Discussion about the shapes of these resolvent modes, regions of different behaviors
based on wave speed. Wave speed localizes reponse of resolvent operator. Low wave
speed implies that all resolvent modes are "attached" to the wall, then an
intermediate region, then a region of wave speed values such that all modes
are "detached".

Dynamics dictated by when wave speed is equal to local turbulent mean, that is to say these
are the values that result in physical behaviors.

In the inner regions the singular values "do all of the hard work".

Required that U-c should be scalable in y for geometric self-similarity, shows after scaling
that.
Self similarity related to attached eddy ideas of Townsend et al.
Shows similar results for broadband forcing.

Nonlinear interacting hierarchies, interaction of triatically consistent modes allows
for excitation of another set?

Everything seems predicated on knowing the mean, or giving a good guess for the mean based
on experiment rather than a priori knowledge.
\item[M. Jovanovi\'c]
\textit{Color of Turbulence}
\HREF{http://online.kitp.ucsb.edu/online/transturb-c17/jovanovic/}{(video)}.
Spatiotemporal spectrum of stochastic forcing. Using linear navier-stokes equations with
stochastic input, linearized around the turbulent mean.

"Can I choose stochastic input based on spectrum such that linear dynamics produce correct
statistical information"

Proposed approach: view second-order statistics as data for an inverse problem. His contribution
is a principled way of turbulence model as an inverse problem.

A good way of getting insight is to look at how quantities are scaled into observable
quantities that we care about.

Stochastic forcing that is white in time and $y$, harmonic in $x,z$.

Linear equations can capture coherent structures.

Using Lyapunov equation propagates white correlation of sources into colored statistics
of coordinates, using spatial covariance matrix. $AX+ XA* = -BWB$.
If W positive definite, then X positive semi-definite, but converse not true.
Colored in time $AX +XA* = -(BH +HB*)$

Convex optimization problem.

White-noise too restrictive, cannot reproduce dns with whit in time forcing.
Suprising: quality of completion of incomplete data, retrieving matrix structure.
Without keeping physics in the mix, you would get complete garbage.

\item[A. Sharma Talk]
\textit{Resolvent Modes and invariant solutions}
\HREF{http://online.kitp.ucsb.edu/online/transturb-c17/sharma/}({video)}
Using resolvent modes to find new invariant solutions of Navier-Stokes.
Linearization about the mean (w=0) mode. leads to equation relating
second order terms to u.
Nonlinear terms are exciting the mean,invert (explanation of resolvent operator equation)
$(wI -L)^-1$. Resolvent is well approximated via projection, use SVD to give a basis.

To fix the coefficients properly is a quadratic optimisation problem of an algebraic
equation.
Hopefully resolvent mode analysis is representative of the effect of invariant
solutions on {\statesp} and the low dimensionality of said solutions.

Showing actual solutions to pipe flow and their projections onto small number
of modes and comparison.

using mean flow of solutions, "at that point in {\statesp}"....this isn't right is it?

Projection of upper branch solution looks like lower branch counterpart. If you put
upper branch projection and search for solution, looks like equilibria.

Using a project-and-search algorithm (project onto resolvent modes, then use Newton-Krylov).
Dissipation continuations graphs.

\item[Clancy Rowley]
\HREF{http://online.kitp.ucsb.edu/online/transturb17/rowley}{(video)}.
\textit{Data-driven methods for identifying nonlinear models of fluid flows}
Dynamic Mode Decomposition, Koopman Operator, data approximating Koopman.

Eigenfunctions of the Koopman operator determine coordinates in which a system
evolves linearly.

If U has enough eigenfunctions so that we can reconstruct
the state x from the values of the eigenfunctions then
there is a coordinate change in which the dynamics is linear.

Data-driven inner product.
Subspace S spanned by a set of observables
Projection Theorem
Approximate Koopman operator by projection onto the subspace S.

Numerical method that is obtained is the same as DMD, but will
be able to say more about correspondence about Koopman operator and DMD.

The data driven inner product converges to normal L2 norm in limit of large
numbers.
Create subspace spanned by n observables, define projection operator.

Theorem(Data-Driven Projetion)
Let A = X+X then for any $v \in \mathcal{C}^n$ then
PU F(v)= F(Av), that is A is the matrix representation of the projection of Koopman.

Turns out this matrix, A, is identical to matrix computed in EDMD. (Extended
Dynamic Mode Decomposition).

What if there is an eigenfunction that is in subspace S, then this method
is able to find it.

If subspace is invariant under U. If v is eigenfunction of A, then $\phi$ is
an eigenfunction of u, with $\phi = F(v)$.

Hard to actually check if S is invariant under U.

Choice of observables to determine the basis is nontrivial.

Using this model with POD modes and comparison to Galerkin truncation
of Fourier modes. Works much better than Galerkin when white noise
introduced.

\item[M. Green Talk]
\textit{Tracking coherent structures in massive-separated and turbulent flows}
\HREF{http://online.kitp.ucsb.edu/online/transturb-c17/green/}{(video)}.
pFTLE and nFTLE: finite time Lyapunov exponents. (positive and negative time).
This is a field that is used to differentiate between dynamically distinct
regions. Negative time FTLE is where fluid is attracted in forward time.

What can we do with this information about these ridges? Non-parallel
intersections interesting. Saddle-like behavior of these intersections.

Intersection starts on cylinder and seems to track vortices.
Extracting physics from the distances of vortex core and cylinder, etc.

Can tie vortex shedding to the dynamics of the saddle point.

Can you track similar coherent structures in 3D turbulence?
Subsequent vortex generation can be tracked by additional saddle
generation.

Tracking material transport that will be involved in secondary hairpin
generation, drawn up from the wall.

nFTLE, pFTLE surfaces calculated in channel flow. Projected onto 2-d plane,
used PIV program to calculate average velocity of saddle points correlations
Comparison to velocity profiles, slower than mean in bulk, faster than mean
near walls. Similar to perturbation profile of ...

Can we use these saddle tracking methods to follow coherent structures in 3d?
not so simple because they actually live on 2d surfaces.
Large amounts of velocity data needed but it's really hard to mess up the general
direction of unstable/stable manifolds of these structures.

\item[Y. Hwang Talk]
\textit{Self-sustaining attached eddies in wall-bounded turbulence}
\HREF{http://online.kitp.ucsb.edu/online/transturb-c17/hwang/}{(video)}.
Attached Eddy Hypothesis, Townsend. Viscous layer, buffer layer, Log region (scales with wall),
wake region.
Linear spanwise length scale growth is intimately related to the log region scaling.
Isolating motions at given spanwise length scale growth, filter out larger scaling,
damping out smaller structures. Smagorinsky scale damping.

Showing self-sustaining process of attached eddies with correlation functions.

"real" part of the talk. Pressure fluctuation generated by nonlinear feeding processes
of vortices.
Showing length scale of streamwise vortices is the same as the length scale of pressure
off of the wall.

Both rapid and slow pressure is generated in streamwise wavy structures. Rapid
pressure is mediating the lift-up effects which controls the generation of
streaks from streamwise vortices.

Slow pressure generation is dominated by streamwise vortical structure.

Skin-friction generation. Not generated by singular structure at a given
length scale. Everyone is important at high Reynolds number.

Three methods for skin-friction, identifying size of structures responsible.
Using Fukugata-Iwamoto-Kasagi identity. Only include structures up to given length scale,
artificial damping of large structures.

FIK say that large scale structures contribute the most to skin-friction. Getting
rid of largest structures doesn't reduce drag that much, it's actually disposing
of log-region scale structures that will reduce drag the most.

Supressing the lift-up reduces drag.

Upper branch is more related to large scale structures.

\item[E. Knobloch]
\textit{A nonlinear model of noise-sustained structure in subcritical systems}
\HREF{http://online.kitp.ucsb.edu/online/transturb-c17/knobloch/}{(video)}.
Lots of interesting things happen when you diverge from periodic boundary conditions.
Changes convective instability to absolute instability. Convective instability
is things that grow and are then advected with the flow, other stability
is things that grow in place and fill the domain.

Complex-Ginzburg Landau. For range of bif. parameter things are convectively
unstable, and outside this region structures are absolutely unstable.

Study the model problem that has the same dichotomy.
Bifurcation diagram with different amplitudes of disturbances.

%Look at this slide in more depth
Scale the equations in limit of large advection, use Li\'enard transformation
to change the problem to a BVP.
Spatial van der Pol oscillator.

The coincidence of the two manifolds results in an "orbit flip bifurcation".

Talking about the front location, originally "found" using phase diagrams,
then looking at inlet perturbations eta and seeing how the location of the
front changes.

"the observed sensitivity finds a natural explanation in the
presence of canard segments on solutions of the spatial BVP
for the steady states of the model. These results are obtained by
recasting the mode as a slow-fast system in space and focusing on its steady states subject to

Even very small changes to inlet boundary conditions have a huge effect.

\item{S. Zammert}
\textit{Coherent structures in boundary layers in the quasilinear approximation}
\HREF{http://online.kitp.ucsb.edu/online/transturb-c17/zammert/}{(video)}.
Exact solutions with smaller scales or multiple scales.
Rescale coherent structure with factor $\lambda$ and plug back into Navier stokes and find
that this is related to scale factor $\lambda$ at $Reynolds / \lambda^2$.
With this scaling, coherent structures of multiple scales can coexist, creating a hierarchy of scaled
solutions.

Lower branch solutions only. Wall normal localization gets lost.
Method also works for localized solutions.
Apply rescaling procedure to axes of wall-normal localization leads to results that are
completely self-similar.

Linear stability analysis, stablity eigenvalues, have same
self-similar spectrum, therefore
can use larger structures at lower reynolds numbers to do
analysis at higher reynolds numbers.

Quasilinear approximation, split into spanwise and no-spanwise variation.
Define projection operators s.t. you get either of these.
Nonlinear term split into parts, some discarded
others kept. $(u2 \cdot \nabla )u1 + p1 (u2 \cdot \nabla) u2$
discard part that effects mean flow.

Much simpler dynamics in quasilinear approximation, attracting state in this approx is
always steady.
Qualitatively the profiles of the nonlinear, quasilinear cases are similar.
Profiles scale in the same way that solutions did.

How turbulent mean profiles are related to exact coherent solutions.
Bifurcations of quasilinear states, what's going on?

The amplitudes of modes vs. reynolds number, decrease quite rapidly at low re, but
dip then plateau at higher Re.
Whenever a new mode becomes active, bifurcations of quasilinear.

Look at how new modes contribute to profile. Modes do not interact, turn off
certain modes as way of initializing initial condition for ECS search.
Low mode numbers lead to linear profiles in the bulk and high contribute to
boundary layer.

Continuation to nonlinear system. Homotopy introduced such that the parameter
equal to one are ECS in quasilinear, 0 equal to ECS in nonlinear.

In order to synthesize turbulence, you need constant energy scaling.
\end{description}
}

\MNGpost{2017-1-13}{:
\begin{description}
\item[L. Tuckerman Talk]
\textit{Transition to turbulence in Waleffe and Kolmogorovian flows}
\HREF{http://online.kitp.ucsb.edu/online/transturb-c17/tuckerman/}{(video)}.
Taking quasi-turbulent (no kolmogorov scaling) and quasi-laminar flow to larger scale domain.

Using coupled map lattice to analyze
structures on this large domain.

Are results from coupled map lattice
continuous? Turbulence fraction
graphs have same slope on log-log
plot as directed percolation.

Shows how large domains are necessary
to pick up on structures.

\textbf{Waleffe Flow}
Reproduces interior of Plane-Couette
well by having sinusoidal body force
and free-slip boundary conditions.

Truncated Waleffe flow results in
system of seven PDEs.

Simulations find that transition is continuous, shares same critical
exponent.

Directed Percolation is the basis
of these exponents, although
never truly realized in physical
systems.

Can be classified by three independent
exponents, turbulent fraction, laminar
gaps in time, laminar gaps in space.

(turbulent and laminar taking the role
which is normally discussed as excited
and unexcited in directed percolation).

Streamwise spatial gaps do not agree.

\item[B. Hof Talk]
\textit{Transition to turbulence in channel flow}
\HREF{http://online.kitp.ucsb.edu/online/transturb-c17/hof/}{(video)}.
Stripes present in channel flow.

Edge-tracking of tilted domain (tilted so that stripes are
perpendicular to domain boundaries). Using Newton method
to converge the solution.

Lower branch bifurcates into RPO solution which becomes the
new edge state, followed to upper branch.
Following shows more bifuracations and then a boundary crisis, this scenario
gives rise to transient striped patterns.

Might be able to find in experiment as it is a stable RPO,
but would need a localized version without periodic boundary conditions.

There is an angular dependence of the
bifurcation diagram (angle of tilted domain). 45 degrees seems optimal.

Transient stripes and puffs to sustained. Time scales are very long
and very important because turbulent
fraction continues to climb as system
is evolved.

Memoryless process of pipe flow puffs
lead to measuring characteristic life
time. Growing more than decay is being
used as a indirect method of characterizing the critical Reynolds
number at which sustained turbulence occurs ($Re_c \approx 2040$).

$Re = 2020$ the turbulent fraction goes
the zero after a large number of advective time units,
while at $Re = 2060$ there appears to be a statistical
steady state w.r.t. turbulent fraction.

Using a directed percolation analogy,
discrete model based on lifetime and splitting statistics, but cannot get
critical exponents because interactions
are not taken into consideration in model. When interactions are taken
into account, the percolation exponent comes out.

\item[B. Eckhardt Talk]
\textit{Dynamical systems analysis of transition in boundary layer flows}
\HREF{http://online.kitp.ucsb.edu/online/transturb-c17/eckhardt/}{(video)}.
Looking at spatiotemporal development of flow downstream from a thin plate.

Five main spatial behaviors of flow, arranged in a streamwise fashion:
\begin{itemize}
\item free-stream turbulence
\item streak growth
\item breakdown into turbulent spots or nucleation.
\item spatio-temporal intermittency (spots spreading and merging)
\item turbulent boundary layer
\end{itemize}

Localized edge-state that produces secondary structures. It's difficult to
access asymptotic dynamics because there is spatial development.

This is complicated, so instead apply
suction such that the developing boundary layer is translationally
invariant. Then can look at asymptotic
suction boundary layer. It is linearly
stable up to $Re_c = 54370$, but goes
turbulent with bypass transition at
much smaller $Re = 270$.

In short-wide domains, edge states evolved in time go through a
process that results in spatial translation; can
relate to self-sustaining mechanism.

Can development multiple edge states the move left, right, oscillate,
and move erratically when tuning parameters.

Can compare edge states and nucleation events. When starting with an initial
random perturbation field, there is a transient period where perturbations decay,
leading to streak growth. Before
the streaks transition into turbulence via nucleation events,
look at the streak structure. These spatiotemporal
structures are similar in width, strength, and sinuous instability
 wavelength to the edge states found previously.

Looking at random initial conditions
in a {\statesp} view. The energies of velocity components of
initial conditions come close to those of
edge states and then are either ejected
towards turbulent behavior, or cross the edge states and decay.

Measuring "intermittency factor" which
seems identical to Tuckerman's "turbulent fraction" from directed
percolation.

The nucleation model in conjunction
wth the probabilistic cellular
automaton gives a good description
of bypass transition.
\end{description}
}

\MNGpost{2016-1-14}{:
Looking at Lan's directory, I was able to find what I believe is the period
of $\overline{01}$ orbit to six decimal digits, $T_{\overline{01}} = (25.6356)82$
which conflicts with my value of $T_p = 25.635484$

Spent some time learning c++.
}H

\MNGpost{2017-1-16}{:
\begin{description}
\item[G. Falkovich Talk]
\textit{Interaction between mean flow and turbulence}
\HREF{http://online.kitp.ucsb.edu/online/transturb17/falkovich/}{(video)}.
Isotropic homogeneous turbulence is basically nonsense because
it discards interactions.

Has a localized edge, so it leaves a wake, edge has asymptotic behavior.
Drag coefficient goes to constant as Re goes to infinity.
Friction factor for pipe flows scales like 1/Log(Re) as Re goes to infinity.

Cannot describe turbulence generated by a flow, so instead try to describe a
flow generated by turbulence. How to predict a glow generated by an inverse
cascade in two dimensions.

How? Set the small scale forcing, change the geometry to change the flow.

Guiding principle to guess is to guess that it occupies the lowest available
wavenumber.

In a box with symmetry, expect dipole or large central vortex with four small
counter rotating vortices at the corners. If discarding symmetry expect
flow along long direction.

Two-dimensional turbulence, including forcing and friction. Condition
for a strong mean flow estimated by kolmogorov factor.
$\epsilon^{-1/3}L^{2/3} \alpha << 1$. Epsilon is total net force.

This friction is not viscous friction, for example its set by the layer
of lubricant \`a la Schatz.

Two-dimensional vortex sustained by small-scale forcing.
J Laurie et al PRL 2014.
Radial forcing generalization Frishman and Falkovich, unpublished.
Next step is describing correlation functions of turbulence,
long calculation, resulting in expression for pressure
as a function of radius.

Prediction that first harmonic is independent of radius.
Graphs of mean flow and momentum flux quantities vs. r showing
valid region of theory.

Numerical modeling in a box with no-slip boundary
Clear symmetry breaking, vortices have direction. Takes many turnover
times to see behavior.

Inverse cascade on a sphere, non-rotating.
Same procedure as before, argument for neglecting large velocity, pressure
correlation terms, leading to a new closed equation on the momentum flux.

Can we have a plane flow out of a two-dimensional turbulence with
forcing and friction? Aspect not equal to one, argument of lowest
wavenumber makes jets, not vortices.

Inverse cascade on a torus. Starting with aspect ratio one, have vortices,
changing the aspect ratio produces jets, but vortices remain.

Long time means restore translational invariance that is broken on small
time scales.

Lowering friction increases the number of vortices.
Radius of curvature going to zero is very interesting limiting case.

Inverse cascade and vortices in compressible 2d turbulence.
Compression creates shock waves which dissipate energy on small scales.
Use of shallow water equations for the density fluctuations, density
equated to height.
Viscosity is an irrelevant perturbation in regards to large vortices
and or inverse cascade.

\item[D. Lecoanet: Introduction to Dedalus]
\HREF{http://online.kitp.ucsb.edu/online/transturb17/lecoanet/}{(video)}.
Pseudo-spectral, open-source, python code. Enter equations as strings.
Hard to install, otherwise easy to use.

Great for equations that do not have specific optimized
implementations (e.g. asymptotic equations) and data analysis.

Can solve reduced model equations to verify their accuracy.

Example problems: Burgers equation, Kelvin-Helmholtz instability.

Initial value problem, need equations, domain, timestepping scheme.

Goes into Solver variable, which gives a state and integrator, and fields.

Implicit, explicit time stepped. Implicit on LHS of equality, explicit
on RHS of equality. i.e. linear terms on LHS and nonlinear on RHS.

Restricted to Fourier and Chebyshev bases, only allowed one Chebyshev basis
at the moment.

Can have non-constant coefficients for Chebyshev but not Fourier bases.

\item[Channelflow]
Got the basics of c++ down, going through Channelflow scripts and documentation to understand
what each of them do in detail.
\item[reading]
Reading and rereading \refref{DingCvit14} to understand the process employed for \KSe. Filling
in gaps with \refref{Trefethen97} as I see fit.
\item[misc]
Rewatched S. Tobias' talk as review and transcribed notes for it.

\end{description}
}

\MNGpost{2017-1-17}{:
\begin{description}
\item[B. Marston Talk]
\textit{El ni\~no as a topological insulator: A surprised connection between geophysical fluid dynamics and
quantum physics}
\HREF{http://online.kitp.ucsb.edu/online/transturb17/marston/}{(video)}.
Kelvin waves play key role.
Intro to integer quantum hall effect and topological insulators.

Immunity to macroscopic messiness is due to topological robustness.
Leftward, Rightward bulk movement prevents back scattering due to
inhomogeneities. This leads to quantum spin hall effect and topological insulators.
Special curves of dispersion relations are the states that propagates
along the edges of the physical object (surface states). Spin-orbit
interactions stratifies the spin states are produce the quantum spin
Hall effect.

Dispersion relation of shallow water waves on a spherical surface,
coriolis force switches direction at equator. Wave that propagates along
equator is trapped. Gravity waves, Yanai waves, Kelvin waves, Rossby waves.

Two layer model, behaves very differently with and without rotation. Need
enough rotation for time-scale separation.

Numerical spectrum from shallow-water simulations. shallow means wavelength
of waves in relation to depth.

Topologically protected systems, i.e coupled gyros, acoustic modes.
Key factors of topologically insulation is gaps in dispersion relation
other than these edge modes. Can interpret equatorial divide as the edge
modes. Need to break time-reversal symmetry in part of the system.

Shallow water equations on Lieb lattice. Leads to dispersion
diagram that has bulk states, and edge states (linearly dispersing modes),
which equal Kelvin, Yanai waves.

Linearized equations of motion, leads to Chern number and Bulk-Edge correspondence.
Parameterize eigenstates of frequency space equations (linearized), kx, ky, f
parameterized by trigonometric functions. Multiply by phase factor to repair the
ill-defined azimuthal coordinate at the north-pole and south-pole.

Chern number comes from quantum analog, magnetic field being applied to a sphere.

Plots verifying protection from obstacles, (analogous to inhomogeneities in quantum systems).

Claims that Kelvin waves at ocean basin boundaries, Magneto-Rossby waves, Fluids
with mean flows are all topologically protected. This is related to the lack
of back scattering, but maybe this is
related to E. knobloch's talk? only persists in convective instability regime? Or
is this back-scattering completely unrelated to upstream growth.

Protection from interactions/nonlinearities, Fractional Quantum Hall effect arises
from strong interactions.

\item[Marie Farge and Kal Schneider Talk]
\textit{Energy dissipation caused by boundary layer instability at vanishing viscosity}
d'Alembert's paradox, dissipative properties of vortices in wall bounded 2d flow,
differences in navier-stokes, euler, prandtl solutions.

Finite-time singularities are a candidate for explaining the dissapation rate
at high reynolds number.

Kato's theorem, navier-stokes solutions converge toward Euler solution iff
the viscosity vanishes.

Volume-penalization (pseudo-spectral). Brickman term added to Navier-Stokes
equation that produces vorticity at the wall, and boundary layers are
detached from the wall. What happens at the wall that makes boundary
layers detach?

Vortex dipole produces many small scall vortices from lifting the boundary
layer off of the wall. Looking locally at the combined vortex structure,
the dissipation of the attached vorticity layer goes to zero, but the
detached vortex has constant dissipation. Detached vortex scales like
$1/Re$ term, while attached vorticity layer scales like $1/Re^{1/2}$.

Wall assumed solid, but there is a porosity ($1/\eta \mbox{where,} \quad \eta = permeability$) that induces
the volume penalization term.

\textit{Comparison between Navier-Stokes and Euler, Prandtl solutions.}
Replacing no-slip with this volume-penalization term. Replacing
spectral method (due to its order) with compact finite differences.
The big plus is that it is applicable at higher Re numbers because
you can take larger permeability values.

Reformulate Prandtl equations to be in terms of vorticity. Only have
viscous term in y-direction, but not in wall-normal direction. Coupling
between Prandtl and Euler is the pressure gradient. Scaling argument
saying that you can discard wall-normal terms, but keep wall-normal
derivatives. Non-uniform grid that depends on time, before and
after the boundary layer detachment.

Prandtl solution finite time singularity well known, expected due
to Kato's theorem that navier stokes converges to euler.

Production of dissipative structures is the key feature of boundary layer
detachment.

In DNS the Prandtl boundary layer is used to justify the coarseness
of numerical grids, trying to show that one needs to be careful and this
thickness may not be the correct guiding factor. Results suggest
that new aymptotic description of flow beyond the breakdown of the
Prandtl regime is possible.

Prandtl is great for when the boundary layer remains attached "keep it alive".

$1/Re$ number scaling is wall-number scaling.

Reynolds number original definition is ratio of norm of nonlinear term,
to the norm of the linear term.
\end{description}
}
\MNGpost{2017-1-19}{:
\begin{description}
\item[T. Schneider]
\textit{Computing Clouds: Why turbulent coherent structures are crucial for
predicting climate change}
\HREF{http://online.kitp.ucsb.edu/online/transturb17/schneider/}{(video)}.

Shows that equilibrium climate sensitivity is a great tool for predicting
various quantities related to climate change. (carbon dioxide concentration, amount of total
reflectance).

Equilibrium climate sensitivity (ECS) vs. Low-cloud reflectance variations
(normalized w.r.t. seasonal fluctuations in temperature).

Most atmospheric water is vapor not in clouds.
Clouds form where small residual of water condenses in coherent turbulent updrafts.

Difficulty in these limited area simulations is that they have typically have not
respected the energy flux balances previously, such that you get an exponential
growth in atmospheric water vapor.

PyCLES code.
Anelastic Navier-Stokes, discontinuity capturing (WENO) advection schemes.

Every climate model has three subscale schemes. Deep convection, shallow convection,
boundary layer turbulence. Parametric and structual discontinuities for processes with
common limits.

Reduce number of free parameters by using adiabatically conserved variables.
Let Coherent structures interact with the isotropic part (mean flow)? But do
not allow coherent structures to interact with each other.

New EDMF scheme, have draft equations and grid-mean equations.

\item[Koopman Coalition]
There is a connection between Koopman and using periodic orbits to
compute manifolds. How can we go beyond doing simulations and
looking at good modes?

Hyperbolic objects as opposed to attracting sets? Can we use
this for transient.

How to tame the continuous spectrum, in order to not just
do point spectrum, need Colm's input and need
functional space to be "correct" somehow.

\textit{Transition to turbulence: highway through the edge of chaos is charted
by Koopman modes}
Plane-couette flow: potential to be turbulent, what is the minimal energy state
and its structure that allows you to transition to turbulence.
Shortest distance to the edge manifold (energy norm). Follow edge state's stable
manifold and then get ejected to turbulence by unstable manifold.

Can do Koopman analysis around plateau of energy of perturbations, specifically
around the GMRES residual minimum. The point is to do the Koopman analysis
around the edge state.

The structure at this instant can almost completely be described by three modes.

Least-squares fitting of the amplitudes of modes that best reproduce the structure
of original structure.

The three modes that best describe the state are labeled "growth, neutral and decaying".
The neutral mode mataches the GMRES solution of the edge state, the growing mode
matches relatively well the evolution in time of DNS. The decaying mode matches simulation
backwards in time.

Minimal seeding accomplished by adjoint looping, look to Rich Kerswell papers for info.
Can accomplish the same with edge tracking.

Pick up a lot of continuous spectrum in different time segments because you're in the
basin of attraction of something else.

Can use pullbacks to linearize anywhere on basin of attraction. How do you expand
about the edge state as it isn't an attractor.

For saddles, define Koopman eigenfunctions about the saddle, can continue it as a strip
about the manifold via the dynamics. This foliation produces a distribution.
Can define bouncing between invariant solutions via the Koopman analysis.

Exponentials give rigorous expansion when dealing with autonomous system. Koopman
is non-normal in dissipative flows.

\item[M. Farge Talk]
Introduction by L. Tuckerman:
L' \'economie 101, inelasticity of journal prices.

Journals aren't evil, well, some are.
\HREF{http://dissem.in}{dessim.in}

\item[misc]
Almost done reviewing the more advanced concepts in c++
that I'm not so familiar with.

\end{description}
}

\MNGpost{2017-1-20}{:
\begin{description}
\item[I. Marusi\'c Talk]
\textit{Modeling high Reynolds number wall turbulence using the attached eddy hypothesis}
\HREF{http://online.kitp.ucsb.edu/online/transturb17/marusic/}{(video)}.
Using data and recent discoveries to test Townsend's attached eddy hypothesis.

Description of different regions of wall bounded flow, viscous region close to wall and
log-region and beyond where attached eddy models are defined (viscous not leading order).

Demonstrating that log region dominates at high $Re$.
Quotes from Townsend 1976. Perry and Chong (1982) claim that hierarchy or structures
lead to equations and log dependence of mean flow.

Inviscid structures can be predicted by inviscid equations, of course expect them to be invariant.
Continuous hierarchy of scales.
Geometric progression of scales. Telephone poles and dead cats attached to horizon.
Flip, shear, and squint and it looks like boundary layer.

"Are attached eddies real or just a statistical construct?"
Interpretation of eddies is energy contributing structures, organized
motion that contributes to kinetic energy. Contributions
to vorticity from attached eddies only give the mean, which is a very small
contribution.

Thinks that instead of looking at energy and enstrophy should look at instantaneous
changes in velocity fields. Thinks of logarithmic layer as stair case that jitters around
a line. Organized velocity fields that have vorticity associated with them but do not
contribute to the enstrophy.

Count the number of realizations of momentum zones, average over Reynolds number. The number
of momentum zones follows logarithmic growth.

Take hierarchical approach, keep dividing by scale factor and rescale velocities of demarcated
structure, randomly scatter them about and look at mean velocity field. Get logarithmic
scaling of the number of momentum zones that agree with DNS, claim that the requirement
is that the objects are self-similar.

Looking at spatial statistics, central limit theorem, poisson statistics, definitions of
moments and cumulants.

Any statistic with a wall-normal part, no logarithm. No wall-normal part means logarithm.
Correlation statistics asymptotically approach Gaussian distribution in limit of high $Re$,
much like Poisson statistics.

Problems arise from treating everything as point processes that have no spatial exclusion
with hierarchical structures of different scales.

\item[J. Jimenez]
\textit{Can we see coherent structures in wall-bounded turbulence?}
\HREF{http://online.kitp.ucsb.edu/online/transturb17/jimenez/}{(video)}.
Can we relate turbulence structures to coherent structures that people have found?
Structures to javier are autonomous identities that determine their futures.

Even if most of the flow may be structureless, it's important to identify key structures
because they are useful. (Seeing a storm cloud, even an isolated one, means you should bring an umbrella).

Javier's requirements:
\begin{itemize}
\item Strength (Stronger than background)
\item Observable
\item Relevant
\item Either: Energy production or sink, or energy repository.
\end{itemize}

Using joint pdf (pdf in two variables) can find important structures by quadrant analysis.
How does one look for these things? Need to look at where things are of comparable scale.
You do not want to mix energy and vorticity. How do you separate energy and dissipation or vorticity.

Energy produced by eddies has to be coupled by the shear. Corrsin parameter.
There is a characteristic time, nonlinear eddy turnover time, and shear deformation time.
Corrsin parameter is the ratio of turnover time to shear deformation time. Shear dominated,
means "linear" energy input. Corrsin parameter of order unity is "the playground" and of
very small means shear independence, nonlinear, no energy input.

Near wall, reynolds stresses small so has to be carried by shear.
By looking at phase velocity of velocity field, minimizing a wave equation relation
you can get phase velocity of individual Fourier components which corresponds to different
distances from the wall.

Stratification of spectra of different components of phase velocity
are similar when subsets of scales are viewed.
Can plot the scaled derivatives with respect to distance to the wall to decompose into
structural regions, from regions of large and small scale structures to transient areas.

Small scale structures have to do intimately with the near-wall viscous region, large scales
are linked to the bulk phase velocity, log layer is the transient regions.

Looking down (spanwise streamwise representation) can get the energy contained in a small
subsection that represents the energy contained in the viscous spectral subregion, and then plotting
vs. the wall normal coordinate leads to a maximum velocity intensity of 11.

Doing the same with the large-scale region, agrees quite well with highest POD mode of the velocity.
i.e. the phase velocity.

Linear transient growth analysis of the flow yields the same result.
Doing the same with logarithmic layer gives good agreement with the mean profile and the advection
velocity, as well as the phase velocity profile as a function of wall normal coordinate.

Things are similiar in the log layer but outside of this the large structures are not self-similar
when you get to the scale of the channel.

Structures bursting. Measured by quick changes in energy production and dissipation, has nothing to
to with the wall.

A theory for tilting things, Linearised Orr-Sommerfeld equation (inviscid Orr).
As the tilting angle changes continuously, get build up of the wall-normal component of the velocity.

Video demonstrating the connection between tilting and growth of wall
normal velocity component.
\end{description}
}

\MNGpost{2017-01-23}{:
\begin{description}
\item[G. Chini Talk]
\textit{An asymptotic theory for coupled uniform momentum zones
and vortical fissures in turbulent wall flows}
\HREF{http://online.kitp.ucsb.edu/online/transturb17/chini2/}{(video)}.
What is the role of invariant solutions in the limit of very large $Re_{\tau}$
Lower branch states seem to be important for transitional turbulence.
Might be able to explain near-wall coherence by these invariant solutions
because they only see a moderate effective $Re_{\tau}$.

Far from the wall, quasi-coherent motions with streamwise length scales approximately
5-15h are observed at high $Re_{\tau}$. Are these associated with ECS?

Hwang and Cossu say yes, with their Smagorinsky constant continuation, others who
claim that large scales come by via small scale hairpin vortex trains advected by
the flow.

Reference to instantaneous velocity profiles of Marusi\'c, decomposition into
uniform momentum zones and vortical fissures.

Can invariant solutions explain these instantaneous velocity profiles near the
wall?

Implications of mean momentum balance analysis. Long-time averaged x-momentum equation
has mean viscous term, turbulent inertia term (gradient of Reynolds
stress), and pressure gradient term.

The claim is that the momentum zone jumps correspond to regions of vorticity, or
vortical fissures.

Want to construct invariant solutions that somehow explain these momentum zones and
velocity profiles.

Tracking the lower branch solution to high Reynolds number, and it also converges
to solution of Vortex-Wave Interaction equations. Looking at Fourier analysis
and scalings of different Fourier modes.

When looking at the streamwise averaged streamwise momentum equation using VWI
scalings you get an effective Reynolds number felt by mean flow is order unity.

Asymptotic analyses of do not show sharp gradients of the rolls and streaks.

Reynolds number continuation of upper branch states leads to converges to lower
branch scaling.

Wants an invariant solution whose mean flow looks like instantaneous profiles.
They get more and more unstable directions but also get more isolated.
Reference to Clancy Rowley's previous work into scaling with symmetries involved,
but cannot do this with mean values.

Do local analysis of vortical fissures, with periodic boundary conditions.
Does Navier-Stokes admit a solution like this? Decompose into mean and fluctuation
term. Two different time scales. Why not two spatial scales? There could be but
for here there is not.

Mean-flows are isolated points in {\statesp} that mean nothing in terms of the dynamics.
Streamwise fluctuations are not of order $u_{\tau}$, thinks that this implies that there
should be discontinuities in the mean flow.

Dennice: There's a physical reason why it's ok to do this, looking at streamwise vortices,
so it seems that it might be ok to average in the streamwise direction to produce a mean
because these vortices are only z and y dependent??

Assume that the scaling of the average is s.t. the mean scales with inverse Reynolds, but
not as strongly as $1/Re$. Employ two-timing, Parse Navier-Stokes into mean and fluctuation equations.

Reduced PDE equations for the mean, fluctuations. Fluctuation equation is quasilinear.
This implies they admit separable and single mode (traveling wave) solutions in x. No direct
coupling between fundamental mode and higher harmonics.
Gradients of mean valued things are small, because of the assumption that they are smooth.

Linear stability analysis of Orr-Sommerfeld to help deduce scaling arguments for fluctuation
terms. Can do this because this is a steady state solution? It's a local, steady state solution?

Reynolds stress not driving the mean-flow seen via the streamwise Fourier transformed quantities,
of rescaled velocity components.

Can restore three dimensional incompressibility because of how the critical layer scales velocities.
Insist spanwise fluctuations drive mean flow?

Get critical layer variants of mean equations and fluctuation equations.
\item[Turbulence \`a la D. Barkley]
A nice recap of the key concepts and ideas; didn't type down much because eating took
priority. Key concepts: Turbulence and scaling laws, dynamical systems approach, manifolds of invariant solutions, self-sustaining
processes, transitional turbulence, edge states, puffs, splitting of puffs and growth and decay rates, directed percolation picture.

\item[Reading]
Started through Blonigan, Wang \etal\
papers\rf{WGBGQ13,BloWan13,BloWan15,BloWan16}. Interesting, I find
\refref{WGBGQ13} similar to variational {\descent}. I'm not really
sure how the spatial evolution is included, and if you look at their
converged solutions they are seemingly on the same spatial domain size as
the initial condition; I might just be missing something that I need to
reread. The ability to parallelize is of course a large distinction, and
the continuation through their advection velocity parameter is distinct
as well.

\end{description}
}

\MNGpost{2017-1-24}{:
\begin{description}
\item[T. Schneider]
\textit{From turbulence transition to shell buckling}
\HREF{http://online.kitp.ucsb.edu/online/transturb17/schneider2/}{(video)}.
Look at when prebuckled state loses linear stability, to see when buckling occurs.
Extreme sensitivity to inhomogeneity, imperfections.

See this via thin shell structures. Studied in reality via axially
loaded cylinders and sphere under uniform pressure.

Classical approach of understanding, bifurcation diagram of shortening
versus load parameter. Linear instability of the imperfect system,
given imperfections can predict buckling load. Of course,
typically imperfections are not known a priori.

Instead of looking at linear instability of the imperfect system,
look at the nonlinear finite amplitude instability of the perfect system.

Characterize basin of attraction as a function of load, get unstable equilibria
edge states in some norm, etc.

Geometric nonlinearity in elastic systems.
Displacement fields in terms of Lagrangian(fixed in material) coordinates.
get change in distance between material points, Green-Lagrange strain tensor,
typically nonlinear function of displacement field, usually linearized.

Relation for cauchy stress tensor for isotropic Hookean material,
relation to original coordinates?

Donnell-Mushtari-Vlassov theory, for deformations of thin elastic shell.
Find in-plane stress of mid-plane to get the nonlinear term that matters,
which is the one who only depends on normal direction of midplane.

Leads to DMV equations. Nonlinear and nonlocal equations like Navier-Stokes.
Want to construct nonlinear equilibria on the basin boundary of the unbuckled state.

Use edge-tracking. the edge state equilibrium is localized dimple on the surface.
Do continuation in axial load parameter. Leads to shrink of depth of dimple,
at zero the eigenmodes are not localized. Dimple pattern grows and wraps around shell.

Azimuthal symmetry, impose reflection invariance, could have also imposed translational
invariance.

Very high number of modes that become unstable at the same parameter value. Increasing
axial length induces snaking along axial direction.

Finite amplitude perturbations of a shell in fluids. Taking a perfect solution and
use perturbations that are known, much like in elastic materials.

Finding edge state with claim that the force should increase, and then return to zero
as you reach the edge-state equilibrium.

\item[S. Bagheri Talk]
\textit{Flow of non-smooth surfaces}
\HREF{http://online.kitp.ucsb.edu/online/transturb17/bagheri/}{(video)}.
Outline:
\begin{itemize}
\item Find favorable fluid-structure interaction mechanisms
\item Effective boundary conditions induces by non-smooth surfaces
\item Fabrication of surfaces and experiments
\end{itemize}

Fiber in cylinder wake breaks symmetry, induces lift and reduces drag.
Filament in oscillating flow, symmetry breaking in intermediate pulsation range.

Look at collection of filaments. Need to do coarse grained effective representation,
such that the average interactions are the same.

Take effective boundary conditions generated by poroelastic surface or reduce to
porous surface. Leads to two second order tensor relations for boundary
conditions on the velocity components taking the place of no-slip.

Streamwise velocity profiles shifted up or down based on the roughness in
the form of slip-lengths.

Wall tangential velocity condition
If structures are small, only contributions to flow are linear shear in the
viscous layer. If the structures extended into the buffer layer, this would
not be the case.

Related to drag reduction via a skin friction coefficient equation, looks like
dispersion relation?

Wall normal velocity condition
Permeability of interior related to boundary conditions on the interface, not
completely intuitive.

In this porous version of the equations, need to solve continuity equation
inside the coating to account for permeability.

How does one generate the two different tensors, $\mathbf{M}$ and $\mathbf{K}$.

DNS of two dimensional model, with porous media. Look at streamwise velocity
close to the interface to identify the slip. FreeFEM plus to do DNS.

Use coarse grained representation to derive the same quantities.
Take unit cell, with multiscale expansion of microscale equations.
Solving leading order equations in unit cell, slip-length comes out. Get
similar slip-length as DNS.

Coupling porous wall boundary conditions to turbulent channel flow, seeing
how it affects near-wall structures.

Experimentally, use UV lithography on polymers to induce roughness.

Inclined water table experiments. PIV measurements and comparison
to smooth wall setup leads to a smaller velocity near the wall and
higher velocity away from the wall (log-layer).

There is a decision that needs to be made of where to decide where the "wall"
is. Currently being taken to be the top of the filaments.

Would like to use filaments as actuators and sensors once studied.

\item[Channelflow]
Began writing some code and chipping away at the checklist I wrote for myself.
Currently trying to figure out how I'm going to numerically obtain small-time
\jacobianMs\ in a reasonable amount of time. I've been looking through the
Channelflow documentation as to use the resources at hand to the fullest extent.
The main problem is that the top-level machinery is all relatively easy but
getting into the guts is somewhat daunting considering the sheer number of
amounts of parts. One step at a time is the name of the game. Thinking of fine-grained
additions I can make to the checklist previously written, again, hopefully the
organization will help in this effort.

\end{description}
}
\MNGpost{2017-1-25}{:
\begin{description}
\item[M. Kiewat Brown Bag talk]
Looking to some how get more yield out of currently unused data created
by their simulations. Using Large eddy simulation and some other methods for
DNS.

Pressure induced drag accounts for most of the drag, but still look at vortex
induced drag.

DMD used but due to intense memory requirements. Discussion from much
of Clancy's talk, streaming DMD missed out.

\item[A. Willis Brown bag talk]
Have \reqva\ , have dynamics so therefore have trajectory. Compose
group orbit with shifts. Interested in association of group orbit
members?

Begin with template point on the group orbit via symmetry reduced point.
Looking at minimal distance between subsequent time snapsnots of template?
Slice condition $< \bar{a} - a'|t'> = 0$, locally $|| g(\theta)\bar{a} || = constant$ s.t.
$<\bar{a}|t'>=0$. this leads to slice condition $<\bar{a}|t'> = 0$. Temporal evolution
equivalent to shifting in space for traveling wave, so can relate.

dynamics within slice using defintions of $\bar{a}$ and symmetry invariant velocity,
$\dot{\bar{a}} = v(\bar{a}) - \dot{\theta} \bar{t}$, slice condition $<\bar{a}|\bar{t}'> =0$
get condition for $\dot{\theta} = <v(\bar{a})|\bar{t}'>/<\bar{t}|\bar{t}'>$

Complicated group orbit, get lifting out of the slice which makes the denominator
equal to zero, which is a problem.

Method of connections (not slicing)
put $\bar{t}' = \bar{t} = t$
Projecting out motion along the group orbit. Dynamics from $a_0$ to $a_1$, project.
dynamics back to $a_0$, means different tangent vector so projection leads to a phase
change such that you aren't brought back to original point, phase change is
known as geometric phase.

Principle component analysis. Reducing shift and reflect symmetry. Two clouds
which are related to shift and reflect. Zooming in on one of the clouds and
using Poincar\'e sections. Get ergodic clouds on Poincar\'e section, zooming
in around dense areas get connection between period doubling by looking at manifold.

\item[Excitable Media Discussion]
\HREF{http://online.kitp.ucsb.edu/online/transturb17/excitability/}{(video)}.
R. Grigoriev beginning. Spiral wave chaos: tiling, local symmetries, and
asymptotic freedom. Phase singularity at spiral core.

Model of cardiac dynamics, reaction-diffusion system. Karma model.
Forget about diffusion initially by ignoring spatial properties connections between
cells.

Looking at nullclines and equilibria. Apply dynamical systems views to the problem,
have varieties of solutions, look for periodic/relative equilibria on chaotic attractor.
Use recurrence plot to initiate search for invariant solutions (exact coherence structures).
Double periodicity includes symmetry operations (translation, finite rotation). Use
minima as initial guesses for Newton-Krylov solver. Notes on symmetry and symmetry breaking,
two dimensional planar model is better model of atria as opposed to ventricles.

Some intial conditions converge to small residuals but not non-zero residuals. Why? THere
is structure that does not disappear after certain time.

\CGLe\ discussion. Split into amplitude and phase (its complex so easily done). The amplitude
has very long time dynamics that slows down over time. Real part is fast dynamics. In amplitude
diagram there are boundaries where the interiors each support one spiral.

Use enclosed area of trajectories instead of amplitude, Can find the elapsed time from crossing a
Poincare section instead of phase.

\CGLe\ is general form for any system that goes through oscillatory instability, i.e. hopf bifurcation
with zero wavenumber.

Can describe tile boundaries analytically. Can describe dynamics by imposing Neumann boundary conditions
at the edges of tiles. Boundaries meander after time evolution.

Meandering of spiral cores is possible but not necessary. Meandering, stable, alternans type cores depending on
parameter. in \CGLe\ the cores all have the same frequency and you would have periodic orbit.

Break-up, drift, and collapse.
Looking at the Floquet multipliers of spiral wave. Multipliers are mostly inside the unit circle in the
complex plane but there are a few multipliers that corresponds to absolute instability. Pair of multipliers
near the boundary of the unit circle correspond to convective instability.

Anternans instability, action potential with multiple length scales. Refractory (relaxation period where it
can not be excited). Breaking up of a plane wave into spiral waves.

Looking at Stroboscopic map (time measured in periods). Get core drifting, tile deformation. In order
to understand this, can show that periods are exponentially decaying functions of linear scale. Matching
phase at tile boundaries, moving boundaries depending on spiral frequencies. Small tiles become larger,
large tile become smaller. Unless tiles are all the same size, cannot have periodic solutions.

Imposing boundaries break global symmetry, right eigenfunctions (goldstone modes), adjoint eigenfunctions
are exponentially localized. Symmetry is perfect on the interior but gets fuzzy near the boundary.

Midway conclusion: do not expect to find periodic solutions. On a small enough domain there is an amount
of spatiotemporal coherence. You will never find solutions on large domains because of the lack of coherent.

Dwight's work of applying oscillating field with near resonant frequency that makes them synchronize.


\item[Channelflow]
Looking through the programs to see what I can incorporate from what John has already written. I'm going to
see if he has time tomorrow to talk through things.
\end{description}
}

\MNGpost{2017-01-26}{:
\begin{description}
\item[M. Graham Talk]
\HREF{http://online.kitp.ucsb.edu/online/transturb17/graham/}{(video)}.
Model turbulent drag reduction in polymer solutions. Polymers reduce drag.
Maxiumum drag reduction asymptote. Asymptotic curve between poiseuille limit and Prandtl-von Karman limit.
Log-law for polymers, Virk law.

Long linear polymers work best, when dissolved in water they turn into a long brownian motion like path.
Relaxation time, time it takes to approximately diffuse its own length.
Weissenberg number, which describes:
Polymer chains tumble in shear flows such that they have little effect on viscosity. In extensional flows,
polymer chains stretch, "Lagrangian chaotic velocity fields are extensional". Hyperbolic flows = extensional?

Intermittent dynamics for Newtonian and moderate Weissenberg number, active and hibernating turbulence.
Area averaged velocity profile switches between quiescent periods near Virk limit and von Karman limit.

Polymer stretching generates torque which works to undo streamwise vortices.
Estimate on polymer stress using Lyapunov exponents. Believes that active turbulence is amount of
time that the polymer stress remains below a critical threshold.

There is a family of solutions that have the property that within a certain parameter range, the
upper branch solution is near von Karman and the lower branch is near Virk.

Could do Newton searches for viscoelastic solutions but hasn't been done
yet, edge-tracking viscoelastic has been done.

At this $Re$ the hypothesis is that polymer effect is to increase the Weissenberg number, which brings
the states closer to the edge state in some sense.

Looking at spatial variation of drag, Detector functions vs. clustering methods, k-means clustering or
Otsu's method. Once you have the regions you can do conditional averaging to get mean profiles. Mean
profiles of these low drag regions, low-speed streaks are similar to lower branch coherent states.

\item[Y. Duguet]
\textit{Turbulent bifurcations in intermittent shear flows: From puffs to stripes}
\HREF{http://online.kitp.ucsb.edu/online/transturb17/duguet/}{(video)}.
Shear flows with linearly stable laminar base flows implies a possibility of sustained
laminar turbulent coexistence. Exposition of D. Barkley's work on expanding slugs.

Exposition of laminar, intermittent and turbulent regions of $Re$ scales for a variety
of geometries.

In 2D plane Couette solutions, look at 2D spectrum, there appears to be a gap in the spectrum
so they used two Gaussian filters to look at large length scale and small length scales separately.

Investigation of overhang regions (preference towards one of two channel flows at edges of puffs).
By mass conservation, there is a constraint includes streamwise velocity deviation and spanwise velocity
deviation w.r.t spanwise. Therefore because there is streamwise velocity variation, there must be
spanwise velocity variation. Interfaces must have oblique growth.

Why do spots organize into stripes, they also conserve mass.

Homotopy with a new parameter which is the aspect ratio of spanwise to wall normal length scale,
in this case things die out near the walls so it doensn't really help show puffs to stripes.

New homotopy, based on annular pipe flow. New homotopy parameter which is the radius ratio
in pipeflow. Relevant length unit is the cylinder gap, the difference between radii. The
dynamical parameter is frictional Reynolds number. Even though the asymptotic limit
is not Hagen-Poiseuille flow, want to show that the same phenomenology exists.

Showing results of simulations at various values of new homotopy parameter. Changing
from helical (oblique on unwrapped cylinder) structures to spatially localized
puffs.

Same mass budgeting as before but now in cylindrical coordinates.
Looking at spectra of each direction. Graphs of length scales as
functions of the homotopy parameter. Get regions of different behaviors
of confinement. The regions are strong confinement at both walls,
confinement only at inner wall, and no confinement.

Statistics, generate PDFs after filtering. PDFs of azimuthal velocities
and angles. Bifurcation diagrams of statistical modes (maximum of PDF).

Goldenfeld hypothesized that it is crucial to think about zonal flows
in a pipe, but Yohann thinks that they are of no statistical importance
due to the lack of peaks in his PDFs.

\item[P. Cvitanovi\'c Math Colloquium]

\item[periodic eigendecomposition]
Plan to meet to with J. Gibson tomorrow, undisclosed time based on his schedule.
\item[Misc.]
Listened to C. Rowley, S. Bagheri, P. Cvitanovi\'c talk about Koopman,
singular eigenfunctions of Koopman operator.
\end{description}
}

\MNGpost{2017-1-29}{
\begin{description}
\textit{Physics of puffs and slugs}
\item[D. Barkley]
\HREF{http://online.kitp.ucsb.edu/online/transturb17/barkley/}{(video)}.
Main mechanism that extracts energy from mean flow is nonlinear
interactions with velocity profile. $-\overline{uv}<u>$. Production
exceeds dissipation at interfacial upstream edge, slightly downstream
dissipation exceeds production, and later there is a region in equilibrium.

In comoving frame, the interfacial edge moves upstream (source of confusion). Turning down Reynolds number serves to eliminate the
region of energy equilibrium. Describing the puffs as flame fronts,
making the analogy that the "flame" is moving upstream, but because
it is also being advected by the flow it leads to confusion. Fluid
parcels are constantly flowing through the interfacial upstream edge,
and are being "entrained" by the structure of the puff. Likewise
for the downstream front, fluid is constantly leaving the downstream
interface and relaminarizing. Can get entrainment at both edges.

\item[M. Avila]
Movies of various slugs, much discussion on the upstream and downstream edges and why you should treat them the same. Dwight
claims that this is because of the entrainment that occurs at both
ends. Roman Grigoriev bringing up the transport of momentum at the
interfaces.

Investigations using Lagrangian frame moving with a patch of isoenstropy surface.

\item[J. Gibson]
Discussed periodic eigendecomposition with John, first question that was
brought up was whether periodic Schur decomposition is required, as opposed
to performing Arnoldi iteration of periodic orbits to get Floquet vectors.

Most of the discussion was both of us trying to reconcile our opinions and
how we both are thinking of the problem in the context of \refref{DCTSCD14}.
\end{description}
}

\MNGpost{2017-1-31}{
\begin{description}
\item[P. Hall]
\textit{Canonical exact coherent structures: The emergence of distributed states
and the Law of the Wall}
\HREF{http://online.kitp.ucsb.edu/online/transturb17/hall/}{(video)}.

Looking at structures as $Re$ goes to infinity gives insight on finite $Re$
solutions.

Looking at two non-localized states for arbitrary shear flows, they are the
foundation for producing families of solutions.
Type 1: (Self sustained processes) Vortex-wave interactions.
Type 2: Freestream coherent structures. Relevant to transition induced
by free-stream turbulence.

VWI reduction applied to asymptotics. $3D$ time dependent navier stokes
transformed to 2d steady navier stokes, convection diffusion, eigenvalue
problem.

Logarithmic jump in pressure when you cross the critical layer, unless
the critical layer is flat.

Three time scales for VWI.
1. slow roll-streak diffusion timescale t
2. fast inviscid wave timescale $Re*t$
3. birth death timescale for vwi states $Re^{1/2}*t$

Drag vs. alpha bifurcation diagram.
Looking at VWI solutions and their Reynolds stresses, which for lower
branch solutions concentrate in critical region. For upper branch
solutions they begin in the critical layer but after continuation
they begin to become exotic, but this can be explained by asymptotics.

Looking at bifurcations of upper and lower branch solutions show
why upperbranch is so much more unstable than lower branch.
Instabilities on three different time scales.

Rescaling by taking streamwise and spanwise wavenumbers

Generate new solution by introducing spatial periodicity, by stacking
solutions on top of each other.
where interactions depend on the spatial period.

Get a time periodic solution from this stacking procedure due to
single mode waves?

Defining the infinite array computational problem.

Starting with spatially periodic solution in Couette flow,
generalizes to arbitrary shear flow, but only in flows that
have a logarithmic profile.

Locally looks like Couette flow locally, but phase speeds slowly
changes similar to WKB, eikonal equation.

\item[R. Monaco]
\textit{Convection: from small plumes to large coherent structures}
\HREF{http://online.kitp.ucsb.edu/online/transturb17/monaco/}{(video)}.
Taylor-Couette and Rayleigh-Benard.
Transition turbulence of Taylor-Couette flow.
Conflicting evidence of Taylor-Couette flows, showing large
scale structures can persist conflicting with previous
experiments.

Azimuthal and time averaging. Aspect ratios are different
but can explain apparently.

Streamwise velocity behavior in large variety of flows follows
Virk log-law.

Adding riblets doesn't really change structure in bulk.

Turning off rotation, rolls persist for relatively large times.
Changing behavior of rotation, corotating, outer rotating, etc.

Rayleigh benard flow, boundary conditions changing the nature
of solutions dramatically.

mixed temperature boundary conditions seemingly have no effect, by
means of different geometries of adiabatic and conducting regions at
boundaries.

\item[Spatial KS]
Had the idea to involve Galilean invariance in spatial integration, i.e.
somehow formulate  $0 = \int u(\conf, \zeit) d\conf$ into a local constraint
that can be used to constrain the integration to the repeller. Burak thinks
is is definitely something worth pursuing.

\item[Shadowing Lemma]
Read some papers, had short discussion with PC, I agree it isn't too
informative and takes too much time to get anything worthwhile out.

\item[Variational methods]
Discussion about the general practice,
Why didn't we use Fourier instead of finite differences, using
fourier would be diagonal, instead of having cyclic border terms.
How to deal with continuous families of tori still needs to be
explained. Roman gave me contact of another GT student working
on such methods.

\item[Num Lin Alg]
Review of QR, arnoldi iteration, GMRES. Will begin using arnoldi
iteration on periodic orbits from J. Gibson's repository and then
comparing angles tomorrow.

\end{description}
}

\MNGpost{2017-02-01}{
\begin{description}
\item[Inertial Manifold]
Used Channelflow's arnoldi iteration on two periodic orbits of plane couette, I am
a little confused as to how the eigenfunctions are represented as output so I wasn't
able to produce any pairwise angles or norms yet. I spent too much time digging through
the code with little benefit; although I hate bugging John with these types of questions it's
probably best to just ask.

\item[Lagrangian Formulation of KS]
Poked through literature on Lagrangian formulations of partial differential
equations tt seems that it is relatively common to formulate these things for KdV equation
but not really any other equations.

\item[KS spatial integration]
Looked around for a way to follow up on the idea of using Galilean invariance as a
constraint condition for spatial integration, haven't found or come up with any local
condition that could be applied as of yet.

\item[Avila meeting]
Shared current projects and work with Marc Avila whom was curious since he saw
my poster. Discussed spatial integration, variational methods, invariant spatiotemporal
tori and the benefits of variational methods versus shooting methods.

He was interested in using spatial integration to reconstruct boundary layer information
from PIV data. He also suggested using a test case of applying the spatiotemporal
variational method that we are trying to deploy to a case where a stable torus exists and
then see if one could track tori and the breakdown of tori.

\item[Li Xi]
viscoelastic flow, Weissenberg number better than Deborah number because it's more physical.
Comparison of relaxation time vs. deformation time (shear rate).

If time scale of polymer contraction is slow relative to flow, then can introduce new
nonlinearity that consists of how the polymer reacts to the "new" dynamically evolved flow.

In general $Re$ and weissenberg number $Wi$ are changed via changing the flow rate, which
simultaneously changes both.

In channel flow, there tends to be a proportionality of $Wi$ and the transition to turbulence.
"above" this transition, only get drag reduction above a critical $Wi$ number.

Velocity of FENE p models is the same as experimental data, but otherwise results are qualitative.
depiction of parameter space.

Description of "phase transition" between LDR and HDR regime. In LDR the velocity profile
in the buffer layer increases in magnitude, until the HDR regime where the slope of the
velocity profile in the log layer increases, so there are obviously two different
processes occuring. He thinks that vortices burst/break up less such that the flow is more
coherent.

Wants to identify the time-scale that identifies the transition.

\item[Ismail Hameduddin]
Continuation of viscoelastic talks. Mathematics of viscoelastic modeling.
Get a linear equation for the elements of the conformation tensor under
some assumptions.

\end{description}
}

\MNGpost{2017-02-02}{:
\begin{description}
\item[J. Gibson]
\textit{Julia: the future of scientific computing}
\HREF{http://online.kitp.ucsb.edu/online/transturb17/gibson/}{(video)}.
Combines best properties of Python, Matlab, LISP, and C.
Demonstration of Julia through notebooks via jupyter.
c
\item[N. Goldenfeld]
\textit{Statistical mechanics of the phase transition to turbulence: zonal flows,
ecological collapse and extreme value statistics}
\HREF{http://online.kitp.ucsb.edu/online/transturb17/gibson/}{(video)}.
Statistical approach to turbulence. Going from finite number of particles
in a box, use statistical mechanics to explain equilibrium behavior. Claim
is that the same is true for an infinite number of particles in a box (fluid)

Comparison of puff splitting decay mean time plot to predator-prey model
as a function of prey birth rate. Does extinction equal turbulence?

There is a possibility that turbulence is a long lived transient, changed
after Hof's experiment that measured the survival probability of puffs.

Believes that superexponential scaling is the correct interpretation.
Relating metastable puffs to directed percolation. Four fundamental
processes.

Phase transition characterized by universal exponents, as functions of
parameter p, the probability of continuing.

Directed percolation done spatiotemporally, using correlation length
as a means of generating scales in the problem. Do statistics on these
simulations.

Reasoning behind super exponential scaling and extreme statistics.
Mean follows central limit theorem, maximum follows three types of
distributions. Probability the largest fluctuation exceeds $Re$
threshold then decay superexponentially. Looking at statistics

The reason you expect super exponential decay, is the means by which
decay is measured.

Once correlation length gets to the order of the spatial extent, the
statistics switch from superexponential to power law.

Logic of modeling phase transitions
\begin{itemize}
\item Magnets
\item Electronic structure
\item Ising Model
\item Landau Theory
\item Renormalization group universality class

\item Turbulence
\item Kinetic Theory
\item Navier Stokes equation
\item ?
\item ?
\end{itemize}

Identification of collective modes at the laminar-turbulent transition,
similar to Landau theory only look at important structures near the
transition.

Zonal Flows. Computing instantaneous reynolds stress is anisotropic,
which flows into the walls, which drives the zonal flow. Zonal flow
inhibits reynolds stress (makes it more isotropic). Zonal flow isn't
driven by the mean flow because there is orthogonal to mean flow.

Comparison of zonal mode-turbulence and predator-prey models. Predator
prey models driven stochastically. Lotka-Volterra. Oscillatory effects
from satiation model. Fluctuations and oscillations produced by
demographic stochasticity.

Predator-prey path integrals and quantum field theory. Try to write
down a Landau theory for this.

Allowed field theoretical processes look like the same four fundamental
processes of directed percolation.

Demonstrated turbulence to directed percolation via predator prey and field
theory.

Superexponential vs. critical scaling.

Crutchfield and Kaneko

\item[B. Eckhardt Secret]
\textit{The transition to turbulence in shear flows: a dynamical systems perspective}

No video because it was held in South hall.
Sort of recap of previous presentation.
How fat is turbulence? Apparently as fat as an elephant.
Nikuradse in 1930s shows the frictional factor dependence on $Re$.

Intermittent case is defined by linear stability of laminar profile.
Expect smooth transition when you have subcritical bifurcation.
Coherent structures and {\statesp}. Create a new basin of attraction
via homoclinic bifurcation. Transition from attracting closed state
to fractal open set.

Crises for attracting set of one coherent structure isn't really enough
to explain statespace. Conglomerate of crises is sufficiently stable such that it is possible to never return to laminar.

\item[Arnoldi]
Producing eigenfunctions and Floquet exponents and multipliers for a range
of periodic orbits from John's repository, pinned down John for a quick talk on whether I'm getting reasonable values for exponents, etc. I was also confused on how the eigenfunctions were being represent, he somewhat
cleared it up and directed me towards matlab scripts to use to reproduce
fields as to begin taking norms and computing angles. Looking through matlab scripts.

\item[Readings]
The first part of Farrell et al. on generalized stability theory.
Horseshoes and blenders via

\item[KAM]
Even though I'm not employed by Rafael de la Llave I found myself sidetracked by
KAM theory to gain some insight into what is generally done with tori.
\end{description}
}

\MNGpost{2017-02-03}{
\begin{description}
\item[Jörg Schumacher]
\textit{Boundary layer dynamics and large-scale structure in turbulent
convection}
Announcement of Euromech colloquium.

Intro to thermal convection.
What are dynamics at bottom and sidewall boundary layer.
Going from laminar boundary layers to turbulent boundary layers,
can see that bulk is turbulent via Kolmogorov spectrum.

Always work with Boussinesq approximation, boundary layers for low and
high $Pr$, changes thickness of thermal boundary layer.

Numerical simulations for liquid mercury required to be parallelizable.

Skin friction field at bottom is complex even in the laminar boundary layer regime. Seems like avoiding calling it turbulent. Looks turbulent to
me.

One can remove the precession of "impact" and "ejection" zones (convection cells component on bottom plate) by proposed coordinate transformation.

Pure fluid turbulent cascade is extended with Prandtl number.
Trying to understand velocity profile in boundary layers, no answers for now?

Large-scale order in convection. Large aspect ratios.

Time averaging at a higher Reynolds number makes results look like lower
$Re$.

Koopman analysis projected back into {\statesp}, grouped around these modes.

Don't lose structure from lower Prandtl numbers under continuation.

\item[P. Arratia Talk]
Flow of polymeric solutions. High shear serves to unwrap polymers.
Reduced navier stokes equations, introduce new nonlinearity with terms that describe the polymers interaction with the fluid. FENE-p models.

Flows that stretch polymers really well, high velocity gradients.
Role of curvature, curved streamlines lead to "hoop stresses". Taylor
Couette flow is linearly unstable even at zero $Re$.

Schematic for transition in viscoelastic similar to Newtonian fluids.

Is there a nonlinear instability to laminar viscoelastic flow at low $Re$?
Discussion of global instability, once you have finite perturbations, no longer talking about linear stability.

Experimental setup, cylinders, subcritical instability demonstrated by
graphs of velocity fluctuations vs $Wi$ for different numbers of cylinders. Saturation implies subcritical instability, hysteresis
in $Wi$ continuation seems to confirm this.

Little bit of discussion of velocity profile in the middle of the channel, as there was confusion on how one of the graphs was being produced.
\end{description}
}

\MNGpost{2017-02-06}{
\begin{description}
\item[Channelflow]
Learning how to write bash shell scripts to automate the process of using
Channelflow commands to compute generate time discretization
of periodic orbits, arnoldi iterations for eigenfunctions, and pairwise angles
of eigenfunctions.
\item[Spatial integration]
Spent more time trying to figure out if I can
use galilean invariance as an integration constraint.
\item[Spatiotemporal parallelization debriefing]
Debriefing of \refref{WGBGQ13} with Ashley Willis taking the lead. Discussion over coffee
he said that he would work through some calculations and then let me compare it to
variational {\descent}.
\item[What would Koopman do]
Listened to Igor Mezi\'c discuss Koopmanism and answer questions from Mohammad Farazmand,
R. Grigoriev, M. Schatz.

\end{description}
}

\MNGpost{2016-2-7}{
\begin{description}
\item[C. Caulfield]
\textit{Telling the time, using the clockwork of turbulence to answer open questions
in fluid dynamics}
What is shortest distance between edge manifold in energy norm, therefore could find
the smallest perturbation to transition to turbulence.

Linear optimal perturbation is not sufficient as it is streamwise invariant? Therefore
use a localized nonlinear perturbation.

Using variational methods that maximizes energy, makes sure that navier-stokes is
imposed for all space and time, imcompressibility, initial energy $c$ of perturbation.

Finding adjoint equations via integration by parts, can use Newton solver to solve
adjoint equation. It's like a backwards in time Newton-solver which doesn't blow
up due to using the adjoint variable.

Identify transition by the energy gain and optimal time.

Localization of initial condition has nothing to do with edge state it's only
to exploit the Orr mechanism. Edge state is a streamwise structure due to the box
size?

Use DNS results as initial condition for GMRES algorithm.
Use Koopman/DMD modes applied to snapshots. Looking at spectrum, relatively well
peaked. Three Koopman modes is enough to capture the DNS results near the energy plateau
region of time, otherwise known as the edge-state.

Discussion about whether this is linear or nonlinear.

Using adjoint somehow requires converge on arbitrarily long time, so adjoint methods.

Stratified flows. Get spontaneous layers that aren't taylor vortices, i.e. structures
that don't scale with the gap width.

Most linearly unstable mode connection to nonlinear interactions.
Recurrent structures in stratified flows.

\item[Rama Govindarajan Talk]
\textit{Droplet Dynamics in model vortices}
Heavy particles converge to regions of high strain and leave regions of high
vorticity, bubbles would do the opposite.

Behavior dependent on spatial position of particles relative to the vortex.
Polar coordinate equations that arise are parameter free, they would normally
take into account time dependence of length scale which is associated with both
the particles and the vortex core. The particles do not interact, do not affect
the flow.

Dynamical simulations show that vortices appear to behave as solid objects that
experience the Magnus effect. Can create vorticity through baroclinic torque.

Buoyancy definitely affects the size of structures that appear in 2-D DNS
simulations.

Looking towards clouds, general description of a self-sustaining process,
equations that now involve the vapor pressures and saturation and Stokes
number.

\item[Solving Turbulence Debrief]
Relation between fictitious time and time, seems to become blended
Discussions of variational {\descent} and how it's really crude.
Discussions about parallelization, no longer doing time integrations,
Parameterizing the space of curves in "fictitious time" isn't obvious
in this method.

How second order equation is sufficient for parallelization.
Looking at secord order equation that Ashley has written down,
the RHS looks like the current formulation of the variational {\descent}.

John seems to be implying that instead of the normal variational {\descent} matrix equation that arises from the fictitious time PDE, one
can use Krylov methods, PC doesn't see how to incorporate small steps
into this.

Is cost function always the $L2$ norm? The issue has to do with the area
where localization is valid, higher norms reduces this size.
Mathematicians like to use a $1+ \epsilon$ norm, which bypasses the
kinky structure of the one norm.

\item[Cost functional version of multishooting]
Ashley wants me to attempt to implement the minimization of the cost
functional \refeq{e-MultishootCost} which leads to the variational
equations (under the assumption that there is no symmetry operation
$g(\phi)$, which is the simplest case to take),
\refeq{e-MultishootVariational}.

The general procedure indicated by Ashley is that you shoot the reference points
i.e. the set of $x_i  i = 0 ... M-1$ to retrieve $y_i = f^{t}(x_i)$ and then
set the second of these variation equations equal to zero to get $w_{i}(T_i)$.
You use an adjoint equation for $w$ to pull back $w_{i}(T_i)$ to $w_{i}(0)$,
and then use the variations in $T_i$ and $x_i$ to update your reference state in
order to proceed until the cost functional described by \refeq{e-MultishootCost} is
minimized to desired cut off.

I'm going to apply this to antisymmetric subspace $\bbU^+$ of \KSe\ and
then work with Ashley to hopefully parallelize it.

I wrote a crude version by repurposing function definitions from
variational {\descent}, as well as the numerical time-stepping algorithm
from \refref{ks05com}.

The main part that is left is to derive the adjoint equation and decide on how
to evaluate it, and then include that in the crude version I have currently.
\end{description}
}

\MNGpost{2017-02-08}{

\begin{description}
\item[M. Gudorf Brown Bag]
Talked about spatial integration and variational methods.
B. Eckhardt brought up similarities to Burgers equation and
was concerned with how \KSe\ deals with shocks.

J. Jimenez was concerned about how well-posed the spatial integration
problem is and what we hope to get out of it
Shortly after the talk in the cafe next door,
B. Eckhardt brought up ideas very similar described $T=0$
steady state equation from \refref{DoLa14} about how the solutions
to the steady state equation (equilibria in time) form a very coarse
way to partition the time dynamics. He also mentioned that the talk was
good, he cited Divakar Viswanath and his approach of thinking
about the problem and adapting methods, and how this was crucially
important for results in Plane-Couette

J. Gibson thinks that the variational {\descent} is bad method
due to its number of steps and formulation as essentially a gradient
descent method. He thinks an alternative description with

He also brought up that although there are continuous symmetries in the
Plane-Couette problem, Newton-Krylov methods are well suited to picking out
solutions without ``sliding" around due to the continuous symmetries.

Other suggestions were to use Krylov methods in combination with
variational {\descent}.

\item[R. Grigoriev]
Excitability talk, review of spirals using Karma model.
Square boundary (no periodic boundary conditions)
, three marginal eigenvalues on square domain.
Discussion of relative equilibria on square domain. As long
as spiral is in center of domain it doesn't drift but when
offset it begins to drift.

Rotation and translation of square domains, don't overlap, so take
smaller circular domain in the interior of the intersection of rotated
square domains has some property? Sort of confused here, Roman said people
taking circular domains were lazy, but ends up having to do that in some
respect.
Once spirals are far from the walls, or far from each other,
exponential decay of response
functions tell you that they spiral-wall or spiral-spiral
pairs do not really interact anymore. The converse statement is
that small domain sizes will strongly impact the spiral core.

Spiral chaos and excitable systems can be decomposed into slow
dynamics of tiles and fast dynamics of spirals.

Implications for fluid turbulence, need to look at adjoint
and the role of pressure on localized solutions. The adjoint in
detail means the adjoint eigenfunctions about the linearization
of localized solutions.

Dwight Barkley explains that perturbations along the adjoint
of a certain eigenvector is what leads to perturbations along
the eigenvector desired. This is due to the fact that the adjoint
is orthogonal to the complement of the desired eigenvector.

\item[Variational multishooting]
I rederived the equations \refeq{e-MultishootCost} and \refeq{e-MultishootVariational}
in order to solidify the thought process. Ashley was kind enough to produce
the equation for the adjoint \refeq{e-msadjoint} which I also went through.

\item[A. Willis Coffee discussion]
Discussed variational methods with Ashley, and in detail discussed how I
implemented the variational {\descent}. I personally feel there is much
room for optimization, and that automation of generating initial conditions
would be a much sought after device for many people interested in exact
coherent solutions. One of his methods that differed from mine was using the
angle between successive descent directions as a means to control the step size.

\end{description}
}

\MNGpost{2017-02-09}{
\begin{description}
\item[M. Farazmand Talk]
\textit{Extreme events in turbulent flows: A variational approach}
Trying to explain intermittent bursts in energy dissipation by
explaining it away by saying there is an ECS that is visited albeit
rarely. The solution wasn't find by Newton-Krylov methods, so needed
another method. Chose to work with Adjoint method.

Lots of discussion on gradient methods, John really would like
to see comparisions with these variational methods with modified
Newton.

Looking at nonlinear energy transfers through by using a graph with Fourier modes
as the vertices.

Needs a new variational method,
\begin{itemize}
\item Classical case of these types of bursts is a symmetry breaking bifurcation. (Near-equilibrium systems)
\item Multi-scale methods, requires separation of slow and fast degrees of freedom
\item Theory of large deviations (multi-equilibrium driven by stochastic force)
\item Optimally time-dependent modes (Using Lyapunov exponents, predictive, not physics)
\end{itemize}

Looking for initial states that trigger extreme growth of observable. The idea
is that the solutions are in some submanifold of {\statesp}, and try to find states
that shoot you away from the submanifold. Look for states that maximize a time
delayed supremum of observable.

%A little bit more goes here I spent the remainder of talk focusing rather than writing. Will
%review video and update

\item[Modified Newton's Method]
Talked with John and Ashley about the method that John always likes to bring up when
the topic of the variational {\descent} is brought up. The only way to know
which is the best is to do testing it seems.

\item[Tower room]
Discussions about the program and where do we go from here. Javier Jimenez thinks
it is time to formulate interactions between different scales, i.e. multiscale theory.
Shell models, examples, and the history of turbulence were discussed.
\end{description}
}

\end{description}
%%%%%%%%%%%%%%%%%%%%%%%%%%%%%%%%%%%%%%%%%%%%%%%%%%%%%%%%%%%%%%%%%%%%%%%
\printbibliography[heading=subbibintoc,title={References}]
