% siminos/reversal/JacobianOrb.tex      pdflatex LC21; bibtex LC21
% temporary: siminos/spatiotemp/chapter/LC21JacobianOrb.tex
% $Author: predrag $ $Date: 2021-12-24 01:25:20 -0500 (Fri, 24 Dec 2021) $

\section{Orbit stability}
\label{s:JacobianOrb}

The {temporal lattice} reformulation gives us deep insights into how to
enumerate and determine all global solutions ({\lattstate}s) of
lattice field theories.

The discretized Euler–\-Lagrange
$F[\Xx_c]=0$ fixed point condition \refeq{LC21eqMotion} is central to the
theory of {global methods} for finding periodic orbits. Instead of
utilizing local, for\-ward-in-time numerical integrations, in
global multi-shooting,
collocation\rf{auto,GM00aut,ChoGuck99}, and Lindstedt-Poincar{\'e}\rf{DV02,DV03,DV04}
searches for \po s, one discretizes a \po\ into $\cl{}$
segments\rf{CvitLanCrete02,lanVar1,DingCvit14,DCTSCD14} temporal lattice
configuration,
and lists the field value at a point of each segment
\beq
\transp{\Xx}=(\ssp_0,\ssp_{1},\cdots,\ssp_{\cl{}-1})
\,.
\ee{LC21nXdCycle}
Starting with an initial guess for \Xx, the zero of function
$F[\Xx_c]$ can then be found by Newton iteration, which requires
an evaluation of the $[\cl{}\!\times\!\cl{}]$ \emph{\jacobianOrb}
\beq
\jMorb_{\zeit\zeit'} =\frac{\delta F[\Xx_c]_\zeit}{\delta \ssp_{\zeit'}}
\,.
\ee{jacobianOrb}

\subsection{Temporal Bernoulli, {\templatt}} % \jacobianOrb}
\label{s:JacobianOrbBern}
% joined with \subsection{{\tempLatt} \jacobianOrb}
% \label{s:tempCatJacobianOrb}

The {temporal Bernoulli} condition
\refeq{tempBern} and
the
{\templatt} discretized Euler–\-Lagrange equation \refeq{catTempLatt}
can be viewed as searches for zeros of the vector of
$\cl{}$ functions
\bea
F[\Xx_\Mm] &=& \jMorb\Xx-\Mm = 0
                \label{tempFixPoint}\\
&& \mbox{temporal Bernoulli: } \qquad \jMorb =  - {\shift} + {s}\id
                \label{bernFixPoint}\\
&& \mbox{\templatt: } \qquad\qquad\;\;\,    \jMorb =  -\shift + s\id - \shift^{-1}
                \label{tempCatFix}
\,,
\eea
with the entire periodic \emph{{\lattstate}} ${\Xx}_{\Mm}$ treated as a
single fixed \emph{point} $(\ssp_0,\ssp_{1},\cdots,\ssp_{\cl{}-1})$ in the
\cl{}\dmn\ \statesp\ unit hyper-cube $\Xx\in[0,1)^\cl{}$, and
the $[\cl{}\!\times\!\cl{}]$ {\jacobianOrb}  $\jMorb$.

The $[\cl{}\!\times\!\cl{}]$ {\jacobianOrb} $\jMorb$ is a
%tri-diagonal
Toeplitz matrix, \ie, matrix constant along each diagonal,
$\jMorb_{k\ell} = j_{k-\ell}$, of circulant form,
for the {temporal Bernoulli},
\beq
\jMorb %  = \shift - s\id + \shift^{-1}
  =
\left(\begin{array}{ccccccc}
{s}&-1 & 0 & 0 &\dots &0        & 0 \\
0 & {s}&-1 & 0 &\dots &0        & 0 \\
0 & 0 & {s}&-1 &\dots &0        & 0 \\
\vdots & \vdots&\vdots & \vdots & \ddots &\vdots &\vdots\\
 0 & 0 & 0     & 0     &\dots   & {s}&-1 \\
-1 & 0 & 0     & 0     &\dots& 0& {s}
        \end{array} \right)
\,.
\ee{bernJacOrb}
and for \templatt
\beq
\jMorb %  = \shift - s\id + \shift^{-1}
  =
\left(\begin{array}{ccccccc}
{s}&-1 & 0 & 0 &\dots &0&-1 \\
-1 & {s}&-1 & 0 &\dots &0&0 \\
0 &-1 & {s}&-1 &\dots &0 & 0 \\
\vdots & \vdots &\vdots & \vdots & \ddots &\vdots &\vdots\\
 0 & 0 & 0     & 0      &\dots  & {s}&-1 \\
-1 & 0 & 0     & 0      &\dots&-1 & {s}
        \end{array} \right)
\,.
\ee{Hessian}

\subsection{Nonlinear field theories}
\label{s:henlattJacobianOrb}
%%%%% siminos/kittens/cat.tex                    2021-04-27 %%%%%

The key to understanding their chaoticity are the eigenvalues of
the {\jacobianOrb},

The {temporal Bernoulli} condition \refeq{tempBern} %{1stepDiffEq}
can be thus viewed as a time-discretized, first-order ODE dynamical
system
\beq
   \dot{\Xx} \,=\, \vel(\Xx) \,,
\ee{XX1stepVecEq}
where the `velocity' vector field $\vel$ is given by
\[
\vel(\Xx) \,=\,
   % \vel(\Xx;\Mm) \,=\,
(s-1)\,\Xx-\Mm
\,,
\]
with the time increment set to $\Delta\zeit=1$, and perturbations that
grow (or decay) with rate $({s}-1)$.

consider a temporal lattice with a set of $d$ fields
$\ssp_{\zeit}=\{\ssp_{\zeit,1},\ssp_{\zeit,2},\dots,\ssp_{\zeit,d}\}$ on
each lattice site $\zeit$, and time evolution given by a $d$\dmn\ map
$\ssp_{\zeit+1}=\hat{\map}(\ssp_{\zeit})$.
A period-$\cl{}$ {\lattstate} \refeq{pathBern} thus
satisfies site-by-site the first-order difference equation
\beq
\ssp_{\zeit} - \hat{\map}(\ssp_{\zeit-1}) = 0
    \,,\qquad
\zeit=1,2,\cdots,\cl{}
\,.
\ee{XX1stepNonlimTemp}


the matrix of second variations of
the action $\action[\Xx]_{\zeit\zeit'}$,
%
\beq
\jMorb[\Xx] =
\left(\begin{array}{ccccccc}
%\begin{pmatrix}
 {s}_{0} &-1 & 0 & 0 & \cdots & 0 &-1 \\
-1 & {s}_{1} &-1 & 0 & \cdots & 0 & 0 \\
0 &-1 & {s}_{2}  &-1 & \cdots & 0 & 0 \\
\vdots & \vdots & \vdots & \vdots & \ddots & \vdots & \vdots \\
0 & 0 & 0 & 0 & \cdots & {s}_{\cl{}-2} &-1 \\
-1 & 0 & 0 & 0 & \cdots & -1 & {s}_{\cl{}-1}
%\end{pmatrix}
          \end{array} \right)
\,,
\ee{jMorb1dFT} %was {jMorb1dField}, {PCJiKoKr20(8)}
%
evaluated on a {\lattstate} \Xx,
and its {\HillDet} $\Det\jMorb[\Xx]$,
with the `stretching factor' ${s}_{\zeit}= V''(\ssp_{\zeit})$ at
lattice site $\zeit$ in general a function of the site field $\ssp_\zeit$
for the given \lattstate\ $\Xx$.
    \PC{2020-06-01}{
Cite Gade and Amritkar\rf{GadAmr93} as an early investigation of a
lattice {\jacobianOrb}. They did not know about `Hill's formula.
    }
