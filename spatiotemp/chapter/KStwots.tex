% siminos/spatiotemp/chapter/KStwots.tex
% $Author: predrag $ $Date: 2019-07-07 02:39:12 -0400 (Sun, 07 Jul 2019) $

% called by
%           siminos/spatiotemp/chapter/spatiotemp.tex
%           siminos/tiles/GuBuCv17.tex

%\section{\KS\ on a torus}
%\label{sect:KStwots} % KSspacetime}

Here we propose to describe all solutions $u(\conf,\zeit)$ of \KSe\ \refeq{e-ks} as the
closure of the union of the set of all prime (non-repeating)
\twots\
\bea
u(\conf,\zeit) &=& u(\conf,t+\period{})
       \;=\; u(\conf+\speriod{},t)
\continue
       &=&  u(\conf+\speriod{},t+\period{})
\,,\quad
(\conf,\zeit)\in([0,\speriod{}),[0,\period{}))
\,.
    \label{e-ksTorus}
\eea

Consider a torus $(\speriod{},\period{})$ with {\spt}ly
doubly-periodic $u(\conf,\zeit)=u(x+\speriod{},t+\period{})$,
$(\conf,\zeit)\in([0,\speriod{}),[0,\period{}))$, see
\reffig{fig:spaceTime1}\,(c), and combine the discrete spatial
Fourier modes expansion \refeq{spatFT} with the discrete temporal Fourier
modes expansion \refeq{BBtemporFourier}.
Discretize $u(\conf_n,\zeit_m)= u_{nm}$ over
$N$ points of a periodic spatial lattice $\conf_n = n \speriod{}/N$,
and
$M$ points of a periodic temporal lattice $\zeit_m = m \period{}/M$,
%\beq
%    \Fu_{k,m}^{(i)} = \frac{1}{M} \sum_{n = 0}^{M-1}
%    \Fu^{(i)}_{k,n} e^{i \omega_n \zeit_m}
%    \, , \quad
%    \Fu^{(i)}_{k,n} = \sum_{m=0}^{M-1}\Fu_{k,m}^{(i)}  e^{-i \omega_n \zeit_m}
%    \, , \quad
%    \mbox{where }
%    \omega_n = 2 \pi n / \period{} \, .
%\eeq
%
\bea
\Fu_{kj} &=&
  \frac{1}{NM} \sum^{N-1}_{n=0} \sum^{M-1}_{m=0}
  u_{mn} \, e^{-i(\wavek \xm + \freqj \tn)}
    \,,\quad
\wavek = \frac{2 \pi k}{\speriod{}}
    \,,\;
\freqj = \frac{2 \pi j}{\period{}}
    \continue
u_{mn} &=&\sum_{k=0}^{M-1}\sum^{N-1}_{j=0}
   \Fu_{kj} \, e^{i(\wavek \xm + \freqj \tn)}
\,,
\label{spattempFT}
\eea
In its Fourier discretization, \KS\ PDE \refeq{e-Fks} is a set of $MN$
algebraic equations for the {\spt} Fourier coefficients $\Fu$,
\beq
\left[ i \freqj - ( \wavek^2 - \wavek^4 ) \right]\Fu_{kj}
+ i \frac{\wavek}{2} \!\sum_{k'=0}^{M-1} \sum^{M-1}_{j'=0}\!\!
\Fu_{k'j'} \Fu_{k-k',j-j'}
    = 0
\,.
\label{e-FksSpattemp}
\eeq
In other words, we have reduced the problem of either temporal or
spatial evolution on a
{\spt}ly periodic domain to the fixed point problem of determining an
invariant 2-torus, a fixed point in the $NM$\dmn\ \statesp.

\MNG{2019-05-13}{Elimination of space and time
translations effectively reduces the codimension of each solution by two, this is
undesirable when numerically solving the \refeq{e-pseudospectral} in a least-squares sense}
First, however, we must eliminate the temporal translation marginal eigenmode
by a {\PoincSec}, and the spatial translation marginal eigenmode by a {\slice}.
Generically, they both can fixed by the first Fourier mode {\PoincSec} /
{\slice}.
The Newton method then requires inversion of $1-J$, \ie,
$\det(1-J)$, where $J$ is the 2-torus \jacobianM.
    %
    \PC{2018-04-08}{see also \refsect{sect:KurSivSpTmpStab}.}

We can generalize the notion of average `energy' density
\refeq{SCD07ksEnergy} and apply it to any \twot\ $p$ by also
averaging over time
\beq
    \expctE_p=
  \frac{1}{\speriod{}\period{}}\!\oint d\pSpace\oint d\zeit \frac{u^2}{2}
  \,,
  \label{twotEnergy}
\eeq
The Fourier space average spatial energy density \refeq{SCD07EFourier}
generalized to any {\spt} solution \refeq{spattempFT} is then:
\beq
\expctE
          =  \sum_{k=0}^{N-1}\sum^{M-1}_{n=0}
             {\textstyle\frac{1}{2}}|\Fu_{kn}|^2
\,.
\ee{twotEFourier}


\subsection{Spatiotemporal $u=0$ equilibrium}
\label{sect:KSu0equiST}

    \PC{2019-03-16}{
    Incorporate here the spatial evolution stability analysis of
    blog posts starting with {\bf 2016-09-02 Matt Stability of u=0 equilibria}.
   }
   %
Consider the $\Fu_{kn}=0$ \eqv\ of \refeq{e-FksSpattemp}. Its
linearization is
    \MNG{2016-09-12 }{
\MNGedit{Added the factor of $i$ that was missing in \refeq{e-FksSpattemp}}
\PCedit{2016-09-23 restored signs to match \refeq{e-Fks} and \refeq{e-FksSpattemp}}
    }
\bea
i \omega_n  &=&  q_k^2 - q_k^4
    \continue
i \frac{2\pi n}{\period{}}   &=&
\left(\frac{2\pi k}{\speriod{}}\right)^2 - \left(\frac{2\pi k}{\speriod{}}\right)^4
\,.
\label{e-FksSpattemp0}
\eea
The two special cases are the spatial strip of \refsect{sect:KStimeInt},
which yields the $2N$ real temporal stability exponents
$\Lyap_k=i\omega(k)$ for a fixed spatial strip of width $\speriod{}$, and the
temporal strip of \refsect{sect:KSspaceInt}, which yields the $4+8(M-1)$ complex
spatial stability exponents $i q^{\pm,\pm}(n)$ for a fixed temporal strip of period
$\period{}$, as the four solutions of the quartic equation \refeq{e-FksSpattemp}
for each $i \omega_n$, $n = 0, \pm 1, \pm 2, \pm M/2$ :
\beq
(i q_k)^4 + (i q_k)^2 + i \omega_n  =0
    \quad\Rightarrow\quad
i q^{\pm,\pm}(n) = \pm \sqrt{(-1\pm \sqrt{1 -4 i \omega_n})/2}
\ee{spatialStabExp}
    \MNG{2016-09-20}{
    I think I should have been looking for $i q_k$ not $q_k$? If so, it
    matches the stability exponents of \refeq{PCeqvaStblty} %\refeq{MNGeqvaStblty}
    }
Each solution appears twice, corresponding to  $(\omega_n,-\omega_n)$,
except for
\beq
q^{\pm,\pm}(0) = \mp i\sqrt{(-1\pm 1)/2} =
(0,0,-1,1)
\,.
\ee{spatialStabExp0}
There are two root magnitudes for each $n \neq 0$, independent of the
sign of $n$:
%\[%beq
%|q^{\pm,\pm}(n)|^2 = (-1\pm \sqrt{1 +4 i \omega_n})
%(-1\pm \sqrt{1 -4 i \omega_n})/4
%\]%ee{spatialStabExp01}
%
\beq
|2q^{\pm,\pm}(n)|^2 = 1+ \sqrt{1 +  16 \omega_n^2}
\pm \sqrt{1 +4 i \omega_n}
\pm \sqrt{1 -4 i \omega_n}
\,.
\ee{spatialStabExp1}

\subsection{Spatiotemporal stability}
\label{sect:KurSivSpTmpStab}

Let
$u_\stagn(\conf,\zeit)$ be a doubly-periodic solution of \refeq{e-FksSpattemp} on the
$(\conf,\zeit)\in[0,\speriod{})\times[0,\period{})$ torus, a point in the
$\speriod{}\,\period{}$\dmn\ \statesp. Add an arbitrary small perturbation field
$\delta u(\conf,\zeit)$, also doubly periodic. Linearize \refeq{e-FksSpattemp}.
The linearized operator acting on $\delta u(\conf,\zeit)$ must annihilate it, \ie,
(WRONG AS IT STANDS)
\[
(J(u_\stagn) -\ExpaEig_j \matId)\delta u =0
\]
where the \jacobianM\ is the linearization over the whole 2-torus
\beq
J(\Fu_\stagn)_{kn} =
\left[ i \omega_n - ( q_k^2 - q_k^4 ) \right]\delta_{kn}
+ i q_k\!\sum_{k'=0}^{N-1} \sum^{M-1}_{m'=0}\!\!
\Fu_{k'm'} \delta_{k-k',m-m'}
\,.
\label{e-FksSpTmpJac}
\eeq
In $d=1$ case (pure temporal evolution) $\det(J(u_\stagn) -\matId)$ is the
usual period $\period{}$ cycle weight, this time obtained not by integration in
time but by evaluating the determinant of the Toeplitz matrix over the whole
cycle in one go (this might be familiar to some from the path-integral
derivation of Gutzwiller formula cycle weight. With \refeq{e-FksSpTmpJac},
$\det(J(u_\stagn) -\matId)$ is the pattern weight in $d=2$, and this
generates to defining the invariant weight for any $d$-torus.
