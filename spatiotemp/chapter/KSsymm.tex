% siminos/spatiotemp/chapter/KSsymm.tex
% $Author: predrag $ $Date: 2020-05-07 17:34:06 -0400 (Thu, 07 May 2020) $

% called by
%           siminos/spatiotemp/chapter/spatiotemp.tex
%           siminos/tiles/GuBuCv17.tex

%\section{Symmetries of \KSe}
%\label{sect:KSsymm}

The \KSe\ \refeq{e-ks} is equivariant under spatial translations, spatial
reflections and temporal translations and Galilean transformations.
The Galilean symmetry $u(\conf,\zeit)$ is a solution,
then $u(x -ct,t) -c $, with $c$ an arbitrary constant
speed, is also a solution. Without loss of generality, in our
calculations we shall set the mean velocity of the front to zero,
\beq
\spaceAver{u}(\zeit)
  \,=\, \int_0^{\speriod{}} d\conf \, u(\conf,\zeit) = 0
\,.
\ee{GalInv}

If the system is compactified on a
2-torus, with periodic boundary conditions
$u(\conf,\zeit)=u(\conf+\speriod{},t+\period{})$, the symmetry group is
\beq
\Group = \On{2}_\conf \times \SOn{2}_\zeit
        = \Dn{1,\conf} \ltimes \SOn{2}_\conf \times \SOn{2}_\zeit
\,.
\ee{KSsymms}
The elements of the 1-parameter group of spatial shifts and reflections are
$\On{2}_\conf:\{\Shift_{\shift/\speriod{}},\Refl \}$, and
the elements of the 1-parameter group of temporal shifts are
$\SOn{2}_\zeit:\{\Shift_{\shift/\period{}}\}$.
If $u(\conf,t)$ is a solution, then $\Shift_{\shift/\speriod{}}\, u(\conf,t) =
u(\conf+\shift,t)$ is an equivalent solution for any shift $0 \leq \shift <
\speriod{}$, as is the reflection (`parity' or `inversion')
\beq
    \Refl \, u(\conf,\zeit) = -u(-\conf,\zeit)
\,.
\ee{KSparity}

%%%%%%%%%%%%%%%%%%%%%%%%%%%%%%%%%%%%%%%%%%%%%%%%%%%%%%%
% from \example{Invariance under fractional rotations.}{\label{exam:FractRot}
Consider a cyclic group
\[
\Cn{m} = \{e,\trHalf{},\trHalf{}^{2},\cdots,\trHalf{}^{m-1}\}
\,,\qquad \trHalf{}^m= e
\,.
\]
where $\trHalf{}$ is an \SOn{2} rotation by $2\pi/m$. $\Cn{m}$
is a discrete subgroup of \SOn{2} for any $m=2,3,\cdots$ .

A field $u$ on the $2\pi/m$ domain is now a
tile whose $m$ copies tile the entire domain. It is periodic on the
$2\pi/m$ domain, and thus has Fourier expansion with Fourier modes
$\exp(2\pi\ii m j \ssp)$. This means that $\SOn{2}$ always has an
infinity of discrete subgroups $\Cn{2}, \Cn{3}, \cdots,\Cn{m}, \cdots$;
for each the non-vanishing coefficients are only for Fourier modes whose
wave numbers are multiples of $m$.
%  } %end\example{exam:FractRot}
%%%%%%%%%%%%%%%%%%%%%%%%%%%%%%%%%%%%%%%%%%%%%%%%%%%%%%%

If we take discrete subgroups in $\Cn{2,\conf}$ in place of both \SOn{2} groups
then the order of the discrete group
\(
\tilde{\Group} = \Dn{1,\conf} \ltimes \Cn{2,\conf} \times \Cn{2,t}
\)
is of order $8$.
All \spt\ symmetries of discussion can be described by \emph{isotropy subgroups},
which are symmetry subgroups which leave solutions invariant.
Specifically the discrete symmetries,
spatial reflection symmetry and \spt\ shift-reflection symmetry. These particular
symmetries have isotropy subgroups
\(
\Group = \Dn{1,\conf}
\)
and
\(
\Group = \Cn{2,t}
\)
respectively. To cover the discrete \spt\ symmetries that
are realized by \twots\ we need to investigate the group
\(
\Group = \Dn{1,\conf} \times \Cn{2,t}\,,
\)
because its description includes
reflection and shift-reflection symmetries. The term shift-reflection
denotes solutions which are left invariant only after spatial reflection
and a time translation by half a period. We have disregarded
\Cn{2,\conf} for the discussion of discrete symmetries. This is
permitted because spatial half-cell shifts, even in combination
with other group elements only permit equivariant solutions,
not invariant. Solutions invariant under half-cell shifts in
space would have to be doubly periodic in space. For combination
with the cyclic group in time it would be a yet undiscovered
\twot which is invariant after a half-cell shifts in space and
then time. The general $\Cn{M,\conf} \times \Cn{N,t}$ case
is harder to describe; if $M=N$ then one example of a way
to construct an invariant solution would be to construct
a solution which would be invariant after $N$ total rotations.
For instance, a solution with the form
\beq
u(x,t) =\left[\begin{array}{c}
1\,2\,3 \\
3\,1\,2 \\
2\,3\,1
\end{array}\right]
\ee{e-uCmCn}
would be invariant after a cycle consisting of
one space rotation and two time rotations or
two space rotations and one time rotation (each by one third
of the domain in the respective, positive directions). This
seems incredibly unlikely as it requires the solution to be comprised of
permutations of three patterns which are all equivalent in domain size.
This unlikelihood only gets worse for higher order cyclic groups
We return from our tangent by getting into the meat of the
discussion by analyzing the group $ \Dn{1,\conf} \times \Cn{2,t}$.
We demonstrate some standard group theoretic calculations such as
looking at the character table \reftab{D1C2table} and projection operators
\refeq{e-D1C2operators}.

%%%%%%%%%%%%%%%%%%%%
\begin{table}[h!]
\caption{\label{D1C2table}
Because the direct product group is abelian we only have one dimensional
representations and as such the character table follows directly.
    }
\centering
\begin{tabular}{|c|c|c|c|c|}
\quad & $e$ & $\Refl_x$ & $\trHalf{t}$ & $\Refl_x \trHalf{t}$ \\
\hline
$E$ & 1 & 1 & 1 & 1 \\
$\Gamma_1$ & 1 & 1 & -1 & -1 \\
$\Gamma_2$ & 1 & -1 & 1 & -1 \\
$\Gamma_3$ & 1 & -1 & -1 & 1 \\
\end{tabular}
\end{table}
The character table \reftab{D1C2table}, leads
to the construction of four linear projection operators
\bea \label{e-D1C2operators}
P^{++} &=& \frac{1}{4}(1 +\Refl_x +\trHalf{\zeit}+ \Refl_x \trHalf{\zeit}) \continue
P^{+-} &=& \frac{1}{4}(1 +\Refl_x - \trHalf{\zeit}- \Refl_x \trHalf{\zeit}) \continue
P^{-+} &=& \frac{1}{4}(1 -\Refl_x  + \trHalf{\zeit} - \Refl_x \trHalf{\zeit}) \continue
P^{--} &=& \frac{1}{4}(1-\Refl_x  - \trHalf{\zeit} + \Refl_x \trHalf{\zeit})
\,,
\eea
where $\Refl_x$,$\trHalf{\zeit}$ denote spatial reflection about the $x=0$ line and time translation
by half a period, respectively.
The solution space can be decomposed into the irreducible subspaces produced
by these projection operators
$\bbU = \bbU^{++} \oplus \bbU^{+-} \oplus \bbU^{-+} \oplus \bbU^{--}$.
In the context of a \rv\ \spt\ Fourier basis each of these subspaces corresponds
to a subset of coefficients in the expansion \refeq{e-RealFourier}
\bea \label{D1C2subspaces}
u^{-+}(\conf,\zeit) &=& \sum_{k} \sum_{j} \akj \cos(\freqj \tn)\cos(\wavek \xm) \continue
u^{--}(\conf,\zeit) &=& \sum_{k} \sum_{j} \bkj \sin(\freqj \tn)\cos(\wavek \xm) \continue
u^{++}(\conf,\zeit) &=& \sum_{k} \sum_{j} \ckj \sin(\wavek \xm)\cos(\freqj \tn) \continue
u^{+-}(\conf,\zeit) &=& \sum_{k} \sum_{j} \dkj \sin(\wavek \xm)\sin(\freqj \tn) \,.
\eea
We won't use these equations just yet but they are good for classifying what each
projection operator corresponds to. This classification comes naturally
from the parity (odd, even) of the trigonometric functions therein. They can later
be used to derive constraints on the \spt\ \Fcs\ pertaining to invariance
under certain symmetry operations.

Before we continue,
it will first be convenient to calculate the relationships between
the projection operators \refeq{e-D1C2operators} and the spatial differentiation operator.
The utility comes later when we apply these projection operators to the \KSe, specifically
when considering the nonlinear term.
\bea \label{D2C2projopderivx}
D_{\conf} P^{++} &=& \frac{1}{4}D_{\conf}(1 +\Refl_x  + \trHalf{\zeit} + \Refl_x \trHalf{\zeit}) \continue
                 &=& \frac{1}{4}(1 -\Refl_x  + \trHalf{\zeit}- \Refl_x \trHalf{\zeit})D_{\conf} \continue
                 &=& P^{-+}D_{\conf} \continue
D_{\conf} P^{+-} &=& \frac{1}{4}D_{\conf}(1 + \Refl_x  - \trHalf{\zeit}- \Refl_x \trHalf{\zeit}) \continue
                 &=& \frac{1}{4}(1 -\Refl_x  - \trHalf{\zeit}+ \Refl_x \trHalf{\zeit})D_{\conf} \continue
                 &=& P^{--}D_{\conf} \continue
D_{\conf} P^{-+} &=& \frac{1}{4}D_{\conf}(1 -\Refl_x  + \trHalf{\zeit} - \Refl_x \trHalf{\zeit}) \continue
                 &=& \frac{1}{4}(1 +\Refl_x + \trHalf{\zeit} + \Refl_x \trHalf{\zeit})D_{\conf} \continue
                 &=& P^{++}D_{\conf} \continue
D_{\conf} P^{--} &=& \frac{1}{4}D_{\conf}(1 -\Refl_x  - \trHalf{\zeit} + \Refl_x \trHalf{\zeit}) \continue
                 &=& \frac{1}{4}(1 +\Refl_x  - \trHalf{\zeit} - \Refl_x \trHalf{\zeit})D_{\conf} \continue
                 &=& P^{+-}D_{\conf}\,.
\eea
These identities allow us to rewrite the nonlinear terms
present in each projection of the \KSe\ as derivatives
of projection components as opposed to projections of derivatives,
which we believe leads to less confusing analysis. Note that
the effect can be summarized by flipping the first $\pm$, pertaining
to the coefficient of the spatial reflection terms in \refeq{e-D1C2operators}
The surviving nonlinear terms after the application of each projection operator are
as follows
\bea \label{e-D1C2nonlinear}
P^{++}(u\partial_x u) &=& u^{\pm \pm}\partial_{\conf}(u^{\pm \pm})\continue
P^{+-}(u\partial_x u) &=& u^{\pm \pm}\partial_{\conf}(u^{\pm \mp})\continue
P^{-+}(u\partial_x u) &=& u^{\pm \pm}\partial_{\conf}(u^{\mp \pm})\continue
P^{--}(u\partial_x u) &=& u^{\pm \pm}\partial_{\conf}(u^{\mp \mp})\,.
\eea
Using these relations \refeq{e-D1C2nonlinear} we can produce the projections
of the \KSe\ onto the different irreducible subspaces, noting that the projection operator
commutes with the linear terms such that
\bea \label{e-KSEprojections}
P^{++}F(u) &=& u_{\zeit}^{++}+u_{\conf \conf}^{++}+u_{\conf \conf \conf \conf}^{++} \continue
           &+& (u^{++}\partial_{\conf}(u^{++}) + u^{+-}\partial_{\conf}(u^{+-}) \continue
           &+& u^{-+}\partial_{\conf}(u^{-+}) + u^{--}\partial_{\conf}(u^{--}))  \continue
P^{+-}F(u) &=& u_{\zeit}^{+-}+u_{\conf \conf}^{+-}+u_{\conf \conf \conf \conf}^{+-}\continue
           &+&(u^{++}\partial_{\conf}(u^{+-}) + u^{+-}\partial_{\conf}(u^{++}) \continue
           &+& u^{-+}\partial_{\conf}(u^{--}) + u^{--}\partial_{\conf}(u^{-+}))  \continue
P^{-+}F(u) &=& u_{\zeit}^{-+}+u_{\conf \conf}^{-+}+u_{\conf \conf \conf \conf}^{-+}\continue
           &+&(u^{++}\partial_{\conf}(u^{-+}) + u^{+-}\partial_{\conf}(u^{--}) \continue
           &+& u^{-+}\partial_{\conf}(u^{++}) + u^{--}\partial_{\conf}(u^{+-})) \continue
P^{--}F(u) &=& u_{\zeit}^{--}+u_{\conf \conf}^{--}+u_{\conf \conf \conf \conf}^{--}\continue
           &+&(u^{++}\partial_{\conf}(u^{--}) + u^{+-}\partial_{\conf}(u^{-+}) \continue
           &+& u^{-+}\partial_{\conf}(u^{+-}) + u^{--}\partial_{\conf}(u^{++}))\,.
\eea
Solutions to \refeq{e-ks} satisfy $F = 0$ by definition so
by extension solutions must also satisfy $P^{\pm \pm}F=0$.
With this we can determine the combinations of projection operators whose equations
are ``self contained''. This is similar to the notion of \textit{flow invariant subspaces}
but because we do not have dynamics we can't really use this term. Instead,
these subspaces correspond to a constrained set of equations that solutions with
particular discrete symmetries must adhere to.
For example, assume that the only nonzero component $u$ is $u=u^{++}$.
Substitution of \refeq{e-KSEprojections} yields
\bea \label{e-KSplusplus}
P^{++}F(u^{++}) &=& u_{\zeit}^{++}+u_{\conf \conf}^{++}+u_{\conf \conf \conf \conf}^{++}
                +u^{++}\partial_{\conf}(u^{++}) \continue
P^{+-}F(u^{++}) &=& 0 \continue
P^{-+}F(u^{++}) &=& 0 \continue
P^{--}F(u^{++}) &=& 0 \,,
\eea
so $\bbU^{++}$ is
an invariant subspace. In fact,
this subspace
corresponds to equilibria solutions which
live on the $\period{}=0$ line. The meaning
of self contained in this example is that we
assumed that $u=u^{++}$ and the only nonzero part
of \refeq{e-KSplusplus} is the $P^{++}F(u^{++})$ component.
Perhaps a more elucidating example is generated
by the assumption that $u=u^{--} \neq 0 $. Substitution
yields
\bea \label{e-KSminusminus}
P^{++}F(u^{--}) &=& u^{--}\partial_{\conf}(u^{--}) \continue
P^{+-}F(u^{--}) &=& 0 \continue
P^{-+}F(u^{--}) &=& 0 \continue
P^{--}F(u^{--}) &=& u_{\zeit}^{--}+u_{\conf \conf}^{--}+u_{\conf \conf \conf \conf}^{--}
\eea
which indicates that the equations are not self contained as components
other than $P^{--}F(u^{--})$ are non-zero. Recall that each of these
components is equivalently equal to zero. Because these equations represent
scalar field values defined at every $\conf, t$ this implies that in order
to satisfy
$u^{--}\partial_{\conf}(u)^{--}=0$
either $u^{--}$, its derivative $\partial_{\conf}(u)^{--}$, or both must equal
to zero at every point on the \spt\ domain. The only nontrivial
possibility is if there are (at least)
two disjoint regions such that $\Omega_u=\{(\conf,\zeit):u(\conf,\zeit)=0\}$
and $\Omega_{u_x}=\{(\conf,\zeit):u_x(\conf,\zeit)=0\}$. By smoothness, if
$u=0$ then $u_x=0$. This implies
that $u_x=0$ for all $(\conf,\zeit)$; if
$u_x=0$ everywhere and $u=0$ for some $(\conf,\zeit)$ then it must
be the case that $u=0$ everywhere
which contradicts our
original assumption that $u=u^{--} \neq 0 $.
The rest of the symmetry invariant subspaces follow from a
similar substitutions. To expedite the derivation process, note
that the equation for $P^{++}F$ contains
all of the symmetric terms $u^{\pm \pm}\partial_{\conf}(u^{\pm \pm})$
such that there is no
possibility of an invariant subspaces
which does not intersect $\bbU^{++}$.
Following a process of elimination we can show that the possible
symmetry invariant subspaces are $\bbU^{++}$, $\bbU^{++}\oplus \bbU^{--}$,
$\bbU^{++}\oplus \bbU^{+-}$ and $\bbU^{++}\oplus \bbU^{-+}$ and
of course the full space $\bbU$. There are no triplet subspaces
(comprised of three components) which can be shown using
the parity of the different subspaces. We can interpret
these subspaces by addition of the corresponding projection
operators \refeq{e-D1C2operators}
\bea \label{e-invariantoperators}
P_{0}\equiv P^{++} &=& \frac{1}{4}(1 +\Refl_x +\trHalf{\zeit}+ \Refl_x \trHalf{\zeit}) \continue
P_{\Refl_x}\equiv P^{++}+P^{+-} &=& \frac{1}{2}(1 + \Refl_x) \continue
P_{\trHalf{\zeit}}\equiv P^{++}+P^{-+} &=& \frac{1}{2}(1 + \trHalf{\zeit}) \continue
P_{\Refl_x \trHalf{\zeit}}\equiv P^{++}+P^{--} &=& \frac{1}{2}(1+\Refl_x \trHalf{\zeit})
\,.
\eea
With these projection operators we can interpret the symmetry
invariant subspaces as follows:
$\bbU^{++}$ represents the fixed point ($\period{}=0$) subspace,
$\bbU^{++}\oplus \bbU^{+-}$ the spatial reflection invariant subspace,
$\bbU^{++}\oplus \bbU^{--}$ the shift-reflection invariant subspace,
and lastly $\bbU^{++}\oplus \bbU^{-+}$ which
contains solutions that are invariant after a half period shift
in time. This subspace of
``twice repeating'' solutions is trivial and not useful; doubly periodic solutions
can always be made to repeat twice in time by definition. The interpretation
of the corresponding subspace is therefore not very intuitive.

The next question to answer is how continuous spatial translation symmetry
manifests itself in this \spt\ context.
How do these subspaces relate to the continuous spatial translation symmetry?
The three subspaces $\bbU_0,\bbU_{\Refl_x},\bbU_{\Refl_x \trHalf{t}}$
share an interesting property in a real valued (\SOn{2}) representation.
Specifically, the subspaces of \spt\ \Fcs\ corresponding
to invariance under these discrete symmetries
are all orthogonal to the space of spatial translations. This can
be seen by acting on the different orbits with the spatial
derivative operator which is the generator of infinitesimal translations.
The subgroup
\(
H = \Cn{M,\conf}
\)
represents continuous spatial translation symmetry after discretization.
We utilize a co-moving frame ansatz to handle this
symmetry, which we will now develop. As previously mentioned,
we use a \rv\ ($\SOn{2}$)
representation for the \spt\ \Fcs. This choice makes the matrix
representations of the group elements slightly more complicated
as they will be block diagonal as opposed to exactly diagonal.
Note that because of doubly periodic boundary conditions,
translations
are the same as rotation.
The matrix representation of the group element which spatially rotates $M$
Fourier modes by a value $\theta$ is a block diagonal matrix with $M$ blocks; each
block being a representation of two dimensional
rotations for the corresponding wavenumber $k$
\beq \label{e-SOnGroupElement}
\tilde{\LieEl}(\theta) \equiv
\begin{bmatrix}
\cos \wavek\theta  & -\sin \wavek\theta \\
\sin \wavek\theta & \cos \wavek\theta
\end{bmatrix}\,.
\eeq
This block diagonal matrix acts on $M$ Fourier modes;
the corresponding extension to the set of \spt\ \Fcs\
is simply $N$ copies of \refeq{e-SOnGroupElement}. In other
words we have $N$ blocks of \refeq{e-SOnGroupElement}.
This form lends itself to the matrix representation
for the co-moving reference frame transformation.
The co-moving reference frame is the reference
frame which makes {\rpo}s periodic by applying
a time-dependent spatial translation to every
point of the \twot. Using
\refeq{e-SOnGroupElement} the matrix representation
of the co-moving frame transformation is as follows
\beq \label{e-comovingRotation}
\LieEl(\frac{\sigma \tn}{\period{}}) \equiv
\begin{bmatrix}
\tilde{\LieEl}(\frac{\sigma t_1}{\period{}}) & 0 & \cdots & 0 \\
0 & \tilde{\LieEl}(\frac{\sigma t_2}{\period{}}) & \cdots & 0 \\
\vdots & \vdots & \ddots & \vdots \\
 0 & 0 & 0 & \tilde{\LieEl}(\frac{\sigma t_{\scalebox{.4}{$N$}}}{\period{}})
\end{bmatrix}
\,.
\eeq
Transformations of the type \refeq{e-comovingRotation}
will be used in our ansatz for doubly periodic solutions of
the \KSe\ which are relatively periodic.



\subsection{OLD: Symmetries of \KSe}
\label{sec:KSeSymm}
% KSe.tex     copied from siminos/rpo_ks/current/
% TEMPORARY, ELIMINATE EVENTUALLY  \label{s-KS}

%MOVED TO KSsymm.tex

%The \KSe\ is Galilean invariant: if $u(\conf,\zeit)$ is a solution,
%then $u(x -ct,t) -c $, with $c$ an arbitrary constant
%speed, is also a solution. Without loss of generality, in our
%calculations we shall set the mean velocity of the front to zero,
%\beq \int dx \, u = 0 \,. \ee{GalInv}
%As $\dot{a_0}=0$ in
%\refeq{SCD07:expan}, $a_0$ is a conserved quantity
%fixed to $a_0=0$ by the condition \refeq{GalInv}.


$G$, the group of actions $ g \in G $ on a
\statesp\ (reflections, translations, \etc) is a symmetry of the KS
flow \refeq{e-ks} if $g\,u_t = F(g\,u)$.
The \KSe\ is time translationally invariant, and space translationally invariant
on a periodic domain under
the 1-parameter group of
$\On{2}: \{\Shift_{\shift/\speriod{}},\Refl \}$.
If $u(\conf,\zeit)$ is a solution, then
$\Shift_{\shift/\speriod{}}\, u(\conf,\zeit) = u(x+\shift,t)$
is an equivalent solution for any shift
$-\speriod{}/2 < \shift \leq \speriod{}/2$,
as is the
reflection (`parity' or `inversion')
\beq
    \Refl \, u(x) = -u(-x)
\,.
\ee{SCD07:KSparity}
The translation operator action on the Fourier coefficients \refeq{eq:ksexp},
represented here by a complex valued vector
$a = \{a_k\in\mathbb{C}\,|\,k = 1, 2, \ldots\}$, is given by
\beq
  \Shift_{\shift/\speriod{}}\, a = \mathbf{g}(\shift) \, a \,,
  \label{eq:shiftFour}
\eeq
where $\mathbf{g}(\shift) = \mbox{diag}( e^{i q_k\, \shift} )$ is a complex
valued diagonal matrix, which amounts to the $k$-th mode complex plane
rotation by an angle $k\, \shift /\tildeL$.  The reflection acts on
the Fourier coefficients by complex conjugation,
\beq
  \Refl \, a = -a^\ast
\,.
\ee{FModInvSymm}
Reflection generates the dihedral subgroup $\Dn{1} = \{1, \Refl\}$
of $\On{2}$.  Let $\bbU$ be the space of
real-valued velocity fields periodic and square integrable
on the interval $\Omega = [-\speriod{}/2,\speriod{}/2]$,
\begin{align}
 \bbU  &= \{u \in \speriod{}^2(\Omega) \; | \; u(x) = u(x+\speriod{})\}  \,.
\end{align}
A continuous symmetry maps each state $u \in \bbU$
to a manifold of functions with identical dynamic behavior.
Relation $\Refl^2 = 1$ induces linear decomposition
$u(x) = u^+(x)+ u^-(x)$,
$u^\pm(x)= P^\pm u(x) \in  \bbU^\pm$,
into irreducible subspaces
$
\bbU = \bbU^+
       \oplus \bbU^-
$, where
\beq
    P^+=(1+\Refl)/2
    \,,\qquad
    P^-=(1-\Refl)/2
\,,
\ee{SCD07:P1P2proj}
are the antisymmetric/symmetric projection operators.
Applying $P^+,\,P^-$ on the \KSe\ \refeq{e-ks} we have\rf{KNSks90}
\bea
 u_t^+ &=& - (u^+u^+_x + u^-u^-_x )
                - u^+_{xx} - u^+_{xxxx}
    \continue
 u_t^- &=& - (u^+u^-_x + u^-u^+_x )
                - u^-_{xx} - u^-_{xxxx}
\,.
\label{SCD07:KSD1}
\eea
If $u^- = 0$, \KSf\ is confined to
the antisymmetric $\bbU^+$ subspace,
\beq
 u_t^+ = - u^+u^+_x
                - u^+_{xx} - u^+_{xxxx}
\,,
\label{SCD07:KSU+}
\eeq
but otherwise the nonlinear terms in \refeq{SCD07:KSD1}
mix the two subspaces.

Any rational shift $ \Shift_{1/m}u(x)=u(x+\speriod{}/m)$ generates a discrete
cyclic subgroup $\Cn{m}$ of $\On{2}$, also a symmetry of \KSe.
Reflection together with $\Cn{m}$ generates another
symmetry of \KSe, the dihedral subgroup $\Dn{m}$ of $\On{2}$.
The only non-zero Fourier components of a solution invariant
under $\Cn{m}$ are $a_{jm} \neq 0$, $j =1,2,\cdots$, while for a
solution invariant under $\Dn{m}$ we also have the condition
$\Re a_j=0$ for all $j$.
$\Dn{m}$ reduces the dimensionality of \statesp\ and aids computation of
\eqva\ and \po s within it. For example, the 1/2-cell translations \beq
    \Shift_{1/2}\, u(x)=u(x+\speriod{}/2)
\ee{KSshift}
and reflections generate $\On{2}$
subgroup $\Dn{2} = \{1, \Refl,\Shift,\Shift\Refl\}$,
which
reduces the \statesp\ into four irreducible subspaces
(for brevity, here $\Shift = \Shift_{1/2}$):
\begin{align}
 & \qquad\qquad\qquad\qquad\qquad
              ~~~ \Shift ~~ \Refl  ~\;  \Shift\Refl
    \nnu\\
P^{(1)} &= \frac{1}{4} (1 + \Shift + \Refl + \Shift\Refl)
           ~~~~  S  ~~  S   ~~   S
    \nnu\\
P^{(2)} &= \frac{1}{4} (1 + \Shift - \Refl - \Shift\Refl)
            ~~~~  S  ~~  A   ~~   A
    \nnu\\
P^{(3)} &= \frac{1}{4} (1 - \Shift + \Refl - \Shift\Refl)
           ~~~~  A  ~~  S   ~~   A
     \label{ek_defn}\\
P^{(4)} &= \frac{1}{4} (1 - \Shift - \Refl + \Shift\Refl)
          ~~~~  A  ~~  A   ~~   S
\,.
    \nnu
\end{align}
$P^{(j)}$ is the projection operator onto
$u^{(j)}$ irreducible subspace, and the last 3 columns
refer to the symmetry (or antisymmetry) of
$u^{(j)}$ functions under reflection and
1/2-cell shift.
By the same argument that identified \refeq{SCD07:KSU+},
the \KSf\
stays within the
 $\bbU^S =  \bbU^{(1)}+ \bbU^{(2)}$
irreducible invariant $\Dn{1}$ subspace  of
$u$ profiles symmetric under 1/2-cell shifts.

While in general the bilinear term $(u^2)_x$  mixes the
irreducible subspaces of $\Dn{n}$, for $\Dn{2}$ there are
four subspaces invariant under the flow\rf{KNSks90}:
\begin{itemize} %{romannum}
 \item
    $\{0\}$:~~~~~~ the $u(x)=0$ {\eqv}
 \item
    $\bbU^+ = \bbU^{(1)}+ \bbU^{(3)} $:\\
    the reflection $\Dn{1}$ irreducible space of antisymmetric $u(x)$
 \item
    $\bbU^S =  \bbU^{(1)}+ \bbU^{(2)}$:\\
    the shift $\Dn{1}$ irreducible space of $\speriod{}/2$ shift symmetric  $u(x)$
 \item
    $\bbU^{(1)}$:~~~~~\\
    the $\Dn{2}$ irreducible  space of $u(x)$ invariant under
    $x\mapsto \speriod{}/2-x,\ u\mapsto -u$.
\end{itemize} %{romannum}
With the continuous
translational symmetry eliminated within each subspace, there are no
\reqva\ and \rpo s, and one
can focus on the \eqva\ and \po s only, as was done
for $\bbU^+$ in \refrefs{Christiansen97,LanThesis,lanCvit07}.
In the Fourier
representation, the
$u \in \bbU^+$
antisymmetry amounts to having purely imaginary
coefficients, since $a_{-k}= a^\ast_k = -a_k$.
The 1/2 cell-size shift $\Shift_{1/2}$
generated 2-element discrete subgroup
$\{1,\Shift_{1/2}\}$ is
of particular interest
because in the $\bbU^+$ subspace the translational invariance of the full system reduces to
invariance under discrete translation \refeq{KSshift} by half a
spatial period $\speriod{}/2$.

Each of the above dynamically invariant subspaces is unstable
under small perturbations, and generic solutions of \KSe\ belong to
the full space.
Nevertheless, since  all \eqva\ of the KS flow studied in \refref{SCD07}
lie in the $\bbU^+$ subspace, $\bbU^+$  plays important role for the global
geometry of the flow.
However, linear stability of these \eqva\ has
eigenvectors both in and outside of $\bbU^+$, and needs to be
computed in the full \statesp.
