% siminos/reversal/abstract.tex      pdflatex LC21; bibtex LC21
% temporary: siminos/chapter/LC21abstract.tex  pdflatex blogCats; biber blogCats
% $Author: predrag $ $Date: 2021-12-24 01:25:20 -0500 (Fri, 24 Dec 2021) $

% Predrag   LC21 version 2.1            2021-12-29
% Predrag   LC21 version 2.0            2021-12-15
% Predrag   LC21 version 1.1            2021-08-12
% Predrag   CL18 version 1.0            2020-09-19
% Predrag   CL18 version 0.1            2019-08-15
% Predrag   GHJSC16 draft               2017-09-26

Motivated by Gutzwiller's semiclassical quantization theory, in which
unstable periodic orbits of low-dimensional deterministic dynamics serve
as a WKB `skeleton' for chaotic quantum mechanics, we
construct the corresponding deterministic skeleton for
infinite-dimensional lattice-discretized scalar field theories. In
field-theoretical formulation, there is no evolution in time, and there
is no `Lyapunov horizon'; there is only an enumeration of {\lattstate}s
that contribute to theory's partition sum, all robust global \spt\
solutions of deterministic Euler–Lagrange equations.

The reformulation aligns `chaos theory' with the standard solid state,
field theory and statistical mechanics. In a \spt,
crystallographer formulation, time-periodic orbits of dynamical systems
theory are replaced by periodic $d$\dmn\ {Bravais cell} tilings
($d$-tori) of spacetime, each weighted by the inverse of its instability,
its Hill determinant. In contrast to conventional solid state
calculations, hyperbolic shadowing of large cells by smaller ones ensures
that the predictions of the theory are dominated by the smallest Bravais
cells.

The form of the partition function of a given field theory is determined
by the group of its \spt\ symmetries, \ie, by the space group of its
lattice discretization, best studied on its reciprocal lattice. Already
1\dmn\ lattice discretization is of sufficient interest to be the focus
of this paper. In particular, from a \spt\ field theory perspective,
`time'-reversal is a purely crystallographic notion, a reflection point
group, leading to a perhaps novel symmetry quotienting of
time-reversible theories and associated topological zeta functions.
