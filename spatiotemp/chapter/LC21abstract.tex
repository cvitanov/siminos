% siminos/reversal/abstract.tex      pdflatex LC21; bibtex LC21
% temporary: siminos/chapter/LC21abstract.tex  pdflatex blogCats; biber blogCats
% $Author: predrag $ $Date: 2021-12-24 01:25:20 -0500 (Fri, 24 Dec 2021) $

% Predrag   LC21 version 2.0            2021-12-15
% Predrag   LC21 version 1.1            2021-08-12
% Predrag   CL18 version 1.0            2020-09-19
% Predrag   CL18 version 0.1            2019-08-15
% Predrag   GHJSC16 draft               2017-09-26

We use lattice-discretized
scalar field theories, in particular \catlatt, \Henon\
{$\phi^3$} and {$\phi^4$} theories
to explain how to reformulate dynamics of {\spt}ly `chaotic' or
`turbulent' PDEs in

Motivated by a Gutzwiller's ideas of periodic orbits of
deterministic dynamics serving as WKB
`skeleton' for chaotic quantum mechanics, in this paper we
focus on deterministic chaotic field theory.

We abandon initial state evolution, local in time, and seek instead to
determine global solutions compatible with system's defining equations.

The reformulation aligns `chaos theory' with the standard solid state,
field theory and statistical mechanics partition functions, related to
the well-known forward-in-time Gutzwiller trace formulas by Hill's
formulas.

In \spt, crystallographer formulation, time-periodic orbits of
dynamical systems theory are replaced by periodic $d$\dmn\ {Bravais cell}
tilings ($d$-tori) of spacetime. In contrast to conventional
solid state calculations, due to hyperbolic shadowing of large
cells by smaller ones, the predictions of the theory are dominated by
the smallest Bravais cells.


In the field-theoretical formulation, there is no evolution in time, and
there is no `Lyapunov horizon'; every contributing {\em \lattstate} is a
robust global solution of a \spt\ fixed point condition.

The form of the partition function of a
given field theory is determined by the group of its \spt\ symmetries, \ie,
by the space group of its lattice discretization,  and its reciprocal
lattice.

In particular, from a \spt\ field theory perspective, `time'-reversal is
a purely crystallographic notion, a reflection point group, leading to a
perhaps unfamiliar symmetry quotienting of time-reversible theories.

The theory is compactly summarized by its {\tzeta}
that counts Bravais lattices.

On the level of counting lattice-states, their {\tzeta}s are purely
group-theoretic Lind zeta functions.
