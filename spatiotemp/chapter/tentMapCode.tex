% siminos/spatiotemp/chapter/tentMapCode.tex
% $Author: predrag $ $Date: 2020-05-07 17:34:06 -0400 (Thu, 07 May 2020) $


%%% input by % siminos/spatiotemp/chapter/catMap.tex %%%%%
\section{Any piecewise linear map has ``linear code''}
\label{exam:tentMapSymbDyn}

                                                            \toCB
For reasons unbeknownst to me, it is below the dignity of any cat to work out
any problem in Chaos\-Book, or in the online course, no matter how often I
point out that it is easier to understand what we do for cat maps if you
first work it out for 1\dmn\ maps.

So I have to do these exercises myself - I'm forced to it, so Li Han can be
motivated to re-derive his polynomials (as described in Bird and
Vivaldi\rf{BirViv}, see my notes of {\bf 2016-05-21, -12-12} below), rather
than to fit them to Mathematica grammar rule counts for integer $s$.

Basically, I am baffled by why should ``linear code'' be such a big deal that
it has to go into the title of our paper\rf{GHJSC16}. {\em Every} example of
symbolic dynamics worked out in Chaos\-Book is a ``linear code.'' The strategy
is always the same - find a topological conjugacy from your map to a
piecewise linear map, and then use the fact that any piecewise linear map has
``linear code.'' The pruning theory is always the same - kneading orbit
separates {\admissible} from the {\inadmissible}, also in the infinite 1\dmn\
discrete lattice case worked out in the {\em Diffusion} chapter in the
Chaos\-Book, and the appendix (\refchap{c-appendStatM} reproduced here) that no
one wants to read either.

A tent map is a 1\dmn\ example (a simpler one would be Bernoulli, and its
sawtooth generalizations). The 2\dmn\ examples are the Belykh map,
\refexam{exam:BelykhLCod}, and the Lozi map, \refexam{exam:LoziLCod}. Belykh map
is of particular interest to us, as it is in form very close to the cat
map. Both maps have the pruning front conjecture is proven for them, for
a some sets of parameters.

%%%%%%%%%%%%%%%%%%%%%%%%%%%%%%%%%%%%%%%%%%%%%%%%%%%%%%%
\hfill    \fastTrackExam{exam:TentLCod}
          \fastTrackExam{exam:TentCycl}

\hfill    \fastTrackExam{exam:BelykhLCod}
          \fastTrackExam{exam:LoziLCod}
