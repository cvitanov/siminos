% siminos/spatiotemp/chapter/symbolicIns.tex  pdflatex blog; biber blog
% $Author: predrag $ $Date: 2021-06-12 01:01:43 -0400 (Sat, 12 Jun 2021) $

%Predrag            2016-12-20

\section{Symbolic dynamics, inserts}
\label{s-SymbDynDefs}
% from ChaosBoook  \Chapter{knead}{15feb2015}{Charting the state space}

{\bf 2019-01-19 Predrag} Merge everything here to \refchap{s-SymbDynGloss}
{\em Symbolic dynamics: a glossary} then \texttt{svn rm} this file.

{\bf 2017-08-05 Predrag}
Consult / harmonize with  ChaosBook.org Chapter~{\em Charting the state space} (source file knead.tex).



%%%%%%%%%%%%%%%%%%%%%%%%%%%%%%%%%%%%%%%%%%%%%%%%%%%%%%%%%%%%%%%
\bigskip

to Predrag: check that all this is in Chaos\-Book, then erase:

\bigskip


The set of all bi-infinite itineraries that can be formed from the
letters of the alphabet ${\cal A}$ is called the
{\em full shift} (or {\em topological Markov chain})
\index{shift!full}
\index{Markov!chain}\index{topological!Markov chain}
% before \ee{FullSh}

Here we refer to this set of all conceivable itineraries
as the {\em covering} symbolic dynamics.
\index{symbolic dynamics!covering}
\index{covering!symbolic dynamics}

Orbit that starts out as a finite {\brick} followed by infinite number of
repeats of another {\brick} $p = (\Ssym{1} \Ssym{2} \Ssym{3} \cdots
\Ssym{\period{}})$ is said to be {\em heteroclinic} to the cycle $p$. An
orbit that starts out as $p^{\infty}$ followed by a different finite
{\brick} followed by $(p')^{\infty}$ of another {\brick} $p'$ is said to be a
{\em heteroclinic connection} from cycle $p$ to cycle $p'$.
    \index{heteroclinic!connection}



Suppose that
the grammar can be stated as a finite number of pruning rules, each
forbidding a {\brick} of finite length,
\index{symbolic dynamics!coding}
\beq
 {\cal G} = \left\{
        b_1, b_2, \cdots b_k
        \right\}
\,,
\ee{grammar1}
where a {\em pruned {\brick}} $b$ is a sequence of symbols
$b=\blok{ \Ssym{1} \Ssym{2} \cdots \Ssym{\cl{b}}}$,
 $\Ssym{} \in {\cal A}$,
of finite length $\cl{b}$.
\index{block, pruning}
\index{pruning!block}
\index{shift!finite type}
\index{symbolic dynamics!recoding}
\index{recoding}


\noindent{\bf Subshifts of finite type.}
A {topological dynamical system} $(\Sigma,\sigma)$ for
which all {\admissible} itineraries are generated by a finite
transition matrix
\beq
\Sigma = \left\{ (\Ssym{k})_{k\in \integers} \,:\, T_{s_k s_{k+1}} = 1
        \quad \mbox{for all $k$} \right\}
\ee{AdmItnr}
is called a subshift of {\em finite type}.

\noindent{\bf Reflection symmetries.}
Symmetries of the cat map induce  invariance with respect to
corresponding symbol exchanges. Define $\bar{m}=s\!-\!m\!-\!2$ to be the
conjugate of symbol $m \in \A$. For example, the two exterior
alphabet \Ae\ symbols are conjugate to each other, as illustrated by
\refeq{eq:StateSpCatMap}.
\PC{2019-05-27}{fix this eq. reference; edit it away}
If ${b}=\Ssym{1} \Ssym{2} \dots \Ssym{\ell}$ is a
\brick, and  $\bar{{b}}=\bar{m}_1 \bar{m}_2 \dots
\bar{m}_\ell$ its conjugate, then by  reflection symmetry of the cat
map we have  $|\Pol_{{b}}|= |\Pol_{\bar{{b}}}|$. Similarly, if
$b^*=\Ssym{l}\Ssym{l-1}\dots \Ssym{1}$, the time reversal invariance implies
$|\Pol_{{b}}|=|\Pol_{{b}^*}|$.

There are many ways to skin a cat. For example, due to the space
reflection symmetry about $\ssp=1/2$ of the \PV\ cat map
\refeq{eq:StateSpCatMap}, it is natural (especially in studies of
deterministic diffusion on periodic
lattices\rf{ArtStr97,CBdiffusion,CBappendDiff}) to center  the \statesp\
unit interval\rf{PerViv} as $\ssp\in[-1/2,1/2)$. In this formulation the
\PV\  cat map has a 5-letter alphabet
$\A=\{\underline{2},\underline{1},0,1,2\}$, in which the spatial
reflection symmetry is explicit (the ``conjugate'' of a symbol $m \in \A$
is $\bar{m}= -\!m$).
% , with \statesp\ partitions and pruning rules taking a symmetric form.

%%%%%%%%%%%%%%%%%%%%%%%%%%
% \renewcommand{\cl}[1]{{\ensuremath{|#1|}}}  % the length of a periodic orbit, Ronnie
