\documentclass{article}
\usepackage[utf8]{inputenc}

\begin{document}
% \title{PhD}
% \author{matthew.gudorf }
% \date{February 2020}
% \maketitle
\section{Introduction}
		\subsection{Background}
			\subsubsection{The problem}
			\subsubsection{Work so far}
		\subsection{Transition 1}
			\subsubsection{No progress, need to change the process}
		\subsection{Why current is bad}
		    \subsubsection{all conventional methods do not work on large}
	    \subsection{Why}
            % we ddiscussed to not go into shadowing so early
		    \subsubsection{all continuous dimensions treated democratically}
			\subsubsection{Infinite spacetime via patterns}
			\subsubsection{its possible to name and give precise meaning to patterns}
            % hyperbolic -> isolated -> discrete symb dyn & topologically robust
            % qualitative patterns -> quantitative results.
            \subsubsection{Topologically robust}
			\subsubsection{domain determined by equation, (T,L) can change}
		\subsection{Transition 2}
			\subsubsection{Know why, now use KS as testing ground }
		\subsection{What}
			\subsubsection{K-S vs. N-S}
			\subsubsection{2-tori translational invariance, Fourier}
			\subsubsection{optimization problem}
			\subsubsection{library of solutions}
			\subsubsection{large to small}
            % (topology is robust enough to make approximations)
			\subsubsection{small to large}
		\subsection{Transition 3}
            \subsubsection{Know what, how do we do it}
		\subsection{How}
            % orbach, cvitanovic, et al, near passages just don't happen in high dim
		    \subsubsection{initial conditions}
			\subsubsection{descent}
			\subsubsection{direct methods}
			\subsubsection{library}
			\subsubsection{clip}
			\subsubsection{glue}
            \subsubsection{continuation}
        \subsection{Transition 4}
            \subsubsection{Know how, does it work?}
        \subsection{}
		% Be sure to explain why these examples are important and
		%why they were chosen.
		\subsection{Library Results}
			\subsubsection{Noise, known solutions}
			\subsubsection{Initial to final}
			\subsubsection{Symmetries}
			\subsubsection{Outliers}
			
		\subsection{Tile results}
			\subsubsection{Tile guesses}
			\subsubsection{Clippings}
			\subsubsection{Continuous families}
			\subsubsection{Symmetric endpoint}
			\subsubsection{Intermediate}
			\subsubsection{Terminal endpoint}
			\subsubsection{don't have tunable parameter for families}
				
		\subsection{Gluing results}
			\subsubsection{Time gluing, used to itineraries}
			\subsubsection{Space gluing}
			\subsubsection{Repeated spatial gluing}
			\subsubsection{Tile combinations}
			\subsubsection{Reproduction of target solution}
			\subsubsection{New solutions}
			
	\section{Conclusions and discussion}
		\subsection{Triumphs}
			\subsubsection{New method to solve}
			\subsubsection{Less memory than expected}
			\subsubsection{Tiles, gluing are completely new idea}
            \subsubsection{Spatiotemporal leads the way for parallelized techniques}
		\subsection{Open Challenges and Future work}
			\subsubsection{Theory of tile families, rubberized, tiles identified globally,
                            not localized}
			\subsubsection{Better numerical methods exist.}
            \subsubsection{no physical predictions yet}
			\subsubsection{Linear spatiotemporal stability and Hill's formula}
			\subsubsection{Lack of systematic method of gluing and its crudeness}
			\subsubsection{Prediction of physical observables}
			\subsubsection{Local Galilean velocity not included}
			\subsubsection{No Symbolic Dynamics (no one has one) or grammar}
			\subsubsection{Have no metric to tell if final tile is the
					realization of the symbolic initial condition}
			\subsubsection{Navier-Stokes or Kolmogorov or spiral waves}
			\subsubsection{Need to have more local conditioning}
			\subsubsection{Subdivision of domain; can't use Fourier}
			\subsubsection{Use spatial translations in smart way}
			\subsubsection{No algorithmic way of gluing yet}
			\subsubsection{no spatiotemporal theory analogous to periodic orbit theory}.

\end{document}
