% siminos/spatiotemp/chapter/stabBlog.tex
% $Author: predrag $ $Date: 2021-08-10 11:56:19 -0400 (Tue, 10 Aug 2021) $

% chapter/stability.tex called by blogCats.tex
% not included in CL18.tex

\section{Stability blog}
\label{sect:stabBlog}

\begin{description}
    \PCpost{2016-09-28}{                                        %\toCB
Amritkar \etal\rf{AGGN91,GadAmr93} have investigated the stability of
{\spt}ly {\po}s in one- and two-dimensional coupled map lattices, \ie,
$1+1$ and $1+2$ \spt\ dimensions. The stability matrices for such {\lattstate}s are block circulant and hence can be brought onto a block diagonal
form through a unitary transformation. They derive conditions for the
stability of periodic solutions in terms of the criteria for smaller
orbits.

Zhilin\etal\rf{ZGBG94}
{\em Spatiotemporally periodic patterns in symmetrically coupled map lattices}
write: ``
The stability of the deduced orbits is investigated and we can reduce the
problem to analyze much smaller matrices corresponding to the building block
of their spatial periodicity or to the building block of the spatial
periodicity of the original orbits from which we construct the new orbits. In
the two-dimensional case the problem is considerably simplified.
''
    }

    \PCpost{2019-10-10}{
Reread Lindstedt-\Poincare\rf{DV02} Fourier method papers by
Viswanath\rf{DV03,DV04}; his most accurate resolution of fractal
structure of the Lorenz attractor. It is a very thin fractal, stable
manifold thickness is of the order $10^{-4}$. He has computed all 111011
periodic orbits corresponding to symbol sequences of length 20 or less,
all with 14 digits accuracy.
    }

    \PCpost{2019-10-13}{
Viswanath\rf{DV02} writes: ``
The Lindstedt-\Poincare\ technique uses a nearby periodic orbit of the
unperturbed differential equation as the first approximation to a
perturbed differential equation.
One of the examples presents what is possibly the most accurate
computation of Hill's orbit of lunation since its justly celebrated
discovery in 1878.

The eigenvalues excluding 1 are called characteristic multipliers.

AUTO\rf{auto,Doedel97} collocation method, Guckenheimer and
Meloon\rf{GM00aut}, Choe and Guckenheimer\rf{ChoGuck99} all set up their
\po s as in \refeq{errorVecs1D}. Since the linear systems that they form
are sparse, the cost of solution is only linear in the number of mesh
points.

There are other variants of this forward multiple shooting algorithm: one
is a symmetric multiple shooting algorithm and another is based on
Hermite interpolation.

He dismisses harmonic balance methods for computing periodic orbits (Lau,
Cheung and Wu [?15], and Ling and Wu [?16]) as being too expensive, of
order $O(n^3)$, where the Fourier series are of width $n$, whereas his
method is of order $O(n\ln{n})$ .

Wisvanath algorithm for computing periodic orbits is a ``polyphony of
three themes:'' the Lindstedt-\Poincare\ technique from perturbation
theory, Newton's method for solving nonlinear systems, and Fourier
interpolation.

To compute $n$ Fourier coefficients of $\ssp(\zeit)$, the fast Fourier
transform (FFT) is applied to the function evaluated at $n$ equispaced
points in $[0,2\pi)$. The width $n$ of the Fourier series must be
sufficiently large to pick up all the coefficients above a desired
accuracy threshold.

If $(\ssp_1,\ssp_2,\cdots,\ssp_m)$ are $2\pi$ periodic, so is
$\map(\ssp_1,\ssp_2,\cdots,\ssp_m)$. To obtain its Fourier series from
those of the $\ssp_i$, interpolate $\ssp_i$ at equispaced points,
evaluate $\map$ at those points, and apply the FFT. The inverse FFT can be
used to interpolate a Fourier series at equispaced points\rf{trefethenSpectral}.
In $d$ \statesp\ dimensions, one needs
$d$ Fourier series, one for each coordinate in $\reals^d$.

His 4 coupled Josephson junctions (10\dmn \statesp) uses 64 Fourier modes.

''

The implementation of the algorithm must pay attention to the possibility of
aliasing.
    }

    \PCpost{2019-10-13}{
Viswanath\rf{DV03} writes: ``
The representation of periodic orbits by Fourier series is both accurate
and efficient because, when a periodic orbit is analytic, the Fourier
coefficients decrease exponentially fast, making its Fourier
representation compact.

''
    }

    \PCpost{2019-10-13}{
Guckenheimer and
Meloon\rf{GM00aut} set up their
\po s as in \refeq{errorVecs1D}, and have the same $d$\dmn\ orbit  \jacobianM\
variant of
\refeq{fcnewt}, but with extra, time-direction fixing diagonals, as they are
looking at continuous time flows. Instead of the cyclic group, they use
LU factorization. They get $1-\jMps_p$ matrix.
    }

    \PCpost{2019-10-14}{
Notes on Choe and Guckenheimer\rf{ChoGuck99}, a clear and enjoyable read:

Instead of relaying on forward-in-time numerical integration,
\emph{global methods} for finding periodic orbits view the vector field
as an equation on a function space of closed curves. Here $\map$ is a
Lipschitz continuous vector field on a smooth manifold \pS, and $p :
S^1\to\pS$ is a $C^1$ closed curve in \pS.

Computer implementation of global methods for computing periodic orbits
requires discretization of closed curves and approximation of the
periodic orbit equations. One defines finite-dimensional
submanifolds  of the space  of closed curves and approximates the
periodic orbit equations as a map defined on this space.

They keep the number of discretization points fixed and increase the
accuracy by \emph{automatic differentiation}, constructing the Taylor
series of trajectories at discretization points. They also compute
\stabmat\ derivatives of the Taylor series coefficients with respect to
the \statesp\ variables for use in the Newton iteration. As the degree of
the computed Taylor series increases, their curves converge since the
trajectories are analytic.

The Taylor series is obtained by repeated differentiation of the
differential equation
\(
\dot{\ssp}=\vel(\ssp)
\)
and recursive substitution of
the values of derivatives $\ssp^{(k)}(\zeit)$ of increasing degree. To
make the approximate curve smooth and continuous, they use a somewhat
funky interpolation function they call $\beta(\zeit)$.

[Predrag's aside: hopefully our strategy of using Fourier transforms has
much faster convergence than Taylor series. Even if one wants
polynomials, I suspect Chebyshev or Hermite or some other orthogonal sets
would be better.]

Indeed, the Hermite splines, interpolating functions that arc polynomials
of degree $2d+1$, gave the best results in their computations.

They eliminate the time translation marginal eigenvalue by using sets of
\PoincSec\ hyperplanes transverse to the vector field, and solving for
points that lie on the intersection of \PoincSec\ with the periodic
orbit.
They use the orthogonal complements to the vector field $\vel(\ssp_i)$ at
the mesh points $\ssp_i$. The normal subspace to the vector field at
$\ssp_i$, is determined by computing the QR factorization of the
$[d\!\times\!(d+1)]$ matrix. There is a whole PhD thesis worth of detail here.

The structure of the \jacobianMs\ that are used in the root finding has a
simple sparsity pattern that can be exploited in its inversion. Explicit
inversion of this block matrix in terms of the inverses of the individual
blocks yields a relationship between the regularity of the root finding
problem and the hyperbolicity of the periodic orbit.
They relate the regularity of orbit \jacobianM\ ${\cal J}$ to the
\po's \monodromyM, their sect.~3.~{\em Analysis},
using LU factorization. They show that ${\cal J}$ is invertible
(needed for Newton schemes) if and only if the \monodromyM\ \monodromy of the
\PoincSec\ does not have 1 as an eigenvalue.

Since their methods produce smooth approximations to periodic orbits,
they can evaluate the distance between the tangent vectors to a computed
curve and the vector field along that curve. These error estimates enable
them to develop strategies for mesh refinement that balance the error in
different mesh intervals. Since the approximating solution in a mesh
interval is determined entirely by its endpoints, mesh refinement is a
simple process and does not change the structure of the discretized
periodic orbit equations.

They define the error field \refeq{errorVecs1D} as operator
$F(p)=\map(p)-\sigma p$, with \po s solutions satisfying $F=0$. $p$ are
analytic curves, but Choe-Guckenheimer approximations are not analytic.

The starting data is an $N$-point \emph{discrete closed curve}
\refeq{nXdCycleErr}, a cyclically ordered collection of $N$ points. Given
a map $S$, on seeks seek systems of $(\cl{p}\!\times\!d)$\dmn\ vector
field equations $F_S=0$ whose solutions yield good approximations to
periodic orbits of \map.
The convergence is takes place on a fixed mesh, but with increasing
degree $d$ of map $S_d$. They compute the orbit \jacobianM\ ${\cal J}$
and invert it to use in the Newton routine,
but do not mention or discus computing $\det\cal J$.

They test their algorithm with the Hodgkin-Huxley equations, a moderately
stiff 4\dmn\ vector field with strongly stable directions. They do not
boast, but their residual errors are of order $10^{-11}$.
    }







\end{description}





%%%%%%%%%%%%%%%%%%%%%%%%%%%%%%%%%%%%%%%%%%%%%%%%%%%%%%%%%%%%%%%%%%%%%%%%%%
\section{Hill's formula blog}
\label{sect:HillBlog}

\hfill   {\color{red} For the latest entry, go to the bottom of this section}

\bigskip

\begin{description}

    \PCpost{2019-10-13}{
Viswanath\rf{DV02} describes, and numerically solves Hill's problem: ``
In 1878, Hill\rf{Hill86} derived the equations that describe
the planar motion of the moon around the earth:
\bea
  \ddot{x}-2\dot{y} &=& \frac{\pde\Omega}{\pde x}
\continue
  \ddot{y}+2\dot{x} &=& \frac{\pde\Omega}{\pde y}
\,,\quad
  \Omega = \frac{3}{2}x^2+(x^2+y^2)^{-1/2}
\,.
\label{DV02:HillEq}
\eea
The Jacobi integral $2\Omega-\dot{x}^2-\dot{y}^2$ is constant along
the solutions of Hill's equation, so each orbit is characterized by
a ``Jacobi constant.''
The orbits Viswanath (and Hill) computes are symmetric with
respect to both the $x$ and the $y$ axes - $\Cn{2}\times\Cn{2}$-symmetric,
so it suffices to compute them to quarter-period $\period{}/4$.
He uses a Fourier series of width 64 and filters out
20\% of the frequencies at the high end after each iteration.

I do not see Hill's formula in this paper.
    ''
    }

\PCpost{2018-10-27}{
Petrisor\rf{Petrisor13}
{\em Twist number and order properties of periodic orbits}
works mostly with the standard-like maps, but we might find her article
useful both as a review of the standard
literature, as well as an aid in understanding the $\jMorb$ of the cat map,
and perhaps the twisted {\bcs} (relative) \twots\ of \catlatt\ as well.

Petrisor\rf{Petrisor14} {\em Monotone gradient dynamics and the location of
stationary $(p,q)$-configurations} might also be of interest.

A standard-like map is a twist map $F_\epsilon$, defined by a Lagrangian
generating function of the form
\[
h(x,x') = \frac{1}{2} (x - x')^2 - \epsilon V(x)
\,,
\]
where $V$ is a fixed 1-periodic even function.
Classical standard map corresponds to the potential
\(
V(x) = - \frac{1}{(2\pi)^2}\cos(2\pi x)
\,.
\)
The twist map $F_\epsilon$ is reversible, i.e. it factorizes as $F_\epsilon = I
\circ R$, where R and I are the involutions. [...] The R-invariant orbits are
called symmetric orbits.

A numerical
characteristic associated with a periodic orbit is the rotation number, which
measures the average rotation of the orbit around the annulus.
Mather\rf{Mather84} defined also the amount of rotation, which is called twist
number or torsion number.
[...]
Angenent\rf{angen88} proved that in the space of $(p,q)$-sequences a critical
point of the $W_{pq}$ action [...]  is connected by the negative gradient flow of
the action.
    \index{twist number}\index{torsion number}

The 1-cone function is defined on the phase space of a twist map and takes
negative values within the region where the map exhibits strong folding property.
We prove that the restriction of this function to a periodic orbit gives
information on the eigenvalues of the {\jacobianOrb} $\jMorb_q$ associated with that
orbit.

[...]
we revisit the definition and properties of the twist number of a periodic orbit
based on the structure of the universal covering group of the group
$\SLn{2}{\reals}$. The twist number is defined as the translation number of a circle
map induced by the monodromy matrix associated with the periodic orbit.
[...]
we give the relationship between the twist number value of a
$(p,q)$-periodic orbit, and the position of the real number 0 with
respect to the sequence of interlaced eigenvalues of the {\jacobianOrb}
$\jMorb_q$, associated with the corresponding $(p,q)$-sequence, and of a
symmetric matrix derived from $H_q$.

Petrisor eq.~(11) expresses the Hessian in terms of the 1-step forward
{\jacobianM}, and gives references to the related literature. In
particular, the discrete Hill's formula, the characteristic polynomial of
a periodic Jacobi matrix, and the Hill discriminant are presented in
Toda\rf{Toda89}.

The {\jacobianOrb} $\jMorb_q$ of the$W_{pq}$ action associated with a
$(p,q)$-periodic orbit, $q \leq 3$, is a Jacobi periodic matrix (i.e. a
symmetric tridiagonal matrix with non-null entries in the upper right,
and left lower corners, and the next-to-diagonal entries have the same
sign), see Petrisor eq.~(19). For q = 2, $H_q$ is simply a symmetric
matrix.

Let $\jMorb_q$ be the {\jacobianOrb} of the action $W_{pq}$ at a critical
point $\ssp = (\ssp_n)$. The signature (the number of negative and
positive eigenvalues) of the Hessian $\jMorb_q$, at a non-degenerate
minimizing sequence is (0, q), while at the corresponding mini-maximizing
sequence it is (1, q-1). The number of negative eigenvalues is called the
Morse index of the critical sequence.
    }

    \PCpost{2020-08-02}{
Toda\rf{Toda89} {\em Theory of Nonlinear Lattices} \CBlibrary{Toda89},
Chapt.~4.{\em Periodic Systems} has way more wisdom than what I am
capable of learning this Sunday.

Toda studies the classical mechanics of one-dimensional
lattices (chains) of particles with nearest neighbor interaction; they
are discrete and infinite in space, continuous in time.

When the force is proportional to displacement, that is, when Hooke's law
is obeyed, the spring is said to be linear, the potential is quadratic.
While for us that leads to site stretching rate ${s}$, for Toda it leads to the
Laplacian (${s}=2$), not sure why..

The {\em inverse scattering method} for an infinite lattice makes use of
the discrete Schrodinger equation. For periodic systems this gives a
discrete Hill's equation, and in place of the scattering data, it is
convenient to use the spectrum of the discrete Hill's equation and the
auxiliary spectrum for fixed boundary conditions of the same equation. In
this case the fundamental solutions and the discriminant of the discrete
Hill's equation play important roles. The discriminant is a polynomial of
the spectrum, and the integral of motion is given in terms of elliptic
integrals. Thus the initial value problem reduces to the inverse problem
(Jacobi's inverse problem), or inverse spectral theory.

His discrete Hill's equation is continuous in time, so presumably most
work is for stationary states; I have probably misunderstood the
formulation...

He works with a 3-term recurrence (4.1.3a), and defines a 2-configuration
monodromy matrix (4.1.11).

For special values of A, the solution of (4.1.4) can be periodic, but
more generally it is relative periodic (4.1.16), or the Bloch function
(it's existence given by the Floquet theorem) which he relates to the
trace of the monodromy matrix (4.1.19). The simplest example is his
(4.1.23). His \jacobianOrb\ (4.1.28) has variable diagonal and off
diagonal elements, corresponding to nontrivial nonlinear solutions for
$d=1$ lattice.

He says that the $\speriod{}=3$ three-particle system [4.8] is important
because, though the simplest, it shows nearly all the characteristic
features which $\speriod{}$-particle systems exhibit.
    }

    \PCpost{2020-07-24}{
Bolotin and Treschev\rf{BolTre10} {\em Hill's formula}:
give two multidimensional generalizations of Hill's formula:
\begin{enumerate}
  \item to discrete Lagrangian systems (symplectic twist maps)
  \item to continuous Lagrangian systems
\end{enumerate}
They discuss additional aspects which appear in the presence
of symmetries or reversibility.
\begin{enumerate}
  \item study the change of the Morse index of a periodic trajectory
        after the reduction of order in a system with symmetries
  \item applications to stability of periodic orbits
\end{enumerate}


``In his study of periodic orbits of the 3 body problem, Hill obtained
a formula relating the characteristic polynomial of the monodromy
matrix of a periodic orbit and an infinite determinant of the
Hessian of the action functional. A mathematically correct
definition of the {\HillDet} and a proof of Hill's formula were
obtained later by Poincar\'e.''

Hill computed $\det H$ approximately  replacing $H$ by a $3\times 3$
matrix, which gave  quite a good approximation.
Hill did not prove convergence for the infinite determinant  $\det H$.
Poincar\'e\rf{poincare}
({\rm Vol.~I:} Solutions p\'eriodiques. Non-existence des int\'egrales
uniformes. Solutions asymptotiques)
explained the meaning of the {\HillDet} and presented a rigorous
proof of Hill's formula. The equation
appeared in 1983 for discrete Lagrangian systems in \refref{MacMei83} and
independently in ref\{4\}. Here $H$ is the finite Hessian
matrix associated with the action functional at the critical point
generated by the periodic solution. In ref\{5\} (see also ref\{6\}) a
general form of Hill's formula was obtained for a periodic solution
of an arbitrary Lagrangian system on a manifold. In this case $H$  is a
properly regularized Hessian operator of the action functional at the
critical point determined by a periodic solution.

Hill's relates $P$, the monodromy matrix of the periodic trajectory, to
the second variation of the action functional at the periodic trajectory,
with $H$ the corresponding  Hessian operator.

                                \toCB
The first dynamical application of Hill's formula is the well known
statement that the Poincar\'e degeneracy of a periodic trajectory (that
is, the condition that 1 is an eigenvalue of $P$) is equivalent to the
variational degeneracy (the condition $\det H=0$).


%\Bibitem{2}
%\Hrefs{http://www.zentralblatt-math.org/zmath/search/?an=JFM 24.1130.01}
%\by A.~Poincar\'e
%\book Les m\'ethodes nouvelles de la m\'ecanique c\'eleste.
%{\rm Vol.~I:} Solutions p\'eriodiques. Non-existence des int\'egrales uniformes.
%Solutions asymptotiques
%\publ Gau\-thier-Villars
%\publaddr Paris
%\yr 1892
%\totalpages 385
%\zmath{http://www.zentralblatt-math.org/zmath/search/?an=JFM 24.1130.01}
%\moreref
%\Hrefs{http://www.zentralblatt-math.org/zmath/search/?an=JFM 25.1847.03}
%\book {\rm Vol.~II:} M\'ethodes de MM.~Newcomb, Gyld\'en, Lindstedt et Bohlin
%\totalpages viii+479
%\yr 1893
%\zmath{http://www.zentralblatt-math.org/zmath/search/?an=JFM 25.1847.03}
%\moreref
%\Hrefs{http://www.zentralblatt-math.org/zmath/search/?an=JFM 30.0834.08}
%\book
%{\rm Vol.~III:}
%Invariants int\'egraux. Solutions p\'eriodiques du deuxi\`eme genre.
%Solutions doublement asymptotiques
%\totalpages iv+414
%\yr 1899
%\zmath{http://www.zentralblatt-math.org/zmath/search/?an=JFM 30.0834.08}
%\moreref
%\Hrefs{http://www.zentralblatt-math.org/zmath/search/?an=Zbl 0651.70002}
%\yr 1987, Vol.~I
%\serial  Les Grands Classiques Gauthier-Villars
%\publaddr Paris
%\publ Librairie Scientifique et Technique Albert Blanchard
%\totalpages xii+387
%\isbn 2-85367-093-7
%\bookinfo Reprint
%\mathscinet{http://www.ams.org/mathscinet-getitem?mr=0926906}
%\zmath{http://www.zentralblatt-math.org/zmath/search/?an=Zbl 0651.70002}
%% New methods of celestial mechanics. Vol. I. Periodic solutions.
%% Nonexistence of uniform integrals.  Asymptotic solutions
%\moreref
%\Hrefs{http://www.zentralblatt-math.org/zmath/search/?an=Zbl 0651.70002}
%\totalpages viii+487
%\bookinfo Vol.~II
%\isbn 2-85367-093-7
%\mathscinet{http://www.ams.org/mathscinet-getitem?mr=0926907}
%\zmath{}
%% The methods of Newcomb, Gylden, Lindstedt and Bohlin
%\moreref
%\Hrefs{http://www.zentralblatt-math.org/zmath/search/?an=Zbl 0651.70002}
%\bookinfo Vol.~III
%\totalpages vi+416
%\isbn 2-85367-093-7
%\mathscinet{http://www.ams.org/mathscinet-getitem?mr=0926908}
%\zmath{}
%% Integral invariants. Periodic solutions of the second kind. Doubly asymptotic solutions

%\bibitem{4} Treschev D.V., On the question of stability of periodic trajectories of the Birkhoff billiard,
%    {\it Vestnik Moskov. Univ. Ser I Mat-Mekh,} (1988) no 2, 44--50.
%\RBibitem{4}
%\zmath{http://www.zentralblatt-math.org/zmath/search/?an=Zbl 0667.58033}
%\yr 1988
%\issue 2
%\pages 44--50
%\mathscinet{http://www.ams.org/mathscinet-getitem?mr=0938065}
%\zmath{http://www.zentralblatt-math.org/zmath/search/?an=Zbl 0667.58033}
%\transl
%English transl.
%\Hrefs{http://www.zentralblatt-math.org/zmath/search/?an=Zbl 0667.58033}
%\by D.\,V.~Treshchev  (Treschev)
%\paper On the problem of the stability of the periodic trajectories of a Birkhoff billiard
%\jour Mosc. Univ. Mech. Bull.
%\vol 43
%\issue 2
%\pages 28--36
%\yr 1988

%\bibitem{5}  Bolotin S.V.,
%    On the Hill determinant of a periodic orbit.
%    {\it Vestnik Moskov. Univ. Ser I Mat-Mekh,} 1988, no.~3, 30--34.
%\RBibitem{5}
%\Hrefs{http://www.zentralblatt-math.org/zmath/search/?an=Zbl 0712.70023}
%\by ‘.\,‚.~®«®â¨­
%\paper Ž¡ ®¯à¥¤¥«¨â¥«¥ •¨««  ¯¥à¨®¤¨ç¥áª®© ®à¡¨âë
%\jour ‚¥áâ­. Œ®áª. ã­-â . ‘¥à.~1. Œ â¥¬., ¬¥å.
%\yr 1988
%\issue 3
%\pages 30--34
%\mathscinet{http://www.ams.org/mathscinet-getitem?mr=966862}
%\zmath{http://www.zentralblatt-math.org/zmath/search/?an=Zbl 0712.70023}
%\adsnasa{http://adsabs.harvard.edu/cgi-bin/bib_query?1988MVSMM.......30B}
%\transl
%English transl.
%\Hrefs{http://www.zentralblatt-math.org/zmath/search/?an=Zbl 0712.70023}
%\by S.\,V.~Bolotin
%\paper The Hill determinant of a periodic orbit
%\jour Mosc. Univ. Mech. Bull.
%\yr 1988
%\vol 43
%\issue 3
%\pages 7--11

%\bibitem{6}  Kozlov V.V. and Treschev D.V., {Billiards: a genetic introduction to the dynamics of systems with impacts.}
%    Translations of Mathematical Monographs, vol. 89, AMS, 1991.
%\RBibitem{6}
%\Hrefs{http://www.zentralblatt-math.org/zmath/search/?an=Zbl 0729.34027}
%\by ‚.\,‚.~Š®§«®¢, „.\,‚.~’à¥éñ¢
%\book ¨««¨ à¤ë. ƒ¥­¥â¨ç¥áª®¥ ¢¢¥¤¥­¨¥ ¢ ¤¨­ ¬¨ªã á¨á⥬ á 㤠ࠬ¨
%\publ ˆ§¤-¢® Œƒ“
%\publaddr Œ.
%\yr 1991
%\totalpages 168
%\isbn 5-211-01566-5
%\mathscinet{http://www.ams.org/mathscinet-getitem?mr=1157370}
%\zmath{http://www.zentralblatt-math.org/zmath/search/?an=Zbl 0729.34027}
%\transl
%English transl.
%\Hrefs{http://www.zentralblatt-math.org/zmath/search/?an=Zbl 0729.34027}
%\by V.\,V.~Kozlov and  D.\,V.~Treschev
%\book Billiards. A~genetic introduction to the dynamics of systems with impacts
%\serial Transl. Math. Monogr.
%\publ Amer. Math. Soc.
%\publaddr Providence, RI
%\totalpages viii+171
%\isbn 0-8218-4550-0
%\vol 89
%\yr 1991
%\mathscinet{http://www.ams.org/mathscinet-getitem?mr=1118378}
%\zmath{http://www.zentralblatt-math.org/zmath/search/?an=Zbl 0729.34027}

%\bibitem{7}  Liu C. and  Long Y., Iterated index formula for closed geodesics with applications, Science in China, 45(1)(2002) 9--28.
%\Bibitem{7}
%\Hrefs{http://www.zentralblatt-math.org/zmath/search/?an=Zbl 1054.53063}
%\by Ch.~Liu and Y.~Long
%\paper Iterated index formulae for closed geodesics with applications
%\jour Sci. China Ser.~A
%\vol 45
%\issue 1
%\yr 2002
%\pages 9--28
%\mathscinet{http://www.ams.org/mathscinet-getitem?mr=1894955}
%\zmath{http://www.zentralblatt-math.org/zmath/search/?an=Zbl 1054.53063}

%\Bibitem{8}
%\Hrefs{http://www.zentralblatt-math.org/zmath/search/?an=Zbl 1012.37012}
%\by Y.~Long
%\paper Index theory for symplectic paths with applications
%\serial Progr. Math.
%\vol 207
%\publ Birkh\"auser
%\publaddr Basel
%\yr 2002
%\totalpages xxiv+380
%\isbn 3-7643-6647-8
%\mathscinet{http://www.ams.org/mathscinet-getitem?mr=1898560}
%\zmath{http://www.zentralblatt-math.org/zmath/search/?an=Zbl 1012.37012}


    }

    \PCpost{2019-01-30}{
Downloaded the monograph by
Treschev and Zubelevich\rf{TreZub09}
{\em Introduction to the Perturbation Theory of Hamiltonian Systems}
\CBlibrary{TreZub09}
which contains a chapter {\em Hill's formula}. Here are some clippings:

``In 1886, in his study of stability of the lunar orbit, Hill\rf{Hill86} published
a formula which expresses the characteristic polynomial of the monodromy matrix
for a second order time periodic equation in terms of the determinant of a
certain infinite matrix.''

    }

    \PCpost{2018-09-29}{
Some papers that follow up on Bolotin and Treschev\rf{BolTre10}:

Xu and Weng\rf{XuWen08} {\em The calculation for characteristic
multiplier of {Hill's} equation in case with positive mean}

Hu and Wang\rf{HuWan11}
{\em Conditional {Fredholm} determinant for the {S}-periodic orbits in {Hamiltonian} systems}

Hu and Wang\rf{HuWan13}
{\em Conditional {Fredholm} determinant and trace formula for {Hamiltonian} systems: a survey}

Hu, Ou and Wang\rf{HuOuWa14} {\em Trace formula for linear {Hamiltonian} systems
with its applications to elliptic {Lagrangian} solutions}. Their eq.~(1.17)
Krein formula for eigenvalues of a linear Hamiltonian systems with $D$ \dof s
is intriguing.

Krein\rf{Krein83}
{\em The basic propositions of the theory of  {$\lambda$}-zones of stability of
a canonical system of linear differential equations with periodic coefficients}
(have not found a free version on line)

Davletshin\rf{Davletshin14}
{\em Hill's formula for {$g$}-periodic trajectories of {Lagrangian} systems}

Hu and Wang\rf{HuWan15}
{\em Eigenvalue problem of {Sturm-Liouville} systems with separated {\bcs}}

Hu and Wang\rf{HuWan16} {\em Hill-type formula and {Krein}-type trace
formula for {$S$}-periodic solutions in {ODEs}},
\arXiv{1504.01815}

Hu, Ou and Wang\rf{HuOuWa17}
{\em Hill-type formula for {Hamiltonian} system with {Lagrangian} {\bcs}},
\arXiv{1711.09182}: ``
The Hill-type formula connects the infinite determinant of the Hessian of the
action functional with the determinant of matrices which depend on the
monodromy matrix and {\bcs}. Consequently, we derive the
Krein-type trace formula and give nontrivial estimation for the eigenvalue
problem.
''

Hu, Wu and Yang\rf{HuWuYa18}
{\em Morse index theorem of {Lagrangian systems} and stability of brake orbit}

Sunada\rf{Sunada80}
{\em Trace formula for {Hill's} operators}

Carlson\rf{Carlson00} {\em Eigenvalue estimates and trace formulas for
the matrix {Hill's} equation}
    }

    \PCpost{2019-01-28}{
Downloaded monographs (have not started studying them as yet):

Maybe \HREF{https://www.encyclopediaofmath.org/index.php/Hill_equation}
{encyclopediaofmath.org} \emph{Hill equation} is a starting point for
this literature.

A whole book on the subject that we might have to have a look at:
Magnus and Winkler\rf{MagWin66} {\em Hill's Equation}
\CBlibrary{MagWin66}

Added a Bolotin conference abstract
\CBlibrary{Bolotin09}
and a Hu conference abstract
\CBlibrary{Hu13}.
    }

    \PCpost{2019-04-21}{
Kozlov\rf{Kozlov11} {\em Problem of stability of two-link trajectories in a
multidimensional {Birkhoff} billiard} has a simple derivation of Hill's formula
for billiards, where the Hessian is the second derivative of the length function.
    }

    \PCpost{2018-09-29}{
Agrachev\rf{Agrachev18} {\em Spectrum of the second variation},
\arXiv{1807.10527} writes:
Second variation of a smooth optimal control problem at a regular extremal is a
symmetric Fredholm operator. We study the spectrum of this
operator and give an explicit expression for its determinant in terms of
solutions of the Jacobi equation.

We study the spectrum of the second variation $D_u^2\varphi$ that is a
symmetric Fredholm operator of the form $I+K$, where $K$ is a compact
Hilbert--Schmidt operator. $K$ is NOT a trace class operator so that the trace
of $K$ and the determinant of $I+K$ are not well-defined in the standard sense.

[...] A simple example: for the 1\dmn\ linear control system
$\dot x=ax+u$ with the quadratic cost
$\varphi(u)=\int_0^1u^2(t)-(a^2+b^2)x^2(t)\,dt$
our determinantal
identity reads:
\beq
\prod\limits_{n=1}^\infty\left(1-\frac{a^2+b^2}{a^2+(\pi n)^2}\right)=
\frac{a\sin b}{b\,\mathrm{sh}\,a}
\,;
\ee{Agrachev18(1-1a)}
the case $a=0$ corresponds to the famous Euler identity
\beq
\prod\limits_{n=1}^\infty\left(1-\frac{b^2}{(\pi n)^2}\right)=\frac{\sin b}{b}
\,.
\ee{Agrachev18(1-1b)}
The example is simple, but, unfortunately, the paper itself is a hell to read...

    }


    \PCpost{2018-12-07}{
Reread Kook and Meiss\rf{kooknewt}
{\em Application of {Newton}'s method to {Lagrangian} mappings}.
They describe an Newton's method algorithm for finding periodic orbits
of {Lagrangian} mappings. The method is based on block-diagonalization
of the {\jacobianOrb} of the action function.
The explicit form of the Hessian displayed by Kook and Meiss reminds me of
Bolotin discrete Hill's formula (\refsect{LiTom17b:GenFctn}, Predrag post {\bf
2018-09-29} above, eq.~\refeq{MKMP84(3.6)PC}), maybe that's the way to derive it.
        }

\HLpost{2018-09-26}{
I worked through the \refexam{exam:MKMP84(3.4)} cat map example. The equation of motion is:
\bea
\left[\begin{array}{c} q_{n+1} \\ p_{n+1} \end{array}\right]
 = \left[\begin{array}{cc} s-1 & 1 \\ s-2 & 1 \end{array}\right]
   \left[\begin{array}{cc} q_n \\ p_n \end{array}\right] \quad \mod 1
\,.
\label{HL1dCatMap1}
\eea
    \PCedit{
Rewrite this equation as:
\bea
q_{n+1} &=&  q_n  + p_n + (s-2) q_n - \Ssym{n+1}^q
    \continue
p_{n+1} &=&  p_n  + (s-2) q_n  - \Ssym{n+1}^q - (\Ssym{n+1}^p - \Ssym{n+1}^q)
\,.
\label{HL1dCatMap2}
\eea
and compare with \refeq{exmPerViv2.1a},
\bea
q_{n+1} &=& q_{n}+p_{n+1} \qquad  \mod 1, \label{exmPerViv2.1b}
    \\
p_{n+1} &=& p_{n} + P(q_{n})             \label{exmPerViv2.1a}
\,,
\eea
so
\bea
q_{n+1} &=& q_{n} + p_{n+1}
    \continue
p_{n+1} &=&  p_n + (s-2) q_n - \Ssym{n+1}^p \, .
\label{PC1dCatMap2}
\eea
where the $\Ssym{n+1}^q$ seems happily absorbed into $p_{n+1}$.
The generating function (1-step Lagrangian density) is
\beq
L(q_{n},q_{n+1}) =
\frac{1}{2}(q_{n+1} - q_{n})^2 - V(q_{n})
    \,,\qquad
P(q) = - \frac{dV(q)}{dq}
\,,
\ee{MKMP84(3.6)HL}
and the potential energy is:
\bea
V(q_n) = - \frac{s-2}{2} q_n^2 + \Ssym{n+1}^p q_n \, .
\label{HL1dCatMap3}
\eea
The problem with this formulation is that
the potential energy contribution is
defined asymmetrically in \refeq{MKMP84(3.6)HL1}.
We should really follow Bolotin and Treschev\rf{BolTre10} eq.~(2.5),
and define a symmetric generating function
\beq
L(q_{n},q_{n+1}) =
\frac{1}{2}  (q_{n+1} - q_{n})^2
    - \frac{1}{2} [V(q_{n}) +V(q_{n+1})]
\,,
\ee{MKMP84(3.6)PC}
The first
variation \refeq{MacMei83(7)} of the action vanishes,
\bea
0 &=& L_2(q_{n+1}, q_n) + L_1(q_{n}, q_{n-1})
                            \label{HL1dCatMap5PC}\\
&=& q_n -q_{n+1} + (s-2)q_n - \Ssym{n+1}^p + q_n - q_{n-1} \continue
&=& - q_{n+1} + s q_n -q_{n-1} - \Ssym{n+1}^p
\,,
\nnu
\eea
hence
\beq
q_{n+1} - s q_n + q_{n-1} = - \Ssym{n+1}^p
\,.
\label{HL1dCatMap5aPC}
\eeq
Letting $\Ssym{n} = - \Ssym{n+1}^p$, we recover the Lagrangian
formulation \refeq{eq:CatMapNewton5}.
    } % end \PCedit

Alternatively,
Han's generating function (1-step Lagrangian density) is:
\bea
L(q_{n+1}, q_n)
&=& \frac{1}{2} \left[ p_{n+1}(q_{n+1}, q_n) \right]^2 - V(q_n)
\label{HL1dCatMap4}\\
&=& \frac{1}{2} (q_{n+1} - q_n +m_{n+1}^q - m_{n+1}^p)^2 + \frac{s-2}{2} q_n^2 -m^p_{n+1} q_n \, . \continue
\nnu
\eea
The action is the sum over the Lagrangian density over the orbit. The first
variation \refeq{MacMei83(7)} of the action vanishes,
\bea
0 &=& L_2(q_{n+1}, q_n) + L_1(q_{n}, q_{n-1})
    \continue
&=& q_n -q_{n+1} + m_{n+1}^p - m_{n+1}^q
\label{HL1dCatMap5}\\
&& + (s-2)q_n -m^p_{n+1} + q_n - q_{n-1} + m_n^q - m_n^p \continue
&=& - q_{n+1} + s q_n -q_{n-1} - (m_{n+1}^q - m_n^q + m_n^p)
\,,
\nnu
\eea
hence
\beq
- q_{n+1} + s q_n -q_{n-1} = m_{n+1}^q - m_n^q + m_n^p
\,.
\label{HL1dCatMap5a}
\eeq
Letting $m_n = m_{n+1}^q - m_n^q + m_n^p$, we recover the Lagrangian
formulation \refeq{eq:CatMapNewton5},
    \PCedit{
except for the wrong sign for $m_n$.
    }
Now I see why $m_n$'s are called `sources'.

But I think I missed something. I believe \refeq{ActionPC} is correct.
Because $\Box - {\mu}^2\mathsf{1}$ has negative determinant, so we can do
the Gaussian integral. But it seems like the action $S[X]$ is the
negative of the sum of Lagrangian along the orbit? Because if we sum
\refeq{HL1dCatMap4} along the orbit, the sign before $s$ should be
positive but in \refeq{ActionPC} it is negative...
}

  \item[2019-09-25 PC]
Levit and Smilansky\rf{LevSmi77} {\em A theorem on infinite products of
eigenvalues of {Sturm-Liouville} type operators} computes a
Gaussian path integral with a Laplacian kernel. Looks simple, but
I do not understand it.

Levit and Smilansky\rf{LevSmi77a} {\em A new approach to {Gaussian} path
integrals and the evaluation of the semiclassical propagator}.

  \item[2019-09-25 PC]
Han, can you clean up the rest, make it \refeq{Verdiere07:Sect11.5a}
and beyond into a derivation of the Hill's
formula for our paper - what follows is just a sketch in (close to) our
notations:

\HREF{http://www-fourier.ujf-grenoble.fr/~ycolver/}
{Colin de Verdi{\`{e}}re}\rf{Verdiere07} {\em Spectrum of the {Laplace}
operator and periodic geodesics: thirty years after} discusses the
mathematical history of ``Semi-classical trace formula,'' a formula
expressing the smoothed density of states of the Laplace operator on a
compact Riemannian manifold in terms of the periodic geodesics.
This seems to be an elaboration oof
\HREF{https://www-fourier.ujf-grenoble.fr/~ycolver/ihp05.pdf}
{these lectures} from which one can clip \& paste.

{Colin de Verdi{\`{e}}re}\rf{Verdiere07} and Levit and
Smilansky\rf{LevSmi77a} are deriving ``semiclassical'' or ``Gaussian path
integral'' evolution trace formulas, which lead to $\oneMinJ{}^{\half}$
rather than the classical $\oneMinJ{}$. That does not matter for our
purposes, which is the derivation of the Hill's formula, sketched in
\refeq{Verdiere07:Sect11.5a}, and on.

What we call \po's {\monodromyM} $\monodromy$ he seems to call (linear)
\Poincare\ map $\Pi$ a closed orbit computed on a hypersurface transverse
to the orbit, an ``inversible (symplectic) endomorphism of the tangent
space''. The \po\ weight $\oneMinJ{}$ shows up in his eq.~(5.5).

In his Theorem~11, Sect.~11, Colin de Verdi{\`{e}}re evaluates the
integral over $\exp(iS(x)/\hbar)$ in the stationary phase approximation,
with $S(W)$ having support on a critical manifold $W$ has a measure
$d\mu_W$ given by the quotient of the measure $|dx|$ by the ``Riemannian
measure'' on the normal bundle to $W$ associated to the Hessian of $S$
\beq
d\mu_W = \frac{|dx|}{|\det(\partial^2_{\alpha\beta}S)|^\half|dz|}
\,,
\ee{Verdiere07:The11}
where $z=(z_\alpha)$ are the coordinates on the normal bundle.

The Hessian is associated to a periodic Sturm-Liouville operator for
which many regularizations have  been proposed.

Elsewhere\rf{VanVleck28,LevSmi77a} $\det(\partial^2_{\alpha\beta}S)$ is
known as the Van Vleck determinant. Irrelevant to the problem at hand,
but a fun history read
\HREF{https://www.reed.edu/physics/faculty/wheeler/documents/}
{Nicholas Wheeler} is
\HREF{https://www.reed.edu/physics/faculty/wheeler/documents/Quantum\%20Mechanics/Class\%20Notes/Chapter\%203.pdf}
{here}.
In Sect.~11.6 {\em Regularized Determinants of continuous Sturm-Liouville
operators} he discretizes the $\cl{}$-cycle Hessian just as we do (except
he allows $s$ to vary along the path), with Dirichlet, or periodic
boundary conditions,
\bea
\mathcal{H}_\cl{} &=& \left (
		\begin{array}{cccccc}
		A & B & 0 & \dots & 0 & \transp{B} \\
		\transp{B} & A & B & \dots & 0 & 0 \\
		0 & \transp{B} & A  & \dots & 0 & 0 \\
		\vdots  & \vdots & \vdots & \ddots & \vdots & \vdots \\
		0 & 0 & 0 & \dots & A & B \\
		B & 0 & 0 & \dots & \transp{B} & A \\
		\end{array}
		\right )
\label{Verdiere07:Sect11.5a}
\eea
In general the [2$\times$2] matrices
\bea
A &=&  \frac{1}{*} \left (
		\begin{array}{cc}
		a_{11}&a_{12}\\
		a_{21}&a_{22}\\
		\end{array}
		\right )
\,,\qquad
B = -\,\frac{1}{*}  \left (
\begin{array}{cc}
* & * \\
* & * \\
\end{array}
\right )
\label{Verdiere07:Sect11.5b}
\eea
are time dependent,
$(A_\zeit,B_\zeit)$, but for our simple \templatt\ they are
constant, $(A_\zeit,B_\zeit)=(A,B)$.
    \ifblog
For some choices, see \refsect{sect:catMapAKS}~{\em Adrien's blog}.
    \fi
Denote $b=-\det B$, and by
\bea
 \left (
		\begin{array}{c}
		x_1\\
		x_2-x_1\\
		\end{array}
		\right )
&=& \jMps
 \left (
		\begin{array}{c}
		x_0\\
		x_1-x_0\\
		\end{array}
		\right )
\label{Verdiere07:Sect11.5c}
\eea
the 1-time step symplectic (canonical) transformation
\beq
\jMps = \left (
		\begin{array}{cc}
		\alpha&\beta\\
		\gamma&\delta\\
		\end{array}
		\right )
\,.
\ee{Verdiere07:Sect11.5d}
From his theorem 13 he gets Hill's formula
$\det H = \det(1-\jMps_p)$ for the periodic case.

\item[2019-09-28 PC]
It is clearer and clearer that the smart way of doing the
multishooting Newton and various related ``noisy'' dynamics problems is
by discrete Fourier (\Cn{n} cyclic group) irreps diagonalization.
So I might have to rework several earlier papers:

I had inverted Newton Jacobian matrix often, see for example
eq.~(16) and onward in Cvitanovi{\'c}, Dettmann, Mainieri and
Vattay\rf{noisy_Fred},
\HREF{http://chaosbook.org/~predrag/papers/noise.pdf} {click here}. I
have also introduced the notation for finite-time (shorter than the
period) Jacobian matrices, see for example eq.~(69) in Cvitanovi{\'c} and
Lippolis\rf{CviLip12},
\HREF{http://chaosbook.org/~predrag/papers/CviLip12.pdf} {click here}.
But I have never done it the way I should have, by a discrete Fourier
transform, into sum of irreps of \Cn{n} (AKA Fourier modes). Probably best
to use characters?

\item[2019-09-28 PC]
I think I have finally committed the long awaited conceptual
breakthrough. To remind everyone - one unsolved problem in Matt and my
work\rf{GuBuCv17} is ``Hill's formula'' for the first-order time
derivative dissipative dynamics, which relates linear stability of a
\spt\ pattern to $(1-\jMps)$ temporal evolution stability.

I'm writing this up in \refchap{chap:stability}~{\em \Spt\ stability}.

\item[2019-10-01 PC]
Cao and Voth\rf{CaoVot96} {\em Semiclassical approximations to quantum
dynamical time correlationfunctions}
\HREF{http://web.mit.edu/jianshucaogroup/pdfdir/JCP104-0273-1996-2.pdf}
{(click here)}: ``
find an alternative to evaluate the Jacobi matrices and have thereby found
it necessary to derive the initial-value expression from a new
perspective. Straight-forward and self-contained, this derivation leads
to a discretized expression for the Jacobi matrices and a simple
interpretation of the Maslov-like index.
''

They derive the Jacobi equations from a dicretization, their Appendix B.
These equations evolve sublocks of the Jacobian that maintain the
symplectic invariance.

Langouche, Roekaerts and Tirapegui\rf{LaRoTi81}
{\em {WKB} Expansion for arbitrary {Hamiltonians}} might be of interest,
but I have not studied it.

    \PCpost{2019-10-13}{
Guckenheimer and Meloon\rf{GM00aut} define a ``symmetric multiple
shooting algorithm'' as is a small modification of the forward multiple
shooting method that makes the method time reversible. Let
\(
p=(\ssp_1,\ssp_2,\cdots,\ssp_\cl{p})
\,,
\)
as in \refeq{nXdCycle}. They evaluate some ``Taylor polynomials'' at
time-interval midpoints, I admit not to see how their formulas are
time reversible. I believe they replace \refeq{errorVecs1D}
by segments that traverse a time interval in both direction
\beq
 F_{GM}(\hat{\ssp}) \, = \,
\left (
\begin{array}{l}
  \hat{\map}_\cl{p} - \hat{\map}^{-1}_1 \\
  \hat{\map}_1 - \hat{\map}^{-1}_2\\
  ~~~\cdots \\
  \hat{\map}_{\cl{p}-1} - \hat{\map}^{-1}_\cl{p}
\end{array}
\right )
\,,\qquad \hat{\map}_k = \map(\hat{\ssp}_k)
\,,
\ee{errorVecsTrevers}
Now the orbit \jacobianM\ \refeq{fcnewt} picks up the derivatives of
inverse map along the diagonal (instead of the identity matrices). They
manipulate it and show it is the same Jacobian as the forward shooting
one.

For my taste having a diagonal and sub-diagonal is not time-symmetric
enough. Inspired by the formula for the discrete Laplacian
\refeq{PerViv2.2}, % PC 2019-10-13: find a better equation number
I suggest trying the tri-diagonal vector field
\beq
 H(\hat{\ssp}) \, = \,
\left (
\begin{array}{l}
  \hat{\map}^{-1}_2 - 2 \hat{\ssp}_1 + \hat{\map}_\cl{p}\\
  \hat{\map}^{-1}_3 - 2 \hat{\ssp}_2 + \hat{\map}_1\\
  ~~~\cdots \\
  \hat{\map}^{-1}_{\cl{p}-1} - 2 \hat{\ssp}_\cl{p} + \hat{\map}_{\cl{p}-1}\\
\end{array}
\right )
\,,\qquad \hat{\map}_k = \map(\hat{\ssp}_k)
\,,
\ee{errorVecsTrv}
that reverses its direction under time reversal.
The orbit  \jacobianM\
\beq
{\cal H} \,=\,
 (\sigma \jMps)^{-1} -2\jMps +\sigma \jMps
\,,
\ee{PCselfAdjMatr}
is of self-adjoint form. We have to show that its determinant is
the Hill's formula.

Is it what we need for the Hessian / Lagrangian case? I suspect
that the \templatt\ case - uniform stretching, is easily brought
to the \templatt\ orbit stability form, by rescaling \refeq{PCselfAdjMatr}
so that off-diagonal $J's$ are absorbed into the diagonal ${\mu}^2$ factors.

Read Doedel \etal\rf{doedel_computation_2003} \CBlibrary{DPKDGV03}. They
show how two-point boundary value problem continuation software like AUTO
[Doedel et al., 1997; Doedel et al., 2000] can be used to compute
families of periodic solutions of conservative systems, \ie\ systems
having a first integral. Seems the same as chaos book as far as adding
and additional constraint is concerned. See no word ``determinant''
anywhere...
    }


    \PCpost{2020-08-01} {
I guess I should have looked more closely. The original the {\HillDet}
(see \HREF{https://mathworld.wolfram.com/HillDeterminant.html}
{mathworld.wolfram}) is, with various terms absorbed into our
stretching parameter ${s}$, a 3-term recurrence which is \emph{exactly} our
$d=1$ \templatt, with our {\HillDet}, except Hill did it for the
oscillatory parameter value ${\mu}^2<0$.

Check Morse, P. M. and Feshbach, H.
\emph{Methods of Theoretical Physics}, Part I.
New York: McGraw-Hill, pp. 555-562, (1953).

Magnus and Winkler\rf{MagWin66} {\em Hill's Equation}
\HREF{http://ChaosBook.org/library/MagWin66djvu} {(click here)}.

Dan Rothman told me to look at J. J. Stoker (but it is not in {\em
Differential geometry}), so it's in the wave mechanics book...
Have not found it yet.
    }

\end{description}


\newpage
\section{Generating function literature}
\label{sect:GenFuncLit}


   % *********************************************************************
\hfill   {\color{red} For the latest entry, go to the bottom of this section}

\bigskip


\begin{description}

    \PCpost{2016-11-11}{
I still cannot get over how elegant the {\GO}\rf{GutOsi15}
{\catlatt} is. It is \emph{linear}! ($\mod 1$, that is - the map
is continuous for integer $s$). A 1\dmn\ cat map has a Hamiltonian
\refeq{BarShc06Ham}, and they have written down the 2\dmn\ Lagrangian,
their Eq.~(3.1) (or the ``generating function'', as this is a mapping).
Their {\catlatt} generating function is defined on a {\spt}
cylinder, infinite in time direction,
\bea
\action(\coord_t,\coord_{\zeit+1})
    &=&
- \sum_{n=1}^N \coord_{nt} \coord_{1+n,t}
- \sum_{n=1}^N \coord_{nt} (\coord_{n,t+1}+m^\coord_{n,t+1})
+\frac{a}{2}\sum_{n=1}^N \coord^2_{nt}
    \ceq
+\frac{b}{2}\sum_{n=1}^N (\coord_{n,t+1}+m^\coord_{n,t+1})^2-m^p_{n,t+1}\coord_{n,t+1}
\,,
\label{GutOsi15-3.1:model}
\eea
where $\coord_t=\{\coord_{nt}\}^N_{n=1}$ is a spatially periodic state
at time $t$, with  $\coord_{nt}$ being  the coordinate of
$n$th ``particle'' $n=1\dots N$ at the moment of time $t\in\mathbb{Z}$, and
$m^q_{n,t+1}, m^p_{n,t+1}$ are integer numbers which stand for  winding
numbers along the $q$ and $p$ directions of the 2N-torus.
Note that
\(
\ssp_{1+n,t} = \ssp_{1+(n\!\!\!\!\!\mod{N}),t}
\,.
\)
The coefficients $a,b$, $s=a+b$ are integers which they specify.
Gutkin and Osipov refer to the map generated by the action
(\ref{GutOsi15-3.1:model}) as non-perturbed \textit{coupled cat map}, and to an
\twot\ $p$ as a ``many-particle periodic orbit'' (MPO) if $q_{nt}$ is
doubly-periodic, or ``closed,'' \ie,
\[
q_{nt} = q_{n+L,t+T}
    \,,\quad
n = 1,2,\cdots, L
    \,,\quad
t = 1,2,\cdots, T
    \,.
\]

{\bf 2D symbolic representation}
Encode  each \twot\ (many-particle periodic orbit)  $p$ by a
two dimensional (periodic) lattice of symbols $a_{nt}$,
$(nt)\in\mathbb{Z}^2$, where symbols  $a_{nt}$ belong to some alphabet $\A$
of a small size. Each \twot\ $p$ is represented  by
$L\times T$ toroidal  array of symbols:
\[
\Aa_{p}=\{a_{nt}| \, (nt)\in \integers^2_{LT} \}
\,.
\]
The Hamiltonian equations of motion can be generated using \refeq{MKMP84(3.2)}
but who needs them? Remember, a field theorist would formulate a spacetime
symmetric field theory in a Lagrangian way, with the invariant action.
    }

\PCpost{2016-11-11}{
Percival and Vivaldi\rf{PerViv} state the Lagrangian variational
principle in
Sect.~6. {\em Codes, variational principle and the static model}:
    \PC{2016-11-12}{eventually move to \refrem{rem:AnosovMaps}}

The Lagrangian variational principle for the sawtooth map on the real
line states that the action sum
\refeq{MKMP84(3.6)HL1}
is stationary with respect to variations of any finite set of
configurations $\ssp_{\zeit}$. Their discussion of how ``elasticity''
works against the ``potential'' is worth reading. For large values of
stretching parameter $s$, the potential wins out, and the state
$\ssp_{\zeit}$ falls into the $m_t$th well:
``the code may be considered as a labelling of the local minima of the
Lagrangian variational principle.''

Dullin and Meiss\rf{DulMei98} {\em Stability of minimal periodic orbits}
does the calculations in great detail.
    }

\PCpost{2016-11-11}{
``mean action'' =  the action divided by the period
    }

    \PCpost{2018-12-07}{
as shown in (copied here from ChaosBook)
\refexam{exam:2DHamiltFlow},
\refexam{exam:Dzeta2dHamMaps}, and
\refexam{exam:DzetaHam}
Hamiltonian \Fd\ and \dzeta\ have a special form. Recheck
against our cat map \zetatop.
    }

    \PCpost{2019-10-14}{
The Jacobi operator acts on a discrete periodic lattice as
\[
L\,u(t)=a(t+1)u(t+1)+b(t)u(t)+a(t-1)u(t-1)\,,
\]
where a(t) and b(t) are real valued for each $t\in\integers$,
and M-periodic in t.
Jacobi operators are the discrete analogue of Sturm–Liouville operators, with
many similarities to Sturm–Liouville theory.
    }





\end{description}
