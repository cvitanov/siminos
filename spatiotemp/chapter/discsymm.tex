\svnkwsave{$RepoFile: siminos/spatiotemp/chapter/discsymm.tex $}
\svnidlong {$HeadURL: svn://zero.physics.gatech.edu/siminos/spatiotemp/chapter/discsymm.tex $}
{$LastChangedDate: 2019-10-18 20:49:56 -0400 (Fri, 18 Oct 2019) $}
{$LastChangedRevision: 6859 $} {$LastChangedBy: mgudorf3 $}
\svnid{$Id: discsymm.tex 6859 2019-05-09 20:51:31Z predrag $}

\chapter{Symmetries of space-time \KSe}
\label{chap:disc_symm}

In this chapter we look at the discrete subgroups of $\On{2} \times
\SOn{2}$ that arise when calculating \twot\ spatiotemporal solutions, and
write up the projection operators algebra needed to restrict the
computations to invariant subspaces. The discussion will be restricted to
solutions that lie in the flow invariant antisymmetric subspace $\bbU^+$,
whose spatiotemporal subgroup is $\Zn{2}\times (e)$, and \ppo s whose
discrete symmetry subgroup is $\Zn{2} \times \Cn{2}$.
    \PC{2018-04-24}{Summarize here the \KS\ analogue of relevant parts of the
    \pCf\ \eqva\ and \reqva\ classification of Gibson, Halcrow and
    Cvitanovi{\'c}\rf{GHCW07}.}

\section{Symmetries of \KSe}
\label{sect:KSsymm}
% siminos/spatiotemp/chapter/KSsymm.tex
% $Author: predrag $ $Date: 2020-05-07 17:34:06 -0400 (Thu, 07 May 2020) $

% called by
%           siminos/spatiotemp/chapter/spatiotemp.tex
%           siminos/tiles/GuBuCv17.tex

%\section{Symmetries of \KSe}
%\label{sect:KSsymm}

The \KSe\ \refeq{e-ks} is equivariant under spatial translations, spatial
reflections and temporal translations and Galilean transformations.
The Galilean symmetry $u(\conf,\zeit)$ is a solution,
then $u(x -ct,t) -c $, with $c$ an arbitrary constant
speed, is also a solution. Without loss of generality, in our
calculations we shall set the mean velocity of the front to zero,
\beq
\spaceAver{u}(\zeit)
  \,=\, \int_0^{\speriod{}} d\conf \, u(\conf,\zeit) = 0
\,.
\ee{GalInv}

If the system is compactified on a
2-torus, with periodic boundary conditions
$u(\conf,\zeit)=u(\conf+\speriod{},t+\period{})$, the symmetry group is
\beq
\Group = \On{2}_\conf \times \SOn{2}_\zeit
        = \Dn{1,\conf} \ltimes \SOn{2}_\conf \times \SOn{2}_\zeit
\,.
\ee{KSsymms}
The elements of the 1-parameter group of spatial shifts and reflections are
$\On{2}_\conf:\{\Shift_{\shift/\speriod{}},\Refl \}$, and
the elements of the 1-parameter group of temporal shifts are
$\SOn{2}_\zeit:\{\Shift_{\shift/\period{}}\}$.
If $u(\conf,t)$ is a solution, then $\Shift_{\shift/\speriod{}}\, u(\conf,t) =
u(\conf+\shift,t)$ is an equivalent solution for any shift $0 \leq \shift <
\speriod{}$, as is the reflection (`parity' or `inversion')
\beq
    \Refl \, u(\conf,\zeit) = -u(-\conf,\zeit)
\,.
\ee{KSparity}

%%%%%%%%%%%%%%%%%%%%%%%%%%%%%%%%%%%%%%%%%%%%%%%%%%%%%%%
% from \example{Invariance under fractional rotations.}{\label{exam:FractRot}
Consider a cyclic group
\[
\Cn{m} = \{e,\trHalf{},\trHalf{}^{2},\cdots,\trHalf{}^{m-1}\}
\,,\qquad \trHalf{}^m= e
\,.
\]
where $\trHalf{}$ is an \SOn{2} rotation by $2\pi/m$. $\Cn{m}$
is a discrete subgroup of \SOn{2} for any $m=2,3,\cdots$ .

A field $u$ on the $2\pi/m$ domain is now a
tile whose $m$ copies tile the entire domain. It is periodic on the
$2\pi/m$ domain, and thus has Fourier expansion with Fourier modes
$\exp(2\pi\ii m j \ssp)$. This means that $\SOn{2}$ always has an
infinity of discrete subgroups $\Cn{2}, \Cn{3}, \cdots,\Cn{m}, \cdots$;
for each the non-vanishing coefficients are only for Fourier modes whose
wave numbers are multiples of $m$.
%  } %end\example{exam:FractRot}
%%%%%%%%%%%%%%%%%%%%%%%%%%%%%%%%%%%%%%%%%%%%%%%%%%%%%%%

If we take discrete subgroups in $\Cn{2,\conf}$ in place of both \SOn{2} groups
then the order of the discrete group
\(
\tilde{\Group} = \Dn{1,\conf} \ltimes \Cn{2,\conf} \times \Cn{2,t}
\)
is of order $8$.
All \spt\ symmetries of discussion can be described by \emph{isotropy subgroups},
which are symmetry subgroups which leave solutions invariant.
Specifically the discrete symmetries,
spatial reflection symmetry and \spt\ shift-reflection symmetry. These particular
symmetries have isotropy subgroups
\(
\Group = \Dn{1,\conf}
\)
and
\(
\Group = \Cn{2,t}
\)
respectively. To cover the discrete \spt\ symmetries that
are realized by \twots\ we need to investigate the group
\(
\Group = \Dn{1,\conf} \times \Cn{2,t}\,,
\)
because its description includes
reflection and shift-reflection symmetries. The term shift-reflection
denotes solutions which are left invariant only after spatial reflection
and a time translation by half a period. We have disregarded
\Cn{2,\conf} for the discussion of discrete symmetries. This is
permitted because spatial half-cell shifts, even in combination
with other group elements only permit equivariant solutions,
not invariant. Solutions invariant under half-cell shifts in
space would have to be doubly periodic in space. For combination
with the cyclic group in time it would be a yet undiscovered
\twot which is invariant after a half-cell shifts in space and
then time. The general $\Cn{M,\conf} \times \Cn{N,t}$ case
is harder to describe; if $M=N$ then one example of a way
to construct an invariant solution would be to construct
a solution which would be invariant after $N$ total rotations.
For instance, a solution with the form
\beq
u(x,t) =\left[\begin{array}{c}
1\,2\,3 \\
3\,1\,2 \\
2\,3\,1
\end{array}\right]
\ee{e-uCmCn}
would be invariant after a cycle consisting of
one space rotation and two time rotations or
two space rotations and one time rotation (each by one third
of the domain in the respective, positive directions). This
seems incredibly unlikely as it requires the solution to be comprised of
permutations of three patterns which are all equivalent in domain size.
This unlikelihood only gets worse for higher order cyclic groups
We return from our tangent by getting into the meat of the
discussion by analyzing the group $ \Dn{1,\conf} \times \Cn{2,t}$.
We demonstrate some standard group theoretic calculations such as
looking at the character table \reftab{D1C2table} and projection operators
\refeq{e-D1C2operators}.

%%%%%%%%%%%%%%%%%%%%
\begin{table}[h!]
\caption{\label{D1C2table}
Because the direct product group is abelian we only have one dimensional
representations and as such the character table follows directly.
    }
\centering
\begin{tabular}{|c|c|c|c|c|}
\quad & $e$ & $\Refl_x$ & $\trHalf{t}$ & $\Refl_x \trHalf{t}$ \\
\hline
$E$ & 1 & 1 & 1 & 1 \\
$\Gamma_1$ & 1 & 1 & -1 & -1 \\
$\Gamma_2$ & 1 & -1 & 1 & -1 \\
$\Gamma_3$ & 1 & -1 & -1 & 1 \\
\end{tabular}
\end{table}
The character table \reftab{D1C2table}, leads
to the construction of four linear projection operators
\bea \label{e-D1C2operators}
P^{++} &=& \frac{1}{4}(1 +\Refl_x +\trHalf{\zeit}+ \Refl_x \trHalf{\zeit}) \continue
P^{+-} &=& \frac{1}{4}(1 +\Refl_x - \trHalf{\zeit}- \Refl_x \trHalf{\zeit}) \continue
P^{-+} &=& \frac{1}{4}(1 -\Refl_x  + \trHalf{\zeit} - \Refl_x \trHalf{\zeit}) \continue
P^{--} &=& \frac{1}{4}(1-\Refl_x  - \trHalf{\zeit} + \Refl_x \trHalf{\zeit})
\,,
\eea
where $\Refl_x$,$\trHalf{\zeit}$ denote spatial reflection about the $x=0$ line and time translation
by half a period, respectively.
The solution space can be decomposed into the irreducible subspaces produced
by these projection operators
$\bbU = \bbU^{++} \oplus \bbU^{+-} \oplus \bbU^{-+} \oplus \bbU^{--}$.
In the context of a \rv\ \spt\ Fourier basis each of these subspaces corresponds
to a subset of coefficients in the expansion \refeq{e-RealFourier}
\bea \label{D1C2subspaces}
u^{-+}(\conf,\zeit) &=& \sum_{k} \sum_{j} \akj \cos(\freqj \tn)\cos(\wavek \xm) \continue
u^{--}(\conf,\zeit) &=& \sum_{k} \sum_{j} \bkj \sin(\freqj \tn)\cos(\wavek \xm) \continue
u^{++}(\conf,\zeit) &=& \sum_{k} \sum_{j} \ckj \sin(\wavek \xm)\cos(\freqj \tn) \continue
u^{+-}(\conf,\zeit) &=& \sum_{k} \sum_{j} \dkj \sin(\wavek \xm)\sin(\freqj \tn) \,.
\eea
We won't use these equations just yet but they are good for classifying what each
projection operator corresponds to. This classification comes naturally
from the parity (odd, even) of the trigonometric functions therein. They can later
be used to derive constraints on the \spt\ \Fcs\ pertaining to invariance
under certain symmetry operations.

Before we continue,
it will first be convenient to calculate the relationships between
the projection operators \refeq{e-D1C2operators} and the spatial differentiation operator.
The utility comes later when we apply these projection operators to the \KSe, specifically
when considering the nonlinear term.
\bea \label{D2C2projopderivx}
D_{\conf} P^{++} &=& \frac{1}{4}D_{\conf}(1 +\Refl_x  + \trHalf{\zeit} + \Refl_x \trHalf{\zeit}) \continue
                 &=& \frac{1}{4}(1 -\Refl_x  + \trHalf{\zeit}- \Refl_x \trHalf{\zeit})D_{\conf} \continue
                 &=& P^{-+}D_{\conf} \continue
D_{\conf} P^{+-} &=& \frac{1}{4}D_{\conf}(1 + \Refl_x  - \trHalf{\zeit}- \Refl_x \trHalf{\zeit}) \continue
                 &=& \frac{1}{4}(1 -\Refl_x  - \trHalf{\zeit}+ \Refl_x \trHalf{\zeit})D_{\conf} \continue
                 &=& P^{--}D_{\conf} \continue
D_{\conf} P^{-+} &=& \frac{1}{4}D_{\conf}(1 -\Refl_x  + \trHalf{\zeit} - \Refl_x \trHalf{\zeit}) \continue
                 &=& \frac{1}{4}(1 +\Refl_x + \trHalf{\zeit} + \Refl_x \trHalf{\zeit})D_{\conf} \continue
                 &=& P^{++}D_{\conf} \continue
D_{\conf} P^{--} &=& \frac{1}{4}D_{\conf}(1 -\Refl_x  - \trHalf{\zeit} + \Refl_x \trHalf{\zeit}) \continue
                 &=& \frac{1}{4}(1 +\Refl_x  - \trHalf{\zeit} - \Refl_x \trHalf{\zeit})D_{\conf} \continue
                 &=& P^{+-}D_{\conf}\,.
\eea
These identities allow us to rewrite the nonlinear terms
present in each projection of the \KSe\ as derivatives
of projection components as opposed to projections of derivatives,
which we believe leads to less confusing analysis. Note that
the effect can be summarized by flipping the first $\pm$, pertaining
to the coefficient of the spatial reflection terms in \refeq{e-D1C2operators}
The surviving nonlinear terms after the application of each projection operator are
as follows
\bea \label{e-D1C2nonlinear}
P^{++}(u\partial_x u) &=& u^{\pm \pm}\partial_{\conf}(u^{\pm \pm})\continue
P^{+-}(u\partial_x u) &=& u^{\pm \pm}\partial_{\conf}(u^{\pm \mp})\continue
P^{-+}(u\partial_x u) &=& u^{\pm \pm}\partial_{\conf}(u^{\mp \pm})\continue
P^{--}(u\partial_x u) &=& u^{\pm \pm}\partial_{\conf}(u^{\mp \mp})\,.
\eea
Using these relations \refeq{e-D1C2nonlinear} we can produce the projections
of the \KSe\ onto the different irreducible subspaces, noting that the projection operator
commutes with the linear terms such that
\bea \label{e-KSEprojections}
P^{++}F(u) &=& u_{\zeit}^{++}+u_{\conf \conf}^{++}+u_{\conf \conf \conf \conf}^{++} \continue
           &+& (u^{++}\partial_{\conf}(u^{++}) + u^{+-}\partial_{\conf}(u^{+-}) \continue
           &+& u^{-+}\partial_{\conf}(u^{-+}) + u^{--}\partial_{\conf}(u^{--}))  \continue
P^{+-}F(u) &=& u_{\zeit}^{+-}+u_{\conf \conf}^{+-}+u_{\conf \conf \conf \conf}^{+-}\continue
           &+&(u^{++}\partial_{\conf}(u^{+-}) + u^{+-}\partial_{\conf}(u^{++}) \continue
           &+& u^{-+}\partial_{\conf}(u^{--}) + u^{--}\partial_{\conf}(u^{-+}))  \continue
P^{-+}F(u) &=& u_{\zeit}^{-+}+u_{\conf \conf}^{-+}+u_{\conf \conf \conf \conf}^{-+}\continue
           &+&(u^{++}\partial_{\conf}(u^{-+}) + u^{+-}\partial_{\conf}(u^{--}) \continue
           &+& u^{-+}\partial_{\conf}(u^{++}) + u^{--}\partial_{\conf}(u^{+-})) \continue
P^{--}F(u) &=& u_{\zeit}^{--}+u_{\conf \conf}^{--}+u_{\conf \conf \conf \conf}^{--}\continue
           &+&(u^{++}\partial_{\conf}(u^{--}) + u^{+-}\partial_{\conf}(u^{-+}) \continue
           &+& u^{-+}\partial_{\conf}(u^{+-}) + u^{--}\partial_{\conf}(u^{++}))\,.
\eea
Solutions to \refeq{e-ks} satisfy $F = 0$ by definition so
by extension solutions must also satisfy $P^{\pm \pm}F=0$.
With this we can determine the combinations of projection operators whose equations
are ``self contained''. This is similar to the notion of \textit{flow invariant subspaces}
but because we do not have dynamics we can't really use this term. Instead,
these subspaces correspond to a constrained set of equations that solutions with
particular discrete symmetries must adhere to.
For example, assume that the only nonzero component $u$ is $u=u^{++}$.
Substitution of \refeq{e-KSEprojections} yields
\bea \label{e-KSplusplus}
P^{++}F(u^{++}) &=& u_{\zeit}^{++}+u_{\conf \conf}^{++}+u_{\conf \conf \conf \conf}^{++}
                +u^{++}\partial_{\conf}(u^{++}) \continue
P^{+-}F(u^{++}) &=& 0 \continue
P^{-+}F(u^{++}) &=& 0 \continue
P^{--}F(u^{++}) &=& 0 \,,
\eea
so $\bbU^{++}$ is
an invariant subspace. In fact,
this subspace
corresponds to equilibria solutions which
live on the $\period{}=0$ line. The meaning
of self contained in this example is that we
assumed that $u=u^{++}$ and the only nonzero part
of \refeq{e-KSplusplus} is the $P^{++}F(u^{++})$ component.
Perhaps a more elucidating example is generated
by the assumption that $u=u^{--} \neq 0 $. Substitution
yields
\bea \label{e-KSminusminus}
P^{++}F(u^{--}) &=& u^{--}\partial_{\conf}(u^{--}) \continue
P^{+-}F(u^{--}) &=& 0 \continue
P^{-+}F(u^{--}) &=& 0 \continue
P^{--}F(u^{--}) &=& u_{\zeit}^{--}+u_{\conf \conf}^{--}+u_{\conf \conf \conf \conf}^{--}
\eea
which indicates that the equations are not self contained as components
other than $P^{--}F(u^{--})$ are non-zero. Recall that each of these
components is equivalently equal to zero. Because these equations represent
scalar field values defined at every $\conf, t$ this implies that in order
to satisfy
$u^{--}\partial_{\conf}(u)^{--}=0$
either $u^{--}$, its derivative $\partial_{\conf}(u)^{--}$, or both must equal
to zero at every point on the \spt\ domain. The only nontrivial
possibility is if there are (at least)
two disjoint regions such that $\Omega_u=\{(\conf,\zeit):u(\conf,\zeit)=0\}$
and $\Omega_{u_x}=\{(\conf,\zeit):u_x(\conf,\zeit)=0\}$. By smoothness, if
$u=0$ then $u_x=0$. This implies
that $u_x=0$ for all $(\conf,\zeit)$; if
$u_x=0$ everywhere and $u=0$ for some $(\conf,\zeit)$ then it must
be the case that $u=0$ everywhere
which contradicts our
original assumption that $u=u^{--} \neq 0 $.
The rest of the symmetry invariant subspaces follow from a
similar substitutions. To expedite the derivation process, note
that the equation for $P^{++}F$ contains
all of the symmetric terms $u^{\pm \pm}\partial_{\conf}(u^{\pm \pm})$
such that there is no
possibility of an invariant subspaces
which does not intersect $\bbU^{++}$.
Following a process of elimination we can show that the possible
symmetry invariant subspaces are $\bbU^{++}$, $\bbU^{++}\oplus \bbU^{--}$,
$\bbU^{++}\oplus \bbU^{+-}$ and $\bbU^{++}\oplus \bbU^{-+}$ and
of course the full space $\bbU$. There are no triplet subspaces
(comprised of three components) which can be shown using
the parity of the different subspaces. We can interpret
these subspaces by addition of the corresponding projection
operators \refeq{e-D1C2operators}
\bea \label{e-invariantoperators}
P_{0}\equiv P^{++} &=& \frac{1}{4}(1 +\Refl_x +\trHalf{\zeit}+ \Refl_x \trHalf{\zeit}) \continue
P_{\Refl_x}\equiv P^{++}+P^{+-} &=& \frac{1}{2}(1 + \Refl_x) \continue
P_{\trHalf{\zeit}}\equiv P^{++}+P^{-+} &=& \frac{1}{2}(1 + \trHalf{\zeit}) \continue
P_{\Refl_x \trHalf{\zeit}}\equiv P^{++}+P^{--} &=& \frac{1}{2}(1+\Refl_x \trHalf{\zeit})
\,.
\eea
With these projection operators we can interpret the symmetry
invariant subspaces as follows:
$\bbU^{++}$ represents the fixed point ($\period{}=0$) subspace,
$\bbU^{++}\oplus \bbU^{+-}$ the spatial reflection invariant subspace,
$\bbU^{++}\oplus \bbU^{--}$ the shift-reflection invariant subspace,
and lastly $\bbU^{++}\oplus \bbU^{-+}$ which
contains solutions that are invariant after a half period shift
in time. This subspace of
``twice repeating'' solutions is trivial and not useful; doubly periodic solutions
can always be made to repeat twice in time by definition. The interpretation
of the corresponding subspace is therefore not very intuitive.

The next question to answer is how continuous spatial translation symmetry
manifests itself in this \spt\ context.
How do these subspaces relate to the continuous spatial translation symmetry?
The three subspaces $\bbU_0,\bbU_{\Refl_x},\bbU_{\Refl_x \trHalf{t}}$
share an interesting property in a real valued (\SOn{2}) representation.
Specifically, the subspaces of \spt\ \Fcs\ corresponding
to invariance under these discrete symmetries
are all orthogonal to the space of spatial translations. This can
be seen by acting on the different orbits with the spatial
derivative operator which is the generator of infinitesimal translations.
The subgroup
\(
H = \Cn{M,\conf}
\)
represents continuous spatial translation symmetry after discretization.
We utilize a co-moving frame ansatz to handle this
symmetry, which we will now develop. As previously mentioned,
we use a \rv\ ($\SOn{2}$)
representation for the \spt\ \Fcs. This choice makes the matrix
representations of the group elements slightly more complicated
as they will be block diagonal as opposed to exactly diagonal.
Note that because of doubly periodic boundary conditions,
translations
are the same as rotation.
The matrix representation of the group element which spatially rotates $M$
Fourier modes by a value $\theta$ is a block diagonal matrix with $M$ blocks; each
block being a representation of two dimensional
rotations for the corresponding wavenumber $k$
\beq \label{e-SOnGroupElement}
\tilde{\LieEl}(\theta) \equiv
\begin{bmatrix}
\cos \wavek\theta  & -\sin \wavek\theta \\
\sin \wavek\theta & \cos \wavek\theta
\end{bmatrix}\,.
\eeq
This block diagonal matrix acts on $M$ Fourier modes;
the corresponding extension to the set of \spt\ \Fcs\
is simply $N$ copies of \refeq{e-SOnGroupElement}. In other
words we have $N$ blocks of \refeq{e-SOnGroupElement}.
This form lends itself to the matrix representation
for the co-moving reference frame transformation.
The co-moving reference frame is the reference
frame which makes {\rpo}s periodic by applying
a time-dependent spatial translation to every
point of the \twot. Using
\refeq{e-SOnGroupElement} the matrix representation
of the co-moving frame transformation is as follows
\beq \label{e-comovingRotation}
\LieEl(\frac{\sigma \tn}{\period{}}) \equiv
\begin{bmatrix}
\tilde{\LieEl}(\frac{\sigma t_1}{\period{}}) & 0 & \cdots & 0 \\
0 & \tilde{\LieEl}(\frac{\sigma t_2}{\period{}}) & \cdots & 0 \\
\vdots & \vdots & \ddots & \vdots \\
 0 & 0 & 0 & \tilde{\LieEl}(\frac{\sigma t_{\scalebox{.4}{$N$}}}{\period{}})
\end{bmatrix}
\,.
\eeq
Transformations of the type \refeq{e-comovingRotation}
will be used in our ansatz for doubly periodic solutions of
the \KSe\ which are relatively periodic.



\subsection{OLD: Symmetries of \KSe}
\label{sec:KSeSymm}
% KSe.tex     copied from siminos/rpo_ks/current/
% TEMPORARY, ELIMINATE EVENTUALLY  \label{s-KS}

%MOVED TO KSsymm.tex

%The \KSe\ is Galilean invariant: if $u(\conf,\zeit)$ is a solution,
%then $u(x -ct,t) -c $, with $c$ an arbitrary constant
%speed, is also a solution. Without loss of generality, in our
%calculations we shall set the mean velocity of the front to zero,
%\beq \int dx \, u = 0 \,. \ee{GalInv}
%As $\dot{a_0}=0$ in
%\refeq{SCD07:expan}, $a_0$ is a conserved quantity
%fixed to $a_0=0$ by the condition \refeq{GalInv}.


$G$, the group of actions $ g \in G $ on a
\statesp\ (reflections, translations, \etc) is a symmetry of the KS
flow \refeq{e-ks} if $g\,u_t = F(g\,u)$.
The \KSe\ is time translationally invariant, and space translationally invariant
on a periodic domain under
the 1-parameter group of
$\On{2}: \{\Shift_{\shift/\speriod{}},\Refl \}$.
If $u(\conf,\zeit)$ is a solution, then
$\Shift_{\shift/\speriod{}}\, u(\conf,\zeit) = u(x+\shift,t)$
is an equivalent solution for any shift
$-\speriod{}/2 < \shift \leq \speriod{}/2$,
as is the
reflection (`parity' or `inversion')
\beq
    \Refl \, u(x) = -u(-x)
\,.
\ee{SCD07:KSparity}
The translation operator action on the Fourier coefficients \refeq{eq:ksexp},
represented here by a complex valued vector
$a = \{a_k\in\mathbb{C}\,|\,k = 1, 2, \ldots\}$, is given by
\beq
  \Shift_{\shift/\speriod{}}\, a = \mathbf{g}(\shift) \, a \,,
  \label{eq:shiftFour}
\eeq
where $\mathbf{g}(\shift) = \mbox{diag}( e^{i q_k\, \shift} )$ is a complex
valued diagonal matrix, which amounts to the $k$-th mode complex plane
rotation by an angle $k\, \shift /\tildeL$.  The reflection acts on
the Fourier coefficients by complex conjugation,
\beq
  \Refl \, a = -a^\ast
\,.
\ee{FModInvSymm}
Reflection generates the dihedral subgroup $\Dn{1} = \{1, \Refl\}$
of $\On{2}$.  Let $\bbU$ be the space of
real-valued velocity fields periodic and square integrable
on the interval $\Omega = [-\speriod{}/2,\speriod{}/2]$,
\begin{align}
 \bbU  &= \{u \in \speriod{}^2(\Omega) \; | \; u(x) = u(x+\speriod{})\}  \,.
\end{align}
A continuous symmetry maps each state $u \in \bbU$
to a manifold of functions with identical dynamic behavior.
Relation $\Refl^2 = 1$ induces linear decomposition
$u(x) = u^+(x)+ u^-(x)$,
$u^\pm(x)= P^\pm u(x) \in  \bbU^\pm$,
into irreducible subspaces
$
\bbU = \bbU^+
       \oplus \bbU^-
$, where
\beq
    P^+=(1+\Refl)/2
    \,,\qquad
    P^-=(1-\Refl)/2
\,,
\ee{SCD07:P1P2proj}
are the antisymmetric/symmetric projection operators.
Applying $P^+,\,P^-$ on the \KSe\ \refeq{e-ks} we have\rf{KNSks90}
\bea
 u_t^+ &=& - (u^+u^+_x + u^-u^-_x )
                - u^+_{xx} - u^+_{xxxx}
    \continue
 u_t^- &=& - (u^+u^-_x + u^-u^+_x )
                - u^-_{xx} - u^-_{xxxx}
\,.
\label{SCD07:KSD1}
\eea
If $u^- = 0$, \KSf\ is confined to
the antisymmetric $\bbU^+$ subspace,
\beq
 u_t^+ = - u^+u^+_x
                - u^+_{xx} - u^+_{xxxx}
\,,
\label{SCD07:KSU+}
\eeq
but otherwise the nonlinear terms in \refeq{SCD07:KSD1}
mix the two subspaces.

Any rational shift $ \Shift_{1/m}u(x)=u(x+\speriod{}/m)$ generates a discrete
cyclic subgroup $\Cn{m}$ of $\On{2}$, also a symmetry of \KSe.
Reflection together with $\Cn{m}$ generates another
symmetry of \KSe, the dihedral subgroup $\Dn{m}$ of $\On{2}$.
The only non-zero Fourier components of a solution invariant
under $\Cn{m}$ are $a_{jm} \neq 0$, $j =1,2,\cdots$, while for a
solution invariant under $\Dn{m}$ we also have the condition
$\Re a_j=0$ for all $j$.
$\Dn{m}$ reduces the dimensionality of \statesp\ and aids computation of
\eqva\ and \po s within it. For example, the 1/2-cell translations \beq
    \Shift_{1/2}\, u(x)=u(x+\speriod{}/2)
\ee{KSshift}
and reflections generate $\On{2}$
subgroup $\Dn{2} = \{1, \Refl,\Shift,\Shift\Refl\}$,
which
reduces the \statesp\ into four irreducible subspaces
(for brevity, here $\Shift = \Shift_{1/2}$):
\begin{align}
 & \qquad\qquad\qquad\qquad\qquad
              ~~~ \Shift ~~ \Refl  ~\;  \Shift\Refl
    \nnu\\
P^{(1)} &= \frac{1}{4} (1 + \Shift + \Refl + \Shift\Refl)
           ~~~~  S  ~~  S   ~~   S
    \nnu\\
P^{(2)} &= \frac{1}{4} (1 + \Shift - \Refl - \Shift\Refl)
            ~~~~  S  ~~  A   ~~   A
    \nnu\\
P^{(3)} &= \frac{1}{4} (1 - \Shift + \Refl - \Shift\Refl)
           ~~~~  A  ~~  S   ~~   A
     \label{ek_defn}\\
P^{(4)} &= \frac{1}{4} (1 - \Shift - \Refl + \Shift\Refl)
          ~~~~  A  ~~  A   ~~   S
\,.
    \nnu
\end{align}
$P^{(j)}$ is the projection operator onto
$u^{(j)}$ irreducible subspace, and the last 3 columns
refer to the symmetry (or antisymmetry) of
$u^{(j)}$ functions under reflection and
1/2-cell shift.
By the same argument that identified \refeq{SCD07:KSU+},
the \KSf\
stays within the
 $\bbU^S =  \bbU^{(1)}+ \bbU^{(2)}$
irreducible invariant $\Dn{1}$ subspace  of
$u$ profiles symmetric under 1/2-cell shifts.

While in general the bilinear term $(u^2)_x$  mixes the
irreducible subspaces of $\Dn{n}$, for $\Dn{2}$ there are
four subspaces invariant under the flow\rf{KNSks90}:
\begin{itemize} %{romannum}
 \item
    $\{0\}$:~~~~~~ the $u(x)=0$ {\eqv}
 \item
    $\bbU^+ = \bbU^{(1)}+ \bbU^{(3)} $:\\
    the reflection $\Dn{1}$ irreducible space of antisymmetric $u(x)$
 \item
    $\bbU^S =  \bbU^{(1)}+ \bbU^{(2)}$:\\
    the shift $\Dn{1}$ irreducible space of $\speriod{}/2$ shift symmetric  $u(x)$
 \item
    $\bbU^{(1)}$:~~~~~\\
    the $\Dn{2}$ irreducible  space of $u(x)$ invariant under
    $x\mapsto \speriod{}/2-x,\ u\mapsto -u$.
\end{itemize} %{romannum}
With the continuous
translational symmetry eliminated within each subspace, there are no
\reqva\ and \rpo s, and one
can focus on the \eqva\ and \po s only, as was done
for $\bbU^+$ in \refrefs{Christiansen97,LanThesis,lanCvit07}.
In the Fourier
representation, the
$u \in \bbU^+$
antisymmetry amounts to having purely imaginary
coefficients, since $a_{-k}= a^\ast_k = -a_k$.
The 1/2 cell-size shift $\Shift_{1/2}$
generated 2-element discrete subgroup
$\{1,\Shift_{1/2}\}$ is
of particular interest
because in the $\bbU^+$ subspace the translational invariance of the full system reduces to
invariance under discrete translation \refeq{KSshift} by half a
spatial period $\speriod{}/2$.

Each of the above dynamically invariant subspaces is unstable
under small perturbations, and generic solutions of \KSe\ belong to
the full space.
Nevertheless, since  all \eqva\ of the KS flow studied in \refref{SCD07}
lie in the $\bbU^+$ subspace, $\bbU^+$  plays important role for the global
geometry of the flow.
However, linear stability of these \eqva\ has
eigenvectors both in and outside of $\bbU^+$, and needs to be
computed in the full \statesp.


\section*{Predrag's notes - temporary section}
\label{sect:KSsymmPC}
% siminos/spatiotemp/chapter/KSsymmPC.tex
% $Author: predrag $ $Date: 2019-04-18 01:01:46 -0400 (Thu, 18 Apr 2019) $

%\section*{Predrag's notes - temporary section}
%\label{sect:KSsymmPC}
% PC    2017-06-20, 2018-05-03

    \PCedit{
Symmetries of \pCf\ are discussed in
Gibson, Halcrow and Cvitanovi{\'c}\rf{HGC08}
{\em Equilibrium and traveling-wave solutions of plane {Couette} flow}.
Here I adopt clips from the repository \texttt{halcrow/n00bs/n00bs.tex} to
the {\spt} \KS, and will use this text as a temporary staging ground
before editing Matt's  \emph{chap:disc\_symm} text. Not very important, but
my life is a bit easier if I harmonize the notation with ChaosBook.org.
    }


We denote the spatial reflection through the origin by $\sigma$.
The $\sigma$ symmetry generates a discrete dihedral group
$\Dn{1,x} = \{e,\sigma\}$ of order 2, where
\beq
\sigma\, u(x,\zeit) = -u(-x,\zeit)
\label{sigma}
\,.
\eeq
With periodic boundary conditions, the spatial and time translation
symmetries become the $\SOn{2}_x \times \SOn{2}_\zeit$ continuous two-parameter
group of commuting spacetime translations
\begin{align}
\tau(\shift_x)_x\tau(\shift_\zeit)_\zeit u(x,\zeit)
    &= u(x+\shift_x, \zeit+ \shift_\zeit)
\,.
\label{translation}
\end{align}
The \KS\ equations are thus equivariant under the group
$\Group = \On{2}_x \times \On{2}_\zeit = \Dn{1,x} \ltimes \SOn{2}_{x} \times \SOn{2}_\zeit$,
where $\ltimes$ stands for a semi-direct product,
$x$  subscripts indicate spatial translations
and reflections in $x$, and $\zeit$ subscripts indicate time translations
in $\zeit$.

The solutions of an equivariant system can satisfy all of
the system's symmetries, a proper subgroup of them, or
have no symmetry at all. For a given solution $\bu$, the
subgroup that contains all symmetries that fix $\bu$ (that satisfy
$s \bu = \bu$) is called the isotropy (or stabilizer) subgroup of
$\bu$\rf{hoyll06,MarRat99, golubitsky2002sp, GL-Gil07b}. For example, a typical
turbulent trajectory $\bu(\bx,t)$ has no symmetry beyond the identity,
so its isotropy group is $\{e\}$. At the other extreme is the laminar
{\eqv}, whose isotropy group is the full symmetry
group $\Group$.

Consider $S = \{e, \trHalf{x}, \trHalf{\zeit}, \trHalf{x\zeit}\}$, where
\begin{align}
\trHalf{x} \, u(x,\zeit) &=  u(x+\speriod{}/2,\zeit) \nnu \\
\trHalf{\zeit} \, u(x,\zeit) &=  u(x,\zeit+\period{}/2) \label{shiftRot}\\
\trHalf{x\zeit} \, u(x,\zeit) &=  u(x+\speriod{}/2,\zeit+\period{}/2) \nnu
\,.
\end{align}
\PCedit{2018-05-01 TO BE CONTINUED\\}
Four isotropy subgroups of order 4 are generated by picking
$\sigma_{x\zeit}$ as the first generator, and $\sigma_{\zeit},\, \sigma_{\zeit}
\trHalf{x},\, \sigma_{\zeit} \trHalf{\zeit},\,$ or $\sigma_{\zeit} \trHalf{x\zeit}$
as the second generator (\emph{R} for reflect-rotate):
\begin{align}
 R~~  &=  \{e, \sigma, \sigma_\zeit, \sigma_{x\zeit}\}
      ~~~~~~~\; = \{e,\sigma_{x\zeit}\} \times \{e,\sigma_{\zeit}\} \nnu\\
 R_x~ &=  \{e,\sigma \trHalf{x}, \sigma_\zeit \trHalf{x}, \sigma_{x\zeit}\}
      ~~ = \{e,\sigma_{x\zeit}\} \times \{e,\sigma\trHalf{x}\}
        \label{subg4RR} \\
 R_\zeit~ &=  \{e, \sigma \trHalf{\zeit}, \sigma_\zeit \trHalf{\zeit}, \sigma_{x\zeit}\}
      ~~\; = \{e,\sigma_{x\zeit}\} \times \{e,\sigma_{\zeit}\trHalf{\zeit}\}
        \nnu\\
 R_{x\zeit} &= \{e, \sigma \trHalf{x\zeit}, \sigma_\zeit \trHalf{x\zeit}, \sigma_{x\zeit}\}
        = \{e,\sigma_{x\zeit}\} \times \{e,\sigma_{\zeit}\trHalf{x\zeit}\}
        \simeq S \,. \nnu
\end{align}
These are the only isotropy groups of order 4 containing $\sigma_{x\zeit}$
and no isolated translation elements. Together with $\{e,\sigma_{x\zeit}\}$,
these 5 isotropy subgroups represent the 5 conjugacy classes in
which expect to find {\eqva}.

The $R_{x\zeit}$ isotropy subgroup is particularly important, as many
{\eqva} belong to this conjugacy class.


\section{Fourier transform normalization factors}
\label{sect:FTnormal}
% siminos/spatiotemp/chapter/FTnormal.tex
% $Author: predrag $ $Date: 2019-05-09 16:51:31 -0400 (Thu, 09 May 2019) $

% called by
%           siminos/spatiotemp/chapter/spatiotemp.tex
%           siminos/tiles/GuBuCv17.tex

%\section{Appendix: Fourier transforms and normalization factors}
%\label{sect:FTnormal}
% Predrag                                           26 May 2016


\MNG{2016-06-17}{Burak Budanur wrote this.}
%
We have time-periodic solutions, namely two repeats of \ppo s,
however, in my intuition (which might be wrong) we don't
necessarily need to start simulations from those. Since the dynamics is
chaotic, after a finite time, say 20 Lyapunov times
($e^{-20} \sim O(10^{-9})$), correlations between initial condition and
final point will be so low that imposing periodicity in time will not
effect the outcome.

With this in mind, let's say that you solved the \KSe\ in spatially
periodic domain, with an initial condition on the strange attractor
(no transients), and obtained $u(\conf, \zeit)$, for $\conf \in [0, \speriod{})$
and $\zeit \in [0, T)$. You can find the spatial derivatives by
inverse Fourier transformations:
\bea
    u_{\conf}( \conf, \zeit) &=&
                \mathcal{F}^{-1} \left\{i q_k u_k \right\} \,, \quad
    u_{\conf \conf}( \conf, \zeit) =
                \mathcal{F}^{-1} \left\{(i q_k)^2 u_k \right\} \,, \quad
    \continue
    u_{\conf \conf \conf}( \conf, \zeit) &=&
                 \mathcal{F}^{-1} \left\{(i q_k)^3 u_k \right\} \,, \quad
    u_{\conf \conf \conf \conf}( \conf, \zeit) =
                 \mathcal{F}^{-1} \left\{(i q_k)^4 u_k \right\} \, .
    \label{e-Spectralderiv}
\eea
In fact, you must compute spatial derivatives as above (not by
approximating with finite-differences) because otherwise they will not
be as accurate numerically. Side note: depending on the implementation
you're using, Fourier transforms would need you to add a normalization
factor, usually a division by $N$ (number of modes).

Now that you have $u_{\conf}( \conf, \zeit)$ and its space derivatives,
you should take their value at $\conf = 0$ for $\zeit \in [0, T)$ as
your initial condition and input it to the space-integrator. Then you
can compare the outcome with the one you already have from
time-integration.
%}


\MNG{2016-07-13}{Burak Budanur wrote this section.}
In this section we go through the derivation of \refeq{e-FksX}
and state the correct normalizations for Fourier transforms.

Let us start from the following definition of the Fourier expansion of the
time-periodic function $u(t) = u(t + \period{})$:
\beq
    u(\zeit) = \sum_{k = -\infty}^{\infty}
    \Fu_k e^{i \omega_k \zeit} \, , \quad \mbox{where }
    \omega_k = 2 \pi k / \period{}
\,.
\label{e-Fseries}
\eeq

In order to find Fourier coefficients $\Fu_k$, we multiply the above equation
from the left by
$\frac{1}{\period{}} \int_0^{\period{}} d\zeit\ e^{-i \omega_m t}$,
on the RHS we get:
\beq
    \sum_{k = -\infty}^{\infty} \Fu_k \frac{1}{\period{}}
    \int_0^{\period{}} d\zeit\ e^{i (\omega_k - \omega_m) \zeit}
    =
    \sum_{k = -\infty}^{\infty} \Fu_k \frac{1}{\period{}}
    \int_0^{\period{}} d\zeit\ e^{i 2 \pi (k - m) \zeit / \period{}}
     \, .
\eeq
If $k \neq m$, then the integral above is integral of a periodic function
over one full period, hence $0$. If $k = m$, then it is the integral of $1$
from $0$ to $\period{}$, and we can replace the integral by $\period{} \delta_{km}$, which
picks out $\Fu_m$ from the sum. Hence we obtain the forward Fourier transform
of $u(\zeit)$ as
\beq
    \Fu_k = \frac{1}{T} \int_0^{\period{}} d\zeit\ u(\zeit)
            e^{- i \omega_k \zeit}
\eeq
We can approximate the above transformation by replacing the integral by a
Riemann sum $\int_0^T dt \rightarrow \sum_{n=0}^{N-1}\Delta t$, $\Delta t ={T}/{N} $,
hence we
obtain the discrete Fourier transform as
\bea
    \Fu_k &=& \frac{1}{N} \sum_{n=0}^{N-1} u(t_n) e^{-i \omega_k t_n} \,,
    \mbox{where } t_n = n T / N \continue
          &=& \frac{1}{N} \sum_{n=0}^{N-1} u(t_n) e^{-i 2 \pi n k / N}
          \, , \continue
          &=& \frac{1}{N} \mathcal{F} \{ u (t_n) \} \, ,
\label{e-FdscrApprx}
\eea
where $\mathcal{F} \{ . \}$ denotes the Fourier transformation in Matlab's
normalization convention. Consequently, if we take $2N+1$ terms from the
series \refeq{e-Fseries}, we obtain the inverse discrete Fourier transform as
\bea
    u(\zeit_n) &=& \sum_{k = - N/2}^{N/2} \Fu_k e^{i \omega_k \zeit_n}
                   \,, \quad
               \,=\, \sum_{k = - N/2}^{N/2} \Fu_k e^{i 2 \pi k n / N}
                   \,, \continue
               &=& N \mathcal{F}^{-1} \{\Fu_k \} \, ,
\eea
where $\mathcal{F}^{-1} \{ . \}$  is the inverse Fourier transform in the
Matlab's convention.

In Matlab it is probably is computationally preferable to carry out the
convolution in the fourth equation of \refeq{e-FksX} in time-domain as
\beq
    \sum_{m = -\infty}^{\infty} \Fu^{(0)}_{\ell - m} \Fu^{(1)}_m
        = \mathcal{F} \left\{ \mathcal{F}^{-1} \left\{\Fu^{(0)} \right\}
                              \mathcal{F}^{-1} \left\{\Fu^{(1)} \right\}
                      \right\} \, ,
\eeq
where  $\mathcal{F}$ and $\mathcal{F}^{-1}$ denote the Fourier transform
and its inverse, respectively. One should experiment with time-domain
sizes and truncation of the Fourier expansion.

One can insert the definition \refeq{e-Fseries} into \refeq{e-ksX} and then
multiply from left by the integral
$\frac{1}{\period{}} \int_0^{\period{}} d\zeit\ e^{-i \omega_m t}$
in order to confirm that the equation \refeq{e-FksX} is correct. But in order
to compute the nonlinear term pseudospectrally, we take the
Fourier transform of $u^{(0)} u^{(1)}$, that is
\bea
    \frac{1}{\period{}} \int_0^{\period{}} d\zeit\ e^{-i \omega_m t}
    u^{(0)} (t) u^{(1)} (t)
    &\approx& \frac{1}{N} \sum_{n = 0}^{N-1} u^{(0)}(t_n) u^{(1)}(t_n)
                                             e^{-i \omega_m t_n}
    \, , \continue
    &=& \frac{1}{N} \mathcal{F} \{ u^{(0)} u^{(1)} \} \, .
\eea


\section{Selection rules for Fourier coefficients}
\label{sect:selection}

\subsection{Selection rules for \KS}
\label{sect:selectionKS}
% siminos/gudorf/thesis/chapter/selectionKS.tex
% $Author: predrag $ $Date: 2020-05-25 15:18:45 -0400 (Mon, 25 May 2020) $

%\subsection{Selection rules for \KS}

\subsection{Selection rules for real-valued Fourier coefficients}
%\label{sect:selectionKS}

%Possible figures
%Pictographic display of selection rules/constraints




Although the \spt\ \KSe\ is easier to write in terms of a complex
Fourier-Fourier basis, the symmetry invariant subspaces generated
by symmetry constraints is easier to describe in terms of \rv\ \Fcs.
The \rv\
{\spt} Fourier expansion can be written

This is the expansion for a general {\spt} solution. For each discrete
symmetry of the \spt\ \KSe\ there is a unique set of constraints or
``selection rules'' for the \spt\ \Fcs. These selection rules constitute
\textit{symmetry invariant subspaces} of solutions of the \spt\ \KSe. In
this section we commit to the description of the selection rules of the
\Fcs. For more discussion on the symmetries themselves we refer the
reader to \refsect{sect:KSsymm}.

The two discrete symmetries we will describe are spatial reflection
symmetry and \spt\ shift-reflection symmetry. The shift-reflection
symmetry is a special case of the broader symmetry group $\Dn{n}\times
\Cn{n}$ ($n=2$). Due to the uncommon appearance of solutions with $n>2$
and the relatively easy generalization of the $n=2$ shift-reflection
case, we shall only consider the $n=2$ symmetry group.

The general procedure for producing these selection rules is not very
complicated. Let $R$ represent an arbitrary symmetry operation. If a
solution is invariant under $R$ then it satisfies the \textit{invariance
condition} $Ru = u$ or equivalently $(R-1)u=0$. Substitution of the
expansion \refeq{e-RealFourier} produces a set of constraints that can
only be satisfied when a subset of \rv\ \Fcs\ are individually equal to
zero. To begin we start with spatial reflection symmetry, as it is almost
trivial. Solutions invariant under spatial reflection only admit
spatially antisymmetric basis functions. Therefore, the selection rules
for spatial reflection symmetry are
\beq \label{e-ReflRules}
\akj,\bkj = 0 \mbox{  for all  } k,j
\,.
\eeq
\Spt\ shift-reflection is the composition of two symmetry operations:
spatial reflection $\Refl_x$ and time translation by $\tau_{\period{}/2}$.
The action of this symmetry is as follows $\Refl_x \tau_{\period{}/2} u(\conf, \zeit) = -u(-\conf, \zeit + \period{}/2)$.
When directly applied to the \rv\ Fourier expansion \refeq{e-RealFourier}
and by virtue of trigonometric identities and the parity of $\sin$ and $\cos$ we have
\beq
\begin{split}
\Refl_x \tau_{\period{}/2}\,u(\xm, \tn) &= -\sum_{k,j}
                                \cos(\wavek (-\xm)) (\akj\cos(\freqj (\tn + \period{}/2)) + \bkj \sin(\freqj (\tn + \period{}/2))) \continue
                                &+ \sin(\wavek(-\xm)) (\ckj\cos(\freqj (\tn + \period{}/2)) + \dkj\sin(\freqj (\tn + \period{}/2))) \continue
                            &= \sum_{k,j} -\cos(\wavek \xm) (\akj \cos(\freqj \tn)\cos(\pi j) + \bkj \sin(\freqj \tn)\cos(\pi j)))\continue
                                &+ \sin(\wavek \xm) (\ckj\cos(\freqj \tn)\cos(\pi j) + \dkj \sin(\freqj \tn)\cos(\pi j)) \continue
                            &= \sum_{k,j} (-1)^{j+1} \cos(\wavek \xm) (\akj\cos(\freqj \tn) + \bkj\sin(\freqj \tn))\continue
                                &+ (-1)^{j} \sin(\wavek \xm) (\ckj \cos(\freqj \tn) + \dkj \sin(\freqj \tn)) \, ,
\end{split}
\ee{e-ShiftReflectBasis}
By combining this with the invariance condition
$(\Refl_x \tau_{\period{}/2} - 1)u = 0$,
we find that the selection rules for \spt\ shift-reflection are as follows
\bea \label{e-ShiftReflRules}
\akj,\bkj &=& 0 \, \mbox{ for }\, j \, \mbox{ even } \continue
\ckj,\dkj &=& 0 \, \mbox{ for }\, j \, \mbox{ odd }
\,.
\eea


\subsection{Selection rules for Fourier coefficients of Navier-Stokes \ppo s}
\MNGedit{
This section makes the assumption that one is not in the flow invariant
subspace defined by Golubitsky and Stewart\rf{golubitsky2002sp} and
Gibson, Halcrow and Cvitanovi{\'c}\rf{GHCW07}, and also assumes that
outside of this flow invariant subspace that there are \ppo s. Then one
can calculate the number of active variables much like
%\refsect
{sect:selection\_KS} (?) if one took real valued transforms.
    }

The main results which I'll describe is for $s_1$ and $s_3$ "\ppo
solutions" if they exist, which I couldn't confirm. Think of this as an
algebra check on the previous work if its worthless or not interesting.

For $s_1$ we have $[u,v,w](x,y,z,t) = [u,v,-w](x+\frac{L_x}{2},y,-z,t+\frac{T}{2})$.
The functions $u,v,w$ have generic expansions (i.e. there is a sum over index $m$ that denotes unit directions $\hat{x}_m$ that
I'm not including) given by the following.
\bea
\mathbf{u}(\mathbf{x},t) &=& \sum_{jkn\ell} T_{\ell}(y)[\cos(q_j x)\cos(q_k z)(a_{jkn\ell}\cos(w_n t)+b_{jkn\ell}\sin(w_n t))\continue
                        &+& \cos(q_j x)\sin(q_k z)(c_{jkn\ell}\cos(w_n t)+d_{jkn\ell}\sin(w_n t))\continue
                        &+& \sin(q_j x)\cos(q_k z)(e_{jkn\ell}\cos(w_n t)+f_{jkn\ell}\sin(w_n t))\continue
                        &+& \sin(q_j x)\sin(q_k z)(g_{jkn\ell}\cos(w_nc t)+h_{jkn\ell}\sin(w_n t))]
\eea
applying symmetry operation $\tau s_1$ results in (with trigonometric identities and parity of functions)
\bea
u(\mathbf{x},t) &=& \sum_{jkn\ell} T_{\ell}(y)[(-1)^{j+n}\cos(q_j x)\cos(q_k z)(a_{jkn\ell}\cos(w_n t)+b_{jkn\ell}\sin(w_n t))\continue
                        &+& (-1)^{j+n+1}\cos(q_j x)\sin(q_k z)(c_{jkn\ell}\cos(w_n t)+d_{jkn\ell}\sin(w_n t))\continue
                        &+& (-1)^{j+n}\sin(q_j x)\cos(q_k z)(e_{jkn\ell}\cos(w_n t)+f_{jkn\ell}\sin(w_n t))\continue
                        &+& (-1)^{j+n+1}\sin(q_j x)\sin(q_k z)(g_{jkn\ell}\cos(w_n t)+h_{jkn\ell}\sin(w_n t))]
\eea
which results in the selection rules that $j+n$ must be even for the terms $a_{jkn\ell},b_{jkn\ell},e_{jkn\ell},f_{jkn\ell}$ and
$j+n$ must be odd for $c_{jkn\ell},d_{jkn\ell},g_{jkn\ell},h_{jkn\ell}$ for terms in the sums to be non-zero. These selection rules
also apply to the $v$ component of the velocity field, but they switch for the $w$ component due to the extra $-1$ that results from
$s_1$ symmetry operation.

For $j+n$ to be even, they either both half to be even, or both have to be odd, which reduces the number of terms in the summation
by a factor of $4$. Likewise, $j+n$ to be odd, the indices $j,n$ need to either be odd,even or even,odd pairs; again reducing the number
of terms in the summation by a factor of four. Therefore by imposing this type of symmetry and having an spatiotemporal discretization
only increases the dimensionality of the problem by a factor of $N_t / 4$, making it more manageable memory wise.

For solutions with invariant under $\tau s_3$ we have
\[
[u,v,w](x,y,z,t) = [-u,-v,-w](-x,-y,-z+\frac{L_z}{2},t+\frac{T}{2})
\,,
\]
using the parity of Chebyshev polynomials (because this transformation includes
changes to y), trigonometric identities, and parity of sine and cosine functions, we get
\bea
u(\mathbf{x},t) &=& \sum_{jkn\ell} T_{\ell}(y)[(-1)^{l+k+n+1}\cos(q_j x)\cos(q_k z)(a_{jkn\ell}\cos(w_n t)+b_{jkn\ell}\sin(w_n t))\continue
                        &+& (-1)^{l+k+n}\cos(q_j x)\sin(q_k z)(c_{jkn\ell}\cos(w_n t)+d_{jkn\ell}\sin(w_n t))\continue
                        &+& (-1)^{l+k+n}\sin(q_j x)\cos(q_k z)(e_{jkn\ell}\cos(w_n t)+f_{jkn\ell}\sin(w_n t))\continue
                        &+& (-1)^{l+k+n+1}\sin(q_j x)\sin(q_k z)(g_{jkn\ell}\cos(w_n t)+h_{jkn\ell}\sin(w_n t))]
\,,
\eea
where the selection rules are identical for each component of the velocity field because the symmetry transformation changes the sign
on all of the components. Because the result has the same requirement but over three indices I believe that this just reduces the number
of variables by a factor of $8$, so including a time dimension only increases the dimensionality by a factor of $8$.

\subsection{Selection Rules for \Cn{n} shift-reflection symmetries in Kolmogorov flow}
I've been working towards two dimensional Kolmogorov flow code and I realized that it
was imperative to figure out the symmetry invariant subspace related for shift reflection where
the shift is not a half domain but rather a cyclic shift of order $n$ as determined by the forcing
profile in the doubly periodic domain of numerical simulations.

Because the algebra get's exceedingly long I'll present the final result for the selection rules
as the equivalent problem of finding the kernel of a matrix operator. The idea is to work
completely with the vorticity field $\omega(x,y,t)$, which has symmetries of n-cell shift and
spatial reflection, as well as rotation by $\pi$. Specifically shift reflection on a spatial domain
$x \in [0,2\pi]$ and $y \in [0,2\pi]$, with forcing profile of $n$ cells is given by,

\beq
\omega (x,y,t) \rightarrow -\omega (-x,y+\frac{\pi}{n},t)
\eeq

and the rotation is given by,

\beq
\omega (x,y,t) \rightarrow \omega (-x,-y,t) \,.
\eeq

By using applying the invariance condition only for shift-reflection invariant (three) tori solutions, we start
with the expansion for the scalar field in terms of the real-valued Fourier basis functions.

\bea
\omega(x,y,t) &=& \sum_{jkn} [\cos(q_j x)\cos(q_k y)(a_{jkn}\cos(w_n t)+b_{jkn}\sin(w_n t))\continue
                        &+& \cos(q_j x)\sin(q_k y)(d_{jkn}\cos(w_n t)+f_{jkn}\sin(w_n t))\continue
                        &+& \sin(q_j x)\cos(q_k y)(g_{jkn}\cos(w_n t)+h_{jkn}\sin(w_n t))\continue
                        &+& \sin(q_j x)\sin(q_k y)(m_{jkn}\cos(w_n t)+p_{jkn}\sin(w_n t))]
\eea
If we assume there are preperiodic solutions under ($\ell$)-cell shift
reflection after a prime period $\period{p}$, the general selection rules can be
written as constraint conditions
\bea
a_{jkn} &=&  (-1)^{n+1}[a_{jkn}\cos(\frac{q_k \pi}{\ell})-b_{jkn}\sin(\frac{q_k \pi}{\ell})] \continue
b_{jkn} &=&  (-1)^{n}[a_{jkn}\cos(\frac{q_k \pi}{\ell})-b_{jkn}\sin(\frac{q_k \pi}{\ell})] \continue
d_{jkn} &=&  (-1)^{n}[d_{jkn}\cos(\frac{q_k \pi}{\ell})+f_{jkn}\sin(\frac{q_k \pi}{\ell})] \continue
f_{jkn} &=&  (-1)^{n}[-d_{jkn}\cos(\frac{q_k \pi}{\ell})+f_{jkn}\sin(\frac{q_k \pi}{\ell})] \continue
g_{jkn} &=&  (-1)^{n+1}[g_{jkn}\cos(\frac{q_k \pi}{\ell})+h_{jkn}\sin(\frac{q_k \pi}{\ell})] \continue
h_{jkn} &=&  (-1)^{n}[g_{jkn}\cos(\frac{q_k \pi}{\ell})-h_{jkn}\sin(\frac{q_k \pi}{\ell})] \continue
m_{jkn} &=&  (-1)^{n}[m_{jkn}\cos(\frac{q_k \pi}{\ell})+p_{jkn}\sin(\frac{q_k \pi}{\ell})] \continue
p_{jkn} &=&  (-1)^{n}[-m_{jkn}\cos(\frac{q_k \pi}{\ell})+p_{jkn}\sin(\frac{q_k \pi}{\ell})]
\eea

By doing some algebra one realizes that the coefficients are zero unless
$k_y$ the wavenumber associated with the direction that the forcing
profile varies over is only nonzero for integer multiples of the forcing
wavelength. In other words, if the forcing repeats four times, the only
nonzero $k_y$ are $k_y = 4,8,12, \cdots$. The way that the preperiodic
(time) condition comes into play is just like how it comes into play in
the \KSe\ equation, half of the modes are zero depending on whether the
time index $n$ is even or odd. Specifically, the modes that are zero are,

For $k_y$ being an odd number of multiples of the forcing profile index
$\ell$, we have the following constraints
\bea
d_{jkn},f_{jkn},m_{jkn},p_{jkn} &=& 0 \, \mbox{for}\, n \, \mbox{odd} \continue
a_{jkn},b_{jkn},g_{jkn},h_{jkn} &=& 0 \, \mbox{for}\, n \, \mbox{even}
\eea

For $k_y$ being an even number of multiples of the forcing profile index
$\ell$, we have the following constraints
\bea
d_{jkn},f_{jkn},m_{jkn},p_{jkn} &=& 0 \, \mbox{for}\, n \, \mbox{even} \continue
a_{jkn},b_{jkn},g_{jkn},h_{jkn} &=& 0 \, \mbox{for}\, n \, \mbox{odd}
\eea

Of course for \eqva\ this simplifies due to not having a third continuous
dimension, and we only have four distinct sets of coefficients
(combinations of $\sin$ and $\cos$ in $x,y$). In fact, it simplifies even
more due to not having the extra factor $-1^n$, such that the only
non-zero coefficients
\bea
a_{jk},b_{jk} &=& 0 \, \mbox{for}\, k_y \, \mbox{even multiple of forcing index} \continue
d_{jk},f_{jk} &=& 0 \, \mbox{for}\, k_y \, \mbox{odd multiple of forcing index} \continue
\eea

In summary, for \eqva\ with discrete shift reflection symmetry, the number of non-zero
modes equals $N_x*\frac{N_y}{2*\ell}$ and for preperiodic orbits it totals $N_x*\frac{N_y}{\ell}*\frac{N_t}{2}$.

Note, this is also useful for the \KSe\ if one desires the selection
rules for $\ell$ cyclic shift reflection, as there is no difference
between shift reflection in two continuous spatial dimensions versus one
space and one time; the analogy is perfect with $y$ playing the role of
$t$ in the \KSe.

\newpage
% siminos/spatiotemp/chapter/GuBuCv17blog.tex
% $Author: mgudorf3 $ $Date: 2021-03-01 18:23:28 -0500 (Mon, 01 Mar 2021) $

\section{Tiles' GuBuCv17 clippings and notes}
\label{sect:GuBuCv17blog}

Move good text not used in \refref{GuBuCv17} to this file, for possible
reuse later.

\begin{description}
\item[2016-11-05 Predrag]
A theory of turbulence that has done away with \emph{dynamics}?
We rest our case.

\item[2019-03-19 Predrag] Dropped this:
\\
In what follows
we shall state results of all calculations either in units of the
`dimensionless system size' $\tildeL$, or the system size $\speriod{} = 2 \pi
\tildeL$.

Due to the hyperviscous damping $u_{xxxx}$, long time solutions of \KSe\
are smooth, $a_k$ drop off fast with $k$, and truncations of
\refeq{SCD07:expan} to $16 \leq N \leq 128$ terms yield accurate
solutions for system sizes considered here (see
\refappe{sec:fourierRLD}).

For the case investigated here, the
\statesp\ representation dimension $d \sim 10^2$ is set by
requiring that the exact invariant solutions that we compute
are accurate to $\sim 10^{-5}$.

\end{description}

\subsection{GuBuCv17 to do's}
\label{sect:GuBuCv17ToDo}
Internal discussions of \refref{GuBuCv17} edits.

\begin{description}

\item[2019-03-17 Predrag to Matt]
My main problem in writing this up is that I see nothing in
the blog that formulates the variational methods that you use,
in a mathematically clear and presentable form.
Perhaps there is some text from\\
\texttt{siminos/gudorf/thesisProposal/proposal.tex}\\
that you can use to start writing up variational justification for
your numerical codes, section~\ref{sect:variational} {\em Variational methods}.

\item[2019-03-17 Predrag to Matt]
Please write up {\em tile extraction} and {\em glueing} in the style of a
SIADS article.


\item[2019-03-17 Predrag to Matt]
Should any of
Appendix \ref{sect:FTnormal}~{\em Fourier transform normalization factors}
be incorporated into {\bf GuBuCv17}\rf{GuBuCv17}?

\item[2019-04-10 Matt writing]
To begin \texttt{variational.tex} I included two equivalent formulations
of the variational problem; the first is written in a more concise manner
while the second is written in a more explicit manner. The longer of the
two is commented out. The more explicit description uses dummy variables
(Lagrange multipliers) which replace parameters $(\speriod{},\period{})$
as independent variables.

I'm including explanations of the numerical algorithms but I don't think
I should present them in their style for algorithms, because we didn't
invent them just applied them in a unique way. If desired I think the
easiest way of including them per SIADS style guide is to use the
algorithm package they suggest: \texttt{algpseudocode} and
\texttt{algorithmic} are the package names.

I feel conflicted as to whether to define the gradient matrix using a new
letter or the ``mathematician way". e.g. $A(x)$ or $DG(x)$. Also, I
started using $\mathbf{z}$ to represent {\statesp} vectors. I'm not a fan
of using $z$ but I don't want to confuse people by using $u$,$x$, etc.

I need to get better at writing or stop being OCD over how sentences are written.

\item[2019-04-16 Matt update]
In an effort to make the chapters and \texttt{GuBuCv17.tex} more modular,
I've split apart some of the chapters into smaller, more manageable pieces.
For example, \texttt{variational.tex} was covering too many topics to be
reflected by the file name and \texttt{numerics.tex} predominately covered discrete
lagrangian systems and Noether's theorem. The algorithms (matrix free adjoint descent,
matrix free GMRES and Gauss-Newton) have yet to be discussed in excruciating detail.
This is my fault, in hindsight I've done a poor job with recording what I do and
how I do it. I'm going to get better at this.

For the time being, until it is deemed unnecessary or unintelligent, I am going
to break the chapters into the files \texttt{adjointdescent.tex} and \texttt{iterativemethods.tex}.
I'm going to change the discourse so that instead of requiring the current order,
namely, \texttt{variational.tex}->\texttt{adjointdescent.tex}->\texttt{iterativemethods.tex}
the pieces will be written as to be independent of one another.

In order to get specific, I needed to include the \KSe\ written in the Fourier-Fourier basis; I put this in \texttt{sFb.tex}

\item[2019-04-17 MNG update]
Realized that in order to get specific with the numerical methods I need to
include both an exposition on the {\spt} Fourier modes  as well
as the matrix-free computations. The latter really stresses the improvements
over the finite-difference approximation of the Jacobian that requires
time integration ubiquitous in plane-couette and pipe numerics.
Expanding on \texttt{adjointdescent.tex} and \texttt{iterativemethods.tex}.
Again, the main stratagem is to make the separate \texttt{.tex} files
as independent as possible to avoid ``long distance references''.

\item[2019-04-18 MNG]
Heavy edits to \texttt{tiles.tex}
Added section on preconditioning \texttt{preconditioning.tex}
Formatting edits to \texttt{matrixfree.tex} can be ignored.
\\
Added details in \texttt{iterativemethods.tex} regarding GMRES
and SciPy wrapper for LAPACK solver GELSD

\item[2019-04-23 MNG]
Converting indices to abide by the conventions: physical space
indices $u(x_m,t_n)$, and \spt\ Fourier space indices
$\Fu_{kj}$.
\\{\bf 2018-05-09 PC} can do. Also, remember that
$u(x_m,\zeit_n)$ implies that everywhere the ordering is
$(\speriod{},\period{})$, and not $(\period{},\speriod{})$.

Luca Dieci asked (borderline pleaded) to abide by the
mathematics convention that $n$ is the index for discrete
time. I'm avoiding $\ell$ and $\tau_t$ due to the unnecessary
confusion with domain size $\speriod{}$ and period $\period{}$.
\\{\bf 2018-05-09 PC} Agreed. $\tau_t$ we usually control by macro
$\setminus zeit$, so currently
$\zeit_n$.

\item[2019-04-24 MNG]
Discussion of how I foresee paper(s) playing out in \texttt{blogMNG.tex}
by considering subject matter, narratives, and paper length.
Perhaps unsurprisingly I lean towards structuring a paper similar
to my thesis.

I'm unsure how to approach \spt\ symmetries in a practical manner.
Projection operators which produces symmetry invariant
subspaces are nice and complements the selection rules for different
symmetries nicely. Specifically it provides the reason for why
the selection rules exist and motivates the use of symmetry
constrained Fourier transforms. The only issue I have with this
is that the results of the formal derivation are not really used
beyong that.
I think this is likely a case of ``It-is-trivial-now-that-I-know-it' syndrome.
Perhaps it would be
sufficient to say that the selection rules constitute these subspaces
without the formalism?

\item[2019-04-29 MNG]
Rewrite of \texttt{KSsymm.tex} after double checking the
derivations.
Going to rewrite \texttt{sFb.tex}, I'm paying for the expedient
manner in which is was written; in other words just use a single
Fourier basis as opposed to a real basis and a complex basis, Matt.

\item[2019-04-30 MNG]
Rewrites to describe the \spt\ \KSe\ only in terms of \rv\
\Fcs\ for consistency. The index notation gets a little rough
but the pseudospectral form of the equation is nice enough.

Tried to find the most concise description of how I handle
\rpo s using mean velocity frame (time dependent rotation
transformation).

\item[2019-05-02 MNG]
Is it necessary to recap all of the results in \refsect{sect:KStimeInt}
in this paper? Other than the spatial integration calculation the results
described in \refrefs{DasBuch,SCD07}. I'm unsure how to connect the
{\spt} calculations to results pertaining to the dynamical system
formulation, e.g. temporal stability and energy budget.

Moved \texttt{SpatTempSymbDyb.tex} to after \texttt{tiles.tex} such that
it proceeds from finding tiles to using tiles.

The bulk of each section is complete; perhaps need to add some more
detail to \texttt{glue.tex} and \texttt{tiles.tex} but mostly need to
work on picking, producing, and inserting figures.

Going to list suggestions for figures at the top of each section in
commented text.

\item[2019-05-02 MNG]
Added tile figures: Extraction and converged results in \texttt{tiles.tex}.

Modifying scripts to produce figures of general numerical convergence
(initial condition to final converged \twot), produce figures
demonstrating step-by-step gluing for repeated gluing,
and produce figures for the ``frankenstein'' plots (combining tiles
to produce \twots). Basically just producing more figures.

\item[2019-05-11 PC]
moved Ibragimov to \texttt{gudorf/thesis/thesis.tex} until we find it useful.

\item[2019-05-13 MNG]
\begin{itemize}
%\item Reduced Whitespace on figures
\item Added spatial gluing figures
\item Added description of gluing procedure
\end{itemize}

\item[2019-05-13 PC]
Figures are looking great, and in my talks people seem to ``get'' tile
extraction and gluing, so they are very important. A few notes, before you
produce the next versions:
\begin{itemize}
%  \item
%never include labels like (a) in this example
%into a figure file - that is job for LaTeX and such labels can change
%from time to time, for the same figure.
%  \item
%If at all possible for your scrips: Never create a white border around a
%figure (see the example above),
%
%\fbox{\includegraphics[width=.4\textwidth]{ks_ux_largeL}}
%
%that wastes valuable real estate, always
%clip it tightly up to the first/last non-white pixel of the figure.
  \item
I think you should label all $u$ color bars in multiples of 1, or
or 0.5 if that is really needed, not different units in every plot.
  \item
Once you have improved a given figure,
% like the \texttt{MNG\_hook\_initial.png},
keep the same name rather than renaming it
(they are often shared between different articles, presentations and blogs)
\end{itemize}

\item[2019-07-05 PC] dropped from trawl.tex: ``
In both formulations there is no guarantee of convergence
but it is clearly better to take less time regardless of convergence.

In our formulation, convergence can not be guaranteed either, but the
resources committed to the initial guesses generation are negligible.
''

\bea \label{e-discretedefsOld}
\wavek &=& 2\pi\frac{k}{\speriod{}}\,,\qquad k=1,\cdots,M/2-1 \continue
\freqj &=& 2\pi\frac{j}{\period{}}\,,\qquad j=0,\cdots,N/2-1 \continue
\xm &=& \frac{m}{M}\speriod{}\,,\qquad m=0,\cdots,M-1 \continue
\tn &=& \frac{n}{N} \period{}\,,\qquad n=0,\cdots,N-1\,.
\eea

\item[2019-08-21 MNG]
Moved discussion of recurrence plots and multiple shooting from trawl.tex
to variational.tex

It seemed more coherent to first describe the disadvantages
of the IVP to motivate the variational problem. I'm going
to refer to what I do as ``solving a variational problem'' as
opposed to boundary value problem because it insinuates
(at least to me) that we're solving a Dirichlet BC in 1 + 1 dimensions
problem.

General narrative of variational.tex
\begin{itemize}
\item Exponential instability bad
\item Variational formulation good
\item How to solve variational problem (general description of optimization)
\item Losses from variational formulation (notion of dynamics, stability, bifurcation analysis).
\item How to recoup from these losses (adjoint sensitivity, Lagrangian, Hill's formula)
\end{itemize}

It's currently a hot mess.


\item[2019-09-20 MNG]
Input references to topological defects and motifs in complex networks. Renamed
the ``defect tile'' to the ``merger tile'' but also made the connection
that similar patterns in crystals are referred to as ``edge dislocations''.

Just clean up and rewriting \texttt{tiles.tex} mainly; it's almost in shape.

\item[2018-05-09 PC] Dropped:
The following definitions will be devoid of symmetry considerations
such that the equations represent the general case.

For $\tildeL<1$ the only \eqv\ of the
system is the globally attracting constant solution
$u(\conf,\zeit)=0$, denoted $\EQV{0}$ from now on. With increasing
system size $\speriod{}$ the system undergoes a series of
bifurcations. The resulting \eqva\ and \reqva\ are described
in the classical papers of Kevrekidis, Nicolaenko and
Scovel\rf{KNSks90}, and Greene and Kim\rf{ksgreene88},
among others. The relevant bifurcations up to the
system size investigated here are summarized in
\reffig{fig:SCD07ksBifDiag}: at $\tildeL=22/2\pi = 3.5014\cdots$,
the {\eqva} are the constant solution \EQV{0},
the  \eqv\ \EQV{1} called GLMRT by Greene and
Kim\rf{laquey74,ksgreene88},
the $2$- and $3$-cell states
\EQV{2} and \EQV{3}, and the pairs of \reqva\ \REQV{\pm}{1},
\REQV{\pm}{2}.
All \eqva\ are in the antisymmetric subspace $\bbU^+$, while
\EQV{2} is also invariant under $\Dn{2}$ and \EQV{3} under $\Dn{3}$.

Due to the translational invariance of {\KSe},
they form invariant circles
in the full \statesp.
In the $\bbU^+$ subspace considered here,
they correspond to $2n$ points, each shifted by $\speriod{}/2n$.
For a sufficiently small $\speriod{}$
the number of {\eqva} is small and
concentrated on the low wave-number end of the Fourier spectrum.

    %\PC{2018-05-04} {
dropped this: \Group, the group of actions $ g \in G $
    on a \statesp\ (reflections, translations, \etc) is a spatial symmetry of
    a given system if $g u_\zeit = F(g\,u)$.
%     u_\zeit + u_{\conf \conf} + u_{\conf \conf \conf \conf} + u u_\conf = 0
%    \refeq{e-ks}

An instructive example is offered by the dynamics for
the  $(\speriod{},\period{})=(22,\period{})$  system
that \refref{SCD07} specializes to.
The size of this
small system is $\sim 2.5$ mean wavelengths
($\tildeL/\sqrt{2}= 2.4758\ldots$),
and the competition between states with wavenumbers 2 and 3.

The two zero Lyapunov exponents are due to the time and
space translational symmetries of the \KSe.

For large system size, as the one shown in \reffig{f:ks_largeL}, it is
hard to imagine a scenario under which attractive periodic states (as
shown in \refref{FSTks86}, they do exist) would have significantly large
immediate basins of attraction.

\item[2019-10-17 MNG]: Merged symmetry discussions.
\texttt{KSsymmMNG1} was deleted because seems to be an old discussion predating
the \spt\ symmetry group discussion as it still mentions
equivariance. The focus should only be on invariance under symmetry operations,
as invariance gives rise to the the practical application
of the symemtry discussion which is constraints on the \spt\
\Fcs.
\texttt{KSsymmMNG} was deleted because it is just an older version
of \texttt{KSsymm}.
\texttt{KSsymmPC} uses different notation and says things better than I do
so I'll have to figure out how to merge it in.

\item[2019-10-25 PC]  dropped from \emph{variational.tex}:

Linear stability analysis has been used in bifurcation
analysis of describe the existence and bifurcations of solutions
as well as the geometry of state spaces corresponding to different
flows \refrefs{GHCW07,lanCvit07,W97}.

Commonly time variational integrators preserve symplectic structure

\item[2019-09-05 MNG] Dropped from \emph{variational.tex}:
{multishooting optimization of cost functional
because it doesn't jive with \spt\ methods (based on integration)}

{Adjoint sensitivity and Hill's formula sections
when I figure them out or they seem useful}:

Section on adjoint sensitivity
The \spt\ reformulation of a dynamical problem also
requires a reformulation of its linear stability
analysis.

Nevertheless, we still have the notions of tangent spaces and derivatives
so the natural replacement is the notion of sensitivity. In the context
of finite element (finite difference) representation, instead of computing
a derivative and transporting it around a periodic orbit, it instead
computes the derivative of the temporal average of the quantity with
respect to whichever parameter is desired\rf{Blonigan17,LaShMe18,Wang13}.
Because there is no transport,
one need not worry about the exponential instability present.
Essentially sensitivity is to stability as boundary value problem is
to initial value problem in this context. Because the \spt\ boundary
problem is defined on a compact domain on which the scalar field does
not diverge, dynamical observables are bounded; they do not experience
numerical overflow (underflow) associated with unstable (stable) manifolds.
%%%%%%%% Hill's formula.
\beq
S = \int_{\mathcal{M}} \mathcal{L}(u,v,u_x,v_x,u_t,v_t,u_xx,v_xx) dx dt
\eeq
such that the matrix of second variations, or Hessian, of this action functional
is defined as
\beq
H = \nabla \nabla^{\top} S
\eeq
such that the derivatives are taken with respect to the infinite dimensional
scalar fields $u,v,\dots,$ such that the Hessian matrix is infinite dimensional
prior to discretization of the scalar fields.
The resultant discrete Lagrangian system and subsequent Hessian should be
the Hessian of Hill's formula, I believe. If one is trying to derive
Hamilton's action principle as a result of discretization (\ie, finite differences)
as in \refref{KraMaj15} then one must take care to define \spt\ differentiation operators
in a manner consistent with an action principle. A large amount of
the derivation of the discrete action principle and discrete Noether's theorem
of\rf{KraMaj15} relates to using a finite element discretization in physical
space. I am unsure how these ideas extend to a Fourier basis; I currently
am assuming that as long as the differentiation operators, and hence the derivatives
(jet bundle) is properly defined then everything should work out.
When two total derivatives of the Lagrangian density are taken, one arrives
at the following matrix representation of the Hessian. Keep in mind that
we have ordered the variables in terms of the
order of the corresponding derivatives $(u,v,u_t,v_t,u_x,v_x,u_xx,v_xx)$.
\beq
\begin{bmatrix}
-v_x(t,x)/3 & u_x(t,x)/3 & 0 & -1/2 &v(t,x)/3 & -2u(t,x)/3 & 0 &0 \\
u_x(t,x)/3& 0 & 1/2 & 0 & u/3 & 0& 0& 0\\
0    &1/2 &0 &0 &0 &0 &0 &0 \\
-1/2 &0 &0 &0 &0 &0 &0 &0 \\
v(t,x)/3  &u(t,x)/3 &0 &0& 0& &-1 &0 &0 \\
-2u(t,x)/3 &0 &0 &0 &-1 &0 &0 &0 \\
0    &0  &0 &0 &0 &0 &0 &1 \\
0    &0  &0 &0 &0 &0 &1 &0 \\
\end{bmatrix} \,.
\ee{NotHessiantake2}
This is an infinite dimensional matrix, but upon discretization each block
will represent a diagonal matrix whose diagonal contains the scalar
field values of the corresponding spacetime coordinates. For instance,
$u_x/3 \equiv \frac{1}{3} u_x(x,t) \to \frac{1}{3} u_x(t_n, x_m)$. Because
each of the blocks are diagonal, \ie, $1 \equiv \mathcal{I}^{N*M}$, the
determinant expansion is long but not impossible to decipher. Note
the presence of the adjoint variables $v,v_x$. There is freedom in
the choice of what these variables should be, because they are non-physical.


\item[2020-02-28 MNG]
Reformatted the paper into sections which follow the outline so far:
,\texttt{tileoutline.tex}\\
\texttt{tileintro.tex}\\
\texttt{tilebody.tex}\\ \texttt{tilesummary.tex}\\
\texttt{tilefuture.tex}

\item[2020-05-04 PC] might reuse these somewhere:

Motivated by the presence of continuous symmetries we recast
chaotic nonlinear dynamical systems via a $(D+1)$-dimensional space-time theory.

Space-time translationally recurrent solutions are
invariant $(D+1)$-tori
%(and not the $1$-dimensional periodic orbits of the traditional periodic
%orbit theory).

larger tori can be constructed from the combination of smaller tori.

the entirety of space-time
can be explained via the shadowing by these tori.


This sets the stage for a 2\dmn\ {\symbolic}
of the infinite space-time \KSe\ wherein the fundamental
patterns constitute the symbolic alphabet.


As longer periods periodic
orbits are shadowed by the shorter ones, truncations of the theory to
finite sets of periodic orbits should suffice to predict any observable
of the `turbulent' flow to a finite accuracy.

There is a vast
literature on {\rpo s} since their first appearance, in
Poincar\'e study of the 3-body problem\rf{ChencinerLink,rtb},
where the Lagrange points are the \reqva.  They arise in
dynamics of systems with continuous symmetries, such as
motions of rigid bodies, gravitational $N$-body problems,
molecules, nonlinear waves  and the plane Couette fluid flow\rf{Visw07b}.

\item[2020-05-12 PC] Who's this Gunzberger02 in
 rf\{BorSch11,Gunzberger02,BoyVan04\}?
\item[2020-05-14 MNG] It's a text on numerical optimization.
\item[2020-05-19 PC]
Cannot find any such Gunzberger textbook, removed the reference,
sticking with the traditional Russian BorSch.

\item[2020-02-26 MNG] Let's get this done.
I'm realizing how little was actually written now that I'm done with my thesis and it is kind of laughable.

\item[2020-02-28] Flushing out \texttt{tileintro.tex}. 

\end{description}




\bigskip
Note to Predrag - send this paper to




%%%%%%%%%%%%%%%%%%%%%%%%%%%%%%%%%%%%%%%%%%%%%%%%%%%%%%%%%%%%%%%%%%%%%%%
\printbibliography[heading=subbibintoc,title={References}]
