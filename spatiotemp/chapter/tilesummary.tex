% siminos/spatiotemp/chapter/tilesummary.tex
% $Author: mgudorf3 $ $Date: 2020-05-15 18:24:44 -0400 (Fri, 15 May 2020) $

% called by
%           siminos/spatiotemp/chapter/spatiotemp.tex
%           siminos/tiles/GuBuCv17.tex

%\section{Summary}
%\label{sect:summary}
% Matt                                           28 February 2020
We have detailed a {\spt} formulation of turbulence
which treats all continuous dimensions with translational
invariance equally and
explains solutions as collections of {\fpo}s.
\Preliminary{\subsection{Many benefits}}
\Preliminary{\subsection{Many new applications}}
The new techniques we developed allow us
to extract small {\po}s from larger {\po}s (clipping) and build larger {\po}s
by combining smaller {\po}s (gluing).
\Preliminary{\subsection{Numerical robustness}}
\Preliminary{\subsection{Stress again how hard it was}}
While we hope to eventually apply these ideas to equations with more continuous
dimensions there are still many tough questions yet to be answered.
\Preliminary{\section{Future directions}}
These methods provides a numerical foundation
with which to investigate a 2{\dmn}{\spt} symbolic dynamics.
Specifically, by gluing members of the three continuous families of {\fpo}s
we can begin to probe the grammar of the proposed {\symbolic} by looking
for admissible orbits.
% \Preliminary{\subsection{Better existent numerical methods}} pruned as I do not actually discuss other methods
\Preliminary{\subsection{Rubber tile families}}
The most important question is how to incorporate
continuous families into the proposed  2\dmn\ \spt\ {\symbolic}.
The existence of continuous families makes the determination of the {\symbolic} grammar
particularly difficult. One reason is because our method for
determining the grammar is ultimately an empirical process.
The admissibility of every {\po} is
dependent upon the convergence of the optimization problem, which in turn may depend on
which {\fpo} family members ares used in the construction of the initial guess.
Therefore, we can be mislead by initial guesses which do not converge to the corresponding
{\po} it shadows. In addition to being sensitive to initial guesses, the
failure to converge can also be the fault of insufficient numerical methods.
The most obvious course of action is to improve the
optimization and gluing methods, with respect to their frequency of convergence.
The former of these two improvements is fairly
straightforward; find and implement better numerical methods. This seems
to be the low hanging fruit as we have employed some of the simplest available
algorithms. Improving the gluing method is less straightforward; but there are
also many potential improvements.

\Preliminary{\subsection{Gluing improvements}}
\Preliminary{\subsection{1. Add local Galilean velocity to gluing}}
The first set of gluing improvements we discuss center around the reduction of
the number of false negatives (not converging to a {\po}). We consider
the inclusion of local Galilean velocity towards this end. Historically,
when performing simulations on a single spatially periodic domain,
Galilean invariance has been invoked to constrain the mean value of
the velocity field to zero. This does not mean, however, that the \emph{local}
Galilean velocity be zero. This detail could be included in each gluing such that the
local Galilean velocity of each {\fpo} would be included as a free parameter. By
doing so we can theoretically construct better guesses by increasing the
agreement between {\fpo}s at their boundaries.
% continuous family inclusions
\Preliminary{\subsection{2. Leveraging continuous families}}
The second gluing improvement we offer is the proper usage of the {\fpo} continuous families.
In other words, instead of using three static representatives of the families
we would reference the entire family during the gluing. This could be employed to
minimize differences at the
boundaries as well as in the periods.
In a similar vein, we need to incorporate symmetries into the construction process.
This extends the freedom of choice from picking a member of a family to
picking a member of each families' group orbit. For instance, during the gluing
process one is free to choose whether to use a {\fpo} or its reflection.
In the process of probing the {\symbolic} grammar, numerical convergence is not
the only factor which is required for success. The {\po} found via optimization
must also correspond to the original {\brick} it was targeting. If this is not
upheld, then the result is deemed a false positive; numerical but not symbolic
convergence.

\Preliminary{\subsection{No {\symbolic} as yet}}
Our only method of classifying false positives is visual inspection;
obviously an inefficient method that needs to be improved.
For example, assume that a {\brick}
contains $N$ symbols representative of the {\defect} family. If the {\po}
which the initial guess converges to does not include the signatures of $N$
defects then we claim that it cannot be a manifestation of the original {\brick}.
We have tried a number of methods which
attempt to identify features of {\fpo}s via image detection
or their topological signatures via persistent homology with
no real success. A possible brute force method would be to attempt to clip
and compare subdomains with {\fpo}s. The problem with this
is the incredible number of subdomains which could be clipped; a crude
approximation is on the order of $\mathcal{(NM)^2}$. Choose a corner
of the subdomain ($N$ choices in time, $M$ in space) then identify the
dimensions of the rectangular subdomain ($N-1$ choices in time, $M-1$ in space).
Therefore this brute force method does not seem to be a realistic option.
Each of these improvements introduces a layer of complexity which
quickly compound with one another making gluing a very complex process.

The guiding principle for all of these improvements
is that they minimize the cost function residual \refeq{e-cost} by
mitigating the error at the gluing boundaries and the error due to difference
in periods. We point out that simply reducing the cost function with no consideration
towards the method is not useful. For instance,
setting $\Fu=\mathbf{0}$ reduces the residual to zero but it quite clearly not of any use.
Better metrics to gauge the merit of these additional techniques would perhaps be
the frequency of convergence to the correct {\po} symbolically (true positive rate).

\Preliminary{\subsection{No \po\ theory}}
\Preliminary{\subsection{Physical predictions?}}
\Preliminary{\subsection{Stability}}
\Preliminary{\subsection{Extend to \NS}}

