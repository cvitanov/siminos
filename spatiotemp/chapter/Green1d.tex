% siminos/spatiotemp/chapter/Green1d.tex
% $Author: predrag $ $Date: 2021-08-10 11:56:19 -0400 (Tue, 10 Aug 2021) $

%\section{Green's functions for  $\integers^1$ and $\integers^2$ lattices}
%\label{sect:Green}
\section{Green's  function for 1\dmn\ lattice}
\label{sect:Green1d}
% derived from siminos/cats/Green.tex                2017-09-06

\renewcommand{\Ssym}[1]{{\ensuremath{m_{#1}}}}    % Boris
% \renewcommand{\Ssym}[1]{{\ensuremath{s_{#1}}}}  % ChaosBook

Cat map is a second order difference equation
\beq
\ssp_{t+1}  -  s \, \ssp_{t} + \ssp_{t-1}
    =
-\Ssym{t}
\,,
\ee{eq:CatMapNewton5}
with the unique integer ``winding number'' $\Ssym{t}$  at every time step $t$
ensuring that $\ssp_{t+1}$ lands in the unit interval.
This is a 1\dmn\ discrete {\sPe} of form
\beq
 \D \ssp = \Ssym
\,,
\ee{eq:CatMapStep}
where $\ssp_t$ are {\lattstate}s, and $\Ssym{t}$ are the `sources'. Since
$\D_{tt'}$
% $(\D_\ell)_{tt'}$
is of a tridiagonal form,  its inverse, or its Green's
discrete matrix $\gd$ on infinite lattice satisfies
\beq
 (\D \gd)_{t0}= \delta_{t0}\,, \qquad t\in \integers
\ee{1DGreenFunInf}
with a point source at $t=0$.
By time-translation invariance $\gd_{tt'}=\gd_{t-t',0}$, and by time-reversal
invariance $\gd_{t',t}=\gd_{t,t'}$.
In this simple, tridiagonal case, $\gd$ can be evaluated
explicitely\rf{varcyc,PerViv},
\beq
\gd_{tt'}= \frac{1}{\ExpaEig^{|t'-t|}}\,\frac{1}{\ExpaEig-\ExpaEig^{-1}}
\,,
\ee{GreenFun00a}
where, in the hyperbolic $s>2$ case, the cat map ``stretching'' parameter
$s$ is related to the 1-time step cat map eigenvalues
$\{\ExpaEig,\ExpaEig^{-1}\}$ by
\beq
s
  = \ExpaEig+\ExpaEig^{-1}
  = e^\Lyap+e^{-\Lyap}
  = 2\cosh\Lyap
\,,\quad \Lyap>0
\,.
\ee{catMapEig}
While the ``Laplacian'' matrix $\D$ is sparse, it is non-local (\ie, not
diagonal), and its inverse is the full matrix $\gd$, whose key feature,
however, is the prefactor $\ExpaEig^{-|t'-t|}$ which says that the magnitude of
the matrix elements falls off exponentially with their distance from the
diagonal. For this it is crucial that the $\D$ eigenvalues \refeq{catMapEig}
are hyperbolic. In the elliptic, $-2<s<2$ case, the sinh's and cosh's are
replaced by sines and cosines,
\(
  s= 2\cos\theta= \exp(i\theta)+\exp(-i\theta)
\)\,,
and there is no such decay of the off-diagonal matrix elements.

For a finite-time lattice with $\cl{}$ sites
we can represent $\D$ by a symmetric tridiagonal
$[\cl{}\!\times\!\cl{}]$ Toeplitz  matrix.
The matrix
\beq
\D_\cl{}= \left(\begin{array}{ccccccc}
 s&-1 & 0 & 0 &\dots &0&0 \\
-1 &  s&-1 & 0 &\dots &0&0 \\
0 &-1 &  s&-1 &\dots &0 & 0 \\
\vdots & \vdots &\vdots & \vdots & \ddots &\vdots &\vdots\\
0 & 0 & \dots &\dots &\dots  & s&-1 \\
0 & 0 & \dots &  \dots &\dots&-1 &  s
        \end{array} \right )
%\,.
\ee{3diagToeplitz}
satisfies \emph{Dirchlet boundary conditions}, in the sense that the first and
the last site do not have a left (right) neighbor to couple to. We distinguish
it from the circulant matrix \refeq{3diagCirculant} by emphasising that
\(
(\D_\cl{})_{0,\cl{}-1}=(\D_\cl{})_{\cl{}-1,0}=0
\,.
\)
As the time-translation invariance is lost, the matrix elements of its inverse,
the Green's $[\cl{}\!\times\!\cl{}]$ matrix $\gd_{tt'}$,
with a delta-function source term and the Dirichlet boundary
conditions
\bea
 (\D \gd)_{tt'}&=&\delta_{tt'}\, \qquad t,t'\in 0,1,2,\cdots,\cl{}-1
                                            \label{1DGreenFun0}\\
  0&=&\gd_{-1,t'}\,=\,\gd_{t,-1}\,=\,\gd_{\cl{} t'}\,=\,\gd_{t \cl{}}
%\,.
\nnu   %\ee{1DGreenFunDirichlet}
\eea
depend on the point source location $t$, and no formula for its matrix elements
as simple as \refeq{GreenFun00a} is to be expected.
In general, a finite matrix inverse is of the form
\[
\D^{-1} = \frac{1}{\det \D} \transp{(\mbox{cofactor matrix of }\D)}
\,.
\]
While the cofactor matrix might be complicated, the key here is, as in formula
\refeq{GreenFun00a}, that the prefactor $1/\det \D$ falls off exponentially, and
for Toeplitz matrices can be computed recursively.

Associated with this simple tridiagonal matrix are the Chebyshev polynomials
of the first and the second kind
\[
  T_{\cl{}}(s/2) = \cosh(\cl{}\Lyap)
\,,\qquad
  U_{\cl{}}(s/2) = {\sinh(\cl{}+1)\Lyap}\,/\,{\sinh(\cl{}\Lyap)}
\,,
\]
generated by a three-term recursion relation (second-order difference
equation\rf{Elaydi05}).

The identity % from https://en.wikipedia.org/wiki/Chebyshev_polynomials
\beq
2\,T_{\cl{}}(s/2) % = 2\,T_{\cl{}}\left(\frac{\Lambda+\Lambda^{-1}}{2}\right)
            = \Lambda^{\cl{}}+\Lambda^{-\cl{}}
%\,.
\ee{ChebSecKindLamda}
follows from $s=\Lambda+\Lambda^{-1}$, see \refeq{catEigs}.

The inverse of the Dirichlet boundary conditions matrix $\D_\cl{}$
\refeq{3diagToeplitz} can be determined explicitly, in a number of different
ways\rf{Streater79, Smith85, HuCon96, Simons97}. Here we find it convenient to
write the inverse of $\D_\cl{}$ in the Chebyshev polynomial
form\rf{YamAbd97}. The determinant of $\D_\cl{}$, \ie, the Jacobian of the
linear transformation \refeq{eq:CatMapStep} is well known\rf{GraRyz}
\beq
  \det \D_{\cl{}}  % = %(-1)^\cl{}
                 % \frac{\sinh(\cl{}+1)\Lyap}{\sinh\Lyap}
                = U_{\cl{}}(s/2)
\,,
\ee{3diagToeplitzDet}
%(where $U_{\cl{}}(x)$ is the Chebyshev polynomial of the second kind),
and the matrix elements of the Green's function in the Chebyshev polynomial
form\rf{HuCon96,YamAbd97} are explicitly
\beq
 \gd_{ij}= \frac{1}{\det \D_{\cl{}}}\,\times
          \left\{
            \begin{array}{ll}
        U_{i-1}(s/2)\,U_{\cl{}-j}(s/2) \qquad &\mbox{for } i\leq j\\[1ex]
        U_{j-1}(s/2)\,U_{\cl{}-i}(s/2) \qquad &\mbox{for } i> j .
        \end{array}
           \right.
 \,.
\ee{BG1dGreen}
$\det \D_{\cl{}}$ is also known as the determinant of the {\em Dirichlet kernel}
(see \HREF{https://en.wikipedia.org/wiki/Dirichlet_kernel} {wiki})
\beq
D_{\cl{}}(x)=\sum_{k=-{\cl{}}}^{\cl{}}
e^{ikx}=1+2\sum_{k=1}^n\cos(kx)=\frac{\sin\left(\left({\cl{}} +1/2\right) x \right)}{\sin(x/2)}
\,.
\ee{DirichletKer}


It follows from the recurrence relation
$\ssp_{i+1}=s\ssp_{i}-\ssp_{i-1}\,,\,\mod~1$, that $U_{n}(s/2)$
Chebyshev polynomials have the generating function
\bea
\sum_{{n}=0}^\infty U_{n}(s/2) z^{n} & = & \frac{1}{1 - s z + z^2}
    \continue
& = & 1 + s z + (s^2 - 1) z^2 + (s^3 - 2s) z^3 + \cdots
\,,
\label{2ndChebGenF}
\eea
with $U_{n}(s/2) \approx s^n \approx \ExpaEig^n$, and for a hyperbolic system
the off-diagonal matrix elements $\gd_{tt'}$ are again falling off
exponentially with their separation $|t'-t|$, as in \refeq{GreenFun00a}, but
this time only in an approximate sense.

    \PC{2017-09-20}{
Probably should do circulants first, then the complicated Dirichlet case next,
in the spirit of starting out with the infinite lattice case
\refeq{GreenFun00a}.
    }
Alternatively, for finite time $\cl{}$ we can represent $\D$ by a symmetric
tridiagonal $[\cl{}\!\times\!\cl{}]$ circulant matrix with
\emph{periodic boundary conditions}
\beq
\D_\cl{}= \left(\begin{array}{ccccccc}
 s&-1 & 0 & 0 &\dots &0&-1 \\
-1 &  s&-1 & 0 &\dots &0&0 \\
0 &-1 &  s&-1 &\dots &0 & 0 \\
\vdots & \vdots &\vdots & \vdots & \ddots &\vdots &\vdots\\
0 & 0 & \dots &\dots &\dots  & s&-1 \\
-1 & 0 & \dots &  \dots &\dots&-1 &  s
        \end{array} \right )
\,.
\ee{3diagCirculant}
In the periodic boundary conditions case the determinant (in contrast to the
Dirichlet case \refeq{3diagToeplitzDet}) is obtained by Fourier-transform
diagonalization
\bea
\det \D_n
 = \prod_{j=0}^{n-1}
             \left[s - 2 \cos\left(\frac{2 \pi j}{\cl{}}\right)\right]
 = 2\,T_{\cl{}}(s/2) -2
\,,
\label{3diagCircDet}
\eea
see \refeq{ChebSecKindLamda}.
    \HL{2018-12-01}{
    I still have to derive and recheck this formula!
    }


Now the discrete
matrix Green's function $\gd_{tt'}$ satisfies periodic boundary conditions
\bea
 (\D \gd)_{t1}&=&\delta_{t1}, \qquad\qquad\qquad t = 1,2,\cdots,\cl{}
                                            \label{1DGreenCirc}\\
 \gd_{\cl{}+1, t'}&=&\gd_{1t'}
\,,\qquad
 \gd_{t,\cl{}+1}\,=\,\gd_{t1}
\,.
\nnu   %\ee{1DGreenFunDirichlet}
\eea
Note that the Green's matrix is strictly negative for both the periodic and
Dirichlet boundary conditions.


%%%%%%%%%%%%%%%%%%%%%%%%%%%%
\bigskip %\bigskip\bigskip

\paragraph{Left over from Boris version:}
Consider the single cat map equation %\refeq{eq:CatMapNewton5}
with a delta-function
source term
\begin{equation}
 (-\Box+ {\mu}^2) \gd_{t}=\delta_{t,0}, \qquad t\in \integers^1
\,.
\label{GreenFun0a}
\end{equation}

An alternative  way to evaluate   $\gd_{i,j}$   is to use  Green's
function $g$ and take antiperiodic sum (similar method can be used for
periodic and Neumann  boundary conditions)
\beq
   \gd_{i,j}=\sum_{n=-\infty}^{\infty}
               \gd_{i,j+2 n(\cl{}+1)}- \gd_{i,-j+2 n(\cl{}+1)}
\,.
\ee{GreenFun0b}
This approach has an advantage of being extendable  to the $\integers^2$
case. After substituting $g$ and taking the sum one obtains
\refeq{BG1dGreen}.

See also \refsect{sect:ChebyshevSer}~{\em Chebyshev series}.

\renewcommand{\Ssym}[1]{{\ensuremath{m_{#1}}}}    % Boris
% \renewcommand{\Ssym}[1]{{\ensuremath{s_{#1}}}}  % ChaosBook

\section{Green's blog}
\label{sect:blog1dGreen}

\begin{description}
    \PCpost{2017-08-24,2017-09-09}{
This to all curious cats, but mostly likely only Boris might care: OK
now I see why Chebyshevs...

Chebyshev expansions are used here because of the recurrence relations that
they satisfy.

                                                           \toCB
A \emph{Toeplitz matrix}, $T$, is a matrix that is constant along each
diagonal, \ie, $T_{jk} = t_{j-k}$.
A \emph{Hankel matrix}, $H$, is a matrix that is constant along each
anti-diagonal, i.e., $H_{jk} = h_{j+k}$.
There is also  the \emph{Laurent matrix} or \emph{doubly infinite Toeplitz
matrix}.


R. M. Gray (2009) % Gray09
\HREF{http://ee.stanford.edu/~gray/toeplitz.html}
{Toeplitz and Circulant Matrices: A Review}
focuses on bounds of sums of eigenvalues - I see nothing here that is of
immediate use to us, with maybe the exception of the discussion of the
diagonalization of circulant matrices (discrete Fourier series).

                                                            \toCB
A look at a {Toeplitz} matrix evokes time evolution of a periodic orbit
symbolic \brick: it looks like successive time shifts stacked upon each
other, every entry is doubly periodic on a torus of size
$[\cl{p}\times\cl{p}]$. Does that have to do something with Chebyshev
polynomials (rather than with the usual discrete Fourier series)? One
uses Chebyshev polynomials of the first, second, third, and fourth kind,
denoted by $T_{\cl{}}, U_{\cl{}}, V_{\cl{}}, W_{\cl{}}$,
if, as an example, one looks at a  pentadiagonal symmetric Toeplitz matrix,
a generalization of the 3rd order spatial derivative.

Circulant matrices are discussed in Aitken\refref{Aitken39} (1939).

Maybe some of the literature cited here illuminates this:
    }

    \PCpost{2017-09-09}{
The eigenvalues and eigenvectors for the finite symmetric tridiagonal
Toeplitz matrix might have been obtained by Streater\rf{Streater79} {\em
A bound for the difference {Laplacian}}, but I do not see where in the
article they are. They seem to also be given in Smith\rf{Smith85} {\em
{Numerical Solution of Partial Differential Equations: Finite Difference
Methods}}.


Hu and {O'Connell}\rf{HuCon96}
{\em Analytical inversion of symmetric tridiagonal matrices}
``
present an analytical formula for the inversion of symmetrical tridiagonal
matrices. As an example, the formula is used to derive an exact analytical
solution for the one-dimensional discrete {\sPe} (DPE) with
Dirichlet boundary conditions.
''
The $\cl{}$ eigenvalues and orthonormal $\cl{}$\dmn\ eigenvectors of  $\D$
are
    \PC{2017-09-09}{recheck!}
\bea
\gamma_k &=& s + 2\cosh\frac{k\pi}{\cl{}+1}
    \,,\qquad k=1,2,\cdots,\cl{}
            \continue
e^{(k)}_n &=& \sqrt{\frac{2}{\cl{}+1}}
     \sinh\frac{kn\pi}{\cl{}+1}
%    \,.
\label{3diagToepEigs}
\eea
(see, for example, \refrefs{HuCon96,YamAbd97}).
This is a typical inverse propagator, see ChaosBook\rf{CBappendDiff}
\beq
(\varphi_k^{\dagger} \cdot \Laplacian  \cdot \varphi_{k'})
 =
 \left(- {2}\cos\left({2\pi \over N}k\right) + 2  \right)
         \delta_{kk'}
\label{Lat-LapCos}
\eeq
The inverse (the Green's function) $\gd\D=1$ is\rf{HuCon96}
    \PC{2017-09-09}{
    It is shown in \refref{YamAbd97} that is the same as the formula
    \refeq{BG1dGreen} Boris uses (without a source citation).
    }
\beq
\gd_{jk} % (T^{-1})_{jk}
=   \frac{
\cosh(\cl{}+1-|k-j|)\Lyap\,-\,\cosh(\cl{}+1-j-k)\Lyap
     }{
2\sinh\Lyap\,\sinh(\cl{}+1)\Lyap
     }
\ee{HuCon96(10)}
The above paper is applied to physical problems in
Hu and {O'Connell}\rf{HuCon94} {\em Exact solution for the charge soliton in
a one-dimensional array of small tunnel junctions},
and in
Hu and {O'Connell}\rf{HuCon95} {\em Exact solution of the electrostatic
problem for a single electron multijunction trap}. The
\HREF{https://doi.org/10.1103/PhysRevLett.76.4097} {erratum} is of no
importance for us, unless the - sign errors affect us.

A cute fact is that they also state the solution for $s=2$,
which, unlike \refeq{HuCon96(10)} has no exponentials - it's a power law.

%    \PCpost{2017-09-09}{
Eigenvalues, eigenvectors and inverse for $[\cl{}\!\times\!\cl{}]$ matrix $\D$
\refeq{3diagToeplitz}, $-2<s<2$,
\bea
\Lyap_k &=& -s + 2\cos\frac{k\pi}{\cl{}+1}
        \continue
e_k &=& \sqrt{\frac{2}{\cl{}+1}}\left(
     \sin\frac{k\pi}{\cl{}+1},\sin\frac{2k\pi}{\cl{}+1},\cdots,\sin\frac{\cl{} k\pi}{\cl{}+1}
                \right)
        \continue
(\D^{-1})_{k\cl{}} &=& \frac{2}{\cl{}+1} \sum_{m=1}^{\cl{}}
     \frac{
\sin\frac{km\pi}{\cl{}+1}\sin\frac{\cl{} m\pi}{\cl{}+1}
     }{
-s + 2\cos\frac{k\pi}{\cl{}+1}
     }
\label{HuCon95Helmh}
\eea
are computed in
Meyer\rf{Meyer00} {\em Matrix Analysis and Applied Linear Algebra}.

Yamani and Abdelmonem\rf{YamAbd97}
{\em The analytic inversion of any finite symmetric tridiagonal matrix}
rederive Hu and {O'Connell}\rf{HuCon96}, using the theory of orthogonal
polynomials in order to write down explicit expressions for the polynomials
of the first and second kind associated with a given infinite symmetric
tridagonal matrix H.

The matrix representation of many physical operators are tridiagonal, and
some computational methods, are based on creating a basis that renders a
given system Hamiltonian operator tridiagonal.  The advantage lies in the
connections between tridiagonal matrices and the orthogonal polynomials,
continued fractions, and the quadrature approximation which can be used to
invert the tridiagonal matrix by finding the matrix representation of the
Green's functions.

The Green's function $G(z)$ associated with the matrix $H$ is
defined by the relation
\beq
(H-zI)G=I
\,.
\ee{YamAbd97inv}
It is more convenient to calculate the inverse of the matrix $(H-zI)$ instead
of the inverse of the matrix $H$.
Note that in this formulation $G$ is the \emph{resolvent} of $H$.

Simons\rf{Simons97}
{\em Analytical inversion of a particular type of banded matrix}
rederives Hu and {O'Connell}\rf{HuCon96} by a ``a simpler and
more direct approach''.
The structure of \refeq{YamAbd97inv} is that of a homogeneous
difference equation with constant coefficients and therefore one looks
for a solution of the form
\beq
  G_{pq} = A_q e^{p\Lyap}+A_q e^{-p\Lyap}
\,,
\ee{YamAbd97fundSol}
with appropriate boundary conditions. This leads to
the Hu and {O'Connell} formulas for the inverses of $G$.

\HREF{https://math.stackexchange.com/questions/179893/how-to-invert-a-very-regular-banded-toeplitz-matrix}
{How to invert} a very regular banded Toeplitz matrix.

Yueh\rf{Yueh06} {\em Explicit inverses of several tridiagonal matrices} has a
bunch of fun tri-diagonal Toeplitz matrix inverses, full of integers - of no
interest to us.

Dow\rf{Dow03}
{\em Explicit inverses of {Toeplitz} and associated matrices}: ``
We discuss Toeplitz and associated matrices which have simple explicit
expressions for their inverses. We first review existing results and
generalize these where possible, including matrices with hyperbolic and
trigonometric elements. In Section 4 we invert a tridiagonal Toeplitz matrix
with modified corner elements. A bunch of fun tri-diagonal Toeplitz matrix
inverses, full of integers - of no interest to us.

Noschese,  Pasquini  and Reichel\rf{NoPaRe13}
{\em Tridiagonal {Toeplitz matrices}: properties and novel applications}
use the eigenvalues and eigenvectors of tridiagonal Toeplitz matrices to
investigate the sensitivity of the spectrum. Of no interest to us.

Berlin and Kac\rf{bk}
{\em The spherical model of a ferromagnet}
  use bloc-circulant matrices; see also

Davis\rf{PJD} {\em Circulant Matrices}

    }

    \PCpost{2017-09-09}{
Gover\rf{Gover94} {\em The Eigenproblem of a Tridiagonal 2-Toeplitz Matrix}
seems less useful:
``
The characteristic polynomial of a tridiagonal 2-Toeplitz matrix is shown to
be closely connected to polynomials which satisfy the three point Chebyshev
recurrence relationship. This is an extension of the well-known result for a
tridiagonal Toeplitz matrix. When the order of the matrix is odd, the
eigenvalues are found explicitly in terms of the Chebyshev zeros. The
eigenvectors are found in terms of the polynomials satisfying the three point
recurrence relationship.
''

Gover\rf{Gover94} motivates his paper by reviewing a tridiagonal 1-Toeplitz,
or Toeplitz matrix, referring to the original literature. Consider a
tridiagonal $[\ell\!\times\!\ell]$ Toeplitz  matrix with Dirchlet boundary
conditions \refeq{3diagToeplitz}, with eigenvalues \refeq{3diagToepEigs}.

K{\"u}bra Duru and Bozkurt\rf{KubBoz16}
{\em Integer powers of certain complex pentadiagonal 2-{Toeplitz} matrices}

Elouafi\rf{Elouafi14} {\em On a relationship between {Chebyshev} polynomials
and {Toeplitz} determinants}:
``Explicit formulas are given for the determinants of a band symmetric
Toeplitz matrix $T_{\cl{}}$ with bandwidth $2r+1$. The formulas involve $r\times r$
determinants whose entries are the values of Chebyshev polynomials on the
zeros of a certain $r$th degree q which is independent of {\cl{}}.''
    }

    \PCpost{2017-09-09}{
\HREF{http://page.math.tu-berlin.de/~felsner/Paper/lattice-paths.pdf}
{Felsner and Heldt} {\em Lattice paths} seem to be all on graphs - I see
no 2\dmn\ lattice here.

\HREF{http://epubs.siam.org/doi/pdf/10.1137/15M1008221}
{\em
Spectral asymptotics} {\em in one-dimensional periodic
lattices with geometric interaction}
    }

\end{description}

 \renewcommand{\Ssym}[1]{{\ensuremath{m_{#1}}}}    % Boris
% \renewcommand{\Ssym}[1]{{\ensuremath{s_{#1}}}}  % ChaosBook

\section{Chebyshev series}
\label{sect:ChebyshevSer}

%  2020-06-03 seminar below
\begin{bartlett}{
Chebyshev series are Fourier (cosine) series in disguise.
        }
\bauthor{
Jason Mireles-James
    }
\end{bartlett}

The canonical reference is
Boyd\rf{boyd01} {\em Chebyshev and Fourier Spectral Methods}
\CBlibrary{boyd01}. Perhaps check also:

\HREF{http://keaton-burns.com/docs/chebyshev_essay.pdf}{Keaton J. Burns}
{\em Chebyshev Spectral Methods with applications to Astrophysical Fluid Dynamics}.

Philippe Grandclement\rf{Grandc06} {\em Introduction to spectral methods}
\arXiv{gr-qc/0609020}.
%DOI \HREF{https://dx.doi.org/10.1051/eas:2006112} {10.1051/eas:2006112}

\subsection{Spectral methods}
\label{sect:SpectrMeth}
                                                            \toCB
The basic idea of all numerical techniques is to approximate any function
$u(x)$ by polynomials, $\hat{u} = \sum_{n=0}^N \hat{u}_n p_n(x)$ where the
$p_n(x)$ are polynomial {\em trial functions}. Depending on the choice of
trial functions, one has various classes of numerical techniques. For
example, the {\em finite difference} schemes are obtained by choosing
local polynomials of low degree. In \emph{spectral methods} the $p_n(x)$
are global polynomials, typically Legendre or Chebyshev. Spectral methods
can reach very good accuracy with only moderate computational resources;
for ${{\mathcal C}^\infty}$ functions, the error decays exponentially, as
one increases the degree of the approximation.

A function $u$ can be described either by its value $u(x_i)$ at each
collocation point $x_i$ or by the coefficients $\tilde{u}_i$ of the  {\em
interpolant} of $u$
\beq
[I_N u](x) = \sum_{n=0}^N \tilde{u}_n p_n(x).
\ee{Grandc(6)} %{interpol}
The computation of $\tilde{u}$ only requires evaluation of $u$ at the $N+1$
collocation points. The interpolant of $u$ is the spectral approximate of
$u$ in terms of polynomials of degree $N$ that coincide with $u$ at each
collocation point:
\[
[I_nu](x_i) = u(x_i) \quad \forall i\leq N.
\]

If the values at collocation points are known one is working in the {\em
configuration space}, and in the {\em coefficient space} if $u$ is given
in terms of its coefficients.

Depending on the operation one has to perform, one choice of space is
usually more suited than the other. The derivative of $u$ can be
evaluated in the coefficient space by approximating
$u'$ by the derivative of the interpolant,
\[
u'(x) \approx [I_N u]'(x) = \sum_{n=0}^N \tilde{u}_n p_n'(x)
\,.
\]
This requires only the knowledge of the coefficients of
$u$ and the derivatives of the basis polynomials. This approximate derivative
is not the interpolant of $u'$, as the polynomials that represent
$(I_Nu)'$ do not coincide with $u'$ at the collocation points.

\subsection{Discretizing with Chebyshev polynomials}
\label{sect:ChebyshevPoly}

To use Chebyshev series as a basis, one shifts the problem to functions
defined over $x\in[-1,1]$, and expands them in {Chebyshev polynomial of the
first kind} $T_k(x)$
\beq
\field(x) = \field_0 + 2 \sum_{n=1}^\infty \field_n T_n(x)
\,,
\ee{Lessard20a:1}
where $T_k(z)$ are
defined by 3-term recurrence \refeq{Rangarajan17:TkRecurr}.

Expanded as Chebyshev series, the product of functions
\[
a(x) = a_0 + 2 \sum_{n=1}^\infty a_n T_n(x)
\,,\qquad
b(x) = b_0 + 2 \sum_{n=1}^\infty b_n T_n(x)
\,,
\]
satisfies the Fourier-like convolution formula
    \PC{2020-06-07}{
I think of Fourier convolution formula as a statement of
the translation invariance condition on matrices, not sure how to think
about this.
    }
\[
(a\cdot b)(x) =(a*b)_o + 2 \sum_{n=1}^\infty(a*b)_n T_n(x)
\]
where
\[
(a*b)_n = \sum_{k_1+k_2= n}
          a_{|k_1|}b_{|k_2|}
\,,\qquad k_1,k_2\in\integers
\]

Chebyshev polynomials are an analogue of the Fourier expansion for non
periodic functions on an interval and, as the Chebyshev polynomials of
the first kind\rf{boyd01} satisfy
\beq
T_{\cl{}}(\cos(x))=\cos(\cl{}x)
\,,
\ee{ChebTkDefined}
they are Fourier series in disguise.
Mapping \refeq{ChebTkDefined} geometric interpretation: the $n$th
Chebyshev polynomial is the projection onto a plane of the function
$y =\cos(\cl{}x)$ drawn on a cylinder.


For $x=0$, $T_n(1) = 1\,.$
For
$x=2\pi k/\cl{}$, $k=0,1,...,\cl{}-1$, $\cos(2\pi k/\cl{})$ is the $k$th
root of equation
\[
T_n(x)-1=0
\,.
\]
This equation can be written as a product over the eigenvalues
% \refeq{appe:tempCatFT}
\beq
T_{\cl{}}(x)-1 =
2^{\cl{}-1} \prod_{k=0}^{\cl{}-1} \left[x - \cos({2\pi k}/{\cl{}})\right]
\,.
\ee{CL18:factorChebPoly}
Here the coefficient $2^{\cl{}-1}$ comes from matching the coefficient
of $x^\cl{}$ term in the definition of $T_{\cl{}}(x)=\cdots+2^{\cl{}-1} x^\cl{}$.
For $x={s}/2$, this is the \jacobianOrb\ determinant formula
%\refeq{POsChebyshev}
\beq
N_\cl{}
 = \prod_{k=0}^{\cl{}-1} \left[ {s} - 2\cos\left({2 \pi k}/{\cl{}}\right)\right]
 = 2 T_{\cl{}} \left({s}/{2}\right) - 2
\,.
\ee{CL18:detTemCatCheb}

Three different types of partial differential equation
solvers\rf{Grandc06} are  the {\em Tau-method}, the {\em collocation
method} and the {\em Galerkin method}.

The basic idea of the Galerkin method is to expand the solution as a
linear combinations of polynomials -the {\em Galerkin basis}- that fulfill
the boundary conditions.

The Chebyshev polynomials $T_n$ are an orthogonal set on $[-1,1]$ for the measure
$w = \frac{1}{\sqrt{1-x^2}}$,
\beq
\int_{-1}^1 \frac{T_n T_m}{\sqrt{1-x^2}} {\rm d}x
   = \frac{\pi}{2}(1+\delta_{0n}) \delta_{mn}
   \,.
\ee{Grandc(14)}

\bigskip

\begin{description}
\item[2020-06-03 Jason Mireles-James]
\HREF{https://primetime.bluejeans.com/a2m/events/playback/2d867419-fd23-4893-9236-39b84991bddd}
{talk},
{\em Parameterization of unstable manifolds for delay differential equations}:
Delay differential equations (DDEs) are important in physical
applications where there is a time lag in communication between
subsystems. They provide natural examples of infinite
dimensional dynamical systems.  He discusses Chebyshev spectral numerical
methods for computing invariant manifolds for DDEs.

\item[2020-06-03 Jean-Phillipe Lessard]
\HREF{https://researchseminars.org/talk/CAPA_UU_SEMINAR/2/}
{talk},
{\em Rigorous integration of infinite
dimensional dynamical systems via Chebyshev series}: In this talk we
introduce recent general methods to rigorously compute solutions of
infinite dimensional Cauchy problems. The idea is to expand the solutions
in time using Chebyshev series and use the contraction mapping theorem to
construct a neighbourhood about an approximate solution which contains
the exact solution of the Cauchy problem. We apply the methods to some
semi-linear parabolic partial differential equations (PDEs) and delay
differential equations (DDEs).

For my screen grabs from the 2 talks,
\CBlibrary{Lessard20a}.

\item[2020-06-03 Predrag] We (John Gibson \HREF{http://channelflow.org}
{channelflow.org}, \etc) use Chebyshev in the
wall-normal directions in Navier-Stokes channel flow high-accuracy
integrators, as the Laplacian is a banded matrix in the Chebyshev basis.
But I do not like them, as they put all wiggles close to the walls, and
lots of interesting turbulence is going on in the middle of the channel,
around the middle of the $[-1,1]$ interval.

Dear Abby, am I just being prejudiced for no good reason?

\end{description}
