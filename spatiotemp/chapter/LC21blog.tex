% siminos/spatiotemp/chapter/LC21blog.tex
% $Author: predrag $ $Date: 2021-12-24 01:25:20 -0500 (Fri, 24 Dec 2021) $

\section{Kittens' LC21blog}
\label{s:CL18blog}

Internal discussions of \refref{LC21} edits:
Move good text not used in \refref{LC21} to this file, for possible
reuse later.

\bigskip

Tentative title:    ``Is there anything cats cannot do?"

\begin{description}

\item[2016-11-18 Predrag]
A theory of turbulence that has done away with \emph{dynamics}?
We rest our case.

As a gentle introduction
for a reader too busy\rf{focusPOT} to study the book\rf{ChaosBook},
we
disguise a brief course on chaos theory as something everyone
understands, a Bernoulli coin toss, \refsect{s:coinToss}.

one determines the total number of
{\lattstate}s by computing the {\HillDet} \refeq{detBern0} of the
\emph{\jacobianOrb}

The observation that a Bernoulli system can be viewed as a discretization
of a first-order in time ODE, eq.~\refeq{1stepVecEq}, with solutions
whose temporal global linear stability is described by the {\jacobianOrb}
$\jMorb_{\zeit\zeit'}=\delta{F[\Xx]}_{\zeit}/\delta\ssp_{\zeit'}$, has
profound implications for dissipative \spt\ systems such as Navier-Stokes
and Kuramoto-Sivashinsky\rf{GuBuCv17}.

As we shall here have to traverse territory unfamiliar to many, we
follow Mephistopheles pedagogical dictum ``You have to say it three
times"\rf{GoetheIstuZim1806}, and sing our song thrice.

The deep insight here is that the two formulations of mechanics, the
for\-ward-in-time Hamiltonian evolution, and the global, Lagrangian,
{\templatt} formulation are related by the {\em Hill's formula}.

The deep insight
here is the realization that the {\em\HillDet}, \ie, the volume of the
{\em\jacobianOrb} (\reffig{fig:BernCyc2Jacob} and \ref{fig:catCycJacob})
partitions system's \statesp.

Next, we address two questions:
(i) how is the high-dimensional orbit \jacobianOrb\ $\jMorb$ related
to the temporal [$d\!\times\!d$] \jacobianM\ $\jMat$?
(\refsect{s:LC21Hill}),
and
(ii) how does one evaluate the orbit \jacobianM\ $\jMorb$?
(\refsects{sect:LC21recip1d}{sect:LC21recip1d}).

The theory is compactly summarized by its {\tzeta} \refeq{Isola90-13}
that counts Bravais lattices.

\PCpost{2021-07-07}{
Experimenting with \refeq{Ryu17eq:1.3C} by:\\
%
%Kim \etal\rf{KiLePa03} prove that every finite index subgroup of
%the {infinite dihedral group} $\Dn{\infty}$
%\refeq{D_infty}
%\[
%\langle r, \Refl \mid \Refl r \Refl=r^{-1} \,,\; \Refl^2=1\rangle
%\]
%is
a flip across the $k$th axis, $k=0, 1, 2, \cdots, \cl{}-1$,
\bea
\mbox{dihedral } \Dn{\cl{}}:\quad
H_{\cl{},k}  &=&
\langle \shift, \Refl_k=\shift^k \Refl \mid \Refl_k \shift \Refl_k=\shift^{-1}
    \,,\;
           \shift^\cl{}=\Refl_k^2=1 \rangle
%            \continue
%            &&
%\cl{}=1, 2, 3, \cdots
\,,
\label{Ryu17eq:1.3C}
\eea
that Han had replaced with \refeq{KiLePa03H}
and $\cl{}$ with $|\cl{}|$ in \refeq{KiLePa03Index}.

%with index
%\beq
%|\Dn{\infty}/\Cn{\cl{}}| =  2\cl{}
%    \qquad \text{or} \qquad
%|\Dn{\infty}/\Dn{\cl{}}| = \cl{}
%\,.
%\ee{Ryu17eq:3.0}
}

\PCpost{2021-07-07}{
A presentation of the \emph{infinite dihedral group}\rf{KiLePa03} is
\beq
\Dn{\infty} = \left\langle {\shift_i,\Refl_j} \mid
      \shift_i\Refl_j= \Refl_j\shift_{-i};\;
      \Refl_j^2 = 1;\; i,j\in\integers
              \right\rangle
\,.
\ee{presntD_infty}
}

\item[2021-08-10 Predrag]
dropped:

Applying the projection operator
\(
\PP_{{0}-} = \frac{1}{2} (\unit-\Refl_{0})/2
\)
we obtain
a {\lattstate}
\beq
\cdots
\underline{{y}_{4}}\,\underline{{y}_{3}}\,\underline{{y}_{2}}\,\underline{{y}_{1}}\,
 \sitebox{0}
      {y}_{1} {y}_{2} {y}_{3} {y}_{4}  \cdots
\,,
\ee{latticeInf0-}
antisymmetric under reflection, where the
field
\(
\sitebox{0} =(\ssp_0-\ssp_0)/2 =0
\)
at the reflection lattice site $0$ vanishes by antisymmetry, while the rest,
\(
{y}_{j} =(\ssp_{j}-\ssp_{-j})/2
\,,
\)
are pairwise antisymmetric under the reflection $\Refl$. The underline
indicates the negative of, \ie, $\underline{{y}_{j}}=-{y}_{j}$.

Applying the antisymmetric projection operator
\(
\PP_{{1}-} = \frac{1}{2} (\unit-\Refl\shift)/2
\)
we obtain a {\lattstate}
\beq
\cdots
\underline{{y}_{4}}\,\underline{{y}_{3}}\,\underline{{y}_{2}}\,\underline{{y}_{1}}\,
        |
      {y}_{1} {y}_{2} {y}_{3} {y}_{4}  \cdots
\,,
\ee{latticeInf1-}
antisymmetric under reflection, where
\(
{y}_{j} =(\ssp_{j}-\ssp_{1-j})/2
\,,
\)
are pairwise antisymmetric under the reflection $\Refl_{1}$.

    \PCpost{2021-08-21}{
The old definition of Bernoulli $\Ssym{\zeit}$ in \refeq{LC21:1dPhi4}
conflicted with the definition \refeq{circ-m}. I changed \refeq{circ-m}
to current form.
    }

    \PCpost{2021-10-29}{Dropped:
Cat maps are beloved by ergodicists and statistical mechanicians because,
even though the field $(\coord_{\zeit},p_{\zeit})$ is 2\dmn, for integer
values of the stretching parameter $s$, a cat map has a finite alphabet
linear code, just like the Bernoulli map, and its
unit torus can be tiled by two rectangles,
    \PC{2020-12-17}{
Link to the ChaosBook.
    }
in analogy with the forward-in-time Bernoulli map subinterval
partitioning of \reffig{fig:BernPart}. From this it follows that all
admissible symbol {\brick}s can be generated as shifts of finite
type, and all periodic points determined and counted.

As all that is well known, and a side issue for this paper, we relegate
the details of the Hamiltonian cat map dynamics and \po\ counting to
??.
    \PC{2020-12-17}{
Link to the ChaosBook.
    }
Here we focus on reformulating the cat dynamics as
a temporal lattice (or discrete Lagrangian) problem, as we have done for
the Bernoulli system in \refsect{s:1D1dLatt}.
    }

    \PCpost{2021-10-29}{Dropped:
The Lagrangian formulation requires only temporal {\lattstate}s
and their actions, replacing the phase space `cat map' \refeq{catMap}
by a `{\templatt}' lattice \refeq{OneCat}. The {\templatt} has no
generating partition analogue of the \AW\ partition for a Hamiltonian cat
map (see
    \PC{2020-12-17}{
Link to the ChaosBook.
    }
).
As we have shown here, no funky Hamiltonian \statesp\ partitioning magic
(such as
    \PC{2020-12-17}{
Link to the ChaosBook.
    }
) is needed to count the
{\lattstate}s of a \templatt. Not only are no such partitions needed to
solve the system, but the Lagrangian,
    }

    \PCpost{2021-08-12}{
    Sidney will chuckle at this comment:
    The usual $a\ssp_{\zeit}^2$ form \refeq{Hen-1dLattA} might be
    preferable, as the `$a$' is a stretching parameter, just like in
    \refeq{LC21:1dTemplatt}. See \refsect{sect:HamHenonMap}~{{\Henlatt}}.
%    The same for $\phi^4$ coupling constant in \refeq{LC21:1dPhi4}.
    }

    \PCpost{2020-12-17}{
Link to the ChaosBook? or drop?\\
In \refsect{s:catPV} we review the traditional cat map in its Hamiltonian
formulation.
(but relegate to the explicit \AW\ generating partition of the cat map
\statesp).

We evaluate and cross-check  {\HillDet}s by two methods, either the
`fundamental fact' evaluation, or by the
discrete Fourier transform diagonalization, \refsect{sect:LC21recip1d}.
    }

    \PCpost{2021-10-13}{
Is there a - sign specific to Sidney's definition of the
\Henon\ {\jacobianOrb}
Han and Predrag have to redefine both
\templatt\ and \henlatt\ {\jacobianOrb} throughout, so we do not pick up an
extraneous `-' sign for odd period {\lattstate}s.
%
See also \refeq{Pozrikidis14(1.1.3)}, and Pozrikidis\rf{Pozrikidis14}
\CBlibrary{Pozrikidis14} eq.~(1.8.2). The main thing is to have a Laplacian
with positive eigenvalues, right? Maybe not, the main thing is to
have hyperbolic eigenvalues for $s>2$. Rethink.
determinants in \po\ formulas.

%       Predrag                          3 feb 2005
$Z[\source]$ notation extracted from \emph{lattFTnotat.tex},
called \emph{by lattFT.tex}.
% \Chapter{lattFT}{3feb2005}{Lattice field theory}

, in field theorist's parlance, $\Ssym{z}$ are `sources', and

The {\jacobianOrb} $\jMorb[\Xx]$ is best understood by starting with the
period-$\cl{}$ Bravais cell stability.

As in \refsect{sect:fundFact}, the {\fundPip} given the stretching of the
\cl{}\dmn\ \statesp\ unit hyper-cube $\Xx\in[0,1)^\cl{}$ by the
{\jacobianOrb} counts {\lattstate}s, with the {\admissible} {\lattstate}s
of period $\period{}$ constrained to field values within
$0\leq\ssp_\zeit<1$. The {\fundPip} contains images of all {\lattstate}s
$\Xx_\Mm$, which are then translated by integer winding numbers $\Mm$
into the origin, in order to satisfy the fixed point condition
\refeq{tempFixPoint}.

    }

    \PCpost{2021-10-21}{
Han, RECHECK all $\Ssym{\zeit}$, as well as formulas starting with
\refeq{1stepNonlimTemp}!!! Bernoulli $\Ssym{\zeit}$ in
\refeq{LC21:1dPhi4} conflicted with the old definition \refeq{circ-m}, so
I changed \refeq{circ-m}.

When the force is proportional to displacement, that is, when Hooke's law
is obeyed, the spring is said to be linear, the potential is quadratic.

A matrix $\jMat$ with no eigenvalue on the unit circle is called
hyperbolic.

Ignoring (mod~1) for a moment, we can use
\refeq{LC21PerViv2.1b} to eliminate
$p_{\zeit}$ from \refeq{LC21PerViv2.1a}
and rewrite the kicked rotor equation as the

For the problem at
hand, it pays to go from the Hamiltonian (configuration, momentum) phase
space formulation to the discrete Lagrangian
$(\ssp_{\zeit-1},\ssp_{\zeit})$ formulation.

\emph{temporal lattice} condition

`Temporal' again refers to the discretized time 1$d$ lattice

In atomic physics applications, the values of the angle $\coord$
differing by integers are identified, but the momentum $p$ is unbounded.
In dynamical systems theory one compactifies the momentum as well, by
adding $(\mbox{mod}\;1)$ to \refeq{LC21PerViv2.1a}, as for the Bernoulli map
\refeq{n-tuplingMap}. This reduces the phase space to a square
$[0,1)\times [0,1)$ of unit area, with the opposite edges identified,
\ie, 2-torus.

Thom-Anosov diffeomorphism

Cat maps with the same $s$ are equivalent
up to a similarity transformation, so it suffices to work out a single
convenient realization, as we shall do here for the
\PV\rf{PerViv} `two-configuration representation'
\refeq{LC21PerViv}.

    }

    \PCpost{2021-11-29} {
    Might need to introduce the inverse temperature $\beta = 1/T$ and the
    free energy $F$, as in \refeq{NKHM15ZJ}, multiplied by `volume' $N$
    the number of lattice sites;
\[
  Z[\source]	= e^{W[\source]}
  \,,\qquad
  W[\source] = \beta N F[\Xx]
\]
    So, $W[\source]$ is not the `free energy'.

Hill's formula here is the discrete Hill's formula\rf{MacMei83,BolTre10}
\refeq{MacMei83(17)}.

The temporal Bernoulli {\jacobianOrb} %\refeq{1stepVecEq},
$\jMorb=\partial/\partial\zeit-(s-1)\,\shift^{-1}$ is a differential
operator whose determinant one usually computes by a Fourier transform
diagonalization (see \refsect{sect:LC21recip1d}). The Fourier discretization
approach goes all the way back to Hill's 1886 paper\rf{Hill86};
    }

\PCpost{2021-11-29} {
                    {\color{red}
!!!WARNING!!!
Following Han \refeq{HLcorrectOrbJac}, we are changing the sign of the
action $\action[\Xx]$ and the {\jacobianOrb}, as in \refeq{LC21:1dTempFT},
THROUGHOUT! (Totally Predrag's fault).
This makes \catlatt\ and {$\phi^4$} theory action strictly positive
for
${s}>2$, as
needed for the probability interpretation \refeq{ProbConf}.
                    }
Han, Sidney and Predrag have to redefine
\templatt, \catlatt\ and \henlatt\ {\jacobianOrbs} throughout, to avoid the
extraneous `-' sign for odd period {\lattstate}s.
%
See also \refeq{Pozrikidis14(1.1.3)}, and Pozrikidis\rf{Pozrikidis14}
\CBlibrary{Pozrikidis14} eq.~(1.8.2).
    }

\PCpost{2016-11-08} {
Say: THE BIG DEAL is

for $d$\dmn\ field theory, symbolic dynamics is not one temporal sequence
with a huge alphabet, but $d$\dmn\ {\spt} tiling by a finite alphabet
   }

\PCpost{2021-12-27} {removed:\\
%\subsection
{\bf {\HillDet}: time-evolution evaluation}
%not used: \label{s:LC21HillHam}

However, in classical and statistical mechanics, one often computes the
{\HillDet} using a  Hamiltonian, or `transfer matrix' formulation.

Define
\[  %\beq
\hat{\xx}_\zeit
=
\left[\begin{array}{l}
 {\ssp}_{\zeit-1}\\
 {\ssp}_\zeit
 \end{array}\right]
,\quad
\hat{\mathsf{\Ssym{}}}_\zeit
=
    \left[\begin{array}{l}
    {0}\\
 {\Ssym{}}_{\zeit}
 \end{array}\right]
\,,
\] %\ee{PV2catlattJ1}
where the hat $\hat{~}$~ indicates a $2$\dmn\
`two-configuration'\rf{PerViv} lattice site $\zeit$ state.

The $1$\dmn\ field theory 3-term recurrence \refeq{LC21:1dTempFT} written
in the \PV\rf{PerViv} `two-configuration representation'
\refeq{LC21PerViv}.


$\mathbf{\jMorb}_1$ is the spatial
$[\speriod{}\!\times\!\speriod{}]$ {\jacobianOrb} of
$d=1$ \templatt\ form \refeq{tempCatFix},


This proves that $\det\hat{\mathbf{\jMorb}}$ of the
`Hamiltonian' or `two-configuration'
$[2\speriod{}\cl{}\times2\speriod{}\cl{}]$ `phase space'
\jacobianOrb\ $\hat{\mathbf{\jMorb}}$ defined by \refeq{eq:orbitJPVJxS}
equals the `Lagrangian' {\HillDet} of the
$[\speriod{}\cl{}\times\speriod{}\cl{}]$ \jacobianOrb\
$\mathbf{\jMorb}$.
    }

    \HLpost {2021-12-26} {
I think \refeqs{LC21PerViv2.1b}{LC21PerViv2.1a} should be written as:
\bea
\coord_{\zeit+1} &=& \coord_{\zeit} + p_{\zeit+1} \qquad  (\mbox{mod}\;1), \\
p_{\zeit+1} &=& p_{\zeit} + P(\coord_{\zeit}) \,.
\eea
Otherwise the angle of the rotor $\coord$ is not constrained to $[0,1)$.
\\Predrag: you are right, corrected.
}

\end{description}


%%%%%%%%%%%%%%%%%%%%%%%%%%%%%%%%%%%%%%%%%%%%%%%%%%%%%%%%%%%%%%%%%%%%%%%%
\bigskip\bigskip

\noindent
Note to Predrag - send this paper to
Vladimir Rosenhaus  <vladr@kitp.ucsb.edu>,
Xiangyu Cao <xiangyu.cao08@gmail.com>,
George Savvidy, ``Demokritos'', Athens,
and
David Berenstein <dberens@physics.ucsb.edu>
