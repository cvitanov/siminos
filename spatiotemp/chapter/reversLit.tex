% siminos/spatiotemp/chapter/reversLit.tex
% $Author: predrag $ $Date: 2021-12-22 18:29:33 -0500 (Wed, 22 Dec 2021) $

\section{Time reversal literature}
\label{sect:reverLit}
\renewcommand{\ssp}{\ensuremath{\phi}}             % lattice site field
\renewcommand{\Ssym}[1]{{\ensuremath{m_{#1}}}}    % Boris
\renewcommand{\Xx}{\ensuremath{\mathsf{\Phi}}}      % kittens lattice field

%\newpage
\subsection{Grava \etal\ 2021 paper {GKMM21}}
\label{sect:GKMM21}
% [2021-03-20 Predrag]
Notes on
Grava, Kriecherbauer,  Mazzuca and McLaughlin\rf{GKMM21}
{\em Correlation functions for a chain of short range oscillators},
\arXiv{2010.09612}:
\bigskip

[C]onsider a system of $N=2M+1$ particles interacting with a short range
harmonic potential with 	 Hamiltonian of the form
	\begin{equation}
	\label{GKMM21:H0}
H=\sum_{j=0}^{N-1}\frac{p_j^2}{2}+\sum_{s=1}^m\frac{\kappa_s}{2}\sum_{j=0}^{N-1}(q_j-q_{j+s})^2\,,
\end{equation}
and
Hamiltonian density
\[
e_j=\frac{p_j^2}{2}+\frac{1}{2} \Big(\sum_{s=1}^m \tau_s (q_{j+s}-q_j)
\Big)^2
\,,
\]
local in the variables $({\bf{p}},{\bf{q}})$ for fixed $m$.  If we let
$N\to\infty$, the quantity $e_j$ involves a finite number of physical
variables  $({\bf{p}},{\bf{q}})$.
We always take periodic boundary conditions,
the indices $j$ are taken from $\integers / N\integers$ and therefore
\[
q_{N+j}=q_j,\quad p_{N+j}=p_j
\]
holds for all $j$.

The relative shift \tilt{}  boundary condition $q_{N+1}=q_1+\tilt{}$ can
be also be considered  (see e.g.~\refref{Spohn14}). The periodic boundary
condition is recovered by  change  of coordinates $q_j\to
q_j-\frac{\tilt{}}{N}(j-1)$.

The coefficients $\tau_s$ are the entries of the circulant localized
square root $T$ of the matrix  $A$ by which we mean a solution of the
equation \refeq{GKMM21:eq:T0} of the form \refeq{GKMM21:form:T}.

The  Hamiltonian \refeq{GKMM21:H0} can be rewritten in the form
			\begin{equation}
	\label{GKMM21:eq:A}
	H({\bf{p}},{\bf{q}}) := \frac{1}{2} \langle{\bf{p}}, {\bf{p}}\rangle +\frac{1}{2}\langle {\bf{q}},A {\bf{q}}\rangle,
	\end{equation}
	where ${\bf{p}}=(p_0,\dots,p_{N-1}) $, ${\bf{q}}=(q_0,\dots,q_{N-1}),$  $\langle\,.,\,.\rangle$ denotes the standard scalar product in $\R^N$ and
	where $A\in\mbox{Mat}(N,\R)$ is a positive semidefinite  symmetric   circulant matrix generated by the vector  ${\bf{a}}=(a_0,\dots,a_{N-1})$ namely $A_{kj}=a_{(j-k)\mbox{\tiny {mod $N$}}}$ or
	\begin{equation}
	\label{GKMM21:A}
			A = {\begin{bmatrix}
				a_{0}&  a_{1}&\dots & a_{N-2}& a_{N-1}
				\\
				a_{N-1}& a_{0}& a_{1}&& a_{N-2}
				\\
				\vdots & a_{N-1}& a_{0}&\ddots &\vdots
				\\
				a_{2}&&\ddots &\ddots & a_{1}
				\\
				a_{1}& a_{2}&\dots & a_{N-1}& a_{0}
				\\
				\end{bmatrix}}\, ,
		\end{equation}
The harmonic oscillator with  only  nearest neighbour interactions  is recovered by choosing
\[
a_0=2\kappa_1,\quad a_1=a_{N-1}=-\kappa_1,
\]
and the remaining coefficients  are set to zero.

The equations of motion  for the Hamiltonian $H$  take the form
\beq
\dfrac{d^2}{dt^2}q_j=\sum_{s=1}^m \kappa_s (q_{j+s}-2q_j+q_{j-s}),\quad j\in\integers / N\integers.
\ee{GKMM21:Lagr}
{
The integration is obtained  by studying the dynamics in Fourier space.
%  (see e.g. \cite{Lukkarinen}).
[...]
Following the standard procedure in the case of
nearest neighbour interactions we replace the vector of position ${\bf{q}}$ by
a new variable ${\bf{r}}$ so that the Hamiltonian takes the form
 \[
H=\frac{1}{2}\langle {\bf{p}},{\bf{p}}\rangle+\frac{1}{2}\langle{\bf{r}},{\bf{r}}\rangle.
\]
Such a change of variables may be achieved by any linear transformation
\begin{equation}
\label{GKMM21:r}
{\bf{r}}=T{\bf{q}},
\end{equation}
with an $N\times N$ matrix $T$ that satisfies
\begin{equation}
	\label{GKMM21:eq:T0}
		A = T^\intercal T,
	\end{equation}
where $ T^\intercal$ denotes the transpose of $T$.

In the case of nearest neighbour interactions one may choose
\[
r_j = \sqrt{\kappa_1} (q_{j+1}-q_j)
\]
corresponding to a circulant matrix $T$
generated by the vector
\[
\boldsymbol{\tau}=\sqrt{\kappa_1}(-1,1,0,\dots,0)
\,.
\]
We show that short range interactions given by matrices $A$ of the
form~\refeq{GKMM21:A} also admit such a {\em localized square root}. More
precisely, there exists a circulant $N\times N$ matrix $T$ of the form
\beq\label{GKMM21:form:T}
			T ={\begin{bmatrix}
				\tau_0&  \tau_{1}&\dots&  \tau_{m}&0 &\dots&0	\\			
				0& \tau_{0}& \tau_{1}&\dots&\tau_m&0&				\\
				&\ddots & \ddots& \ddots &\ddots &&\\
%				&&&&&&&&&\\
%	&&&&&&&&&\\
%	&&&&&&&&&\\
%	&&&&&&&&&\\
%	&&&&&&&&&\\
                                 \tau_m&0&\ddots & \ddots& \ddots &\ddots &\\
                                 				&\ddots & \ddots& \ddots &\ddots & \ddots&\ddots\\
				\tau_2&\dots&\tau_m&0&\dots&\tau_0&\tau_1\\
				\tau_1&\tau_2&\dots&\tau_m &0&0&\tau_{0}
				\\
				\end{bmatrix}}
\,.
\eeq
that satisfies \refeq{GKMM21:eq:T0}. The crucial point here is that $T$ is not the standard (symmetric) square root of the positive semidefinite matrix $A$ but a localized version generated by some vector $\boldsymbol{\tau}$
with zero entries everywhere, except possibly in the first $m+1$ components.
[...]
Note that $\boldsymbol{1}=(1,\ldots, 1)^\intercal$ satisfies $T \boldsymbol{1}=0$ since $\langle \boldsymbol{1}, A\boldsymbol{1}\rangle =0$. This implies
$$\sum_{s=0}^m\tau_s=0\,, \quad
r_j= \sum_{s=1}^{m} \tau_s (q_{j+s}-q_j)\,  \quad \text{and} \quad \sum_{j=0}^{N-1}r_j= (1,\ldots, 1) T {\bf{q}} = 0.$$
The local energy $e_j$ takes the form
\[
e_j=\dfrac{1}{2}p_j^2+\dfrac{1}{2}r_j^2\,.
\]
Due to the spatial translation invariance of the Hamiltonian
\[
H({\bf{p}},{\bf{q}}) = H({\bf{p}},{\bf{q}}+\lambda \boldsymbol{1})
\,,
\] $\lambda \in \R$,
that corresponds to the conservation of total momentum, we reduce the
Hamiltonian system by one degree of freedom, with the reduced phase space
		\begin{equation}
	\label{GKMM21:eq:phase_space}
	{\mathcal M}
:=
\left\{({\bf{p}},{\bf{q}})\in \R^{N}\times \R^{N}
\, : \, \sum_{k=0}^{N-1} p_k =0\, ;\, \sum_{k=0}^{N-1} q_k = 0 \right\}
\,.
	\end{equation}

[...]
the dispersion relation  $|\omega(k)|$  for the harmonic oscillator with
short range interaction in the limit $N\to\infty$ obtaining
\beq
	\label{GKMM21:dispersion0}
	f(k)=|\omega(k)|
=\sqrt{2\sum_{\ell=1}^m\kappa_\ell\left(1-\cos(2\pi k\ell)\right)}
\,,
\eeq
[...]
we show that the evolution equations for the generalized position,
momentum can be written in the form of  conservation laws which have a
potential function.
 For the case of the harmonic oscillator with nearest neighbour interaction, we show that this function is  a Gaussian random variable
and determine the leading order behaviour of its variance as $t\to\infty$.

[...] some notation.   First of all,  a matrix $A$ of the form
\refeq{GKMM21:A} with ${\bf{a}}\in\R^N$ is called a circulant matrix generated by the vector ${\bf{a}}$.
	\paragraph{$m$-physical vector and half-$m$-physical vector}
	\label{GKMM21:def:ph}
			Fix $m \in {\mathbb N}$.
			For any odd $ N > 2m$, a   vector $\tilde{\bx}\in \R^{N}$  is said to be  {\em $m$-physical} generated by $\bx=(x_0,x_1,\dots, x_m)\in \R^{m +1}$ if $x_0=-2\sum_{s=1}^mx_s$ and 				
			\begin{align}
            \label{GKMM21:HalfVector}
				\tilde{x}_0 = & x_0 \, ,\\
				\tilde{x}_1= &\tilde{x}_{N-1} = x_1<0 ,\;\;\tilde{x}_m= \tilde{x}_{N-m} = x_m<0,\\
			\tilde{x}_k= &\tilde{x}_{N-k} = x_k \leq 0, \mbox{ for  $1< k <  m$},\\
			 \tilde{x}_{k} =& 0, \mbox{  otherwise,}
						\end{align}
	while the  vector  $\tilde{\bx}\in \R^{N}$  is called  {\em half-$m$-physical} generated by ${\bf{y}}\in  \R^{m +1}$ if  $y_0=-\sum_{s=1}^my_s$  and	\begin{align*}
		\tilde{x}_k=  & y_k,  \mbox{ for  $0 \leq k \leq m$   } \\
		\tilde{x}_k=&0,  \mbox{ for  $m< k \leq N-1$.   }
		\end{align*}
Following  the proof of a lemma by Fej\'er and Riesz, one can show that a
circulant symmetric matrix $A$ of the form \refeq{GKMM21:eq:A}  generated
by  a $m$-physical vector ${\bf{a}}$ always has a circulant localized
square root
%{\em Golub - Toepliz} \cite{Golubov}, the next proposition shows that
% a circulant symmetric matrix $A$ of the form \refeq{GKMM21:eq:A}  generated by
% a $m$-\th{physical} vector ${\bf{a}}$ always has a circulant localized square root
$T$  that is generated by a half-$m$ physical vector ${\boldsymbol{\tau}}$.

\paragraph{Fej\'er and Riesz~\cite[pg.~117 f]{RieSzo55} lemma}
\label{Lemma:FejerRiesz}
asserts that every positive
trigonometric polynomial can be represented by the square of the absolute
value of another trigonometric polynomial whose coefficients are, in
general, complex.
\bigskip

Fix $m \in {\mathbb N}$. Let the circulant matrix $A$ be generated by an
$m$-physical vector ${\bf{a}}$, then there exist a circulant matrix $T$
generated by an half-$m$-physical vector ${\boldsymbol{\tau}}$ such that:
	\begin{equation}
	\label{GKMM21:eq:T}
		A = T^\intercal T\,.
	\end{equation}
Moreover, we can choose ${\boldsymbol{\tau}}$ such that $\sum_{s=1}^m s
\tau_s > 0$. Then one has $\sum_{s=1}^m s
\tau_s=\sqrt{\sum_{s=1}^ms^2\kappa_s}$.

For  example,  if  we consider $m=1$, and $a_0=2\kappa_1$ and
$a_1=a_{N-1}=-\kappa_1$.  The matrix $T$ is generated by the vector
${\boldsymbol{\tau}}=(\tau_0,\tau_1)$ with $\tau_0=-\sqrt{\kappa_1}$ and
$\tau_1= \sqrt{\kappa_1}$.
When $m=2$ and  $a_0=2\kappa_1+2\kappa_2$,  $a_1=a_{N-1}=-\kappa_1$,  $a_2=a_{N-2}=-\kappa_2$. The matrix $T$ is generated by the vector ${\boldsymbol{\tau}}=(\tau_0,\tau_1,\tau_2)$
with
\begin{align*}
&\tau_0=- \frac{\sqrt{\kappa_1}}{2}-\frac{1}{2}\sqrt{\kappa_1+4\kappa_2}, \;\;\tau_1=\sqrt{\kappa_1},\\
&\tau_2=-\frac{\sqrt{\kappa_1}}{2}+\frac{1}{2}\sqrt{\kappa_1+4\kappa_2},\end{align*}
so that the quantities $r_j$ are defined as
\[
r_j=\tau_1(q_{j+1}-q_j)+\tau_2 (q_{j+2}-q_j)\,,\quad j\in\integers / N\integers \,.
\]
[...]
The Hamiltonian $H({\bf{p}},{\bf{q}})$	represents clearly an integrable system  that can be integrated passing through Fourier transform.
Let ${\mathcal F}$ be the discrete Fourier transform with entries  ${\mathcal F}_{j,k}:=
\frac{1}{\sqrt{N}} e^{- 2\im \pi j k/N}$ with $j,k=0,\dots, N-1$. It is immediate to verify that
	\begin{equation}
	\label{GKMM21:DHT_prop}
	{\mathcal F}^{-1} = \bar{{\mathcal F}} \qquad {\mathcal F}^\intercal = {\mathcal F}.
	\end{equation}
	Thanks to the above  properties,  the transformation defined by
	\begin{equation}
	\label{GKMM21:eq:F_variables}
	(\hat{\bf{p}}, \hat{\bf{q}}) = (\bar{\mathcal F} p, {\mathcal F} q)
	\end{equation}
	is canonical.  Furthermore
$\bar{\hat{\bf{p}}}_j = \hat{\bf{p}}_{N-j}$  and $\bar{\hat{\bf{q}}}_j = \hat{\bf{q}}_{N-j}$,
for $j=1,\dots,N-1$, while  $\hat{\bf{p}}_0$ and  $\hat{\bf{q}}_0$ are real variables.
The matrices  $T$ and $A$ are  circulant matrices and so they are reduced to diagonal form by
${\mathcal F}$:
\[
{\mathcal F} A {\mathcal F}^{-1}
= {\mathcal F} T^\intercal T{\mathcal F}^{-1}
=\overline{({\mathcal F} T{\mathcal F}^{-1})}^\intercal ({\mathcal F} T{\mathcal F}^{-1})
\;.
\]
Let  $\omega_j$ denote the eigenvalues of the matrix $T$ ordered so that
${\mathcal F} T{\mathcal F}^{-1}=$ diag$(\omega_j)$. Then  $|\omega_j|^2$
are the (non negative) eigenvalues of the matrix $A$ and
\begin{equation}
\label{GKMM21:omega}
|\omega_j|^2=\sqrt{N}( \overline{ {\mathcal F}} \tilde{{\bf{a}}})_j,\quad \omega_j=\sqrt{N}( \overline{{\mathcal F}}\tilde{ \boldsymbol{\tau}})_j, \quad j=0,\dots, N-1,
\end{equation}
%\todo{T. We need to say that $\omega_0=0$}
where  $\tilde{{\bf{a}}}$ is  the $m$-physical vector generated by
${\bf{a}}$ and  $ \tilde{\boldsymbol{\tau}}$ is the half $m$-physical
vector generated by $ \boldsymbol{\tau}$.
% according to Definition~\ref{def:ph}.
It follows that
  \begin{equation}
  \label{GKMM21:sym_omega}
\omega_0=0,\quad   \omega_j=\overline{\omega}_{N-j}\,,\quad j=1,\dots,N-1,
  \end{equation}
  which implies $|\omega_{j}|^2=|\omega_{N-j}|^2$, $j=1,\dots,N-1$.

\subsubsection{Circulant hierarchy of integrals}
	In this section we construct a complete set of conserved quantities that have local densities.
	 The harmonic oscillator with short range interaction is clearly an integrable system. A  set of integrals of motion is given
	  by the harmonic oscillators in each of  the  Fourier variables:
$\hat{H}_j=\frac{1}{2}\left( |\hat{\bf{p}}_j| ^2 + |\omega_j|^2 |\hat{\bf{q}}_j|^2  \right)$,
$j=0,\dots \frac{N-1}{2}$.
However, when written in the physical variables ${\bf{p}}$ and $ {\bf{q}}$,  the quantities
$$\hat{H}_j=
 \frac{1}{2}\sum_{k,l=0}^{N-1} {\mathcal F}_{j,k} \overline{{\mathcal F}_{j,l} }( p_k  p_l+|\omega_j|^2 q_k q_l)$$  depend on all components of the physical variables.
We now construct integrals of motion each having a density that involves only a limited number of components of the physical variables and this number only depends on the range~$m$ of interaction.

	For this purpose we denote by $\{\be_k\}_{k=0}^{N-1}$  the canonical  basis in $\R^N$.
	\paragraph{Local conserved quantities}
		\label{GKMM21:thm:first}
		Let us consider the Hamiltonian 		
		\begin{equation}
		\label{GKMM21:eq:general_sys}
			H({\bf{p}},{\bf{q}}) = \frac{1}{2} {\bf{p}}^\intercal {\bf{p}} + \frac{1}{2}{\bf{q}}^\intercal A {\bf{q}}\,,
		\end{equation}
		with the symmetric circulant matrix $A$ as in \refeq{GKMM21:eq:A}, \refeq{GKMM21:A}.
	 Define the matrices $\{G_k\}_{k=1}^{M}$ to be the symmetric circulant matrix generated by  the vector  $\frac{1}{2}(\be_k + \be_{N - k})$
	  and $\{S_k\}_{k=1}^{M}$ to be the  antisymmetric  circulant matrix generated by the vector $\frac{1}{2}(\be_k - \be_{N - k})$.
	  Then  the family of Hamiltonians defined as
			\begin{align}
			\label{GKMM21:eq:even_hamiltonian}
			H_k({\bf{p}},{\bf{q}}) =& \frac{1}{2} {\bf{p}}^\intercal G_k {\bf{p}} + \frac{1}{2}{\bf{q}}^\intercal T^\intercal G_kT {\bf{q}}=\frac{1}{2}\sum_{j=0}^{N-1}[p_jp_{j+k}+r_jr_{j+k}]\, ,\\
			\label{GKMM21:eq:odd_hamiltonian}
			H_{k+ \frac{N-1}{2}}({\bf{p}},{\bf{q}}) =&  {\bf{p}}^\intercal T^\intercal S_k T {\bf{q}} =\frac{1}{2}\sum_{j=0}^{N-1}\left[\left(\sum_{\ell=0}^m\tau_\ell p_{j+\ell}\right)(r_{j+k}-r_{j-k})\right]\,,\, \quad k=1,\dots,\frac{N-1}{2}
			\end{align}
together with $H_0:=H$ forms a complete family $(H_j)_{0\leq j \leq N-1}$
of integrals of motion that, moreover, is in involution.
				% 	 the set  $$\fH = \{H\} \, \bigcup_{k=1}^{N}\left(\{H^{G_k}\} \, \bigcup\, \{H^{S_k}\} \right)$$ is a complete set of integral of motion for $H$. We will call it {\em circulant hierarchy}.
[...]
Now we introduce the local densities corresponding to the just defined integrals of motion
\beq
\label{GKMM21:symAntisym}
e_j^{(k)}=
\begin{cases}
& \frac{1}{2}\left(p_jp_{j+k} +  r_jr_{j+k}\right)
\, , \mbox{for $k=1, \dots,\frac{N-1}{2}$} \\
&\left(\sum_{l=0}^m \tau_l p_{j+l}\right) \left(r_{j+k} - r_{j-k}\right)
,\;\; \mbox{for $k=\frac{N+1}{2}, \dots,N$}
\,.	
\end{cases}
\eeq	
[...]

\subsubsection{Nonlinear regime}\label{GKMM21:sect3.2}
In this section we consider a nonlinear perturbation of the harmonic
oscillators with short range interactions of the form

\begin{equation}
\label{GKMM21:HN}
H({\bf{p}},{\bf{q}})=
\sum_{j=0}^{N-1}\frac{p_j^2}{2}+\sum_{s=1}^m\kappa_s\left(\frac{1}{2}\sum_{j=0}^{N-1}(q_j-q_{j+s})^2
+ \frac{\chi}{3}\sum_{j=0}^{N-1}(q_j-q_{j+s})^3 + \frac{\gamma }{4}\sum_{j=0}^{N-1}(q_j-q_{j+s})^4\right)
\,.
 \end{equation}

We consider examples
%~\ref{example1} and Example~\ref{example2}
with different strengths of nonlinearity  namely
\[
\mbox{  $m=2, \, \kappa_1 = 1,$   $\kappa_2 = \frac{1}{4}$,}\;\;
\begin{cases}
	&\mbox{$\chi=0.01$ and $\gamma=0.001$}\\
	&\mbox{$\chi=0.1$ and $\gamma=0.01$ }
\end{cases}
\]
\[
\mbox{  $m=3, \, \kappa_1 = 1,$
$\kappa_2 = \frac{1}{8}$, $\kappa_2 = \frac{7}{72}$, }\;\;
\begin{cases}
	&\mbox{$\chi=0.01$ and $\gamma=0.001$}\\
	&\mbox{$\chi=0.1$ and $\gamma=0.01$ }
\end{cases}\, .
\]


\subsection{Baake \etal\ 2008 paper {BaRoWe08}}
\label{sect:BaRoWe08}

\begin{description}

 \item[2017-09-27, 2021-02-03 Predrag] reading
 Baake, Roberts andWeiss\rf{BaRoWe08}
{\em Periodic orbits of linear endomorphisms on the 2-torus and its lattices}
\arXiv{0808.3489}.
 \item[2021-02-03 Predrag] Summary
 \begin{enumerate}
   \item
The main result is the third matrix invariant: the `gmc' that fixes the
conjugacy class of a given lattice in an invariant way, unlike the
Hermite normal form \refeq{Holmin12-Hermite} that breaks the `\spt'
symmetry; mention and cite in CL18\rf{CL18}, even if we do not use it.
   \item
They do counting for the golden (Fibonacci\rf{BaNeRo13}) cat map \refeq{goldenCat}, see
\reftab{BaRoWe08fib-tab} and \refeq{zetasqrt-N}, as the simplest example;
we do not need to cite their counting in CL18\rf{CL18}, cite  1997
\refsect{sect:BaHePl97}~{\em Baake,
Hermisson and Pleasants}\rf{BaHePl97} instead. (I'm somewhat sure that
name `golden cat'
is not already in 1967 Smale\rf{smale}, or 1995 Katok and
Hasselblatt\rf{Katok95}. Perhaps better to call it ``Fibonacci''\rf{BaNeRo13}?)
    There is no mention that this is a time-reversal reduction of
    the $s=3$ cat map in  Baake \etal\rf{BaRoWe08}. Perhaps
    Katok and Hasselblatt\rf{Katok95} mention that?
   \item
They do not mention any time-reversal symmetry reduction connection to the
{\lattstate}s and %{prime}
orbit counts for the ${s}=3$ cat map
\reftab{tab:catMapN_n-s=3}; do not cite them for that.
 \end{enumerate}

\end{description}

% Moved to here from  \emph{catMap.tex}, might be relevant to 2\dmn\
% lattices orbit counting, but then went back.

Their focus on the relation between global and local aspects and between the
dynamical zeta function on the torus and its analogue on finite lattices. The
situation on the lattices, up to local conjugacy, is completely determined by
the determinant, the trace and a third invariant of the matrix defining the toral
endomorphism.

\PCedit{In introduction they refer to much literature on cat maps on
lattices, and I've not read much of it.}


[...] the                                                     \toCB
system $(\varOmega,T)$ is called \emph{chaotic} when the {\po}s of $T$ are dense in $\varOmega$ and when also a dense orbit
exists, see Banks \etal\rf{BBCDS92}
\textit{On Devaney's definition of chaos} for details.
Knowledge of the {\po}s can be used to detect characteristic properties of $T$. For
example, if $T'$ represents another continuous mapping of $\varOmega$,
then a necessary condition for $T$ and $T'$ to be topologically
conjugate is that they share the same number of periodic points of
each period.

[...] endomorphisms of the $2$-torus, represented by the action (mod $1$)
of an integer matrix $M\in{Mat} (2,\integers)$ on $\mathbb{T}^2 \simeq \RR^2/\integers^2$. A
well-studied subclass consists of the toral automorphisms, represented by
elements of the group $GL (2,\integers)$, being the subgroup of matrices with
determinant $\pm 1$ within the ring ${Mat} (2,\integers)$. Particularly
important are the hyperbolic ones (meaning that no eigenvalue is on the
unit circle), which are often called \emph{cat maps}.  Since these are
expansive, all periodic point counts are finite.  Hyperbolic toral
automorphisms are also topologically mixing and intrinsically ergodic,
see \refrefs{Katok95,Walt82}. By the Bowen-Sinai theorem, this has the
consequence that the integral of a continuous function over $\mathbb{T}^2$
equals its average value over the points fixed by $M^m$ in the limit as
$m \rightarrow \infty$.

The topological entropy of a hyperbolic toral automorphism $M\in GL
(2,\integers)$ is given by $\log\,\lvert \lambda_{\mathrm{max}}\rvert$,
where $\lambda_{\mathrm{max}}$ is the eigenvalue of $M$ with modulus
$>1$.  This is also the metric (or Kolmogorov-Sinai) entropy of $M$,
and completely determines the dynamics up to metric isomorphism,
compare \refref{AdWei67}.  This does not imply topological conjugacy though,
and one important difference emerges from the {\po}s, which
live on a set of measure $0$. Indeed, on $\mathbb{T}^2$, it is well-known
that the {\po}s of hyperbolic linear endomorphisms lie on the
invariant lattices given by the sets of rational points with a given
denominator $n$, also known as $n$-division points.  One of our main themes in
this paper is the interplay between the {\po} statistics on a
certain lattice (which we call \emph{local statistics}) versus
{\po} statistics on the union of all lattices (which we call
\emph{global statistics}). What determines when two cat maps have the
same global statistics? What determines when two cat maps have the
same local statistics on a certain lattice or on all lattices?

The time of recurrence of a hyperbolic $M \in GL (2,\integers)$ on
the toral rational lattice with denominator $n$ is denoted by ${per}{\hspace{0.5pt}}
(M,n)$, where this is the least common multiple of the periods present
on the $n$-division points.

[...] for symmetries or (time)
reversing symmetries of a cat map, these being automorphisms of the
torus that commute with the cat map, respectively conjugate it into
its inverse.

[...] there has been quite some
interest in dealing with this challenge of so-called pseudo-symmetries
of quantum maps that are not quantisations of symmetries of the cat
map on the torus, but instead are manifestations of local symmetries
of the cat map restricted to some lattice\rf{KeaMez00,KurRud00}. % \cite{KM,KR,DW}.

Conjugacy of $GL (2,\integers)$
matrices is another topic that has arisen in a broad variety of
contexts and has been considered by many
% \cite{T,Rade,ATW}
%   \rf{T} O.~Taussky,
%\textit{Introduction into connections between algebraic number
%theory and integral matrices}, appendix to:\
%H.\ Cohn, \textit{A Classcial Invitation to Algebraic Numbers
%and Class Fields}, Springer, New York (1978), pp.~305--321.
%    \rf{Rade} H.~Rademacher,
%\textit{Zur Theorie der Dedekindschen Summen},
%Math.\ Z.\  \textbf{63} (1956) 445--463;
%see also \texttt{MR$\,$0079615} by H.\thinspace D.\
%Kloosterman for a short summary.
%    \rf{ATW} R.~Adler, C.~Tresser and P.\thinspace A.~ Worfolk,
%\textit{Topological conjugacy of linear endomorphisms of the $2$-torus},
%Trans.\ AMS \textbf{349} (1997) 1633--1652.
Conjugacy is determined by a triple of
invariants, namely the determinant, the trace and one other invariant
which can be related to ideal classes, representation by binary
quadratic forms or topological properties. % \cite{ATW}.
Conjugacy in
$GL (2,\integers)$ can also be completely decided by using the amalgamated
free product structure of ${PGL} (2,\integers)$, which attaches a finite
sequence of integers to each element which corresponds to its normal
form as a word in the generators of the amalgamated free product\rf{BaaRob97}.

There are various ways
of deciding $GL(2,\integers)$-conjugacy, amounting to exploiting a third
and final conjugacy invariant.

{\bf Result of this paper:}
The \emph{matrix gcd} is a key
quantity. It is preserved by $GL (2,\integers)$ conjugacy, so it provides a
quick tool to see that two $GL (2,\integers)$ matrices with different
matrix gcd are not conjugate on the torus.  If two
integer matrices share the same determinant, trace and matrix gcd they
are linearly conjugate on all rational lattices of the torus. As an
illustration of this result, consider
cat maps and time-reversal symmetry. The fact that any $M \in {SL}
(2,\integers)$ shares determinant, trace and matrix gcd with $M^{-1}$ means
that the two matrices are conjugate on \emph{all} rational lattices,
though not necessarily by matrices that derive from one and the same
matrix on the torus.

Consider a compact space $\varOmega$ and some (continuous) mapping $T$
of $\varOmega$ into itself. Let ${Fix}_m (T) := \{ x\in X \mid T^m x =
x\}$ be the set of fixed points of $T^m$.  Of particular interest are
the \emph{fixed point counts}, defined as
\begin{equation} \label{def-a}
     a_m \, := \, {card{\hspace{0.5pt}}} \{x\in \varOmega \mid T^m x=x\}
     \, = \,  {card{\hspace{0.5pt}}} ({Fix}_m (T)){\hspace{0.5pt}}
\,.
\end{equation}
The quantity $a_m$ has the disadvantage that one keeps recounting the
contributions $a^{}_{\ell}$ for all $\ell | m$. Clearly, the fixed
points of \emph{genuine} order $m$ permit a partition into disjoint
cycles, each of length $m$. If $c_m$ is the number of such cycles, one
thus has the relation
\begin{equation} \label{a-from-c}
   a_m \, = \, \sum_{d{\hspace{0.5pt}} | m} d\, c_d {\hspace{0.5pt}} .
\end{equation}
An application of a standard inclusion-exclusion argument, here
by means of the M\"obius inversion formula from elementary
number theory, results in the converse identity,
\begin{equation} \label{c-from-a}
    c_m \, = \, \frac{1}{m} \sum_{d{\hspace{0.5pt}} |  m}
   \mu \big( \tfrac{m}{d}\big) {\hspace{0.5pt}}  a_d{\hspace{0.5pt}} ,
\end{equation}
where $\mu(k)$ is the M\"obius function.
% compare \cite{Hasse,PW} and references therein for details.
%    \rf{Hasse} H.~Hasse,
%\textit{Number Theory}, Springer, Berlin (1980).
%    \rf{PW} Y.~Puri and T.~Ward,
%\textit{Arithmetic and growth of periodic orbits},
%J.\ Integer Sequences  \textbf{4} (2001) 01.2.1

[...] a toral endomorphism $M\in {Mat}(2,\integers)$ is
\emph{hyperbolic} if it has no eigenvalue on the unit circle
$\mathbb{S}^1$. The standard $2$-torus is $\mathbb{T}^2 \simeq \RR^2/
\integers^2$, where $\integers^2$ is the square lattice in the plane. It is a
compact Abelian group, which can be written as $\mathbb{T}^2 := [0,1)^2$,
with addition defined mod $1$.

[...] the abbreviation $\integers_n = \integers/n\integers$ for the
finite integer ring mod $n$, and $\integers_n^{\times}=\{1\le k\le n \mid
gcd (k,n) = 1\}$ for its unit group.

\PCedit{Some `obvious' number theory defines gcd, but I have not put in
the effort needed to understand it.}

For counting orbits, this
might be useful:

Let $M\in\mbox{Mat}(2,\complex)$ be a non-singular matrix, with $D:=\det(M)\neq
0$ and $T:=\tr(M)$. Define a two-sided sequence of (possibly
complex) numbers $p_m$ by the initial conditions $p^{}_{-1} = -1/D$
and $p^{}_{0}=0$ together with the recursion
\begin{equation} \label{recursion}
  \begin{split}
     p_{m+1} & \,  = \, T p_{m} - D p_{m-1} ,
      \quad \text{for $m\ge 0$,}   \\
     p_{m-1} & \,  = \, \frac{1}{D} (T p_{m} - p_{m+1}) ,
      \quad \text{for $m\le -1$.}
  \end{split}
\end{equation}
This way, as $D\neq 0$, $p_m$ is uniquely defined for all $m\in\integers$.
Note that the sequence $(p_m)_{m\in\integers}$ depends only on the
determinant and the trace of $M$. When $M\in\mbox{Mat}(2,\integers)$, one has
$p_m\in\rationals$, and $p_m\in\integers$ for $m\ge 0$. When $M\in\mbox{GL}(2,\integers)$, all
$p_m$ are integers.

Note an interesting property, which follows from a
straight-forward induction argument (in two directions):

%\begin{fact} %\label{quadratic-relation}
  The two-sided sequence of rational numbers defined
  by the recursion $\refeq{recursion}$ satisfies the
  relation
\beq
p_{m}^{2} - p_{m+1}^{} p_{m-1}^{} = D^{m-1}_{}
\,,
\ee{BaRoWe08(8a)}
  for all $m\in\integers$.  %\qed
%\end{fact}

To deal with combinatorial quantities such as the fixed point counts
$a_m$, it is advantageous to employ generating functions.
Here, the concept of a
\emph{dynamical zeta function} is usually most
appropriate.  Consequently, given a matrix $M\in{Mat} (2,\integers)$, we set
\begin{equation} \label{define-zeta}
   \zeta^{}_{M} (t) \, := \; \exp\Bigl( \sum_{m = 1}^{\infty}
        \frac{a_m}{m}\, t^m \Bigr),
\end{equation}
where, from now on, $a_m := {card{\hspace{0.5pt}}} \{ x\in{Fix}_m (M)\mid x \text{ is
isolated} \}$ is the number of \emph{isolated} fixed points of $M^m$.

The ordinary power series generating function for the counts $a_m$ can
be calculated from $\zeta^{}_{M} (t)$ as $\sum_{m\ge 1} a_m{\hspace{0.5pt}} t^m =
t{\hspace{0.5pt}}\frac{\dd}{\dd t} \log \big(\zeta^{}_{M} (t)\big)$. The
significance of the formulation used in Eq.~\refeq{define-zeta}
follows from the fact that it has a unique Euler product decomposition
as
\begin{equation} \label{euler1}
   \frac{1}{\zeta^{}_{M} (t)} \;\, = \!
   \prod_{\text{cycles } \mathcal{C}}
   \big( 1- t^{\lvert \mathcal{C}\rvert}\big) \; = \,
   \prod_{m\ge 1}\, \big(1-t^m \big)^{c_m},
\end{equation}
where $\lvert \mathcal{C}\rvert$ stands for the length of the cycle
$\mathcal{C}$ and $c_m$ is now the number of \emph{isolated} cycles
of $M$ on $\mathbb{T}^2$ of length $m$, as determined from Formula
\refeq{c-from-a}. Consequently, the role of cycles in dynamics is
similar to that of primes in elementary number theory.

The dynamical zeta function, a special case of which was
also given in \refref{EspIso95}.

% \begin{prop} \label{zeta1}
   Let\/ $M\in GL (2,\integers)$ be hyperbolic, and define\/
   $\sigma=sgn \bigl(\tr (M)\bigr)$. Then,
   with the coefficients\/ $a_m = {card{\hspace{0.5pt}}} \{ x\in\mathbb{T}^2\mid M^m x=x
   \mbox{ \rm (mod $1$)}\}$, the dynamical zeta function\/
   \refeq{define-zeta} of\/  $M$ on\/ $\mathbb{T}^2$ is given by
\[
    \zeta^{}_{M} (t)  \, = \,
    \frac{(1-\sigma t) (1-\sigma t{\hspace{0.5pt}} \det(M))}
    {\det (\unit-\sigma t M )}  \, = \,
    \frac{(1-\sigma{\hspace{0.5pt}} t) (1-\sigma \det(M)\, t)}
    {1-\lvert\tr (M)\rvert\, t+\det(M)\, t^2} {\hspace{0.5pt}} .
\]
   In particular, $\zeta^{}_{M} (t)$ is a rational
   function, with numerator and denominator in\/ $\integers[t]$. The
   denominator is a quadratic polynomial that is irreducible over
   $\integers$.  Its zero $t_{\rm min}$ closest to $0$ gives the radius of
   convergence of $\zeta^{}_{M} (t)$, as a power series around\/ $0$,
   via $r_{\rm c} = \lvert t_{\rm  min} \rvert$.

If $M$ is hyperbolic, the general formula for the $a_m$
\[
    a_m \, = \, \sigma^m \bigl( \tr (M^m)
    - (1 + \det(M)^m )\bigr)
\]
can be derived
by observing that the two eigenvalues of $A$ can be written as $\lambda$
and $\det(A)/\lambda$. For the detailed argument, one may assume
$\lvert\lambda\rvert>1$ and check the different cases. Note that a
hyperbolic toral automorphism is never of trace $0$.

The formula for the zeta function now follows from \refeq{define-zeta}
by inserting the expression for $a_m$.
% and using the relation \refeq{zeta-ruelle} together with the power series
% identity \refeq{log-series}.
The statement on the nature of the rational
function is then clear. With $M\in GL (2,\integers)$, the denominator only
factorises for $\tr (M)=0$, $\det(M)=-1$ or for $\tr (M)=\pm 2$,
$\det(M)=1$, both cases being impossible for hyperbolic matrices.


Two hyperbolic
$GL (2,\integers)$-matrices with the same trace and determinant
possess the same dynamical zeta function, hence the same fixed point
counts. The converse is slightly more subtle.

\emph{3.3. Generating functions on lattices}:
\PCedit{I tried reading this before, I tried on 2017-09-27 again, and on
2021-02-03 again, and I still do not get it.}

Consider a $2\!\times\! 2$-matrix
\begin{equation} \label{two-matrices}
  M \, = \, \begin{pmatrix} a & b \\ c & d \\
    \end{pmatrix}
\end{equation}
   If $M\in{Mat} (2,\integers)$, the quantity
\[
   mgcd (M) \, := \, gcd (b,c,d-a) {\hspace{0.5pt}} ,
\]
is called the \emph{matrix gcd} of $M$, or $mgcd$ for
short.  Here, we take the gcd to be a non-negative
integer, and set $mgcd(M)=0$ when $b=c=d-a=0$.
The last convention matches that of the ordinary gcd,
and is compatible with modular arithmetic.

For $M\in{Mat} (2,\integers)$, the following statements are equivalent:
\begin{itemize}
\item[{\rm (a)}] The matrix gcd satisfies $mgcd(M)=0$.
\item[{\rm (b)}] $M=k\unit$ for some $k\in\integers$.
\item[{\rm (c)}] The minimal polynomial of $M$ is of degree $1$.
\end{itemize}
Consequently, whenever $mgcd(M)=r\in\NN$, $M$ cannot be a multiple
of the identity, and its characteristic and minimal polynomials
coincide.

Most significantly, the matrix gcd satisfies the following invariance
property:

% \begin{lemma} \label{mat-gcd}
    If $M,M^{{\hspace{0.5pt}}\prime}\in{Mat} (2,\integers)$ are two integer matrices that
    are conjugate via a $GL (2,\integers)$-matrix, one has
    $mgcd (M^{{\hspace{0.5pt}}\prime}) = mgcd (M)$. In particular, the matrix gcd
    %of Definition~$\ref{def:mgcd}$
    is constant on the conjugacy classes
    of\/ $GL (2,\integers)$.

consider the integer matrices
\begin{equation} \label{mat-defs}
    M \, = \, \begin{pmatrix} a & b \\ c & d \end{pmatrix}
    \quad \text{and} \quad
    C \, = \, \begin{pmatrix} 0 & -D \\ 1 & T \end{pmatrix}
\end{equation}
with $D=\det (M)$ and $T=\tr (M)$. Here, $C$ is the standard
companion matrix for the characteristic polynomial
\begin{equation} \label{charpoly}
      x^2 - T{\hspace{0.5pt}} x + D
\end{equation}
of the matrix $M$.
[...]

\begin{description}

 \item[2017-09-27, 2021-02-03 Predrag] read superficially
Llibre and Neum{\"a}rker\rf{LliNeu15}
{\em Period sets of linear toral endomorphisms on{$T^2$}};
did not understand much.

\end{description}


\subsection{Baake \etal\ 1997 paper {BaHePl97}}
\label{sect:BaHePl97}

In 1997 Baake, Hermisson and Pleasants\rf{BaHePl97}
{\em The torus parametrization of quasiperiodic {LI}-classes}
\CBlibrary{BaHePl97}
refer to time-reversal as \emph{`inversion' symmetry}, and discuss
the golden (Fibonacci\rf{BaNeRo13}) cat map.
Before giving up on them, do have a look at their eq.~(18) zeta function
and Table~2. {\em Inflation orbit counts for 1D cut-and-project patterns
with inflation}, compare with \reftab{BaRoWe08fib-tab}; compare their
Table 4. with \reftab{tab:catMapN_n-s=3}.
Their eq.~(18) zeta function is not our Kim \etal\rf{KiLePa03}
\refeq{Ryu17eq:2.1}.
%
                                    \PCedit{Do cite in CL18\rf{CL18}!}

\paragraph{Sect.~2.3 {\em Symmetry}}
The only kind of point symmetry possible for 1D chains is mirror
symmetry, which we shall usually refer to as \emph{inversion symmetry} in
order to have the same terminology for all dimensions (`inversion'
meaning the isometry $x \to -x$). Inversion symmetric chains correspond
to points $\bf{t}$ on the torus with $\bf{t} = -\bf{t}$, i.e. $2\bf{t} =
0$. There are four such points
\beq
(0,0)\,\; (\frac{1}{2},0)\,\;  (0,\frac{1}{2})\,\;  (\frac{1}{2},\frac{1}{2})\,\;
\,,
\ee{BaHePl97(1)}
that form the discrete subgroup of ‘two-division points’ of $T^2$,
isomorphic to $\Cn{2}\times\Cn{2}$.

They count many inversion-symmetric patterns in various dimensions, but
I do not think any of that applies to \templatt\ or \catlatt.

\subsection{Baake 2018 paper {Baake18}}
\label{sect:Baake18}

Read this:\\
Michael Baake\rf{Baake18}
{\em A brief guide to reversing and extended symmetries of dynamical systems}
\arXiv{1803.06263}.


\subsection{Lamb and Roberts 1998 paper {lamb98}}
\label{sect:lamb98}

Lamb and Roberts\rf{lamb98}
{\em Time reversal symmetry in dynamical systems: {A} survey}
(1998)
is a very extensive compendium of references on reversibility.
Even though they touch upon discrete lattice settings
(the Frenkel-Kontorova
model\rf{AuAb90}) I see no place a reference to group-theoretic description of
$\Dn{\infty}$ lattices that we undertake in LC21\rf{LC21}.


Example 3.4.                          \toSect{sect:FrenkelKontorova}
Symmetric difference equations of the form
\beq
\ssp_{n+l} - f(\ssp_n) + \ssp_{n-1} = 0
\ee{lamb98(3.15)}
the Frenkel-Kontorova
model which is equivalent to the area-preserving
Chirikov-Taylor standard mapping.

Remarkably, \refeq{lamb98(3.15)} is not only reversible, but
the associate `time' mapping is also
area-preserving. Many area-preserving (symplectic)
mappings studied in the literature are reversible
(e.g. the well-studied area-preserving \HenonMap,
cf. Roberts and Quispel\rf{RobQui92} (1992) and references
therein).

    {\bf 2021-03-25 Predrag}{
See \refexam{exam:RevHenonMap}; shouldn't ``time-reversal operator'' just
reverse momentum/velocity \refeq{KeMaWi07(9)}? The definitive review of
the nomenclature is Roberts and Quispel\rf{RobQui92} {\em Chaos and
time-reversal symmetry. {Order} and chaos in reversible dynamical
systems}.
    }

It turns out that symmetry naturally arises in the
study of return maps of flows of time-periodic vector
fields with mixed space-time symmetries. In a natural
way these space-time symmetries form a group
under composition.

4.1. Symmetric periodic orbits

a result on periodic
orbits is by far the most well known and used result
in reversible dynamical systems. In 1915, Birkhoff
[Birkhoff, 1915] described the use of reversibility
to find periodic orbits of the restricted three-body
problem. In 1958 DeVogelaere\rf{DeVogelaere58}
described the method again, but now as a tool for
searching for symmetric periodic orbits of reversible
systems (by computer).

Definition 4.1 (Symmetric orbits). An orbit
of a dynamical system is $\Refl$-symmetric
or symmetric with respect to $\Refl$ when the orbit
is setwise invariant under $\Refl$.

Theorem 4.2 (Symmetric orbits for maps)
is the same as LC21\rf{LC21} classification of 3 kinds of symmetric orbits
(Predrag believes).

Theorem 4.1 or 4.2 is used in almost every paper discussing
reversible dynamical systems. In particular,
these theorems imply efficient techniques for tracking
down $\Refl$-symmetric periodic orbits, as it justifies
searching for them in only a subset of the full phase
space.

A well-known property of linear reversible systems
is that their eigenvalue structure is similar to that of
Hamiltonian systems.

Theorem 4.4 (Eigenvalues of linear reversible
systems)...


\subsection{Calogero 2007 paper {BrCaDr07}}
\label{sect:BrCaDr07}

Bruschi, Calogero and Droghei\rf{BrCaDr07} {\em Tridiagonal matrices,
orthogonal polynomials and {Diophantine} relations: {I}}.

If the equations of motion and the solution of their initial-value
problems involve only algebraic operations: finding the zeros of
explicitly known polynomials of degree $N$, finding the eigenvectors and
eigenvalues of explicitly known $N{\times}N$ matrices, the dynamical
system is called \emph{solvable}.

It is well known that the eigenvalues of tridiagonal matrices can be
identified with the zeros of polynomials satisfying three-term recursion
relations and being therefore members of an orthogonal set.
They consider the class of monic polynomials $p_{n}(s)$, of degree n in
the variable ${s}$, defined by the three-term recursion relation
\beq
p_{n+1}(s) - ({s}+a_n)p_{n}(s) - b_{n}p_{n-1}(s)=0
\,,
\ee{BrCaDr07(1a)}
They associate a tridiagonal $[\cl{}\!\times\!\cl{}]$ matrix $M$
with it,
related to $p_{n}(s)$ via the ``well-known'' formula
\beq
p_{n}(s) = \det(s-M)
\,.
\ee{BrCaDr07(2)}
The $n$ zeros
of the polynomial $p_{n}(s)$ coincide with the $n$
eigenvalues of the tridiagonal matrix $M$.


\HREF{https://en.wikipedia.org/wiki/Favard\%27s_theorem}
{Favard's theorem}:
a sequence of polynomials satisfying a suitable 3-term recurrence
relation of the form
\(p(s)_{n+1}=(s-c_n)p(s)_n-d_{n}p(s)_{n-1}\)
for some numbers $c_{n}$ and $d_{n}$,
then the polynomials $p(s)_{n}$ form a sequence of orthogonal polynomials.
    \PCedit{
    We are interested in this, because we would like to
    understand \HillDet s polynomials factorization, such as in \refeq{1stChebGenF}.
    }

See also
\HREF{https://en.wikipedia.org/wiki/Jacobi_operator} {Jacobi operator}.
The self-adjoint {\em Jacobi operators} act on the Hilbert space of square
summable sequences over the $\ell^2(\mathbb{N})$:
\[
Jf_0 = a_0 f_1 + b_0 f_0, \quad Jf_n
     =  a_n f_{n+1} + b_n f_n + a_{n-1} f_{n-1}, \quad n>0
\,,
\]
where the coefficients are assumed to satisfy
\[
a_n >0, \quad b_n \in \reals
\,.
\]
The solution $p_n(s)$ of the recurrence relation
\[
J\, p_n(s) = {s}\, p_n(s), \qquad p_0(s)=1 \text{ and } p_{-1} (s)=0
\,,
\]
is a polynomial of degree $n$ and these polynomials are orthonormal.
Here $J$ can be interpreted as a lattice right-shift operator, \ie, for
a temporal lattice this is related to the evolution in time.
This recurrence relation can also be written as
\beq
a_{n+1}p_{n+1}(s) - ({s}-b_n)p_n(s) + a_{n}p_{n-1}(s)=0
\,,
\ee{JacOperCount}
    \PCedit{
   (Compare with \refeq{genFuncts:CatRec-s}.)
    }

or (if one replaces ${s}\to\mu^2$)
\[
(-J\,+\,\mu^2)p_n(\mu^2) = 0
\,,
\]
reminiscent of the Klein–Gordon equation \refeq{KleinGtInd}.
The operator will be bounded if and only if the coefficients are bounded.
The case $a(n)=1$ is known as the discrete one-dimensional Schr\"odinger operator.
It also arises in:
\begin{itemize}
  \item
The Lax pair of the Toda lattice (see {\bf 2020-08-02 Predrag} Toda post)
  \item
The three-term recurrence relationship of orthogonal polynomials.
  \item
Algorithms devised to calculate Gaussian quadrature
rules, derived from systems of orthogonal polynomials.
\end{itemize}

%\newpage
\subsection{Variational method, Dong 2020 paper {DoLiLi20}}
\label{sect:DoLiLi20}

Some of the text that follows is copy \& paste from Dong, Liu and
Li\rf{DoLiLi20} {\em Unstable periodic orbits analysis in the generalized
{Lorenz}-type system} {2020). That is probably copy \& paste from
earlier Dong papers, have not checked that.

Their description of our variational
method\rf{CvitLanCrete02,lanVar1} looks better than our own, or the
ChaosBook text (which is not in the public edition yet).

                                                    \toCB
[...] a local, time-dependent scaling factor is used to adjust the period
\beq
\lambda \left({s}_{n}\right)\equiv {\Delta}{t}_{n}/{\Delta}{s}_{n}
\,,
\ee{LDJL21(8)}
where
${\Delta}{s}_n = {s}_{n+1} - {s}_n, n = 1, ..., N - 1$,
${\Delta}{s}_{N}=2{\pi}-\left({s}_{N}-{s}_{1}\right)$
 and
 ${\Delta}{t}_{n}$
follows the same pattern. The scaling factor guarantees the loop
increment
${\Delta}{s}_{n}$ is proportional to its counterpart
$\Delta t_n + \delta t_n$
on the {\po} when the loop approaches the cycle, with
$\delta t_n \to 0$ as $L \to p$.

The {\jacobianM}
\(
\boldsymbol{J}\left(x,t\right)=\mathrm{d}x\left(t\right)/\mathrm{d}x\left(0\right)
\)
is obtained by integrating
\beq
\frac{\mathrm{d}\boldsymbol{J}}{\mathrm{d}t}=\boldsymbol{A}\boldsymbol{J}, {\boldsymbol{A}}_{ij}=\frac{\partial {\boldsymbol{v}}_{i}}{\partial {\boldsymbol{x}}_{j}}, \quad \mathrm{w}\mathrm{i}\mathrm{t}\mathrm{h}\quad \boldsymbol{J}\left(x, 0\right)=1
\,.
\ee{LDJL21(9)}
[Predrag is getting tired of copying LaTeX from
Dong, Liu and Li\rf{DoLiLi20}. Whoever continues this,
remember to turn on \emph{MathJax}, upper right corner of paper's homepage.]

The variational evolution equation containing the crux of the method of
finding {\po}s
\beq
\frac{{\partial }^{2}\tilde{\boldsymbol{x}}}{\partial s\partial \tau }-\lambda \boldsymbol{A}\frac{\partial \tilde{\boldsymbol{x}}}{\partial \tau }-\boldsymbol{v}\frac{\partial \lambda }{\partial \tau }=\lambda \boldsymbol{v}-\tilde{\boldsymbol{v}}
\,.
\ee{LDJL21(13)}
Rewriting as
\beq
\frac{\partial \tilde{\boldsymbol{v}}}{\partial \tau }-\lambda \frac{\partial \boldsymbol{v}}{\partial \tau }=-\left(\tilde{\boldsymbol{v}}-\lambda \boldsymbol{v}\right)
\,,
\ee{LDJL21(14)}
yields
\beq
\left(\tilde{\boldsymbol{v}}-\lambda \boldsymbol{v}\right)={\mathit{\text{e}}}^{-\tau }\left(\tilde{\boldsymbol{v}}-\lambda \boldsymbol{v}\right){\vert }_{\tau =0}
\,,
\,,
\ee{LDJL21(15)}
a minimizing cost function:
\beq
{F}^{2}\left[\tilde{x}\right]=\frac{1}{2{\pi}}\underset{L\left(\tau \right)}{\oint } \mathrm{d}\tilde{x}{\left[\tilde{v}\left(\tilde{x}\right)-\lambda v\left(\tilde{x}\right)\right]}^{2}
\,.
\ee{LDJL21(16)}
As the loop descends toward a \po, the cost function
decreases monotonically the differences between
$\tilde{v}\left(\tilde{x}\right)$ and ${v}\left(\tilde{x}\right)$,
converging in the $\tau\to\infty$ limit to the \po. On the \po, by
\refeq{LDJL21(8)},
\(
\lambda \left(s,\infty \right)=\left(\mathrm{d}t/\mathrm{d}s\right)\left(\tilde{\boldsymbol{x}}\left(s,\infty \right)\right)
\),
and the
period is given by
\beq
{T}_{\mathit{\text{P}}}={\int }_{0}^{2\pi }\lambda \left(\tilde{x}\left(s,\infty \right)\right)\mathrm{d}s
\,.
\ee{LDJL21(17)}

%\beq
%\,.
%\ee{LDJL21(XX)}

The finite difference scheme is employed in a discretization of a loop
\beq
\tilde{\boldsymbol{v}}_{n}
    \equiv
{\left.\frac{\partial \tilde{\boldsymbol{x}}}{\partial s}\right\vert }_{\tilde{x}=\tilde{x}\left({s}_{n}\right)}
    \approx
{\left(\hat{\boldsymbol{D}}\tilde{\boldsymbol{x}}\right)}_{n}
\ee{LDJL21(18)}
and the five-point approximation is adopted
\beq
\hat{\boldsymbol{D}}=\frac{1}{12h}\left(\begin{array}{lccccccccccc}\hfill 0\hfill & \hfill 8\hfill & \hfill -1\hfill & \hfill \hfill & \hfill \hfill & \hfill \hfill & \hfill \hfill & \hfill \hfill & \hfill \hfill & \hfill 1\hfill & \hfill -8\hfill \\ \hfill -8\hfill & \hfill 0\hfill & \hfill 8\hfill & \hfill -1\hfill & \hfill \hfill & \hfill \hfill & \hfill \hfill & \hfill \hfill & \hfill \hfill & \hfill \hfill & \hfill 1\hfill \\ \hfill 1\hfill & \hfill -8\hfill & \hfill 0\hfill & \hfill 8\hfill & \hfill -1\hfill & \hfill \hfill & \hfill \hfill & \hfill \hfill & \hfill \hfill & \hfill \hfill & \hfill \hfill \\ \hfill \hfill & \hfill \hfill & \hfill \hfill & \hfill \hfill & \hfill \hfill & \hfill \cdot \cdot \cdot \hfill & \hfill \hfill & \hfill \hfill & \hfill \hfill & \hfill \hfill \\ \hfill \hfill & \hfill \hfill & \hfill \hfill & \hfill \hfill & \hfill \hfill & \hfill \hfill & \hfill 1\hfill & \hfill -8\hfill & \hfill 0\hfill & \hfill 8\hfill & \hfill -1\hfill \\ \hfill -1\hfill & \hfill \hfill & \hfill \hfill & \hfill \hfill & \hfill \hfill & \hfill \hfill & \hfill \hfill & \hfill 1\hfill & \hfill -8\hfill & \hfill 0\hfill & \hfill 8\hfill \\ \hfill 8\hfill & \hfill -1\hfill & \hfill \hfill & \hfill \hfill & \hfill \hfill & \hfill \hfill & \hfill \hfill & \hfill \hfill & \hfill 1\hfill & \hfill -8\hfill & \hfill 0\hfill \end{array}\right),
\ee{LDJL21(19)}
where $h = 2\pi/N$, and each entry represents a $[d\times{d}]$ matrix in
\refeq{LDJL21(19)}, $8 \to 8\,\unit$, etc; the blanks in the matrix
represent zeros. The two $[2d\times{2d}]$ matrices, found in the upper
right-corner and the lower-left corner of \refeq{LDJL21(19)},
respectively, can be written as
\[
{\boldsymbol{M}}_{1}
    =
\left(\begin{matrix}\hfill \unit\hfill & \hfill -8\,\unit\hfill \\ \hfill 0\hfill & \hfill \unit\hfill \end{matrix}\right)
    \,, \quad
{\boldsymbol{M}}_{2}
    =
\left(\begin{matrix}\hfill -\,\unit\hfill & \hfill 0\hfill \\ \hfill 8\,\unit\hfill & \hfill -\,\unit\hfill \end{matrix}\right),
\]
and are related to the periodic boundary conditions.

After discretization, \refeq{LDJL21(13)} can be written as
\beq
      \left(\begin{matrix}\hfill \hat{\boldsymbol{A}}\hfill
                        & \hfill -\hat{\boldsymbol{v}}\hfill
                        \\ \hfill \hat{\boldsymbol{a}}\hfill
                        & \hfill 0\hfill \end{matrix}\right)
      \left(\begin{matrix}\hfill {\delta}\tilde{\boldsymbol{x}}\hfill
                        \\ \hfill {\delta}\lambda \hfill \end{matrix}\right)
   ={\delta}\tau
      \left(\begin{matrix}\hfill \lambda \hat{\boldsymbol{v}}-\hat{\tilde{\boldsymbol{v}}}\hfill
                         \\ \text{0}\hfill \end{matrix}\right)
\,,
\ee{LDJL21(20)}
where
\(
\boldsymbol{\hat{A}}
 =\hat{\boldsymbol{D}}-\lambda\,\mathrm{diag}
    \left[{A}_{1}, {A}_{2},\cdots,{A}_{N}\right]
\)
and
\(
{A}_{n}=\boldsymbol{A}\left(\tilde{x}\left({s}_{n}\right)\right)
\)
is defined in \refeq{LDJL21(9)}.
\[
\hat{\boldsymbol{v}}={\left({v}_{1},{v}_{2}, \cdots , {v}_{N}\right)}^{\mathrm{T}}\quad \mathrm{w}\mathrm{i}\mathrm{t}\mathrm{h}\quad {\boldsymbol{v}}_{\mathrm{n}}=\boldsymbol{v}\left(\tilde{x}\left({s}_{n}\right)\right)
\,,
\]
are the two column vectors that we match everywhere during the
evolution of the loop.
$\hat{\boldsymbol{a}}$ is an $Nd$\dmn\ row vector that restricts
the coordinate variations. The deformation of the loop coordinates
${\delta}\tilde{\boldsymbol{x}}$
and period
${\delta}\lambda$ can be calculated by inverting the
$[(Nd+1)\times(Nd+1)]$
matrix on the left-hand side of \refeq{LDJL21(20)}.

We use the
banded lower-upper (LU) decomposition scheme. Due
to the structure of the matrix in \refeq{LDJL21(20)}, the
Woodbury formula is adopted on the cyclic and boundary terms in the
calculations\rf{Press86}, thus enabling an efficient search for
{\po}s.

The variational approach is a good choice to search cycles in a
low-dimensional dissipative system. This method is not only suitable for
the determination of {\po}s but also for the homoclinic and
heteroclinic orbits\rf{DoLa14a}. In the previous work, the {\po}s in various chaotic systems were calculated efficiently using the
variational method\rf{Dong18a,Dong18b,DoLa14}, which illustrates the
practicability of this method in the GLTS.

The variational method can also be used to analyze bifurcation phenomena.
When the parameters of a dynamical system change continuously, we can
observe the conditions under which the {\po} is created or
disappears through deformations of the {\po}.




%\beq
%\,.
%\ee{LDJL21(XX)}

%\beq
%\,.
%\ee{LDJL21(XX)}

%\beq
%\,.
%\ee{LDJL21(XX)}



\subsection{Dong 2021 paper {LDJL21}}
\label{sect:LDJL21}

Liu, Dong, Jie, and Li\rf{LDJL21}
{\em Topological classification of periodic orbits in the generalized Lorenz-type system with diverse
symbolic dynamics} (2021) \CBlibrary{LDJL21}.

I like their explanation of the variational method for finding
{\po}s, and their use of it to study bifurcations by homotopy
evolution, which I believe to be authors' original contribution to the
subject.

I wish they would consider quotienting symmetries of a given dynamical
system prior to their symbolic dynamics analysis. For the 3-disk system
quotienting the \Dn{3} symmetry vastly improves the convergence of cycle
expansions, see \toChaosBook{table.caption.412} {Table 23.2}, with
symmetry reduction illustrated by \toChaosBook{figure.caption.175}
{Figure 11.1}, and explained at length in ChaosBook, also for the Lorenz
flow, see \toChaosBook{exmple.14.4}{Example 14.4}, and the text leading
up to it.

Reduction of the Lorenz  \Dn{1} symmetry might not seem like much, but
even that is pretty impressive - the period of a prime or pre-{\po} is $|\Group|/|\Group_p|$-th root of the full state space orbit
period, and that leads to a significant simplification of the given
problem. For example, in \refref{lanCvit07},
\HREF{http://chaosbook.org/~predrag/papers/ks.pdf\#f8e} {Figure 8 (e)}
that led to the 3-letter, bi-modal return map instead of a numerically
unmanageable 9-letter return map in the symmetry-unreduced, original
state space.

In authors' example, symmetry reduction would reduce their Table 1 cycles
0,1; 001, 011; 0001,0111 etc to a single cycle, and 01 to repeat 1-cycle,
0011 to a repeat of a prime 2-cycle, and in general of 1/2 period. Table
2 is another illustration, with 2,3; barred 2,3 to a single letter, 23 a
repeat of 1-cycle, ..., etc, and the co-existing attractors of Figure 9
reduced to a single attractor. Would be nice to visualize self-linking in
the symmetry-reduced state space.

How the reduced symbolic dynamics is related to unreduced one (the kind
used by the  authors) is explained in ref. [34]
       \toChaosBook{section.25.5} {Sect. 25.5 \Dn{1} factorization}.

\newpage
\subsection{Letter from Ping Ao}
\label{sect:Pin Ao21}
\begin{quote}
2020-12-16
from
\HREF{https://scholar.google.com/citations?user=JQyz-BoAAAAJ&hl=en}
{Ping Ao}, aoping@sjtu.edu.cn
\\
\emph{Distinguished Professor Shanghai Center for Quantitative
Life Sciences and Physics Department  Shanghai University; and  Shanghai
Center for Systems Biomedicine  Shanghai Jiao Tong University Shanghai,
China}
\end{quote}

Dear Prof. Predrag Cvitanovi\,c,
Many thanks for your inspiring pandeminar on "Spatiotemporal Cat: A
Chaotic Field Theory". It is very interesting to use "chaotic attractors"
as building blocks for field theories. Here I have one remark and one
question which may be of an interest to you.

In dynamical systems it is known that there are three generic classes of systems:
 fixed points or linear; limit cycles; chaotic attractors.
The last two must be nonlinear, as well explained during your talk.
For dissipative dynamical systems we have explicitly constructions for all three:

\begin{enumerate}
  \item \emph{Structure of stochastic dynamics near fixed points},
   C Kwon, P Ao, DJ Thouless. PNAS 102 (2005) 13029
\HREF{https://www.pnas.org/content/pnas/102/37/13029.full.pdf}
{(click here)}

  \item \emph{Limit cycle and conserved dynamics},
   XM Zhu, L Yin, P Ao. Intl J Modern Physics B20 (2006) 817
\HREF{https://www.worldscientific.com/doi/abs/10.1142/S0217979206033607}
{(click here)}

  \item
\emph{Exploring a noisy van der Pol type oscillator with a stochastic approach},
   Yuan, R.-S.; Wang, X.-A.; Ma, Y.-A.; Yuan, B. \& Ao, P. . Phys Rev E87
   (2013) 062109
\HREF{https://journals.aps.org/pre/abstract/10.1103/PhysRevE.87.062109}
{(click here)}

\item \emph{Potential function in a continuous dissipative chaotic system:
{Decomposition} scheme and role of strange attractor},
   Yian Ma, Qijun Tan, Ruoshi Yuan, Bo Yuan and Ping Ao. Intl J Bifur Chaos 24 (2014) 1450015
\HREF{https://www.worldscientific.com/doi/abs/10.1142/S0218127414500151}
{(click here)}

\item A summary of our method is here:
 \emph{SDE decomposition and {A}-type stochastic interpretation in
 nonequilibrium processes},
  Ruoshi Yuan, Ying Tang, Ping Ao %YuTaAo17
Frontiers of Physics 12 (2017) 120201
% DOI 10.1007/s11467-017-0718-2
\HREF{https://academic.hep.com.cn/fop/CN/10.1007/s11467-017-0718-2\#1}
{(click here)}
\end{enumerate}

To my knowledge, we were the first group to explicitly construct the
"Hamiltonian" for limit cycles and chaotic attractors, thought not
possible before our work. I would be happy to be updated on this, due to
our limited knowledge.

My remark is that, our explicit construction for chaotic attractors
revealed a hidden structure which may be useful for your construction,
too.

My question is, is there a field theory constructed upon limit cycles?

Will be happy to receive your feedback.










\renewcommand{\ssp}{x}
\renewcommand{\Ssym}[1]{{\ensuremath{s_{#1}}}}    % Boris
\renewcommand{\Xx}{\ensuremath{\mathsf{X}}}      % Boris
