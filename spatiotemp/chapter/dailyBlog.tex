\svnkwsave{$RepoFile: siminos/spatiotemp/chapter/dailyBlog.tex $}
\svnidlong {$HeadURL: svn://zero.physics.gatech.edu/siminos/spatiotemp/chapter/dailyBlog.tex $}
{$LastChangedDate: 2021-10-04 12:20:04 -0400 (Mon, 04 Oct 2021) $}
{$LastChangedRevision: 7383 $} {$LastChangedBy: predrag $}
\svnid{$Id: dailyBlog.tex 7383 2020-04-19 05:28:36Z predrag $}

\chapter{Space-time, blogged}
\label{chap:dailyBlog}
% Predrag                                           12 January 2016

\begin{bartlett}{
% \HREF{http://www.jango.com/music/The+Verve} {Space And Time}
\HREF{https://soundcloud.com/redwantingblue/spacetime}
     {I'm a space and time continuum }
                }\bauthor{
% The Verve on Urban Hymns
\HREF{http://music.redwantingblue.com/track/spacetime}
{Red Wanting Blue}
                }

\end{bartlett}

\bigskip

   % *********************************************************************
\hfill   {\color{red} The latest entry at the bottom for this blog}

\bigskip


\begin{description}

    \PCpost{2016-10-27}{
{\bf
The revolution {\scriptsize
\HREF{http://chaosbook.org/tutorials/index.html}
{(in strongly nonlinear field theory)}
            } will not
\HREF{https://www.youtube.com/watch?v=BS3QOtbW4m0}
{be televised.}
             }
But it will be on
\HREF{http://chaosbook.org/course1/index2.html} {YouTube}.
}

\item[2011-05-15 Predrag]
A. Hramov and A. Koronovskii\rf{HraKor07}, {\em Detecting unstable periodic
spatio-temporal states of spatial extended chaotic systems},
\arXiv{0708.4349}.

\item[2016-01-12 PC] also of possible interest:

Llibre\rf{Llibre15}
{\em The averaging theory for computing periodic orbits}.

Also, Gutkin and Osipov\rf{GutOsi15} write ``In general, calculating periodic
orbits of a  non-integrable  system  is a  non-trivial  task.   To  this
end  a  number  of  methods have  been developed,'' and then,  for some
reason, they refer to \refref{baranger88}.


    \PCpost{2016-03-02}{
Also Paz\'o \etal\rf{PaSzLoRo09} {\em Structure of characteristic
{Lyapunov} vectors in spatiotemporal chaos}. Actually (I hesitated to
bring it up) this line of inquiry goes smoothly into Xiong Ding's
inertial manifold dimension project.

Not sure Li \etal\rf{LXSH08} {\em Lyapunov spectra of coupled chaotic
maps} is of any interest, but we'll know only if we read it.

Takeuchi and Sano\rf{TaSa07}
{Role of unstable periodic orbits in phase transitions of coupled map lattices}.
    }

\PCpost{2016-08-15}{:
Before I start sounding critical: Rana, Adrien, Matt and Li are all good
students / postdoc, and the work and what people learned this summer is
very good. Now, to my first impressions.

Matt's report is most in my taste. We still do not know whether there is
something seriously wrong with the spatiotemporal proposal  for \KS, the
core part of this whole research project, or is there a coding problem,
but that's research. We'll sort it out eventually.
}

\PCpost{2016-09-28}{
Learned much more from \HREF{https://www.math.gatech.edu/users/rll6}
{Rafael de la Llave} that I can possibly remember.

I explained that we turn a dissipative PDE with commuting continuous
symmetries (time, space translations) into a set of 1st order PDEs in
both time and space directions. Then we integrate spatio\-temporally
periodic solutions along the space directions just like we used to
integrate the along the time direction. I explained that as a discretized
version of that, we studdy 1D chain of diffusively \catlatt s.

Rafael was very happy to hear that, because he has hardly ever done
anything else in his life. Except piss off the Smale cult by publicly
refusing to prove generic results. ``Would you give your fiance a diamond,
or a generic pebble of the street?''
    }

    \PCpost{2016-10-03}{
Various people focus on
proving ``local rigidity of partially hyperbolic algebraic actions.''
Katok's papers on this are absolutely unreadable.

Reinhardt and Mireles James
\HREF{http://cosweb1.fau.edu/~jmirelesjames/preprints/Wu_submitJDE.pdf}
{paper} deals with constructing unstable manifolds of steady states in
parabolic PDE's and gets very close to establishing homoclinic
intersections.

The concrete example they use is the Fisher equation, but it seems that the
\KS\ could work. The recoding is not trivial and that there are many things
that will go wrong.

We had done some work on extending these techniques to periodic orbits.
Of course, one needs to have the periodic orbits and, I know full well
that this is not trivial (I have spent some time doing something and
realized that I am not tough enough to do it). On the other hand, may be
some infusion of strength and new ideas could be enough.
}

\newpage
    \PCpost{2016-10-24}{
A note on a conversation between Stephen Shenker and Paul Wiegmann. Shenker
presented his work on a quantum black hole in a box, \ie, in a thermal equilibrium,
and mentioned that this field theory has a leading positive Lyapunov exponent.
Paul argued afterwards that a system in equilibrium cannot have positive
Lyapunov exponents - they are a property of externally driven, out of equilibrium
systems (let's say, Navier-Stokes turbulence). I believe here Stephen is right,
will try to explain that first for finite-dimensional dynamical systems, and
then for infinite-dimensional field theories.

\paragraph{What is `chaos'?}
(\wwwcb{}, ver.~15.7, Sect.~1.3.1)
In a `chaotic' dynamical system,
any two trajectories that start out very close to each other
separate exponentially with time, and in a finite (and in practice,
short) time their separation $\delta {\bf x}(t)$
attains the magnitude of $L$, the characteristic linear extent of the
whole system.
%, \reffig{fgLLN-4}.
This property of {\em sensitivity to initial conditions} can be quantified as
\[
|\delta {\bf x}(t)| \approx e^{\Lyap t} |\delta {\bf x}(0)|
\]
where $\Lyap$, the mean rate of separation of trajectories of the system,
is the {\em Lyapunov exponent}.

A positive Lyapunov exponent does not in itself lead to chaos. One
could try to play 1- or 2-disk pinball game, but it would not be much
of a game; trajectories would only separate, never to meet again. What
is also needed is {\em mixing}, the coming together
again and again of trajectories.  While locally the nearby
trajectories separate, the interesting dynamics is confined to a
globally finite region of the {\statesp} and thus the
separated trajectories are necessarily
folded back and can re-approach each other
arbitrarily closely, infinitely many times.
The number of
trajectories that return by time $t$ can be quantified as
\[
    N(n) \approx e^{h t}
\]
where $h$, the growth rate of the
number of topologically distinct trajectories,
is called the {\em ``topological entropy"}.
The word
`chaos' has in this context taken on a narrow technical meaning.  If
a deterministic system is locally unstable (positive Lyapunov
exponent) and globally
\indx{mixing} (positive {entropy}) it is said
to be {\em chaotic}.

While mathematically correct, the definition of chaos as `positive
Lyapunov + positive entropy' is useless in practice, as a measurement
of these quantities is intrinsically asymptotic and beyond reach for
systems observed in nature or simulated on computers.
% \DV{very well said}
More powerful is Poincar\'e's vision
of chaos as
the interplay of local instability (unstable periodic orbits)
and global mixing (intertwining of their stable and unstable
manifolds).

\paragraph{Escape rates}
(\wwwcb{}, ver.~15.7, Sect.~22.4)
The above paragraph describes the essence of ``chaos'' for
finite-dimensional dynamical systems, but not how to compute its
consequences. That is accomplished by the periodic orbit theory.
Consider the simplest possible chaotic system, with the {\statesp}
partitioned into two intervals, with equal expanding multipliers,
$|\ExpaEig_0| = |\ExpaEig_1| = e^\lambda  $ (Bernoulli map, tent map).

[A side remark to Paul: Hamiltonian versions with uniform stretching,
such as Arnold cat map, baker's map, Selberg zeta function, \etc, work
the same way, in equilibrium or away from it, modulo inessential details,
such as pairing of eigenvalues, due to the symplectic invariance].

In the above $\Lyap = \ln |\ExpaEig| /\period{}$ is the cycle Lyapunov
exponent. For an open system, the real part of the eigenvalue
$\eigenvL_\alpha$ gives the decay rate $\gamma$ of $\alpha$th eigenstate. If there was
only one \po\ (zero entropy),  the decay rate would  equal the cycle
Lyapunov exponent.

Our task is to determine the leading zero $z = e^{-\gamma}$ of the dynamical
zeta function. The exponentially growing number of cycles with growing period
balances the escape rate from individual cycles, and conspires to shift the
zeros of the zeta function, and for this particular uniform stretching map
correct formula is
% follows from \refeq{2scale}
\beq
0 = 1 - e^{\gamma-\lambda+h}\,, \quad\quad h = \ln 2 .
\label{addendum}
\eeq
This particular formula for the escape rate $\gamma$is a special case of a
general relation between escape rates, Lyapunov exponents and entropies.

Physically this means that the escape induced by the repulsion by each
unstable fixed point is diminished by the rate of backscatter from other
repelling regions; the difference of the two is the actual escape rate.


What about nonlinear field theory and turbulence?
}

    \PCpost{2016-10-27}{
\textbf{A prologue}.
By 1972 I had completed, together with Kinoshita, what at the time was
the largest and most expensive QFT calculation ever\rf{CviKin72} (it
broke CERN theory division computing budget). Once I emerged from the
trenches, wiser for the experience, tthe main thing I had learned
is that
\HREF{http://chaosbook.org/~predrag/papers/preprints.html\#FiniteFieldTheo}
{perturbative QED is stupid} (colored text is live hypelinks), and doing
QCD by Feynman diagrams is plain wrong.

If you are an atomic or nuclear
physicist you think that quantum mechanics is a Hamiltonian and the
energy spectrum. From that perspective, the Wigner surmise is the first
thing to do if you see a complicated spectrum. It tells you what
``random'' means for distributions constrained by unitarity,
time-reversal invariance, \etc. It's a diagnostic, not a theory. Not
then, and not now.

If you are a field theorist, you think that quantum mechanics is path
integrals, Lagrangians, their extrema (WKB/semiclassics), and $\hbar$
expansions around these classical solutions. And you strive (i) write
down equations, (2) solve them, and (3) predict, with no statistical
assumptions or mindless averaging. One thing that ``chaos'' is not
are Gaussians.

\HREF{http://chaosbook.org/~predrag/papers/universalFunct.html}
{Since 1976} I knew that nonlinear equations can have infinitely many
distinct unstable solutions, and, in order to organize them for the period-doubling
and circle-maps\rf{FeKaSh82} into the fundamental terms, and systematic series
of exponentially
small corrections, I invented cycle expansions\rf{inv}, in an esoteric
setting - not time evolution flow of a conventional dynamical system,
but fictitious ``time'' of renormalization flows. Working backwards, I
was able to relate them to zeta functions of Gutzwiller and Ruelle.

I derived the equation for the period doubling fixed point
function (not a big step - it is the limit of his functional recursion
sequence), which has since played a key role in the theory of transitions
to turbulence. Since then we have generalized the universal equations to
period n-tuplings; constructed universal scaling functions for all
winding numbers in circle maps, and established universality of the
Hausdorff dimension of the critical staircase.

This all reads like a Physics Today obituary for a buddy of Lenny Susskind.
But it brings us to today, when for the next 6 weeks me and my plumber friends
will work on how to fashion a theory of turbulence from myard ...


    }
\newpage

    \PCpost{2016-11-06}{ Early references on the spatiotemporal
invariants and invariant measures:

Ruelle\rf{ruelext} {\em Large volume limit of the distribution of
characteristic exponents in turbulence} writes: ``
For spatially extended conservative or dissipative physical systems,
it appears natural that a density of characteristic exponents per unit volume
should exist when the volume tends to infinity. In the case of a turbulent
viscous fluid, however, this simple idea is complicated by the phenomenon of
intermittency.

In the case of a Hamiltonian system, and taking for $\rho$ an ergodic
component of the Liouville measure, one finds that the spectrum is invariant
under change of sign.

For certain classes of physical systems \emph{a large volume limit} exists
(for equilibrium statistical mechanics this is the \emph{thermodynamic
limit}). We want to investigate the possibility of defining a large volume
limit for certain dynamical systems. The idea is that, if several systems
with independent dynamics occupy disjoint regions, the spectrum
of the joint system is simply the union of the spectra of the subsystems
(repeated characteristic exponents appear with added multiplicity). If an
interaction between the subsystems is introduced, one may hope that this does
not alter the spectrum much and that, in the large volume limit, a number of
characteristic exponents per unit volume may be defined.
''

He refers to the leading Lyapunov exponent as ``characteristic exponent.''
Then the things get pretty technical, and I skimmed through the rest.

``
The `barber pole' turbulence which fascinated Feynman (Feynman et al.
[5, II, Sect. 41-6]) appears in the flow between two concentric rotating cylinders
and consists of an helical turbulent band alternating with a `laminar' region.
''
    }

\item[2017-01-24 Predrag] Celebrating Finkelstein's
{\em Reckless Ideas in Physics}
by a night of thinking. Dunno whether the above papers are any good, but
they have pushed us over the edge, and the log germinating process is
over.

Dynamics is dead - the theory of turbulence is now again a branch of
physics, on par with Ising models and quantum field theories, but this
time around based on fundamental equations, with no statistical
assumptions.

In turbulence, there
is no more time, there is only spacetime, and DNS will have to follow -
integrating forward in time from specified initial conditions is over.
From now one has to parallelize and solve the equations in spacetime,
globally and variationally, not step them incrementally in any given 1D
spacetime direction - that problem is ill posed. And the game is to
enumerate admissible patterns - symbolic dynamics will be more crucial
than ever before.

\item[2017-02-03 Bj\"orn Birnir] A nice book to learn basic facts about
PDEs is {\em Partial Differential Equations} by Fritz John\rf{John78}.

Also, study the work of
\HREF{http://www-m8.ma.tum.de/personen/kuehn/} {Christian Kuehn}



\item[2017-02-15 Predrag]
The {\em 2017-01-24} declaration above that the dynamics is dead is not
just another piece of Predrag bombast. In the Santa Barbara secret
seminar (\ie, the one talk that was not recorded) it met some resonance
with fluid dynamicists. In the Dresden MPIPKS talk it barely raised
anyone's pulse (a quantum optics crowd - makes sense, what do they care
about chaos). Only Denis Ullmo responded, and that is because he's been
thinking about ``mean field games,'' a new field of social sciences which
uses constraint optimization methods. But I digress.

For me, this is a time of intellectual turmoil: dynamics is dead. I make
fun of rocket scientists\rf{WGBGQ13} discovering now, in 2013, that ``the
initial value problem of a chaotic dynamical system is
\emph{ill-conditioned},'' but they are right.
Since high school, I have been thinking incorrectly, like a physicist
(formulating chaotic dynamics locally in time, as an initial value
problem, to be integrated forward in time) while all along working
correctly, as an engineer (solving for chaotic orbits globally, by
variational methods). So DasBuch has to be rewritten, entirely, from
chapter 4 on, with the ideas of Chapter 38 {\em Relaxation for cyclists}
to be moved to the beginning of the exposition. All bits and pieces are
already available, but the puzzle has to be fully reconfigured.

I credit the moronic citizenry for this insight - since November 9, the
National Day of Shame, I have read no news, all my life is now local.
Sit, think, talk to friends and family. It is amazing how peaceful and
liberating is this quiet period, in-between the shameful deed and the
major disasters that are about to befall us.

The real credit goes to Boris Gutkin and trying to decode how thinks:
{\catLatt} is a gift that goes on giving. For a single cat, things seem
business as usual - dynamics is just a product of matrices, and if the
eigenvalues are hyperbolic, at any finite discrete
time the usual partition of \statesp\ issues, into
exponentially shrinking regions of given finite symbolic dynamics.
But the moment one goes to 2\dmn\ {\catlatt} business as usual is impossible -
advancing a spatial configuration one discrete step forward in time is impossible,
as it is infinitely unstable.

This message is really nailed in by going to continuous spacetime.

% Multiple shooting shadowing for sensitivity analysis of chaotic systems and turbulent fluid flows

\PCpost{2017-03-26}{ This might merit a quick read:
  and M. Balajewicz\rf{MojBal17}
  {\em Lagrangian basis method for dimensionality
  reduction of convection dominated nonlinear flows}
  }

\PCpost{2017-03-26}{ This might merit a quick read:
A. Gouasmi, E. Parish and K. Duraisamy\rf{GoPaDu16}
{\em Characterizing memory effects in coarse-grained nonlinear systems using
the {Mori-Zwanzig} formalism} looks interesting:
``
Reduced representations of complex, non-linear dynamical systems often
require closure, that is finding a good representation of the contribution of
the discarded physics to the retained physics. In this work, we pursue the
construction of closure models within the context of the Mori-Zwanzig (M-Z)
formalism of irreversible statistical mechanics. In this setting, the effect
of the unresolved states on the resolved states can be exactly represented as
a time-history dependent integral, commonly referred to as memory, and a
noise term. Evaluating the memory kernel requires the solution to the
so-called orthogonal dynamics equation. An understanding of the structure and
mechanics of the orthogonal dynamics is critical to the development of
M-Z-based reduced models. The orthogonal dynamics equation, however, is a
high-dimensional partial differential equation in free-space that is
intractable in general. We propose an alternative method to compute the
memory kernel that builds on the approximation that the orthogonal dynamics
locally retains the structure of a Liouville equation. The method is
demonstrated on Fourier-Galerkin simulations of
the Kuramoto-Sivanshinsky (K-S) equation.

The proposed procedure provides
accurate reconstruction of the memory integral and valuable insight into the
structure and scaling of the memory kernel.
''
  }

\PCpost{2017-06-20}{
Elder \etal\rf{ElXiDeMa95}
{\em Spatiotemporal chaos in the damped {Kuramoto-Sivashinsky} equation}: ``
A discretized version of the damped Kuramoto-Sivashinsky (DKS) equation
is constructed to provide a simple computational model of spatiotemporal
chaos in one dimension. The discrete map is used to study the transition
from periodic solutions to disordered solutions (i.e., spatiotemporal
chaos). The numerical evidence indicates a jump discontinuity at this
transition.
}

\PCpost{2017-06-13}{
Mike Schatz told us in February to look at
Xu and Paul\rf{XuPaul16}
{\em Covariant {Lyapunov} vectors of chaotic {Rayleigh-B{\'e}nard} convection}.
We really should.
}

\PCpost{2017-06-20}{
I have asked Mark Paul about their calculation. He said that they
computed ridiculously many \cLv s - lake a thousand - and never saw and
indication of the physical (invariant manifold) dimension. It is
certainly above 100. They compute a large ``fractal dimension;''
according to a ChaosBook remark, any fractal dimension above 5 or so is
not credible. Considering how much care estimating physical dimension
took in Xiong's work, one would need to critically read through the paper
before taking their results at the face value.

Bala(chandra) Suri feels that his 2D Kolmogorov flow would be the best
experimental flow to try to find the physical dimension. I worry about
that example - it has a rather complicated approximate discrete symmetry
--I remember something like $\Dn{8}$ from Mohammad's elton blog-- and a
broken translation symmetry, that must produce tons of nearly degenerate
\cLv s.

My best candidates are still the small domain \pCf\ and pipe flow.
}

\PCpost{2017-09-20} { {\bf ruminations on spatiotemporal stability.}
Let's first recall how the measure (``inverse'' of the ``stability'') of a
pattern is computed for the \catlatt\rf{GHJSC16}. The $d=2$ \catlatt\
``equations of motion'' in the {Lagrangian} form are given by
\beq
(-\Box + s - 4)\,\ssp_{nt} =\Ssym{nt}
\,,\qquad
\Ssym{z} \in \A
\,,
\ee{CoupledCatsPC}
with $\Box $ being the discrete spacetime Laplacian
on $\integers^2$,
\[
\Box\,\ssp_{nt}:=                   \ssp_{n,t-1} + \ssp_{n-1,t}
                   - 4\,\ssp_{nt} + \ssp_{n,t+1} + \ssp_{n+1,t}
\,.
\]
The  symbols  $\Ssym{nt}$ from  the set
$\A=\{\underline{3},\underline{2}, \cdots, s\!-\!2, s\!-\!1\}$
on the right hand side of \refeq{CoupledCatsPC} are  necessary  to keep
$\ssp_{nt}$ within the  interval $[0,1)$
The  symbol   $ \underline{|\ssp_{nt}|} $  here denotes $\ssp_{nt}$
with the negative sign, \ie, `$\underline{3}$' stands for symbol `$-3$'.
The {\brick}
\(
\Mm= \{\Ssym{nt} \in \A \,,\; (n,t)\in \integers^2 \}
\)
can be used as a 2\dmn\ symbolic representation of the lattice system
state.
Any solution \Xx\ of \refeq{CoupledCatsPC} can be
uniquely recovered from its symbolic representation \Mm. By inverting
\refeq{CoupledCatsPC} we obtain
\begin{equation}
  \ssp_{z}=\sum_{z'\in\integers^2}g_{z z'} \Ssym{z'}, \qquad  g_{z z' }
       =\left(\frac{1}{-\Box +s -4}\right)_{zz'}
       \,,
\label{GreenFuncCoupledPC}
 \end{equation}
where  $g_{z z'}$
% $z=(n,t)\in\Zz$,  $z'=(n',t')\in\Zz$,
is the  Green's
function for the 2\dmn\ discretized heat equation.
A \catlatt\ lattice state $\Mm$ %=\{\Ssym{z}|z\in\Zz\}$
is admissible if and only if all  $\ssp_{z}$ given by
\refeq{GreenFuncCoupledPC} fall into the interval $[0,1)$.

Moving onto \KS: first on solves the nonlinear fixed point equation,
something like
\[
{\scriptsize v} (\ssp^*) = 0
\]
where the state vector $\ssp$ is a finite discretization of the set of
fields  \refeq{e-ksX} over a compact $L\times{T}$
spatiotemporal \twot, and ${\scriptsize v}$ is constructed from a
linear operator ${\cal L}(\ssp)$, a matrix of
first order space and time derivatives acting on $\ssp$, plus
a nonlinear term ${\cal N}(\ssp)$.
The equilibrium solution $\ssp^*$ is now taken as a background field
(hopefully something related to the right side of \refeq{CoupledCatsPC}), and
one looks at small deformations $y=\delta\ssp = \ssp-\ssp^*$,
\beq
({\cal L}-1)\,y(x,t) =\ssp^*(x,t)
\,.
\ee{StabCatLattPC}
Inverting $({\cal L}-1)$ yields a $\det({\cal L}-1)^{-1}\times($co-matrices),
so for large unstable eigenvalues, the inverse is exponentially small, just
like for temporal dynamics $\zeta$ functions. This does not make sense as
yet, but you get my drift... The $\det()^{-1}$ term should yield the
likelihood of a given pattern, nothing to do with the ``fictitious time'' used
in paranoid Newton to find the equilibrium pattern $\ssp^*$.

    }

\PCpost{2017-09-27} {
Knobloch suggests that we study:

Klaus Kirchg{\"a}ssner\rf{Kirchgassner82},
{\em Wave-solutions of reversible systems and applications} has
296 Google Scholar citations.

Bj{\"o}rn Sandstede and Arnd Scheel. This one (not sure ?):
{\em On the structure of spectra of modulated travelling waves}\rf{SanSch01}
    }

%\NBBpost{2017-10-31}{I looked everywhere but could not find
%Politi \& Torcini\rf{PolTor92b}
%``Towards a statistical mechanics of spatiotemporal chaos'' (1992)
%in this blog or the bib files. The abstract reads ....
%}
%
%\PCpost{2018-01-20}{
%Moved the above, and all other Politi and Torcini\rf{PolTor92b} blog posts
%to \texttt{blogCats}, dailyCats.tex.
%    }

\PCpost{2017-11-03} {
This flew by below the radar, sorry - in \refchap{chap:blogMNG} above, blog
entry {\bf 2017-03-23} Matt wrote:
``
Also a note on symmetry, the way that J.~F. Gibson
\HREF{www.channelflow.org}{Channelflow} handles the spatial and temporal
translation symmetry is to constrain the Newton steps to only progress in
directions transverse to the spatial and temporal equivariance tangent
directions.
''

This is not a symmetry reduction.
Separating the flow {\em locally} into group dynamics and a transverse,
`horizontal' flow,\rf{Smale70I,AbrMars78} by the `method of
connections',\rf{rowley_reduction_2003}, does not reduce the dynamics to a
lower\dmn\ \reducedsp\ $\pS/\Group$.
In contrast to the method of co-moving frames, where one defines a mean
{\phaseVel} of a \rpo, the method of connections is inherently local. The two
methods coincide for relative equilibria.

This is explained many places:
{\bf 2013-09-19} entry in
\texttt{siminos/blog/},
{\bf 2013-10-28}  entry in
\texttt{pipes/blog/} (there is actually a whole chapter there, currently commented,
on getting Kreilos and Eckhardt to understand that, and correct their
Kreilos, Zammert and Eckhardt\rf{KrZaEc14}
{\em Comoving frames and symmetry-related motions in parallel shear flows}
prior to publication),
sect.~VI. {\em Bridges to nowhere}  in \refref{atlas12},
sect.~3.1.
{\em Method of connections} in Budanur \etal\rf{WFSBC15}
{\em Relative periodic orbits form the backbone of turbulent pipe flow},
and towards the end of the very scholarly
\HREF{http://ChaosBook.org/slice.pdf} {Remark~13.1}
{\em A brief history of relativity,
or, `Desymmetrization and its discontents'}.

If that is what \HREF{www.channelflow.org}{Channelflow} still does, please
alert John that this is wrong; give him a pdf printout of this blog, so he can
reread the references himself. And please do tell me whether you had any luck
communicating with him about this; it is important to us, because fluid
dynamics community has by now learned a bit about {\po}s, but almost nothing
about the necessity of symmetry reduction for analysis of turbulent flows.
    }

\PCpost{2017-11-03} {
I find it very hard to understand mathematical physics when it it is only
described in words, without formulas. As an example of the way I would blog an
article that I am reading it,  I am starting here a discussion of L\'{o}pez
articles.
In this case writing formulas is easy, as one can download her source files
from \arXiv{1502.03862}. In this way one can discuss the particular step in
her calculations by referring to the formula she is using.
    }

\PCpost{2017-11-03} {
\phantomsection\label{2017-03-14MNG}
Matt mentioned that he is rereading Vanessa L\'{o}pez paper\rf{lop05rel},
because she found {\rpo}s of \cGL\ using spatiotemporal methods
(He had already discussed the paper in blog entries {\bf 2017-03-14} and
{\bf 2017-03-23} in \refchap{chap:blogMNG}, though not in any detail.).
My bad - I simply paid no attention to their numerical method.

L\'{o}pez has since written a more detailed paper\rf{Lopez2015} on her PhD work.
We usually cite L{\'o}pez, Boyland, Heath and Moser\rf{lop05rel} {\em Relative
periodic solutions of the complex {Ginzburg-Landau} equation} for being the
first to determine {\rpo}s in a spatiotemporal PDE, though we had never used her
method of finding {\rpo}s. (I do object to her using a nonsensical formula from
literature to average over {\rpo}s, but that is unrelated to the problem of
finding them.)

Indeed, in both papers she discretizes using Fourier series expansions in both
space and time, in order to derive an underdetermined system of nonlinear
algebraic equations from which invariant solutions of the {\cGLe} are sought.
That is described in sect.~3 {\em Numerical Method} of \refref{Lopez2015}.

She first defines the symmetries of the problem:

The \cGL\ have a three-parameter group\rf{AKcgl02}
    \PC{2018-03-20}
{She thinks of a ``group'' not as a collection of group elements,
but as its parameter space. In \refeq{eqn:Ggroup} $\reals$ refers to
time $t\in(-\infty,\infty)$, and $\torus^2$ refers to
the complex phase $\theta \in (0,2\pi)$, and the configuration space restricted
to a periodic domain of length $x \in (0,\Lx)$. Remember, for us also
$x\in(-\infty,\infty)$.
}
\begin{equation}    \label{eqn:Ggroup}
    \cglegroup = \torus^2 \times \reals
\end{equation}
of continuous symmetries generated by spacetime translations $x \rightarrow x
+ \spacetrans$, $t \rightarrow t + \timetrans$ and a rotation $A \rightarrow
\e^{\ii\Arot}A$ of the complex field $A(x, t)$, in addition to being invariant
under the action of the discrete group of transformations $A(x,t) \rightarrow
A(-x,t)$ of spatial reflections.  If $A(x, t)$ is a solution,
% of equations \refeqs{eqn:cgle_pde}{eqn:cgle_bcs}, then
so are
\begin{align}
    \e^{\ii\Arot} A(x, t)&,            \label{cglesym1} \\
    A(x + \spacetrans, t)&,       \label{cglesym2} \\
    A(x, t + \timetrans)&,          \label{cglesym3} \\
    A(-x, t)&,                              \label{cglesym4}
\end{align}
for any $\LieEl(\Arot, \spacetrans, \timetrans) \in \cglegroup$.
For a given solution $A(x,t)$ of the {\cGLe}, consider the
isotropy subgroup $\cgleisotropy{A}$ of $\cglegroup$ at $A$
    \PC{2018-03-20}
{Looks like she is defining a triply-periodic \rpo\ $A(x,t)$?},
\begin{equation}     \label{eqn:isotropy_subgroup}
    \cgleisotropy{A}  =  \{ (\varphi, S, T) \in \cglegroup \ \, | \ \,  A(x,t) =  \e^{\ii \varphi} A(x+S, t+T) \},
\end{equation}
which consists of elements of the symmetry group $\cglegroup = \torus^2 \times
\reals$  leaving the \cGLe\ invariant.

L\'{o}pez seeks solutions $A(x,t)$ of the \cGLe\ satisfying
\begin{equation}   \label{eqn:cgle_invariant_solution}
    A(x,t)  =  \e^{\ii \varphi} A(x+S, t+T),
\end{equation}
for $(\varphi, S, T) \in \cglegroup$ also unknown and to be determined.

She represents $A(x,t)$ as a spatial Fourier series
\begin{equation}  \label{eqn:xFseries}
     A(x,t)   =  \sum_{m \in \integers} a_{m}(t) \e^{\ii \wavek x},
\end{equation}
where $\wavek = 2\pi m / \Lx$ denotes the $m$-th wavenumber in the expansion. From
the group-invariance condition \refeq{eqn:cgle_invariant_solution} it then
follows that the com\-plex-valued Fourier coefficient functions $a_{m}(t)$ in
\refeq{eqn:xFseries} satisfy
\begin{equation}   \label{eqn:am_invariant}
a_{m}(t)  =  \e^{\ii\varphi} \e^{\ii \wavek S} a_{m}(t +T)
\end{equation}
for all $m \in \integers$.  Because of the presence of spatial translational symmetry
% \refeq{eqn:cgle_bcs},
the solutions sought can be restricted to those with
elements $\LieEl(\varphi,S,T) \in \cglegroup$ having $S \in [0,\Lx)$.

{\bf \On{2} symmetry}.
Since
the {\cGLe} is invariant under the action of the group $\integers_2$ of spatial
reflections $A(x,t) \rightarrow A(-x,t)$, to any solution $A(x,t)$ of the
{\cGLe} having $(0,\Lx,0)$ and $(\varphi,S,T)$ as generators of subgroups of the
isotropy subgroup $\cgleisotropy{A}$
%(defined in \refeq{eqn:isotropy_subgroup})
there corresponds a solution $\tilde{A}(x,t) := A(-x,t)$ having $(0,\Lx,0)$ and
$(\varphi,\Lx\!-\!S,T)$ as generators of subgroups of the isotropy subgroup
$\cgleisotropy{\tilde{A}}$.
[... some details of incorporating the reflection symmetry we should also study ...]
She calls the invariant
solutions $(A; \varphi, \Lx/2 \pm \delta, T)$ and $(\tilde{A}; \varphi, \Lx/2
\mp \delta, T)$, as well as their corresponding orbits $\cglegroup \cdot A$ and
$\cglegroup \cdot \tilde{A}$, \textit{conjugate} to each other under the
(involutive) action of the group $\integers_2$ of spatial reflection symmetry
of the {\cGLe}.

She writes: ``The {\cGLe} may admit solutions having symmetries other than (or
in addition to) that defined by \refeq{eqn:cgle_invariant_solution} and
several of the solutions resulting from our study do have additional
symmetries.
For instance, there may exist solutions of the {\cGLe} satisfying''
\begin{eqnarray}
     A(x,t)  & = &   \e^{\ii 2\pi / \lsym} A(x + \Lx / \lsym, t),     \hspace{2.25em} \mbox{for some $\lsym \in \naturals$, $\lsym > 1$,}
                       \label{additionalsym1}  \\
     A(x, t)  & = &   \hspace*{1em}A(-x + 2\ce, t)          \hspace{3.75em} \mbox{for some $\ce \in \reals$,}
                       \label{additionalsym2}  \\
     A(x, t)  & = &   -A(-x + 2\co, t)                                  \hspace{4.00em} \mbox{for some $\co \in \reals$.}
                       \label{additionalsym3}
\end{eqnarray}
The first one is standard:
\refeq{additionalsym1} describes solutions fixed by a composition of
the actions \refeq{cglesym2} and \refeq{cglesym1}, and gives $(2\pi / \lsym,
\Lx / \lsym, 0)$ as one generator of a subgroup of $\cgleisotropy{A}$.
Symmetries \refeq{additionalsym2} and \refeq{additionalsym3} might be our
\ppo s, they are, respectively, even about $x = \ce$ or odd
about $x = \co$ for some real numbers $\ce, \co$.  Presumably awkwardly placed
symmetry points in a conjugacy class, should all really be conjugated to the
standard origin $x=0$.

A solution having both symmetries \refeq{additionalsym1} and
\refeq{additionalsym2} also satisfies
\begin{equation}    \label{eqn:lsym_even}
    A(-x + 2(\ce + \Lx/(2\lsym)), t)  = \e^{\ii 2\pi / \lsym} A(x, t) \, .
\end{equation}
In particular, note that a solution satisfying \refeq{additionalsym1} for
$\lsym = 2$ and which is even about $x = \ce$ is also odd about
$x =\co=\ce+\Lx/4$.

Since the boundary conditions %\refeq{eqn:cgle_bcs}
are periodic in $x$, she usea
the spatial Fourier series \refeq{eqn:xFseries} and substitutes into the
{\cGLe} %~\refeq{eqn:cgle_pde}
to obtain an infinite system of ordinary
differential equations ({ODEs}),
\begin{equation}   \label{eqn:odes}
    \frac {\dd a_{m}}{\dd t}  =  R a_m - \wavek^2 (1 + \ii\nu) a_m - (1 + \ii\mu) \sum_{m_1+m_2-m_3=m}
	a_{m_1} a_{m_2} a_{m_3}^{*},
\end{equation}
for the complex-valued functions $a_{m}(t)$.
Under this transformation the symmetries \refeqs{cglesym1}{cglesym4} of
{\cGLe} %\refeqs{eqn:cgle_pde}{eqn:cgle_bcs}
become symmetries of~\refeq{eqn:odes}.  Thus, if
$\mmbf{a}(t) = (a_{m}(t))$ is a solution of the
system of {ODEs} \refeq{eqn:odes}, then so are
\begin{align}
    (\e^{\ii\Arot} a_{m}(t))&,                    \label{odesym1} \\
    (\e^{\ii m\spacetrans} a_{m}(t))&,   \label{odesym2} \\
    (a_{m}( t + \timetrans))&,                 \label{odesym3} \\
    (a_{-m}( t))&,                                     \label{odesym4}
\end{align}
for any $(\Arot, \spacetrans, \timetrans) \in  \torus^2 \times \reals$.
In particular, \refeq{odesym1} and \refeq{odesym2} say that the
{ODEs}~\refeq{eqn:odes} are invariant under the $\torus^2$-action
\begin{equation*}  \label{eqn:torusact}
    (\Arot, \spacetrans)\cdot (a_{m}(t))  = (\e^{\ii\Arot} \e^{\ii m \spacetrans} a_{m}(t)).
\end{equation*}
She employs a spectral-Galerkin projection obtained by fixing an even number
$N_x$ and truncating the expansion \refeq{eqn:xFseries} to include only the
terms with indices $m$ satisfying $-N_x/2+1 \leq m \leq N_x/2-1$.
Both theory and computation\rf{doelman89, JoTeXi95} shows that for sufficiently
large $N_x$ the behavior of this truncation captures the essential features of
the {\cGL} dynamics. % of~\refeqs{eqn:cgle_pde}{eqn:cgle_bcs}.

From the condition \refeq{eqn:cgle_invariant_solution} defining an invariant
solution of the {\cGLe}, it follows that the corresponding solution
$\mmbf{a}(t)$ of the system of {ODEs} \refeq{eqn:odes} satisfies
\begin{equation}   \label{eqn:am_rpo}
    a_k(t)  =  \e^{\ii\varphi} \e^{\ii \wavek S} a_{m}(t +T)
\end{equation}
for all $m$ and $t$ (and where $\varphi, S, T$ are to be determined).

As Matt pointed out in a conversation, a solution of the system of functional
equations \refeq{eqn:am_rpo} can be expressed in the spacetime Fourier-Fourier
basis for \refeq{eqn:odes} as
\begin{equation}   \label{eqn:am_ansatz}
    a_k(t)  =  \e^{-\ii\frac{\varphi}{T}t} \e^{-\ii \wavek \frac{S}{T}t}
            \sum_{j \in \integers} \akj \e^{\ii \omegaj t} \, ,
\end{equation}
where $\omegaj = 2 \pi n / T$ denotes the $n$-th frequency in the expansion. It
might be most economical to do for \KS\ precisely what L\'{o}pez did for \cGL.
It should be easier, as there is one less continuous symmetry.
    }

\SFIG{LauretteTimings}{}{
A Chanellflow \eqv\ solution comparison between Laurette's Stokes method (in red) and the
standard method (in blue).
}{fig:LauretteTimings}

\PCpost{2017-11-13} {
As fate would have it, I talked to Laurette Tuckerman, and she reminded me of her method
of determining \eqva\ and \reqva\ that is some 20 times faster than the
usual Newton, and it reminds me very much of Matt's approach.
The references are
\HREF{https://www.pmmh.espci.fr/~laurette/highlights/BifAnalTimeStep.pdf}
    {BifAnalTimeStep.pdf},
\HREF{https://www.pmmh.espci.fr/~laurette/papers/mamun.pdf}
    {mamun.pdf},
\HREF{https://www.pmmh.espci.fr/~laurette/papers/timesteppers.pdf}
    {timesteppers.pdf},
and
\HREF{https://www.pmmh.espci.fr/~laurette/papers/invpow_CICP.pdf}
    {invpow\_CICP.pdf}.

\refFig{fig:LauretteTimings} is an example of (to me very impressive)
convergence acceleration.
    }

\PCpost{2018-02-06}{
I went to 5 (five!) seminars, colloquia and public lectures yesterday, plus
spent couple of hours working with students one-on-one. Twitter is aflame
with a debate whether professors really pull 60-hour weeks. Not sure that
counts as `work' (or, what in the notes that follow we call `action'),
especially the two math seminars that made impenetrable the topics I
currently work on but - that might be counted as `relaxation', but it sure
was many hours. The second seminar of the day, while I still had some neurons
lighting up, was {\em Discrete stochastic Hamilton-Jacobi equation} by
Renato Iturriaga, CIMAT. He is a collaborator Georgia Tech's A. Fathi.
This should be a flip side of Marsden's discrete Lagrangian methods, to
be blogged here at some point.
 Here it goes:
\bea
\mbox{Lagrangian   } L &:& T T^d \to \reals
\label{Iturriaga1a}\\
\mbox{Hamiltonian  } H &:& T^* T^d \to \reals
\label{Iturriaga1b}
\,.
\eea
The dynamics is on the tangent bundle. $T^*$ means symplectic structure.
Minimizing $L$ gives Euler-Lagrange equation (ODE).
Minimizing $H$ gives  Hamilton-Jacobi equation (PDE).
Can discretize Hamilton-Jacobi in two ways (1) add viscosity, (2) add discount factor (for
`discount factors' see, for example,
\HREF{https://arxiv.org/abs/1801.06008}
{here}):
\beq
H(x,d_xU) = \Delta U + \epsilon
\,.
\ee{Iturriaga2}
A Lax-Oleinik transformation gives a unique solution of \refeq{Iturriaga2}.
(And so on, but I give up - have another 17 things awaiting me.)
}

\PCpost{2018-02-06}{
I ran into an interesting discussion by
Miles Stoudenmire:
    {\em
Periodic vs Open or Infinite Boundary Conditions for DMRG,
\HREF{http://itensor.org/docs.cgi?page=articles/periodic}
{Which Should You Choose?}
    }
Stoudenmire says that when open BC achieves a given accuracy when keeping $m$
states, then to reach the same accuracy with periodic BC one must keep $m^2$
states.
I find it interesting, because I have a strong prejudice for periodic BC, but
it really should not matter - if you are doing a large scale turbulence
simulation free BC might make sense, if that decreases the cost of the
computation.
    }

\PCpost{2018-02-26}{
A snippet from internet:

Topological quantum computing (TQC) is a newer type of quantum computing that
uses ``braids'' of particle tracks, rather than actual particles such as ions and
electrons, as the qubits to implement computations. Using braids has one
important advantage: it makes TQCs practically immune to the small
perturbations in the environment that cause decoherence in particle-based
qubits and often lead to high error rates.
}

\PCpost{2018-03-09}{
I uploaded my
\HREF{https://absuploads.aps.org/presentation.cfm?pid=13452}
{APS March Meeting 2018 slides}.
}

\PCpost{2018-03-12}{
We are not alone. It is an
\HREF{https://sinews.siam.org/Details-Page/self-organization-in-space-and-time}
{obvious idea}.

Kevin O'Keeffe, Hyunsuk Hong and Steven Strogatz of Cornell University,
developed a `swarmalator' model, consisting of oscillators whose phase
dynamics and spatial dynamics are coupled, leading to simultaneous
spatially-coordinated and synchronous behavior\rf{KeHoSt17}.
They say that `synchronization' is self-organization in time, such as
activation in heart cells, and that `aggregations' is self-organization in
space, like alignment of electron spins in magnetic material.
In the synchronized state, the individual cells coordinate the timing of
their oscillations, but they do not move through space.
In swarming individuals move through space, but without conspicuously
altering their internal states.

Now, what they actually do is very much in the style of the Winfree,
Kuramoto, Strogatz' life work (fireflies in synch, etc.) and it might have too
much detail of that kind for us to truly enjoy the work. They claim that
``insights from biological synchronization have shed light on neutrino
oscillations, phase locking in Josephson junction arrays (that one with Kurt
as coauthor), ...,'' but I doubt that.

Say what you want, but they sure get the prize for the ugliest new
terminology: ``A rich phenomenology is expected for mobile oscillators whose
phases affect their motion. We call these hypothetical systems `swarmalators'
because they generalize swarms and oscillators.''

As they work with $N$ `particles' in continuous time and space, the simplest
model is $2N$ all-to-all coupled ODEs (like Kuramoto models), a bit
complicated at get go. Emphasis on ``space'' and ``time'' is fraudulent: what
they really mean is that space dynamics of $N$ interacting ``particles'' is
coupled to the internal dynamics of these $N$ ``particles.'' It will provide
employment for folks who have run out of ideas in their work on Kuramoto
systems. In other words, this is just an ordinary, time evolving dynamical
$2N$\dmn\ system, where the physical interpretation of the first $N$
dimensions differs from the second $N$ dimensions. It is not even
Hamiltonian, or anything. They do not use symmetries, even permutation ones.
For us, a downer.
    }

\item[2018-02-06 Alex Haro]
You can compute Lyapunov exponents of the tangent dynamics to the torus.
If you want something more explicit, as eigenvalues and ``eigenvectors"
(technically invariant bundles), you can discretize what is known as
transfer operator. Angel Jorba has a paper on this brute force method.
Rafael de la Llave, I. G. Kevrekidis and R. A. Adomaitis\PC{2018-03-20}
{
I was not able to pinpoint the particular references. Perhaps
Adomaitis, Kevrekidis and de la Llave\rf{AdKeLl07}
{\em A computer-assisted study of global dynamic transitions for a
noninvertible system}
or
{\em Predicting the complexity of disconnected basins of attraction for a
noninvertible system}, a technical report from 1991.
}
used this method many years ago.
In my papers with Rafael (see also chapter 3 of our book\rf{HCFLM16} {\em The
parametrization method for invariant manifolds}) we use several efficient
methods.

You could ask Rafael, or some PhD student from Barcelona, as Joan Gimeno.

\PCpost{2018-03-20}{ We can track the references through their recent
numerical paper, Canadell and Haro\rf{CanHar17}
{\em Computation of quasi-periodic normally hyperbolic invariant tori:
{Algorithms}, numerical explorations and mechanisms of breakdown}.
From the abstract: ``
We present several algorithms for computing normally hyperbolic invariant
tori [...]. The algorithms use different hyperbolicity and
reducibility properties and compute also the invariant
bundles and Floquet transformations.
''
}

\PCpost{2018-04-12}{Stephen Wolfram writes:

Cool.  I like your idea (if I understood it) of making symbolic dynamics out of
spacetime lumps of PDEs. (somewhat reminiscent of my
\HREF{http://www.wolframscience.com/nks/notes-6-7--spacetime-patches-in-cellular-automata/}
{spacetime patches}.

Simpler than \KS, but seemingly just as ``turbulent'' is
\HREF{http://www.wolframscience.com/nks/p165--partial-differential-equations/}
{my PDE}. I've been asking PDE people about this equation for years … nobody
has ever told me anything interesting about it, beyond what's already in
\HREF{http://www.wolframscience.com/nks/notes-4-9--equation-for-the-background-in-my-pdes/}
{my notes}.

P.S.  Common response from kids: ``What a strange coincidence that your name is
the same as the name in Wolfram|Alpha.  Oh, and that your computer has that
same logo on it.''  :)
}

\PCpost{2018-05-02}{
    In 2012 Wilczek hyothesized existence of `Time crystals', see
\HREF{http://gizmodo.com/scientists-finally-observed-time-crystals-but-what-the-1793061377}
{...observed time crystals...} and
\HREF{https://physics.aps.org/synopsis-for/10.1103/PhysRevLett.120.180603}
{``Time Crystals Multiply''}:
``
Time crystals are different from other time-periodic systems (like pendula
and beating hearts) in that they don't move to the rhythm set by their
driving mechanism. Instead, they oscillate with a period that is an integer
multiple of the driving period. In addition, according to most models, the
periodic driving would overheat the crystal, and a discrete time crystal
could only exist if the system is stabilized against heating by a
disorder-induced phenomenon called many-body localization.
''

I do not think we will run into them, because one needs higher time derivatives,
see \refeq{SacZak17(10)}, but I am not sure.
The credit for looking at things spatiotemporal globally perhaps goes back to
Lagrange (the action is the cost function for mechanical systems). Any time
you have a stable limit cycle you have a time crystal (they call that
``spontaneous time-symmetry breaking", but chaotic dynamics selects an
infinite set of periods (periods of unstable periodic orbits, generically not
rationally related). But I keep looking at these papers, such as
Sacha and Zakrzewski\rf{SacZak17}
{\em Time crystals: a review}. Who knows, we might learn something.
They write:

Time crystals are time-periodic self-organized structures postulated by Frank
Wilczek in 2012. Discrete (or Floquet) time crystals are structures that
appear in the time domain due to spontaneous breaking of discrete time
translation symmetry. The struggle to observe discrete time crystals is
reviewed here together with propositions that generalize this concept
introducing condensed matter-like physics in the time domain. We review
strategies aimed at spontaneous breaking of continuous time translation
symmetry.

Switching from space to time crystals exchanges the role of space and time.
In the space crystal case we expect periodic behaviour in space at a fixed
instant of time (i.e. at the moment when we perform a measurement of a
system) while in the time crystal case we fix the position in space and ask
whether a detector clicks periodically in time.

Assume the energy of a particle of the form
\beq
E =
\frac{\dot{x}^4}{4} - \frac{\dot{x}^2}{2}
\,.
\label{SacZak17(10)}
\eeq
The lowest energy corresponds to particle motion with velocity $\dot{x} = \pm
1$. Note that the energy \refeq{SacZak17(10)} cannot be converted to the
Hamiltonian smoo\-th\-ly: the Hamiltonian is a multi-valued function of the
momentum with cusps corresponding precisely to energy minima at $\dot{x} =
\pm 1$ where the Hamilton equations are not defined (Shapere and
Wilczek\rf{ShWi12}).

The quantum version is based on taking a time-periodic (Floquet) Hamiltonian
\(H(t+T)=H(t)\)
and then working with the complete set of Floquet eigenstates
\(
\ket{u_n(t+T)}=\ket{u_n(t)}
\,,
\)
quasi-energies $E_n$,
\[
\ket{\psi(t)}=\sum c_n e^{-iE_n t}\ket{u_n(t)}
\,.
\]
Quasi-energy spectrum is not bounded from below. It is periodic with a period
$2\pi/T$ and it is sufficient to consider only a single Floquet zone in order
to fully describe a system, in analogy to a Brillouin zone in condensed
matter physics.

The general motivation is the same as our cat map discussion -
they consider a rotor motivated by a Rydberg electron perturbed by a microwave
field. Then they relate the quantum version to a tight-binding model.
}

\PCpost{2018-07-10}{
Of possible interest to Matt, as we will have to continue our \twots:
 Engelnkemper \etal\rf{EGUWT19}, \arXiv{1808.02321},
{\em Continuation for thin film hydrodynamics and related scalar problems}.
They write
``
[...] how to apply continuation techniques [...] applied to a number of
common examples of variational equations, namely, Allen-Cahn- and
Cahn--Hilliard-type equations including certain thin-film equations for
partially wetting liquids on homogeneous and heterogeneous substrates as well
as Swift--Hohenberg and Phase-Field-Crystal equations. Second we consider
nonvariational examples as the Kuramoto--Sivashinsky equation [...]. Through
the different examples we illustrate how to employ the numerical tools
provided by the packages auto07p and pde2path to determine steady, stationary
and time-periodic solutions in one and two dimensions and the resulting
bifurcation diagrams.
''

%I cannot access it through GaTech login in, can you?
%\MNG{2018-07-10}{
%No I cannot; in fact, it doesn't even appear when I search directly through
%the GAtech library website.}
%
%\HREF{https://www.uni-muenster.de/Physik.TP/en/institute/members/svetlana_gurevich.html}
%{Svetlana V. Gurevich} does not have her papers online; mostly does optics, condensed matter;
%Try requesting the full text from
%\HREF{https://www.researchgate.net/publication/326251733_Continuation_for_Thin_Film_Hydrodynamics_and_Related_Scalar_Problems}
%{here}?
}

\PCpost{2018-08-19}{
I have the \po\ theory formula for an \eqv\ point (for the last 20-30 years!)
in the boyscout version of ChaosBook (current sect.~22.3 {\em Equilibrium
points}), but do not know what to do with it.

Very frustrating.

Now it is pressing - in the spatiotemporal formulation of turbulence the zeta
functions (Fredholm determinants) are presumably 2-d or (1+3)-d Laplace/Fourier
transforms of trace formulas, one dimension for each continuous symmetry: one
Laplace transform for time, and one Fourier transform for each infinite spatial
direction.

We have not written either the trace or the determinant formulas yet. The
\catlatt\ periodic points (\twots) counting suggests a way, so far unexplored.

Or, a deeper insight: in the spatiotemporal formulation of turbulence there are no
\po s, as there is no evolution, neither in space nor in time. All
solutions are fixed points, and the important measure is not the natural
(Ruelle-Bowen-Sinai) infinite time measure, but the measure concept is more
like the stat mech understanding of the Ising model - what is the likelihood of
occurrence of a given spacetime configuration (admissible by the defining
equations of the system)?
}

\PCpost{2018-08-21}{
Dong\rf{Dong18}
{\em Organization of the periodic orbits in the {R{\"o}ssler} flow}
 writes:
``Numerical implementation of the variational method
[...]
a finite difference scheme is used to obtain accurate discrete loop
derivatives. We use the five-point approximation.
[that is of interest to us]
We invert the matrix using the banded LU decomposition on the embedded
band diagonal matrix, and treat the cyclic and border terms with the
Woodbury formula.''

Have to read these as well (Matt has had a look at Dong\rf{Dong18c},
did not see how to apply it to his own project):

Dong, Wang, Du, Uzer and Lan\rf{DWDUL16}
{\em The ionized electron return phenomenon of {Rydberg} atom in crossed-fields}

Dong\rf{Dong18c} {\em Topological classification of periodic orbits in the
{Kuramoto-Sivashinsky} equation}

Dong\rf{Dong18a}
{\em Topological classification of periodic orbits in {Lorenz} system}

Dong\rf{Dong18b}
{\em Topological classification of periodic orbits in the {Yang-Chen} system}
}

    \PCpost{2018-09-02}{
Ziessler, Dellnitz and R. Gerlach\rf{ZiDeGe18} {\em The numerical
computation of unstable manifolds for infinite dimensional dynamical
systems by embedding techniques}, \arXiv{1808.08787}: `` we extend the
novel framework developed by Dellnitz, Hessel-von Molo and
Ziessler\rf{DeHeZi15}, \arXiv{1508.07182}, to the computation of finite
dimensional unstable manifolds of infinite dimensional dynamical systems.
To this end, we adapt a set-oriented continuation technique for the
computation of such objects of finite dimensional systems. We show how to
implement this approach for the analysis of partial differential
equations and illustrate its feasibility by computing unstable manifolds
of the one-dimensional Kuramoto-Sivashinsky equation as well as for the
Mackey-Glass delay differential equation.
''
    }

    \PCpost{2018-10-06}{A summary of
Wang, Wang and Lan\rf{WaWaLa18}
{\em Accelerated variational approach for searching cycles}.

Lan and Cvitanovi{\'c}\rf{lanVar1} variational approach eliminates most
\Poincare\ sections by discretizing continuous time evolution into small time
steps. The approach requires storage of an entire trajectory in the computer
memory, so the computational load is proportional to the number of
discretization points. In order to extend the method to determination of
connecting (homo/hetero-clinic) orbits, Dong and Y. Lan\rf{DoLa14a} designed an
automatic mesh allocation algorithm which makes the guess points evenly
distributed in arc length instead of in time, and thus avoids their
accumulation near the ends of the connection, where the flow is exponentially
approaching zero. In this paper they introduce several new schemes to allocate
mesh points in different situations.

In their calculations of \po s, they approximate the loop derivatives by a
five-point banded matrices, with the periodic boundary condition.

Following Zhou,  Ren  and E\rf{ZhoRenE08}, they reparametrize the loop
parameter by a desired density of points (their eq.~(6)), for example uniform
in time, or uniform in arc length, and derive an effective equation
to automatically allocate lattice points which capture local fine
orbit structures, while keeping the exponential convergence of the variational
approach.

The time-translational invariance of a \po\ leads to neutral direction in the
update of the coordinates\rf{lanVar1}, which is eliminated by a gauge-fixing
condition\rf{KawKida01,DoLa14a,ZhoRenE08}. They use the usual \Poincare\
section plane condition (below their eq.~(6)). A conservation law may supply
another neutral direction for the variation. They use a two-dimensional
Hamiltonian energy conservation neutral direction as an example of a way to
remove such neutral direction by adding a Lagrange multiplier (their eq.~(16)),
while keeping the dynamics on the desired energy surface.

Utilizing the special structure of the matrix involved in the homotopy
evolution, the lower-upper (LU) decomposition is implemented with gre\-at\-er
efficiency and less memory. Their number of unknown elements in L and U is
proportional to $Nd$ instead of $(Nd)^2$ as in the usual LU decomposition,
which greatly speeds up the computation.

Three examples are used to demonstrate the validity of the accelerated
algorithm.

Left for the future:
\begin{enumerate}
  \item
The computation of the Jacobian in high-dimensional systems is time and
resource consuming. How to overcome this difficulty remains a major challenge.
  \item
The computation of the weight function is quite cumbersome. There may exist an
equivalent way for redistributing the lattice point but with much lighter
computation load.
\end{enumerate}

If you want to discuss this with Lan, we can get him online any time. Unfortunately
Georga Tech prevents him from coming here
February 2019, or perhaps as long as we live in Trumplandia.
    }

    \item[2019-02-03 John Gibson]
I met Datseris at the last JuliaCon and attended his talk. His

{\em {DynamicalSystems}.jl:\\ {A Julia} software library for chaos and nonlinear
dynamics}\rf{Datseris18}

and

{\em {DynamicalBilliards}.jl:\\ An easy-to-use, modular and extendable {Julia}
package for dynamical billiard systems in two dimensions}\rf{Datseris17}

are really spectacular packages. You should assign a student to use them in a
project.

Any volunteers? I have a dynamite (but difficult) turbulence project for you :)

His JuliaCon talk is here
\videoLink{www.youtube.com/watch?v=7y-ahkUsIrY&t=0s&list=PLP8iPy9hna6Qsq5_-zrg0NTwqDSDYtfQB&index=70}
(click!).
John is in the second row to the left, light blue shirt.

    \item[2019-02-21 John Gibson]
Saw this on Julia Discourse this morning: an interactive tool for
exploring dynamical system

\HREF{https://discourse.julialang.org/t/announcing-interactivechaos/21046}
{discourse.julialang.org/t/announcing-interactivechaos/21046}

\HREF{https://juliadynamics.github.io/DynamicalSystems.jl/dev/\#interactivechaos}
{juliadynamics.github.io/DynamicalSystems.jl/dev/\#interactivechaos}

    \item[2019-02-25 Predrag]
As fate would have it, the new theoretical condensed matter physics
assistant professor here,
\HREF{https://www.glenevenbly.com/} {Glen Evenbly}
<glen.evenbly@gmail.com>
does all his tensor networks computations in Julia. Han Liang
visited him, explained the \catlatt, and they were both happy.

    \item[2019-03-24 John Gibson] on defining the \KS\ time scale:
\[
u_t = - u_{xx} - u_{xxxx} - u u_x
\]
Linear stability around $u=0$ gives a maximally unstable mode
    \PC{2019-03-25}{I think he means $\exp(\alpha t)$.}
$\exp(i\alpha t)$ with $\alpha = 1/\sqrt{2}$ and
wavelength $L = 2\pi/\alpha = 2 \pi\sqrt{2}$.

We get a wavespeed $c \approx 3$ from the nonlinear wave term $-u u_x$,
and the fact that $u$ saturates at about $|u| = 3$.
That gives a nonlinear time scale of about $t=3$ from
$t = L/c = 2 \sqrt{2} \pi / 3 \approx 3$.

Seems to be about right from viewing simulations. I don't have a good
argument why $u$ saturates at  $|u| = 3$.

    \item[2019-02-26 Predrag]
Objection! We know that as $\speriod{}$ grows, the flame front
$u(\conf,\zeit)$ is a random walk in $u$ with mean
$\spaceAver{u}(\zeit)=0$ (we used Galilean invariance to enforce that)
and the variance $\spaceAver{u^2}(\zeit)$ (the ``kinetic energy'') growing
linearly with $\speriod{}$ (have a look at the color bar units in Matt's
\twots), so $|u|$ certainly does not saturate at 3. This global argument
has to be turned into some local rate of growth of $|u|$, perhaps
$|u_x|$. The steepness of kinks in the flame front might be bounded by
the hyperviscosity...

    \item[2019-02-27 Predrag]
\HREF{http://cmsa.fas.harvard.edu/machine-learning/} {Machine learning} workshop:
\\
I moved my notes to \emph{pipes/blog/dailyBlog.tex}

    \item[2019-02-27 Predrag]
Ispolatov\rf{IMAD15}
{\em Chaos in high-dimensional dissipative dynamical systems}
cites a huge amount of literature that studies high dimensions
and might have to be cited in our work.

They solve numerically coupled systems of equations which contain second-
and third-order nonlinear terms, with coefficients drawn from Gaussian
distributions with zero mean and unit variance.

They find that the probability of chaos increases with the dimension of
the phase space and that, essentially all trajectories become chaotic for
$d>50$, while for intermediate dimensions $d\succeq 15$, the majority of
chaotic trajectories essentially fill out the available phase space.

[...] We assume that for sufficiently high $d$, all Jacobian eigenvalues
are statistically independent. This assumption [...] is a rather
strong approximation without which it seems impossible to derive
analytical estimates, and which seems to result in reasonable results.

\PCpost{2019-05-19}{SIAM DS19 talk by
Joanna Slawinska, Abbas Ourmazd, Dimitrios Giannakis,
and Joerg Schumacher
{\em Vector-Valued Spectral Analysis of Complex Flows} have
been doing this for a while, and a bunch of old literature comes with it...

Vector-Valued Spectral Analysis (VSA) is a recently developed
framework\rf{GOSW19} for spatiotemporal pattern extraction, based on the
eigendecomposition of a kernel integral operator acting on vector-valued
observables (spatially extended fields) of the dynamical system
generating the data, constructed by combining elements of the theory of
operator-valued kernels for multitask machine learning with
delay-coordinate maps of dynamical systems. The method utilizes a kernel
measure of similarity that takes into account both temporal and spatial
degrees of freedom (whereas classical techniques such as EOF analysis are
based on aggregate measures of similarity between ``snapshots''). As a
result, VSA extracts physically meaningful patterns with intermittency in
both space and time, while factoring out any symmetries present in the
data.
}

\PCpost{2019-06-01}{
Giannakis \etal\rf{GOSW19} {\em Spatiotemporal pattern extraction by
spectral analysis of vector-valued observables}:

``Vector-valued spectral analysis (VSA) is based on an eigendecomposition
of a kernel integral operator acting on a Hilbert space of vector-valued
observables of the system.
[...]
conventional eigendecomposition techniques decompose the input data into
pairs of temporal and spatial modes with a separable, tensor product
structure.
The patterns recovered by VSA can be manifestly non-separable, requiring
only a modest number of modes to represent signals with intermittency in
both space and time.
The kernel construction naturally quotients out dynamical symmetries in
the data and exhibits an asymptotic commutativity property with the
Koopman evolution operator of the system, enabling decomposition of
multiscale signals into dynamically intrinsic patterns. Application of
VSA to the Kuramoto-Sivashinsky model demonstrates significant performance
gains in efficient and meaningful decomposition over eigendecomposition
techniques utilizing scalar-valued kernels.
[...]
the techniques described above recover from the data a set of temporal
patterns and a corresponding set of spatial patterns, sometimes referred
to as ``chronos'' and ``topos'' modes, respectively.
[...] ''

This looks like something it will take time to digest, but digest we must...
}

\PCpost{2020-01-22}{
Well, if we are going to have a revolution, nothing will remain
untouched. In particular, we have to give up on the forward-in-time
\HREF{http://chaosbook.org/chapters/ChaosBook.pdf\#section.20.2}
{\evOper s}, replace them by enforcing infinitesimal
\HREF{http://chaosbook.org/chapters/ChaosBook.pdf\#section.19.5}
{continuity equation} at any  spacetime instant.

Scary, as the solutions are supposed to be invariant spacetime measures,
and those are horrible nowhere differential beasts. But \twot s should
save the day.
    }

\PCpost{2020-08-04}{
When it comes to proofs that solutions like our \KS\ exist,
Akitoshi Takayasu want us to refer to 3 of his papers:
``A method of verified computations for solutions to semilinear parabolic
equations using semigroup theory,'' SIAM J. Numer. Anal., Vol. 55,
pp. 980-1001,(2017)

Numerical verification for existence of a global-in-time solution to
semilinear parabolic equations, J. Comput. Appl. Math., Vol. 315, pp.
1-16, (2017)

``Accurate method of verified computing for solutions of semilinear heat
equations,'' Reliable Computing, Vol. 25, pp. 74-99, (2017).

I have not read them.
Other work of this rigorous kind is by Piotr Zgliczynski \etal.
    }

\newpage
\item[2020-10-20 Erik Aurell]
Thanks for the presentation, it was very enjoyable, and impressive.
Thanks also for making it interactive, which is much more enjoyable than
a zoom talk with slides. To follow up on the remark I made, the paper I
was thinking about was Frisch, She and Thual\rf{FSTks86}

\item[2020-10-25 Predrag to Erik] (\& Lan, Roberto),

you are right, and thanks for reminding me:
Frisch, She and Thual\rf{FSTks86} {\em Viscoelastic behaviour of cellular
solutions to the Kuramoto-Sivashinsky model} (1986)
is an amazing paper (nobody writes papers like that one any more), we
always cite it, but I had forgotten the "viscoelastic" in the title.

As you had noticed immediately, there are very long correlated patterns
in the large scale simulations (not because of transients not having died
out - the long-range correlations are always there), and - as Frisch et
al write, they are a consequence of Galilean symmetry, in the following
way:


\begin{itemize}
  \item[`Visco': ]
While on small finite domains we always work by fixing $\spaceAver{u}=0$,
on large and infinity domains, $u$ can attain any value, ie, the flame
front velocity ``diffuses" or executes a ``random walk" - that's how I
interpret a part of their paper. We actually routinely compute the
covariance $\spaceAver{u^2}/2$ of this random walk, as that goes into our
power-in/dissipation-out plots, ChaosBook
\HREF{http://chaosbook.org/chapters/ChaosBook.pdf\#section.30.3}
{Sect.~30.3} {\em Energy budget}, Figure 30.3.

  \item[`-elastic':]
I know precisely what spatiotemporal stability of a relative
doubly-periodic solution is, but I do not understand how to identify the
`elastic' eigen-exponents when we study stability of our rubbery tiles.
\end{itemize}

For me the paper is very hard to digest (my fault, not theirs), but you
have worked on this, you'll figure it out.

Basically, while they discuss long wavelengths compared to the typical
wiggle-spacing, our problem is a single nearly marginal
eigenvalue / eigendirection in deforming a small rubbery tile while
infinitesimally changing both spatial and temporal period in determining
the continuous family of what we would like to identify uniquely by a
single spatiotemporal array of symbols.

Wow, that was a long sentence. My hunch is that we should pick the unique
value where this eigenvalue is zero (the "least-strained" tile shape) to
describe the whole continuous family of relative doubly-periodic
solutions corresponding to a given ruberry shape. It should die at both
ends through some abrupt, precise, shape-changing bifurcation. All other
eigen-exponents are strictly away from zero, so the family should be
isolated. It should not be possible to continuously transform distinct
rubber tiles into one another for spatiotemporal discrete symbolic
dynamics to make sense.

Matt has explored the bifurcations at the ends of the continuous families
of rubber-tile deformations, but not sufficiently precisely to establish
that.

\item[2019-02-16 Predrag to Matt]
Eventually the study of the deformation of our ``rubber alphabet tiles'' should
go to a chapter of its own. Currently the text above, around
{\bf 2018-07-23 Matt continuous families} is what we have,
and the main conceptual problem remains what point on the 1D continuous
family of a rubber tile to take as the representative tile.

Here is a simple proposal - pick the point at which an ``alphabet'' tile is
reflection symmetric;
than the continuum of other solutions belongs to relative periodic tiles,
with continuum of non-zero phase shifts.

Is this good enough?

    \PCpost{2019-05-14}{
We might chose Galilean invariant elementary tiles (spatial derivatives
of $u$), instead of Galilean equivariant $u_j$ tiles.
    }

\item[2019-12-06 Matt \& Predrag] excerpt from a draft
{\em spatiotemp/chapter/tiles.tex}:

[$\cdots$] we have
not investigated how
the local Galilean velocities
of the tiles affect
the tiling procedure. We [$\cdots$ use]  the
Galilean invariance \refeq{GalInv} to set the mean velocity of
the overall front to zero.
For an arbitrary subregion of width $\speriod{1}< \speriod{}$, the
mean velocity is generically $\spaceAver{u}(\zeit)\neq0$. Actually, we
know that as function of $\speriod{}$ the velocity front executes a
random walk,
    \PC{2019-12-06}{make sure that this is explained in the
text elsewhere, then link here to the variance equation
``with variance
$E(\zeit)=\half\spaceAver{u^2}(\zeit)\propto\speriod{}$ by the
extensivity of \KS,''
    }
and hence the range of the color bar in a figure such
as \reffig{f:ks_largeL} has to grow proportionally to
$\sqrt{\speriod{}}$. The variance grows only in the spatial direction, in
the time direction $E(\zeit)\to{E}$.
That implies that in gluing letters $u_j$ of alphabet
\reffig{fig:tiles} into larger patterns, one also has to vary
$\spaceAver{u_j}(\zeit)$ averaged over the tile of width $\speriod{j}$,
in order to glue optimally. In other words, we have to use the Galilean
symmetry group orbit of the letter $u_j$, and slice that group orbit at
$\spaceAver{u_j}(\zeit_0)=0$ for purposes of plotting its representative
in \reffig{fig:tiles}. The tiles of \reffig{fig:tiles} were all converged
with under the zero mean velocity condition; locally, subdomains of
the tiling have non-vanishing local Galilean velocities.

    \PCpost{2020-10-23}{
We refer to Frisch \etal\ after \refeq{e-ksL}, in
\reffig{fig:MNG_Linfty_u_vs_ux}, and in posts {\bf 2018-05-09 PC} and
{\bf 2013-12-29 PC}. In

Cvitanovi{\'c}, Davidchack and Siminos\rf{SCD07} write: ``For large
system size, it is hard to imagine a scenario under which attractive
periodic states (as shown in \refref{FSTks86}, they do exist) would have
significantly large immediate basins of attraction.''
    }

\item[2021-03-20 Predrag]
I'm worried about `rubbery tiles' in Matt's \KS\ project, specifically
about Frisch \etal\ ``viscoelasticity'' paper\rf{FSTks86} that seems to
describe long range correlations.

Graham know what they are, see {\bf 2017-01-26 Matt}, {\bf M. Graham Talk}
above.

See
{\bf 2020-10-22 Matt} eq.~\refeq{WrongBurgers}, {\bf 2020-10-25 Predrag}
above.

For some viscoelasticity
papers to possibly ponder (have not tried to read them), search for
{\bf 2021-03-20 Predrag} in svn repo \emph{pipes}, \emph{blog/blog.tex}.

\item[2020-10-25 Predrag]
notes on Frisch, She and Thual\rf{FSTks86}
 {\em Viscoelastic
behaviour of cellular solutions to the Kuramoto-Sivashinsky model}:

``A multiple-scale analysis of the Kuramoto-Sivashinsky one-dimensional
model of a flame front with $2\pi$-periodic boundary conditions is presented.
For arbitrary large values of the number M of linearly unstable modes
there exist stable steady solutions of period $2\pi/N$ where N = O(M). These
`cellular solutions' exhibit elastic behaviour under perturbations of
wavelength much larger than $2\pi/N$. The results are illustrated by
numerical experiments. Elasticity has its origin in the translation and
Galilean invariances. Similar invariance properties are likely to be at
the root of the viscoelastic behaviour of turbulent flows conjectured by
many authors.''

They show that the stability of cellular solutions is related to their
viscoelastic behaviour under large-scale weak perturbations. Actually
this stability is at best marginal: the KS equation is invariant under
translations and Galilean transformations ; thus there are perturbations
which cannot relax, namely weak uniform translations and addition of weak
uniform velocities. When such perturbations are taken to be slightly
non-uniform, slow but non-trivial dynamical behaviour sets in. This is a
particular case of what is known as `phase dynamics'
(Kuramoto\rf{kuturb78,kura84tur},1984a,b; Pomeau \& Manneville 1979;
Coullet \& Fauve 1984, 1985; Fauve 1985).

Technically, weak large-scale perturbations of the cellular solutions are
governed by a linear p.d.e. with (spatially) rapidly varying
coefficients. This may be asymptotically analysed by the same multiscale
homogenization methods that are used in deriving the bulk properties of
periodically inhomogeneous materials or flows (Bensoussan, Lions and
Papanicolaou 1978 ; Papanicolaou \& Pironneau 1981). Most problems studied
so far by these techniques have only translation invariance. The presence
in our case of the additional Galilean invariance gives rise to
second-order rather than first-order dynamics in time.

\item[2020-11-13 Predrag]
Johnson, Noble, Rodrigues  and Zumbrun\rf{JNRZ13} {\em  Behavior of
periodic solutions of viscous conservation laws under localized and
nonlocalized perturbations}: ``
when there exist conserved quantities, whether deriving from Hamiltonian
structure/symmetries of the equations, or, as in the case of
parabolic conservation laws considered here, simply from divergence form
of the equations/conservation of mass, then there exist additional
critical modes, and the formal WKB prediction becomes that of a more
complicated hyperbolic-parabolic system of conservation laws rather than
the scalar convected Burgers equation of the reaction-diffusion case.

Perhaps the best-known example of such a model is the
Kuramoto-Sivashinsky equation, for which the formal asymptotic
description of behavior via a hyperbolic-parabolic system of conservation
laws was pointed out already in  Frisch, She and Thual\rf{FSTks86}
under the alternative form of a
damped scalar wave equation (the ``viscoelastic behavior'' of the title).
''

but, on the whole, I doubt this paper will help us.

\item[2020-11-15 Predrag]
Matt writes: ``In this symbolic representation the columns code
admissible time itineraries, and rows encode the admissible spatial profiles.''
I like the phrasing.

\item[2020-11-23 Predrag]
Could it be that continuous $[\speriod{}\times\period{}]$ families
somehow correspond to Floquet of `Brillouin' bands?

    \PCpost{2021-04-12}{
\HREF{https://www.linkedin.com/in/mgudorf/} {Matt}
 started working as ``Data Scientist'' for a AI logistics startup
\HREF{https://verusen.com/} {Verusen.com}.

Burak and Matt have committed to writing up the first draft of the Matt
thesis paper.
}

    \PCpost{2021-04-12}{
From Sep 2019 to present Matt has
designed, developed, documented, deployed and dockerized Python package
as Jupyter Notebook,

\begin{quote}
\HREF{https://orbithunter.readthedocs.io} {Orbithunter guide}
\\
\HREF{https://github.com/mgudorf/orbithunter} {GitHub Orbithunter}
%\item[2021-10-01] Matt:
\\
docker.com login:    orbithunter    beThebest!  \\
orbithunter port is:   -p 8887:8887
\end{quote}

which serves as a framework for the study of nonlinear dynamics and
chaos. Designed to maximize user friendliness and modularity to enable
collaboration between scientists and comparison of their results.
Acts as a high-level user interface of the SciPy and NumPy numerical Python
packages and partial differential equations.
}

    \PCpost{2021-06-28}{
Potentially of interest.

Salihah Alwadani, Kelowna, B. C., \HREF{https://people.ok.ubc.ca/bauschke/}
{Heinz H. Bauschke}, Julian P. Revalski and Xianfu Wang, {\em The difference
vectors for convex sets and a resolution of the geometry conjecture},
\arXiv{2012.04784}, write:

Let $X$ be a real Hilbert space,
and $C_1,\ldots,C_m$ are nonempty closed convex subsets of $X$,
with projectors
$P_{C_1},\ldots,P_{C_m}$ which we also write more simply as
$P_1,\ldots,P_m$,
and with $m\in\{2,3,\ldots,\}$.
We define the fixed point sets of the cyclic compositions by
\bea
  F_m &:=& \mbox{Fix}(P_m\cdots P_1),
\continue
  F_{m-1} &:=& \mbox{Fix}(P_{m-1}\cdots P_1P_m),\;
  \ldots,
\continue
  F_1 &:=& \mbox{Fix}(P_1P_m\cdots P_2).
\label{AKBRW20e:defF_i}
\eea
Compositions of projectors are often employed in projection methods. This is
a vast area which we will not summarize here; however, we refer the reader to
\refref{CenZak18}, \arXiv{1802.07529} as a starting point, as well as the
\refref{ComPes20}, \arXiv{2008.02260}.
    }

    \PCpost{2021-08-27}{
Potentially of interest.

Zeng and Graham\rf{ZenGra21} {\em Symmetry reduction for deep
reinforcement learning active control of chaotic spatiotemporal dynamics}
(2021):
``Deep reinforcement learning (RL) is a data-driven, model-free method
capable of discovering complex control strategies
[...] systems of flow control interest possess
symmetries
[...]  Kuramoto-Sivashinsky equation (KSE), equally spaced actuators, and
a goal of minimizing dissipation and power cost, we
move the deep RL problem to a symmetry-reduced space
[...] symmetry-reduced deep RL yields improved data efficiency
[...] the symmetry aware control agent
drives the system toward an equilibrium state of the forced KSE
[...] despite
having been given no explicit information regarding its existence.''

Deep learning approaches that respect symmetries automatically
rather than learn to approximate them from data
are likely to have superior performance,
[...] incorporating symmetries of the learning
domain into the deep neural-network (NN) models. For example,
the state-of-the-art\rf{LenVed18}
AlexNet NN image classifier spontaneously learns redundant
internal representations that are equivariant to flips,
scalings, and rotations.

Many flow geometries of interest
possess symmetries to incorporate into the deep RL model.



Read also:

Lenc and Vedaldi\rf{LenVed18} {\em Understanding image representations by
measuring their equivariance and equivalence},
(2018).

Bucci, Semeraro, Allauzen, Wisniewski, Cordier and Mathelin\rf{BSAWCM19}
{\em Control of chaotic systems by deep reinforcement learning},
(2019).
    }



\end{description}

%%%%%%%%%%%%%%%%%%%%%%%%%%%%%%%%%%%%%%%%%%%%%%%%%%%%%%%%%%%%%%%%%%%%%%%
\printbibliography[heading=subbibintoc,title={References}]
