% siminos/spatiotemp/chapter/abstract.tex
% $Author: predrag $ $Date: 2020-11-21 17:09:41 -0500 (Sat, 21 Nov 2020) $

% called by
%           siminos/spatiotemp/chapter/spatiotemp.tex
%           siminos/tiles/GuBuCv17.tex

Motivated by space-time translational invariance
`spatiotemporally chaotic' or `turbulent' flows are recast
as a $D+1$ dimensional \spt\ theory which treats space and time equally.
In this formulation time evolution is replaced by
a repertoire of spatiotemporal patterns taking the form of $D+1$ invariant tori.
Infinite space-time is then explained by the shadowing of these tori.
This is formalized by the development of a $D+1$ dimensional symbolic dynamics whose
alphabet is comprised of space-time tori of minimal size.
Enumerating these \spt\ building blocks enables
the construction of all admissible \spt\ patterns.
These ideas are investigated in the context of
the \KSe\ using new open source \spt\ computational codes.
These codes offer easy access to new \spt\ techniques,
persistent homology, convolutional neural networks and more.
\bigskip

\noindent
{\bf
Spatiotemporal tiling of the Kuramoto-Sivashinsky system
\\
Abstract Nov 19, 2020}

Using (1) spacetime translational invariance, and (2) exponentially
unstable dynamics, `spatiotemporally chaotic' or `turbulent' flows are
reformulated as a (D+1)-dimensional spatiotemporal theory which treats
space and time on equal footing. In this theory there is no evolution in
time: time evolution is replaced by the enumeration of the repertoire of
the spatiotemporal solutions (translationally invariant (D+1)-dimensional
invariant tori, or `periodic orbits') of system's equations, very much as
the statistical mechanics' weighted partition function of Ising model is
constructed as a sum formed by enumerating all its lattice states.

Our hypothesis is that the entirety of the spacetime solutions can be
constructed by gluing together tiles from a finite collection of
`fundamental tiles' that shadow larger solutions.  We demonstrate that
(1) these fundamental tiles can be extracted from generic,  large
spacetime domain solutions, and (2) that they in turn can be used as the
`building blocks' of turbulence.  Left for the future work is (3) our
conjecture that these results should enable us to construct a
(D+1)-dimensional symbolic dynamics with an alphabet (whose `letters' are
these fundamental tiles) enables one to systematically enumerate and
label all `turbulent' solutions.

These ideas are investigated in the context of the 1+1 dimensional
space-time of the Kuramoto-Sivashinsky equation in one spatial dimension,
using my open source Python package `OrbitHunter'.
