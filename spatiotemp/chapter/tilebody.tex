% siminos/spatiotemp/chapter/tilebody.tex
% $Author: mgudorf3 $ $Date: 2020-05-19 16:24:55 -0400 (Tue, 19 May 2020) $

% called by
%           siminos/spatiotemp/chapter/spatiotemp.tex
%           siminos/tiles/GuBuCv17.tex

%\section{Results}
%\label{sect:body}
% Matt                                           28 February 2020



\Preliminary{Know how, does it work\subsection{Collection}}
So far we have motivated a {\spt} theory of turbulence
which replaces unstable dynamics with \spt\ patterns.
We formulate these ideas using the \KSe, creating a number of
new numerical techniques in the process.
The following section describes the results of our numerical investigation.

\Preliminary{\subsubsection{Different symmetries}}
    \PCedit{ % 2020-05-07
In previous work, the \statesp\ geometry and the natural measure for
this system have been
studied\rf{Christiansen97,LanThesis,lanCvit07} in terms of unstable
periodic solutions restricted to the antisymmetric subspace of the
\KS\ dynamics.

The focus of \refref{SCD07} was on the role continuous symmetries
play in {\spt} dynamics. The notion of exact
periodicity in time is replaced by the notion of relative
{\spt} periodicity, and \reqva\ and \rpo s here play
the role the \eqva\ and \po s played in the earlier studies.

In presence of a continuous symmetry any solution belongs to a group
manifold of equivalent solutions. The problem: If one is to
generalize the periodic orbit theory to this setting, one needs to
understand what is meant by solutions being nearby (shadowing) when
each solution belongs to a manifold of equivalent solutions.
        }
\PC{2020-05-07}{In \refref{BudCvi15} we resolve this puzzle by implementing
symmetry reduction.}

\Preliminary{\subsubsection{Noise $\to$ final}}
Before we began searching for {\po}s in earnest, we first
tested the efficacy of the {\spt} numerical methods %\MNG{reference section here}
using known {\po}s \rf{SCD07}.
The first test was to find solutions of \refeq{e-Fks} using coarse {\spt} discretizations.
This is important because the main limiting factor
for {\spt} methods is the number of {\cdofs}; the entire orbit must be
kept in the computational memory. It is imperative that we be able to
find {\po}s with coarse resolutions, otherwise the problem is not computationally
feasible unless we consider more advanced computing resources. 
Each test performed simply rediscretized a known solution to create an coarse
initial guess, which was then pass to our numerical methods.
In all of the tests we performed, the guess converged to the 
``same'' {\po} that it originated from.
Technically, they are not returning
to exactly the same {\po}, but rather, the same family of {\po}s. This
topic arises organically later on when we detail the existence
of continuous families of {\fpo}s, and so we leave the discussion until then.
These tests showed that we could now solve \refeq{e-Fks} with coarse
discretizations, however, using known solutions to perform these tests however
is not very convincing; therefore we then applied
to initial guesses created by adding a substantial amount of noise
to these known {\po}s.

The idea behind this test was simply to see whether or not
the noisy initial guess would converge. 
At each field site in the discretization, we drew a value from a
normal distribution with mean zero and whose standard deviation was the 
$L_{\infty}$ norm of the {\po}'s field. We only performed a single test
but the results were still  Not only did the 
sum of the original field and the noise converge, it converged to the original {\po}
up to translations and small differences in periods.
This exceeded our expectations as the noise was much larger in magnitude
than the original {\po} itself.
Note that we did not include perturbations to the periods but
they remained unconstrained. This test case used a {\po}
with small periods, however, the extent of the noise made us
optimistic that initial guesses of poor quality could be used to find {\po}s.
After these two tests we began the search and collection of {\po}s.
This search applied the methods of \refsect{sect:intro} and
spanned all symmetry types and a range of domain sizes; specifically,
the periods were drawn from the intervals
$L \times T \in [22, 88] \times [20, 200]$.
As a reminder our hypothesis does not require the collection of exceptionally large {\po}s;
they must only be large enough to capture all unique fundamental patterns. If we believed
that this range of periods was insufficient we would have expanded our
search.

\Preliminary{\subsubsection{Initial $\to$ final}}
Disregarding a few outliers (to be discussed later) the vast majority of {\po}s in this collection
look more or less the same. More precisely, if one were to zoom in on a {\po} and examine
the local features and patterns in a small {\spt} window, neither
the global {\spt} symmetry nor the {\extent} would be ascertainable. This reinforced our belief that {\fpo}s are indeed universal and
can be used to explain all solutions.
The figure %\reffig{fig:waldo}
displays {\po}s of the four different
{\spt} symmetry classes that we considered in our investigation. We believe they are essentially
indistinguishable when only the fundamental domains are plotted (obviously if the symmetry
is visible they can be distinguished from one another).%\MNG{reference figure here}
%\begin{figure}
%\begin{minipage}[height=.05\textheight]{.3\textwidth}
%\centering \small{\texttt{(a)}}\\
%\includegraphics[width=.3\textwidth,height=.1\textheight]{MNG_waldo1}
%\end{minipage}
%\begin{minipage}[height=.05\textheight]{.3\textwidth}
%\centering \small{\texttt{(b)}}\\
%\includegraphics[width=.4\textwidth,height=.1\textheight]{MNG_waldo2}
%\end{minipage}
%\begin{minipage}[height=.05\textheight]{.3\textwidth}
%\centering \small{\texttt{(c)}}\\
%\includegraphics[width=.4\textwidth,height=.1\textheight]{MNG_waldo3}
%\end{minipage}
%\begin{minipage}[height=.05\textheight]{.3\textwidth}
%\centering \small{\texttt{(d)}}\\
%\includegraphics[width=.8\textwidth,height=.1\textheight]{MNG_waldo4}
%\end{minipage}
%\caption{ \label{fig:waldo}
%Four {\po}s which demonstrate the ubiquity of patterns irrespective of
%the global symmetry of the housing {\po}.
%}
%\end{figure}
\Preliminary{\subsubsection{Outliers}}
Another means of developing our intuition regarding {\spt} patterns is
to look at the \textit{atypical} patterns of the \KSe.
This classification consists of three main categories: {\po}s which
are `too symmetric', contain `uncommon' patterns, or
which follow an \eqv\ for an extended time.
In dynamical systems context `outliers' is our blanket term for very isolated (unstable) solutions which would hardly ever realized.
Our searches find these outliers
because stability has no affect on the optimization problem.
This implies that our variational method also captures
`rare events', a notably hard problem that has utility in other situations. \MNG{mentioning this worthwhile (or accurate)?}

The usage of `outliers' as opposed to 'isolated' is because solutions exist continuous families
and hence are not `isolated'. Support for our claim that
they are not frequented often is that they are typically found
to exist in the antisymmetric subspace which is very unstable and hence not
visited for long periods of time.  % sources? Lorenz z axis analogy?
Originally we found {\po}s classified as outliers everywhere; but
it may be that they all exist in the antisymmetric subspace. This seems
contradictory if not for a particular computational detail.
It is possible to find antisymmetric solutions
without specifically constraining an initial guess to the antisymmetric subspace.
This is because an antisymmetric {\po}s have the same topology as other {\po}s;
it manifests only as a constraint on the {\Fcs}.
What does this mean? It depends
on the symmetries being compared but here are two examples. In figure %\MNG{figure reference here}
demonstrates a {\po} found with no imposed symmetry.
Another possibility occurs with shift-reflection invariant {\po}s. Due to
the nature of the constraints on the {\Fcs}, imposing shift-reflection symmetry
can also find even multiples of prime periods of antisymmetric {\po}s;
i.e. two, four, etc. repeats of antisymmetric solutions in time.
These aren't just assumptions either; these solutions can actually be converged to the
antisymmetric subspace if their reflection axes are restored, as seen in {}.
This demonstrates that {\po}s with discrete symmetries are simply special representatives
of their group orbit produced via translations.
This would have never been realized in the dynamical
systems context due to the instability of the antisymmetric subspace.%\MNG{is this actually a new revelation?}
%The solution %\reffig{rposlant}
%demonstrates a behavior that is not particularly common on smaller domain sizes.
%\subsubsection{Analysis of the library}
%examples of initial conditions and their converged results.
%examples depicting similar structures regardless of symmetry
%examples demonstrating repeats of the structures after a prescribed length
%examples of similar yet deformed structures / the structure's scales.
%\subsubsection{Outliers}
%Under resolved solutions
%Large equilibria solutions?

We have displayed {\po}s that we consider to be both typical and atypical.
Our intuition evolved to the point where we could identify {\fpo} candidates.
The notable property of these candidates is that they not only occur frequently throughout the
{\po} collection; they also occur frequently within individual {\po}s.
To further motivate the existence of
{\fpo}s as well as some initial guesses, we direct the reader to %\MNG{figure reference here}
where a very large time integrated trajectory (aperiodic in time)
is displayed. Notable in this figure is the {\spt}
frequency with which a single pattern occurs.
\Preliminary{\subsection{Clipping}}
\Preliminary{\subsubsection{Tiles, cutouts, final} Lion's share}
The first guess was the {\wiggle} which appears in% \MNG{figure reference here}
It very distinctly repeats twice with respect to time in
and is relatively small in spatial scale, containing only two notable wavelengths.
%\MNG{figure reference here}
\Preliminary{\subsubsection{Iterative}}
This was the inaugural search for a {\fpo} and so extra
care was taken in the clipping process \refsect{sect:intro}.
In this instance the clipping was done iteratively, finding
a sequence of progressively smaller {\po}s each of which contained the {\fpo}.
We now know that it is was possible to clip and converge {\fpo}s in a single clipping,
but this example is kept in its entirety to demonstrate additional utility.

The results of the {\wiggle} search are as displayed in %\MNG{figure reference here}
The final result was an antisymmetric {\fpo} whose velocity field on
the fundamental domain consisted of a wavelength or streak ``wiggling'', hence the name.
To check our work, comparison of this pattern with our {\po} collection was made.
While the {\wiggle} was converged in the antisymmetric subspace, the
pattern most often as a pair of wiggles.
As previously mentioned in the discussed of ``outlier'' {\po}s, the antisymmetric {\wiggle}
should be viewed as a special case of the continuous family of {\wiggle} solutions as
this case will never appear in actuality.
\Preliminary{\subsubsection{Direct}}
The collection of the {\wiggle} not only provided us with our first {\fpo} but also
provided us with an obvious guess for the second {\fpo}. If we look back
at the iterative clipping procedure that we just performed, %\MNG{figure reference here}
we see that the {\wiggle} is spatially adjacent to an additional,
single wavelength {\eqv}. %\MNG{figure reference here}
By appealing to the notion of {\spt} {\symbolic} we postulated that
this {\eqv} would be our second {\fpo}. Indeed, by clipping this
single wavelength {\eqv} from the original {\po} another {\fpo} was found,
which we now refer to as the {\streak}.

At this point we knew that our library was yet incomplete as we had not
captured the pattern emphasized in %large cutouts.
This pattern, now named the {\defect}, captures two very important physical processes (they
could be interpreted as two sides of the same coin, really). First, two spatially adjacent
wavelengths merge into a single wavelength. In the gap in space-time left behind by this
merger, a new wavelength emerges. The result is a new pair of wavelengths which
exhibit an approximate phase difference of a
half wavelength with the original pair (approximately a quarter of
the spatial period).
This {\defect} epitomizes two very important mechanisms of the \KSe:
fluctuations in global wavelength count and local spatial drift velocity.
Neither the {\streak} nor the {\wiggle} can account for
these behaviors, highlighting the importance of the {\defect}.
The actual search for the {\defect} was quite challenging relative to the other
{\fpo}s. Due to the spatial
shift it should be no surprise that the {\fpo} was assumed to be a {\rpo}.
Locating the {\defect} proved to be much more difficult than both the {\wiggle}
and {\streak}, in part due to the extra {\cdof} from the spatial shift.
\Preliminary{\subsubsection{Families of tiles} A collecting of members}

\MNGedit{After our first search concluded, we believed that there were four unique {\fpo}s.
Upon further review two of these four {\fpo}s looked quite similar visually, leading us
to believe that they were related. This was verified by pseudo-arclength continuation using
the spatial period as the parameter. Continuation was performed until the spatial periods
of this {\fpo} pair were the same. By fixing the phases of the
{\Fcs} it is quite clear that these {\fpo}s are equivalent.
This brought about a revelation regarding
{\fpo}s that had been previously glossed over;
{\fpo}s exist in continuous families. In hindsight it could not have been
any other way. Shadowing by {\fpo}s is not an exact process;
clippings from larger trajectories always look similar but not identical. This
property is precisely captured by continuous families: which are
infinite sets of solutions related by continuous deformations.}

From a dynamical systems perspective
this is not shocking nor new as previous bifurcation analyses
track different branches of {\po}s.
The main effect of these families is that our {\symbolic}
now consists of a ``rubbery'' continuous alphabet, as opposed to a static discrete one.
For each symbol in any {\spt} {\brick}, the symbol now represents a continuous
family of {\fpo}s. To best illustrate the effect of this,
we delve into the continuous family of the {\defect}.
We display the numerical continuation of the defect as a
function of the spatial period noting
that there are other manners with which {\po}s can be continued. A collection of representatives
are displayed in {}. The main result of the investigation is
that the family is perhaps more properly parameterized
by the spatial shift parameter.
% more on defect family.

It might be accurate to say that all {\po}s
exist in continuous families \emph{because} {\fpo}s exist in
continuous families. This initially betrayed our intuition; in hyperbolic systems, {\po}s are
isolated solutions by virtue of their unstable manifolds. In any case each continuous
family seems to exist on a finite interval of spatial periods, punctuated by what are
presumed to be bifurcations. It could be that this assumption is incorrect and that we
are failing to track the correct branch of the solutions.
Exploration of the {\fpo} continuous families reduced the alphabet
to three. To reiterate; there may only be three {\fpo}s
needed to describe all solutions of the {\KSe}.

It is absolutely essential to understand the notion of continuous families of {\fpo}s
and so we summarize it here before moving on to the description of gluing results. We postulated
and then found a collection of {\fpo}s whose existence can ostensibly describe all solutions.
We believe that this collection consists of three unique {\fpo} continuous families.
We currently do not have a clear manner with which to develop a {\spt} {\symbolic}.
To begin the investigation, however, we can test our ideas by creating
{\spt} combinations of {\fpo}s. This is the aforementioned gluing method,
whose results shall now be described.

\Preliminary{\subsection{Gluing}}
\Preliminary{\subsubsection{Reproduce solutions} proof of concept}

First as a proof of concept we reproduce a known solution by combining many small,
custom tailored components. Next, we demonstrate how to glue known {\po}s along a single spacetime dimension: either time or space.
Lastly, we glue {\spt} combinations of {\fpo}s to find {\po}s.
The first question we aimed to answer was ``Can a known solution be reproduced by combining many different clippings from
other solutions?''.  This is demonstrated in %reffig{}.
where a handmade {\spt} combination was used to create an initial guess for a specific {\po} which
is known to exist.

The resultant {\po} is essentially a reproduction of the target up to continuous family considerations. This prototype of the gluing method was developed concurrently with the search
for {\fpo}s. In this case the constituent pieces are not converged {\fpo}s but merely clippings from {\po}s not including the target. The success of this trial gave credence to the argument for {\po}s
existing as collections of {\fpo}s.
While this result was very encouraging it was also somewhat contrived.
The initial guess was hand tailored to be a rough approximation to the
targeted {\po}. Therefore we did not know whether this result would generalize.
The next step was to implement unsupervised gluing along a single continuous dimension.
In this context ``unsupervised'' implieds that the only decision
that is made is which {\po}s to glue together.
We focus on only gluing one pair of {\po}s at a time but also demonstrate
that this process can be applied iteratively.
\Preliminary{\subsubsection{Time}}

We begin by temporally gluing {\po}s with small spatial periods,
a familiar notion to those cognizant of periodic orbit theory. Our computation
is completely different however as the spatial periods are allowed to differ
as well as freely vary.
Upon success we were able to find {\po}s with large temporal periods.
For instance, in %reffig
the final converged {\po} is a shift-reflection invariant orbit whose fundamental domain
has period $386.088... / 2$. Similarly in %reffig
is an examples of a converged {\rpo} with time period $231.419...$. These periods
are what we would expect if the gluing was properly functioning, as they are
approximately the sum of the periods of the original {\po}s.
These {\po}s are much longer than anything we have seen
in the literature which again exemplifies how robust
our {\spt} methods are.
\Preliminary{\subsubsection{Space}}

In line with our identical treatment of space and time we also glue {\po}s spatially.
The figures %reffig reffig
show exactly this where the original {\po}s, their gluing, and the resultant {\po} are shown.
In %reffig{MNGppo12spaceglue}
two {\po}s with shift-reflection symmetry were chosen to be glued in space.
\Preliminary{\subsubsection{Repeated space} $1+2\to12$ then $12+3\to 123$}

Much like the clipping process we can also iteratively glue to create a sequence of
progressively larger {\po}s. To demonstrate this, we glue the result
from the previous example to yet another {\po} with shift-reflection symmetry. Originally
the choice was made to only glue in a symmetry preserving manner; shift-reflection symmetric
orbits were only glued to other shift-reflection symmetric orbits, and the result was
also constrained to have shift-reflection symmetry. It was realized that there
was no real motivation behind this choice and so this requirement was removed, such that
it is not applied in the next section when discussing gluing {\fpo}s together.

\Preliminary{\subsubsection{Space-time}New solutions}

Showing that we can both glue in space and time we proceed to
the ultimate method: gluing {\spt} combinations of {\fpo}s.
To review our motivation, let us again explain the process in terms of {\symbolic}.
The initial guesses can be thought of as symbolic blocks, where each symbol
is representative of a {\fpo}'s continuous family. Quite literally, the symbols are replaced
by these {\fpo}s to construct an initial guess; an approximation to a possible {\po}.
We elect to use a static set of
representative {\fpo}s for each of these substitutions with no motivation other
than simplicity. Specifically: the {\defect},
{\wiggle}, and {\streak} displayed in %reffig final tiles.

\Preliminary{\subsubsection{Cannot tell if final solution $=$ target symbol block}}
To deem each gluing as a success, they must satisfy a {\symbolic} requirement
in addition to numerical convergence. Namely, the initial guesses must converge
to a {\po} which has the ``correct'' symbolic representation.
That is, it must be contain the {\fpo}s which befits its construction.
Yet again, however, we are forced to rely upon visual inspection to validate our results.
First, we provide examples which we claim are successful under our new requirement.
Next, are examples of numerical success but ``symbolic failure''.
The existence of this class of outcomes, in combination with our lack of verification of results,
contributes to the challenge of making the {\symbolic}. Specifically,
this prevents us from determining the correct grammar of
the {\symbolic}; the rules which dictate whether each {\brick} is admissible {\po} or not.

In summary we have shown that it is possible to use our techniques to find {\po}s by construction,
clipping and gluing. In essence
clipping and gluing are inverses of one another, however, gluing is
typically more difficult than clipping due to the inherent differences in complexity. In practice,
we usual clip in a single step but glue iteratively.
Currently we have a very surface level method and there are currently many unknowns in regards to best practices. Essentially we have made the simplest choices which result in the gluing method to be
numerically well defined. While these results are very informative they leave us with more open
questions than we started with \refsect{sect:future}. The future is bright, however, as the number of potential improvements seems to be limited only by our creativity.
