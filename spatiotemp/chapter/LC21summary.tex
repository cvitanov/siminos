% siminos/reversal/summary.tex      pdflatex LC21; bibtex LC21
% temporary: siminos/spatiotemp/chapter/LC21summary.tex
% $Author: predrag $ $Date: 2021-12-24 01:25:20 -0500 (Fri, 24 Dec 2021) $

% was siminos/kittens/summary.tex  CL18

\section{Summary}
\label{s:summary}


%%%%%%%%%%%%%%%%%%%%%%%%%%%%%%%%%%%%%%%%%%%%%%%%%%%%%%%%%%%%%%%%%%%%%%%%
%   \PC{2016-11-10}{
How to think about matters {\spt}? As our intuition about motions of
fluids is so much better than about turbulent quantum field theories,
here we briefly describe recent advances in turbulence that underpin
ideas developed in this paper.

%%%% clip from siminos/cats/GHJSC16.tex 2020-11-18
While dynamics of a turbulent system might appear so complex to defy any
precise description, the laws of motion drive a spatially extended system
through a repertoire of recognizable unstable patterns (clouds, say),
each defined over a finite  {\spt} region\rf{GuBuCv17,GudorfThesis}.
The local dynamics, observed through such finite spatiotemporal windows,
can be thought of as a visitation sequence of a finite repertoire of
finite patterns.
To make predictions about such system,  one needs to know how often a
given pattern  occurs, and that is a purview of \po\ theory\rf{focusPOT}.
Early 2000's rapid progress in description of turbulence in terms of such
`{\ecs}s'\rf{KawKida01,science04,GHCW07} has since slowed down to a crawl
due to our inability to extend the theory and the computations to
spatially large or infinite computational domains\rf{WFSBC15}.

In dynamics, we have no fear of the infinite extent in time. That is \po\
theory's\rf{DasBuch} raison d'\^{e}tre; the dynamics itself describes the
infinite time strange sets by a hierarchical succession of \po s, of
longer and longer, but always finite periods (with no artificial external
periodicity imposed along the time axis). And, since 1990's we know how
to deal with both spatially and temporally infinite regions by tiling
them with {\spt}ly periodic tiles\rf{Christiansen97,FE03,GHCW07}. A time
{\po} is computed in a finite time, with period \period{}, but its
repeats ``tile'' the time axis for all times. Similarly, a {\spt}ly
periodic ``tile'' or ``\twot'' is computed on a finite spatial region
$L$, for a finite period \period{}, but its repeats in both time and
space directions tile the infinite spacetime.

These ideas open a path to determining exact solutions on \emph{spatially
infinite} regions, and many physical turbulent flows of come equipped
with $D$ spatial translational symmetries. For example, in a pipe flow at
transitional Reynolds number, the azimuthal and radial directions
(measured in viscosity length units) are compact, while the pipe length
is infinite. If the theory is recast as a $d$\dmn\ space-time theory,
\(d= D +1\,,\)
{\spt}ly translational invariant recurrent solutions are \dtors\
(and \emph{not} the $1$\dmn\ temporal \po s of the traditional {\po} theory),
and the symbolic dynamics is likewise $d$\dmn\
(rather than a 1\dmn\ temporal string of symbols).

This changes everything. Instead of studying time evolution of a chaotic
system, one now studies the repertoire of {\spt} patterns compatible with
a given PDE, or, in the discretized spacetime, partial difference
equation. There is no more \emph{time} in this vision of nonlinear
\emph{dynamics}! Instead, there is a \statesp\ of all {\spt} patterns,
and the likelihood that a given finite {\spt}ly pattern can appear, like
the mischievous grin of Cheshire cat, anywhere, anytime in the turbulent
evolution of a flow. A bold vision, but how does it work?
\\
%   } %end censored \PC{2016-11-10 Curb you enthusiasm}
%%%%%%%%%%%%%%%%%%%%%%%%%%%%%%%%%%%%%%%%%%%%%%%%%%%%%%%%%%%%%%%%%%%%%%%%

It is in the context of working out the geometry of infinite-dimensional
{\statesp}s of turbulent fields\rf{hopf48} (not 3\dmn\ visualizations of
fluids!) that we find the lessons learned from discretized field
theories studied in this paper very helpful.

We have learned that, in order to understand turbulence, we have to think
globally, but act locally. Turbulence is described by a catalogue of
\spt\ patterns, ({`{\lattstate}s'} in the present, discretized field theory
context), each a numerically exact \emph{global} solution of \emph{local}
deterministic Euler–Lagrange equations. The problem, stripped to its
essentials, is the problem od systematically enumerating them, computing
them, and determining their relative importance:

\begin{enumerate}
              \item
\emph{{\Lattstate}s:} Each solution is a zero of
the global Euler–Lagrange {\em fixed point} condition
\[
F[\Xx] = 0
\,.
\]
              \item
{\em Global stability:} is given by the {\jacobianOrb}
\[
\jMorb_{ij} =\frac{\delta F[\Xx]_i}{\delta \ssp_j}
\,.
\]
              \item
{\em Fundamental fact:} the numbers of \spt\ orbits
and the weight of each is given by orbit's {\HillDet}
\[
\Det\jMorb
\,.
\]
              \item
{\em Symmetries:} you must use them.
              \item
{\em Zeta function:} If you have $\zeta(z)$,
you have all predictions of the theory.
            \end{enumerate}


semiclassical quantization theory,

deterministic dynamics serve

 WKB `skeleton'

deterministic skeleton for
infinite-dimensional lattice-discretized scalar field theories.

In
field-theoretical formulation, there is no evolution in time

Hill's formulas

and thus a $d$\dmn\ {\spt} pattern is
mapped one-to-one onto a $d$\dmn\ discrete {\lattstate}, symbolic
dynamics labelled configuration - a configuration very much like that of an
Ising model of statistical mechanics.


there is only an enumeration of {\lattstate}s
that contribute to theory's partition sum,

robust global \spt\
solutions of deterministic Euler–Lagrange equations.

The reformulation aligns `chaos theory' with the solid state,
field theory and statistical mechanics.

In a \spt,
crystallographer formulation, time-periodic orbits of dynamical systems
theory are replaced by periodic $d$\dmn\ {Bravais cell} tilings
($d$-tori) of spacetime, each weighted by the inverse of its instability,
its Hill determinant.

In contrast to conventional solid state
calculations, hyperbolic shadowing of large cells by smaller ones ensures
that the predictions of the theory are dominated by the smallest Bravais
cells.

Already
1\dmn\ lattice discretization is of sufficient interest to be the focus
of this paper.

from a \spt\ field theory perspective,
`time'-reversal is a purely crystallographic notion, a reflection point
group, leading to a perhaps novel symmetry quotienting of
time-reversible theories and associated topological zeta functions.


in particular, a \spt\ lattice formulation of time
reversal.



corresponding dynamical zeta functions
should be sums over $d$\dmn\ Bravais cells, rather than $1$\dmn\ time-\po s.


While the setting is classical,
such deterministic field-theory advances offer new semi-classical
approaches to quantum field theory and many-body problems.
