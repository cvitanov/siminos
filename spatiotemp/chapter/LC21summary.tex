% siminos/reversal/summary.tex      pdflatex LC21; bibtex LC21
% temporary: siminos/spatiotemp/chapter/LC21summary.tex
% $Author: predrag $ $Date: 2021-12-24 01:25:20 -0500 (Fri, 24 Dec 2021) $

% was siminos/kittens/summary.tex  CL18

\section{Summary}
\label{s:summary}


%%%%%%%%%%%%%%%%%%%%%%%%%%%%%%%%%%%%%%%%%%%%%%%%%%%%%%%%%%%%%%%%%%%%%%%%
%   \PC{2016-11-10}{
How to think about matters {\spt}?
%%%% clip from siminos/cats/GHJSC16.tex 2020-11-18
While dynamics of a turbulent system might appear so
complex to defy any precise description,
the laws of motion drive a spatially extended system (clouds, say) through a
repertoire of unstable patterns, each defined over a finite  {\spt}
region\rf{GuBuCv17,GudorfThesis}.
The local dynamics, observed through such
finite spatiotemporal windows, can be thought of as a visitation
sequence of a finite repertoire of finite patterns.
To make
predictions about such system,  one needs to know how often a given
pattern  occurs, and that is a purview of \po\ theory\rf{focusPOT}.

However, the initial rapid progress in description of turbulence in terms
of such `{\ecs}s'\rf{KawKida01,GHCW07} has slowed down to a crawl due
to our inability to extend the theory and the computations to
spatially large or infinite computational domains\rf{WFSBC15}.

But in dynamics, we have no fear of the infinite extent in time. That is \po\
theory's\rf{DasBuch} raison d'\^{e}tre; the dynamics itself describes the
infinite time strange sets by a hierarchical succession of \po s, of longer
and longer, but always finite periods (with no artificial external
periodicity imposed along the time axis). And, since 1996 we know how to deal
with both spatially and temporally infinite regions by tiling them with
finite {\spt}ly periodic tiles\rf{Christiansen97,GHCW07}. More
precisely: a time periodic orbit is computed in a finite time, with period
\period{}, but its repeats ``tile'' the time axis for all times. Similarly, a
{\spt}ly periodic ``tile'' or ``\twot'' is computed on a finite
spatial region $L$, for a finite period \period{}, but its repeats in both
time and space directions tile the infinite spacetime.

Taken together, these open a path to determining exact solutions on
\emph{spatially infinite} regions.
This is important, as many turbulent flows of physical interest come equipped
with $D$ continuous spatial symmetries. For example, in a pipe flow at
transitional Reynolds number, the azimuthal and radial directions (measured
in viscosity length units) are compact, while the pipe length is infinite.
If the theory is recast as a $d$\dmn\ space-time theory,
\(d= D +1\,,\)
{\spt}ly translational invariant recurrent solutions are \dtors\
(and \emph{not} the $1$\dmn\ \po s of the traditional periodic orbit theory),
and the symbolic dynamics is likewise $d$\dmn\ (rather than what is
today taken for given, a single 1\dmn\ temporal string of symbols).

This changes everything. Instead of studying time evolution of a chaotic
system, one now studies the repertoire of {\spt} patterns allowed by
a given PDE, or, in the discretized spacetime, partial difference equations.
To put it more provocatively: junk your old equations and look for guidance
in clouds' repeating patterns.
There is no more \emph{time} in this vision of nonlinear \emph{dynamics}!
Instead, there is the space of all {\spt} patterns, and the
likelihood that a given finite {\spt}ly pattern can appear, like the
mischievous grin of Cheshire cat, anywhere in the turbulent evolution of a flow.
A bold vision, but how does it work?
\\

and thus a $d$\dmn\ {\spt} pattern is
mapped one-to-one onto a $d$\dmn\ discrete {\lattstate}, symbolic
dynamics labelled configuration - a configuration very much like that of an
Ising model of statistical mechanics.
%   } %end censored \PC{2016-11-10 Curb you enthusiasm}
%%%%%%%%%%%%%%%%%%%%%%%%%%%%%%%%%%%%%%%%%%%%%%%%%%%%%%%%%%%%%%%%%%%%%%%%

In this paper we have analyzed

in particular, a \spt\ lattice formulation of time
reversal.



corresponding dynamical zeta functions
should be sums over $d$\dmn\ Bravais cells, rather than $1$\dmn\ time-\po s.


While the setting is classical,
such deterministic field-theory advances offer new semi-classical
approaches to quantum field theory and many-body problems.
