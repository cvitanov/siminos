\svnkwsave{$RepoFile: siminos/spatiotemp/chapter/chronotope.tex $}
\svnidlong {$HeadURL: svn://zero.physics.gatech.edu/siminos/spatiotemp/chapter/chronotope.tex $}
{$LastChangedDate: 2021-12-08 12:21:27 -0500 (Wed, 08 Dec 2021) $}
{$LastChangedRevision: 7383 $} {$LastChangedBy: predrag $}
\svnid{$Id: chronotope.tex 7383 2020-04-19 05:28:36Z predrag $}

\chapter{Chronotopic musings}
\label{chap:chronotope}
% Predrag                                           18 April 2020
% called by siminos/spatiotemp/blog.tex

\renewcommand{\ssp}{\ensuremath{\phi}}             % lattice site field
\renewcommand{\Xx}{\ensuremath{\mathsf{\Phi}}}      % kittens lattice field
\renewcommand{\Laplacian}{\square}
\renewcommand\speriod[1]{{\ensuremath{L_{#1}}}}  %continuous spatial period
\renewcommand\period[1]{{\ensuremath{T_{#1}}}}  %continuous time period

\begin{description}
    \PCpost{2020-10-25}{
ChaosBook.org
\HREF{http://ChaosBook.org/overheads/spatiotemporal/index.html}
{spatiotemporal homepage}
    }

    \PCpost{2020-10-25}{
2021 APS March Meeting (held virtually) March 15 - 19\\
Abstracts deadline has been extended to
 Friday, \textbf{November 6} at 5:00 p.m. ET.

\HREF{https://march.aps.org/abstracts/} {Sorting Categories}\\
03.0 Statistical and Nonlinear Physics (GSNP)\\
03.07.00 Pattern Formation and Spatio-temporal Chaos\\
03.08.00 Chaos and Nonlinear Dynamics
    }

    \PCpost{2020-10-25}{
SIAM  DS21 May 23 - 27, 2021\\
Minisymposium Proposal
\HREF{https://www.siam.org/conferences/cm/submissions-and-deadlines/ds21-submissions-deadlines}
{Submission Deadline}:
\textbf{November 23, 2020}, 11:59 p.m. ET
    }

    \PCpost{2020-10-25}{
2021 Dynamics Days DD21-Europe\\
August 24-28, 2020, Nice, France\\
Minisymposium Proposal
\HREF{https://dynamicsdays2020.univ-cotedazur.fr/}
{Submission Deadline}:
not announced yet.
    }
\end{description}


\section{Chronotopic literature}
\label{sect:chronotopic}
\bigskip

   % *********************************************************************
\hfill   {\color{red} The latest entry at the bottom for this blog}

\bigskip

\begin{description}

    \BGpost{2016-03-02}{
Just stumbled upon Lepri, Politi and Torcini\rf{LePoTo96}
{\em Chronotopic {Lyapunov} analysis.
{I. A} detailed characterization of {1D} systems}.

Are you familiar with this? Somewhere in the direction I thought about.
    }

    \PCpost{2016-03-02, 2016-09-03}{
Politi and collaborators work is very close to our way of spatiotemporal
thinking. See
\refsect{sect:GiLePo95},
\refsect{sect:LePoTo96},
\refsect{sect:LePoTo97} and
\refsect{sect:PoToLe98}.
If you read that literature, please share what you have learned
by writing it up there.
    }

    \PCpost{2016-03-02}{
Also Paz\'o \etal\rf{PaSzLoRo09} {\em Structure of characteristic
{Lyapunov} vectors in spatiotemporal chaos}. Actually (I hesitated to
bring it up) this line of inquiry goes smoothly into Xiong Ding's
inertial manifold dimension project.

Not sure Li \etal\rf{LXSH08} {\em Lyapunov spectra of coupled chaotic
maps} is of any interest, but we'll know only if we read it.

    }

\MNGpost{2016-09-06}{ {\bf Chronotopic Approach}
I've been reading \refrefs{LePoTo96, LePoTo97, PoToLe98,GiLePo95}
on chronotopic approach to spatiotemporal chaos.
    }

\MNGpost{2016-09-06}{
Branching out to get a better grasp of what's out there, I read
Pikovsky and Politi\rf{PikPol98}
{\em Dynamic localization of {Lyapunov} vectors in spacetime chaos},
It discusses the complex Ginzburg-Landau, two types of coupled maps, as
well as \KSe, so I thought it might be useful somehow.
}

\PCpost{2016-09-06}{
   Pikovsky and Politi\rf{PikPol98} is a stat mech paper about using KPZ
   equations in pattern formation, do not waste time on it right now.
    }

\item[Rafael 2016-09-29]
I spent a lot of time in the coupled cat maps, but in the regime of small
coupling. One thing I did explore numerically is when the conjugacy given
by the structural stability breaks down as one turns up the coupling.

\item[Rafael 2016-10-10]
The main paper about the coupled maps is the paper with Miaohua Jiang. We
show that the local chains of Anosov remain Anosov under local couplings.
The partitions remain the same if you make changes of coordinates that
are essentially local.

Of course, the fact that there is a regime of large perturbations in
which this does not happen begs the question of studying the transition.
Some of this has been studied also by Bastien Fernandez\rf{Fernandez14}.

I have done some preliminary numerics (too crude to show). One
possibility is that some of the Lyapunov exponents go to zero. Another is
that the Lyapunov exponents remain away from zero but that the angle
between the splittings goes to zero. There are heuristic arguments that
the second possibility should occur. (Almost a proof when there is a
system small coupling to another more massive one).

There were other points of the discussion. The space dynamics for PDE's.
This contains references to older papers notably Kirchgassner and Mielke
as well as applications to some papers.

I think Weinstein\rf{Weinstein85} is related.

\item[2011-02-17 PC]
\emph{A large-deviation approach to space-time chaos}
% used to say ``A statistical-mechanics approach to space-time chaos"
by  Pavel V. Kuptsov and Antonio Politi\rf{KupPol11},
\arXiv{1102.3141}. They say:

``
We show that the analysis of Lyapunov-exponents fluctuations
contributes to deepen our understanding of high-dimensional chaos. This is
achieved by introducing a Gaussian approximation for the entropy function that
quantifies the fluctuation probability. More precisely, a diffusion matrix $D$ (a
dynamical invariant itself) is measured and analyzed in terms of its principal
components. The application of this method to four (conservative, as well as
dissipative) models, allows: (i) quantifying the strength of the effective
interactions among the different degrees of freedom; (ii) unveiling microscopic
constraints such as those associated to a symplectic structure; (iii) checking
the hyperbolicity of the dynamics.
''

    \PCpost{2016-09-28}{                                        %\toCB
Isola, Politi, Ruffo and Torcini\rf{IPRT90}
{\em Lyapunov spectra of coupled map lattices}.

Fontich, de la Llave and Martín\rf{FoLlMa11}
{\em Dynamical systems on lattices with decaying interaction
{I: A} functional analysis framework}.
[...] consider weakly coupled map lattices with a decaying interaction.
[...] applications of the framework are the study of the structural
stability of maps with decay close to uncoupled possessing hyperbolic
sets and the decay properties of the invariant manifolds of their
hyperbolic sets, in the companion paper by Fontich et al. (2011).
    }

\MNGpost{2016-11-18}{:
There is a storm in the distance however, as this general procedure is
ruined for the spatial problem.  According to the chronotopic
literature\rf{LePoTo96,LePoTo97,PoToLe98,GiLePo95},
iteration in space typically does not converge to the same attractor as
iteration in time, and generally corresponds to a strange repeller.
Therefore I cannot hope to form an initial guess loop from using a
Poincar\'e section in the spatial direction, as typically all of my
Fourier coefficients go off to infinity before a recurrence is found.
    }

\PCpost{2017-03-02}{
I asked GaTech library to order
Pikovsky and Politi\rf{PikPol16}
{\em Lyapunov Exponents: A Tool to Explore Complex Dynamics}.
    }

\MNGpost{2017-11-07}{
My \emph{spatiotemp/blog} comments of
{\bf 2017-11-02} % on page~\pageref{2017-11-02MNG}
were mainly predicated by the fact that once we find these roots, I don't think
we can apply the same type of reasoning as Politi and Torcini\rf{PolTor92b} as
they aren't truly fixed points of a fictitious dynamical system, like so,
$F(\akj, T, L) = {\akj, T, L}$. But rather, like I have already
described, they are the roots of a system of nonlinear algebraic equations,
$F(\akj, T, L) = 0$.
    }

    \PCpost{2019-05-11}{
My extensive notes on extensivity in
Carlu, Ginelli, Lucarini and Politi\rf{CGLP19} {\em Lyapunov analysis of
multiscale dynamics: the slow bundle of the two-scale {Lorenz} 96 model}
are in \texttt{lyapunov/dailyBlog.tex}.
    }

\item[2020-12-16  Alessandro Torcini] alessandro.torcini@u-cergy.fr \\
to Domenico:

si avevo visto l'annuncio della tesi di Gudorf ma alla fine non avevo
partecipato, se ho capito bene lui e' riuscito  a fare nello spazio-tempo
continuo una cosa che io e Politi avevamo tentato nel 1990 in tempo
discreto e spazio discreto (mappe accoppiate), cioè'  riscrivere come un
modello Markoviano la evoluzione spazio-temporale di un sistema con caos
spazio temporale in termini di unità spazio temporali, tipo i mattoncini
del Tetris.

(I had seen the announcement of the thesis of Gudorf but in the end I had
not participated, if I understood well he was able to do in continuous
space-time something that Politi and I had tried in 1990 in discrete time
and discrete space (coupled maps), that is to rewrite as a Markovian
model the spatio-temporal evolution space-time evolution of a system with
space-time chaos in terms of space-time units time-space units, like
Tetris bricks.)

Io ci ho speso 6 mesi sopra e credo di avere ancora quaderni su quaderni,
ma alla fine non pubblicammo mai nulla con Politi. A parte un PRL del 1992
molto poco citato ed un Chaos.

(I spent 6 months on it and I think I still have notebooks upon notebooks,
but in the end we never published anything with Politi. Except for a 1992 PRL
very little cited and a Chaos.)

Poi scrivemmo 3 lavori con Lepri su chronotopic approach al caos spazio
temporale, questa roba e' finita in 2 o 3 libri, ma di fatto la linea di
ricerca e' stata molto poco seguita , infine nel 2013 siamo riusciti a
calcolare tutto lo spettro dei i comoving Lyapunov exponents (lavoro
ignoto ai piu),

(Then we wrote 3 papers with Lepri on chronotopic approach to space-time
chaos. time chaos, this stuff ended up in 2 or 3 books, but in fact the
research line of research was very little followed, finally in 2013 we
managed to calculate the whole spectrum of the the whole spectrum of
comoving Lyapunov exponents, work unknown to most,)

A. K. Jiotsa, A. Politi, and A. Torcini, {\em Convective Lyapunov Spectra}
J. Phys. A 46 (2013) 254013.


potete trovare tutto nella mia web page e scaricare tutto

(you can find everything in my web page and download everything)

\HREF{https://perso.u-cergy.fr/~atorcini/prepri.html}
{perso.u-cergy.fr/~atorcini/prepri.html}

Per la review  a SIAM grazie va bene, per i fondi sino al 31 marzo non ho
problemi, dopo si, spero di poter pagare prima .... la FEE.

(For the review in SIAM thank you is fine, for the funds until March 31 I
don't have any problem, after that I hope to be able to pay before ....
the FEE)

Spero di riuscire a studiare la tesi di Gudorf prima o poi.

(I hope to be able to study Gudorf's thesis sooner or later.)

A presto


\end{description}


\section{PolTor92b Towards a statistical mechanics of spatiotemporal chaos}
\label{sect:PolTor92b}

\begin{description}

\NBBpost{2017-10-31}{I looked everywhere but could not find
Politi and Torcini\rf{PolTor92b}
{\em Towards a statistical mechanics of spatiotemporal chaos} (1992)
in this blog. The abstract:

\begin{quotation}
Coupled H{\'e}non maps are introduced to model in a more
appropriate way chaos in extended systems. An effective technique allows
the extraction of spatiotemporal periodic orbits, which are then used to
approximate the invariant measure. A further implementation of the
$\zeta$-function formalism reveals the extensive character of entropies and
dimensions, and allows the computation of the associated multifractal
spectra. Finally, the analysis of short chains indicates the existence of
distinct phases in the invariant measure, characterized by a different
number of positive Lyapunov exponents.
\end{quotation}

We should all read it very
carefully. They use Biham-Wenzel\rf{afind} to infer spatiotemporal periodic orbits
and their symbolic dynamics by introducing a continuous
fictitious time.
%The method probably works only for maps but nevertheless
%seems quite interesting.

I'm not sure if they are using periodic orbits of different chain spatial
period $\speriod{}$ in order to estimate the statistics of the system at
thermodynamic limit $\speriod{}\rightarrow \infty$. If that's the case
and the dynamical $\zeta$ function is designed for this purpose, then
this paper is very similar to what we have in mind.
}

%{\bf Predrag 2018-01-20} moved all Politi and Torcini\rf{PolTor92b} blogposts
%to \texttt{blogCats}, dailyCats.tex.
% Predrag 18 April 2020 returned to siminos/spatiotemp/blog.tex

\MNGpost{2017-10-31}{
% Reading Politi and Torcini\rf{PolTor92b}
% {\em Towards a statistical mechanics of spatiotemporal chaos}
I find it interesting that they use a continuous fictitious
Biham-Wenzel\rf{afind} dynamics for a spatiotemporal system
of mappings, while I have a discrete fictitious time (in the form of my
spatiotemporal mapping) introduced for continuous (albeit discretized)
spatiotemporal equations.

If a paper is worth something, other people cite it. Currently it has
\HREF{https://journals.aps.org/prl/cited-by/10.1103/PhysRevLett.69.3421}
{\bf 13} APS and
\HREF{https://scholar.google.com/scholar?cites=17889061931444888709&as_sdt=80005&sciodt=0,11&hl=en}
{\bf 22} Google Scholar citations.
%\rf{PolTor92b}
    }
\end{description}

%\MNGpost{2017-11-08}{
To motivate the Politi and Torcini\rf{PolTor92b} coupled H\'enon map
lattice, we start by a review a single H\'enon map, written using
the conventions of \wwwcb{}.

\MNGedit{Note: Politi and Torcini say that in the case of $\epsilon = 0$ that
the original H\'enon map is retrieved, but if you actually do this then indices
don't match the original equation.}
\bea
\label{e-singleHenon}
x_{n+1} &=& 1 - a y_{n+1}^2 + b x_{n-1}
       \continue
y_{n+1} &=& x_{n}
\,,
\eea
or, as a 3-term recurrence
\beq \nonumber
\ssp_{n+1} + a \ssp_n^2 - b\,\ssp_{n-1} = 1
\,.
\eeq
The parameter $a$ quantifies the ``stretching'' and
$b$ quantifies the ``contraction''.

The single H\'enon map is nice because the system is not linear, but has binary dynamics.

The deviation of an approximate trajectory from the 3-term recurrence is
\beq \nonumber
v_n = \ssp_{n+1} - (1 - a \ssp_n^2 + b\,\ssp_{n-1})
\eeq
In classical mechanics force is the gradient of potential, which
Biham-Wenzel\rf{afind} construct as a cubic potential
\beq \nonumber
V_n = \ssp_n(\ssp_{n+1} - b\,\ssp_{n-1} - 1) + a \ssp_n^3
\ee{BWpotential}
With the cubic potential of a single H\'enon map we can start to look for
orbits with initial conditions of two points (two point recurrence
relation requires this) and make the guess as we iterate in time. A
particular guess is to choose a sequence of maxima/minima of the
potential.

In order to accurately enumerate the orbit with symbolic dynamics, in
order to choose which way you "roll" the potential needs to be modified
by $\pm 1$, as when viewed from the perspective of the cubic potential,
trajectories roll downhill, so in order to flip the direction one must
apply a flip. The symbolic dynamics is therefore binary and determined by
these flips. One can just list the binary sequences and see if the orbit
is realized by the system. \HREF{Chaosbook.org}{Chaosbook} does this up
to sequences of length $13$, while it has been done by Grassberger, Kantz
and Moenig\rf{grass89} to symbol length $32$. This means that
there could be as many as ${2^{32}}/{32}$ distinct \po s.

%Making x a complex number you can find non-physical solutions; what we're
%really doing is finding roots of polynomials. For a cycle of length N,
%the degree of the polynomial is $e^{2N}$? Finding the real roots of
%these equations is one way of finding these solutions, but this
%fictitious time is really just a way of finding these solutions.

In the Politi-Torcini\rf{PolTor92b} coupled H\'enon map equations $\zeit$
is the index associated to time, while $n$ is the index associated with
space:
    \MNG{2017-11-08}{
Politi-Torcini \Henon\ map form  differs from the ChaosBook convention
\refeq{e-singleHenon}. If we take their claim that in the $\epsilon = 0$
case we should retrieve the classical H\'enon equations very strictly,
this is how they should appear I believe. The difference lies in the time
index $\zeit+1$ versus $\zeit$ of the $y$ terms.
\bea
\ssp_{n,\zeit+1} &=& 1 - a(y_{n,\zeit+1})^2 + b\,\ssp_{n,\zeit-1}
         \continue
y_{n,\zeit+1} &=&
(1+\epsilon)\ssp_{n\zeit} + \frac{\epsilon}{2}(\ssp_{n+1,\zeit} + \ssp_{n-1,\zeit})
\,.
\label{e-coupledHenon}
\eea
        }
\bea
&& \ssp_{n,\zeit+1} + y_{n\zeit}^2 - b\,\ssp_{n,\zeit-1} = a
         \continue
&& y_{n\zeit} = \ssp_{n\zeit}
   + \frac{\epsilon}{2}(\ssp_{n+1,\zeit} -2\ssp_{n\zeit}+ \ssp_{n-1,\zeit})
\label{PolTor92(2.1)}
\eea

In the Hamiltonian $b = -1$ case the only parameter is the stretching
$a$; when it is small some orbits become stable, not everyone is
unstable. If there's not strong stretching then there is a mixture as the
hyperbolicity isn't dominating. All of these coupled maps, say something
wild happens at each site (alternating between 1 and -1 is far in this
case), and THEN you couple it weakly to its neighbors.
%Two adjoining
%shapes are not coupled at all due to the fact that its so hyperbolic.
% Usually in chaotic dynamics there is stretching, but due to folding in
% this case a binary sumbolic dynamics is created.

But when the coupling with neighbors is very strong, its a very different
phenomena. The cats have written a Helmholtz in space and time where the
Laplacian is weighted by $+1$ as opposed to $1, -1$ in the Minkowski
case.
Second order operator that has different weights (as determined by some metric)
its called the Beltrami operator.
Predrag also thinks the sign is different.
This is 1992 though, and there's not much there because its Phys. Rev.
Lett. so there must be real work somewhere.

They cite the fact that most people don't use it for invertible dynamics.
Then they say the spatially and temporally periodic orbits are extracted
using the Newton method. What they actually do, they take $\speriod{} =
1,2,3,4$ only. They don't elucidate on interesting tricks, like how to
use the method for coupled maps. The density of periodic orbits in the
invariant measure is stated, but this is only really known for single
maps, Predrag doubts this.

Politi and Torcini\rf{PolTor92b} is one of the first papers that uses
spatiotemporal symbolic dynamics, as seen by Predrag in the literature.
Symbol of a torus is really just a lattice label. Politi and Torcini use
doubly periodic boundary conditions, with canonical values from H\'enon
for the parameters.

When you write the equations in diagonal,
form you'll likely take a square root of it. Then, they say some orbits are pruned,
some exist only with uncoupled case. That helps them because if you can prove you only
lose orbits and never gain orbits you can see if certain orbits are realized or not.

It is important to realize that not all \twots\ belong to the
inertial manifold. There are isolated orbits, corresponding to fixed points.

% Do not allow invariant subspaces? maybe. Depends on how unstable the
% invariant subspace is because sometimes you can be glued to it.
The orbits are found with Newton method.

They discarded all of the cycles that had a sequence, specifically
$\bar{000}$ in symbolic dynamics, is not allowed. The application of the
zeta function, "they don't know what they are doing, so ignore it".

Their intuition is that the inertial manifold is extensive: if you double the spatial
length, you should double the number of physical Lyapunov exponents as you go
in time. They compute some Jacobian, they are just the normal Jacobians
in time. Using these they compute $2J \times 2J$ sized matrices for the spatiotemporal
domain, (thinks Bloch theorem doesn't apply, maybe).
% Maybe this Bloch theorem would be worthwhile in understanding.
They also say to take the logarithm and divide it
by the size of the domain. Their entropy is a sum of temporal Lyapunov exponents
divided by
$\speriod{}$. They have an intuition but its hard to check because of the small $\speriod{}$. One possible
solution is to use the fixed point, as its a representative of the stretching rate
in the neighborhood. "Let's just say the typical stretching rate is roughly the same no
matter the domain size." I.e. they assume its a frozen state so it doesn't matter how
big the domain gets. Grand canonical formalism for the Zeta function, but it can
be "safely ignored". Multifractal stuff seems useless, look at this $h$, instead
of plotting everything from infinity to infinity, but can subtract something and
it works out and they get a plot. But there seems to be an envelope which
may be connected to the quantity they try to claim, maybe.

Different phases, but they notice that different periodic orbits have different number
of positive Lyapunov exponents. They interpret periodic orbits with the same number belong
to the same 'phase'.

The problem only uses Biham in time; Not clear why they did it because they say its
invertible, which makes sense because Hamiltonian.
% This is different from Cat map
% because the symmetry imposed makes it same in space and time.

%All symbolic dynamics in the cat map paper, the way the equations work,
%the 'Laplace' equation (symmetric in space and time) has a stretching
%rate that depends \MNGedit{on the number of neighbors)}? (I had this
%written but I thought it only depended on $s$).

%The source terms on the RHS, make sure that its modulo two pi. They act
%like symbolic dynamics, and PC has showed that all of these types of maps
%have to be put back into the torus (unit square in cat map case).
%Optimistic because it will be a general equation.

For a particular solution, the potential is just a set of numbers, even though it
depends on $x$. It will have to be evaluated at the linearization of the nonlinear
equations and evaluated along the orbit. e.g. stablity of orbits do not look like
constant anything. PC Thinks the Jacobian will work out somehow.

\begin{description}
\MNGpost{2017-10-31}{
{\bf Part Two: Discussions from the Invariant Solns Meeting}

Ignore the last third of the paper. The one thing PC doesn't understand,
is that they are studying coupled {\HenonMap}s because they're invertible,
but he doesn't understand why they care.

They uses the standard parameter values of H\'enon.

Fictitious time isn't important, the method is the method of Biham where
they derive a cubic potential such that its derivative is the H\'enon map.

%The people who used coupled map lattices before Gutkin and Osipov,
%you always assumed weakly coupled with nearest neighbors, such
%that the coupling takes the form of the Laplacian.
%
%Gutkin and Osipov, PC, impose a symmetry such that the coupling
%in time is the same as the coupling in space such that there
%is something like Helmholtz equation in space-time.

The coupling is what determines the type of behavior here,
in the weak coupling limit, everything looks unstable on
its own, but when the coupling is strong
The model is then determined to be an ergodic model, because
it is fully developed due to the length scale imposed but
no laminar patches (No point where there are
a different number of unstable directions).

If we write it carefully, take a Hamiltonian map,
rewrite in terms of Beltrami operator (Laplacian with
metric). Looking at this operator, its bilinear in
derivative, but the derivatives have different weights
and can have opposite signs.

Elliptic structure in PC's case, but thinks the H\'enon
there should be a negative sign.

If you have a spatially periodic chain, (they still
think of a chain in space evolved in time, but they
change this in the future). They say if you have periodic
lattice, you should use Bloch theorem, which is what
is done in condensed matter. They write the Bloch
theorem, but we don't know why. PC believes it may
be because the coupling is weak, so deformations are
long wavelength deformations.

Explicitly the stretching parameter, changed from small
to large, the problem becomes hyperbolic and sines and
cosines change to hyperbolic sines and cosines.

In order to rewrite the equations in one variable that has a spatiotemporal
Beltrami operator, we need to modify \refeq{e-coupledHenon}. First
we write the set of equations as two step recurrence in one variable,
\beq
\ssp_{n+1}^j + b \ssp_{n-1}^j= 1 - a((1+\epsilon)\ssp_{n}^j + \frac{\epsilon}{2}(\ssp_n^{j+1} + \ssp_n^{j-1}))^2 \, ,
\eeq
Then we can add and subtract $2 \ssp_n^j$ to the LHS of the equation,
\beq
(\Box_t - 2)\ssp_n^j =
1 - a((1+\epsilon)\ssp_{n}^j + \frac{\epsilon}{2}(\ssp_n^{j+1} + \ssp_n^{j-1}))^2 \, ,
\eeq
Because the quadratic nonlinearity is where the spatial part of the
Beltrami or Laplacian is, I don't know how to get around this and combine
space and time currently. I recall Predrag mentioning something about a
square root but it doesn't seem very well motivated because the spatial
coupling is completely separated from time coupling unless I'm missing
some type of approximation.
    }

\item[2020-06-25 Predrag]
Perhaps the easiest thing would be to replace the \Henon\ in
\refeq{PolTor92(2.1)} by the Lozi map?
\end{description}


\section{PolTor92 Periodic orbits in coupled {H{\'e}non} maps}
\label{sect:PolTor92}

\begin{description}

\item[2020-05-31 Predrag] Politi and Torcini\rf{PolTor92} {\em Periodic
orbits in coupled {H{\'e}non} maps: {Lyapunov} and multifractal analysis}
is quite close to our \catlatt. The problem is harder,
as the {H{\'e}non} map is nonlinear; MUST CITE in \refref{CL18}.

They study \emph{\spt\ \Henon}, a (1+1)-spacetime lattice of
\Henon\ maps orbits which are periodic both in space and time, and note
that the dependence of the lattice field at time $\field_{t + 1}$ on the
two previous time steps prevents an interpretation of dynamics as
the composition of a local chaotic evolution with a diffusion process,
and that the $|b|=1$ case(s) could be important as examples of Hamiltonian
lattice field theories. (\textbf{Predrag}: they do not comment on
the role of the spacetime asymmetry of \spt\ \Henon.)

Their numerical method is an extension of Biham and Wenzel\rf{afind} for
the single \Henon\ map, with symbols $\Ssym{n\zeit}$ in $\A=\{0,1\}$. Any
fixed point in fictitious time corresponds to a spatio-temporal cycle
$\BravCell{\speriod{}}{\period{}}{\tilt{}}$.

The search of periodic orbits is further simplified by the fact that a
small coupling prunes some of cycles which are present for $\epsilon=O$.
Therefore, the knowledge of the topological structure of the single
\Henon\ map yields a symbolic encoding of the dynamics and allows for
restricting the set of candidate admissible symbol {\brick}s to be
investigated. For $a=1.4$, $b=0.3$ this works well for $\epsilon=0.1$.

They comment on existence both time-\eqva\ $\BravCell{\speriod{}}{1}{0}$
and time-\reqva\ $\BravCell{\speriod{}}{1}{\tilt{}}$,  $\tilt{}\neq0$
(seen as stationary patterns in a reference frame moving with a constant
velocity).

A problem in reconstructing the statistical properties of an attractor
from periodic orbits is ensuring that all orbits used belong to the
natural invariant measure. For instance, in the single \Henon\ map, one
of the two fixed points is isolated and it does not belong to the strange
attractor. Something similar should occur in the CML.

A family of specific Lyapunov exponents is defined, which estimate the
 growth rate of spatially inhomogeneous perturbations,
related to the comoving Lyapunov exponents.

The \emph{$\zeta$‐function formalism} is used to analyze the scaling
structure of the invariant measure both in space and time.
\\
(\textbf{Predrag}: here things fall apart. They do numerics for various
small fixed $\speriod{}$ or $\period{}$, but have no path to constructing
a spacetime $\zeta$‐function.)

In the case of a CML, the \po\ weights depend exponentially both on space
and time variables, $t_j=r_j^{\speriod{}\period{}}$. This suggests that
the $\zeta$‐function formalism could be effectively extended, by
performing an additional sum over all spatial periods. Unfortunately, a
straight implementation of this scheme is not so effective as in the
low-dimensional case. Therefore, we limit ourselves to apply the standard
formalism, checking afterwards the dependence on the length chain
$\speriod{}$.


\end{description}

\section{PoToLe98 {Lyapunov} exponents from node-counting}
\label{sect:PoToLe98}

\begin{description}

\item
Politi, Torcini and Lepri\rf{PoToLe98}
{\em {Lyapunov} exponents from node-counting arguments}
is a promising start, but there does not seem to have been any followup
since 1998...

\end{description}

The \emph{chronotopic approach} aims to extending the concept of Lyapunov
spectrum to spatially inhomogeneous perturbations.

The main result of the chronotopic approach is the existence of a
dynamical invariant, the entropy potential, the knowledge of which allows
to determine all properties of the evolution of localized as well as
extended perturbations.

One can describe the spatial structure of a generic Lyapunov vector with
a single complex number \(\tilde{\mu}= \mu+ik\), the real part of which
is the exponential growth rate, while the imaginary part is the
wavenumber. The frequency $\omega$ can be read as the imaginary part of
the complex number \(\tilde{\lambda}= \lambda+i\omega\), where $\lambda$
is the temporal growth rate (i.e. the Lyapunov exponent) of the given
perturbation. The analyticity properties of the ``dispersion relation"
connecting $\tilde{\mu}$ with $\tilde{\lambda}$ furnish the last
ingredient to ``prove" the existence of an entropy potential\rf{LePoTo97}.

This paper introduces a wavenumber by define ``rotation numbers" as the
imaginary counterpart of the Lyapunov exponents. They compute the
Lyapunov spectrum by using the transfer matrix approach. The approach is
limited to a class of coupled map lattices (CMLs) with everywhere
expanding multipliers.

(First) they assume a time-stationary \spt\ solution and
compute the 1 spatial dimension {\jacobianOrb}. That resembles the
tight-binding approximation of the lD Schr\"odinger equation (with
imaginary time) in the presence of a random potential, \ie., the Anderson
model. They note the close analogy with the computation of the
vibrational spectrum of a chain with random masses.
The spectrum of the Schr\"odinger problem can be determined without
diagonalizing the operator (which is the sum of the discretized spatial
Laplacian and a diagonal operator). Its symmetry ensures the
applicability of the node theorem which states that the eigenfunctions
are ordered according to the number of their zeros\rf{PasFig92}.

For a time-stationary \spt\ solution, one can always redefine site fields
$\field_{n\zeit}$ to make them all a constant field $\field_{oo}$. Then
the spatial structure of the corresponding Lyapunov vector counts the
nodes. Furthermore, all eigenvalues are real, \ie, no rotations in
tangent space.

(Second), they consider orbits of temporal period $\period{}>2$. The
operator is a banded matrix (of width $2\period{} + 1$) so that we are
dealing with a sort of Schrodinger problem with long-range hopping. The
fundamental difference is that no similarity transformation can turn the
operator into a symmetric matrix, hence generic existence of complex
eigenvalues. That's a problem for them, as the node theorem is proved
only for operators with a strictly real and positive spectrum. They waffle.

Still, they do a numerical calculation for \LTS{11}{7}{0} and
\LTS{7}{5}{0} Bravais cells and get the correct counting, their figs.~2
and 3. They finish with

``More important, in our opinion, is the question whether the same
approach can be extended to continuous-time and -space systems. We
believe that instead of checking numerically whether this is true or not,
it is more important to look for the possibly deep reasons that lie
behind the apparent validity of the conjectures presented in this
paper.''

\begin{description}

\PCpost{2020-04-18}{
A long shot, but maybe
Pastur and Figotin\rf{PasFig92}
{\em Spectra of Random and Almost-Periodic Operators} Chap.~III
offers some estimate of the \KS\ spectra, and provides mean node-count for
\KS?
}

\end{description}

\section{PolPuc92 Invariant measure in coupled maps}
\label{sect:PolPuc92}

\begin{description}

    \PCpost{2016-11-06}{
Politi and Puccioni\rf{PolPuc92} {\em Invariant measure in coupled maps}
write: ``
The state of affairs is much less clear when we pass from closed chains (as
above) to sub-chains of an-in principle-infinite lattice. This corresponds to
the canonical-ensemble picture of statistical mechanics: the system of
interest (sub-chain of length E) is coupled with a thermal bath given by the
rest of the chain. From the previous considerations, the attractor
corresponding to an isolated system would fill, for E sufficiently large, a
$\rho E$\dmn\ manifold. The main effect of the coupling with the heat bath is
to add a sort of ``external noise'' dressing the manifold along all
directions, and thus making the resulting invariant measure to become
$E$\dmn.
''

This paper is lots of hand-waving, so I gave up on reading it.
    }
\end{description}

\section{PolTor09 Stable chaos}
\label{sect:PolTor09}

\begin{description}
\item
Politi and Torcini\rf{PolTor09} {\em Stable chaos} (2009),
\arXiv{0902.2545}
\end{description}

Chaos is associated with an exponential sensitivity to tiny perturbations
in the initial conditions, so that the presence of at least one positive
Lyapunov exponent is considered as a necessary and sufficient condition
for the occurrence of irregular dynamics in deterministic dynamical
systems. In fact, the first observation in coupled-map models of
stochastic-like behaviour accompanied by a negative maximum Lyapunov
exponent came as a big surprise. % [2,3].
The unexpected coexistence of local stability and chaotic behaviour, due
to the phenomenon was called stable chaos (SC). The irregular behaviour
is a transient phenomenon that is restricted to finite-time scales.

We might want to study the `chronotopic approach' of eqs.~(10) to (17).

\renewcommand{\ssp}{x}
\renewcommand{\Xx}{\ensuremath{\mathsf{X}}}      % Boris
\renewcommand{\Laplacian}{\Delta}
\renewcommand\speriod[1]{{\ensuremath{\ell_{#1}}}}  %continuous spatial period
\renewcommand\period[1]{{\ensuremath{\ell_{#1}}}}  %continuous time period
\renewcommand{\Ssym}[1]{{\ensuremath{m_{#1}}}}    % Boris
% \newcommand{\Ssym}[1]{{\ensuremath{s_{#1}}}}  % ChaosBook

%%%%%%%%%%%%%%%%%%%%%%%%%%%%%%%%%%%%%%%%%%%%%%%%%%%%%%%%%%%%%%%%%%%%%%%
\printbibliography[heading=subbibintoc,title={References}]
