% siminos/reversal/Lind1d.tex      pdflatex LC21; bibtex LC21
% temporary:  siminos/spatiotemp/chapter/LC21Lind1d.tex
% $Author: predrag $ $Date: 2021-12-20 23:24:25 -0500 (Mon, 20 Dec 2021) $

% Predrag 2021-08-08: shared with siminos/reversal/LC21.tex

%%%%%%%%%%%%%%%%%%%%%%%%%%%%%%%%%%%%%%%%%%%%%%%%%%%%%%%%%%%%%%%%%%%%%%%%%%
\section{Lind zeta function}
\label{sect:LC21Lind1d}

\Poincare\rf{poincare} was the first to  recognize the fundamental role
{\po}s play in shaping ergodic dynamics. The first step
in this program is the count of {\po}s, addressed in
\refrefs{ArtMaz65,ChAnPi85,brucks,Lutzky88,Lutzky93,BLMS91,XH94,
CheLou97,BriPer05,CBcount}, starting with 1950's
\HREF{https://en.wikipedia.org/wiki/Pekka_Myrberg} {Myrberg}
investigations of {\po}s of quadratic maps, in what was arguably the
first application of computers to
dynamics\rf{Myrberg58a,Myrberg58b,Myrberg59,Myrberg62,Myrberg63}.
Such orbit counts are most elegantly encoded by \emph{topological zeta
functions}.

For a discrete time dynamical system
\(
\ssp_{\zeit+1}=\flow{}{\ssp_{\zeit}}
\,,\)
period-$\cl{}$ solutions (in present context, the period-$\cl{}$ Bravais
cell {\lattstate}s) are \emph{fixed point}s of the $\cl{}$th iterate map
$\map^\cl{}$.
While the symmetry of a time-invariant dynamical `law' $\map$ is
the {infinite cyclic group} \Cn{\infty},
the group of all temporal lattice translations, for a period-$\cl{}$
solution the symmetry is $\cl{}$-steps translation subgroup $H_{\cl{}}$.

This observation motivates definition of a general group-theoretic
\emph{fixed-points counting} generating  function that relates the number
of {\lattstate}s to the number of prime orbits for any \statesp\ map
\(
\map : \pS \to \pS
\)
with a symmetry group $\Group$, a counting function that applies equally
well to multi\dmn\ lattice field theories\rf{CL18} as to the 1\dmn\
theories considered here.

Let $\Group$ be a group whose action
$\alpha: \Group \times \pS \to \pS$
permutes elements of a set $\pS$.
The Lind zeta function\rf{Lind96} is then defined as
\beq
\zeta_{Lind}(t) =
\exp \left( \sum_{H} \;
            \frac{N_{H}}{|\Group/{H}|}t^{|\Group/H|}
      \right)
\,,
\ee{LindZeta} % was {LindZeta}
where the sum is over all subgroups $H$ of $\Group$
of multiplicity $|G/H| < \infty$, and $N_{H}$ is the number of states
in $\pS$ invariant under action of (\ie, fixed points of) subgroup $H$,
\beq
N_{H} =
   |\{\Xx \in \pS : \mbox{ all } h \in H \quad \alpha(h,\Xx) = \Xx\}|
\,.
\ee{LindN_H} % was {LC21Ryu17eq:1.3A}
The multiplicity factor $|\Group/{H}|$ is best explained by working out a
few examples.

In the lattice field theory setting, $\pS$ is the set of all
{\lattstate}s.
For 1\dmn\ lattice field theories, the group $\Group$ is either the
{infinite cyclic group}  \Cn{\infty} of all lattice translations \refeq{C_infty},
or
the {infinite dihedral group} $\Dn{\infty}$  of all translations and
reflections \refeq{LC21D_infty}.
Their finite-multiplicity subgroups $H$ are,
respectively, infinite translation subgroups $H_{\mathbf{a}}$
\refeq{H(n)subgroup}, or $H_{\mathbf{a}}$ and infinite dihedral subgroups
$H_{\mathbf{a},k}$ \refeq{H(n,k)subgroup}.

\subsection{Artin-Mazur zeta function}
\label{sect:ArtinMazur}

Assume that the symmetry group of a given discrete time dynamical system
is the group of temporal lattice translations \Cn{\infty}, \ie, that the
defining law of dynamics is time invariant. 1\dmn\ infinite translation
subgroup $H_{\mathbf{a}}$ \refeq{H(n)subgroup} leaves {Bravais cell} of
{period} \cl{} \refeq{1DBravLatt} invariant, so the sum over $H$ in
\refeq{LindZeta} can be replaced by the sum over periods $\cl{}$, with
$N_\cl{}$ the number of {\lattstate}s of period $\cl{}$. For \Cn{\infty}
subgroups (contemplate the \Cn{\infty} column of
\reffig{fig:1dLatStatC_5}) %\,(1)-($\shift_4$)
the multiplicity is
\beq
|\Cn{\infty}/H_\cl{}|=\cl{}
\,.
\ee{CinfHnMult}
The Lind zeta function
\refeq{LindZeta} now takes form of the Artin-Mazur zeta
func\-tion\rf{ArtMaz65,CBcount}
\beq
\zeta_{\mbox{\footnotesize AM}}(z) =
     \exp\left(\sum_{\cl{}=1}^\infty
\frac{N_\cl{}}{\cl{}} z^\cl{}
         \right)
\,,
\ee{AMzeta}
which, from group-theoretic perspective, is the statement that the law
governing the dynamical system is time invariant.

In what sense is $\zeta_{\mbox{\footnotesize AM}}(z)$ a {\lattstate}
count generating  function? Given the $\zeta_{\mbox{\footnotesize
AM}}(z)$, the number of periodic points of period $\cl{}$ is given by its
logarithmic derivative (see
\HREF{http://chaosbook.org/chapters/ChaosBook.pdf\#section.18.7}
{ChaosBook}\rf{CBcount})
\beq
\sum_{\cl{}=1}N_\cl{} z^\cl{}
    = \frac{1}{\zeta_{\mbox{\footnotesize AM}}}
            \,z\frac{d}{dz} \zeta_{\mbox{\footnotesize AM}}
\,.
\ee{zetatop-N}
Let us evaluate $\zeta_{\mbox{\footnotesize AM}}$ for scalar lattice
field theories studied here.

\paragraph{Bernoulli.}
The number of Bernoulli system period-\cl{} {\lattstate}s is given in
\refeq{noPerPtsBm}, so
\bea
\zetatop(z)
 =  \exp \left(-\sum_{\cl{}=1}^\infty
\frac{{s}^\cl{} - 1}{\cl{}}z^\cl{}
         \right)
 =
\frac{1 -  {s}z}{1 - z}
\,.
\label{BernZeta}
\eea
The numerator $(1 - {s}z)$ says that a Bernoulli system is a full
shift\rf{CBcount}: there are $s$ fundamental {\lattstate}s, in this case
fixed points $\{\ssp_0,\ssp_1,\cdots,\ssp_{s-1}\}$, and every other
{\lattstate} is built from their concatenations and repeats. The
denominator $(1 - z)$ compensates for the single overcounted
{\lattstate}, the fixed point $\ssp_{{s}-1}=\ssp_{0}$ $(\mbox{mod}\;1)$
of \reffig{fig:BernPart} and its repeats. If the stretching factor
${s}=\beta$ is not an integer, the map \refeq{n-tuplingMap} is called the
`$\beta$-transformation'. For its Artin-Mazur zeta function see
\refref{FlLaPo94}.

\paragraph{\tempLatt.}
The number of \templatt\ period-\cl{} {\lattstate}s is given in
\refeq{1stepDiffSolu} into the Artin-Mazur zeta \refeq{AMzeta}, so
Isola\rf{Isola90} obtains
\bea
\zetatop(z)
 =  \exp \left(-\sum_{\cl{}=1}^\infty
\frac{\ExpaEig^\cl{} + \ExpaEig^{-\cl{}} - 2}{\cl{}} z^\cl{}
         \right)
 =
\frac{1 - s z + z^2}
     {(1 - z)^2}
\,.
\label{Isola90-13}
\eea
Conversely, given the \tzeta, the generating function for the number of
temporal {\lattstate}s of period $\cl{}$ is given by the logarithmic
derivative \refeq{zetatop-N},
\bea
\sum_{{n}=0}^\infty N_{n} z^{n}
    & = & \frac{2-{s}z}{1 - s z + z^2}-\frac{2}{1 - z}
    \continue
& = & (s-2)\left[z + ({s}+2) z^2 + ({s}+1)^2 z^3 \right.
    \ceq
      \left.\qquad\quad
      +\,({s}+2)\,{s}^2 z^4 + (s^2+ s-1)^2 z^5  +  \cdots\right]
\,,
\label{1stChebGenF}
\eea
which is indeed the generating function for $T_{\cl{}}(s/2)$, the
{Chebyshev polynomial of the first kind} \refeq{POsChebyshev}.
Why Chebyshev? Essentially because $T_k(x)$ are also
defined by a 3-term recurrence:
\bea
T_0(x) &=& 1\,,\quad
T_1(x)=x \,,
    \continue
T_k(x)  &=&  2x T_{k-1}(x) - T_{k-2}(x)
\quad \mbox{for } k \geq 2
\,.
\label{Cheb1stRecurr} %{Rangarajan17:TkRecurr}
\eea

\paragraph{\Henlatt.}
The problem of counting orbits for the {\HenonMap} was addressed
already in 1979
pioneering work by Sim{\'o}\rf{Simo79}.
For the complete horseshoe, ${a}=6$ {\HenonMap} repeller
there are $2^\cl{}$
period-$\cl{}$ {\lattstate}s, so
\bea
\zetatop(z)
 =  \exp \left(-\sum_{\cl{}=1}^\infty
\frac{2^\cl{}}{\cl{}} z^\cl{}
         \right)
\,=\,
1- 2z
\,.
\label{HenonZeta}
\eea
The numbers of the shortest period {\lattstate}s and prime orbits are
listed in \reftab{tab:LC21HamHenon}.

%%%%%%%%%%%%%%%%%%%%%%%%%%%%%%%%%%%%%%%%%%%%%%%%%%%%
\begin{table}
\begin{tabular}{c|rrrrr|rrrrr|rrrrr}
$\cl{}$ &  1 &  2 &  3 &  4 &  5 &
       6 &  7 &  8 &  9 & 10 &
      11 & \\%& 12 & 13 & 14 & 15 \\
\hline
$N_\cl{}$ &  2 &  4 &  8 & 16 &  32 &
       64 &  128 &  256 & 512 & 1024 &
      2048 & % \\%& 12 & 13 & 14 & 15 \\
             \rule[-1ex]{0ex}{3.5ex} \\
$M_\cl{}$ &   2 &   1 &   2 &  3 &  6 &
         9 & 18 &  30 & 56  & 99 &
       186 &  %% &     &     &
\end{tabular}
\bigskip
\caption{\label{tab:LC21HamHenon}
$N_\cl{}$ and $M_\cl{}$ are the numbers of
the period-$\cl{}$ {\lattstate}s and
orbits for the ${a}=6$ {\HenonMap}.
}
\end{table}
%%%%%%%%%%%%%%%%%%%%%%%%%%%%%%%%%%%%%%%%%%%%%%%%%%%%
%

\paragraph{{$\phi^4$} field theory.}
Essentially the same as the \henlatt, with $2\to3$ replacement in
\refeq{HenonZeta}.


\subsection{Kim-Lee-Park zeta function}
\label{sect:KiLePa}

There is a problem with the \templatt\ Artin-Mazur zeta
\refeq{Isola90-13}: it counts pairs of orbits related by time reversal as
distinct orbits, but the \templatt\ Euler–\-Lagrange equation
\refeq{catMapNewt} is manifestly time-reversal invariant, so
time-reversal pairs are physically equivalent.
If the symmetry group $\Group$ is not the maximal symmetry, Lind zeta
function erroneously counts some of the symmetry-related {\lattstate}s as
belonging to separate `prime orbits', a problem that repeatedly bedevils
the \wwwcb{} exposition of \po\ theory whenever dynamics is time-reversal
invariant.

In particular, if the symmetry group of temporal lattice of a
given dynamical system is the {infinite dihedral group} $\Dn{\infty}$,
the group of all translations and reflections\rf{KiLePa03}, \ie, the
defining law of dynamics is time and time-reversal invariant.

For the infinite translation $H_{\mathbf{a}}$ subgroup
\refeq{H(n)subgroup}
the multiplicity is (as illustrated by \reffig{fig:1dLatStatC_5})
\beq
|\Dn{\infty}/H_{\cl{}}| =  2\cl{}
\,.
\ee{HnMult}
As explained in \refsect{s:LattStateSyms}, the \Dn{\infty} orbits of
reflection-symmetric {\lattstate}s contain only translations, so the
multiplicity of each infinite dihedral subgroup $H_{\mathbf{a},k}$
\refeq{H(n,k)subgroup} is
\beq
|\Dn{\infty}/H_{\cl{},k}| = \cl{}
\,.
\ee{HnkMult}
The Lind zeta function \refeq{LindZeta} now counts separately the four
types of (a)symmetric {\lattstate}s\rf{KiLePa03} of
\refsect{s:LattStateSyms}
\beq
%\zeta_{\map,\Refl}(t) =
\zeta_{\mbox{\footnotesize KLP}}(t) =
\exp \Big(\sum_{\cl{}=1}^{\infty} \, \frac{N_{\cl{}}}{2\cl{}}t^{2\cl{}}
          +\sum_{\cl{}=1}^{\infty} \, \sum_{k=0}^{\cl{}-1}\,
                     \frac{N_{\cl{},k}}{\cl{}}t^{\cl{}} \Big)
\,.
\ee{KLPzeta}
The first sum is a square-root of the {Artin-Mazur} zeta func\-tion
\refeq{AMzeta}:
    % was {ArtMaz5} and  {Ryu17eq:2.1A}
\beq
\exp \left(\sum_{\cl{}=1}^\infty
\frac{t^{2\cl{}}}{2\cl{}} N_\cl{}
         \right)
= \sqrt{\zeta_{\mbox{\footnotesize AM}}(t^2)}
\,.
\ee{LC21Ryu17eq:2.1B}
This takes care of {\lattstate}s \refeq{reflSymNo} with no reflection
symmetry.
The number of reflection-symmetric {\lattstate}s does not depend on the
location of the reflection point $k$, only on the type of symmetry (see
the class counts \refeq{H(n,k)class} and \refeq{DnClassRefl}), so
\beq
N_{\cl{},k} =
          \left\{
            \begin{array}{ll}
\, N_{\cl{}, 0} \qquad \mbox{ if } \cl{} \mbox{ odd} \\ %[1ex]
\, N_{\cl{}, 0} \qquad \mbox{ if } \cl{} \mbox{ and } k \mbox{ are even} \\ %[1ex]
\, N_{\cl{}, 1} \qquad \mbox{ if } \cl{} \mbox{ even and } k \mbox{ is odd}
\,,
        \end{array}
           \right.
\ee{N_nkCount}
and the Lind zeta function takes the form
that we refer to as the Kim-Lee-Park\rf{KiLePa03} zeta function
\beq
\zeta_{\mbox{\footnotesize KLP}}(t)
    = \sqrt{\zeta_{\mbox{\footnotesize AM}}(t^2)} \; e^{h(t)},
\ee{KLPzetaFact}
where
\beq
h(t) = \sum_{m=1}^{\infty} \left\{
       N_{2m-1, 0}\,t^{2m-1}
       + \left(N_{2m,0}+N_{2m,1}\right)\,\frac{ t^{2m}}{2}
                               \right\}
\,.
\ee{KLPzetaExp}

\subsection{Euler product form of zeta function }
\label{sect:KiLePaEuler}

In 1966 Ihara\rf{Ihara66} defined the zeta function of an undirected
graph $\Gamma$ by analogy to Euler's product form of a zeta function,
\beq
\zeta_{\mbox{\footnotesize Ihara}}(z)_\Gamma =
        \prod_{[C]}\frac{1}{1-z^{|C|}}
\,,
\ee{Sato05Zeta}
where the product is over all equivalence classes of prime (non-self
retracing) loops $C$ of $\Gamma$, and $|C|$ denotes the length of $C$.
Ihara zeta functions%
\rf{Ihara66,Bass92,Pollicott01,Sato05,GuIsLa08,Terras10,RAEWH10,
Clair14,ZhXiHe15,Deitmar15,daCosta16,Saito18}
are ``graph-theoretic analogues of discrete Laplacians''\rf{Sunada13}
defined here in \refeq{LC21:Lap}. Even though
``undirected" refers to no preferential
time direction, they do not appear related to
the time-reversal, group-theoretic Kim-Lee-Park zeta function
\refeq{KLPzetaFact} derived above.

Still,  the idea that zeta function $\zeta_{\mbox{\footnotesize KLP}}(t)$
can be written as a product over prime {\orbit}s holds. The \statesp\
\pS\ of a $\Dn{\infty}$ invariant dynamical system is union of 2
subspaces:
\beq
\pS = \pS_{a} \cup \pS_{s}
\,,
\ee{pSunion2}
where $\pS_{a}$ is the set of pairs of asymmetric orbits \refeq{reflSymNo},
each element of the set a forward-in-time orbit and the time-reversed orbit,
and $\pS_{s}$ is the set of time reversal symmetric orbits, invariant under
reflections \refeqs{reflSymOdd}{reflSymEvens1}. Kim \etal\rf{KiLePa03}
show that the contribution of a single prime orbit $p$  to
the Kim-Lee-Park zeta function is:
\bea
1/\zeta_{\mbox{\footnotesize KLP}}(t)|_{p}=
          \left\{
            \begin{array}{ll}
\, 1 - t^{\cl{p}} \qquad\qquad\qquad\qquad\quad
                    \mbox{ if } p \in \pS_{a} \,, \\[1ex]
\, \sqrt{1 - t^{2\cl{p}}} \,\exp\left(-\frac{t^{\cl{p}}}{1-t^{\cl{p}}}
\right)     \qquad \mbox{ if } p \in \pS_{s}
\,,
        \end{array}
           \right.
\label{KLPzetap}
\eea
with the zeta function written as a product over prime {\orbit}s:
\beq
1/\zeta_{\mbox{\footnotesize KLP}}(t)=
\prod_{p_a\in\pS_{a}} (1 - t^{\cl{p_a}})
\prod_{p_s\in\pS_{s}} \sqrt{1 - t^{2\cl{p_s}}}
\exp\left(-\frac{t^{\cl{p_s}}}{1-t^{\cl{p_s}}}\right)
\,,
\ee{KLPzetaProd}
to be expanded as a power series in $t$.

To sumarize, the Euler product form of {\tzeta s} makes it explicit that
they count {\em prime orbits}, \ie, \emph{sets} of equivalent
{\lattstate}s related by symmetries of the problem. The remainder of this
section are numerical checks that this is indeed what {\tzeta s} do.

\subsection{Counting {Bernoulli} prime orbits}
\label{s:bernPrime}

Substituting the Bernoulli map \tzeta\ \refeq{BernZeta}
into \refeq{zetatop-N}
we obtain
% ------------------- siminos/mathematica/ ----------
% CatMaptopZeta.nb                                    2020-01-18
% Bernoulli map periodic points counting for CL18.tex
\bea
\sum_{n=1}N_n z^n
    &=&
 z+3 z^2+7 z^3+15 z^4+31 z^5+63 z^6+127 z^7
    \ceq
+255 z^8+511 z^9+ 1023z^{10} +2047 z^{11}
\cdots
\,,
\label{bernN_n-s=2}
\eea
in agreement with the {\lattstate}s count \refeq{LC21detBern}.
The number of \emph{prime} cycles of period $\cl{}$ is given recursively by
subtracting repeats of shorter prime cycles\rf{CBcount},
\beq
M_n\,=\,\frac{1}{n}\left( N_n - \sum _{d|n}^{d<n}\,d M_d \right)
\,,
\ee{primeCount}
where $d$'s are all divisors of $n$, hence
% ------------------- siminos/mathematica/ ----------
% CatMaptopZeta.nb                                    2020-01-18
% Bernoulli map periodic points counting for CL18.tex
\bea
\sum_{n=1}M_n z^n
    &=&
 z+  z^2+2 z^3+3 z^4+6 z^5+9 z^6+18 z^7
    \ceq
+30 z^8+56 z^9+99 z^{10} +186 z^{11}
\cdots
\,,
\label{bernM_n-s=2}
\eea
in agreement with the usual numbers of binary symbolic dynamics prime
cycles\rf{CBcount}.

\newpage % REMOVE
%%%%%%%%%%%%%%%%%%%%%%%%%%%%%%%%%%%%%%%%%%%%%%%%%%%%%%
\subsection{Counting \templatt\ {\lattstate}s}
\label{sect:LC21catCounts}    % derived from blogCats.tex {sect:PeriodicPsCount}

%%%%%%%%%%%%%%%%%%%%%%%%%%%%%%%%%%%%%%%%%%%%%%%%%%%%
\begin{table}
\begin{tabular}{c|rrrrr|rrrrr|rrrrr}
$n$ &  1 &  2 &  3 &  4 &  5 &
       6 &  7 &  8 &  9 & 10 &
      11 \\%& 12 & 13 & 14 & 15 \\
\hline
$N_n$ &   1 &   5 &  16 &  45 &  121 &
        320 & 841 & 2205 &5776 &15125&
       39601& %   &      &     & 1364
             \rule[-1ex]{0ex}{3.5ex} \\
$M_n$ &   1 &   2 &   5 &  10 &   24 &
         50 & 120 & 270 & 640 & 1500 &
       3600 &  %% &     &     &
\end{tabular}
\bigskip
\caption{\label{tab:catMapN_n-s=3}
Lattice states and {prime} orbit counts for the ${s}=3$ cat map
(Bird and Vivaldi\rf{BirViv}).
        }
\end{table}
%%%%%%%%%%%%%%%%%%%%%%%%%%%%%%%%%%%%%%%%%%%%%%%%%%%%

The number of {\lattstate}s for {\templatt} can be computed from the fundamental fact
{\HillDet} $\Det\jMorb$ \refeq{1stepDiffSolu}.
The fundamental fact also holds for the time reversal invariant {\lattstate}s.
As an example, consider {\templatt} with $s=3$. Using the
determinant of the {\jacobianOrb} \refeq{1stepDiffSolu} and symmetry
reduced {\jacobianOrbs}
\refeq{HillDetSymmReduced}, the numbers of {\lattstate}s are:
\bea
N_\cl{} &=& \left(\ExpaEig^{\cl{}/2}-\ExpaEig^{-\cl{}/2}\right)^2 \,, \continue
N_{\cl{},0} &=& \ExpaEig^{\cl{}/2}-\ExpaEig^{-\cl{}/2} \, , \qquad \qquad\;\;
\cl{} \mbox{ odd,} \continue
N_{\cl{},0} &=& \sqrt{5}\left(\ExpaEig^{\cl{}/2}-\ExpaEig^{-\cl{}/2}\right) \, , \qquad
\cl{} \mbox{ even,} \continue
N_{\cl{},1} &=& \frac{1}{\sqrt{5}}\left(\ExpaEig^{\cl{}/2}-\ExpaEig^{-\cl{}/2}\right) \, , \qquad
\cl{} \mbox{ even.}
\eea
    \PC{2021-12-30}{These do not make sense to me for \cl{} odd...
\bea
N_\cl{} &=& \left(\ExpaEig^{\cl{}/2}-\ExpaEig^{-\cl{}/2}\right)^2 \,, \continue
N_{\cl{},0} &=& \ExpaEig^{\cl{}/2}-\ExpaEig^{-\cl{}/2}
\eea
but I do remember all $\ExpaEig^{1/2}$ eventually going away... Never mind.
    }
Using \refeq{KLPzetaExp} we have:
\bea
h(t) &=& \sum_{m=1}^{\infty} \left[
       N_{2m-1, 0}\,t^{2m-1}
       + \left(N_{2m,0}+N_{2m,1}\right)\,\frac{ t^{2m}}{2}
                               \right] \continue
&=&
\frac{\ExpaEig^{1/2} t}{1-\ExpaEig t^2}
-\frac{\ExpaEig^{-1/2}t}{1-\ExpaEig^{-1}t^2}
\label{HLsymmCatZetaExp}
+
\sqrt{\frac{9}{5}}\frac{\ExpaEig t^2}{1- \ExpaEig t^2}
-\sqrt{\frac{9}{5}}\frac{\ExpaEig^{-1} t^2}{1- \ExpaEig^{-1} t^2} \,.
\eea
Using the definition of the Kim-Lee-Park zeta function \refeqs{KLPzeta}{KLPzetaExp},
generating functions of the {\lattstate} counts are
\bea
t \frac{\partial}{\partial t}\ln\zeta_{\mbox{\footnotesize AM}}(t^2)
&=& \sum_{\cl{}=1}^{\infty} N_{\cl{}} t^{2\cl{}} \continue
&=& t^2 + 5 t^4 + 16 t^6 + 45 t^8 + 121 t^{10} + 320 t^{12} + 841 t^{14} \continue
&&+ 2205 t^{16} + 5776 t^{18} + 15125 t^{20} + \dots
\,,
\eea
and
\bea
h(t) &=& \sum_{m=1}^{\infty} \left[
       N_{2m-1, 0}\,t^{2m-1}
       + \frac{\left(N_{2m,0}+N_{2m,1}\right)}{2}\, t^{2m}
                               \right] \continue
&=&
t + 3 t^2 + 4 t^3 + 9 t^4 + 11 t^5 + 24 t^6 +  \continue
&&29 t^7 + 63 t^8 + 76 t^9 +165 t^{10}
+ \dots
\,,
\eea
in agreement with \reftab{tab:lattstateCountCat}.

%%%%%%%%%%%%%%%%%%%%%%%%%%%%%%%%%%%%%%%%%%%%%%%%%%%%
\begin{table}
\begin{center}
\begin{tabular}{c|rrrrr|rrrrr|rrrrr}
$n$ &  1 &  2 &  3 &  4 &  5 &
       6 &  7 &  8 &  9 & 10 \\
\hline
$N_\cl{}$ &   1 &   5 &  16 &  45 &  121 &
        320 & 841 & 2205 &5776 &15125 \\
$N_{\cl{},0}$ &   1 &   5 &   4 &  15 &  11 &
         40 & 29 & 105 & 76 & 275 \\
$N_{\cl{},1}$ &   1 &   1 &   4 &  3 &  11 &
         8 & 29 & 21 & 76 & 55
\end{tabular}
\bigskip
\caption{\label{tab:lattstateCountCat}
Numbers of period-$\cl{}$ {\lattstate}s of $s=3$ {\templatt}.
$N_{\cl{}}$, $N_{\cl{},0}$ and $N_{\cl{},1}$ are numbers of {\lattstate}s
that are invariant under group actions of $H_{\cl{}}$, $H_{\cl{},0}$
and $H_{\cl{},1}$ respectively.
    }
\end{center}
\end{table}
%%%%%%%%%%%%%%%%%%%%%%%%%%%%%%%%%%%%%%%%%%%%%%%%%%%%

\bigskip\bigskip
========================================================================
We have found
MacKay\rf{Bmack93} %1982 PhD thesis,
% {\em Renormalisation in Area-preserving Maps} has a
%chapter on reversible maps.
listed the periodic {\lattstate}s and orbits counts, together with
the counts of time reversal invariant {\lattstate}s and orbits.
    \PC{2021-12-24}{
Do cite in our paper(s).
MacKay had these numbers already
listed in Table 1.2.3.5.1 of his 1982 PhD thesis\rf{Bmack93}.
    }
