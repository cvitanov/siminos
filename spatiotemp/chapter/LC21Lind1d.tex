% siminos/reversal/Lind1d.tex      pdflatex LC21; bibtex LC21
% temporary:  siminos/spatiotemp/chapter/LC21Lind1d.tex
% $Author: predrag $ $Date: 2021-12-20 23:24:25 -0500 (Mon, 20 Dec 2021) $

% Predrag 2021-08-08: shared with siminos/reversal/LC21.tex

%%%%%%%%%%%%%%%%%%%%%%%%%%%%%%%%%%%%%%%%%%%%%%%%%%%%%%%%%%%%%%%%%%%%%%%%%%
\section{Topological $\Dn{\infty}$-zeta function}
\label{sect:LC21Lind1d}    % derived from blogCats.tex {sect:dihedralZeta}


Let $\Group$ be a group, $\pS$ a set and $\map: \Group \times \pS \to
\pS$ a $\Group$-action on $\pS$. The Lind zeta function\rf{Lind96} is
defined as
\beq
\zeta_{Lind}(t) =
\exp \left( \sum_{H} \;
            \frac{N_{H}}{|\Group/{H}|}t^{|\Group/H|}
      \right)
\,,
\ee{LindZeta} % was {LindZeta}
where the sum is over all subgroups $H$ of $\Group$
of multiplicity $|G/H| < \infty$, and $N_{H}$ is the number of states
in $\pS$ invariant under action of (fixed points of) subgroup $H$,
\beq
N_{H} =
   |\{\Xx \in \pS : \mbox{ all } h \in H \quad \map(h,\Xx) = \Xx\}|
\,.
\ee{LindN_H} % was {LC21Ryu17eq:1.3A}
The multiplicity $|\Group/{H}|$ is best explained by working out a few examples.

In the 1\dmn\ lattice field theory setting, $\pS$ is the set of
all {\lattstate}s. The group $\Group$ is either the
{infinite cyclic group}  \Cn{\infty} of all lattice translations \refeq{C_infty},
or the
{infinite dihedral group} $\Dn{\infty}$  of all translations and
reflections \refeq{LC21D_infty}. Their finite-multiplicity subgroups $H$ are,
respectively, infinite translation subgroups $H_{\mathbf{a}}$
\refeq{H(n)subgroup}, or $H_{\mathbf{a}}$ and infinite dihedral subgroups
$H_{\mathbf{a},k}$ \refeq{H(n,k)subgroup}.

Assume that the symmetry group of a given discrete time dynamical system
is the group of temporal lattice translations \Cn{\infty}, \ie, that the
defining law of dynamics is time invariant. 1\dmn\ infinite translation
subgroup $H_{\mathbf{a}}$ \refeq{H(n)subgroup} leaves {Bravais cell} of
{period} \cl{} \refeq{1DBravLatt} invariant, so the sum over $H$ in
\refeq{LindZeta} can replaced by the sum over periods $\cl{}$.
$N_\cl{}$ is the number of {\lattstate}s of period $\cl{}$. For \Cn{\infty} subgroups --see
\reffig{fig:1dLatStatC_5}\,(1)-($\shift_4$)-- the multiplicity is
\beq
|\Cn{\infty}/H_\cl{}|=\cl{}
\,,
\ee{CinfHnMult}
and the Lind zeta function
\refeq{LindZeta} reduces to the familiar Artin-Mazur zeta
func\-tion \refeq{topoZeta}.

Consider next the case that the symmetry group of temporal lattice of a
given dynamical system is the {infinite dihedral group}
$\Dn{\infty}$, the group of all translations and reflections\rf{KiLePa03}, \ie,
the defining law of dynamics is time and time-reversal invariant.

For the infinite translation $H_{\mathbf{a}}$ subgroup
\refeq{H(n)subgroup}
the multiplicity is --as illustrated by \reffig{fig:1dLatStatC_5}--
\beq
|\Dn{\infty}/H_{\cl{}}| =  2\cl{}
\,,
\ee{HnMult}
and, as explained in \refsect{s:LattStateSyms}, the \Dn{\infty} orbits of
reflection-symmetric {\lattstate}s contain only translations, so the
multiplicity of each infinite dihedral subgroup $H_{\mathbf{a},k}$
\refeq{H(n,k)subgroup} is
\beq
|\Dn{\infty}/H_{\cl{},k}| = \cl{}
\,.
\ee{HnkMult}
The Lind zeta function
\refeq{LindZeta}
is now of form\rf{KiLePa03}
\beq
%\zeta_{\map,\Refl}(t) =
\zeta_{\Refl}(t) =
\exp \Big(\sum_{\cl{}=1}^{\infty} \, \frac{N_{\cl{}}}{2\cl{}}t^{2\cl{}}
          +\sum_{\cl{}=1}^{\infty} \, \sum_{k=0}^{\cl{}-1}\,
                     \frac{N_{\cl{},k}}{\cl{}}t^{\cl{}} \Big)
\,.
\ee{LC21Ryu17eq:3.0B}
The first sum yields a square-root of the {Artin-Mazur} zeta func\-tion
\refeq{topoZeta}:
    % was {ArtMaz5} and  {Ryu17eq:2.1A}
\beq
\exp \left(\sum_{\cl{}=1}^\infty
\frac{t^{2\cl{}}}{2\cl{}} N_\cl{}
         \right)
= \sqrt{\zeta_{top}(t^2)}
\ee{LC21Ryu17eq:2.1B}
This takes care of {\lattstate}s \refeq{reflSymNo} with no reflection
symmetry.
The number of reflection-symmetric lattice states does not depend on $k$,
only on the type of symmetry, so

\bigskip\bigskip
================== stopped editing here ========================

The class count \refeq{H(n,k)class}, \refeq{DnClassRefl} tells us that
\beq
N_{\cl{},k} =
    \begin{cases}
\, N_{\cl{}, 0} \qquad \mbox{ if } \cl{} \mbox{ is odd}, \\
\, N_{\cl{}, 0} \qquad \mbox{ if } \cl{} \mbox{ and } k \mbox{ are even}, \\
\, N_{\cl{}, 1} \qquad \mbox{ if } \cl{} \mbox{ is even and } k \mbox{ is odd}
\,.
    \end{cases}
\ee{LC21Ryu17eq:3.0D}
Hence
\beq
\sum_{k=0}^{\cl{}-1} \, \frac{N_{\cl{},k}}{\cl{}}
=   \begin{cases}
\, N_{\cl{},0} \qquad \qquad \qquad \qquad
                            \mbox{if } \cl{} \mbox{ is odd}
\,,\\ \\ \,
\displaystyle{\frac{N_{\cl{},0}+N_{\cl{},1}}{2} \qquad
                            \mbox{if } \cl{} \mbox{ is even}\,.}
    \end{cases}
\ee{LC21Ryu17eq:3.0E}
so the Lind zeta function teks the form
that we shall refer to as the Kim-Lee-Park\rf{KiLePa03} zeta function
\beq
\zeta_{\Refl}(t) = \sqrt{\zeta_{top}(t^2)} \; e^{h(t)},
\ee{LC21Ryu17eq:2.1}
where
\beq
h(t) = \sum_{m=1}^{\infty} \left\{
       N_{2m-1, 0}\,t^{2m-1}
       + \left(N_{2m,0}+N_{2m,1}\right)\,\frac{ t^{2m}}{2}
                               \right\}
\,.
\ee{LC21Ryu17eq:2.11}

The zeta function $\zeta_{\Refl}$ can be written as a product over
{\orbit}s. Let $O_1$ be the collection of finite orbits with time
reversal (flip) symmetry, and $O_2$ be the collection of the pairs of
orbits without time reversal symmetry, each an orbit and the flipped
orbit. A finite \orbit\ $p$ is a periodic points set
\[
p = \{\ssp, \map(\ssp), \dots, \map^{\cl{p}-1}(\ssp)\}
\]
if $p \in O_1$, and
\[
p = \{\ssp, \map(\ssp), \dots, \map^{k-1}(\ssp)\} \cup
\{\Refl(\ssp), \map\circ\Refl(\ssp), \dots, \map^{k-1}\circ\Refl(\ssp)\}
\]
if $p \in O_2$, where $k=\cl{p}/2$.

If $p \in O_1$,
\beq
\zeta_{p}(t) =
\sqrt{\frac{1}{1-t^{2\cl{p}}}}\exp\left(\frac{t^{\cl{p}}}{1-t^{\cl{p}}}\right)
\,,
\ee{ZetaO1}
and if $p \in O_2$,
\beq
\zeta_{p}(t) =
\frac{1}{1-t^{\cl{p}}}
\,.
\ee{ZetaO2}
The product form of the zeta function is:
\beq
1/\zeta_{\Refl}(t) =
\sqrt{\prod_{p_1\in {O_1}}(1-t^{2\cl{p_1}})}
      \;\exp\left(-\frac{t^{\cl{p_1}}}{1-t^{\cl{p_1}}}\right)
\prod_{p_2\in O_2} (1-t^{\cl{p_2}})
\,.
\ee{ZetaProd}

\newpage % REMOVE
%%%%%%%%%%%%%%%%%%%%%%%%%%%%%%%%%%%%%%%%%%%%%%%%%%%%%%
\subsection{Counting \templatt\ {\lattstate}s}
\label{sect:LC21catCounts}    % derived from blogCats.tex {sect:PeriodicPsCount}

\refeq{AABHM99-46b}, \refeq{catlattMass} introduce ${\mu}$.
See also \refeq{3diagCircEigs}, \refeq{3diagCircEigs1},
\refeq{HL:detTemCatCheb}, \refeq{GraRyz1.395.2}, \refeq{HLhalfHalf},


For $s>2$ the stability multipliers
\(
(\ExpaEig^{+},\ExpaEig^{-})
\,=\,(\ExpaEig\,,\; \ExpaEig^{-1})
\)
are real,
\beq
\ExpaEig^{\pm}=\frac{1}{2}(s\pm \surd{D})
\,,\qquad
\ExpaEig=e^{\Lyap}
\,,
\ee{LC21catEigs}
where
\bea
s&=&\ExpaEig+\ExpaEig^{-1}
  =  2\cosh(\Lyap)
    \continue
s-2 &=&{\mu}^2
    \continue
\surd{D}&=&\ExpaEig-\ExpaEig^{-1}
  =  2\sinh(\Lyap)
    \continue
\mbox{discriminant }
{D}  &=& {s}^{2}-4
      =  {\mu}^2({\mu}^2+4)
\label{LC21catEigs1}
\eea

The sine, sinh, cos, cosh are related by the identities
\refeq{GraRyzSect1.30sin} and \refeq{GraRyzSect1.30cos}.
%    \item[2021-03-10 Predrag]
For \Dn{\cl{}} irreps, I think we should use the \refeq{tildejMorbDisg} form
of eigenvalues, and Klein-Gordon mass $\mu$
\bea
\lambda_m &=& {\mu}^2+ 4 \sin^2\left(\alpha_m/2\right)
\continue
   &=& \left({\mu} - i\,2\sin\left(\frac{\alpha_m}{2}\right)\right)
   \,  \left({\mu} + i\,2\sin\left(\frac{\alpha_m}{2}\right)\right)
\continue
%\,,\quad
\alpha_m &=& 2\pi{m}/{\cl{}}
\,,
\label{LC21tildejMorbDisg1}
\eea
rather than \refeq{HLEigenvalueD62} and stretching $s$, which is
appropriate to \Cn{\cl{}} irreps. Identities like \refeq{IvIzHu02:GraRyz},
\refeq{IvIzHu02:HL},
\refeq{Zab00} and {Gradshteyn and Ryzhik}\rf{GraRyz} Eq.~1.317.1
\CBlibrary{GraRyz}
\beq
2\sin^2\left(\theta/2\right)= 1-\cos\left(\theta\right)
\ee{LC21GraRyz1.317.1}
are also suggestive in this context.

%    \item[2021-08-26 Han]
How to count the number of {\lattstate}s for \templatt?

\emph{No symmetry} {\lattstate}s \HillDet:
\[
N_\cl{} = \prod_{j=0}^{\cl{}-1} \left( s - 2\cos\frac{2\pi j}{\cl{}}\right)
\,.
\]
The products of eigenvalues for the $\Cn{\cl{}}$ discrete Fourier
case follows from \refeq{TableOfISP13952-2}:
\beq
\prod_{j=0}^{\cl{}-1} \left( s - 2\cos\frac{2\pi j}{\cl{}}\right)
= (\ExpaEig^{\cl{}/2}-\ExpaEig^{-\cl{}/2})^2
\,,
\ee{eigsProduct}
It's a square, because of the  $\Dn{\cl{}}$ symmetry.
Consider even, odd casses, use $\cos0=1$, $\cos\pi=-1$,
$\cos(-\theta)=\cos\theta$. The product over non-trivial eigenvalues is:
\bea
\cl{}=2m
     &&
M_{\cl{},0} =
 \prod_{j=1}^{m-1}\left({s}-2\cos\frac{\pi{j}}{m}\right)
      =  \frac{|\ExpaEig^{\cl{}/2}-\ExpaEig^{-\cl{}/2}|}
              {{\mu}\sqrt{{\mu}^2+4}}
\,,
\label{LC21:eigsProdEven}
\eea

\bea
\cl{}= 2m-1
     &&
M_{\cl{},1} =
 \prod_{j=1}^{m-1}\left({s}-2\cos\frac{2j\pi}{2m-1}\right)
     = \frac{|\ExpaEig^{\cl{}/2}-\ExpaEig^{-\cl{}/2}|}
              {{\mu}}
\,,
\label{LC21:eigsProdOdd}
\eea

Next, look at the \emph{symmetric} {\lattstate}s {\HillDet}s:

For odd $\cl{}=2m-1$,
\bea
N_{\cl{},1} = \prod_{j=0}^{m-1} \left(s-2\cos\frac{2\pi j}{\cl{}}\right)
={\mu}M_{\cl{},1}
\,.
\eea
For $\cl{}=2m$,
\bea
N_{\cl{},1} &=& \prod_{j=0}^{m-1} \left(s-2\cos\frac{2\pi j}{\cl{}}\right)
            \continue
N_{\cl{},0} &=& % \prod_{j=0}^{m} \left(s-2\cos\frac{2\pi j}{\cl{}}\right) =
                (s+2)\,N_{\cl{},1}
\,,
\eea
and
\bea
\frac{1}{2}\left(N_{\cl{},0}+
N_{\cl{},1} \right)
= \frac{\mu^2+5}{2}\prod_{j=0}^{m-1} \left(s-2\cos\frac{2\pi j}{\cl{}}\right)
= \frac{\mu^2+5}{2\mu}
\sqrt{\frac{\left(\ExpaEig^{\cl{}} + \ExpaEig^{-\cl{}} - 2\right)}
           {\mu^2+4}}
\,.
\eea
The number of lattice states can be written as polynomials:
For $\cl{}=2m-1$:
\bea
N_{\cl{},0} &=&
\mu\left(\ExpaEig^{\cl{}/2}-\ExpaEig^{-\cl{}/2}\right)
\continue
&=&
\mu^2\ExpaEig^{-1/2}\left(\ExpaEig^{m}-\ExpaEig^{-m+1}\right)
\,.
\eea
For $\cl{}=2m$:
\bea
\frac{1}{2}\left(N_{\cl{},0}+
N_{\cl{},1} \right)&=&
\frac{s+3}{2(\ExpaEig-\ExpaEig^{-1})}
\left(\ExpaEig^{\cl{}/2}-\ExpaEig^{-\cl{}/2}\right)
\continue
&=&
\frac{{\mu}^2+5}{2{\mu}\sqrt{{\mu}^2+4}}
\left|\ExpaEig^{m}-\ExpaEig^{-m}\right|
\,.
\eea
Now we can compute the $h(t)$ from \refeq{LC21Ryu17eq:2.11}
\bea
h(t) &=& \sum_{m=1}^{\infty} \left[
       N_{2m-1, 0}\,t^{2m-1}
       + \left(N_{2m,0}+N_{2m,1}\right)\,\frac{ t^{2m}}{2}
                               \right]
\continue
&=&
\mu\frac{\ExpaEig^{1/2} t}{1-\ExpaEig t^2}
-\mu\frac{\ExpaEig^{-1/2}t}{1-\ExpaEig^{-1}t^2}
\continue
&&+
\frac{\mu^2+5}{2(\ExpaEig-\ExpaEig^{-1})}\frac{\ExpaEig t^2}{1- \ExpaEig t^2}
-\frac{\mu^2+5}{2(\ExpaEig-\ExpaEig^{-1})}\frac{\ExpaEig^{-1} t^2}{1- \ExpaEig^{-1} t^2}
\,.
\eea
Using \refeq{LC21Ryu17eq:2.1} we have the "flip" part of the zeta function. Testing
this zeta function using \refeq{HLFlipGeneratingFunction}, we have:
\bea
- t \frac{\partial}{\partial t}(\ln e^{-h(t)}) &=&
t + 6 t^2 + 12 t^3 + 36 t^4 + 55 t^5 +144 t^6
\continue
&&+ 203 t^7 + 504 t^8 + 684 t^9
+1650 t^{10} + \dots
\,,
\eea
which is in agreement with \refeq{HLFlipGeneratingFunction}
and \reftab{tab:Bmack93Fixed}.

%%%%%%%%%%%%%%%%%%%%%%%%%%%%%%%%%%%%%%%%%%%%%%%%%%%%%%
\subsection{Counting {\lattstate}s}
\label{sect:LC21poCounts}    % derived from blogCats.tex {sect:PeriodicPsCount}


Given the {\tzeta} \refeq{LC21Ryu17eq:3.0B} we can count the
number of lattice states from the generating function:
    \PC{2021-08-25}{
    We have the counts of the Bravais lattice states $N_\cl{}$,
    $N_{\cl{},k}$ already, from \refeq{LC21Ryu17eq:3.0D}, so why don't we
    reverse the logic, start here, and get the zeta function
    \refeq{LC21Ryu17eq:2.1} by integration? Mention that this is an
    example of Lind zeta function\rf{Lind96} \refeq{LindZeta}
    without ever writing it down, so we do not have to explain it? It's a
    side issue for us, really.
    }
\beq
\frac{-t\frac{d}{dt}(1/\zeta_{\Refl}(t))}{1/\zeta_{\Refl}(t)}
= \sum_{\cl{}=1}^\infty N_\cl{}t^{2\cl{}}
+ \sum_{\cl{}=1}^\infty\sum_{k=0}^{\cl{}-1}N_{\cl{},k}t^{\cl{}}
= \sum_{m=1}^\infty a_m t^m \, ,
\ee{LC21zetatopGenerating}
where the coefficients are:
% https://tex.stackexchange.com/questions/401201/difference-between-align-and-alignedt
\beq
a_m =
\left\{
\begin{array}{ll}
\sum_{k=0}^{m-1} N_{m,k}^{\Refl}
= m N_{m,0}^{\Refl}
\,,\quad & \mbox{$m$ is odd}
 \, ,\\
N_{m/2} + \sum_{k=0}^{m-1} N_{m,k}^{\Refl}
= N_{m/2} + \frac{m}{2} \left(N_{m,0}^{\Refl} + N_{m,1}^{\Refl}\right)
\,,\quad & \mbox{$m$ is even}
\,.
 \end{array}\right.
\ee{LC21zetatopCoefficients}
Using the product formula of {\tzeta} \refeq{ZetaProd} and
the numbers of orbits with length up to 5 from the \reftab{tab:Bmack93Fixed},
we can write the {\tzeta}:
\bea
1/\zeta_{\Refl}(t) &=&
\sqrt{1-t^2} \exp\left(-\frac{t}{1-t}\right) (1-t^4) \exp\left(-\frac{2t^2}{1-t^2}\right)
\left(\sqrt{1-t^6}\right)^3 \continue
&& \exp\left(-\frac{3t^3}{1-t^3}\right)(1-t^6)(1-t^8)^3
\exp\left(-\frac{6t^4}{1-t^4}\right) \continue
&& (1-t^8)^2(1-t^{10})^5\exp\left(-\frac{10t^5}{1-t^5}\right)
(1-t^{10})^6 \dots \, .
\eea
The generating function is:
\bea
\frac{-t\frac{d}{dt}(1/\zeta_{\Refl})}{1/\zeta_{\Refl}}
=
t + 7t^2 + 12t^3 + 41t^4 + 55t^5 + \dots \, ,
\eea
which is in agreement with \refeq{LC21zetatopCoefficients}, where the $N_\cl{}$ and $N_{\cl{}}^{\Refl}$
are the $C_\cl{}$ and $SF_\cl{}$ in the \reftab{tab:Bmack93Fixed}.

We are not able to retrieve the numbers of fixed points by their symmetry groups using this {\tzeta} \refeq{LC21Ryu17eq:3.0B}, unless we rewrite the {\tzeta} with two variables:
\beq
\zeta_{\Refl}(t,u) =
\exp \Big(\sum_{\cl{}=1}^{\infty} \, \frac{N_{\cl{}}}{2\cl{}}t^{2\cl{}}
          +\sum_{\cl{}=1}^{\infty} \, \sum_{k=0}^{\cl{}-1}\,
                     \frac{N_{\cl{},k}}{\cl{}}u^{\cl{}} \Big)
\,.
\ee{LC21FlipZetaTU}
Using this {\tzeta} $\zeta_{\Refl}(t,u)$ we can write two generating functions:
\beq
\frac{-t\frac{\partial}{\partial t}(1/\zeta_{\Refl}(t,u))}{1/\zeta_{\Refl}(t,u)}
= \sum_{\cl{}=1}^\infty N_\cl{}t^{2\cl{}}
\, ,
\ee{LC21zetatopGeneratingT}
and
\beq
\frac{-u\frac{\partial}{\partial u}(1/\zeta_{\Refl}(t,u))}{1/\zeta_{\Refl}(t,u)}
= \sum_{\cl{}=1}^\infty\sum_{k=0}^{\cl{}-1}N_{\cl{},k}u^{\cl{}}
\, .
\ee{LC21zetatopGeneratingU}
Using the product formula of this {\tzeta} and the numbers of orbits with length
up to 5 from the \reftab{tab:Bmack93Fixed}, the {\tzeta} is:
\bea
1/\zeta_{\Refl}(t,u) &=&
\sqrt{1-t^2} \exp\left(-\frac{u}{1-u}\right) (1-t^4) \exp\left(-\frac{2u^2}{1-u^2}\right)
\left(\sqrt{1-t^6}\right)^3 \continue
&& \exp\left(-\frac{3u^3}{1-u^3}\right)(1-t^6)(1-t^8)^3
\exp\left(-\frac{6u^4}{1-u^4}\right) \continue
&& (1-t^8)^2(1-t^{10})^5\exp\left(-\frac{10u^5}{1-u^5}\right)
(1-t^{10})^6 \dots \, .
\eea
And the generating function from this {\tzeta} is:
\bea
\frac{-u\frac{\partial}{\partial u}(1/\zeta_{\Refl}(t,u))}{1/\zeta_{\Refl}(t,u)}
=
u + 6u^2 + 12u^3 + 36u^4 + 55u^5 + \dots \, ,
\label{HLFlipGeneratingFunction}
\eea
which is in agreement with \refeq{LC21zetatopGeneratingU}, where the $N_{\cl{}}^{\Refl}$ is the $SF_{\cl{}}$
in the \reftab{tab:Bmack93Fixed}.

\bigskip\bigskip
========================================================================

We use the temporal 1D lattice for
systems with time-reversal symmetry to explain how such zeta functions
are constructed.

A {\tzeta} is the generating function of all distinct orbits \refeq{GroupOrbDisc}
for a given system.

We shall refer to $\Group$-orbit \refeq{GroupOrbDisc} counting generating
function for a symmetry group $\Group$ as the topological $\Group$-zeta
function, or the Lind zeta function\rf{Lind96}. The classic example is
\refeq{topoZeta}, the {\em Artin-Mazur} zeta
func\-tion\rf{ArtMaz65,CBcount} for the \emph{infinite cyclic group}
\Cn{\infty} of all integer time translations. Here we focus on the
\emph{infinite dihedral group} topological $\Dn{\infty}$-zeta function
(Kim-Lee-Park\rf{KiLePa03} flip-zeta function).

The \emph{infinite dihedral group} $\Dn{\infty}$ \refeq{D_infty}




the {\em trace formula for maps}:
\index{trace!formula!maps}
%  This formula yields the spectrum of $\Lop$ as
% the poles of $\tr z\Lop /( 1 - z\Lop)$.
% The relation
\PC{} {box this relation}
\index{generating!function}
\beq
%\sum_{\alpha=0}^\infty {z e^{\eigenvL_\alpha} \over 1 - z e^{\eigenvL_\alpha} }
%   =
    \sum_p \cl{p} \sum_{r=1}^\infty
    \frac{ z^{\cl{p} r}} %\, e^{r \beta \Obser_p}
    {\oneMinJ{r}}
\,.
\ee{tr-L-s}

the trace formula \refeq{tr-L-s} can be recovered
from the \Fd\ by taking a derivative
\beq
    %\tr {z\Lop \over 1- z\Lop} =
    -z \frac{d~}{dz}
          \ln \det(1 - z\Lop)
    \,.
\ee{der-det-map}




For every integer temporal period
$\cl{}$, we first determine $N_\cl{}$, the number of all periodic
\emph{{\lattstate}} ${\Xx}_{\Mm}$ solutions on a tile of length
$\cl{}$.


Due to
the time invariance of the defining equations, there are $\cl{p}$
physically equivalent copies of a given solution in the orbit of
every $\Xx_p$. So all we really have to do is to enumerate $M_\cl{}$
{\em orbits} of the time-invariance equivalent \po\ solutions,
whose generating function is the {\tzeta}.

\bigskip\bigskip

That {\HillDet} factors as
\(
\Det\jMorb  = \Det\jMorb_{-}\,\Det\jMorb_{+}
\)
for all symmetric states must propagate into the factorization of the
{\dzeta}, mirroring the Kim \etal\rf{KiLePa03} topological (counting)
Lind eta function \refeq{Ryu17eq:2.1}.

multiplicity $m_\Xx$ of orbit $\pS_\Xx$ \refeq{GroupOrbMult}
