% siminos/reversal/LC21intro.tex      pdflatex LC21; bibtex LC21
% temporary: siminos/chapter/LC21intro.tex  pdflatex blogCats; biber blogCats
% $Author: predrag $ $Date: 2021-12-24 01:25:20 -0500 (Fri, 24 Dec 2021) $

\begin{quote}
     Dedicated to Fritz Haake 1941--2019\rf{GGSZ21}
\end{quote}



The year was 1988.
Roberto Artuso, Erik Aurell and P.C. had just worked out the cycle
expansions formulation of the deterministic and semiclassical chaotic
systems\rf{AACI,AACII}, and a Niels Bohr Institute ``Quantum Chaos
Symposium'' was organized to introduce the new-fangled theory to
unbelievers (for a history, see \toChaosBook{section.A.4} {Append.~A.4
{\em Periodic orbit theory}}).
In the first row of the famed auditorium, where long ago Niels Bohr and
his colleagues used to nod off, sat a man with an impressive butterfly
and a big smile. At the end of our presentation, Fritz --for that was
Fritz Haake-- stood up and exclaimed

\begin{quote}
     % Nordita Quantum Chaos Symposium, 25 Nov 1988
     ``Amazing! I did not understand a single word!''
\end{quote}

And indeed, there is a problem of understanding what is `chaos' as
encountered in different disciplines, so we start this offering to Fritz
Haake's memory by {`a fair coin toss'} theory of chaos,
\refsect{s:coinToss}, as was presented in the 1988 symposium, but in a
modern, field theorist's vision. In those days `chaos' was a phenomenon
exhibited by low-dimensional systems. In this and companion papers%
\rf{CL18,CL18,GuBuCv17} we explain how to think of `chaotic' or
`turbulent' $\infty$\dmn\ deterministic field theories. Deterministic
chaotic field theory is of interest on its merits, as a method of
describing turbulence in strongly nonlinear deterministic field theories,
such as \NS\ or \KS\rf{GudorfThesis,GuBuCv17}, or as a Gutzwiller WKB
`skeleton' for a chaotic quantum field theory\rf{gutbook,CFTsketch}
or a stochastic field theory%
\rf{noisy_Fred,conjug_Fred,diag_Fred,LipCvi08,CviLip12}.
The lattice reformulation aligns `chaos' with standard solid state, field
theory and statistical mechanics, but the claims are radical: we've been
doing `turbulence' all wrong.
In ``explaining'' chaos we talk the talk as though we have never moved
beyond Newton; here is an initial state of a system, at an instant in
time, and here are the differential equations that evolve it forward in
time. But when we -all of us- walk the walk, we do something altogether
different (see the references preceding eq.~\refeq{LC21nXdCycle}), much
closer to the 20th century spacetime physics.
Our papers realign the theory to what we
actually {\em do} when solving `chaos equations', using not much more
than linear algebra.
In the field-theoretical formulation, there is no evolution in time, and
there is no `Lyapunov horizon'; every contributing {\em \lattstate} is a
robust global solution of a \spt\ fixed point condition, there is no
dynamicist's exponential blowup of initial state perturbations.

To a field theorist, the fundamental object is \emph{global}, the
partition function sum over probabilities of all possible spacetime field
configurations.
To a dynamicist, the fundamental notion is \emph{local}, an ordinary or
partial differential time-evolution equation. From the field-theoretic
perspective, the spacetime formulation is fundamental, elegant and
computationally powerful, while moving in step-lock with time is only one
of the methods, a `transfer matrix' method for construction of the
partition sum, at times awkward and computationally unstable.

We start our introduction to chaotic field theory (\refsect{s:LC21FT}) by
rewriting the two most elementary examples of deterministic chaos,
the for\-ward-in-time first order difference equation for the Bernoulli
map (\refsect{s:coinToss}), and
the for\-ward-in-time second order difference equation for a 1\dmn\
lattice of coupled rotors (\refsect{s:kickRot}) as, respectively,
the `{temporal Bernoulli}' two-term discrete lattice recurrence relation,
% (\refsect{s:1D1dLatt}),
and
the `\templatt'  three-term discrete lattice recurrence relation.
% (\refsect{s:catLagrange}).
We then apply the approach to the simplest
nonlinear field theories, the 1\dmn\ discretized
scalar {$\phi^3$} and {$\phi^4$} theories
(\refsects{s:henlatt}{s:phi4latt}).

Their spacetime generalization, the simplest of all chaotic field theories,
is the `\catlatt'\rf{GutOsi15,GHJSC16,CL18}, a discretization of the
Klein-Gordon equation, a deterministic scalar field theory on a $d$\dmn\
hyper-cubic lattice, with an unstable ``anti-harmonic" rotor $\ssp_{z}$
at each lattice site $z$, a rotor that gives, rather than pushes back,
coupled to its nearest neighbors.
In contrast to its elliptic sister, the Helmholtz equation and its
oscillatory solutions, {\catlatt}'s {\lattstate}s are hyperbolic and
`turbulent', just as in contrast to oscillations of a harmonic
oscillator, Bernoulli coin flips are unstable and chaotic.

The key to constructing partition sums for deterministic field theories
(\refsect{s:LC21FT}) are {\HillDet}s, determinants of  the
`{\jacobianOrbs}' (\refsect{s:JacobianOrb}) that describe the global stability of the
linearized deterministic equations.
How is this global, high-dimensional orbit stability related
to the stability of the conventional low-dimensional, forward-in-time evolution?
The two notions of stability are related by Hill's
formulas (\refsect{s:LC21Hill}), relations that rely on higher-order
derivative equations being rewritten as sets of first order ODEs,
relations equally applicable to the mechanical, energy conserving
systems, as to the viscous, dissipative systems.

The lattice discretization of the theory enables us to apply the
solid state computational methods, such as the reciprocal lattice and
space group crystallography, to what are conventionally dynamical
system problems (\refsect{sect:LC21recip1d}). In order to explain the
key ideas,
we focus in this paper on 1\dmn\ field theories, postponing the
2\dmn\ Bravias lattices' subtleties to the sequel\rf{CL18}.
In \refsect{s:latt1d} we show that the partition function of a
given field theory is determined by the group of its symmetries, \ie,
by the space group of its lattice discretization,
and its reciprocal lattice.
On the level of counting lattice states, their {\tzeta}s are purely
group-theoretic Lind zeta functions, see \refsect{sect:LC21Lind1d}.
As long as the symmetry is only the time translation, we recover the
conventional \po\ theory\rf{ChaosBook}
(\refsect{s:PoThe}).
However, from a \spt\ field theory perspective, `time'-reversal is a
purely crystallographic notion, leading to --to us very surprising--
dihedral space group Kim-Lee-Park\rf{KiLePa03} zeta function
\refeq{LC21Ryu17eq:2.1} for the  `time-reversible' theories.

Our results are summarized and open problems discussed in the
\refsect{s:summary}.
