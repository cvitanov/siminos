\svnkwsave{$RepoFile: siminos/spatiotemp/chapter/blogRJ.tex $}
\svnidlong {$HeadURL: svn://zero.physics.gatech.edu/siminos/spatiotemp/chapter/blogRJ.tex $}
{$LastChangedDate: 2019-05-09 18:30:52 -0400 (Thu, 09 May 2019) $}
{$LastChangedRevision: 6420 $} {$LastChangedBy: predrag $}
\svnid{$Id: blogRJ.tex 6420 2018-05-01 11:56:57Z predrag $}

\chapter{Rana's blog}
\label{chap:blogRJ}

\bigskip

\hfill   {\large Rana Jafari <jafari.rana@gmail.com> work blog}

   % *********************************************************************
\hfill   {\color{red} The latest entry at the bottom for this blog}

\bigskip

\begin{description}

    \RJpost{2016-01-19}{
Here is an example of \RJedit{text edit by me},
and here one of a footnote by me\RJ{2016-01-19}{Rana test footnote}.
   }

    \RJpost{2015-11-20}{
(Discussion with Predrag, \KS\ \eqva\ and \reqva\ project Spring 2016:
\begin{itemize}
\item blog the project progress here
\item blog whatever I'm reading and learning about
    dynamical systems here
\end{itemize}
    }

    \RJpost{2016-01-19}{
 I've been working on reading the ChaosBook materials and doing the
 online Course 1.
    }

	\RJpost{2016-06-05 to 06-11}{
Read Chapters 14, 17, part of 18, and 19 of Chaosbook. Did homework of
Weeks 9 and 10. Read 1st chapter of Hao and Zheng\rf{hao98} {\em Applied
Symbolic Dynamics and Chaos},  and started reading 2nd chapter.
    }

\PCpost{2016-06-13}{
Try not to get further into Hao and Zheng\rf{hao98} than necessary.
You want to return to symbolic dynamics for cat map as fast as possible,
and then understand Gutkin and Osipov coupled lattice generalization.
}

\PCpost{2016-07-08}{to Rana:
Do you agree with Li Han \reftab{tab:LHarnoldPruned} on the numbers of pruned blocks?
}

	\RJpost{2016-07-10}{
Li's and my results are in agreement.
}

\PCpost{2016-07-11}{to Rana:
Can you complete \reftab{tab:LHarnoldPruned}? Li Han did not write
what the numbers of new pruned blocks are.
}

\PCpost{2016-07-11}{to Rana:
I can already see that several of my posts has not been recovered after
you erased my edits (by mitting to \emph{svn up} before starting your
edits). For example, all my instructions about how to add figures are
gone (but there is a copy at the end of Adrien's blog. You have to add
\emph{RJfrequencies.png} (or \emph{RJfrequencies.pdf}) to
\emph{siminos/figs/}, then \emph{svn add Jfrequencies.png}.

Can you quickly recheck your blog, make sure I have not introduced new
errors while trying to recover my edits?
}

	\RJpost{2016-07-12}{
Li and I checked our the results at the meeting, and we had the same
results for $\cl{a}$ up to $5$ . I saw \reftab{tab:LHarnoldPruned} today, the number of
inadmissible sequences that I found for $\cl{a}=7$ differs from Li's. I
wrote my results in \reftab{RJpruning}.
}

\PCpost{2016-07-12}{You might have  \emph{svn add RJfrequencies.png} (or
\emph{RJfrequencies.pdf}) to \emph{siminos/figs/}, but it has not been
added to the repository yet (you can see that from the commit emails).
Wonder why? Perhaps this:

Remember, always, before \emph{svn up} and \emph{svn ci-m"added entropy figures"}
you have to go to the root directory, \emph{cd [...]/siminos}. Otherwise
you are not refreshing bibtex, figures and other files in the repository.
}

\PCpost{2016-07-12}{ to Rana:

On \emph{zero.physics.gatech.edu}, or
Matt's \emph{light.physics.gatech.edu}, or
\\
Xiong's \emph{hard.physics.gatech.edu}, or
Boris' \emph{love.physics.gatech.edu}, or
any other CNS linux workstation your userid is

rjafari7 (the same passwd as for svn - please change it once you log in)

Do not do calculations on the server \emph{zero.physics.gatech.edu}
- from any machine (including your laptop)

ssh \emph{rjafari7@light.physics.gatech.edu}

save all data on the local hard disk \emph{/usr/local/home/rjafari7/}.
I have made a link in your CNS home directory:

\emph{cd homeLight}

Help for CNS system is on

\HREF{http://www.cns.gatech.edu/CNS-only/index.html}
{www.cns.gatech.edu/CNS-only} cnsuser cnsweb

but current crop of grad students, as a matter of
principle,  never look at any info, or add to these homepages.

Good luck - Xiong knows \texttt{linux} best, also Chris Marcotte and
Burak Budanur (via Skype) know a lot. All our documentation is on
        }

\end{description}




%%%%%%%%%%%%%%%%%%%%%%%%%%%%%%%%%%%%%%%%%%%%%%%%%%%%%%%%%%%%%%%%%%%%%%%
\printbibliography[heading=subbibintoc,title={References}]
