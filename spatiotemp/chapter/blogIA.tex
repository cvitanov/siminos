% siminos/spatiotemp/chapter/blogIA.tex
% $Author: predrag $ $Date: 2021-12-22 18:29:33 -0500 (Wed, 22 Dec 2021) $

% called by siminos/spatiotemp/blogCats.tex


\chapter{Ibrahim's blog}
\label{chap:blogIA}
% Predrag                                       2021-12-07
\renewcommand{\Ssym}[1]{{\ensuremath{m_{#1}}}}    % Boris
\renewcommand{\Refl}{\ensuremath{{\sigma}}} % Dihedral wiki convention
\renewcommand{\shift}{\ensuremath{r}}
\renewcommand{\ssp}{\ensuremath{\phi}}             % lattice site field

	
{\large Ibrahim Abu-hijeh}\\
iabuhijleh3@gatech.edu\\
subversion siminos : iabuhijleh3 \\ % spatioTemp \\
cell +1 714 488 8926

   % *********************************************************************
\hfill   {\color{red} The latest entry at the bottom for this blog}
\bigskip

\bigskip

\begin{description}

\item[2021-12-07 Predrag] to Ibrahim:

As you go along, write up your narrative in this file, ask questions -
this is your personal blog, like an experimentalist's log - everything
that you learn and want to share goes in here.
Clip \& paste anything from other sections you
want to discuss, that saves you LaTeXing time.

\item[2021-12-07 Predrag]
The 3rd line of  \emph{siminos/spatiotemp/blogCats.tex}
says ``process only the files you are editing'',
\begin{verbatim}
% siminos/spatiotemp/inclOnlyCats.tex
% $Author: predrag $ $Date: 2021-12-18 15:14:03 -0500 (Sat, 18 Dec 2021) $

%%%%%%% Recompiling a smaller chunk %%%%%%%%%%%%%%%%%%%%%%%%%%
%%
%% Instead of recompiling the whole blogCats every time,
%% uncomment to compile only a chapter, or a subset of chapters
%%     (only one \includeonly{chapter/...} is allowed at a time)

% \includeonly{chapter/catMap}
% \includeonly{chapter/Henon}
% \includeonly{chapter/LC21}
% \includeonly{chapter/groups}
% \includeonly{chapter/catMapLatt}
% \includeonly{chapter/zeta2D}
% \includeonly{chapter/stability}
% \includeonly{chapter/action}
% \includeonly{chapter/chronotope}
% \includeonly{chapter/editing} % for CL18, GHJSC16 article comments
% \includeonly{chapter/symbolic}
% \includeonly{chapter/prime}
% \includeonly{chapter/appendStatM} % from ChaosBook: cat map chapter
% \includeonly{chapter/Ising}
% \includeonly{chapter/checkers}
% \includeonly{chapter/blogSVW}
% \includeonly{chapter/blogIA}
% \includeonly{chapter/blogXW}
% \includeonly{chapter/blogHL}
% \includeonly{chapter/reportRJ}
% \includeonly{chapter/reportAKS}
% \includeonly{chapter/blogRJ}
% \includeonly{chapter/blogAKS}
% \includeonly{chapter/dailyCats}
% \includeonly{chapter/powwows}

% \includeonly{chapter/catMap,chapter/catMapLatt}
% \includeonly{chapter/catMap,chapter/catMapLatt,chapter/dailyCats}
% \includeonly{chapter/catMap,chapter/blogHL}
% \includeonly{chapter/catMap,chapter/Henon,chapter/blogSVW,chapter/blogHL,chapter/powwows}
% \includeonly{chapter/catMap,chapter/Henon,chapter/LC21,chapter/groups,%
%              chapter/action,chapter/blogSVW,chapter/blogHL}
% \includeonly{chapter/catMap,chapter/prime,chapter/blogHL}
% \includeonly{chapter/catMap,chapter/blogHL,chapter/dailyCats}
% \includeonly{chapter/catMap,chapter/catMapLatt,chapter/blogHL}
% \includeonly{chapter/catMap,chapter/blogHL,chapter/Ising}
% \includeonly{chapter/catMap,chapter/groups,chapter/editing,chapter/blogHL}
% \includeonly{chapter/catMap,chapter/Henon,chapter/groups}
% \includeonly{chapter/catMap,chapter/blogSVW,chapter/blogHL,chapter/powwows}
% \includeonly{chapter/catMap,chapter/catMapLatt,chapter/Henon,%
%               chapter/stability,chapter/action,chapter/blogSVW,%
%               chapter/blogHL,chapter/powwows}
% \includeonly{chapter/catMap,chapter/catMapLatt,chapter/stability,
%              chapter/chronotope,chapter/action,
%              chapter/prime,chapter/Ising,chapter/blogHL,chapter/dailyCats}
% \includeonly{chapter/catMapLatt,chapter/Ising,chapter/dailyCats}
% \includeonly{chapter/catMapLatt,chapter/groups,chapter/blogHL}
% \includeonly{chapter/catMapLatt,chapter/Ising}
% \includeonly{chapter/catMapLatt,chapter/chronotope,chapter/Ising}
% \includeonly{chapter/catMapLatt,chapter/blogHL}
% \includeonly{chapter/catMapLatt,chapter/stability}
% \includeonly{chapter/catMapLatt,chapter/stability,chapter/action,chapter/Ising,chapter/blogHL}
% \includeonly{chapter/LC21,chapter/groups}
% \includeonly{chapter/LC21,chapter/groups,chapter/blogHL}
% \includeonly{chapter/Henon,chapter/LC21,chapter/groups,chapter/blogSVW,chapter/blogHL}
% \includeonly{chapter/stability,chapter/action}
% \includeonly{chapter/chronotope,chapter/dailyCats}
% \includeonly{chapter/blogHL,chapter/dailyCats}
% \includeonly{chapter/blogHL,chapter/dailyCats,chapter/powwows}
% \includeonly{chapter/symbolic,chapter/blogHL}
% \includeonly{chapter/Henon,chapter/blogSVW}
% \includeonly{chapter/Henon,chapter/blogSVW,chapter/blogHL}
% \includeonly{chapter/Henon,chapter/LC21,chapter/groups,chapter/catMapLatt,chapter/blogSVW,chapter/blogHL} %,chapter/powwows}

\end{verbatim}
you uncomment a single line in that file to "process  only the files you
are editing".
% Don't comment out files called further down by the
% masterfile \emph{blogCats.tex}, that can get confusing for other people.

\item[2021-12-07 Predrag to Ibrahim]:

%You refer to an equation like this: CL18 eq.~{tempBern};

%to figure like this: CL18 figure~{fig:BernCyc2Jacob};

%to table like this: \reftab{tab:LxTs=5/2};

You refer to a reference like this: Gutkin and Osipov\rf{GutOsi15}
(\emph{GutOsi15}
refers to an article listed in \emph{../bibtex/siminos.bib}).

and to external link like this:
``For great wallpapers, see overheads in
\HREF{http://www-personal.umich.edu/~engelmm/lectures/ShortCourseSymmetry.html}
{Engel's} course\rf{Engel11}.''

Pro tip: compile \emph{blogCats.tex} often, as you write, and fix errors as
you write. I had to go all they way back to May to find one of Sidney's
unbalanced ``$\{$'' and make the entire blog compile without errors...

\end{description}

\section{Spring 2021 blog}
\label{sect:ibrahim2021}

\begin{description}

\item[2021-12-07 Predrag]
My notes are in
\refsect{sect:phi4latt}~{\em Classical {$\phi^4$} lattice field theory}.
Harrison and you should form a study group to understand this -
it's absolutely essential.

%However, I am hoping he will focus on implementing variational methods
%for computing {\lattstate}s.

\item[2021-12-08 Predrag]
The \HenonMap/$\phi^3$ approaches should be safe for multimodal maps with
complete repelling sets, and it should work for finite-grammar Smale
horseshoe repellers.
Smale's original horseshoe\rf{smale}, his fig.~1 was unimodal, but he
also explicitly gives our $\phi^4$ bimodal repeller, his fig.~5.

    \PCpost{2021-01-27} {
For some background on $\Dn{\infty}$ symmetry of the \templatt\ and
{$\phi^4$} 1$d$ lattice field theory, \refsect{s:latt1d} {\em
Translations and reflections}, see \refrem{rem:timeRev}~{Time reversal};
not a priority as yet.
    }

\item[2021-12-07 Predrag]
The quick \& dirty calculation of short period {\lattstate}s is outlined
by Sidney in \refeq{invertedhen}, see Vattay's \toChaosBook{Item.91}
{exercise 7.2}, copied to here as \refexer{exer:HamHenonCyc}.
We need something like that for {$\phi^4$}.

The Biham-Wentzel method \refeq{LC21BWcubic} (search throught this blog)
might be better. We'll need better methods, but not yet.

\item[2021-12-07 Ibrahim]
I do not understand sources $\Ssym{\zeit}$ in \refeq{LC21:1dHenlatt} and
\refeq{LC21:1dPhi4}. In \templatt\ \refeq{LC21:1dTemplatt} they are a
finite integer-valued alphabet that translates the field to the right
fixed point, but for the nonlinear field theories they are simply a
constant? A constant that can be changed by shifting fields?

\item[2021-12-08, 2021-12-10 Predrag]
You are right - we should think of $\Ssym{\zeit}$ is a (possibly
noninteger) shift that centers the map on a given nonlinear segment. Read
\refsect{sect:noisyGabor} and footnote after \refeq{LC21BWdeviate}, see
whether you have a good formulation that cover all three cases.

\item[2021-12-07 Predrag]
First task:

Start reading
\refchap{chap:LC21}~{\em LC21: Time reversal for lattice field theories}.

Write your study notes up here. And now it's done.

\item[2021-09-12 to 2021-12-22 Predrag]
I have added my guess \refeq{PCphi4} for the infinite coupling $g$
anti-integrable limit of $\phi^4$ theory. That gives a 3-letter
alphabet $\A=\{-1,0,1\}$. One can use it to find by
continuation any {\lattstate}, at $g$ as low as possible.
`Generalized {\HenonMap}s' AKA $\phi^4$ field theory posts are in
\refsect{sect:phi4latt}~{\em Classical {$\phi^4$} lattice field theory}.







\end{description}

\bigskip

%%%%%%%%%%%%%%%%%%%%%%%%%%%%%%%%%%%%%%%%%%%%%%%%%%%%%%%%%%%%%%%%%%%
\renewcommand{\ssp}{\ensuremath{x}}             % lattice site field
\renewcommand{\Ssym}[1]{{\ensuremath{s_{#1}}}}    % Boris
\renewcommand{\Refl}{\ensuremath{\sigma}}             % in DasBuch
\renewcommand{\shift}{\ensuremath{d}}                 % in DasBuch

%%%%%%%%%%%%%%%%%%%%%%%%%%%%%%%%%%%%%%%%%%%%%%%%
\printbibliography[heading=subbibintoc,title={References}]

\ChapterEnd % formatted for ChaosBook.org
