% siminos/spatiotemp/chapter/symGroupXD.tex
% $Author: predrag $ $Date: 2021-08-03 16:02:22 -0400 (Tue, 03 Aug 2021) $

\section{Group theory and symmetries: a review}
\label{sect:group}

    \PC{2021-06-19}{
    A copy of the 2017-03-09 section from Xiong Ding's thesis\\
    siminos/xiong/thesis/chapters/symGroup.tex.
    }
    \PC{2021-06-19}{
    Update/replace the ChaosBook version, as now
    $\Zn{2}\to\Dn{1}, \Zn{3}\to\Cn{3}$.
    }
In quantum mechanics, whenever a system exhibits some symmetry, the
corresponding symmetry group commutes with the Hamiltonian of this
system, namely, $[U(\LieEl), H] = U(\LieEl)H - HU(\LieEl) = 0$. Here
$U(\LieEl)$ denotes the operation corresponding to symmetry $g$ whose
meaning will be explained soon. The set of eigenstates with degeneracy
$\ell$, $\{\phi_1, \phi_2, \cdots, \phi_\ell\}$, corresponding to the same
system energy $H\psi_i = E_n\psi_i$, is invariant under the symmetry
since $U(\LieEl)\psi_i$ are also eigenvectors for the same energy.
This information helps us understand the spectrum of a Hamiltonian and
the quantum mechanical selection rules. We now apply the same idea to
the classical {\evOper} $\Lop^t(\ssp_e, \ssp_s)$
for a system $\flow{t}{\ssp}$ equivariant under a discrete symmetry group
$\Group=\{e, \LieEl_2, \LieEl_3,\cdots, \LieEl_{|\Group|}\}$ of order
$|\Group|$:
\begin{equation}
  \label{eq:equiva}
  \flow{t}{{D}(\LieEl)}\ssp)={D}(\LieEl)\,\flow{t}{\ssp} \quad \text{for}
  \quad \forall
  \LieEl\in\Group
  \,.
\end{equation}
We start with a review of some basic facts of
the group representation theory. Some examples of good references
on this topic are \refref{Hamermesh62, Tinkham}.

Suppose group $\Group$ acts on a linear space $V$ and function
$\rho(\ssp)$ is defined on this space $\ssp\in V$. Each element
$\LieEl\in\Group$ will transform point $\ssp$ to ${D}(\LieEl)\ssp$. At
the same time, $\rho(\ssp)$ is transformed to $\rho'(\ssp)$. The value
$\rho(\ssp)$ is unchanged after state point $\ssp$ is transformed to
${D}(\LieEl)\ssp$, so $\rho'({D}(\LieEl)\ssp) = \rho(\ssp)$. Denote
$U(\LieEl)\rho(\ssp)=\rho'(\ssp)$, so we have
\begin{equation}
  \label{eq:ogfx}
  U(\LieEl)\rho(\ssp) = \rho({D}(\LieEl)^{-1}\ssp)
  \,.
\end{equation}
This is how functions are transformed by group operations. Note, $D(\LieEl)$
is the representation of $G$ in the form of space transformation matrices.
The
operator $U(\LieEl)$, which acts on the function space, is not the same as
group operation ${D}(\LieEl)$, so \refeq{eq:ogfx} does not mean that
$\rho(\ssp)$ is invariant under $\Group$. \refExam{exam:C3matrRep} gives
the space transformation matrices of $\Cn{3}$.

%%%%%%%%%%%%%%%%%%%%%%%%%%%%%%%%%%%%%%%%%%%%%%%%%%%%%%%%%%%%%
% \example{A matrix representation of cyclic group $\Cn{3}$.}
\fastTrackExam{exam:C3matrRep}     % \toExam
%%%%%%%%%%%%%%%%%%%%%%%%%%%%%%%%%%%%%%%%%%%%%%%%%%%%%%%%%%%%%

\subsection{Regular representation}
An operator $U(\LieEl)$ which acts on an infinite\dmn\ function space
is too abstract to analyze.
We would like to represent it in a more familiar way.
Suppose there is a function $\rho(\ssp)$ with symmetry $\Group$ defined in full
\statesp\ $\pS$, then full \statesp\ can be decomposed as a union
of $|\Group|$ tiles each of which is obtained by transforming the fundamental
domain,
\begin{equation}
  \label{eq:domain}
  \pS =  \bigcup_{g\in \Group}g\pSRed
  \,,
\end{equation}
where $\pSRed$ is the chosen fundamental domain.
So $\rho(\ssp)$ takes $|G|$ different forms by \refeq{eq:ogfx} in each sub-domain
in \refeq{eq:domain}. Now, we obtained a natural choice of a set of bases in this
function space called the \emph{regular basis},
\beq
\label{eq:RegBasis}
\{ \rho_1^{reg}(\sspRed), \rho_2^{reg}(\sspRed), %\rho_3^{reg}(\sspRed),
\cdots, \rho_{|\Group|}^{reg}(\sspRed)\}
=
\{
\rho(\sspRed), \rho(\LieEl_2\sspRed), % \rho(\LieEl_3\sspRed),
\cdots, \rho(\LieEl_{|\Group|}\sspRed) \}
\,.
\eeq
Here, for notation simplicity we use
$\rho(\LieEl_i\sspRed)$ to represent $\rho(D(\LieEl_i\sspRed))$ without
ambiguity.
These bases are
constructed by applying $U(\LieEl^{-1})$ to $\rho(\sspRed)$ for each
$\LieEl\in\Group$, with $\sspRed$ a point in the fundamental domain.
The [$|G|\!\times\!|G|$] matrix representation of the
action of $U(\LieEl)$ in basis \refeq{eq:RegBasis} is called the \emph{(left)
regular representation} $D^{reg}(\LieEl)$. Relation \refeq{eq:ogfx} says that
$D^{reg}(\LieEl)$ is a permutation matrix, so each row or column has only one nonzero
element.

We have a simple trick to obtain the regular representation quickly.
Suppose the element at the $i$th row and
the $j$th column of $D^{reg}(\LieEl)$
is $1$. It means
$\rho(\LieEl_i\sspRed) = U(\LieEl) \rho(\LieEl_j\sspRed)$, which
is $g_i=\LieEl^{-1}g_j \implies g^{-1} = g_i g_j^{-1}$. Namely,
\beq
D^{reg}(\LieEl)_{ij} = \delta_{\LieEl^{-1},\, g_i g_j^{-1}}
\,.
\ee{eq:RegRep}
So if we arrange the
columns of the multiplication table by the inverse of the group elements,
then setting positions with $\LieEl^{-1}$ to 1 defines the regular
representation $D^{reg}(\LieEl)$. Note, the above relation can
be further simplified to $g = g_jg_i^{-1}$, but it exchanges the rows and
columns of the multiplication table, so $g = g_jg_i^{-1}$
should not be used to get $D^{reg}(\LieEl)$.
On the other hand, it is easy to see
that the regular representation of group element $e$ is always the identity matrix.

%%%%%%%%%%%%%%%%%%%%%%%%%%%%%%%%%%%%%%%%%%%%%%%%%%%%%%%%%%%%%%%%%%%%%%
% \example{The regular representation of cyclic group $\Cn{3}$.}
\fastTrackExam{exam:C3regularRep}     % \toExam
%%%%%%%%%%%%%%%%%%%%%%%%%%%%%%%%%%%%%%%%%%%%%%%%%%%%%%%%%%%%%

\subsection{Irreducible representations}

$U(\LieEl)$ is a linear operator under the regular basis.
Any linearly independent combination of the regular bases can be used as
new basis, and then the representation of $U(\LieEl)$ changes respectively.
So we ask a question: can we find a new set of bases
\begin{equation}
  \rho^{irr}_i=\sum_j S_{ij}\rho^{reg}_j
  \label{eq:trans}
\end{equation}
such that the new representation $D^{irr}(\LieEl) = SD^{reg}(\LieEl)S^{-1}$ is block-diagonal
for any $\LieEl\in\Group$ ?
\begin{equation}
  D^{irr}(\LieEl) =
  \begin{bmatrix}
    D^{(1)}(\LieEl) & & \\
    & D^{(2)}(\LieEl) & \\
    & & \ddots \\
  \end{bmatrix}
  = \bigoplus_{\mu=1}^\shift d_\mu D^{(\mu)}(\LieEl)
  \,.
  \label{eq:irre}
\end{equation}
In such a block-diagonal representation, the
subspace corresponding to each diagonal block is invariant under $\Group$
and the action of $U(\LieEl)$ can be analyzed subspace by subspace.
It can be easily checked that for each $\mu$, $D^{(\mu)}(\LieEl)$ for all
$\LieEl\in\Group$ form another representation (\emph{irreducible
  representation}, or \emph{irrep}) of group $\Group$.
Here, $r$ denotes the total number of
irreps of $\Group$. The same irrep may show up more than once in the decomposition
\refeq{eq:irre}, so the coefficient $d_{\mu}$ denotes the number of its
copies.  Moreover, it is proved\rf{Hamermesh62} that $d_{\mu}$ is also equal to the dimension
of $D^{(\mu)}(\LieEl)$ in \refeq{eq:irre}.
Therefore, we have a relation
\[
  \sum_{\mu=1}^r d_\mu^2 = |G|
  \,.
\]

%%%%%%%%%%%%%%%%%%%%%%%%%%%%%%%%%%%%%%%%%%%%%%%%%%%%%%%%%%%%%%%%%%%%%%
% \example{Irreps of cyclic group $\Cn{3}$.}
\fastTrackExam{exam:C3irReps}     % \toExam
%%%%%%%%%%%%%%%%%%%%%%%%%%%%%%%%%%%%%%%%%%%%%%%%%%%%%%%%%%%%%

\paragraph{Character tables.}
Finding a transformation $S$ which simultaneously block-diagonalizes the
regular representation of each group element sounds difficult.
However, suppose it can be achieved and we obtain a set of irreps $D^{(\mu)}(\LieEl)$,
then according to Schur's lemmas\rf{Hamermesh62}, $D^{(\mu)}(\LieEl)$ must satisfy a set of
orthogonality relations:
\begin{equation}
  \label{eq:ortho}
  \frac{d_\mu}{|G|} \sum_g D_{il}^{(\mu)}(\LieEl) D_{mj}^{(\nu)}(g^{-1}) = \delta_{\mu \nu}
  \delta_{ij} \delta_{lm}
  \,.
\end{equation}
Denote the trace of irrep $D^{(\mu)}$ as $\chi^{(\mu)}$, which is referred to as
the \emph{character} of $D^{(\mu)}$. Properties of irreps can be derived from
\refeq{eq:ortho}, and we list them as follows:
\begin{enumerate}
\item The number of irreps is the same as the number of
  classes.
\item Dimensions of irreps satisfy
  $\sum_{\mu=1}^\shift d^2_\mu = |G| $
\item Orthonormal relation I :
  $\sum_{i=1}^\shift |K_i| \chi_i^{(\mu)} \chi_i^{(\nu)*} = |G|\delta_{\mu \nu} $. \\
  Here, the summation goes through all classes of this group, and $|K_i|$ is
  the number of elements in class $i$. This weight comes from the fact that
  elements in the same class have the same character. Symbol $*$ means
  the complex conjugate.
\item Orthonormal relation II :
  $\sum_{\mu=1}^\shift \chi_i^{(\mu)} \chi_j^{(\mu)*} = \frac{|G|}{|K_i|}\delta_{ij} $. \\
\end{enumerate}
The characters for all classes and irreps of a finite group are
conventionally arranged into a \emph{character table}, a square matrix
whose rows represent different
classes and columns represent different irreps.
Rules 1 and 2 help determine the number of irreps and
their dimensions. As the matrix representation of class $\{e\}$ is always the
identity matrix, the first row is always the dimension of the
corresponding representation. All entries of the first column are always 1,
because the symmetric irrep is always one\dmn. To compute the
remaining entries, we should use properties 3, 4 and the class multiplication
tables.   Spectroscopists conventions use labels $A$ and $B$ for
symmetric, respectively antisymmetric nondegenerate irreps, and
$E$, $T$, $G$, $H$ for doubly, triply, quadruply, quintuply degenerate irreps.

%%%%%%%%%%%%%%%%%%%%%%%%%%%%%%%%%%%%%%%%%%%%%%%%%%%%%%%%%%%%%%%%%%%%%%
% \example{Character table of $\Dn{3}$.}
\fastTrackExam{exam:D3charTab}     % \toExam
%%%%%%%%%%%%%%%%%%%%%%%%%%%%%%%%%%%%%%%%%%%%%%%%%%%%%%%%%%%%%

\subsection{Projection operator}
We have listed the properties of irreps and the
techniques of constructing a character table, but we still do not know how to
construct the similarity transformation $S$ which takes a regular representation into a
block-diagonal form. Think of it in another way,
each irrep is associated with an invariant subspace, so by
projecting an arbitrary function $\rho(\ssp)$ into its invariant subspaces, we
find the transformation \refeq{eq:trans}.
One of these invariant subspaces is $\sum_g \rho(g\sspRed)$, which is the basis of
the one\dmn\ symmetric irrep $A$. For $\Cn{3}$, it is \refeq{eq:c3f1}.
But how to get the others? We resort to the projection operator:
\begin{equation}
  \label{eq:projecIrre}
  P^{(\mu)}_{i} = \frac{d_\mu}{|G|}\sum_g \left(D^{(\mu)}_{ii} (\LieEl)\right)^* U(g)
  \,.
\end{equation}
It projects an arbitrary function into the $i$th basis of irrep
$D^{(\mu)}$ provided the diagonal elements of this representation
$D^{(\mu)}_{ii}$ are known. $ P^{(\mu)}_{i} \rho(\ssp) = \rho^{(\mu)}_i$.
Here, symbol $*$ means the complex conjugate. For unitary groups
$\left(D^{(\mu)}_{ii} (\LieEl)\right)^* = D^{(\mu)}_{ii} (\LieEl^{-1})$.
Summing $i$ in \refeq{eq:projecIrre} gives
\begin{equation}
  \label{eq:projectSum}
  P^{(\mu)} = \frac{d_\mu}{|G|}\sum_g \left(\chi^{(\mu)}(\LieEl)\right)^*U(g)
  \,.
\end{equation}
This is also a projection operator which projects an arbitrary function onto
the sum of the bases of irrep $D^{(\mu)}$.

Note, for one\dmn\ representations, \refeq{eq:projectSum}
is equivalent to \refeq{eq:projecIrre}. The projection operator is
known after we obtain
the character table, since the character of an one\dmn\ matrix is the matrix itself.
However, for two\dmn\ or higher\dmn\ representations, we need to know the
diagonal elements $D^{(\mu)}_{ii}$ in order to get the basis of invariant subspaces. That is to say,
\refeq{eq:projecIrre} should be used instead of \refeq{eq:projectSum} in this case.
\refExam{exam:D3irrepBases} illustrates this point. The two one\dmn\ irreps are obtained
by \refeq{eq:projectSum}, but the other four two\dmn\ irreps are obtained by
\refeq{eq:projecIrre}.

%%%%%%%%%%%%%%%%%%%%%%%%%%%%%%%%%%%%%%%%%%%%%%%%%%%%%%%%%%%%%%%%%%%%%%
% \example{Bases for irreps of $\Dn{3}$.}{
\fastTrackExam{exam:D3irrepBases}     % \toExam
%%%%%%%%%%%%%%%%%%%%%%%%%%%%%%%%%%%%%%%%%%%%%%%%%%%%%%%%%%%%%

The $\Cn{3}$ and $\Dn{3}$ examples used in this section can be
generalized to any $\Cn{n}$ and $\Dn{n}$. For references,
\refExam{exam:CnChars}, \refexam{exam:DnOddChars} and
\refexam{exam:DnEvenChars} give the character tables of $\Cn{n}$ and
$\Dn{n}$.

%%%%%%%%%%%%%%%%%%%%%%%%%%%%%%%%%%%%%%%%%%%%%%%%%%%%%%%%%%%%%%%%%%%%%%
% \example{Character table of cyclic group \Cn{n}.}
\fastTrackExam{exam:CnChars}     % \toExam
%%%%%%%%%%%%%%%%%%%%%%%%%%%%%%%%%%%%%%%%%%%%%%%%%%%%%%%%%%%%%
%
%%%%%%%%%%%%%%%%%%%%%%%%%%%%%%%%%%%%%%%%%%%%%%%%%%%%%%%%%%%%%%%%%%%%%%
% \example{Character table of dihedral group $\Dn{n}$, $n$ odd.}
\fastTrackExam{exam:DnOddChars}     % \toExam
%%%%%%%%%%%%%%%%%%%%%%%%%%%%%%%%%%%%%%%%%%%%%%%%%%%%%%%%%%%%%
%
%%%%%%%%%%%%%%%%%%%%%%%%%%%%%%%%%%%%%%%%%%%%%%%%%%%%%%%%%%%%%
% \example{Character table of dihedral group $\Dn{n}$, $n$ even.}
\fastTrackExam{exam:DnEvenChars}     % \toExam
%%%%%%%%%%%%%%%%%%%%%%%%%%%%%%%%%%%%%%%%%%%%%%%%%%%%%%%%%%%%%
