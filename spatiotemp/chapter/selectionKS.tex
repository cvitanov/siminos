% siminos/spatiotemp/chapter/selectionKS.tex
% $Author: mgudorf3 $ $Date: 2020-10-22 10:19:52 -0400 (Thu, 22 Oct 2020) $

%\subsection{Selection rules for \KS}

\subsection{Selection rules for real-valued Fourier coefficients}
%\label{sect:selectionKS}

%Possible figures
%Pictographic display of selection rules/constraints




Although the \spt\ \KSe\ is easier to write in terms of a complex
Fourier-Fourier basis, the symmetry invariant subspaces generated
by symmetry constraints is easier to describe in terms of \rv\ \Fcs.
The \rv\
{\spt} Fourier expansion can be written

This is the expansion for a general {\spt} solution. For each discrete
symmetry of the \spt\ \KSe\ there is a unique set of constraints or
``selection rules'' for the \spt\ \Fcs. These selection rules constitute
\textit{symmetry invariant subspaces} of solutions of the \spt\ \KSe. In
this section we commit to the description of the selection rules of the
\Fcs. For more discussion on the symmetries themselves we refer the
reader to \refsect{sect:KSsymm}.

The two discrete symmetries we will describe are spatial reflection
symmetry and \spt\ shift-reflection symmetry. The shift-reflection
symmetry is a special case of the broader symmetry group $\Dn{n}\times
\Cn{n}$ ($n=2$). Due to the uncommon appearance of solutions with $n>2$
and the relatively easy generalization of the $n=2$ shift-reflection
case, we shall only consider the $n=2$ symmetry group.

The general procedure for producing these selection rules is not very
complicated. Let $R$ represent an arbitrary symmetry operation. If a
solution is invariant under $R$ then it satisfies the \textit{invariance
condition} $Ru = u$ or equivalently $(R-1)u=0$. Substitution of the
expansion \refeq{e-RealFourier} produces a set of constraints that can
only be satisfied when a subset of \rv\ \Fcs\ are individually equal to
zero. To begin we start with spatial reflection symmetry, as it is almost
trivial. Solutions invariant under spatial reflection only admit
spatially antisymmetric basis functions. Therefore, the selection rules
for spatial reflection symmetry are
\beq \label{e-ReflRules}
\akj,\bkj = 0 \mbox{  for all  } k,j
\,.
\eeq
\Spt\ shift-reflection is the composition of two symmetry operations:
spatial reflection $\Refl_x$ and time translation by $\tau_{\period{}/2}$.
The action of this symmetry is as follows $\Refl_x \tau_{\period{}/2} u(\conf, \zeit) = -u(-\conf, \zeit + \period{}/2)$.
When directly applied to the \rv\ Fourier expansion \refeq{e-RealFourier}
and by virtue of trigonometric identities and the parity of $\sin$ and $\cos$ we have
\beq
\begin{split}
\Refl_x \tau_{\period{}/2}\,u(\xm, \tn) &= -\sum_{k,j}
                                \cos(\wavek (-\xm)) (\akj\cos(\freqj (\tn + \period{}/2)) + \bkj \sin(\freqj (\tn + \period{}/2))) \continue
                                &+ \sin(\wavek(-\xm)) (\ckj\cos(\freqj (\tn + \period{}/2)) + \dkj\sin(\freqj (\tn + \period{}/2))) \continue
                            &= \sum_{k,j} -\cos(\wavek \xm) (\akj \cos(\freqj \tn)\cos(\pi j) + \bkj \sin(\freqj \tn)\cos(\pi j)))\continue
                                &+ \sin(\wavek \xm) (\ckj\cos(\freqj \tn)\cos(\pi j) + \dkj \sin(\freqj \tn)\cos(\pi j)) \continue
                            &= \sum_{k,j} (-1)^{j+1} \cos(\wavek \xm) (\akj\cos(\freqj \tn) + \bkj\sin(\freqj \tn))\continue
                                &+ (-1)^{j} \sin(\wavek \xm) (\ckj \cos(\freqj \tn) + \dkj \sin(\freqj \tn)) \, ,
\end{split}
\ee{e-ShiftReflectBasis}
By combining this with the invariance condition
$(\Refl_x \tau_{\period{}/2} - 1)u = 0$,
we find that the selection rules for \spt\ shift-reflection are as follows
\bea \label{e-ShiftReflRules}
\akj,\bkj &=& 0 \, \mbox{ for }\, j \, \mbox{ even } \continue
\ckj,\dkj &=& 0 \, \mbox{ for }\, j \, \mbox{ odd }
\,.
\eea

%%%%%%%%%%%%%%%%%%
%There is an argument to be made to either
%1. Discuss everything in in terms of real-valued Fourier coefficient (sFb.tex equations -> real-valued basis)
%2. Discuss the selection rules in terms of the complex coefficients only.
%Both of these options have their disadvantages:
%(1) requires description of operators in terms of
%in an \On{2} representation vs. U(1), I find introducing matrix operators in the spatiotemporal equations
%to be less intuitive.
%(2) The selection rules for complex-valued coefficients are not nearly as nice as {Subset of Fourier Coefficients = 0},
%instead the constraints include conjugation and multiplication by powers of -1. This makes the description of the
%Fourier transform operations to be much less intuitive.
%%%%%%%%%%%%%%%%%%


%\subsection{Selection rules for complex Fourier coefficients}
%Firstly just as a matter of note,
%for the complex-valued {\spt} Fourier transform, $m$ will be the spatial mode number
%and $n$ will be the temporal mode number. The discrete Fourier transform is defined as follows,

%\beq
%u(\xm,\tn) = \frac{1}{\sqrt{KJ}}\sum_{k=-K/2}^{K/2-1}\sum_{j=-J/2,J/2-1} \utensor_{kj}e^{\ii(\wavek \xm + \freqj \tn)},
%\eeq

%which we then perform shift and reflect symmetry operation on,

%\beq
%\Refl_x \tau_{\period{}/2} u(\xm,\tn) = \frac{1}{\sqrt{KJ}}\sum_{k=-K/2}^{K/2-1}\sum_{j=-J/2,J/2-1} -\utensor_{kj}e^{\ii(\wavek (-\xm) + \freqj %(\tn+\period{}/2))},
%\eeq
%which after some simplification can be written
%\beq
%\Refl_x \tau_{\period{}/2} u(\xm,\tn) =\frac{1}{\sqrt{KJ}}\sum_{k=-K/2}^{K/2-1}\sum_{j=-J/2,J/2-1} (-1)^{j+1} \utensor_{kj}e^{\ii(\wavek (-\xm) + \freqj %\tn)}.
%\eeq
%Equating the expansion before and after the symmetry operations in addition to some identities
%between the \Fcs\ yields the selection rules
%\beq \label{e-ComplexShiftReflRules}
%(-1)^{n+1} \utensor_{n,m} = \utensor_{n,-m} \,.
%\eeq
%These selection rules allow us to reduce the dimensionality of the system as some
%of the \Fcs\ are redundant. The velocity field
%of any solution with this symmetry can be reproduced with the non-redundant \Fcs.
%Therefore we may implement a symmetry specific \xDft\ which eliminates the redundant
%\Fcs\ for the purpose of computation and then reconstructs the spectrum upon
%application of the inverse \xDft.
%We believe that this reconstruction is best demonstrated in a pictographic fashion
%as the {\spt} \Fcs\ can be easily represented with a matrix. The \Fcs\ are arranged
%in the manner in which they would appear computationally after using SciPy's
%implementation of the Fast Fourier Transform routine.
%\begin{table}[h!]
%\caption{Shift-reflect spectrum reconstruction.}
%\centering
%\begin{tabular}{|c|c|c|c|c|}
%\hline
%\quad                & $k = 0$  & $k = 1,\ldots,K/2-1$            & $k = -\frac{K}{2}$ & $k = -K/2+1,\ldots,-1$    \\
%\hline
%$j = 0,\ldots,J/2-1$ & $0$      & $\utensor_{kj}$                     & $0$                  & $(-1)^{n+1}\utensor_{kj}$   \\
%\hline
%$j=\frac{-J}{2}$     & $0$      & $0$                             & $0$                  & $0$                     \\
%\hline
%$j=-J/2+1,\ldots,-1$ & $0$      & $(-1)^{j+1}\utensor_{kj}^{\dagger}$ & $0$                  & $\utensor_{kj}^{\dagger}$   \\
%\hline
%\end{tabular}
%\end{table}
%Where $\utensor_{kj}$ represents the set of \Fcs\ with positive indices, $j>0$ and $k>0$. From \refeq{e-ComplexShiftReflRules}
%we can reconstruct the original field $u(\xm,\tn)$ using only these modes.
%The same type of table can be constructed for the antisymmetric subspace
%$\bbU^+$, which follows the following rules,
%\begin{table}[h!]
%\caption{Antisymmetric spectrum reconstruction.}
%\centering
%\begin{tabular}{|c|c|c|c|c|}
%\hline
%\quad                & $k = 0$  & $ = 1,\ldots,K/2-1$            & $k = -\frac{K}{2}$ & $k = -K/2+1,\ldots,-1$    \\
%\hline
%$j = 0,\ldots,J/2-1$ & $0$      & $\utensor_{kj}$                     & $0$                  & $(-1)\utensor_{kj}$   \\
%\hline
%$j=\frac{-J}{2}$     & $0$      & $0$                             & $0$                  & $0$                     \\
%\hline
%$j=-J/2+1,\ldots,-1$ & $0$      & $(-1)\utensor_{kj}^{\dagger}$ & $0$                  & $\utensor_{kj}^{\dagger}$   \\
%\hline
%\end{tabular}
%\end{table}
