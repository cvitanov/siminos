% siminos/spatiotemp/catMapLatt.tex
% $Author: predrag $ $Date: 2020-12-14 00:22:48 -0500 (Mon, 14 Dec 2020) $

\section{Counting {\twots}}
\label{s:2DcatCounting}
% until 2019-08-13 was siminos/kittens/twots2D.tex      pdflatex CL18

%%%%%%%%%%%%%%%%%%%%%%%%%%%%%%%%%%%%%%%%%%%%%%%%%%%%%%%%%%%%%%%%%%%%%%%%
	\HL{2019-06-25}{
    This section is a version of kittens refsect~{s:dDcatMap}
    that starts from 2D cat map
    without giving the formula of general $d$\dmn\ \catlatt. I feel
    this is less clear than start with the $d$\dmn\ \catlatt, but
    it directly follows the section of {\spt} cat map.
    Eventually this text was not used not used in kittens\rf{CL18}.
    }
An {\twot} on a 2\dmn\ {\spt}ly infinite $\integers^2$ lattice has
more complicated pattern than a cat map \po. An
{\twot} can tile the infinitely large 2\dmn\ space not only by
repeating in the time or space direction, but also by moving in both of
the {\spt} directions. The repeating pattern can generally be described
by a Bravais lattice:
\beq
\lattice = \{n_1 {\bf a}_1 + n_2 {\bf a}_2 | n_i \in \mathbb{Z}\}.
\ee{2DBravLatt}
And the {\twots} tile the infinitely large 2\dmn\ space by:
\beq
\ssp_{{\bf z}} = \ssp_{{\bf z}+{\bf R}} \, , \quad {\bf R} \in \lattice \, .
\ee{2DPeriodicField}
The ${\bf z}$ here is a two\dmn\ vector which labels the position and time of the field.
The {\sPe} \refeq{2dCoupledCats} can be written as:
\beq
(-2s + \sigma_1 + \transp{\sigma}_{1} + \sigma_2 + \transp{\sigma}_{2}) \ssp_{\bf z} = -m_{\bf z} \, ,
\ee{2DCat}
where the $\sigma_i$ is a translation operator which can translate the
field in the positive $i$th direction by length one and
$\transp{\sigma}_{i}$ is the inverse of the operator $\sigma_i$ which
translates the field in the negative $i$th direction. Here we can assume
that $\sigma_1$ is a translation in the time and $\sigma_2$ is a
translation in space. But since the system is invariant under the
exchange of space and time, we don't need to distinguish these two
directions.

Note that in \refeq{2DCat} the operators, field and source are defined on
infinitely large 2\dmn\ space (lattice). For an {\twot}, which is
a periodic tile, the {\sPe} \refeq{2DCat} is also satisfied
on this finite tile. But in this case, the translation operators need to
satisfy the periodic {\bcs} specified by this {\twot}. And
the
$-2s +\sigma_1 +\transp{\sigma}_{1} +\sigma_2 +\transp{\sigma}_{2}$
on the finite region is the {\jacobianOrb} matrix of this specific periodic
pattern.

Following the same procedure as counting the periodic points of a cat
map, we know that the number of periodic points is given by the
determinant of the {\jacobianOrb}. To find the determinant and the
inverse of the {\jacobianOrb}, we need to first find the eigenvectors
and eigenvalues.

The eigenvectors here are fields defined in this finite tile. The
elements of these eigenvectors are generally complex numbers. If we tile
the whole 2\dmn\ space with one of these finite fields using the periodic
condition, we will get an eigenvector of the operator in \refeq{2DCat}
defined in the infinite 2\dmn\ space. And the eigenvalue remains
unchanged. So we can find the eigenvectors and eigenvalues in the
infinite 2\dmn\ space then reduce the field into the finite tiles.

	\HL{2019-06-17}{
	I will need to rewrite this paragraph to make it clearer.
	}

For a 2\dmn\ {\spt} cat map, we want to find eigenvectors with
periodicity given by the Bravais lattice \refeq{2DBravLatt}, where ${\bf
a}_1$ and ${\bf a}_2$ are two 2\dmn\ basis vectors. The general form of
these basis vectors are ${\bf a}_1 = \{l_1, l_2\}$ and ${\bf a}_2 =
\{l_3, l_4\}$. For a given Bravais lattice, the choice of basis vectors
is not unique. It is shown by Lind\rf{Lind96}, \CBlibrary{Lind96} that we can choose basis
vectors with form ${\bf a}_1 = \{l_1, 0\}$ and ${\bf a}_2 = \{l_3, l_4\}$
without loss of generality.
    \PC{2020-02-15}{
    This is called `Hermite normal form', see
    \refeq{Holmin12-Hermite}.
    }
Then the reciprocal lattice is:
\beq
\overline{\lattice} = \{n_1 {\bf b}_1 + n_2 {\bf b}_2 | n_i \in \mathbb{Z}\}
\,,
\ee{2DReciprLatt}
where the vectors ${\bf b}_1$ and ${\bf b}_2$ satisfy:
\beq
{\bf b}_i \cdot {\bf a}_j = 2 \pi \delta_{ij}
\,.
\ee{2DReciprocalVectors}
The eigenvectors of the translation operator which satisfy the
periodicity of the Bravais lattice \refeq{2DBravLatt} are
plane waves of form:
\beq
f_{\bf k}({\bf z}) = e^{i {\bf k} \cdot {\bf z}}
  \,, \quad
{\bf k} \in \overline{\lattice}
\,,
\ee{2DEigenvect}
where the wave vector ${\bf k}$ is on the reciprocal lattice
$\overline{\lattice}$. For the basis vectors ${\bf a}_1 = \{l_1, 0\}$
and ${\bf a}_2 = \{l_3, l_4\}$, the basis vectors of the corresponding
reciprocal lattice are ${\bf b}_1 = 2 \pi / l_1 l_4 \, \{l_4, -l_3\}$ and
${\bf b}_2 = 2 \pi / l_1 l_4 \, \{0, l_1\}$. The expression of
eigenvector with wave vector ${\bf k} = n_1 {\bf b}_1 + n_2 {\bf b}_2$
is:
\beq
f_{\bf k}({\bf z}) = e^{i {\bf k} \cdot {\bf z}} = \exp[i\frac{2 \pi}{l_1 l_4}(n_1 l_4 z_1 - n_1 l_3 z_2 + n_2 l_1 z_2)] \, ,
\ee{2DEigenvect1}
where the ${\bf z}=(z_1,z_2)$. The eigenvalue of the operator $s -
\sigma_1 - \transp{\sigma}_{1} - \sigma_2 - \transp{\sigma}_{2}$
corresponding to this eigenvector is:
\beq
\lambda_{\bf k} = s - 2 \cos(\frac{2 \pi n_1}{l_1}) - 2 \cos(-\frac{2 \pi n_1 l_3}{l_1 l_4} + \frac{2 \pi n_2}{l_4})
\, .
\ee{2DEigenval}
It is sufficient to use the wave vectors ${\bf k}$ with $n_1$ from 0 to
$l_1-1$ and $n_2$ from 0 to $l_4-1$ to get all of the eigenvectors. Any
wave vector on the reciprocal lattice outside of this range will give an
eigenvector which is equivalent to an eigenvector with the wave vector in
the range. So the number of eigenmodes we can get is $l_1 l_4$, which is
the number of lattice sites in a smallest repeating tile.

Using the counting formula \refeq{perOrbits:Fourier}, we can find the
number of the periodic points by computing the determinant of the
{\jacobianOrb}, which is the operator $-2s +\sigma_1 +\transp{\sigma}_{1}
+\sigma_2+\transp{\sigma}_{2}$ defined on the finite tile with periodic
{\bcs}:
\beq
N
= \prod_{\bf k} \lambda_{\bf k}
= \prod_{n_1=0}^{l_1-1} \prod_{n_2=0}^{l_4-1}
\left[
2s - 2 \cos(\frac{2 \pi n_1}{l_1}) - 2 \cos(-\frac{2 \pi n_1 l_3}{l_1 l_4} + \frac{2 \pi n_2}{l_4})
\right]
 \,.
\ee{2DCount}
This is the number of periodic points with the periodicity given by
Bravais lattice \refeq{2DBravLatt} with the basis vectors ${\bf a}_1 =
\{l_1, 0\}$ and ${\bf a}_2 = \{l_3, l_4\}$.

Using the eigenvectors we can do a Fourier transform to the
{\jacobianOrb} and get the inverse which is the Green's function.
%But first we need to set the range of the repeating tile. We know that given a Bravais lattice the choice of the lattice cell is not unique. The area of the lattice cell is given by the wedge product of the two basis vectors.
