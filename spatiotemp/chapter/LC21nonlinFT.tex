% siminos/reversal/LC21nonlinFT.tex      pdflatex LC21; bibtex LC21
% temporary: siminos/spatiotemp/chapter/LC21nonlinFT.tex
% $Author: predrag $ $Date: 2021-12-24 01:25:20 -0500 (Fri, 24 Dec 2021) $

% was \section{Nonlinear lattice field theories}
%\label{s:nonlinFT}

\section{A $\ssp^3$ field theory}
% was {{\Henlatt}}
\label{s:henlatt}

The `mod~1' in the definition of the `linear' kicked rotor, the cat map
\refeq{catMap}, makes cat map a highly nonlinear, discontinuous map. In
contrast, field theory action $\action[\Xx]$ is typically polynomial and
smooth. The simplest such nonlinear action turns out to correspond to the
paradigmatic dynamicist's model of a 2\dmn\ nonlinear dynamical system,
the {\HenonMap}\rf{henon}
\bea
    x_{\zeit+1} &=& 1-a\,x_{\zeit}^2 + b\,y_{\zeit}
        \continue
    y_{\zeit+1} &=& x_{\zeit}
\,.
\label{LC21eq2.1}
\eea
For the contraction parameter value $b=-1$ this is a Hamiltonian map.

The \emph{temporal evolution} \jacobianM\ for the $n$th iterate of the
Hamiltonian {\HenonMap} is the product of consecutive  one time-step
\jacobianMs
\beq
\jMps^\cl{}(x_0,y_0) =
\prod_{m=\cl{}-1}^{0}
            \left(\begin{array}{cc}
                -2a\,x_m & -1 \\
                         1 & 0
            \end{array}\right)
\,,\qquad x_m = \map^{m}_1 (x_0,y_0)
\,,
\ee{Henlatt-e_her}
where the successive 1-time step \jacobianMs\ are multiplied in the order
they are applied, as
$\jMps^\cl{}(x_0)=\jMps(x_{\cl{}-1})\cdots\jMps(x_0)$. So, once we have a
{\HenonMap} {\po}, we also have its \FloquetM\
(monodromy matrix). When $\jMps^\cl{}$ is
hyperbolic, only the expanding
eigen\-value $\ExpaEig_1=1/\ExpaEig_2$ needs to be determined, as the
determinant of the {\Henon} 1-time step \jacobianM\ is unity,
\beq
\det\jMps = \ExpaEig_1 \ExpaEig_2 = 1
\,.
\ee{LC21:HenDet}
The map is Hamiltonian in the sense that it preserves areas in the
$[x,y]$ plane.


The {\HenonMap} is the simplest map that captures chaos that arises from
the smooth stretch \& fold dynamics of nonlinear {\PoincMap}s of flows
such as R\"ossler\rf{ross}.
Written as a  2nd-order inhomogeneous difference equation\rf{DulMei00},
\refeq{LC21eq2.1} takes the
{\em \henlatt} 3-term recurrence form, explicitly time-translation
and time-reversal invariant Euler–Lagrange equation
% \refeq{LC21:1dHenlatt}:
\beq
-\ssp_{\zeit+1} + {a}\,\ssp_{\zeit}^2 - \ssp_{\zeit-1} = 1
\,.
\ee{LC21:2-step}
Just as the kicked rotor (\ref{LC21PerViv2.1b},\ref{LC21PerViv2.1a}), the map can
be interpreted as a kicked driven anaharmonic oscillator\rf{Heagy92},
with the nonlinear, cubic Biham-Wenzel\rf{afind} lattice site potential
\refeq{polynPotent}
\beq
V(\Xx,\Mm) = \sum_{\zeit\in\lattice}\left(
\frac{a}{3}\ssp_{\zeit}^3 -\Ssym{\zeit}\,\ssp_\zeit\right)
    \,,\qquad
        \Ssym{\zeit} = -1
\,,
\ee{LC21BWcubic}
giving rise to kicking pulse \refeq{LC21PerViv2.1a}.
    \PC{2021-12-10} {
Note the cubic Biham-Wenzel potential includes
the source term, as  \refeq{LC21:1dTempFT}.
Please recheck the signs!
    }
Devaney, Nitecki, Sterling and Meiss\rf{Devaney79,StMeiss98,SteDuMei99}
have shown that the Hamiltonian {\HenonMap} has a complete Smale
horseshoe for sufficiently large `stretching parameter' values
\beq
      a > 5.699310786700\cdots
\;.
\ee{LC21:SterlHen}
In numerical\rf{ChaosBook} and analytic\rf{EndGal06} calculations we fix
(arbitrarily) the stretching parameter value to $a=6$, in order to
guarantee that all $2^\cl{}$ periodic points  $\ssp=\flow{\cl{}}{\ssp}$
of the {\HenonMap} \refeq{LC21eq2.1} exist.
    \PC{2021-12-10} {
Explain that $\phi^3$ theory is correspond to the unimodal \henlatt, with
the Smale horseshoe repeller cleanly split into the left (negative) and
right (positive) lattice site field values.
This is very much like the temporal Bernoulli, in contrast to the
\templatt\ which has nontrivial pruning, see \reftab{tab:catMapN_n-s=3}.
    }

    \PC{2021-06-04}{
S. Aubry\rf{aub95ant}
{\em Anti-integrability in dynamical and variational problems}

The \eqva\ and \reqva\ of Frenkel-Kontorova models\rf{AuAb90,MraRin12}, widely studied in
literature, might be closely related to \henlatt\ and $\phi^4$ lattices.



D. G. Sterling\rf{SterlingThesis99}
much (undeservedly)
un-cited \HREF{https://www.proquest.com/docview/304508605} {PhD thesis},
Univ. of Colorado,
{\em Anti-Integrable Continuation and the Destruction of Chaos} has much
to teach us. He studies \emph{coupled {\HenonMap} lattices} in both
Hamiltonian and Lagrangian formulations; his definition seems pretty much
consistent with our \refeq{SVWhenSTlatt}, though he has a coupling
parameter $c$ used to make spatial couplings weak. The
``anti-integrable'' refers to our choice $a\geq6$, I believe - parameter
regimes in which all of the horseshoe orbits exists.

``Specifying the anti-integrable state for an orbit of a coupled map
lattice requires a multidimensional symbolic object which we call a
symbol tensor.''

His Figures 6.7, 6.18 are reminiscent of my pruning front.

Sterling and Meiss\rf{StMeiss98}
{\em Computing periodic orbits using the anti-integrable limit}

    } % end     \PC{2021-06-04}

% \PC{2020-05-31} {
Politi and Torcini\rf{PolTor92} note that
a problem in reconstructing the statistical properties of an
{\spt\ H{\'e}non} attractor
is ensuring that all \twots\  used are embedded into the inertial manifold.
For instance, in the single H{\'e}non map, one
of the two fixed points is isolated and it does not belong to the strange
attractor.

We resolve this problem by construction, all our solutions belong to the
non-wondering set.
%   }

%\section{PolTor92 Periodic orbits in coupled {H{\'e}non} maps}
%\label{sect:PolTor92}
%\item[2020-05-31 Predrag]
Politi and Torcini\rf{PolTor92} {\em Periodic
orbits in coupled {H{\'e}non} maps: {Lyapunov} and multifractal analysis}

They study \emph{\spt\ \Henon}, a (1+1)-spacetime lattice of
\Henon\ maps orbits which are periodic both in space and time.

Their numerical method is an extension of Biham and Wenzel\rf{afind} for
the single \Henon\ map, with symbols $\Ssym{n\zeit}$ in $\A=\{0,1\}$. Any
fixed point in fictitious time corresponds to a spatio-temporal cycle
$\BravCell{\speriod{}}{\period{}}{\tilt{}}$.


\section{A {$\phi^4$} field theory}
\label{s:phi4latt}

If a potential that is bounded from below is needed to make sense of the
probabilistic interpretation of the configuration weight
\refeq{ProbConf}, or a symmetry forbids the odd-power potentials such as
\refeq{LC21BWcubic}, one starts instead with a quartic potential
$\sim{g}\ssp_{\zeit}^4$, leading to the `{$\phi^4$} lattice field theory'
\refeq{LC21:1dPhi4}. As $\phi^4$ example adds little to understanding
over what is learned from \henlatt, we will not discus is further in this
paper.
    \PC{2021-12-10} {
No! Explain that $\phi^4$ theory is the bimodal extension of
what we had learned for the unimodal \henlatt.

The \HenonMap/$\phi^3$ approaches should be safe for multimodal maps with
complete repelling sets, and it should work for finite-grammar Smale
horseshoe repellers.
Smale's original horseshoe\rf{smale}, his fig.~1 was unimodal, but he
also explicitly gives our $\phi^4$ bimodal repeller, his fig.~5.
    }

    \PC{2021-10-13}{RECHECK,
maybe applies to \catlatt: Equilibria or steady solutions of
Frenkel-Kontorova lattices, the smooth function $V:\reals\to\reals$ is a
periodic onsite potential, $V[\ssp+1]=V[\ssp]$ for all $\xi\in\reals$.

%\PC{2019-06-26}{
%Mramor and Rink
The $d$-dimensional Frenkel-Kontorova Hamiltonian lattice differential
equation\rf{AuAb90,MraRin12}
\beq
\frac{d^2 \ssp_i}{dt^2} + V'[\ssp_i] - \Box\,\ssp_i
    = 0 \ \mbox{for all} \ i\in\mathbb{Z}^d.
\ee{LC21FKHam}
describes the motion of particles under the
competing influence of an onsite periodic potential field and nearest
neighbor attraction.

the goal is to find a
$d$-dimensional ``lattice configuration''
(for us, {\lattstate})
$x:\integers^d\to \reals$ that satisfies
\beq
V'[\ssp_i] - \Box\,\ssp_i = 0 \  \ \mbox{for all} \ i\in \mathbb{Z}^d
\,.
\ee{FKeq} %\ref{RR} in {MraRin12}
% \PC{2019-06-26}{
Eq.~\refeq{FKeq}
is relevant for statistical mechanics\rf{MraRin12}, because it is
related to Eq.~\refeq(FKeq) describes its stationary solutions.
%  }
    }


\section{Computing {\lattstate}s for nonlinear theories}
\label{s:nonlinLattStates}

Unlike the {temporal Bernoulli} \refeq{LC21:1dBernLatt} and the
{\templatt} \refeq{LC21:1dTemplatt}, for which the {\lattstate} fixed
point condition \refeq{LC21eqMotion}
% {tempCatFixPoint}
is linear and easily solved, for
nonlinear lattice field theories the {\lattstate}s are roots of
polynomials of arbitrarily high order. While Gallas and collaborators%
\rf{EndGal01,EndGal02,EG05,EG05a,EndGal06,Gallas18,Gallas20,Gallas20a}
have developed a powerful theory that yields {\HenonMap} {\po}s in
analytic form, it would be unrealistic to demand such explicit solutions for
general field theories on multi-dimensional lattices. We take a
pragmatic, numerical route, and search for the fixed-point solutions
\refeq{LC21eqMotion}
starting with the deviation of an approximate trajectory from the 3-term
recurrence \refeq{LC21:1dTempFT}, given by the lattice deviation vector
\beq
v_{\zeit} = -\ssp_{\zeit+1} + (V'(\ssp_{\zeit})-\Ssym{\zeit}) - \ssp_{\zeit-1}
\,,
\ee{LC21BWdeviate}
and minimizing this error term by any convenient variational or
optimization method, perhaps in conjunction with a high-dimensional
variant of the Newton method\rf{CvitLanCrete02,lanVar1}.
    \PC{2021-12-10} {
Form $(V'(\ssp_{\zeit})-\Ssym{\zeit})$ looks like the most convenient
definition of the ``$\Ssym{}$-centered" subregion $\pS_\Ssym{}$ potential,
applicable to both linear and nonlinear field theories?
    }



\subsection{Papers to refer to?}

Sim{\'o}\rf{Simo79} {\em On the {H{\'e}non-Pomeau} attractor}
is a very fine early paper. Cite it in \Henon\ remark.

Miguel, Sim{\'{o}} and Vieir\rf{MiSiVi13} {\em From the {H{\'{e}}non}
conservative map to the {Chirikov} standard map for large parameter
values} \CBlibrary{MiSiVi13}:

Endler and Gallas\rf{EG05}.
method resembles the methods
earlier employed for quadratic polynomials (and their Julia sets) by
Brown\rf{Brown81}
and Stephenson\rf{Stephen1992A}.

Brown gives cycles up to length 6 for the logistic map,
employing symmetric functions of periodic points.

Hitzl and Zele\rf{HitZe85}
study the of the {\HenonMap} for cycle lengths  up to period  6.
