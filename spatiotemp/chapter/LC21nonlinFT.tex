% siminos/reversal/LC21nonlinFT.tex      pdflatex LC21; bibtex LC21
% temporary: siminos/spatiotemp/chapter/LC21nonlinFT.tex
% $Author: predrag $ $Date: 2021-12-24 01:25:20 -0500 (Fri, 24 Dec 2021) $

% was \section{Nonlinear lattice field theories}
%\label{s:nonlinFT}

\section{A $\ssp^3$ field theory}
% was {{\Henlatt}}
\label{s:henlatt}

The `mod~1' in the definition of the `linear' kicked rotor, the cat map
\refeq{catMap}, makes cat map a highly nonlinear, discontinuous map. In
contrast, field theory action $\action[\Xx]$ is typically polynomial and
smooth. The simplest such nonlinear action turns out to correspond to the
paradigmatic dynamicist's model of a 2\dmn\ nonlinear dynamical system,
the {\HenonMap}\rf{henon}
\bea
    x_{\zeit+1} &=& 1-a\,x_{\zeit}^2 + b\,y_{\zeit}
        \continue
    y_{\zeit+1} &=& x_{\zeit}
\,.
\label{LC21eq2.1}
\eea
For the contraction parameter value $b=-1$ this is a Hamiltonian map.

The \emph{temporal evolution} \jacobianM\ for the $n$th iterate of the
Hamiltonian {\HenonMap} is the product of consecutive  one time-step
\jacobianMs
\beq
\jMps^\cl{}(x_0,y_0) =
\prod_{m=\cl{}-1}^{0}
            \left(\begin{array}{cc}
                -2a\,x_m & -1 \\
                         1 & 0
            \end{array}\right)
\,,\qquad x_m = \map^{m}_1 (x_0,y_0)
\,,
\ee{Henlatt-e_her}
where the successive 1-time step \jacobianMs\ are multiplied in the order
they are applied, as
$\jMps^\cl{}(x_0)=\jMps(x_{\cl{}-1})\cdots\jMps(x_0)$. So, once we have a
{\HenonMap} {\po}, we also have its Floquet
(monodromy) matrix. When $\jMps^\cl{}$ is
hyperbolic, only the expanding
eigen\-value $\ExpaEig_1=1/\ExpaEig_2$ needs to be determined, as the
determinant of the {\Henon} 1-time step \jacobianM\ is unity,
\beq
\det\jMps = \ExpaEig_1 \ExpaEig_2 = 1
\,.
\ee{LC21:HenDet}
The map is Hamiltonian in the sense that it preserves areas in the
$[x,y]$ plane.


The {\HenonMap} is the simplest map that captures chaos that arises from
the smooth stretch \& fold dynamics of nonlinear {\PoincMap}s of flows
such as R\"ossler\rf{ross}.
Written as a  2nd-order inhomogeneous difference equation\rf{DulMei00},
\refeq{LC21eq2.1} takes the
{\em \henlatt} 3-term recurrence form, explicitly time-translation
and time-reversal invariant Euler–Lagrange equation
% \refeq{LC21:1dHenlatt}:
\beq
-\ssp_{\zeit+1} + {a}\,\ssp_{\zeit}^2 - \ssp_{\zeit-1} = 1
\,.
\ee{LC21:2-step}
Just as the kicked rotor (\ref{LC21PerViv2.1b},\ref{LC21PerViv2.1a}), the map can
be interpreted as a kicked driven anaharmonic oscillator\rf{Heagy92},
with the nonlinear, cubic Biham-Wenzel\rf{afind} lattice site potential
\refeq{polynPotent}
\beq
V(\ssp_\zeit,\Ssym{\zeit}) =  \frac{a}{3}\ssp_{\zeit}^3
                            - \ssp_{\zeit}^2 -\Ssym{\zeit}\,\ssp_\zeit
    \,,\qquad
        \Ssym{\zeit} = -1
\,,
\ee{LC21BWcubic}
giving rise to kicking pulse \refeq{LC21PerViv2.1a}, so we
refer to this field theory as $\ssp^3$ theory.

Thus the lattice site field values of this $\phi^3$ theory are in
one-to-one correspondence to the unimodal \HenonMap\
Smale horseshoe repeller, cleanly split into the `left', positive stretching and
`right', negative stretching lattice site field values.
Devaney, Nitecki, Sterling and Meiss\rf{Devaney79,StMeiss98,SteDuMei99}
have shown that the Hamiltonian {\HenonMap} has a complete Smale
horseshoe for sufficiently large `stretching parameter' values
\beq
      a > 5.699310786700\cdots
\;.
\ee{LC21:SterlHen}
In numerical\rf{ChaosBook} and analytic\rf{EndGal06} calculations we fix
(arbitrarily) the stretching parameter value to $a=6$, in order to
guarantee that all $2^\cl{}$ periodic points  $\ssp=\flow{\cl{}}{\ssp}$
of the {\HenonMap} \refeq{LC21eq2.1} exist, see \reftab{tab:LC21HamHenon}.
The symbolic dynamics is as simple as the temporal Bernoulli, in contrast to the
\templatt\ which has nontrivial pruning, see \reftab{tab:lattstateCountCat}.


\section{A {$\phi^4$} field theory}
\label{s:phi4latt}

If a potential that is bounded from below is needed to make sense of the
probabilistic interpretation of the configuration weight
\refeq{ProbConf}, or a symmetry forbids the odd-power potentials such as
\refeq{LC21BWcubic}, one starts instead with a quartic potential
\refeq{polynPotent}
\beq
V(\ssp_\zeit,\Ssym{\zeit}) =  \frac{g}{4}\ssp_{\zeit}^4
                            - \ssp_{\zeit}^2 -\Ssym{\zeit}\,\ssp_\zeit
\,,
\ee{LC21phi4pot}
leading to an example of the `{$\phi^4$} lattice field theory'.
\refeq{LC21:1dPhi4},
\bea
- \ssp_{\zeit+1} + {g}\,\ssp_{\zeit}^3 - \ssp_{\zeit-1}
    &=&
\Ssym{\zeit}
\,.
\label{LC21:1dPhi4a}
\eea
Topology of the \statesp\ of $\phi^4$ theory is
very much like
what we had learned for the unimodal \HenonMap\ $\phi^3$ theory,
except that the repeller set is now bimodal. As long as coupling $g$
is sufficiently large, the repeller is a full 3-letter shift.
Indeed, while Smale's first horseshoe\rf{smale}, his fig.~1, was unimodal, he
also sketched the $\phi^4$ bimodal repeller, his fig.~5.

As $\phi^4$ example adds little to understanding
over what is learned from \henlatt, we will not discus is further in this
paper.


\subsection{Computing {\lattstate}s for nonlinear theories}
\label{s:nonlinLattStates}

Unlike the {temporal Bernoulli} \refeq{LC21:1dBernLatt} and the
{\templatt} \refeq{LC21:1dTemplatt}, for which the {\lattstate} fixed
point condition \refeq{LC21eqMotion}
% {tempCatFixPoint}
is linear and easily solved, for
nonlinear lattice field theories the {\lattstate}s are roots of
polynomials of arbitrarily high order. While Gallas and collaborators%
\rf{EndGal01,EndGal02,EG05,EG05a,EndGal06,Gallas18,Gallas20,Gallas20a}
have developed a powerful theory that yields {\HenonMap} {\po}s in
analytic form, it would be unrealistic to demand such explicit solutions for
general field theories on multi-dimensional lattices. We take a
pragmatic, numerical route, and search for the fixed-point solutions
\refeq{LC21eqMotion}
starting with the deviation of an approximate trajectory from the 3-term
recurrence \refeq{LC21:1dTempFT}, given by the lattice deviation vector
\beq
v_{\zeit} = -\ssp_{\zeit+1} + V'(\ssp_{\zeit},\Ssym{\zeit}) - \ssp_{\zeit-1}
\,,
\ee{LC21BWdeviate}
and minimizing this error term by any convenient variational or
optimization method, perhaps in conjunction with a high-dimensional
variant of the Newton method\rf{CvitLanCrete02,lanVar1,orbithunter}.

\subsection{Remarks}

Equilibria or steady solutions of
the $d$-dimensional Frenkel-Kontorova Hamiltonian lattice differential
equation\rf{AuAb90,MraRin12}
\beq
\frac{d^2 \ssp_i}{dt^2} + V'(\ssp_i) - \Box\,\ssp_i
    = 0 \ \mbox{for all} \ i\in\mathbb{Z}^d.
\ee{LC21FKHam}
is an example of what we here call a `nonlinear lattice field theory'.
The model describes the motion of particles under the
competing influence of an onsite periodic potential field and nearest
neighbor attraction.

In 1992 Politi and Torcini\rf{PolTor92} had
studied \emph{\spt\ \Henon}, a (1+1)-spacetime lattice of
orbits periodic both in space and time, using in their
 numerical work an extension of Biham and Wenzel\rf{afind} for
the single \Henon\ map.

Much of the work on 1\dmn\ \henlatt/$\ssp^3$ and $\ssp^4$ field theory
has already been carried out within the framework of {\em
anti-integrability}\rf{AuAb90,aub95ant,StMeiss98}.
The multi\dmn\ \spt\ $\ssp^3$ field theory was pioneered by
Sterling\rf{SterlingThesis99} in his excellent but unpublished
\HREF{https://www.proquest.com/docview/304508605} {PhD thesis}. He
studies {coupled {\HenonMap} lattices} in both Hamiltonian and Lagrangian
formulations, introduces our multidimensional symbolic object $\Mm$ which
he calls a `symbol tensor', and focuses on the `destruction of chaos' as
one lowers the stretching parameter $a$, a much harder problem than what
we address here. Luckily for us, the strong coupling, strong local
stretching field theories' deterministic solutions live on horseshoes,
safely away from the regions of intermediate stretches, where dragons
live.


 \PC{2020-05-31} {
\subsection{Papers to refer to?}

Sim{\'o}\rf{Simo79} {\em On the {H{\'e}non-Pomeau} attractor}
is a very fine early paper. Cite it in \Henon\ remark.

Miguel, Sim{\'{o}} and Vieir\rf{MiSiVi13} {\em From the {H{\'{e}}non}
conservative map to the {Chirikov} standard map for large parameter
values} \CBlibrary{MiSiVi13}:

Endler and Gallas\rf{EG05}.
method resembles the methods
earlier employed for quadratic polynomials (and their Julia sets) by
Brown\rf{Brown81}
and Stephenson\rf{Stephen1992A}.

Brown gives cycles up to length 6 for the logistic map,
employing symmetric functions of periodic points.

Hitzl and Zele\rf{HitZe85}
study the of the {\HenonMap} for cycle lengths  up to period  6.
   }
