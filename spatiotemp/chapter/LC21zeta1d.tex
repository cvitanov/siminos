% siminos/reversal/zeta1d.tex      pdflatex LC21; bibtex LC21
% temporary: siminos/spatiotemp/chapter/LC21zeta1d.tex
% $Author: predrag $ $Date: 2021-12-24 01:25:20 -0500 (Fri, 24 Dec 2021) $

\section{\Po\ theory}
\label{s:zeta1D}

% clipped from Sterling\rf{SterlingThesis99}
Recurrent phenomena, the
simplest of which is periodic motion, are particularly interesting and
practical objects of study. Scientists have long been captivated by the
near periodic motions of the planets. Among them, \Poincare\ was the first
to truly recognize the importance of periodic solutions in understanding
more complex dynamical behavior.

The problem of counting periodic orbits and how they partition
the \statesp\ partitions is among the
first problems that one needs to address%
\rf{ArtMaz65,ChAnPi85,brucks,Lutzky88,Lutzky93,BLMS91,XH94,
CheLou97,BriPer05,CBcount}.
For
the quadratic map it was addressed in 1950's by
\HREF{https://en.wikipedia.org/wiki/Pekka_Myrberg} {Myrberg}, in perhaps
the first application of computers to
dynamics\rf{Myrberg58a,Myrberg58b,Myrberg59,Myrberg62,Myrberg63}.

The problem of counting orbits for the
{\HenonMap} was also addressed very early, in a pioneering work by
Sim{\'o}\rf{Simo79} using an approach centered in the strange attractor
creation/destruction.

MacKay\rf{Bmack93} 1982 PhD thesis, published as
{\em Renormalisation in Area-preserving Maps} has a chapter on
reversible maps. Do cite in our paper(s).
MacKay had these numbers already
listed in Table 1.2.3.5.1 of his 1982 PhD thesis\rf{Bmack93}.

This is the `\po\ theory'. And if you don't know,
\HREF{https://www.youtube.com/watch?v=_JZom_gVfuw} {now you know}.

%%%%%%%%%%%%%%%%%%%%%%%%%%%%%%%%%%%%%%%%%%%%%%%%%%%%
\begin{table}
\begin{tabular}{c|rrrrr|rrrrr|rrrrr}
$\cl{}$ &  1 &  2 &  3 &  4 &  5 &
       6 &  7 &  8 &  9 & 10 &
      11 \\%& 12 & 13 & 14 & 15 \\
\hline
$N_\cl{}$ &  2 &  4 &  8 & 16 &  32 &
       64 &  128 &  256 & 512 & 1024 &
      . % \\%& 12 & 13 & 14 & 15 \\
             \rule[-1ex]{0ex}{3.5ex} \\
$M_\cl{}$ &   2 &   1 &   2 &  3 &  6 &
         9 & 18 &  30 & 56  & 99 &
       186 &  %% &     &     &
\end{tabular}
\bigskip
\caption{\label{tab:LC21HamHenon}
{\Lattstate}s and %{prime}
orbit counts for the ${a}=6$ {\HenonMap}.
}
\end{table}
%%%%%%%%%%%%%%%%%%%%%%%%%%%%%%%%%%%%%%%%%%%%%%%%%%%%
%

\subsection{\Po\ theory}
\label{s:PoThe}

How come that a `$\Det$' in \refeq{detBern0} counts {\lattstate}s?

For a general, nonlinear fixed point condition $F[\Xx]=0$, expansion
\refeq{LnDet=TrLn2} in terms of traces is a cycle
expansion\rf{inv,AACI,ChaosBook}, with support on \emph{periodic orbits}.
Ozorio de Almeida and Hannay\rf{OzoHan84} were the first to relate the
number of periodic points to a \JacobianM\ generated volume; in 1984 they
used such relation as an illustration of their `principle of uniformity':
``periodic points of an ergodic system, counted with their natural
weighting, are uniformly dense in phase space.'' In \po\
theory\rf{inv,CBgetused} this principle is stated as a
\HREF{http://chaosbook.org/chapters/ChaosBook.pdf\#section.27.4} {flow
conservation} sum rule, where the sum is over all {\lattstate}s $\Mm$ of
period $\cl{}$,
\beq
\sum_{|\Mm|=\cl{}}
    \frac{1}{|\det (\id - \jMat_\Mm)|}
    \;= 1
\,,
\label{H-OdeA_mapsOrb}
\eeq
or, by Hill's formula \refeq{detDet},
\beq
\sum_{|\Mm|=\cl{}}
%\sum_{\ssp_i{\in\mbox{\footnotesize Fix}\map^{\cl{}}}}
    \frac{1}{|\Det\jMorb_\Mm|}
    \;=1
\,.
\label{Det(jMorb)eights}
\eeq
For the Bernoulli system the `natural weighting' takes a particularly
simple form, as the {\HillDet} of the {\jacobianOrb} is the same for all
periodic points of period $\cl{}$, $\Det\jMorb_\Mm=\Det\jMorb$, whose
number is thus given by \refeq{detBern0}.
For example, the sum over the $\cl{}=2$ {\lattstate}s is,
\beq
      \frac{1}{|\Det{\color{green}\jMorb_{00}}|}
   +    \frac{1}{|\Det{\color{red}\jMorb_{01}}|}
   + \frac{1}{|\Det{\color{yellow}\jMorb_{10}}|}
    =1
\,,
\ee{H-OdeA_mapsOrb2}
see \reffig{fig:BernCyc2Jacob}\,(b).
Furthermore, for any piece-wise
linear system all curvature corrections\rf{CBcount} for orbits of
periods $k>\cl{}$ vanish, leading to explicit {\lattstate}-counting
formulas of kind reported in this paper.

In the case of temporal Bernoulli or \templatt, the hyperbolicity is the same
everywhere and does not depend on a particular solution $\Xx_p$, counting
\po s is all that is needed to solve a cat-map dynamical system
completely; once \po s are counted, all {\cycForm s}\rf{CBtrace} follow.

A zeta function relates the totality of admissible states to the orbits
generated by all symmetries of a given theory.

\subsection{\Tzeta}
\label{s:bernZeta}
% moved here from siminos/reversal/Bernoulli.tex 2021-12-20

Now that we have the numbers of {\lattstate}s $N_{\cl{}}$ for any
period $\cl{}$, we can combine them all into a single generating function by
substituting $N_{\cl{}}$ into the {\em topological} or {\em Artin-Mazur}
zeta func\-tion\rf{ArtMaz65,CBcount},
\index{topological!zeta function}
\index{zeta function!topological}
\index{Artin-Mazur zeta function}
\index{zeta function!Artin-Mazur}
    \PC{2020-02-16}{
Do I need to derive this, rather than refer to ChaosBook?
    }
\beq
\zetatop(z) =
     \exp\left(-\sum_{\cl{}=1}^\infty
\frac{z^\cl{}}{\cl{}} N_\cl{}
         \right)
\,,
\ee{topoZeta}
which, when expanded as a Taylor series in $z$, is built from
\emph{primitive} (or \emph{prime}), \ie, non-repeating {\lattstate}s\rf{inv}. Conversely, given the \tzeta, the number of periodic
points of period $\cl{}$ is given by the logarithmic derivative of the
{\tzeta} (see
\HREF{http://chaosbook.org/chapters/ChaosBook.pdf\#section.18.7}
{ChaosBook}\rf{CBcount})
\bea
\sum_{\cl{}=1}N_\cl{} z^\cl{}
    &=& - \,\frac{1}{\zetatop}\,z\frac {d}{dz} (\zetatop)
\,.
\label{zetatop-N}
\eea

For a Bernoulli system \refeq{detBern2},
\bea
\zetatop(z)
 &=&  \exp \left(-\sum_{\cl{}=1}^\infty
\frac{z^\cl{}}{\cl{}} ({s}^\cl{} - 1)
         \right)
\,=\,
\exp \left[\ln(1 -  {s}z) - \ln(1 - z) \right]
\continue
 &=&
\frac{1 -  {s}z}{1 - z}
\,.
\label{BernZeta}
\eea
The numerator $(1 - {s}z)$ says that a Bernoulli system is a full
shift\rf{CBcount}: there are $s$ fundamental {\lattstate}s, in this case
fixed points $\{\ssp_0,\ssp_1,\cdots,\ssp_{s-1}\}$, and every other
{\lattstate} is built from their concatenations and repeats. The
denominator $(1 - z)$ compensates for the single overcounted {\lattstate}, the fixed point $\ssp_{{s}-1}=\ssp_{0}$ $(\mbox{mod}\;1)$ of
\reffig{fig:BernPart} and its repeats.

For non-integer values of the stretching factor ${s}=\beta$, the map
\refeq{n-tuplingMap} is called the `$\beta$-transformation'. For its
{\tzeta} see \refref{FlLaPo94}.

\subsection{Counting {temporal Bernoulli} prime \po s}
\label{s:bernPrime}

Substituting the Bernoulli map \tzeta\ \refeq{BernZeta}
into \refeq{zetatop-N}
we obtain
% ------------------- siminos/mathematica/ ----------
% CatMaptopZeta.nb                                    2020-01-18
% Bernoulli map periodic points counting for CL18.tex
\bea
\sum_{n=1}N_n z^n
    &=&
 z+3 z^2+7 z^3+15 z^4+31 z^5+63 z^6+127 z^7
    \ceq
+255 z^8+511 z^9+ 1023z^{10} +2047 z^{11}
\cdots
\,,
\label{bernN_n-s=2}
\eea
in agreement with the {\lattstate}s count \refeq{detBern2}.
The number of \emph{prime} cycles of period $\cl{}$ is given recursively by
subtracting repeats of shorter prime cycles\rf{CBcount},
\beq
M_n\,=\,\frac{1}{n}\left( N_n - \sum _{d|n}^{d<n}\,d M_d \right)
\,,
\ee{primeCount}
where $d$'s are all divisors of $n$, hence
% ------------------- siminos/mathematica/ ----------
% CatMaptopZeta.nb                                    2020-01-18
% Bernoulli map periodic points counting for CL18.tex
\bea
\sum_{n=1}M_n z^n
    &=&
 z+  z^2+2 z^3+3 z^4+6 z^5+9 z^6+18 z^7
    \ceq
+30 z^8+56 z^9+99 z^{10} +186 z^{11}
\cdots
\,,
\label{bernM_n-s=2}
\eea
in agreement with the usual numbers of binary symbolic dynamics prime
cycles\rf{CBcount}.

%%%%%%%%%%%%%%%%%%%%%%%%%%%%%%%%%%%%%%%%%%%%%%%%
\subsection{\Tzeta}
\label{s:tempCatZeta}

Substituting the numbers of {\lattstate}s $N_{\cl{}}$
\refeq{1stepDiffSolu} into the {\em {\tzeta}} \refeq{topoZeta},
Isola\rf{Isola90} obtains
\index{topological!zeta function}
\index{zeta function!topological}
\index{Artin-Mazur zeta function}
\index{zeta function!Artin-Mazur}
\bea
\zetatop(z)
 &=& \exp \left(-\sum_{\cl{}=1}^\infty
\frac{z^\cl{}}{\cl{}} N_\cl{}
         \right)
 =  \exp \left(-\sum_{\cl{}=1}^\infty
\frac{z^\cl{}}{\cl{}} (\ExpaEig^\cl{} + \ExpaEig^{-\cl{}} - 2)
         \right)
\continue
 &=&
\exp \left[\ln(1 - z \ExpaEig) + \ln(1 - z \ExpaEig^{-1}) - 2 \ln(1 - z) \right]
\continue
 &=&
\frac{(1 - z \ExpaEig)(1 - z \ExpaEig^{-1})}{(1 - z)^2}
 =
\frac{1 - s z + z^2}
     {(1 - z)^2}
\,.
\label{Isola90-13}
\eea
\Tzeta s count {\em prime
orbits}, \ie, time invariant \emph{sets} of equivalent {\lattstate}s
related by translations (cyclic permutations), and other
symmetries\rf{CBcount}.

Conversely, given the \tzeta, the generating function for the number of
temporal {\lattstate}s of period $\cl{}$ is given by the logarithmic
derivative of the {\tzeta} \refeq{zetatop-N},
\bea
\sum_{{n}=0}^\infty N_{n} z^{n}
    & = & \frac{2-{s}z}{1 - s z + z^2}-\frac{2}{1 - z}
    \continue
& = & (s-2)\left[z + ({s}+2) z^2 + ({s}+1)^2 z^3 \right.
    \ceq
      \left.\qquad\quad
      +\,({s}+2)\,{s}^2 z^4 + (s^2+ s-1)^2 z^5  +  \cdots\right]
\,,
\label{1stChebGenF}
\eea
which is indeed the generating function for $T_{\cl{}}(s/2)$, the
\emph{Chebyshev polynomial of the first kind} \refeq{POsChebyshev}.

Why Chebyshev? Essentially because $T_k(x)$ are also
defined by a 3-term recurrence:
\bea
T_0(x) &=& 1\,,\quad
T_1(x)=x
    \continue
T_k(x)  &=&  2x T_{k-1}(x) - T_{k-2}(x)
\quad \mbox{for } k \geq 2
\,,
\label{Rangarajan17:TkRecurr}
\eea


The
numbers of {periodic points} and \emph{prime} cycles $M_n$ (given by
\refeq{primeCount}) are tabulated in \reftab{tab:catMapN_n-s=3}.

%%%%%%%%%%%%%%%%%%%%%%%%%%%%%%%%%%%%%%%%%%%%%%%%%%%%
\begin{table}
\begin{tabular}{c|rrrrr|rrrrr|rrrrr}
$n$ &  1 &  2 &  3 &  4 &  5 &
       6 &  7 &  8 &  9 & 10 &
      11 \\%& 12 & 13 & 14 & 15 \\
\hline
$N_n$ &   1 &   5 &  16 &  45 &  121 &
        320 & 841 & 2205 &5776 &15125&
       39601& %   &      &     & 1364
             \rule[-1ex]{0ex}{3.5ex} \\
$M_n$ &   1 &   2 &   5 &  10 &   24 &
         50 & 120 & 270 & 640 & 1500 &
       3600 &  %% &     &     &
\end{tabular}
\bigskip
\caption{\label{tab:catMapN_n-s=3}
Lattice states and {prime} orbit counts for the ${s}=3$ cat map
(Bird and Vivaldi\rf{BirViv}).
        }
\end{table}
%%%%%%%%%%%%%%%%%%%%%%%%%%%%%%%%%%%%%%%%%%%%%%%%%%%%
