% siminos/spatiotemp/chapter/CML.tex
% $Author: predrag $ $Date: 2021-12-08 16:28:14 -0500 (Wed, 08 Dec 2021) $

% called by siminos/spatiotemp/chapter/catMapLatt.tex

\section{Coupled map lattices}
\label{s:CML}

Diffusive coupled map lattices (CML) were introduced by
Kaneko\rf{Kaneko83,Kaneko84}:
\beq
x_{n, t+1}=g(x_{n,t}) + \frac{\epsilon}{2}
            [
            g(x_{n-1,t}) - 2g(x_{n,t}) + g(x_{n+1,t})
            ]
          = (1+\epsilon\,\Box) g(x_{n,t})
\ee{KanekoCML}
where the individual site dynamical system $g(x)$ is a 1D map such as the
logistic map.


In the discretization of a spacetime field $q(x,t)$ on lattice points
$(x_n,t_j)$, the field is replaced by its lattice point value
$q_{n,j}= q(x_n,t_j)$. For a Hamiltonian set of fields we also have
$p_{n,j}= p(x_n,t_j)$.
In the {\catlatt}, a cat map at each  periodic lattice site
is coupled diffusively to its nearest neighbors:
\bea
q_{n, j+1}&=&p_{n,j}+(s-3)q_{n,j} - (q_{n+1,j} - 2q_{n,j} + q_{n-1,j}) - {m}^q_{n,j+1}
\continue
p_{n,j+1}&=&p_{n,j} + (s-4) q_{n,j} - (q_{n+1,j} - 2q_{n,j} + q_{n-1,j}) - {m}^p_{n,j+1}
\label{HamEqs}
\eea
The spatiotemporal symbols follow from the Newtonian equations
in $d$ spatiotemporal dimensions
\bea
&& (q_{n,j+1} -2 q_{nj} + q_{n,j-1})
         + (q_{n+1,j}  -2 q_{nj} + q_{n-1, j}) -  (s-4) q_{nj}
= m_{nj}
        \continue
&& \left(-\Box + {\mu}^2\mathsf{1} \right)\, \mathsf{q}
= -\mathsf{m}
\,.
\label{catLattPC}
\eea
The $\Box + 2d\mathsf{1}$ part is the standard statistical mechanics
diffusive inverse propagator that counts paths on a $d$\dmn\
lattice\rf{DasBuch},
${\mu}^2=d(s-2)$  is the Yukawa mass parameter \refeq{catlattMass},
and $-s\mathsf{1}$ is the on-site cat map dynamics, described by the
stretching parameter ${s}$.
For $d=1$ lattice, $s=3$ is the usual Arnol'd cat map.

\bigskip\bigskip

\begin{description}

\PCpost{2018-12-15}{
Frahm and Shepelyansky\rf{FraShe18} {\em Small world of {Ulam} networks for
chaotic {Hamiltonian} dynamics}, and the related
\HREF{https://scholar.google.com/citations?user=n_XVZ1EAAAAJ&hl=en&oi=sra}
{Shepelyansky} work is of potential interest.

``Ulam method'' replaces discrete dynamics by an Ulam approximate\rf{FraShe10} of
the {\FPoper} (UPFO). The Ulam method produces directed
``Ulam networks'' with weighted probability transitions between nodes
corresponding to phase-space cells.
From a physical point of view the finite cell
size of UPFO corresponds to the introduction of a finite noise with amplitude
given by a discretization cell size. Ulam networks have small-world properties,
meaning that almost any two nodes are indirectly connected by a small number of
links.

They show that the Ulam method applied to symplectic maps generates Ulam networks
which belong to the class of small-world networks. They analyze the small-world
examples of the Chirikov standard map and the Arnold cat map, showing that the
number of degrees of separation grows logarithmically with the network size for
the regime of strong chaos, due to the instability of chaotic dynamics. The
presence of stability islands leads to an algebraic growth with the network size.

The usual case of the cat map corresponds to $L=1$. The map on a torus of longer
integer size $L>1$ generates a diffusive dynamics\rf{ErmShe12}. For $L\gg1$ the
diffusive process for the probability density is described by the Fokker-Planck
equation.

The time scales related with the degrees of separation and the relaxation times
of the {\FPoper} have different behaviors.
The largest relaxation times remain size independent in the case of a diffusive
process, like for the Arnold cat map on a long torus.

In the Appendix they show that the exact linear form of the cat map allows for
very efficient and direct \emph{ exact Ulam network} network size $10^8$
computation of the transition probabilities needed for the UPFO. Shepelyansky
tends to omit boring formulas, so I see no stability multipliers which are so
important in our computations.

In UPFO discretizations of the standard map they use the Arnoldi method. The main
idea of the Arnoldi method is to construct a subspace of ``modest'', but not too
small, dimension (the Arnoldi-dimension) generated by the vectors that span a
Krylov space); the Arnoldi method in \refref{FraShe10} is quite interesting.

The construction of the Ulam networks is (verbally?) described in
\refref{FraShe10}, but I have not understood it. As graphs are directed (?),
there is probably no Laplacian. There might be a related undirected network
model, with a graph Laplacian \refeq{gaphLapl}. In that case a Lagrangian
formulation (in terms of graph Laplacians) might be a more powerful formulation
than their Hamiltonian one. ``Arrow of time'' is perhaps encoded by the
orientations of the links in a directed complex network.
    }

\PCpost{2018-12-15}{                                            \toCB
Ermann and Shepelyansky\rf{ErmShe12}
{\em The {Arnold} cat map, the {Ulam} method and time reversal}
show that the ``Ulam method'' coarse-graining leads to irreversibility.
    }

\item[2020-02-19 Predrag]
In
Berenstein and Garc{\'i}a-Garc{\'i}a15\rf{BerGar15}
{\em A universal quantum constraints on the butterfly effect},
\arXiv{1510.08870},
cat maps were generalized to products of such vector spaces that can be
put on a lattice, with a variables at each site and variables at
different sites commuting with each other, for nearest neighbors on a
lattice in any dimension. As they write, ``This generates a system with
nearest neighbor hopping and local scrambling.'' They write down a
periodic (circulant) banded matrix, so the  eigenvalues have a band
structure similar to a periodic potential with nearest neighbor hopping.
The evolve forward in time, \ie, their is a quantized Hamiltonian
formulation.

Berenstein\rf{Berenstein18}
{\em A toy model for time evolving QFT on a lattice with controllable chaos},
\arXiv{1803.02396} from UC Santa Barbara.
is perhaps a precursor to Gutkin and our \catlatt.

He discusses two points of view on how the Lyapunov exponents appear in
real time correlation functions in quantum field theory and will them
compute them in the case of the cat map dynamics. The Kubo's formula
point if view is in semiclassical physics, and the second point of view
is statistical. They both amount to different ways of making a quantity
with an indefinite sign positive.

He considers 1\dmn\ spatial lattice, and caries out computations on a
spatial lattice with only two sites. He composes a local cat map at each
site with a nearest neighbor entangler and after the system is
constructed one iterates the automorphism. The system will also be
determined by a $[2m\times 2m]$ matrix. Each $[2\times 2]$ block on the
diagonal represents $(P_i;Q_i)$. The local cat map acts on each of these
as a $[2\times 2]$ matrix, and the nearest neighbor entangler is a matrix
that, at least for a lattice on a line, is near the diagonal giving rise
to a banded matrix. The eigenvalues of this bigger matrix are the
Lyapunov exponent of the system.

Iterating over a general M produces a cat map dynamics on the Q, and the
`inverse' cat dynamics on the P. The dynamics on P is actually built from
the inverse transpose, but that has the same eigenvalues as the inverse
of M.

As noted in \refref{BerGar15}, models with nearest neighbor properties
also mimic the Lieb-Robinson bound [18] for propagation of information
and thus can in principle serve as toy models for relativistic field
theories (they have the equivalent of a speed of light).

He sets up a one dimensional spatial lattice, with nearest neighbor
entanglers whose dynamics can be encoded by an 'Dirichlet' upper
triangular matrix, his eq.~(64) and sketch (65), with eigenvalues 1, and
thus not chaotic. However, with periodic \bcs\ it is chaotic. The paper
here falls short of Gutkin and Osipov\rf{GutOsi15}.

Berenstein and Teixeira\rf{Berenstein18} {\em Maximally entangling states
and dynamics in one dimensional nearest neighbor {Floquet} systems},
\arXiv{1901.02944},
describe conditions for generating entanglement between two regions at
the optimal rate in a class of one-dimensional quantum circuits with
Floquet dynamics. I do not get it, but it does cite Prosen\rf{BeKoPr19-1}
see {\bf 2019-11-18 Boris} below.

I do not think we need to cite him, but should send him links to our
papers: \href{https://www.physics.ucsb.edu/people/david-berenstein}
{David Berenstein}

\item[2020-05-31 Predrag]
Houlrik\rf{Houlrik92} {\em Periodic orbits in a two-variable coupled map}
computes \po s in $1+1$ spacetime CML for a linear map composed of two
coupled Chat{\'e}-Manneville maps\rf{CM87tran} [tent map + linear
branch] (see \refeq{KanekoCML})
\beq
x_{n, t+1}=f(x_{n,t}) + \frac{\epsilon}{2}
            [
            f(x_{n-1,t}) - 2f(x_{n,t}) + f(x_{n+1,t})
            ]
%          = (1+\epsilon\,\Box) f(x_{n,t})
\ee{Houlrik92(1)}
what we call the $\BravCell{2}{\cl{}}{0}$ family \po s,
using symbolic dynamics \brick s $\Mm$ defined as the direct product of
the single‐map symbols $\A=\{0,1,2\}$. He credits Bunimovich and
Sinai\rf{BunSin88} with introducing the ($D$+1)\dmn\ spatiotemporal
symbolic dynamics.

The $[2\!\times\!2]$ matrix
\beq
A =
 \left(\begin{array}{cc}
  1-\epsilon & \epsilon \\
  \epsilon   & 1-\epsilon
 \end{array} \right)
  = (1-\epsilon)\unit + \epsilon({\shift}+{\shift}^{-1})
\ee{Houlrik92(3)}
and sources
\beq
B(\Mm) = \sum_{\cl{}-1}^{k=0}
J(\Ssym{\cl{}-1})\cdots J(\Ssym{k+1}) A {\bf{b}}
\ee{Houlrik92(10)}

He finds the $\BravCell{2}{\cl{}}{0}$ \po s by solving
\beq
(1-J(\Mm))\,\Xx=B(\Mm)
\ee{Houlrik92(9)}
The fixed
point condition \refeq{Houlrik92(9)} has a \po\ solution
\beq
\Xx=\frac{1}{1-J(\Mm)}\,B(\Mm)
\ee{Houlrik92(12)}
for each admissible brick \Mm, where this needs still to be rewritten in
the $\cl{}$-dimensional temporal {\lattstate} formulation, hence the
partial products of $[2\times2]$ stability matrices in
\refeq{Houlrik92(10)}. The admissible {\lattstate}s and the
pruning criterion are easily visualized in the
$(\field_{1,0},\field_{2,0})$ plane.

\item[2020-06-01 Predrag]
Gade and Amritkar\rf{GadAmr93}
{\em Spatially periodic orbits in coupled-map lattices}
(a preliminary version of a part of this work was published as
Amritkar, Gade, Gangal and Nandkumaran\rf{AGGN91}
{\em Stability of periodic orbits of coupled-map lattices}):

They take CMLs with \po s over $\BravCell{\speriod{}}{\period{}}{0}$ and
study the stability of their \po\ `replicas'
$\BravCell{k\speriod{}}{\period{}}{0}$ obtained by repeating
$\BravCell{\speriod{}}{\period{}}{0}$ $k$ times in the spatial direction,
and show that {\jacobianOrb} eigenvalues of the replica follow from the
small \po. Not obvious, as the replica \po\ has more directions to be
stable/unstable in. The trick is observing that the replica
{\jacobianOrb} is a block circulant with circulant blocks.

% 2020-06-01 Predrag
Cite Gade and Amritkar\rf{GadAmr93} as an early investigation of a
lattice {\jacobianOrb}. They did not know about `Hill's formula.

% was \section{{\catLatt}  blog} \label{sect:couplCatBlog}
\item[2016-01-12, 2016-08-04 PC] Literature
related to Gutkin and Osipov\rf{GutOsi15}
{\em Classical foundations of many-particle quantum chaos}:

The  existence of 2D symbolic dynamics was demonstrated in
\refref{PetCorBol07}, for a particular model of coupled lattice map.

``In general, calculating periodic orbits of a  non-integrable  system  is
a  non-trivial  task.   To  this  end  a  number  of  methods have  been
developed,'' and then,  for some reason, they refer to \refref{baranger88}.

Pethel \etal\rf{PetCorBol06} {\em Symbolic dynamics of coupled map lattices}

Pethel \etal\rf{PetCorBol07}
{\em Deconstructing {\spt} chaos using local symbolic dynamics}

Amig{\'o}, Zambrano and Sanju{\'a}n\rf{AmZaSa10}
{\em Permutation complexity of {\spt} dynamics}
study diffusive logistic coupled map lattices
(CML) \refeq{KanekoCML}.
Sun \etal\rf{SunKanZha11} {\em A method of recovering the initial vectors
of globally coupled map lattices based on symbolic dynamics} study
CMLs with logistic, Bernoulli, and tent chaotic maps. They  cite
\refRefs{PetCorBol06,PetCorBol07}.
Gundlach and  Rand\rf{GunRan93I} study coupled circle maps (the results of
subsequent papers in this series are claimed to be wrong, see
Jiang\rf{Jiang95}), which is too mathematical for me to understand. Sad.

Just\rf{Just01} {\em Equilibrium phase transitions in coupled map
lattices: {A} pedestrian approach}.
A class of piecewise linear coupled map lattices with simple symbolic
dynamics is constructed. It can be solved analytically in terms of the
statistical mechanics of spin lattices. The corresponding Hamiltonian is
written down explicitly in terms of the parameters of the map. The method
works only for map lattices with repelling invariant sets.

Just\rf{Just05}
{\em On symbolic dynamics of space-time chaotic models}

Sakaguchi\rf{saka90br} {\em Breakdown of the phase dynamics}
was the first to study a coupled Bernoulli maps lattice (in $D=2$).

Kawasaki and Sasa\rf{KaSa05} {\em Statistics of unstable periodic orbits
of a chaotic dynamical system with a large number of degrees of freedom},
study a coupled Bernoulli maps lattice (in spatial $D=1$); Bernoulli
forward in time, but tanh-coupled to the nearest spatial neighbors, so
that the natural invariant measure for spin configurations coincides with
the canonical distribution for an Ising spin Hamiltonian. The most
significant feature of the Bernoulli CML is that it respects a detailed
balance and the resulting  measure coincides with the canonical
distribution of the 1D Ising model. There  is  a  one-to-one
correspondence  between symbol sequences and \po s,
as proven by
Yutaka Ishii, % https://doi.org/10.1088/0305-4470/39/45/012
{\em Note on a paper by Kawasaki and Sasa on Bernoulli coupled map lattices}.
Then they commit the Japanese heresy: ``In summary, we have demonstrated
that the macroscopic properties of the Bernoulli CML can be calculated
with high accuracy using only one \po\ sampled from the special \po\
ensemble.''

Takeuchi and Sano\rf{TaSa07} {\em Role of unstable periodic orbits in
phase transitions of coupled map lattices} also study the spatially
periodic Bernoulli CML (in spatial $D=1$).


\bigskip

Atay, Jalan and Jost\rf{AtJaJo10} study coupled map networks with
multiple time delays; of no current interest for us.

\bigskip

Chen, Chen, and Yuan\rf{ChChYu14} {\em Topological horseshoes in
travelling waves of discretized nonlinear wave equations} is a
mathematical paper. They concentrate on describing \reqva\ of a
discretized version of a PDE that has \KS, KdV and Burgers as special
cases. They define discretized derivatives up to the 5th, if we ever need
them.
                                            \toCB
                They write
``
Applying the concept of anti-integrable limit to coupled map lattices
originated from space-time discretized nonlinear wave equations, we show
that there exist topological horseshoes in the phase space formed by the
initial states of travelling wave solutions. In particular, the coupled
map lattices display {\spt} chaos on the horseshoes.
''

Afraimovich and Pesin\rf{AfrPes90} {\em Hyperbolicity of
infinite-dimensional drift systems} study the discrete versions of the
complex  Ginzburg-Landau equation
\beq
u_{n, t+1}=u_{n,t}
            + \tau(u_{n,t},\sigma)
            + \gamma(u_{n,t}-u_{n,t-1})
            + \frac{\epsilon}{2}[u_{n-1,t} - 2u_{n,t} + u_{n+1,t}]
\,,
\ee{AfrPes90GinLan}
where $\tau(u_{n,t},\sigma)$ is local time dynamics, $\gamma$ is a
parameter of ``connection'', a memory of the previous step, so this has a
time evolution component that could be written as a time Laplacian, with
remainder presumably playing role of a friction. Not sure why this would
be a good idea, as Ginzburg-Landau is the first order in time. Literature
worries about the stability of of the space-homogeneous state in chains
of maps. They consider a special `drift' type of perturbation, at which
point they lost me.


\bigskip

Elder \etal\rf{ElXiDeMa95}
{\em Spatiotemporal chaos in the damped {Kuramoto-Sivashinsky} equation}: ``
A discretized version of the damped Kuramoto-Sivashinsky (DKS) equation
is constructed to provide a simple computational model of {\spt}
chaos in one dimension. The discrete map is used to study the transition
from periodic solutions to disordered solutions (i.e., {\spt}
chaos). The numerical evidence indicates a jump discontinuity at this
transition.
''


\end{description}
