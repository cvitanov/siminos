\chapter{Statistical mechanics applications}
\label{c-appendStatM}  % temporary, while being edited in spatiotemp/
%\Chapter{appendStatM}{22jun2016}{Statistical mechanics applications}
%\listofsections{0}
% $Author: predrag $ $Date: 2020-12-18 01:29:07 -0500 (Fri, 18 Dec 2020) $
% siminos/spatiotemp/chapter/appendStatM.tex
% Predrag 		22 jun 16
% Roberto 		 5 jul 00
% Predrag 		 8 feb 98
% catmap section from
% Roberto ~artuso/book/diff/diff.tex	11 oct 96
% Roberto                16 mar 95
% Roberto 13 feb 95 --->

\section{Cat map}
\label{sect:CatMap}

\noindent
\lettrine[lines=3,lraise=0.1,lhang=0.2]{H}{ere we will deal with}
%Here we will deal with
the prototype example of chaotic
Hamiltonian maps, \emph{hyperbolic toral automorphisms}, % \refref{ArnAve}.
(subspecies of which, known as the as the `Arnol'd cat
map', you have most likely already encountered), acting on a
cylinder or over $\reals ^2$. Their dynamics restricted to the
elementary cell involves maps on ${\mbox{\bf T}}^{2}$
(two--dimensional torus). On such torus an action of a matrix in
$SL_2(\naturals)$ with unit determinant and absolute value of the trace
bigger than $2$ is known as the Anosov map.
    \index{Anosov!map}\index{cat map}\index{Arnold!cat map}
    \index{hyperbolic!toral automorphisms}

\section{New example: Arnol'd cat map}
\label{sect:ArnCatMap}

\begin{bartlett}{
the Arnol'd-Sinai cat is a practical cat
                }\bauthor{
Ian Percival and Franco Vivaldi\rf{PerViv}
                }
\end{bartlett}

% PC 2016-06-01
The `standard' generating partition code of Arnol'd and Avez\rf{ArnAve}
is rather simple - it is described in Devaney\rf{deva87}.
    \PC{2016-08-03}{I do not have either monograph at hand, but
    Creagh\rf{Creagh94} summary in \refsect{sect:Creagh94} is pretty
    clear.}
It relies on a 3-rectangles complete partition of the torus. It is a
subshift of finite type - it is well suited to the generation and
counting of \po s on the torus, see Isola\rf{Isola90} rational
{\tzeta} in \refsect{sect:Isola90}.

However, the Arnol'd--Avez alphabet has no easy translation to the
integers shift on the unfolded torus (on the lattice, most of torus \po s
are \rpo s).
Furthermore, for $N$ coupled cat maps the number of such rectangles would
grow exponentially\rf{GutOsi15}.
    \PC{2016-08-03}{Have not checked that, or whether this is explained in
    \refref{GutOsi15}.}

There is a general consensus in the cat map community\rf{Keating91a} that
the `linear code' of Percival and Vivaldi\rf{PerViv}
(here \refsect{sect:PerViv})
is deeper and more powerful.
For deterministic diffusion developed in Chaos\-Book
(\refchap{c-appendStatM} here) that is the only choice, as one needs to
convert symbolic dynamics of an \rpo\ to the integer shift (translation)
on the lattice.
    \PC{2016-08-03}{I do not know why this symbolic dynamics is natural
    for extensions to $N$ nearest-neighbor coupled maps.}
The downside is that the Markov/generating partition is infinite,
meaning that for longer and longer orbits there are more and more new
pruning ({\inadmissible} blocks) rules, ad infinitum.

    \PC{2016-06-02}{verbatim from Percival and Vivaldi\rf{PerViv}}
Iterated area preserving maps of the form
\bea
p' &=& p + F(x)             \label{PerViv2.1a}
    \\
x' &=& x+p' \qquad  \mod 1, \label{PerViv2.1b}
\eea
where F(x) is periodic of period 1, are widely studied because of their
importance in dynamics. They include the standard map of Taylor, Chirikov
and Greene\rf{Lichtenberg92,Chirikov79}, and also the sawtooth and cat
maps that we describe here. Because values of x differing by integers are
identified, whereas the corresponding values of p are not, the phase
space for these equations is a cylinder.

These maps describe `kicked' rotors that are subject to a sequence of
angle-dependent impulses F(x), with $2\pi x$ as the configuration angle of
the rotor, and p as the momentum conjugate to the configuration
coordinate x.
    \PCedit{ % 2016-05-29
The time step has been set to $\Delta t= 1$. Eq.~\refeq{PerViv2.1a} says that
the momentum $p$ is accelerated to $p'$ by the force pulse $F(x)\Delta
t$, and eq.~\refeq{PerViv2.1b} says in that time the trajectory $x$ reaches
$x' = x+p'\Delta t$.
    }

The phase space of the rotor is a cylinder, but it is often convenient to
extend it to the plane or contract it to a torus. For the former case the
``$\mod 1$'' is removed from \refeq{PerViv2.1b} and for the latter it is
included in \refeq{PerViv2.1a}.

Eqs.~(\ref{PerViv2.1a},\ref{PerViv2.1b}) are a discrete time form of Hamilton's
equations. But for many purposes we are only interested in the values of
the configuration coordinate x, which satisfy the second-order difference equation
        \PCedit{ % 2016-05-29
(the discrete Laplacian in time)
        }
\beq
\delta^2 x_t \equiv x_{t+1} - 2x_{t} + x_{t-1} = F(x_{t})  \qquad  \mod 1
\ee{PerViv2.2}
where t is a discrete time variable that takes only integer values. This
equation may be considered as the Lagrangian or Newtonian equation
corresponding to the Hamiltonian form (2.1), with
\(
p_t = x_{t} + x_{t-1} \,.
\)

Rewrite \refeq{PerViv2.2} as
\beq
 x_{t+1} =  2x_{t} + F(x_{t}) - x_{t-1} \qquad  \mod 1
\ee{PerViv2.3}
Call the 1-step configuration point forward in \refeq{PerViv2.1b} $x_{t}=y$,
and the next configuration point $x_{t+1} = y'$.
This recasts the dynamical equation in the form of an area preserving map
in which only configurations at different times appear,
\bea
x' &=& y
    \continue
y' &=&  2y + F(y) - x  \qquad  \mod 1
\,.
\label{PerViv2.4}
\eea
    \PCedit{ % 2016-05-29
This they call the `two-configuration representation'.
    }

The sawtooth map represents a rotor subject to
an impulse F(x) that is linear in x, except for a
single discontinuity. The impulse is standardised
to have zero mean, the origin of x is chosen so that F(0)= 0, so
\beq
 F(x) = Kx \quad (-1/2 \leq x < 1/2)
\ee{PerViv2.5}
With these conventions Hamilton's equations
for the sawtooth are
\bea
x' &=& y        \qquad\qquad  \mod 1
    \continue
y' &=&  - x  +sy \quad  \mod 1
\label{PerViv2.4a}
\eea
in the two-configuration representation, where
\beq
 s = K+2
 \,.
\ee{PerViv2.5a}
For $s>2$ the map is unstable.
In the two-configuration representation, Hamilton's
equations can be written in matrix form as
\beq
\left (
\begin{array}{c}
x' \\
y' \\
\end{array}
\right ) = {M} \left (
\begin{array}{c}
x \\
y \\
\end{array}
\right ) \quad  \mod\: 1
\ee{PerViv2.9}
with
\beq {M}= \left (
		\begin{array}{cc}
		0&1\\
		-1&s\\
		\end{array} \right)
\,,
\ee{PerViv2.10}
characteristic polynomial
\beq
\ExpaEig^2 - s \ExpaEig + 1
 \,.
\ee{PerViv2.11}
and eigenvalues
\beq
\ExpaEig  = (s+\surd{D})/2
\,,\qquad
\ExpaEig' = (s-\surd{D})/2
 \,,
\ee{PerViv3.7}
where
\(
D = s^2-4
\,.
\)
When $s$ is an integer, then the map \refeq{PerViv2.9} is continuous on the
torus, because the discontinuity of the sawtooth is an integer absorbed
into the modulus. The map is then continuous; it is a toral automorphism,
of a class called \emph{cat maps}, of which the Arnol'd-Sinai cat
map\rf{ArnAve} with $s=3$ is a special case,
\beq {M}= \left (
		\begin{array}{cc}
		0 &1\\
		-1&3\\
		\end{array} \right)
\,.
\ee{PerViv2.10a}

    \PCedit{ % 2016-05-29
If instead of \refeq{PerViv2.3} dynamics on a torus, one considers motion on
a line (no $\mod 1$), one can land in any unit interval along the
$q$-axis. Let then $-b_t$ be the sequence of integer shifts that ensures
that for all $t$ the dynamics
            }
\beq
 x_{t+1} =  2x_{t} + F(x_{t}) - x_{t-1} -b_t
\ee{PerViv3.3}
    \PCedit{ % 2016-05-29
stays confined to the elementary cell $x_{t} \in [-1/2,1/2)$.
}
The Newton equation \refeq{PerViv2.2} then takes the form
\beq
(\delta^2  - K) x_{t} = -b_t
\ee{PerViv3.6}
The linear operator or infinite tridiagonal matrix
on the left of \refeq{PerViv3.6} has a Green's function or inverse
matrix given by the unique bounded solution $\gd_{ts}$
of the inhomogeneous equation
        \PC{2016-05-29}{
still have to check this calculation
    }
\beq
 \gd_{t+1,t'} -  s\,\gd_{tt'} + \gd_{t-1,t'} =\delta_{tt'}
 \,,
\ee{PerViv3.9a}
which is given by
\beq
 \gd_{tt'} = -\ExpaEig^{-|t-t'|}/\surd{D}
 \,.
\ee{PerViv3.9ab}
This solution is obtained by a method that is directly analogous to the
method used for second order linear differential
equations\rf{varcyc} \CBlibrary{varcyc}. The solution of \refeq{PerViv3.6} for
the orbit is therefore
    \PC{2018-03-11}{Mestel and Percival\rf{varcyc} is very systematic, with
    Wronskians, \etc, but I do not see this solution there.
    Percival and Vivaldi\rf{PerViv} state it, say it can be derived
    by the method of Mestel and Percival\rf{varcyc}, and say ``as may be
    verified by substitution.''}
\beq
 t_{t} = \sum_t' \gd_{tt'} (-b_t)
  = \frac{1}{\surd{D}} \sum_t' \ExpaEig^{-|t-t'|} b_{t'}
 =\delta_{tt'}
 \,,
\ee{PerViv3.10}
defining the orbit uniquely in terms of the symbol sequence. That is to
say that the code is complete. We shall refer to an integer code such as
$\{b_t\}$ for a linear system as a \emph{linear code}: the orbit and the
code are related to one another by a linear transformation. Clearly, a
shift in the symbol sequence $\{b_t\}$ corresponds to an equivalent time
shift of the orbit.

The past and the future sums in \refeq{PerViv3.10} resembles the expression
for a real number in terms of the digits $b_t$, using a representation of the
reals in the nonintegral base $\ExpaEig$, in contrast with the past and
future coordinates for the baker's transformation, which have a similar form,
to base 2. The `present' symbol $b_0$ is incorporated with the past in our
convention.



%%%%%%%%%%%%%%%%%%%%%%%%%%%%%%%%%%%%%%%%%%%%%%%%%%%%%%%%%%%%%%%%%%%%%%%%%%%%%
\Remarks

%%%%%%%%%%%%%%%%%%%%%%%%%%%%%%%%%%%%%%%%%%%%%%%%
% PC 2018-02-11 merge into
% \Chapter{appendStatM}{22jun2016}{Statistical mechanics applications}
\remark{Deterministic diffusion in Hamiltonian maps.}{\label{rem:AnosovMaps}
~~(Continued from \refrem{rem:DetDiffHam})
The quasilinear estimate \refeq{qlD} was given in \refref{CarMei}
and evaluated in \refrefs{ArtStr97,Strepparava95}.
Circulant matrices are discussed in Aitken\refref{Aitken39} (1939).
The result \refeq{FinD} agrees with the saw-tooth result of
\refref{CarMei}; for the cat maps \refeq{FinD}
is the exact value of the diffusion coefficient.
This result was also obtained, by using periodic orbits, in
\refref{dana89}, where Gaussian nature of the diffusion
process is explicitly assumed.
Measure polytopes are discussed in \refref{Coxeter12}.
    } %end \remark{Anosov maps} %{Who's talked about it?}{
%%%%%%%%%%%%%%%%%%%%%%%%%%%%%%%%%%%%%%%%%%%%%%%%


\RemarksEnd
%%%%%%%%%%%%%%%%%%%%%%%%%%%%%%%%%%%%%%%%%%%%%%%%%%%%%%%%%%%%%%%%%%%%%%%%%%%%%

    \PC{2018-12-01}{
Had here Problems/exerAppStatM until 30dec2017, now only copy is the renamed
ChaosBook exerAppDiff.tex.\\
REMEMBER: move the cat map exercises to exerCatMap.tex
    }

\printbibliography[heading=subbibintoc,title={References}]

\ChapterEnd
