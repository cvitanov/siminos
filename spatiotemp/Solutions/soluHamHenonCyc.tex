% soluHamHenonCyc.tex
% $Author: predrag $ $Date: 2021-10-14 11:00:27 -0400 (Thu, 14 Oct 2021) $

\Solution{Henon}{soluFixed}{26oct2014}{{\HenonMap}}
% Predrag                                       26oct2014
%       from ChaosBook soluFixed

%%%%%%%%%%%%%%%%%%%%%%%%%%%%%%%%%%%%%%%%%%%%%%%%%%%%%%%%%%%%%%%%%%%%%%%
\solution{exer:HamHenonCyc}
         {Inverse iteration method for a Hamiltonian repeller.}{
\index{inverse!iteration, Hamiltonian repeller}
\index{iteration!inverse!Hamiltonian repeller}
\index{periodic!orbit!Hamiltonian repeller}
\index{Hamiltonian!repeller, periodic orbits}
For the complete repeller case (all binary sequences are realized),
the cycles can be evaluated variationally, as follows.
According to \refeq{eq2.1a},
% (\ref{times}),
the coordinates of a periodic orbit of
length $n_p$ satisfy the  equation
                                \toCB
\beq
\ssp_{p,i+1}+\ssp_{p,i-1}=1 -  a \ssp_{p,i}^2 \, ,
\quad i=1,...,n_p
\,,
\label{e:pere}
\eeq
with the periodic boundary condition
%$\ssp_{p,n_p +1}=\ssp_{p,1}$ and
$\ssp_{p,0}=\ssp_{p,n_p}$.

In the complete repeller case, the {\HenonMap} is a realization
of the Smale horseshoe, and the symbolic dynamics has a very simple
description in terms of the
binary alphabet  $\epsilon \in \{ 0,1 \}$,
$\epsilon_{p,i}=(1+S_{p,i})/2$, where $S_{p,i}$ are the
signs of the corresponding periodic point coordinates,
$S_{p,i}=\ssp_{p,i}/|\ssp_{p,i}|$.
We start with a preassigned sign sequence
$S_{p,1},S_{p,2}, \dots, S_{p,n_p}$, and a
good  initial guess for the coordinates $\ssp_{p,i}'$.
Using the inverse of the equation (\ref{pere})
\[
\ssp_{p,i}''=S_{p,i}\sqrt{\frac{1-\ssp_{p,i+1}'-\ssp_{p,i-1}'}{a}}
\,\quad i=1,...,n_p
\]
we converge iteratively, at exponential rate, to the desired
periodic points $\ssp_{p,i}$. Given the periodic points, the cycle
stabilities and periods are easily computed using
\refeq{Henlatt-e_her}. The itineraries and the stabilities of the short
periodic orbits for the {\Henon} repeller \refeq{e:pere} for
$a=6$ are listed in \reftab{gv:cycles}; in actual
calculations all {\orbit}s up to topological length $n=20$
have been computed.
%
\hfill G.~Vattay
} % end \solution{A Hamiltonian repeller}
%%%%%%%%%%%%%%%%%%%%%%%%%%%%%%%%%%%%%%%%%%%%%%%%%%%%%%%%%%%%%%%%%%%%%%%
