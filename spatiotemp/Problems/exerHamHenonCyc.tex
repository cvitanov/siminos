% siminos/spatiotemp/Problems/exerHamHenonCyc.tex called by Henon.tex
% $Author: predrag $ $Date: 2021-10-15 15:55:44 -0400 (Fri, 15 Oct 2021) $

\Problems{exerFixed}{14oct2021}

%%%%%%%%%%%%%%%%%%%%%%%%%%%%%%%%%%%%%%%%%%%%%%%%%%%%%%%%%%%%%%%
% Predrag                   27apr2021, 14oct2021
% Predrag                   26oct2014
\exercise{Inverse iteration method for a {\Henon} repeller.}{
                \label{exer:HamHenonCyc}
\index{inverse!iteration, Hamiltonian repeller}
\index{iteration!inverse!Hamiltonian repeller}
\index{periodic!orbit!Hamiltonian repeller}
\index{Hamiltonian!repeller, periodic orbits}

%%%%%%%%%%%%%%%%%%%%%%%%%%%%%%%%%%%%%%%%%%%%%%%%%%%%%%%%%%
\begin{table}
\caption{\label{gv:cycles}
 All {\po}s up to $\cl{}=6$
for the Hamiltonian {\HenonMap} repeller
%\refeq{e:pere}
\refeq{pere}
with $a=6$.
Listed are the cycle itinerary, its expanding eigenvalue $\ExpaEig_p$,
and its ``center of mass.'' The ``center of mass'' is listed because
it turns out that it is often a simple rational
or a quadratic irrational.
All {\orbit}s up to topological length $n=20$
have been computed.
    }
\renewcommand{\arraystretch}{0.7}
\noindent
{ \small
\begin{tabular}{lrr}
% \setdec 00.00000
% V                    &\dec 1.04323   &\dec 0.48474   &  5&     -1 \\
~~~~$p$~~~~ & $\ExpaEig_p$~~~~~~~~~~~~
          &    $\sum \ssp_{p,i}~~~~~$ \\
    \hline
0 & 0.715168$\times 10^1$ & -0.607625 \\
1 &-0.295285$\times 10^1$ &  0.274292 \\
%   \hline
10 &-0.989898$\times 10^1$ &  0.333333 \\
%   \hline
100 &-0.131907$\times 10^3$ & -0.206011 \\
110 & 0.558970$\times 10^2$ &  0.539345 \\
%   \hline
1000 &-0.104430$\times 10^4$ & -0.816497 \\
1100 & 0.577998$\times 10^4$ &  0.000000 \\
1110 &-0.103688$\times 10^3$ &  0.816497 \\
%   \hline
10000 &-0.760653$\times 10^4$ & -1.426032 \\
11000 & 0.444552$\times 10^4$ & -0.606654 \\
10100 & 0.770202$\times 10^3$ &  0.151375 \\
11100 &-0.710688$\times 10^3$ &  0.248463 \\
11010 &-0.589499$\times 10^3$ &  0.870695 \\
11110 & 0.390994$\times 10^3$ &  1.095485 \\
%   \hline
100000 &-0.545745$\times 10^5$ & -2.034134 \\
110000 & 0.322221$\times 10^5$ & -1.215250 \\
101000 & 0.513762$\times 10^4$ & -0.450662 \\
111000 &-0.478461$\times 10^4$ & -0.366025 \\
110100 &-0.639400$\times 10^4$ &  0.333333 \\
101100 &-0.639400$\times 10^4$ &  0.333333 \\
111100 & 0.390194$\times 10^4$ &  0.548583 \\
111010 & 0.109491$\times 10^4$ &  1.151463 \\
111110 &-0.104338$\times 10^4$ &  1.366025
\end{tabular}
} %end { \small
\end{table}
\renewcommand{\arraystretch}{1.0}
%%%%%%%%%%%%%%%%%%%%%%%%%%%%%%%%%%%%%%%%%%%%%%%%%%%%%%%%%%

                                                \toCB
    \PC{14oct2021}{Return to ChaosBook eventually}
    \PC{27dec2004}{give the center of mass paper reference somewhere}
%
Consider the  {\HenonMap} \refeq{eq2.1a}
for the area-preserving (``Hamiltonian'') parameter value $b=-1$.
The coordinates of a {\po} of
length $n_p$ satisfy the  equation
\index{Henon@H\'enon map!cycles}
\beq
\ssp_{p,i+1}+\ssp_{p,i-1}=1 -  a \ssp_{p,i}^2 \, ,
\quad i=1,...,n_p
\,,
\label{pere}
\eeq
with the periodic boundary condition
%$\ssp_{p,n_p +1}=\ssp_{p,1}$ and
$\ssp_{p,0}=\ssp_{p,n_p}$.
Verify that the itineraries and the stabilities of the short
{\po}s for the {\Henon} repeller \refeq{pere}
at $a=6$ are as listed in \reftab{gv:cycles}.

{\bf Hint}: you can use any cycle-searching routine you wish, but for the
complete repeller case (all binary sequences are realized), the cycles
can be evaluated by inverse iteration
$\ssp_{p,i}^{(m+1)}\to\ssp_{p,i}^{\infty}=\ssp_{p,i}$, estimating
the midpoint by square-root
of \refeq{pere}
\beq
\ssp_{p,i}^{(m+1)}=\sign{p,i}
     \sqrt{\frac{1-\ssp_{p,i+1}^{(m)}-\ssp_{p,i-1}^{(m)}}{a}}
\,.
\ee{GVinver}
Here $\sign{p,i}$ are the
signs of the corresponding periodic point coordinates,
$\sign{p,i}=\ssp_{p,i}/|\ssp_{p,i}|$,
% and $i=1,...,n_p$
related in the obvious way to desired \po's binary itinerary,
\beq
\sign{i}+1=2\,\Ssym{i}\,,\quad \Ssym{i}\in\{0,1\}
\,.
\ee{henonSigns}
see \reffig{fig:PChenlatt5cyc}.
%
\hfill G.~Vattay
    } % end \exercise{A Hamiltonian repeller}
%%%%%%%%%%%%%%%%%%%%%%%%%%%%%%%%%%%%%%%%%%%%%%%%%%%%%%%%%%%%%%%

%%%%%%%%%%%%%%%%%%%%%%%%%%%%%%%%%%%%%%%%%%%%%%%%%%%%%%%%%%%%%%%
\exercise{``Center of mass'' puzzle.}{\label{e:CenterOfMass}
% \tangent
Why is the ``center of mass,'' tabulated
in \refexer{exer:HamHenonCyc}, often a rational number?
\index{center of mass}
%    \else listed in \reftab{gv:cycles},
    } % end \exercise{``Center of mass'' puzzle
%%%%%%%%%%%%%%%%%%%%%%%%%%%%%%%%%%%%%%%%%%%%%%%%%%%%%%%%%%%%%%%

    \ProblemsEnd
