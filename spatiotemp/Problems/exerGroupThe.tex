% siminos/spatiotemp/Problems/exerGroupThe.tex called by blogSVW.tex
% $Author: predrag $ $Date: 2021-09-14 18:16:58 -0400 (Tue, 14 Sep 2021) $


% \Problems{exerGroupThe}{6mar2021}

%%%%%%%%%%%%%%%%%%%%%%%%%%%%%%%%%%%%%%%%%%%%%%%%%%%%%%%%%%%%%%%%%%%
% Sidney                        2021-03-07
\exercise{Engel Point Groups 1.}{\label{exer:Engel11PG1}
(Engel's\rf{Engel11}
\HREF{http://www-personal.umich.edu/~engelmm/lectures/ShortCourseSymmetry1.pdf}
{Point Groups} Exercise 1):
The molecule on the left has $C_{1s}$ which signifies
that it has reflection symmetry over one axis. The molecule on the right
has $C_3$ symmetry, signifying that it is symmetric by rotations of
$\frac{2\pi}{3}$ or one third of a full circle.

\hfill M.~Engel
    } % end \exercise{exer:Engel11PG1}
%%%%%%%%%%%%%%%%%%%%%%%%%%%%%%%%%%%%%%%%%%%%%%%%%%%%%%%%%%%%%%%%%%%

%%%%%%%%%%%%%%%%%%%%%%%%%%%%%%%%%%%%%%%%%%%%%%%%%%%%%%%%%%%%%%%%%%%
% Sidney                        2021-03-07
\exercise{Engel Point Groups 3.}{\label{exer:Engel11PG3}
(Engel's\rf{Engel11}
\HREF{http://www-personal.umich.edu/~engelmm/lectures/ShortCourseSymmetry1.pdf}
{Point Groups} Exercise 3):
Three point groups for $C_2H_6$: a. $C_{3v}$ because rotating it by 1/3
of a circle leaves it invariant, and one can cut the molecules into three
identical pieces. b. $C_s$ because the top and bottom have the same
orientation, it is like looking in a mirror, so can apply reflection
symmetry. c. Unsure, perhaps inversion symmetry $C_i$.

\hfill M.~Engel
    } % end \exercise{exer:Engel11PG3}
%%%%%%%%%%%%%%%%%%%%%%%%%%%%%%%%%%%%%%%%%%%%%%%%%%%%%%%%%%%%%%%%%%%
