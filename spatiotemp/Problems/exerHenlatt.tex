% $Author: predrag $ $Date: 2021-09-02 23:44:14 -0400 (Thu, 02 Sep 2021) $
% siminos/spatiotemp/Problems/exerHenlatt.tex called by blogSVW.tex

\section{Sidney exercises}
\begin{enumerate} %TEMPORARY
% \Problems{exerHenlatt}{30nov2020}

% Predrag                                               2020-12-27
%  moved GitHub/reducesymm/Problems/exerHenlatt.tex to here
% Predrag                                               2020-11-30

%%%%%%%%%%%%%%%%%%%%%%%%%%%%%%%%%%%%%%%%%%%%%%%%%%%%%%%%%%%%%%%%%%%
\exercise{{\Henon} temporal lattice.}{\label{exer:tempHen}

1\dmn\ temporal {\Henon} lattice (see \toChaosBook{section.3.4} {Example
3.5}) is given by a 3-term recurrence
\beq \nonumber
\ssp_{n+1} + a \ssp_n^2 - b\,\ssp_{n-1} = 1
\,.
\eeq
The parameter $a$ quantifies the ``stretching'' and $b$ quantifies the
``contraction''.

The single H\'enon map is nice because the system is a nonlinear
generalization of \templatt\ 3-term recurrence
CL18 eq.~{catMapNewt}, with no restriction to the unit hypercube XXX, but has
binary dynamics.

There is still a tri-diagonal {\jacobianOrb} $\jMorb$
CL18 eq.~{tempCatFixPoint}, but CL18 eq.~({Hessian}) is now {\lattstate}
dependent. Also, I beleive Han told me that
CL18 sect.~{s:Hill}~{\em {\HillDet}:
            stability of an orbit vs. its time-evolution stability}
block matrices derivation of Hill's formula does not work any more.
Neither does the `fundamental fact', as each {\lattstate}'s {\jacobianOrb}
is different, and presumably does not count periodic states, as there is
no integer lattice within the {\HillDet} volume.

Does the \toChaosBook{section.27.4} {flow conservation} sum rule
\toChaosBook{equation.27.4.15}{ed.~(27.15)}
(or CL18.tex eq.~{Det(jMorb)eights}) still work?

The assignment: Implement the variational searches for
periodic states in Matt's \HREF{https://github.com/farom57/Orbit-hunter}
{OrbitHunter}, find all {\lattstate}s up to $n=6$.

(a) $a=1.4\,\;b=0.3$, compare with
\toChaosBook{table.caption.559} {Table~34.2}.

(b) For $b=-1$ the system is time-reversible or `Hamiltonian',
see
\toChaosBook{exmple.8.5}{Example 8.5}.
For definitiveness, in numerical calculations in examples to follow we
fix (arbitrarily) the stretching parameter value to $a=6$, a value large
enough to guarantee that all roots of the periodic point condition
$0=f^n(x)-x$  are real.

Note also
\toChaosBook{section.J.3}
{sect~A10.3} {\em H\'enon map symmetries}
and
\toChaosBook{Item.91}
{Exer.~7.2} {\em Inverse iteration method}.

The deviation of an approximate trajectory from the 3-term recurrence is
\beq \nonumber
v_n = \ssp_{n+1} - (1 - a \ssp_n^2 + b\,\ssp_{n-1})
\eeq
In classical mechanics force is the gradient of a potential, which
Biham-Wenzel\rf{afind} construct as a cubic potential
\beq
V_n = \ssp_{n+1}\ssp_n - b\,\ssp_n\ssp_{n-1} + (a \ssp_n^3 - \ssp_n)
\,.
\ee{BWcubic}
With the cubic potential at lattice site $n$ we can start to look for
orbits variationally. Note that the potential is time-reversal invariant
for $b=1$.


Compare with XXX
    } % end \exercise{exer:catMapGreenInf}
%%%%%%%%%%%%%%%%%%%%%%%%%%%%%%%%%%%%%%%%%%%%%%%%%%%%%%%%%%%%%%%%%%%%%%%%

%%%%%%%%%%%%%%%%%%%%%%%%%%%%%%%%%%%%%%%%%%%%%%%%%%%%%%%%%%%%%%%%%%%
%\exercise{XXX.}{\label{exer:XXX}
%XXX
%    } % end \exercise{exer:XXX}
%%%%%%%%%%%%%%%%%%%%%%%%%%%%%%%%%%%%%%%%%%%%%%%%%%%%%%%%%%%%%%%%%%%%%%%%

%%%%%%%%%%%%%%%%%%%%%%%%%%%%%%%%%%%%%%%%%%%%%%%%%%%%%%%%%%%%%%%%%%%
%\exercise{XXX.}{\label{exer:XXX}
%XXX
%    } % end \exercise{exer:XXX}
%%%%%%%%%%%%%%%%%%%%%%%%%%%%%%%%%%%%%%%%%%%%%%%%%%%%%%%%%%%%%%%%%%%%%%%%

% \end{enumerate} %TEMPORARY, moved to exerGroupThe.tex
%    \ProblemsEnd
