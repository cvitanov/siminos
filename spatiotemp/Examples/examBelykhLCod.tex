% siminos/spatiotemp/chapter/examBelykhLCod.tex
% $Author: predrag $ $Date: 2021-12-24 01:25:20 -0500 (Fri, 24 Dec 2021) $

% siminos/spatiotemp/chapter/examTentLCod.tex
% $Author: predrag $ $Date: 2021-12-24 01:25:20 -0500 (Fri, 24 Dec 2021) $
% called by siminos/spatiotemp/chapter/tentMapCode.tex
%\section{Any piecewise linear map has ``linear code''}
%\label{exam:tentMapSymbDyn}

%%%%%%%%%%%%%%%%%%%%%%%%%%%%%%%%%%%%%%%%%%%%%%%%%%%%%%%
\example{Belykh map linear code.}{\label{exam:BelykhLCod}
Li and Xie\rf{LiXie16}
{\em Symbolic dynamics of {Belykh}-type maps}:
``
The symbolic dynamics of a Belykh-type map (a two-dimensional
discontinuous piecewise linear map) is investigated. The pruning front
conjecture (the admissibility condition for symbol sequences)  is proved
under a hyperbolicity condition. Using this result, a symbolic dynamics
model of the map is constructed according to its pruning front and
primary pruned region.
''

The Belykh map is a piecewise linear map given by
\[  \left( \begin{array}{c}
        x_{n+1} \\
        y_{n+1} \\
        \end{array}\right)
        = \left(\begin{array}{c}
                    \sign{n} -a x_n + b y_n     \\
                    x_n                         \\
                \end{array} \right)
        =  \left( \begin{array}{c}
            \sign{n} \\
            0        \\
        \end{array}\right)
        +
\left(\begin{array}{cc}
            -a  & b      \\
             1  & 0      \\
                \end{array} \right)
\left( \begin{array}{c}
        x_{n} \\
        y_{n} \\
        \end{array}\right)
\,.\]
where
\[
\sign{n} = \left\{\begin{array}{rcc}
                                1 & \mbox { if } &  x_n \geq 0\\
                                -1 & \mbox { if } &  x_n < 0\\
                                                \end{array} \right.
\,.
\]
The two branches of the map are
\[
f_{\pm} = \left\{\begin{array}{l}
                                \pm 1 -a x + b y\\
                                 x \\
                                                \end{array} \right.
\,.\]
In the 3-term recurrence formulation (the linear code), the map is
an asymmetric tridiagonal Toeplitz matrix
\[
 x_{n+1} +a x_n - b x_{n-1} = \sign{n}
 \,,
\]
or
\beq
\Box x_n +(2+a) x_n - (1+b) x_{n-1} = \sign{n}
 \,.
\ee{s:BelykhMapDiff}
For $b=-1$ (the Hamiltonian, time-reversible case) this is almost the cat
map, with $a=-s$, except that the single sawtooth discontinuity is across
$x=0$, there is no $\mod\ 1$ condition.

Li and Xie consider the
\(
a,\, b\, > 0
\)
case. The strange attractor (for example, for $a = 1.5$ and $b = 0.3$)
looks like a fractal set of parallel lines.
They define the pruning front, the primary pruned region, plot them in
the symbol plane, and prove the pruning front conjecture  for this map.
In the symbol plane there is a symmetry under rotation by $\pi$, but they
do not seem to exploit that.

They call the past and the future itineraries of the tail and the head,
and start the head with $s_0$.

T{\'{e}}l\rf{Tel83} {\em Fractal dimension of the strange attractor in a
piecewise linear two-dim\-en\-si\-on\-al map} computes the box-counting dimension
of this map (which he does not call Belykh map).
    } % end \example{Belykh map linear code.}{{exam:BelykhLCod}
