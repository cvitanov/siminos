% siminos/spatiotemp/Examples/examD3charTab.tex
% $Author: predrag $ $Date: 2021-07-24 23:53:43 -0400 (Sat, 24 Jul 2021) $

%%%%%%%%%%%%%%%%%%%%%%%%%%%%%%%%%%%%%%%%%%%%%%%%%%%%%%%%%%%%%%%%%%%%%%
\begin{table}[h]
  \caption[Character tables of $\Dn{1}$, $\Cn{3}$ and $\Dn{3}$]{
    Character tables of $\Dn{1}$, $\Cn{3}$ and $\Dn{3}$.
    The classes
    $\{\Refl_{12},\Refl_{13},\Refl_{14}\}$, $\{ \shift, \shift^2 \}$
    are denoted $3\Refl$, $2C$, respectively.
  }
  \label{tab:D3charac}
  \centering
  \begin{tabular}{c|ccc}
    $\Dn{1}$ & $A$ & $B$ \\
    \hline
    $e$  & 1 & 1  \\
    $\Refl$ & 1 & -1
  \end{tabular}
  \qquad
  \begin{tabular}{c|ccc}
    $\Cn{3}$ & $A$ & \multicolumn{2}{c}{$E$}  \\
    \hline
    $e$ & 1 & 1  & 1 \\
    $\shift$ & 1 & $\omega$ & $\omega^2$ \\
    $\shift^2$ & 1 & $\omega^2$  & $\omega$
  \end{tabular}
  \qquad
  \begin{tabular}{c|ccc}
    $\Dn{3}$ & $A$ & $B$ & $E$ \\
    \hline
    $e$ & 1 & 1  & 2 \\
    $3\Refl$ & 1 & -1 & 0 \\
    $2C$   & 1 & 1  & -1
  \end{tabular}
\end{table}
%%%%%%%%%%%%%%%%%%%%%%%%%%%%%%%%%%%%%%%%%%%%%%%%%%%%%%%%%%%%%%%%%%%%%%

%%%%%%%%%%%%%%%%%%%%%%%%%%%%%%%%%%%%%%%%%%%%%%%%%%%%%%%%%%%%%%%%%%%%%%
\example{Character table of $\Dn{3}$.}{\label{exam:D3charTab}         \toCB
%% 2021-07-23 from Xiong's siminos/xiong/thesis/chapters/symGroup.tex
  (continued from \refexam{exam:D3regularRep??})~~
  Let us construct \reftab{tab:D3charac}.
  one\dmn\ representations are denoted by $A$ and $B$, depending on
  whether the basis function is symmetric or antisymmetric with respect to
  transpositions $\Refl_{ij}$. $E$ denotes the two\dmn\ representation.
  As
  $\Dn{3}$ has 3 classes, the dimension sum rule $d_1^2+d_2^2+d_3^2 = 6$
  has only one solution $d_1=d_2=1$, $d_3=2$. Hence there are two one\dmn\
  irreps and one two\dmn\ irrep. The first row is $1,1,2$, and the first
  column is $1,1,1$ corresponding to the one\dmn\ symmetric representation. We take
  two approaches to figure out the remaining 4 entries. First, since $B$
  is an antisymmetric one\dmn\ representation, so the characters should be $\pm 1$.
  We anticipate $\chi^{B}(\Refl) = -1$ and can quickly figure out the
  remaining 3 positions. Then we check that the obtained table satisfies the
  orthonormal relations. Second, denote $\chi^{B}(\Refl)=x$ and
  $\chi^E(\Refl)=y$, then from the orthonormal relation of the second
  column with the first column and itself, we obtain $1+x+2y=0$ and
  $1+x^2+y^2=6/3$. Then we get two sets of solutions, one of which is
  incompatible with other orthonormal relations, so we are left with
  $x=-1$, $y=0$.
  Similarly, we can get the other two characters.
                                        \jumpBack{exam:D3charTab}
\authorXD{}
} %end \example{exam:D3charTab}
%%%%%%%%%%%%%%%%%%%%%%%%%%%%%%%%%%%%%%%%%%%%%%%%%%%%%%%%%%%%%%%%%%%%%%
