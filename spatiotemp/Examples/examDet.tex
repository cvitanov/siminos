% examDet.tex called by {det}{9mar2015}{Spectral determinants}
% $Author: predrag $ $Date: 2021-08-10 11:56:19 -0400 (Tue, 10 Aug 2021) $

% Predrag if edited, return to ChaosBook    2018-12-13
% Predrag  added Examples                    9mar2015
% Raenell                                   19Apr2005


%\section{Examples}
%\label{exam:det}

%%%%%%%%%%%%%%%%%%%%%%%%%%%%%%%%%%%%%%%%%%%%%%%%%%%%%%%%%%%%%%%%%
\example{Two-\dofs\ Hamiltonian flows:}{\label{exam:2DHamiltFlow}
    \PC{2018-12-13}{if edited, return to ChaosBook}
For a 2-{\dofs} Hamiltonian flow the energy conservation eliminates one
phase-space variable, and restriction to a \PoincSec\ eliminates the
marginal longitudinal eigenvalue $\ExpaEig=1$, so a periodic orbit of
2-{\dofs} hyperbolic Hamiltonian flow (or of a 1-{\dof} hyperbolic
Hamiltonian map) has one expanding transverse eigenvalue $\ExpaEig$,
$|\ExpaEig|>1$, and one contracting transverse eigenvalue $1/\ExpaEig$.
The weight in \refeq{prod-Lamb} is expanded as follows:
\beq
{1 \over \oneMinJ{r} }
 =  {1 \over |\ExpaEig|^r (1-1/\ExpaEig_p^{r})^2 }
 =   {1 \over |\ExpaEig|^r} \sum_{k=0}^\infty {k+1 \over \ExpaEig_p^{kr}}
\,\, .
\ee{Jac-exp}
The \Fd\ exponent can be resummed,
\[
   - \sum_{r=1}^\infty \frac{1}{r}
    {       e^{(\beta \Obser_p-\eigenvL \period{p}) r}
         \over \oneMinJ{r} }
 =  %\sum_p
     \sum_{k=0}^\infty (k+1)
         \log \left( 1 -  { e^{\beta \Obser_p -\eigenvL \period{p} } %z^{\cl{p}}
                \over
                           |\ExpaEig_p| \ExpaEig_p^{k} } \right)
\,\, ,
%\label{expa}
\]
and the \Fd\ for a  2-dimen\-si\-on\-al hyperbolic Hamiltonian flow rewritten
as an infinite product over {\orbit}s
% Selberg-type product %\rf{pexp}
\index{Hamiltonian!flow!spectral determinant}
\index{Hamiltonian!map!spectral determinant}
\index{spectral!determinant!$2$\DOF\ hyperbolic Hamiltonian flow}
\index{spectral!determinant!$1$\DOF\ hyperbolic Hamiltonian map}
\beq
\det(\eigenvL - \Aop) =  \prod_p  \prod_{k=0}^\infty
                        \left( 1 -   t_p /\ExpaEig_p^k \right)^{k+1}
% \, , \quad\quad
% t_p =  z^{\cl{p}}  e^{\beta \Obser_p-\eigenvL \period{p}} / |\ExpaEig_p|
% t_p = {e^{-\eigenvL \period{p}} g_p \over |\ExpaEig_p| } z^{\cl{p}}
\,\,.
\label{2d-Fred} %cl_weig1}
\eeq
    \PC{2015-03-09}{state here also the $d$-dimensional result ala Gaspard}
                                            \exerbox{e_2-d_prod}
                                            \jumpBack{exam:2DHamiltFlow}
    } %end \example{Two-degree of freedom Hamiltonian flows:}{
%%%%%%%%%%%%%%%%%%%%%%%%%%%%%%%%%%%%%%%%%%%%%%%%%%%%%%%%%%%%%%%%%

%%%%%%%%%%%%%%%%%%%%%%%%%%%%%%%%%%%%%%%%%%%%%%%%%%%%%%%%%%%%%%%%%
\example{\Dzeta\ in terms of determinants,  $2$\dmn\ Hamiltonian maps:}{
                    \label{exam:Dzeta2dHamMaps}
    \PC{2018-12-13}{if edited, return to ChaosBook}
    For 2-dimen\-si\-on\-al Hamiltonian flows the above identity yields
\PC{2015-03-09}{define $\det(1-z\Lop_{(2)})$}
\[
{1\over |\ExpaEig|} =
{1 \over {|\ExpaEig|(1-1/\ExpaEig)^2}}(1 - 2/\ExpaEig+1/\ExpaEig^2)
\,,
\]
so
\beq
1/\zeta = {\det(1-z\Lop) \, \det(1-z\Lop_{(2)}) \over \det(1-z\Lop_{(1)})^2 }
\,.
\ee{Hami-rat}
\PC{2015-03-09}{compare with $1/\zeta= {F^2 \over F_{1} F_{-1} }$
    of the preceeding section}
This establishes that for nice $2$\dmn\ hyperbolic flows the
\dzeta\ is meromorphic.
\PC{2015-03-09}{write out Ruelle's alternating product for any dimensions}
\index{hyperbolic!systems}
\index{dynamical system!hyperbolic}
                                            \jumpBack{exam:Dzeta2dHamMaps}
    } %end \example{\Dzeta\ $2$\dmn\ Hamiltonian maps:}{
%%%%%%%%%%%%%%%%%%%%%%%%%%%%%%%%%%%%%%%%%%%%%%%%%%%%%%%%%%%%%%%%%

%%%%%%%%%%%%%%%%%%%%%%%%%%%%%%%%%%%%%%%%%%%%%%%%%%%%%%%%%%%%%%%%%
\example{\Dzeta s for $2$\dmn\ Hamiltonian flows:}{ \label{exam:DzetaHam}
    \PC{2018-12-13}{if edited, return to ChaosBook}
The relation \refeq{Hami-rat} is not particularly useful for our purposes.
Instead we insert the identity
\[
1= {1 \over {(1-1/\ExpaEig)^2}}
   -{2\over \ExpaEig} {1 \over {(1-1/\ExpaEig)^2}}
   + {1 \over {\ExpaEig^2}} {1 \over {(1-1/\ExpaEig)^2}}
\]
\PC{2015-03-09}{seems the same as \refeq{Hami-rat}?}
into the exponential representation \refeq{dynzeta} of $1/\zeta_k$,
and obtain
\PC{2015-03-09}{recheck, looks wrong}
\beq
1/\zeta_k = {   { \det(1-z\Lop_{(k)}) \det(1-z\Lop_{(k+2)}) }
        \over
        \det(1-z\Lop_{(k+1)})^2
        }
% 1/\zeta_k = { {F_k F_{k+2}} \over {F_{k+1}^2}}
\,\, .
\ee{doub_pole}
Even though we have no guarantee that $\det(1-z\Lop_{(k)})$ are entire, we
do know
      \PublicPrivate{
      }{% switch \PublicPrivate{
(by arguments explained in sect.~?!)
\PC{2015-03-09}{find the sect. referred to}
      }% end \PublicPrivate{
that the
upper bound on the leading zeros of $\det(1-z\Lop_{(k+1)})$ lies strictly
below the leading zeros of $\det(1-z\Lop_{(k)})$, and therefore we expect
that for 2-dimensional Hamiltonian flows the
\dzeta\  $1/\zeta_k$ generically has a {\em double} leading pole
coinciding with the leading zero of the $\det(1-z\Lop_{(k+1)})$
\Fd.  This might fail if the poles and leading eigenvalues come in
wrong order, but we have not encountered such situations in our numerical
investigations.
This result can also be stated as follows: the theorem establishes that
the \Fd\  \refeq{2d-Fred}
is entire, and also implies that the
poles in $1/\zeta_k$ must have the right multiplicities to cancel in the
$ \det(1-z\Lop) = \prod 1/\zeta_k^{k+1}$ product.
                                            \jumpBack{exam:DzetaHam}
    }   %end \example{\Dzeta s for $2$\dmn\ Hamiltonian flows}{
%%%%%%%%%%%%%%%%%%%%%%%%%%%%%%%%%%%%%%%%%%%%%%%%%%%%%%%%%%%%%%%%%
