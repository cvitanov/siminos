% siminos/spatiotemp/Examples/examReflect4.tex
% $Author: predrag $ $Date: 2021-08-10 11:56:19 -0400 (Tue, 10 Aug 2021) $

%%%%%%%%%%%%%%%%%%%%%%%%%%%%%%%%%%%%%%%%%%%%%%%%%%%%%%%%%%%%%%%%%%
\example{\Dn{4} reflection symmetric, antisymmetric
         permutation representation subspaces.}
{ \label{exam:Reflect4}
The characteristic equation $\Refl^2=1$,
with eigenvalues $\{+1,-1\}$ , enables us to start the symmetry
reduction of the $\cl{}$\dmn\ permutation representation of $\Dn{\cl{}}$
by splitting it into the reflection symmetric or antisymmetric subspaces
by means of projection operators.

When the period  $\cl{}$ of the {\lattstate}s is even, there are two
classes of reflections.
For example, when the period of the {\lattstate}s is 4, reflection
operators $\Refl$ and $\Refl_1=\Refl\shift$ (see \refexam{exam:DinftyMultTab})
belong to distinct dihedral group
$\Dn{4}$ classes:
\beq
\Refl=
\left[
\begin{array}{cccc}
 0 & 0 & 0 & 1 \\
 0 & 0 & 1 & 0 \\
 0 & 1 & 0 & 0 \\
 1 & 0 & 0 & 0 \\
\end{array}
\right]
\,,\quad
\Refl_1=
\left[
\begin{array}{cccc}
 0 & 0 & 1 & 0 \\
 0 & 1 & 0 & 0 \\
 1 & 0 & 0 & 0 \\
 0 & 0 & 0 & 1\\
\end{array}
\right] \, ,
\ee{PCreflect4}
where $\shift$ is the shift matrix.
Either one splits the
$\cl{}$\dmn\ permutation representation of $\Dn{\cl{}}$
into the reflection symmetric and antisymmetric subspaces.
For $\Refl$ the two  projection operators are
\bea
\PP_{0+} &=& \frac{\Refl - (-1) \unit}{1-(-1)} =
\frac{1}{2}
\left[
\begin{array}{cccc}
 1 & 0 & 0 & 1 \\
 0 & 1 & 1 & 0 \\
 0 & 1 & 1 & 0 \\
 1 & 0 & 0 & 1
\end{array}
\right]
                   \continue
\PP_{0-} &=& \frac{~\Refl - \unit}{-1-1} =
\frac{1}{2}
\left[
\begin{array}{cccc}
 1 & 0 & 0 & -1 \\
 0 & 1 & -1 & 0 \\
 0 & -1 & 1 & 0 \\
 -1 & 0 & 0 & 1 \\
\end{array}
\right]
\,,
\label{Reflect4Refl}
\eea
and
for $\Refl_1$ they are
\bea
\PP_{1+} &=& \frac{\unit-(-1)\Refl_1}{1+1} =
\frac{1}{2}
\left[
\begin{array}{cccc}
 1 & 0 & 1 & 0 \\
 0 & 2 & 0 & 1 \\
 1 & 0 & 1 & 0 \\
 0 & 0 & 0 & 2 \\
\end{array}
\right]
                   \continue
\PP_{1-} &=& \frac{\unit+(-1)\Refl_1}{1+1} =
\frac{1}{2}
\left[
\begin{array}{cccc}
 1 & 0 &-1 & 0 \\
 0 & 0 & 0 & 0 \\
-1 & 0 & 1 & 0 \\
 0 & 0 & 0 & 0 \\
\end{array}
\right]
\,.
\label{Reflect4reflShift}
\eea
Either splits the $\cl{}$\dmn\ permutation representation, but in a
different way. The dimensions $d_\alpha=\Tr\PP_\alpha$ of the pairs of
subspaces are
$d_{\Refl+}=2$,
$d_{\Refl-}=2$,
and
$d_{\Refl_1+}=3$,
$d_{\Refl_1-}=1$.
They are reducible further by each other, and
by the translation operator characteristic equation $\shift^4=1$.
Of course, there is no reason to single out reflection operators $\Refl$
and $\Refl_1$. For a systematic, all commuting operator approach, see
\refexam{exam:D6mult} for the Burnside, class operator full reduction.
                                        \jumpBack{exam:Reflect4}
} % end {exam:Reflect4}
%%%%%%%%%%%%%%%%%%%%%%%%%%%%%%%%%%%%%%%%%%%%%%%%%%%%%%%%%%%%%%%%%%
