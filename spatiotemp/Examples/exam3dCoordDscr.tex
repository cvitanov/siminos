% siminos/spatiotemp/Examples/exam3dCoordDscr.tex
% $Author: predrag $ $Date: 2021-07-23 16:24:01 -0400 (Fri, 23 Jul 2021) $

%%%%%%%%%%%%%%%%%%%%%%%%%%%%%%%%%%%%%%%%%%%%%%%%%%%%%%%%%%%%%%%%%%
\example{Discrete operations on \Rls{3}.}{\label{exam:3dCoordDscr}
% in examFiniteGr.tex, called by \Chapter{finiteGr}{}{Flips, slides and turns}
(Continued from \refexam{exam:3dFlowsDscr}.)          \toCB
The matrix representation of reflections, rotations
and inversions defined by \refeq{3dOrder2symm} is
 \beq
 \matrep{\Refl} =   \left(\barr{ccc}
    1  &  0 & ~0  \\
    0  &  1 & ~0 \\
    0  &  0 & -1
    \earr\right)
     ,\quad
 \matrep{\shift} =   \left(\barr{ccc}
    -1  &  ~0 & 0  \\
    ~0  &  -1 & 0 \\
    ~0  &  ~0 & 1
    \earr\right)
     ,\quad
 \matrep{P} =   \left(\barr{ccc}
    -1  &  ~0 & ~0  \\
    ~0  &  -1 & ~0 \\
    ~0  &  ~0 & -1
    \earr\right)
\,,
 \label{3dCoordDiscr}
 \eeq
with $\det{\matrep{\shift}} = 1$,
$\det{\matrep{\Refl}} = \det{\matrep{P}} = -1$; that is why we
refer to $\shift$ as a rotation, and $\Refl$, $P$ as inversions. As
$\LieEl^2 = e$ in all three cases, these are groups of order 2.
~~(continued in \refexam{exam:3dinvDscr})
                                        \jumpBack{exam:3dCoordDscr}
        } %end \example{Discrete operations on \Rls{3}.
%%%%%%%%%%%%%%%%%%%%%%%%%%%%%%%%%%%%%%%%%%%%%%%%%%%%%%%%%%%%%%%%%%
