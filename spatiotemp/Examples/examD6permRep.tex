% siminos/spatiotemp/Examples/examD6permRep.tex
% $Author: predrag $ $Date: 2021-08-10 11:56:19 -0400 (Tue, 10 Aug 2021) $

%%%%%%%%%%%%%%%%%%%%%%%%%%%%%%%%%%%%%%%%%%%%%%%%%%%%%%%%%%%%%%%%%%%%%
\example{\Dn{6} permutation rep.}{\label{exam:D6permRep}
For period-6 {\lattstate}s, the
class operators are the identity, and:
\beq
{\cal R}_{3}=
\left[
\begin{array}{cccccc}
 0 & 0 & 0 & 1 & 0 & 0 \\
 0 & 0 & 0 & 0 & 1 & 0 \\
 0 & 0 & 0 & 0 & 0 & 1 \\
 1 & 0 & 0 & 0 & 0 & 0 \\
 0 & 1 & 0 & 0 & 0 & 0 \\
 0 & 0 & 1 & 0 & 0 & 0 \\
\end{array}
\right] \,,\quad
\ee{D6R3}

\beq
{\cal R}_{1}=
\left[
\begin{array}{ccc|ccc}
 0 & 1 & 0 & 0 & 0 & 1 \\
 1 & 0 & 1 & 0 & 0 & 0 \\
 0 & 1 & 0 & 1 & 0 & 0 \\\hline
 0 & 0 & 1 & 0 & 1 & 0 \\
 0 & 0 & 0 & 1 & 0 & 1 \\
 1 & 0 & 0 & 0 & 1 & 0 \\
\end{array}
\right] \,,\quad
{\cal R}_{2}=
\left[
\begin{array}{ccc|ccc}
 0 & 0 & 1 & 0 & 1 & 0 \\
 0 & 0 & 0 & 1 & 0 & 1 \\
 1 & 0 & 0 & 0 & 1 & 0 \\\hline
 0 & 1 & 0 & 0 & 0 & 1 \\
 1 & 0 & 1 & 0 & 0 & 0 \\
 0 & 1 & 0 & 1 & 0 & 0
\end{array}
\right] \,.
\ee{D6R1R2}

\beq
{\cal S}_{0}=
\left[
\begin{array}{ccc|ccc}
 0 & 1 & 0 & 1 & 0 & 1 \\
 1 & 0 & 1 & 0 & 1 & 0 \\
 0 & 1 & 0 & 1 & 0 & 1 \\\hline
 1 & 0 & 1 & 0 & 1 & 0 \\
 0 & 1 & 0 & 1 & 0 & 1 \\
 1 & 0 & 1 & 0 & 1 & 0 \\
\end{array}
\right]
\,,\quad
{\cal S}_{1}=
\left[
\begin{array}{ccc|ccc}
 1 & 0 & 1 & 0 & 1 & 0 \\
 0 & 1 & 0 & 1 & 0 & 1 \\
 1 & 0 & 1 & 0 & 1 & 0 \\\hline
 0 & 1 & 0 & 1 & 0 & 1 \\
 1 & 0 & 1 & 0 & 1 & 0 \\
 0 & 1 & 0 & 1 & 0 & 1
\end{array}
\right] \,.
\ee{PCreflect6}
The reflection operators eigenvalues are $+1$ and $-1$,
corresponding to the reflection symmetric and antisymmetric subspaces.

To compute the dimensions of irreps obtained from
\refeq{Ham-CayD6}, we need the characters of the permutation representation
class operators:
\beq
\tr{\id}=6\,,\;
\tr{{\cal R}_{3}}=0\,,\;
\tr{{\cal R}_{1}}=0\,,\;
\tr{{\cal R}_{2}}=0\,,\;
\tr{{\cal S}_{0}}=0\,,\;
\tr{{\cal S}_{1}}=6
\,.
\ee{D6clasOpsTr}
(Predrag: I do not see why ${\cal S}_{1}$ is special...)
In particular, we have a vanishing dimension
$\lambda=-3$ representation, so
\bea
{\cal S}_{1}, \lambda=-3 &\rightarrow& \qquad
\PP_{1,-3}
    = (\id+\ensuremath{{\cal R}_{2}} - {\cal S}_{1})/2
    = 0
\,,
\label{Ham-CayD6zero}
\eea
Taking trace of \refeq{Ham-CayD6PPpm1} we find that also $\PP_{1}\PP_{0,-3}$
is 0\dmn, so, 6\dmn\ permutation representation is not faithful, and the
classes are not independent (you can check this by inspecting eqs.
\refeq{PCreflect6} to \refeq{D6R1R2}):
\bea
{\cal S}_{0} &=& {\cal R}_{3} + {\cal R}_{1}
    \continue
{\cal S}_{1} &=& \id + {\cal R}_{2}
\,,
\label{D6PermClassOps}
\eea
so forget the last two equations in \refeq{Ham-CayD6} for the $\cl{}$\dmn\
permutation representations of \Dn{\cl{}}. I think it is clear from
\refeq{Ham-CayD6PPpm1} that this means no
antisymmetric 1\dmn\ reps.

Lecturing about the "projector analysis" of \Dn{3} I was such a fool - I
forgot to follow \wwwgt{}, which explains very clearly that whenever there is
a matrix equation =0, that means a relationship between matrices, they are
not independent.

Now one can eliminate ${\cal S}_{j}$ from projection operators \refeq{Ham-CayD6S0S1PP}:
\bea
{\cal S}_{0}:\; \lambda=3 &\rightarrow& \qquad
\PP_{0,3}
    %= {\cal S}_{0}({\cal S}_{0} + 3\id)/3\cdot6
    = (\id+{\cal R}_{3}+{\cal R}_{1} + {\cal R}_{2})/6
    \continue{\cal S}_{1}:\; \lambda=3 &\rightarrow& \qquad
\PP_{1,3}
%    = {\cal S}_{0}({\cal S}_{0} - 3\id)/3\cdot6
    = (\id+{\cal R}_{2})/3
\,,
\label{am-CayD6S0S1PP1}
\eea
To summarize - this is rather inelegant, but the main result is that the flip
classes ${\cal S}_{0},{\cal S}_{1},{\cal S}_{2},\cdots$ do not contribute to
the reduction of the permutation representation; is can be done purely in
terms of the rotation classes
${\cal R}_{1},{\cal R}_{3},{\cal R}_{3},\cdots$. This strikes me as a big
deal, as this is isomorphic - I believe - to the cyclic group \Cn{\cl{}/2}
(for the even period $\cl{}$). I tentative submit \reftab{tab:D6multTabperm}
being sufficient to construct all irreducible projection operators. Of
course, \Cn{6} is the only normal subgroup of \Dn{6}, but we do not use that
- we use only 4 classes rather than the 6 of \Cn{6}. Looks pretty illegal:)

Can you check that you get 2 symmetric
1\dmn\ irreps, and the two 1\dmn\ ones?
        \jumpBack{exam:D6permRep}
} % end \example{exam:D6permRep}
%%%%%%%%%%%%%%%%%%%%%%%%%%%%%%%%%%%%%%%%%%%%%%%%%%%%%%%%%%%%%%%%%%%%%


%%%%%%%%%%%%%%%%%%%%%%%%%%%%%%%%%%%%%%%%%%%%%%%%%%%%%%%%%%%%%%%%%%%%%%
% Predrag                                                   2021-06-14
\begin{table}
\caption[]{
A tentative \Dn{6} class operator multiplication table restricted to the
permutations matrix representation, with flip classes eliminated using
\refeq{D6PermClassOps}.
    }
\begin{center}
\begin{tabular}{c|c c c c|}
\Dn{6}&\id&\ensuremath{{\cal R}_{3}}&\ensuremath{{\cal R}_{1}}&\ensuremath{{\cal R}_{2}}\\\hline
\id   &\id      &\ensuremath{{\cal R}_{3}}  &\ensuremath{{\cal R}_{1}}&\ensuremath{{\cal R}_{2}}\\
\ensuremath{{\cal R}_{3}}&.&\id&\ensuremath{{\cal R}_{2}}  &\ensuremath{{\cal R}_{1}}\\
\ensuremath{{\cal R}_{1}}&.&.&2\id+\ensuremath{{\cal R}_{2}} &2\ensuremath{{\cal R}_{3}}+\ensuremath{{\cal R}_{1}}\\
\ensuremath{{\cal R}_{2}}&.&.&.&2\id+\ensuremath{{\cal R}_{2}}\\\hline
\end{tabular}
\end{center}
  \label{tab:D6multTabperm}
\end{table}
%%%%%%%%%%%%%%%%%%%%%%%%%%%%%%%%%%%%%%%%%%%%%%%%%%%%%%%%%%%%%%%%%%%%%%
