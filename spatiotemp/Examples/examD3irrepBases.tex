% siminos/spatiotemp/Examples/examD3irrepBases.tex
% $Author: predrag $ $Date: 2021-07-25 19:52:46 -0400 (Sun, 25 Jul 2021) $

%%%%%%%%%%%%%%%%%%%%%%%%%%%%%%%%%%%%%%%%%%%%%%%%%%%%%%%%%%%%%%%%%%%%%%
\example{Bases for irreps of $\Dn{3}$.}{
  \label{exam:D3irrepBases}                                       \toCB
%% 2021-07-23 from Xiong's siminos/xiong/thesis/chapters/symGroup.tex
  (continued from \refexam{exam:D3regularRep??})~~
  We use projection operator \refeq{eq:projectSum} to obtain a
  basis of irreps of $\Dn{3}$.
  From \reftab{tab:D3charac}, we have
  \begin{align}
    P^{A}\rho(\sspRed)
    & = \frac{1}{6}
      \left[
      \rho(\sspRed) + \rho(\Refl\sspRed) + \rho(\Refl_{2}\sspRed)
      + \rho(\Refl_{1}\sspRed) + \rho(\shift \sspRed) + \rho(\shift^2 \sspRed)
      \right] \\
    P^{B}\rho(\sspRed)
    & = \frac{1}{6}
      \left[
      \rho(\sspRed) - \rho(\Refl\sspRed) - \rho(\Refl_{2}\sspRed)
      - \rho(\Refl_{1}\sspRed) + \rho(\shift \sspRed) + \rho(\shift^2 \sspRed)
      \right]
      \,.
  \end{align}
  For projection into irrep E, we need to figure out the explicit
  matrix representation first. Obviously, the following 2 by 2 matrices are E irreps.
  \begin{equation}
    D^E(e) =
    \begin{bmatrix}
      1 & 0\\
      0 & 1 \\
    \end{bmatrix} \,,\quad
    D^E(\shift) =
    \begin{bmatrix}
      \omega & 0 \\
      0 & \omega^2 \\
    \end{bmatrix}  \,,\quad
    D^E(\shift^2) =
    \begin{bmatrix}
      \omega^2 & 0 \\
      0 & \omega \\
    \end{bmatrix}  \,
    \label{eq:c3E_1}
  \end{equation}
  \begin{equation}
    D^E(\Refl) =
    \begin{bmatrix}
      0 & 1 \\
      1 & 0\\
    \end{bmatrix} \,,\quad
    D^E(\Refl_{2}) =
    \begin{bmatrix}
      0 & \omega^2 \\
      \omega & 0 \\
    \end{bmatrix}  \,,\quad
    D^E(\Refl_{1}) =
    \begin{bmatrix}
      0 & \omega  \\
      \omega^2 & 0 \\
    \end{bmatrix}  \,.
    \label{eq:c3E_2}
  \end{equation}
  So apply projection operator \refeq{eq:projecIrre} on $\rho(\sspRed)$ and
  $\rho(\Refl\sspRed)$, we get
  \begin{align}
    P^E_1\rho(\sspRed)
    & = \frac{1}{6}
      \left[
      \rho(\sspRed) + \omega \rho(\shift \sspRed) + \omega^2 \rho(\shift^2 \sspRed)
      \right] \\
    P^E_2\rho(\sspRed)
    & = \frac{1}{6}
      \left[
      \rho(\sspRed) + \omega^2 \rho(\shift \sspRed) + \omega \rho(\shift^2 \sspRed)
      \right]  \\
    P^E_1\rho(\Refl\sspRed)
    & = \frac{1}{6}
      \left[
      \rho(\Refl\sspRed) + \omega \rho(\Refl_{1} \sspRed) + \omega^2 \rho(\Refl_{2} \sspRed)
      \right] \\
    P^E_2\rho(\Refl\sspRed)
    & = \frac{1}{6}
      \left[
      \rho(\Refl\sspRed) + \omega^2 \rho(\Refl_{1} \sspRed) + \omega \rho(\Refl_{2} \sspRed)
      \,.
      \right]
  \end{align}
  The above derivation has used formulas \refeq{eq:C3relations}.
  In the invariant basis
  \[
    \left\{
      P^A\rho(\sspRed), P^B\rho(\sspRed), P^E_1\rho(\sspRed), P^E_2\rho(\Refl\sspRed),
      P^E_1\rho(\Refl\sspRed),  P^E_2\rho(\sspRed)
    \right\}
    \,,
  \]
  we have
  \[
    D^{irr}(\Refl_{2}) =
    \begin{bmatrix}
      1 & 0 & 0 & 0 & 0 & 0 \\
      0 & -1 & 0 & 0 & 0 & 0 \\
      0 & 0 & 0 & \omega^2 & 0 & 0 \\
      0 & 0 & \omega & 0 & 0 & 0 \\
      0 & 0 & 0 & 0 & 0 & \omega^2 \\
      0 & 0 & 0 & 0 & \omega & 0 \\
    \end{bmatrix}
    \quad
    D^{irr}(\shift) =
    \begin{bmatrix}
      1 & 0 & 0 & 0 & 0 & 0 \\
      0 & 1 & 0 & 0 & 0 & 0 \\
      0 & 0 & \omega & 0 & 0 & 0 \\
      0 & 0 & 0 & \omega^2 & 0 & 0 \\
      0 & 0 & 0 & 0 & \omega & 0 \\
      0 & 0 & 0 & 0 & 0 & \omega^2 \\
    \end{bmatrix}
    \,.
  \]
                                        \jumpBack{exam:D3irrepBases}
\authorXD{}
} % \example{Basis for irreps of $\Dn{3}$.}{\label{exam:D3irrepBases}
%%%%%%%%%%%%%%%%%%%%%%%%%%%%%%%%%%%%%%%%%%%%%%%%%%%%%%%%%%%%%%%%%%%%%%
