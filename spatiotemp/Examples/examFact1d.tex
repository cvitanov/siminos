% siminos/spatiotemp/Examples/examFact1d.tex
% $Author: predrag $ $Date: 2021-08-25 23:18:52 -0400 (Wed, 25 Aug 2021) $

%%%%%%%%%%%%%%%%%%%%%%%%%%%%%%%%%%%%%%%%%%%%%%%%%%%%%%%%%%%%%%%%%%%%%%%
\example{\Dn{1} factorization.}{  \label{exam:Fact1d}
% cut out of % siminos/spatiotemp/Examples/examSymm1d.tex
% was \section{$\Zn{2} = \Dn{1}$ factorization}
% was \label{s-C-2-fact} in ChaosBook - REPLACE!
% was section siminos/spatiotemp/chapter/symm1d.tex
(Continued from \refexam{exam:Symm1d})~~

Depending on the maximal symmetry group ${\cal H}_p$ that
leaves an orbit $p$ invariant (see
        \PublicPrivate{
        }{
 refsects {degene} {Dynami} as well as
        }
\refexam{exam:Reflecti}),
the contributions to the full \statesp\ \dzeta\  factor as
\bea
                     &  & ~~~~A_1~~~~~~A_2
                \continue
{\cal H}_{{p}} = \{e\}: \quad
( 1 - t_{\hat p} )^2 &  = & (1 - t_{\hat p})(1 - t_{\hat p})
                \continue
{\cal H}_{{p}} = \{e,\Refl\}: \quad
( 1 - t^2_{\hat p} ) & = & (1 - t_{\hat p}) (1 + t_{\hat p})
\,\, ,
\label{symm-isin}
\eea
For example:
\bea
                     &  & ~~~~A_1~~~~~~~~~~A_2
                \continue
{\cal H}_{{RRL}} = \{e\}: \quad
( 1 - t_{RRL} )^2 &  = & (1 - t_{001})(1 - t_{001})
                \continue
{\cal H}_{{RL}} = \{e,\Refl\}: \quad
( 1 - t_{RL} )~~ & = & ~(1 - t_{0})~~(1 + t_{0})
    \,, \quad
\mbox{ where } t_{RL}=t_{0}^2
\,.
\nonumber
\eea

The $A_1$ subspace \dzeta\ has
the same form as the full \statesp\ \pS\ binary expansion
refeq {curvbin}:
\index{cycle!fundamental}
\index{curvature!correction}
\bea
1/\zeta_{A_1} &= &  1 - t_0 - t_1
 - (t_{01} - t_1 t_0)
 - (t_{001}- t_0t_{10})
 - (t_{011}-  t_1t_{10})
               \ceq
 - (t_{0001} - t_{0}t_{001})
 - (t_{0111} - t_{1}t_{011})
               \ceq
 - (t_{0011}  -  t_{001}t_{1} - t_{0}t_{011} + t_{0}t_{01}t_{1}) -
\cdots
\,.
\label{Fact1d:curvbin}
\eea
The form is the same, however, the weights $t_{\tilde{p}}$ are different -
a symmetric orbit weight is a square root of the corresponding full
\statesp\ orbit weight. The asymmetric orbits retain the same weight,
but contribute only once.

The antisymmetric $A_2$
subspace \dzeta\ $\zeta_{A_2}$
differs from $\zeta_{A_1}$ by a minus sign for
cycles with an odd number of $0$'s:
\bea
1/\zeta_{A_2} & = & (1+t_{0})(1-t_{1})(1+t_{10})(1-t_{100})(1+t_{101})
            (1+t_{1000}) \ceq
            (1-t_{1001})(1+t_{1011})
            (1-t_{10000})(1+t_{10001}) \ceq (1+t_{10010})
            (1-t_{10011})(1-t_{10101})(1+t_{10111})
            \dots \continue
 &= &  1 + t_0 - t_1 + (t_{10} - t_1 t_0)
 - (t_{100}- t_{10} t_0)
 + (t_{101}- t_{10} t_1) \ceq
 - (t_{1001}  - t_1 t_{001} - t_{101} t_0 + t_{10}t_0 t_1) - \dots
\dots
\label{zetaA2}
\eea
Note that the group theory factors do not destroy the
curvature corrections (the cycles and pseudo cycles are
still arranged into shadowing combinations).

If the system under consideration has a boundary orbit
({\em cf.}  refsect {bound-o}) with
group-theoretic factor $h_p=(e+\sigma)/2$,
the boundary orbit does not contribute to the antisymmetric
subspace
\bea
& & \quad A_1 \quad \quad A_2   \nonumber\\
\mbox{boundary:} \quad (1-t_{p}) & = &
(1-t_{\hat{p}})(1-0 t_{\hat{p}})
\eea
This is the $1/\zeta$ part of the boundary orbit factorization
discussed in  \refexam{exam:Reflecti}, where the
factorization of the corresponding {\Fd s} for
the 1\dmn\ reflection symmetric maps is worked out in detail.
                                        \jumpBack{exam:Fact1d}
    } %end \example{exam:Fact1d}
%%%%%%%%%%%%%%%%%%%%%%%%%%%%%%%%%%%%%%%%%%%%%%%%%%%%%%%%%%%%%%%%%%%%%%%
