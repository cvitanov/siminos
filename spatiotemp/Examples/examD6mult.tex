% siminos/spatiotemp/Examples/examD6mult.tex
% $Author: predrag $ $Date: 2021-08-25 23:18:52 -0400 (Wed, 25 Aug 2021) $

%%%%%%%%%%%%%%%%%%%%%%%%%%%%%%%%%%%%%%%%%%%%%%%%%%%%%%%%%%%%%%%%%%%%%%
% Predrag                                                   2021-06-14
\begin{table}
\caption[]{
The \Dn{6} Cayley  table (group multiplication \refeq{eq:DinftyMultTab}
table), and the class operator multiplication table.
The class operator multiplication table is symmetric under transposition,
so it suffices to fill up the upper half-triangular region. The 6 classes
correspond to 4 1\dmn\ irreps, and the 2 1\dmn\ irreps.
    }
\begin{center}
\begin{tabular}{c||c|c|cc|cc|ccc|ccc|}
\Dn{6}&$1  $ &$r^3$ &$r  $ &$r^5$ &$r^2$ &$r^4$ &$s  $ &$s_2$ &$s_4$ &$s_1$ &$s_3$ &$s_5$\\\hline\hline
$1  $ &$1  $ &$r^3$ &$r  $ &$r^5$ &$r^2$ &$r^4$ &$s  $ &$s_2$ &$s_4$ &$s_1$ &$s_3$ &$s_5$\\ \hline
$r^3$ &$r^3$ &$1  $ &$r^4$ &$r^2$ &$r^5$ &$r  $ &$s_3$ &$s_5$ &$s_1$ &$s  $ &$s_1$ &$s_2$\\ \hline
$r  $ &$r  $ &$r^4$ &$r^2$ &$1  $ &$r^3$ &$r^5$ &$s_1$ &$s_3$ &$s_5$ &$s_2$ &$s_4$ &$s  $\\
$r^5$ &$r^5$ &$r^2$ &$1  $ &$r^4$ &$r  $ &$r^3$ &$s_5$ &$s_1$ &$s_3$ &$s  $ &$s_2$ &$s_4$\\ \hline
$r^2$ &$r^2$ &$r^5$ &$r^3$ &$r  $ &$r^4$ &$1  $ &$s_2$ &$s_4$ &$s  $ &$s_3$ &$s_5$ & $s_1$\\
$r^4$ &$r^4$ &$r  $ &$r^5$ &$r^3$ &$1  $ &$r^2$ &$s_4$ &$s  $ &$s_2$ &$s_5$ &$s_1$ &$s_3$\\ \hline
$s  $ &$s  $ &$s_3$ &$s_1$ &$s_5$ &$s_2$ &$s_4$ &$1  $ &$r^4$ &$r^2$ &$r^5$ &$r^3$ &$r $\\
$s_2$ &$s_2$ &$s_5$ &$s_3$ &$s_1$ &$s_4$ &$s  $ &$r^2$ &$1  $ &$r^4$ &$r  $ &$r^5$ &$r^3$\\
$s_4$ &$s_4$ &$s_2$ &$s_5$ &$s_3$ &$s  $ &$s_2$ &$r^4$ &$r^2$ &$1  $ &$r^3$ &$r  $ &$r^5$\\ \hline
$s_1$ &$s_1$ &$s_4$ &$s_2$ &$s  $ &$s_3$ &$s_5$ &$r  $ &$r^5$ &$r^3$ &$1  $ &$r^4$ &$r^2$\\
$s_3$ &$s_3$ &$s  $ &$s_4$ &$s_2$ &$s_5$ &$s_1$ &$r^3$ &$r  $ &$r^5$ &$r^2$ &$1  $ &$r^4$\\
$s_5$ &$s_5$ &$s_3$ &$s  $ &$s_4$ &$s_1$ &$s_3$ &$r^5$ &$r^3$ &$r  $ &$r^4$ &$r^2$ &$1  $\\
\hline%$[2ex]
\end{tabular}
\bigskip

\begin{tabular}{c|c c c c c c|}
\Dn{6}&\id&\ensuremath{{\cal R}_{3}}&\ensuremath{{\cal R}_{1}}&\ensuremath{{\cal R}_{2}}&\ensuremath{{\cal S}_{0}}&\ensuremath{{\cal S}_{1}}\\\hline
\id   &\id      &\ensuremath{{\cal R}_{3}}  &\ensuremath{{\cal R}_{1}}&\ensuremath{{\cal R}_{2}}      &\ensuremath{{\cal S}_{0}}&\ensuremath{{\cal S}_{1}}\\
\ensuremath{{\cal R}_{3}}&.&\id&\ensuremath{{\cal R}_{2}}  &\ensuremath{{\cal R}_{1}}&\ensuremath{{\cal S}_{1}}&\ensuremath{{\cal S}_{0}}\\
\ensuremath{{\cal R}_{1}}&.&.&2\id+\ensuremath{{\cal R}_{2}} &2\ensuremath{{\cal R}_{3}}+\ensuremath{{\cal R}_{1}} &2\ensuremath{{\cal S}_{1}}&2\ensuremath{{\cal S}_{0}}\\
\ensuremath{{\cal R}_{2}}&.&.&.&2\id+\ensuremath{{\cal R}_{2}} &2\ensuremath{{\cal S}_{0}}&2\ensuremath{{\cal S}_{1}}\\
\ensuremath{{\cal S}_{0}}&.&.&.&.&3(\id+\ensuremath{{\cal R}_{2}})&3(\ensuremath{{\cal R}_{3}}+\ensuremath{{\cal R}_{1}})\\
\ensuremath{{\cal S}_{1}}&.&.&.&.&.&3(\id+\ensuremath{{\cal R}_{2}})\\\hline
\end{tabular}
\end{center}
  \label{tab:D6multTab}
\end{table}
%%%%%%%%%%%%%%%%%%%%%%%%%%%%%%%%%%%%%%%%%%%%%%%%%%%%%%%%%%%%%%%%%%%%%%

%%%%%%%%%%%%%%%%%%%%%%%%%%%%%%%%%%%%%%%%%%%%%%%%%%%%%%%%%%%%%%%%%%%%%
\example{\Dn{6} multiplication tables.}{\label{exam:D6mult}
From the \Dn{6} class operator multiplication table follow the
Hamilton-Cayley equations (for any matrix representation; in our application
\refeq{PCreflect} that is the 6\dmn\ matrix
representation of permutations), with 16 eigenvalues as listed,
    \PC{2021-06-16}{${\cal R}_{2}$ is a guess, I have not derived it.}
\bea
({\cal R}_{3} - \id)({\cal R}_{3} + \id) &=& 0
    \continue
({\cal R}_{1} - \id)({\cal R}_{1} + \id)
({\cal R}_{1} - 2\id)({\cal R}_{1} + 2\id) &=& 0
    \continue
({\cal R}_{2} - \id)({\cal R}_{2} + \id)
({\cal R}_{2} - 2\id)({\cal R}_{2} + 2\id) &=& 0
    \continue
{\cal S}_{0}
({\cal S}_{0} - 3\id)({\cal S}_{0} + 3\id) &=& 0
    \continue
{\cal S}_{1}
({\cal S}_{1} - 3\id)({\cal S}_{1} + 3\id) &=& 0
\,,
\label{Ham-CayD6}
\eea
so there is lots of redundancy - there are only 6 irreps.
\bea
{\cal R}_{3}:\; \lambda=1 &\rightarrow& \qquad
\PP_{1} = (\id+{\cal R}_{3})/2
     \continue
{\cal R}_{3}:\; \lambda=-1 &\rightarrow& \qquad
\PP_{-1} = (\id-{\cal R}_{3})/2
     \continue
{\cal S}_{0}:\; \lambda=0 &\rightarrow& \qquad
\PP_{0,0}
%    = ({\cal S}_{0} - 3\id)({\cal S}_{0} + 3\id)/9
%    = (9\id -{\cal S}_{0}^2)/9
    =  (2\id - {\cal R}_{2})/3
    \continue
{\cal S}_{0}:\; \lambda=3 &\rightarrow& \qquad
\PP_{0,3}
    %= {\cal S}_{0}({\cal S}_{0} + 3\id)/3\cdot6
    = (\id+{\cal R}_{2}+{\cal S}_{0})/6
     \continue
{\cal S}_{0}:\; \lambda=-3 &\rightarrow& \qquad
\PP_{0,-3}
%    = {\cal S}_{0}({\cal S}_{0} - 3\id)/3\cdot6
    = (\id+{\cal R}_{2} - {\cal S}_{0})/6
     \continue
{\cal S}_{1}:\; \lambda=0 &\rightarrow& \qquad
\PP_{1,0}
%    = ({\cal S}_{0} - 3\id)({\cal S}_{0} + 3\id)/9
%   = (9\id -{\cal S}_{1}^2)/9
%   =  (2\id - {\cal R}_{2})/3
    = \PP_{0:\;0}
     \continue
{\cal S}_{1}:\; \lambda=3 &\rightarrow& \qquad
\PP_{1,3}
%    = {\cal S}_{0}({\cal S}_{0} - 3\id)/3\cdot6
    = (\id+{\cal R}_{2} + {\cal S}_{1})/6
   \continue
{\cal S}_{1}:\; \lambda=-3 &\rightarrow& \qquad
\PP_{1,-3}
    = (\id+\ensuremath{{\cal R}_{2}} - {\cal S}_{1})/3
\,.
\label{Ham-CayD6S0S1PP}
\eea
Split $\PP_{0,-3}$ using $\PP_{1}$:
\bea
\PP_{1}\PP_{0,-3}
%    &=& (\id+{\cal R}_{3})(\id+{\cal R}_{2} - {\cal S}_{0})/2.6
%     \continue
    &=& (\id+{\cal R}_{3}+{\cal R}_{1}+{\cal R}_{2} - {\cal S}_{0} - {\cal S}_{1})/12
\,.
\label{Ham-CayD6PPpm1}
\eea


${\cal S}_{j}$ equations are the same form as for $\Dn{3}$ 1\dmn\ irrep, so
the number of such equations presumably equals the number of 1\dmn\ irrep,
and the same for ${\cal R}_{j}$, $j\neq n/2$ equations.

${\cal S}_{j}$ equations presumably contain symmetric/antisymmetric 
solutions, in the spirit of \refeq{HLreflect6} and \refeq{HLreflect6other}.

For even dimensions
${\cal R}_{n/2}$ presumably leads to 4 1\dmn\ irreps, of which I assume the
two antisymmetric ones do not contribute to the$n$\dmn\ matrix representation
of permutations, while all 1\dmn\ irrep do.

That is probably easier to count using the character formulas.

       \jumpBack{exam:D6mult}
    } % end {exam:exam:D6mult}
%%%%%%%%%%%%%%%%%%%%%%%%%%%%%%%%%%%%%%%%%%%%%%%%%%%%%%%%%%%%%%%%%%%%%
