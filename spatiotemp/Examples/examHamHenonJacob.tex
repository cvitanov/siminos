% examHamHenonJacob.tex
% $Author: predrag $ $Date: 2021-09-02 23:44:14 -0400 (Thu, 02 Sep 2021) $

% Predrag extracted from                    2021-02-15
%                   ChaosBook.org
% based on                                  2021-04-27
%    \example{{\HenonMap} \jacobianM.}{ \label{exam:HenonJacob}
%    \example{Stability of cycles for maps.}{ \label{exam:cycStabMap}
%    siminos/kittens/cat.tex                2021-04-27
%%%%%%%%%%%%%%%%%%%%%%%%%%%%%%%%%%%%%%%%%%%%%%%%%%%%%%%%%%%%%%%%%%%%%%%%%%
\example{{\Henlatt} stability.}{ \label{exam:HamHenonJacob}
For the {\HenonMap} \refeq{Henlatt-eq2.1.0} the \emph{temporal evolution
\jacobianM} for the $n$th iterate of the map is the product of
consecutive  one time-step \jacobianMs\
\index{Henon@H\'enon map!stability}    \index{jacobian!H\'enon map}
\beq
\jMps^n(\field_0) =
\prod_{m=n}^{1}
          \MatrixII{-2\,\field_m}{-1}
                   {1}  {0}
\,,\qquad \field_m = \map^{m}_1 (\field_0,p_0)
\,.
\ee{Henlatt-e_her}
The decreasing order in the indices of the products in above formulas is
a reminder that the successive time steps correspond to multiplication
from the left,
$\jMps_p(\field_1)  = \jMps(\field_\cl{p})\cdots \jMps(\field_1)$.

The determinant of the {\Henon} one time-step \jacobianM\ in
\refeq{Henlatt-e_her} is a constant,
\beq
\det\jMps =
 1
\,,
\ee{Henlatt-HenDet}
so the map is Hamiltonian (symplectic) in the sense that it
preserves areas in the $[\field,p]$ plane.

The \FloquetM\ $\jMps_p$ for a {\orbit} $p$ of length
$\cl{p}$ of the {\HenonMap} \refeq{Henlatt-eq2.1.0} is evaluated by
picking any periodic point as a starting point, running once around a
{\orbit}, and multiplying the individual periodic point {\jacobianMs}
\refeq{Henlatt-e_her},
% restore     \ according to \refeq{jacoB}.
\index{Henon@H\'enon map!stability}
\beq
\jMps_p(\xInit) = \prod_{k=\cl{p}}^{1}
          \MatrixII{-2\field_k}{-1}
                   {1}  {0}
\,,\qquad \field_k \in \pS_p
\,,
\ee{Henlatt-Floq}
Once we have a {\po} of {\HenonMap}, we also have its \FloquetM.
Only the expanding eigen\-value $\ExpaEig_1=1/\ExpaEig_2$ needs to be
determined, as $\det\jMps = \ExpaEig_1 \ExpaEig_2=1$.

The {\em \jacobianOrb} is the $\delta/\delta\field_k$ derivative of the
{\henlatt} \refeq{Henlatt-2-step} 3-term recurrence relation
\bea
\jMorb_p &=&\sigma+2\,{\mathbb{X}}_p+\sigma^{-1}
\continue
 &=&
\begin{bmatrix}
2\,\field_0 & 1 & 0 & 0 & \dots & 0 & 1\\
1 & 2\,\field_1 & 1 & 0 & \dots & 0 & 0\\
0 & 1 & 2\,\field_2 & 1 & \dots & 0 & 0\\
\vdots & \vdots & \vdots & \vdots & \ddots & \vdots & \vdots\\
0 & 0 & \dots & \dots & \dots & 2\,\field_{\cl{}-2} & 1\\
1 & 0 & \dots & \dots & \dots & 1 & 2\,\field_{\cl{}-1}
\end{bmatrix}
\,,
\label{Henlatt-orbitJac}
\eea
where ${\mathbb{X}}_p$ is a diagonal matrix with $p$-{\lattstate} $\field_k$ in the
$k$th row/column, and the `$1$'s in the upper right and lower left corners
enforce the periodic boundary conditions.

The trace of the {\jacobianOrb} is twice the orbital sum\rf{Gallas20a}
\beq
\sigma_p = \sum_{i \in p} \field_{p,i}
\ee{Henlatt-COMdef}
% also \refeq{COMdef}
a \emph{prime} cycle $p$ \emph{invariant} that satisfies a
polynomial equation
${\mathbb S}_\cl{}(\sigma)=0$
of the order $\cl{p}$, the period of the cycle.
    \PC{2021-05-04}{Perhaps include the example \refeq{Gallas20a:S_5}?}

The two fixed points \refeq{Henlatt-exer:eq2} are hyperbolic for $a>3$,
with expanding eigenvalues
 % from \refeq{COM:eq3}.
\bea
      \ExpaEig_{0}  &=&  1+\sqrt{1+a}+ (1+a)^{1/4}\sqrt{\sqrt{1+a}+2}
                        \continue
      \ExpaEig_{1}  &=&  1-\sqrt{1+a}- (1+a)^{1/4}\sqrt{\sqrt{1+a}-2}
\,,
\label{Henlatt-COM:eq3}
\eea
\index{Henon@H\'enon map!fixed points}

%%%%% based on siminos/kittens/cat.tex                    2021-04-27 %%%%%
The action of the \henlatt\ {\jacobianOrb} can be hard to visualize,
as a period-2 {\lattstate} is a 2-torus,
period-3 {\lattstate} a 3-torus, \etc. Still, the {\fundPip} for the period-2
and period-3 {\lattstate}s, should suffice to
convey the idea. The {\fundPip} basis vectors \refeq{lattJac} are the
columns of $\jMorb$. The $[2\!\times\!2]$ {\jacobianOrb}
and its {\HillDet} follow from \refeq{Henlatt-2-cycle}
\beq
\jMorb =
 \left(\begin{array}{cc}
2\,\field_0 & 2 \\
          2 & 2\,\field_1
 \end{array} \right)
\,,\quad
\Det\jMorb = 4\,(\field_0\field_1-1)
           = -4\,(a-3)
\,.
\ee{Henlatt-catFundPar2}
% $( 1+ \sqrt{a-3} )( 1- \sqrt{a-3} )-1 = -a+3$
The resulting {\fundPip} shown in \reffig{fig:Henlatt-catCycJacob}\,(a).
Period-3
{\lattstate}s for $s=3$ are contained in the half-open {\fundPip} of
\reffig{fig:Henlatt-catCycJacob}\,(b),
defined by the columns of $[3\!\times\!3]$
{\jacobianOrb}
\beq
\jMorb =
\left(
\begin{array}{ccc}
2\,\field_0 & 1           & 1 \\
          1 & 2\,\field_1 & 1 \\
          1 & 1           & 2\,\field_2
\end{array}
\right)
\,,
\qquad
\Det \jMorb
    = 8\,\field_0\field_1\field_2-2\,(\field_0+\field_2+\field_3)+2
\,,
\label{Henlatt-catFundPar3}
\eeq
and for an period-$\cl{}$ {\lattstate},
    \PC{2021-05-04}{A guess, {\color{red}absolutely wrong - Fix!}}
\beq
\Det \jMorb
    = 2^\cl{}\field_0\field_1\field_2\cdots\field_{\cl{}-1}
    \,,\qquad \cl{}>3
\,.
\label{Henlatt-DetjMorb-n}
\eeq


% restore     \jumpBack{exam:HamHenonJacob}
% restore     \PC{explain \Henon\ = normal form in conjug.tex}
    } % end \example{exam:HamHenonJacob}
%%%%%%%%%%%%%%%%%%%%%%%%%%%%%%%%%%%%%%%%%%%%%%%%%%%%%%%%%%%%%%%%%%%%%%%%%%
