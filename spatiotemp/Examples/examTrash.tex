% siminos/spatiotemp/Examples/examD3permRep.tex
% $Author: predrag $ $Date: 2021-07-24 23:53:43 -0400 (Sat, 24 Jul 2021) $

%%%%%%%%%%%%%%%%%%%%%%%%%%%%%%%%%%%%%%%%%%%%%%%%%%%%%%%%%%%%%%%%%%%%%
\example{\Dn{3} multiplication tables and the permutation rep.}{\label{exam:D3permRep}
For period-3 lattice states, the
class operators are the identity $\id$ and
\beq
{\cal R}=
\left[
\begin{array}{ccc}
 0 & 1 & 1 \\
 1 & 0 & 1 \\
 1 & 1 & 0
\end{array}
\right] \,,\quad
{\cal S}=
\left[
\begin{array}{ccc}
 1 & 1 & 1 \\
 1 & 1 & 1 \\
 1 & 1 & 1
\end{array}
\right] = \id+{\cal R}
\,,
\ee{D3RS}
so either ${\cal R}$ or ${\cal S}$ can be eliminated from the class
multiplication \reftab{tab:D3multTab}. In the spirit of the presentation of a
dihedral group in terms of two flips, let's eliminate ${\cal R}={\cal S}-\id$:
\beq
\begin{tabular}{c|cc|}
\Dn{3}      &\id        &${\cal S}$\\\hline
\id         &\id        &${\cal S}$\\
${\cal S}$  &${\cal S}$ &$3\,{\cal S}$\\\hline
\end{tabular}
\ee{tab:D3multClasPerm}
From this \Dn{3} class operator multiplication table follows the
Hamilton-Cayley equation
for its 3\dmn\
permutation rep, with two eigenvalues,
\beq
{\cal S}({\cal S} - 3\id) = 0
\,,
\label{D3Ham-CayPerm}
\eeq
with projection operators
\beq
\begin{tabular}{l|rl|r}
$\lambda$&               & projection op.        &$d$ \\\hline
3        &$\PP_{3} =$    &${\cal S}/3$           &1 \\
0        &$\PP_{0} =$    &$(3\id - {\cal S})/3$  &2 \\\hline
\end{tabular}
\ee{tab:Ham-CayD3proj}
Note that the zero-eigenvalue $\PP_{0}$ is the Laplacian operator.

Take {\jacobianOrb} of form common to both the \templatt\
\refeq{jMorbInfin} and the \henlatt\ \refeq{Henlatt-orbitJac}, and
use the spectral resolution $\id=\PP_{0}+\PP_{3}$:
\beq
d=3 \qquad\qquad
\jMorb=
\left[
\begin{array}{ccc}
-\jMorb_{00} & 1           & 1 \\
          1  &-\jMorb_{11} & 1 \\
          1  & 1           & -\jMorb_{22}
\end{array}
\right] = \PP_{0} - (\jMorb-\id)
\,,
\ee{D3orbJac}

Study also
\HREF{https://en.wikipedia.org/wiki/Permutation_representation\#Character_of_the_permutation_representation}
{wiki: {\em Character of the permutation representation}}.

Dixon, J. D. and Mortimer, {\em Permutation Groups}, Springer
%ISBN 9781461207313.

        \jumpBack{exam:D3permRep}
    } % end {exam:D3permRep}
%%%%%%%%%%%%%%%%%%%%%%%%%%%%%%%%%%%%%%%%%%%%%%%%%%%%%%%%%%%%%%%%%%%%%

% siminos/spatiotemp/Examples/examD6mult.tex
% $Author: predrag $ $Date: 2021-07-24 23:53:43 -0400 (Sat, 24 Jul 2021) $

%%%%%%%%%%%%%%%%%%%%%%%%%%%%%%%%%%%%%%%%%%%%%%%%%%%%%%%%%%%%%%%%%%%%%%
% Predrag                                                   2021-06-14
\begin{table}
\caption[]{
The \Dn{6} Cayley  table (group multiplication \refeq{D_nComp} table), and
the class operator multiplication table.
The class operator multiplication table is symmetric under transposition,
so it suffices to fill up the upper half-triangular region. The 6 classes
correspond to 4 1\dmn\ irreps, and the 2 1\dmn\ irreps.
    }
\begin{center}
\begin{tabular}{c||c|c|cc|cc|ccc|ccc|}
\Dn{6}&$1  $ &$r^3$ &$r  $ &$r^5$ &$r^2$ &$r^4$ &$s  $ &$s_2$ &$s_4$ &$s_1$ &$s_3$ &$s_5$\\\hline\hline
$1  $ &$1  $ &$r^3$ &$r  $ &$r^5$ &$r^2$ &$r^4$ &$s  $ &$s_2$ &$s_4$ &$s_1$ &$s_3$ &$s_5$\\ \hline
$r^3$ &$r^3$ &$1  $ &$r^4$ &$r^2$ &$r^5$ &$r  $ &$s_3$ &$s_5$ &$s_1$ &$s  $ &$s_1$ &$s_2$\\ \hline
$r  $ &$r  $ &$r^4$ &$r^2$ &$1  $ &$r^3$ &$r^5$ &$s_1$ &$s_3$ &$s_5$ &$s_2$ &$s_4$ &$s  $\\
$r^5$ &$r^5$ &$r^2$ &$1  $ &$r^4$ &$r  $ &$r^3$ &$s_5$ &$s_1$ &$s_3$ &$s  $ &$s_2$ &$s_4$\\ \hline
$r^2$ &$r^2$ &$r^5$ &$r^3$ &$r  $ &$r^4$ &$1  $ &$s_2$ &$s_4$ &$s  $ &$s_3$ &$s_5$ & $s_1$\\
$r^4$ &$r^4$ &$r  $ &$r^5$ &$r^3$ &$1  $ &$r^2$ &$s_4$ &$s  $ &$s_2$ &$s_5$ &$s_1$ &$s_3$\\ \hline
$s  $ &$s  $ &$s_3$ &$s_1$ &$s_5$ &$s_2$ &$s_4$ &$1  $ &$r^4$ &$r^2$ &$r^5$ &$r^3$ &$r $\\
$s_2$ &$s_2$ &$s_5$ &$s_3$ &$s_1$ &$s_4$ &$s  $ &$r^2$ &$1  $ &$r^4$ &$r  $ &$r^5$ &$r^3$\\
$s_4$ &$s_4$ &$s_2$ &$s_5$ &$s_3$ &$s  $ &$s_2$ &$r^4$ &$r^2$ &$1  $ &$r^3$ &$r  $ &$r^5$\\ \hline
$s_1$ &$s_1$ &$s_4$ &$s_2$ &$s  $ &$s_3$ &$s_5$ &$r  $ &$r^5$ &$r^3$ &$1  $ &$r^4$ &$r^2$\\
$s_3$ &$s_3$ &$s  $ &$s_4$ &$s_2$ &$s_5$ &$s_1$ &$r^3$ &$r  $ &$r^5$ &$r^2$ &$1  $ &$r^4$\\
$s_5$ &$s_5$ &$s_3$ &$s  $ &$s_4$ &$s_1$ &$s_3$ &$r^5$ &$r^3$ &$r  $ &$r^4$ &$r^2$ &$1  $\\
\hline%$[2ex]
\end{tabular}
\bigskip

\begin{tabular}{c|c c c c c c|}
\Dn{6}&\id&\ensuremath{{\cal R}_{3}}&\ensuremath{{\cal R}_{1}}&\ensuremath{{\cal R}_{2}}&\ensuremath{{\cal S}_{0}}&\ensuremath{{\cal S}_{1}}\\\hline
\id   &\id      &\ensuremath{{\cal R}_{3}}  &\ensuremath{{\cal R}_{1}}&\ensuremath{{\cal R}_{2}}      &\ensuremath{{\cal S}_{0}}&\ensuremath{{\cal S}_{1}}\\
\ensuremath{{\cal R}_{3}}&.&\id&\ensuremath{{\cal R}_{2}}  &\ensuremath{{\cal R}_{1}}&\ensuremath{{\cal S}_{1}}&\ensuremath{{\cal S}_{0}}\\
\ensuremath{{\cal R}_{1}}&.&.&2\id+\ensuremath{{\cal R}_{2}} &2\ensuremath{{\cal R}_{3}}+\ensuremath{{\cal R}_{1}} &2\ensuremath{{\cal S}_{1}}&2\ensuremath{{\cal S}_{0}}\\
\ensuremath{{\cal R}_{2}}&.&.&.&2\id+\ensuremath{{\cal R}_{2}} &2\ensuremath{{\cal S}_{0}}&2\ensuremath{{\cal S}_{1}}\\
\ensuremath{{\cal S}_{0}}&.&.&.&.&3(\id+\ensuremath{{\cal R}_{2}})&3(\ensuremath{{\cal R}_{3}}+\ensuremath{{\cal R}_{1}})\\
\ensuremath{{\cal S}_{1}}&.&.&.&.&.&3(\id+\ensuremath{{\cal R}_{2}})\\\hline
\end{tabular}
\end{center}
  \label{tab:D6multTab}
\end{table}
%%%%%%%%%%%%%%%%%%%%%%%%%%%%%%%%%%%%%%%%%%%%%%%%%%%%%%%%%%%%%%%%%%%%%%

%%%%%%%%%%%%%%%%%%%%%%%%%%%%%%%%%%%%%%%%%%%%%%%%%%%%%%%%%%%%%%%%%%%%%
\example{\Dn{6} multiplication tables.}{\label{exam:D6mult}
From the \Dn{6} class operator multiplication table follow the
Hamilton-Cayley equations (for any matrix representation; in our application
\refeq{PCreflect} that is the 6\dmn\ matrix
representation of permutations), with 16 eigenvalues as listed,
    \PC{2021-06-16}{${\cal R}_{2}$ is a guess, I have not derived it.}
\bea
({\cal R}_{3} - \id)({\cal R}_{3} + \id) &=& 0
    \continue
({\cal R}_{1} - \id)({\cal R}_{1} + \id)
({\cal R}_{1} - 2\id)({\cal R}_{1} + 2\id) &=& 0
    \continue
({\cal R}_{2} - \id)({\cal R}_{2} + \id)
({\cal R}_{2} - 2\id)({\cal R}_{2} + 2\id) &=& 0
    \continue
{\cal S}_{0}
({\cal S}_{0} - 3\id)({\cal S}_{0} + 3\id) &=& 0
    \continue
{\cal S}_{1}
({\cal S}_{1} - 3\id)({\cal S}_{1} + 3\id) &=& 0
\,,
\label{Ham-CayD6}
\eea
so there is lots of redundancy - there are only 6 irreps.
\bea
{\cal R}_{3}:\; \lambda=1 &\rightarrow& \qquad
\PP_{1} = (\id+{\cal R}_{3})/2
     \continue
{\cal R}_{3}:\; \lambda=-1 &\rightarrow& \qquad
\PP_{-1} = (\id-{\cal R}_{3})/2
     \continue
{\cal S}_{0}:\; \lambda=0 &\rightarrow& \qquad
\PP_{0,0}
%    = ({\cal S}_{0} - 3\id)({\cal S}_{0} + 3\id)/9
%    = (9\id -{\cal S}_{0}^2)/9
    =  (2\id - {\cal R}_{2})/3
    \continue
{\cal S}_{0}:\; \lambda=3 &\rightarrow& \qquad
\PP_{0,3}
    %= {\cal S}_{0}({\cal S}_{0} + 3\id)/3\cdot6
    = (\id+{\cal R}_{2}+{\cal S}_{0})/6
     \continue
{\cal S}_{0}:\; \lambda=-3 &\rightarrow& \qquad
\PP_{0,-3}
%    = {\cal S}_{0}({\cal S}_{0} - 3\id)/3\cdot6
    = (\id+{\cal R}_{2} - {\cal S}_{0})/6
     \continue
{\cal S}_{1}:\; \lambda=0 &\rightarrow& \qquad
\PP_{1,0}
%    = ({\cal S}_{0} - 3\id)({\cal S}_{0} + 3\id)/9
%   = (9\id -{\cal S}_{1}^2)/9
%   =  (2\id - {\cal R}_{2})/3
    = \PP_{0:\;0}
     \continue
{\cal S}_{1}:\; \lambda=3 &\rightarrow& \qquad
\PP_{1,3}
%    = {\cal S}_{0}({\cal S}_{0} - 3\id)/3\cdot6
    = (\id+{\cal R}_{2} + {\cal S}_{1})/6
   \continue
{\cal S}_{1}:\; \lambda=-3 &\rightarrow& \qquad
\PP_{1,-3}
    = (\id+\ensuremath{{\cal R}_{2}} - {\cal S}_{1})/3
\,.
\label{Ham-CayD6S0S1PP}
\eea
Split $\PP_{0,-3}$ using $\PP_{1}$:
\bea
\PP_{1}\PP_{0,-3}
%    &=& (\id+{\cal R}_{3})(\id+{\cal R}_{2} - {\cal S}_{0})/2.6
%     \continue
    &=& (\id+{\cal R}_{3}+{\cal R}_{1}+{\cal R}_{2} - {\cal S}_{0} - {\cal S}_{1})/12
\,.
\label{Ham-CayD6PPpm1}
\eea


${\cal S}_{j}$ equations are the same form as for $\Dn{3}$ 1\dmn\ irrep, so
the number of such equations presumably equals the number of 1\dmn\ irrep,
and the same for ${\cal R}_{j}$, $j\neq n/2$ equations.

${\cal S}_{j}$ equations presumably contain symmetric/antisymmetric self-dual
solutions, in the spirit of \refeq{HLreflect6} and \refeq{HLreflect6other}.

For even dimensions
${\cal R}_{n/2}$ presumably leads to 4 1\dmn\ irreps, of which I assume the
two antisymmetric ones do not contribute to the$n$\dmn\ matrix representation
of permutations, while all 1\dmn\ irrep do.

That is probably easier to count using the character formulas.

       \jumpBack{exam:D6mult}
    } % end {exam:exam:D6mult}
%%%%%%%%%%%%%%%%%%%%%%%%%%%%%%%%%%%%%%%%%%%%%%%%%%%%%%%%%%%%%%%%%%%%%

% siminos/spatiotemp/Examples/examD6permRep.tex
% $Author: predrag $ $Date: 2021-07-24 23:53:43 -0400 (Sat, 24 Jul 2021) $

%%%%%%%%%%%%%%%%%%%%%%%%%%%%%%%%%%%%%%%%%%%%%%%%%%%%%%%%%%%%%%%%%%%%%
\example{\Dn{6} permutation rep.}{\label{exam:D6permRep}
For period-6 lattice states, the
class operators are the identity, and:
\beq
{\cal R}_{3}=
\left[
\begin{array}{cccccc}
 0 & 0 & 0 & 1 & 0 & 0 \\
 0 & 0 & 0 & 0 & 1 & 0 \\
 0 & 0 & 0 & 0 & 0 & 1 \\
 1 & 0 & 0 & 0 & 0 & 0 \\
 0 & 1 & 0 & 0 & 0 & 0 \\
 0 & 0 & 1 & 0 & 0 & 0 \\
\end{array}
\right] \,,\quad
\ee{D6R3}

\beq
{\cal R}_{1}=
\left[
\begin{array}{ccc|ccc}
 0 & 1 & 0 & 0 & 0 & 1 \\
 1 & 0 & 1 & 0 & 0 & 0 \\
 0 & 1 & 0 & 1 & 0 & 0 \\\hline
 0 & 0 & 1 & 0 & 1 & 0 \\
 0 & 0 & 0 & 1 & 0 & 1 \\
 1 & 0 & 0 & 0 & 1 & 0 \\
\end{array}
\right] \,,\quad
{\cal R}_{2}=
\left[
\begin{array}{ccc|ccc}
 0 & 0 & 1 & 0 & 1 & 0 \\
 0 & 0 & 0 & 1 & 0 & 1 \\
 1 & 0 & 0 & 0 & 1 & 0 \\\hline
 0 & 1 & 0 & 0 & 0 & 1 \\
 1 & 0 & 1 & 0 & 0 & 0 \\
 0 & 1 & 0 & 1 & 0 & 0
\end{array}
\right] \,.
\ee{D6R1R2}

\beq
{\cal S}_{0}=
\left[
\begin{array}{ccc|ccc}
 0 & 1 & 0 & 1 & 0 & 1 \\
 1 & 0 & 1 & 0 & 1 & 0 \\
 0 & 1 & 0 & 1 & 0 & 1 \\\hline
 1 & 0 & 1 & 0 & 1 & 0 \\
 0 & 1 & 0 & 1 & 0 & 1 \\
 1 & 0 & 1 & 0 & 1 & 0 \\
\end{array}
\right]
\,,\quad
{\cal S}_{1}=
\left[
\begin{array}{ccc|ccc}
 1 & 0 & 1 & 0 & 1 & 0 \\
 0 & 1 & 0 & 1 & 0 & 1 \\
 1 & 0 & 1 & 0 & 1 & 0 \\\hline
 0 & 1 & 0 & 1 & 0 & 1 \\
 1 & 0 & 1 & 0 & 1 & 0 \\
 0 & 1 & 0 & 1 & 0 & 1
\end{array}
\right] \,.
\ee{PCreflect6}
The reflection operators eigenvalues are $+1$ and $-1$,
corresponding to the reflection symmetric and anti-symmetric subspaces.

To compute the dimensions of irreps obtained from
\refeq{Ham-CayD6}, we need the characters of the permutation representation
class operators:
\beq
\tr{\id}=6\,,\;
\tr{{\cal R}_{3}}=0\,,\;
\tr{{\cal R}_{1}}=0\,,\;
\tr{{\cal R}_{2}}=0\,,\;
\tr{{\cal S}_{0}}=0\,,\;
\tr{{\cal S}_{1}}=6
\,.
\ee{D6clasOpsTr}
(Predrag: I do not see why ${\cal S}_{1}$ is special...)
In particular, we have a vanishing dimension
$\lambda=-3$ representation, so
\bea
{\cal S}_{1}, \lambda=-3 &\rightarrow& \qquad
\PP_{1,-3}
    = (\id+\ensuremath{{\cal R}_{2}} - {\cal S}_{1})/2
    = 0
\,,
\label{Ham-CayD6zero}
\eea
Taking trace of \refeq{Ham-CayD6PPpm1} we find that also $\PP_{1}\PP_{0,-3}$
is 0\dmn, so, 6\dmn\ permutation representation is not faithful, and the
classes are not independent (you can check this by inspecting eqs.
\refeq{PCreflect6} to \refeq{D6R1R2}):
\bea
{\cal S}_{0} &=& {\cal R}_{3} + {\cal R}_{1}
    \continue
{\cal S}_{1} &=& \id + {\cal R}_{2}
\,,
\label{D6PermClassOps}
\eea
so forget the last two equations in \refeq{Ham-CayD6} for the $\cl{}$\dmn\
permutation representations of \Dn{\cl{}}. I think it is clear from
\refeq{Ham-CayD6PPpm1} that this means no
antisymmetric 1\dmn\ reps.

Lecturing about the "projector analysis" of \Dn{3} I was such a fool - I
forgot to follow \wwwgt{}, which explains very clearly that whenever there is
a matrix equation =0, that means a relationship between matrices, they are
not independent.

Now one can eliminate ${\cal S}_{j}$ from projection operators \refeq{Ham-CayD6S0S1PP}:
\bea
{\cal S}_{0}:\; \lambda=3 &\rightarrow& \qquad
\PP_{0,3}
    %= {\cal S}_{0}({\cal S}_{0} + 3\id)/3\cdot6
    = (\id+{\cal R}_{3}+{\cal R}_{1} + {\cal R}_{2})/6
    \continue{\cal S}_{1}:\; \lambda=3 &\rightarrow& \qquad
\PP_{1,3}
%    = {\cal S}_{0}({\cal S}_{0} - 3\id)/3\cdot6
    = (\id+{\cal R}_{2})/3
\,,
\label{am-CayD6S0S1PP1}
\eea
To summarize - this is rather inelegant, but the main result is that the flip
classes ${\cal S}_{0},{\cal S}_{1},{\cal S}_{2},\cdots$ do not contribute to
the reduction of the permutation representation; is can be done purely in
terms of the rotation classes
${\cal R}_{1},{\cal R}_{3},{\cal R}_{3},\cdots$. This strikes me as a big
deal, as this is isomorphic - I believe - to the cyclic group \Cn{\cl{}/2}
(for the even period $\cl{}$). I tentative submit \reftab{tab:D6multTabperm}
being sufficient to construct all irreducible projection operators. Of
course, \Cn{6} is the only normal subgroup of \Dn{6}, but we do not use that
- we use only 4 classes rather than the 6 of \Cn{6}. Looks pretty illegal:)

Can you check that you get 2 symmetric
1\dmn\ irreps, and the two 1\dmn\ ones?
        \jumpBack{exam:D6permRep}
} % end \example{exam:D6permRep}
%%%%%%%%%%%%%%%%%%%%%%%%%%%%%%%%%%%%%%%%%%%%%%%%%%%%%%%%%%%%%%%%%%%%%


%%%%%%%%%%%%%%%%%%%%%%%%%%%%%%%%%%%%%%%%%%%%%%%%%%%%%%%%%%%%%%%%%%%%%%
% Predrag                                                   2021-06-14
\begin{table}
\caption[]{
A tentative \Dn{6} class operator multiplication table restricted to the
permutations matrix representation, with flip classes eliminated using
\refeq{D6PermClassOps}.
    }
\begin{center}
\begin{tabular}{c|c c c c|}
\Dn{6}&\id&\ensuremath{{\cal R}_{3}}&\ensuremath{{\cal R}_{1}}&\ensuremath{{\cal R}_{2}}\\\hline
\id   &\id      &\ensuremath{{\cal R}_{3}}  &\ensuremath{{\cal R}_{1}}&\ensuremath{{\cal R}_{2}}\\
\ensuremath{{\cal R}_{3}}&.&\id&\ensuremath{{\cal R}_{2}}  &\ensuremath{{\cal R}_{1}}\\
\ensuremath{{\cal R}_{1}}&.&.&2\id+\ensuremath{{\cal R}_{2}} &2\ensuremath{{\cal R}_{3}}+\ensuremath{{\cal R}_{1}}\\
\ensuremath{{\cal R}_{2}}&.&.&.&2\id+\ensuremath{{\cal R}_{2}}\\\hline
\end{tabular}
\end{center}
  \label{tab:D6multTabperm}
\end{table}
%%%%%%%%%%%%%%%%%%%%%%%%%%%%%%%%%%%%%%%%%%%%%%%%%%%%%%%%%%%%%%%%%%%%%%
