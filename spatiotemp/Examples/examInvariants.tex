% examInvariants.tex called by {invariants}{25may2014}{Cycle stability}
% $Author: predrag $ $Date: 2019-03-16 17:14:50 -0400 (Sat, 16 Mar 2019) $

% Predrag                            5nov2017
% Predrag                           25may2014

% called by
%           siminos/spatiotemp/chapter/spatiotemp.tex

\section{Examples}
\label{exam:invariants}

%%%%%%%%%%%%%%%%%%%%%%%%%%%%%%%%%%%%%%%%%%%%%%%%%%%%%%%%%%%%%%
\example{Equivariance under infinitesimal transformations.}
        { \label{exam:EquiInfntmsl}
% Predrag                           2018-02-02
% from \Chapter{continuous}{31jan2015}{Relativity for cyclists}
A flow $\dot{\ssp}= \vel(\ssp)$ is $\Group$-equivariant,
%\refeq{GvCommut},
if symmetry transformations commute with time evolution
\beq
\vel(\ssp)=\LieEl^{-1} \, \vel(\LieEl \, \ssp)
\,,\qquad \mbox{for all } \LieEl \in {\Group}
\,.
\ee{eq:FiniteRot}
For an infinitesimal transformation % \refeq{intsmLieTransf}
the $\Group$-equivariance condition % \refeq{eq:FiniteRot}
becomes
\[
\vel(\ssp) =(1-\gSpace \cdot \Lg) \, \vel(\ssp+\gSpace \cdot \Lg \ssp) + \cdots
       =\vel(\ssp)- \gSpace \cdot \Lg \vel(\ssp)
             + \frac{d\vel}{d\ssp} \,\gSpace \cdot \Lg \ssp + \cdots
\,.
\]
The $\vel(x)$ cancel, and $\gSpace_a$ are arbitrary. Denote
the \emph{group flow tangent field} at \ssp\ by
$\groupTan_a(\ssp)_{i}= (\Lg_a){}_{ij} \ssp_j$. Thus the
infinitesimal, Lie algebra $\Group$-equivariance condition is
\index{group!tangent field}\index{tangent!field, group}
\beq
%  \left(
%    \Lg_a  - \groupTan_a(\ssp) \cdot \frac{\partial}{\partial \ssp}
%  \right) \vel(\ssp) =
  \groupTan_a(\vel)  - \Mvar(\ssp) \, \groupTan_a(\ssp) =0
  \,,
\ee{inftmInv}
where $\Mvar = {\pde \vel}/{\pde \ssp}$ is the \stabmat. %\refeq{DerMatrix}.
A learned remark: The directional derivative along
direction $\xi$ is
\(
    %Df(x)\xi
    \lim_{t \to 0} ({f(x+t\xi)-f(x))}/{t}
    \,.
\)
The left-hand side of \refeq{inftmInv} is
the {\em Lie derivative} of the dynamical flow
field $\vel$ along the direction of the infinitesimal
group-rotation induced flow $\groupTan_a(\ssp)= \Lg_a \ssp$,
\beq
{\cal L}_{\groupTan_a} \vel =
\left.\left(
  \Lg_a - \frac{\partial}{\partial y}(\Lg_a \ssp)
 \right) \vel(y)\right|_{y=\ssp}
\,.
\ee{LieDeriv}
The equivariance condition \refeq{inftmInv} states that the two
flows, one induced by the dynamical vector field $\vel$, and
the other by the group tangent field $\groupTan$, commute if
their Lie derivatives (or the `Lie brackets ' or `Poisson
brackets') vanish.
    \index{Lie!derivative} \index{Lie!bracket}
    \index{Poisson!bracket}
                \jumpBack{exam:EquiInfntmsl}
    } %end \example{Equivariance under inf. transf.}{exam:EquiInfntmsl
%%%%%%%%%%%%%%%%%%%%%%%%%%%%%%%%%%%%%%%%%%%%%%%%%%%%%%%%%%%%%%

%%%%%%%%%%%%%%%%%%%%%%%%%%%%%%%%%%%%%%%%%%%%%%%%%%%%%%%%%%%%%%
\example{Stability of spatiotemporal patterns.}{ \label{exam:spatioTempStab}
% Predrag                           2018-02-02
%   from \Chapter{stability}{25may2014}{Local stability}
%   section Stability of flows
% Predrag                           25may2014
How does one determine the eigen\-values of the finite time local
deformation $\jMps^\zeit$ for a general nonlinear smooth flow? The
\jacobianM\ is computed by integrating the equations of variations
% \refeq{die}
\beq
\ssp(\zeit)=\flow{\zeit}{\xInit} \,, \quad
\deltaX (\xInit,\zeit) \,= \,\jMps^\zeit(\xInit) \, \deltaX (\xInit,0)
\,.
\ee{xit_1}
% with the initial tangent space vector ${\bf \eta}$ transported by
% the \jacobianM\ $\jMps^\zeit(\xInit) $.
The equations are linear, so we should be able to integrate them--but
in order to make sense of the answer, we derive this integral step by
step.

Consider the case of a general, non-stationary trajectory
$\ssp(\zeit)$. The exponential of a constant matrix can be defined
either by its Taylor series expansion or in terms of the Euler limit:
% \refeq{EqDyn141}:
\index{Euler!limit}
\beq
e^{\zeit{\Mvar}} \,=\, \sum_{k=0}^\infty \frac{\zeit^k}{k !} {\Mvar}^k
    \,=\, \lim_{m \to \infty}\left(\matId + \frac{\zeit}{m} {\Mvar} \right)^{m}
\,.
\ee{exp_prod}
Taylor expanding is fine if ${\Mvar}$ is a constant matrix. However,
only the second, tax-accountant's discrete step definition of an
exponential is appropriate for the task at hand.  For dynamical
systems, the local rate of neighborhood distortion ${\Mvar}(\ssp)$
depends on where we are along the trajectory. The linearized
neighborhood is deformed along the flow, and the $m$
discrete time-step approximation to $\jMps^\zeit$ is therefore given
by a generalization of the Euler product \refeq{exp_prod}:
\index{Euler!product}
\bea
\jMps^\zeit(\xInit) &=&
\lim_{m \to \infty}\prod_{n=m}^1
\left(\matId + \delta t {\Mvar}(\ssp_n) \right)
          =
\lim_{m \to \infty}\prod_{n=m}^1 e^{\delta t\, {\Mvar}(\ssp_n)}
                            \label{Jprod} \\
         &= &
\lim_{m \to \infty}
e^{\delta t\, {\Mvar}(\ssp_m)}
e^{\delta t\, {\Mvar}(\ssp_{m-1})}
\cdots
e^{\delta t\, {\Mvar}(\ssp_2)}
e^{\delta t\, {\Mvar}(\ssp_1)}
\,,
\nnu
\eea
where $\delta t = {(t-t_0)/ m}$, and $\ssp_n=\ssp(t_0+n\delta t)$.
Indexing of the product indicates that the successive infinitesimal
deformation are applied by multiplying from the left. The
$m\to\infty$ limit of this procedure is the formal integral
\index{time!ordered integration}
\beq
\jMps^\zeit_{ij}(\xInit)
= \left[ {\bf T} e^{ \int_0^\zeit d\tau {\Mvar} (\ssp(\tau)) } \right]_{ij}
\,,
\label{hodes}
\eeq
where ${\bf T}$ stands for time-ordered integration, {\em defined} as
the continuum limit of successive multiplications
\refeq{Jprod}.
This integral formula for $\jMps^\zeit$ is the main conceptual result of
the present chapter. This formula is the finite time companion of the differential
definition \refeq{Bew_Miaw}.
The definition makes evident important properties of \jacobianMs,
such as their being multiplicative along the flow,
\beq
\jMps^{\zeit+\zeit'}\!(\ssp)
    = \jMps^{\zeit'}\!(\ssp')\, \jMps^\zeit(\ssp), \qquad \mbox{where} \,\;
\ssp'=\flow{\zeit}{\xInit}
\,,
\ee{Jmultiplic}
which is an immediate consequence of the time-ordered product structure of
\refeq{Jprod}.
\index{semigroup!dynamical}
                \jumpBack{exam:spatioTempStab}
    } %end \example{Stability of spatiotemporal patterns.}{exam:spatioTempStab
%%%%%%%%%%%%%%%%%%%%%%%%%%%%%%%%%%%%%%%%%%%%%%%%%%%%%%%%%%%%%%

%%%%%%%%%%%%%%%%%%%%%%%%%%%%%%%%%%%%%%%%%%%%%%%%%%%%%%%%%%%%%%
\example{Floquet multipliers of a spatiotemporal torus are invariant.}
        { \label{exam:FloqTorus}
% Predrag                           2018-02-02
%   from \Chapter{invariants}{}{Cycle stability}
%   \example{Stability of cycles for maps.}{ \label{exam:cycStabMap}
% Predrag                           25may2014
The 1\dmn\ map Floquet multiplier %\refeq{Lambda_1p}
is a product of derivatives over all points around the cycle, and is
therefore independent of which periodic point is chosen as the
initial one.
In higher dimensions the form of the \FloquetM\
$\jMps_p(\xInit)$ % in \refeq{Jrepeat}
does depend on the choice
of coordinates and the initial point $\xInit \in \pS_p$.
Nevertheless, as we shall now show, the cycle {\em Floquet
multipliers} are intrinsic property of a cycle in any
dimension. Consider the $i$th eigenvalue, eigen\-vector pair
$(\ExpaEig_{j},\, \jEigvec[j])$ computed from $\jMps_p$
evaluated at a periodic point $\ssp$,
    \PC{}{fix scale in \reffig{f:covariantPO}, refer to it}
\beq
\jMps_p (\ssp) \,\jEigvec[j](\ssp) =
    \ExpaEig_{j} \,\jEigvec[j](\ssp)\,,  \quad
\ssp \in \pS_p \,.
\ee{e-transp}
%and at
Consider another point on the cycle at time $t$ later,
$\ssp'=f^t(\ssp)$ whose \FloquetM\ is $\jMps_{p} (\ssp')$.  By the
semigroup property \refeq{Jmultiplic},
$\jMps^{\period{}+t} = \jMps^{t+\period{}}$,
and the \jacobianM\ at $\ssp'$ can be written either as
\index{semigroup!dynamical}
\[
\jMps^{\period{}+t} (\ssp) = \jMps^{\period{}} (\ssp') \, \jMps^{t} (\ssp)
               = \jMps_p (\ssp') \, \jMps^t (\ssp)
\,,
\]
%\PC{}{Benny + Lan want me to be more explicit}
or
$ % \jMps_p (\ssp') \, \jMps^t (\ssp) =
  \jMps^t (\ssp) \, \jMps_p (\ssp)$.
Multiplying \refeq{e-transp} by $\jMps^t (\ssp)$, we find that the
\FloquetM\ evaluated at $\ssp'$ has the same Floquet multiplier,
\beq
\jMps_p (\ssp') \,\jEigvec[j](\ssp') =
    \ExpaEig_{j} \,\jEigvec[j](\ssp')\,,  \quad
     \jEigvec[j] (\ssp') = \jMps^t (\ssp) \,\jEigvec[j](\ssp)
\,,
\ee{EigsInvar}
but with the eigen\-vector $\jEigvec[j]$ transported along the flow
$\ssp \to \ssp'$ to $\jEigvec[j](\ssp')=\jMps^t(\ssp)
\,\jEigvec[j](\ssp)$. Hence, in the spirit of the Floquet theory
% (\refappe{s-FloquetTheory})
one can define time-periodic eigen\-vectors (in a
co-moving `Lagrangian frame')
\index{co-moving frame}\index{Lagrangian!frame}
\beq
\jEigvec[j](t) =
 e^{- \eigExp[j] t } \,\jMps^t(\ssp)\,\jEigvec[j](0)
    \,,\qquad
\jEigvec[j](t) = \jEigvec[j](\ssp(t))
    \,,\;\;
\ssp(t) \in \pS_p
\,.
\ee{coMovingEig}
$\jMps_p$ evaluated anywhere
along the cycle has the same set of Floquet multipliers
$\{\ExpaEig_{1}, \ExpaEig_{2},$
$\cdots, 1,\cdots$ $,\ExpaEig_{d-1}\}$.
% \PC{}{YL: elaborate!}
%\beq
%\left( \jMps_p (\ssp') - \E\ssppaEig_{p,j}\matId \right) \jEigvec[j] (\ssp') =
%0\,, \quad  \ssp' \in \pS_p \,.
%\ee{EigsInvar}
 As quantities such as $\tr \jMps_{p}(\ssp)$, $\det \jMps_{p}(\ssp)$ depend
only on the eigenvalues of $\jMps_{p}(\ssp)$ and not on
the starting point $\ssp$, in expressions such as
$ \det \! \left( \matId - \jMps^{r}_{p}(\ssp) \right)$
we may omit reference to $\ssp$,
\beq
\det \! \left( \matId - \jMps^{r}_{p} \right)
            =
\det \! \left( \matId - \jMps^{r}_{p}(\ssp) \right)
\quad \mbox{for any $\ssp \in \pS_p $}
\,.
\ee{NoXdepJp}
We postpone the proof that the cycle Floquet multipliers are smooth
conjugacy invariants of the flow; % to  \refsect{s-SmoothConj};
time-forward map \refeq{EigsInvar} is the special case of this
general property of smooth manifolds and their tangent spaces.
                \jumpBack{exam:FloqTorus}
    } %end \example{Floquets of a spatiotemporal torus}{exam:FloqTorus
%%%%%%%%%%%%%%%%%%%%%%%%%%%%%%%%%%%%%%%%%%%%%%%%%%%%%%%%%%%%%%
