% examHamHenonMap.tex called by examNewton.tex
% $Author: predrag $ $Date: 2021-05-04 15:35:29 -0400 (Tue, 04 May 2021) $

% Predrag                              2021-04-03
% called by \Chapter{newton}{17dec2020}{Hamiltonian dynamics}

%%%%%%%%%%%%%%%%%%%%%%%%%%%%%%%%%%%%%%%%%%%%%%%%%%%%%%%%%%%%%%%%%%
\example{{\Henlatt}.}{ \label{exam:HamHenonMap}
    \index{Henon@H\'enon map!Hamiltonian}                   \toCB
    \index{map!H\'enon!Hamiltonian}
    \index{map!Hamiltonian!H\'enon}
    \index{Hamiltonian!H\'enon map}
    \index{symplectic!H\'enon map}
    \index{area preserving!H\'enon map}
For $b=-1$ parameter value the {\HenonMap} \refeq{eq2.1a}  is the
simplest example of a nonlinear Hamiltonian map, a $2$\dmn\ {\opres},
area preserving map, often studied to better understand topology and
symmetries of \PoincSec s of 2-{\dofs} Hamiltonian flows.

We find it convenient\rf{EG05} to multiply \refeq{2-step} by
$a$ and absorb the $a$ factor into the definition the lattice field
$\field=a\,\ssp$.
This brings the Hamiltonian {\HenonMap} to the form
% analogous to the Fatou\rf{fatou} parabola parametrization:
\index{Henon@H\'enon map}
\bea
    \field_{n+1}&=&a-\field^2_n - p_n
                \continue
    p_{n+1}&=& \field_n
\,,
\label{Henlatt-eq2.1.0}
\eea
or, equivalently, the {\henlatt}
\refeq{EG05:ar_pres} 3-term recurrence relation of form
\beq
\field_{i+1} +\field_{i}^2 +\field_{i-1}=a  \, ,
\quad i=1,...,n_p
\,.
\label{Henlatt-2-step}
\eeq
We can write this as a
lattice field equation with lattice Laplacian \refeq{Box(2.1.1)}
\beq
  % \field_{n-1}-2\,\field_{n}+\field_{n+1}
\Box\,\field_{n}+(\field_{n}+2)\,\field_{n}\,=\, a
\,.
\ee{Henlatt-Lapl}
The field equation is nonlinear, with the
cubic potential \refeq{BWcubic}.

For definitiveness, in numerical calculations in examples
to follow we shall fix (arbitrarily)
the stretching parameter value to $a=6$, a value large enough to guarantee
that all roots of  $0=f^n(\ssp)-\ssp$ (periodic points) are real.
\exerbox{ex_birkhoff}
%\PC{link to next exer}
                                        \jumpBack{exam:HamHenonMap}
    } %end \example{HamHenonmap}
%%%%%%%%%%%%%%%%%%%%%%%%%%%%%%%%%%%%%%%%%%%%%%%%%%%%%%%%%%%%%%%%%%

%%%%%%%%%%%%%%%%%%%%%%%%%%%%%%%%%%%%%%%%%%%%%%%%%%%%%%%%%%%%%%%%%%
\example{{\Henlatt} fixed points.}{\label{exam:Henlatt-fixed}
% derived from ChaosBook {e-Henon-fixed}     Sept 6 2001
\index{Henon@H\'enon map!fixed points}
Since we are looking for fixed points $p_\stagn$ of
\refeq{Henlatt-eq2.1.0}, each successive step is the same as the
previous,
\[  \left( \begin{array}{c}
        \field_\stagn \\
        p_\stagn \\
        \end{array}\right)
         = \left(\begin{array}{c}
            a - \field_\stagn^2 -p_\stagn   \\
            \field_\stagn                 \\
            \end{array} \right)
\,.
\]
Thus there two fixed points, given by the roots of
the quadratic equation
$
\field^2 - 2\,\field - a = 0
\,,
$
\PC{2021-04-28}{agrees with Endler and Gallas\rf{EG05} $a=6$ values}
\bea
      \field_0 &=& -1 -\sqrt{1+a}
                        \continue
      \field_1 &=& -1 +\sqrt{1+a}
\,.
\label{Henlatt-2-cycle}
\eea

\PC{2021-04-28}{make into na exercise: Show that \refeq{Henlatt-2-cycle}...}
\cycle{01} periodic points are \refeq{COM:2-cycle}
\beq
      \field_{10}  =  { 1+ \sqrt{a-3} }
                        \,,\qquad
      \field_{01}  =  { 1- \sqrt{a-3} }
\,.
\label{Henlatt-exer:eq2}
\eeq


%\PC{link to next exam}
                                        \jumpBack{exam:Henlatt-fixed}
} %end \example{exam:Henlatt-fixed}
%%%%%%%%%%%%%%%%%%%%%%%%%%%%%%%%%%%%%%%%%%%%%%%%%%%%%%%%%%%%%%%
