% examHenonMap.tex
% $Author: predrag $ $Date: 2021-12-24 01:25:20 -0500 (Fri, 24 Dec 2021) $

% Predrag extracted from                   2021-02-15
%                   ChaosBook.org
%%%%%%%%%%%%%%%%%%%%%%%%%%%%%%%%%%%%%%%%%%%%%%%%%%%%%%%
\example{{\HenonMap}.}{ \label{exam:HenonMap}
The map
\index{Henon@H\'enon map}\index{map!H\'enon}
\index{stretch \& fold}                                     \toCB
\bea
    x_{n+1}&=&1-ax^2_n+b y_n
        \continue
    y_{n+1}&=& x_n
\label{eq2.1b}
\eea
is a nonlinear 2\dmn\ map frequently employed in testing various hunches
about chaotic dynamics. Written as a  2nd-order inhomogeneous difference equation
(3-term recurrence relation), the {\henlatt} is
\beq
    x_{n+1}=1-ax^2_n+bx_{n-1} \, .
\ee{2-step}
An $(n+1)$-term recurrence relation is the discrete-time analogue
of an $n$th order differential equation, and it can always
be replaced by a set of $n$ 1-step relations.

% restore     The {\HenonMap} is the simplest map that captures the \stretchf\ dynamics
% restore     of return maps such as R\"ossler's, \reffig{f:RosslSect}.
% restore     It can be obtained by a truncation of a polynomial approximation
% restore     \refeq{approxPoinc} to a \PoincMap\ \refeq{approxPoinc} to second order.

%\index{map!once-folding}
%\index{once-folding map}
% restore     A quick sketch of the long-time dynamics of such a mapping
% restore     (an example is depicted in \reffig{FigHenonPer7}), is obtained
% restore     by picking an arbitrary starting point and
% restore     iterating \refeq{eq2.1a} on a computer.

Always plot the dynamics of such maps in the $(x_n,x_{n+1})$ plane,
rather than in the $(x_n,y_n)$ plane, and make sure that the ordinate and
abscissa scales are the same, so $x_n = x_{n+1}$ is the $45^o$ diagonal.
There are several reasons why one should plot this way: (a) we think of
the {\HenonMap} as a model return map $x_n \to x_{n+1}$, and (b) as
parameter $b$ varies, the attractor will change its $y$-axis scale, while
in the  $(x_n,x_{n+1})$ plane it goes to a parabola as $b\to0$, as it
should.
% restore     \exerbox{e-Henon-fixed}

% restore     As we shall soon see, {\po}s will be key to
% restore     understanding the long-time dynamics, so we also plot a typical
% restore     {\po} of such a system, in this case an unstable
% restore     period 7 cycle.
% restore     Numerical determination of such cycles will be
% restore     explained in \refsect{s-biham},
% restore     \PC{recheck this refsect}
% restore     and the periodic point labels
% restore     $0111010$, $1110100$, $\cdots$ in \refsect{s-smale}.
% restore            \jumpBack{exam:HenonMap}
    } %end \example{exam:HenonMap}
%%%%%%%%%%%%%%%%%%%%%%%%%%%%%%%%%%%%%%%%%%%%%%%%%%%%%%%
