% siminos/spatiotemp/chapter/examTentLCod.tex
% $Author: predrag $ $Date: 2018-05-04 12:19:38 -0400 (Fri, 04 May 2018) $
% called by siminos/spatiotemp/chapter/tentMapCode.tex
%\section{Any piecewise linear map has ``linear code''}
%\label{exam:tentMapSymbDyn}

%%%%%%%%%%%%%%%%%%%%%%%%%%%%%%%%%%%%%%%%%%%%%%%%%%%%%%%
\example{Tent map linear code.}{\label{exam:TentLCod}
The simplest example of a piece-wise linear unimodal map with a binary (in
general, pruned) symbolic dynamics is the {\em tent map,}
%\reffig{f-log-repeller}\,(a),
\beq
\flow{}{x} =  \left\{ \begin{array}{ll}
f_0(x) = \ExpaEig x       & \mbox{ if } x  < 1/2 \\
f_1(x) =  \ExpaEig (1-x)  & \mbox{ if } x  > 1/2
         \end{array}\right.
\,,
\ee{anyTentSplit}
with $1<\ExpaEig<\infty$ and $x\in\pS=[0,1]$. (Everything would go through for
a skew tent map with $\ExpaEig_0\neq-\ExpaEig_1$, but there is no need here
for that complication.)
For this family of unimodal maps the coarse (covering) partition of the unit
interval $\pS=\pS_0\cap{C}\cap\pS_1$ is given by intervals $\pS_0=[0,1/2)$,
 $\pS_1=(1/2,1]$, and the critical point $C=1/2$.
Let's rewrite this as a linear first-order difference equation, in the manner
of cat lovers enamoured of matters feline:
\beq
\frac{1}{\ExpaEig}  \, x_{t+1}  +(2\Ssym{t}-1)x_{t} = \Ssym{t}
\,,\qquad
\left\{ \begin{array}{ll}
\Ssym{t}=0 & \mbox{ if }  x_{t}  < 1/2 \\
\Ssym{t}=1 & \mbox{ if }  x_{t}  > 1/2
         \end{array}\right.
\,.
\ee{TentLinCode}
That every such code is a `linear code' is best understood by
computing a periodic orbit for a specified itinerary.

The fixed point condition $\flow{n}{x}=x$ for $n$-cycle
\cycle{\Ssym{1}\Ssym{2}\Ssym{3}\cdots \Ssym{n-1}\Ssym{n}}
is a linear relation between the finite alphabet $\Ssym{t}\in\{0,1\}$ code, and
the $x_{t}\in\reals$ orbit
\beq
  \Delta(\Ssym{})q(\Ssym{}) = m(\Ssym{})
\ee{tentMapDiffEq}
with orbit-dependent inverse propagator $\Delta(\Ssym{})= $
\[
{\small
\left(\begin{array}{cccccc}
      {2\Ssym{n}-1}     & 0     &     0  & \dots  & 0 & \ExpaEig^{-1}\\
      \ExpaEig^{-1}     & {2\Ssym{n-1}-1} &     0     &\dots  &  0 & 0\\
       0 & \ExpaEig^{-1} &    {2\Ssym{n-2}-1} &\dots  &  0 & 0\\
      \vdots&\vdots &   \vdots & \ddots &\vdots  &\vdots \\
       0 &  0 &    0  & \dots  & {2\Ssym{2}-1}& 0\\
      0     &  0 &    0  & \dots  &\ExpaEig^{-1}  & {2\Ssym{1}-1}\\
     \end{array} \right)
\,,
} % end \small
\]
\[
q(\Ssym{})= \left(\begin{array}{c}
      x_{{n}}\\
      x_{{n-1}}\\
      x_{{n-2}}\\
      \vdots\\
      x_{{2}}\\
      x_{{1}}\\
     \end{array} \right)
\,,\qquad
m(\Ssym{})= \left(\begin{array}{c}
      {\Ssym{n}}\\
      {\Ssym{n-1}}\\
      {\Ssym{n-2}}\\
      \vdots\\
      {\Ssym{2}}\\
      {\Ssym{1}}\\
     \end{array} \right)
\,,
\]
and $m(\Ssym{})$ is needed to fold the stretched orbit back into the unit
interval.
While the off-diagonal ``1''s do generate cyclic shifts, the diagonal $\pm
\ExpaEig$ terms are not shift invariant, so I do not believe this can be
diagonalized by a discrete Fourier transform. I had worked it out for
$\ExpaEig=2$ in Chaos\-Book, but not sure if there are elegant tricks for
arbitrary $\ExpaEig\neq2$. For an orbit
\beq
  q(\Ssym{}) = \Delta(\Ssym{})^{-1}m(\Ssym{})
\ee{tentMapOrbit}
to be admissible, no point should be to the right of
the kneading value $x_\kappa = f(C)$. It follows from
the kneading theory for unimodal maps (dike map with slope
$\ExpaEig=2$ being the canonical example) that if a \po \ exists for
a given $\ExpaEig$, it exists for all larger $\ExpaEig$,
and that all orbits exist for $\ExpaEig \geq 2$.

In other words, $\ExpaEig$ is the ``stretching parameter'' for this
problem, and the rational polynomial expressions in $\ExpaEig$
for $x_{t}$ correspond to Li Han's polynomials for cat maps.
        \jumpBack{exam:TentLCod}
    } %end \example{Tent map linear code.}{exam:TentLCod}
%%%%%%%%%%%%%%%%%%%%%%%%%%%%%%%%%%%%%%%%%%%%%%%%%%%%%%%%%%%%%%

%%%%%%%%%%%%%%%%%%%%%%%%%%%%%%%%%%%%%%%%%%%%%%%%%%%%%%%%%%%%%%
\example{Periodic points of a tent map.}{\label{exam:TentCycl}

\paragraph{Exercise}
Check \refeq{tentMapOrbit} for fixed point(s).


\paragraph{Exercise}
Check \refeq{tentMapOrbit} for the 2-cycle \cycle{01}.
\[
\Delta(\Ssym{})=
\left(\begin{array}{cc}
      -\ExpaEig& 1\\
      1     & \ExpaEig
     \end{array} \right)
\,,\qquad
m(\Ssym{})= \ExpaEig \left(\begin{array}{c}
      0\\
      1
     \end{array} \right)
\,.
\]
\[
\Delta^{-1} =
\frac{1}{\ExpaEig^2+1}
\left(\begin{array}{cc}
      -\ExpaEig& 1\\
      1     & \ExpaEig
     \end{array} \right)
\,,\qquad
  \det\Delta(\Ssym{})= -(\ExpaEig^2+1)
\]
\[
\left(\begin{array}{c}
      x_{01}\\
      x_{10}
     \end{array} \right)
=
\frac{\ExpaEig}{\ExpaEig^2+1}
\left(\begin{array}{cc}
      -\ExpaEig& 1\\
      1     & \ExpaEig
     \end{array} \right)
\,\left(\begin{array}{c}
      0\\
      1
     \end{array} \right)
=
\frac{\ExpaEig}{\ExpaEig^2+1}
\,\left(\begin{array}{c}
      1\\
      \ExpaEig
     \end{array} \right)
\,.
\]
For the Ulam tent map this
yields the correct periodic points
\(
\{x_{01},x_{10}\}
=\{2/5,4/5\}
\,.
\)
In the $\ExpaEig\to1$ limit, this 2-cycle collapses into the
critical point $C=1/2$.


\paragraph{Exercise}
Check \refeq{tentMapOrbit} for the two 3-cycles.
For the Ulam tent map case, the periodic points are
\bea
\{\gamma_{001},\gamma_{010},\gamma_{100}\}
&=& \{2/9,4/9,8/9\}
\continue
\{\gamma_{011},\gamma_{110},\gamma_{101}\}
&=& \{2/7,4/7,6/7\}
\,.
\nnu
\eea


\paragraph{Exercise}
Check \refeq{tentMapOrbit} for $\ExpaEig =$ golden mean. The $\cycle{001} \to
\cycle{0C1}$ as $\ExpaEig \to $ golden mean from above. Do you get all
admissible cycles? That is worked out in Chaos\-Book, but not in this
formulation.

\paragraph{Exercise}
Is there a systematic solution to \refeq{tentMapOrbit} for arbitrary
$n$-cycle? The $\ExpaEig=2$ case has the elegant solution described in
Chaos\-Book; whatever polynomials you find, they should agree with that
particular factorization. In other words, think of the sums
\refeq{tentEvenCyclePts} and \refeq{tentOddCyclePts} as the expansion of a
real number in terms of the digits $w_t$ in the nonintegral base $\ExpaEig$.
As the symbolic dynamics of a cycle is independent of  $\ExpaEig$, the Ulam
tent map calculation, in the familiar base 2 clinches the arbitrary tent map
case.

\bigskip

    {\color{red}
    The rest of the section might even be right - has to factorize in
    agreement with my Ulam tent map computations. Please fix at your leisure,
    if I am wrong.
    }

\bigskip

If the repeating string                                         \toCB
$\Ssym{1}\Ssym{2}\ldots \Ssym{n}$ contains an even number of
`1's, the repating string of well ordered symbols $w_1w_2\ldots
w_{n}$ is of the same length. The cycle-point $x$ is a geometrical sum which we
can rewrite as the odd-denominator fraction
\bea
   x(\cycle{\Ssym{1}\Ssym{2}\ldots \Ssym{n}})
   &=& \sum_{t=1}^{n} \frac{w_t}{\ExpaEig^{t}}
      +  \frac{1}{\ExpaEig^{-n}} \sum_{t=1}^{n} \frac{w_t}{\ExpaEig^{t}}
      + \cdots
   \continue
   &=& \frac{1}{\ExpaEig^{n}-1}
        \sum_{t=1}^{n} {w_t}{\ExpaEig^{n-t}}
\label{tentEvenCyclePts}
\eea
If the repeating string
$\Ssym{1}\Ssym{2}\ldots \Ssym{n}$ contains an odd number of
`1's, the string of well ordered symbols $w_1w_2\ldots
w_{2n}$ has to be of the double length before it repeats
itself. The cycle-point $x$ is a geometrical sum which we
can rewrite as the odd-denominator fraction
\bea
   x(\cycle{\Ssym{1}\Ssym{2}\ldots \Ssym{n}})
   &=& \sum_{t=1}^{2n} \frac{w_t}{\ExpaEig^{t}}
      +  \frac{1}{\ExpaEig^{-2n}} \sum_{t=1}^{2n} \frac{w_t}{\ExpaEig^{t}}
      + \cdots
   \continue
   &=& \frac{1}{(\ExpaEig^{n}-1)(\ExpaEig^{n}+1)}
        \sum_{t=1}^{2n}{w_t}{\ExpaEig^{2n-t}}
\label{tentOddCyclePts}
\eea
        \jumpBack{exam:TentCycl}
    } %end \example{Tent map linear code.}{exam:TentCycl}
%%%%%%%%%%%%%%%%%%%%%%%%%%%%%%%%%%%%%%%%%%%%%%%%%%%%%%%%%%%%%%
