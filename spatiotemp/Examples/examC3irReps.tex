% siminos/spatiotemp/Examples/examC3irReps.tex
% $Author: predrag $ $Date: 2021-07-28 17:52:52 -0400 (Wed, 28 Jul 2021) $

%%%%%%%%%%%%%%%%%%%%%%%%%%%%%%%%%%%%%%%%%%%%%%%%%%%%%%%%%%%%%%%%%%%%%%
\example{Irreps of cyclic group $\Cn{3}$.}{
  \label{exam:C3irReps}                                     \toCB
%% 2021-07-23 from Xiong's siminos/xiong/thesis/chapters/symGroup.tex
  (continued from \refexam{exam:C3regularRep})~~
  For $\Dn{1}$, whose multiplication table is in \reftab{tab:C3MultTab}, we can form
  the symmetric base $\rho(\sspRed) + \rho(\Refl\sspRed)$
  and the antisymmetric base $\rho(\sspRed) - \rho(\Refl\sspRed)$. You can verify
  that in this new basis, $\Dn{1}$ is block-diagonalized.
  We would like to generalize this symmetric-antisymmetric
  decomposition to the order 3 group $\Cn{3}$. Symmetrization
  can be carried out on any number of functions, but there is no obvious
  anti-symmetrization. We draw instead inspiration from the Fourier
  transformation for a finite periodic lattice, and construct from the
  regular basis \refeq{eq:C3RegRep} a new set of bases
  \bea
  \rho^{irr}_0(\sspRed) &=& \frac{1}{3}\left[
    \rho(\sspRed) ~+~ \rho(\shift \sspRed) ~+~ \rho(\shift^2 \sspRed)
  \right]
  \label{eq:c3f1}\\
  \rho_1^{irr}(\sspRed) &=& \frac{1}{3}\left[
    \rho(\sspRed) + \omega\,\rho(\shift \sspRed) + \omega^2 \rho(\shift^2 \sspRed)
  \right]
  \label{eq:c3f2}\\
  \rho_2^{irr}(\sspRed) &=& \frac{1}{3}\left[
    \rho(\sspRed) + \omega^2 \rho(\shift \sspRed) + \omega\,\rho(\shift^2 \sspRed)
  \right]
  \label{eq:c3f2}
  \,.
  \eea
  Here $\omega = e^{2i\pi/3}$.
  The representation of group $\Cn{3}$ in this new basis is block-diagonal
  by inspection:
  \begin{equation}
    D^{irr}(e) =
    \begin{bmatrix}
      1 & & \\
      & 1 & \\
      & & 1 \\
    \end{bmatrix} \,,\quad
    D^{irr}(\shift) =
    \begin{bmatrix}
      1 & 0 & 0\\
      0 & \omega & 0\\
      0 & 0 & \omega^2 \\
    \end{bmatrix}  \,,\quad
    D^{irr}(\shift^2) =
    \begin{bmatrix}
      1 & 0 & 0\\
      0 & \omega^2 & 0\\
      0 & 0 & \omega \\
    \end{bmatrix}
    \,.
    \label{eq:c3irr}
  \end{equation}
  So $\Cn{3}$ has three 1\dmn\ irreps. Generalization to any
  $\Cn{n}$ is immediate: this is just a finite lattice, discrete Fourier transform.
                                        \jumpBack{exam:C3irReps}
\authorXD{}
}% end \example{examC3irReps}
%%%%%%%%%%%%%%%%%%%%%%%%%%%%%%%%%%%%%%%%%%%%%%%%%%%%%%%%%%%%%%%%%%%%%%
