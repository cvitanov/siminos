% siminos/spatiotemp/Examples/examDinftyCosets.tex
% $Author: predrag $ $Date: 2021-08-10 11:56:19 -0400 (Tue, 10 Aug 2021) $

%%%%%%%%%%%%%%%%%%%%%%%%%%%%%%%%%%%%%%%%%%%%%%%%%%%%%%%%%%%%%
\example{\Dn{\infty} subgroups and cosets.}{ \label{exam:DinftyCosets}
                                       \toCB
%\item[2021-07-23 Han]
%The structure of the quotient groups $\Dn{\infty}/H(\cl{})$ and
%$\Dn{\infty}/H(\cl{},k)$:
%
$H(\cl{})$, any $\cl{}$, is a translation subgroup of
$\Dn{\infty}$ (a 1\dmn\ {Bravais sublattice} \lattice\
\refeq{1DBravaisLattice},
with a {basis} vector $\mathbf{a}$ that defines
the {Bravais cell} of length \cl{}) with group elements
$\langle\shift^n\rangle>$, or, more explicitely:
\beq
H(\cl{}) = \{ \cdots, \shift_{-2 \cl{}}, \shift_{-\cl{}},
1, \shift_{\cl{}}, \shift_{2 \cl{}}, \cdots\} \, .
\ee{H(n)subgroup}
There are a $2\cl{}$ left cosets of subgroup $H(\cl{})$ in $\Dn{\infty}$:
\bea
H(\cl{}) &=& \{ \cdots, \shift_{-2 \cl{}}, \shift_{-\cl{}},
1, \shift_{\cl{}}, \shift_{2 \cl{}}, \cdots\}
    \label{H(n)cosets}\\
\Refl H(\cl{}) &=& \{ \cdots, \Refl_{-2 \cl{}}, \Refl_{-\cl{}},
\Refl, \Refl_{\cl{}}, \Refl_{2 \cl{}}, \cdots\} \continue
\shift H(\cl{}) &=& \{ \cdots, \shift_{-2 \cl{} + 1}, \shift_{-\cl{} + 1},
\shift, \shift_{\cl{}+1}, \shift_{2 \cl{}+1}, \cdots\} \continue
\Refl_{1} H(\cl{}) &=& \{ \cdots, \Refl_{-2 \cl{} + 1}, \Refl_{-\cl{} + 1},
\Refl_{1}, \Refl_{\cl{}+1}, \Refl_{2 \cl{}+1}, \cdots\} \continue
&\vdots& \continue
\shift_{\cl{}-1} H(\cl{}) &=& \{ \cdots, \shift_{-\cl{}-1}, \shift_{-1},
\shift_{\cl{}-1}, \shift_{2\cl{}-1}, \shift_{3\cl{}-1}, \cdots\} \continue
\Refl_{\cl{}-1} H(\cl{}) &=& \{ \cdots, \Refl_{-\cl{}-1}, \Refl_{-1},
\Refl_{\cl{}-1}, \Refl_{2\cl{}-1}, \Refl_{3\cl{}-1}, \cdots\}
\,. \nnu
\eea
Using elements
\(
\{1,\Refl, \shift, \Refl_{1}, \cdots, \shift_{\cl{}-1}, \Refl_{\cl{}-1}\}
\)
as representatives of these cosets we
see that the quotient group $\Dn{\infty}/H(\cl{})$ is isomorphic to the
dihedral group $\Dn{\cl{}}$.

There are $\cl{}$ infinite dihedral $H(\cl{},k)$ subgroups  of
$\Dn{\infty}$, for any $\cl{}$, $0\leq{k}<\cl{}$
({Bravais cell} of length \cl{}, with reflection point shifted $k$ steps):
\[
H(\cl{},k) = \{
\cdots, \shift_{-2 \cl{}}, \Refl_{-2\cl{}+k}, \shift_{-\cl{}},
\Refl_{-\cl{}+k}, 1,
\Refl_{k}, \shift_{\cl{}}, \Refl_{\cl{}+k}, \shift_{2 \cl{}},
\Refl_{2\cl{}+k}, \cdots
             \}
\,.
\]

The left cosets of the subgroup $H(\cl{},k)$ in $\Dn{\infty}$ are:
    \PC{2021-07-24}{
Explain that $\Refl_{j} H(\cl{},k)$ is a rearrangement.
    }
\bea
H(\cl{},k) &=& \{ \cdots, \shift_{-2 \cl{}}, \Refl_{-2\cl{}+k}, \shift_{-\cl{}},
\Refl_{-\cl{}+k}, 1, \continue
&& \Refl_{k}, \shift_{\cl{}}, \Refl_{\cl{}+k}, \shift_{2 \cl{}},
\Refl_{2\cl{}+k}, \cdots\}
    \label{H(n,k)cosets}\\
\shift H(\cl{},k) &=& \{ \cdots, \shift_{-2 \cl{}+1}, \Refl_{-2\cl{}+k+1},
\shift_{-\cl{}+1},
\Refl_{-\cl{}+k+1}, \shift \continue
&& \Refl_{k+1}, \shift_{\cl{}+1}, \Refl_{\cl{}+k+1}, \shift_{2 \cl{}+1},
\Refl_{2\cl{}+k+1}, \cdots\} \continue
&\vdots& \continue
\shift_{\cl{}-1} H(\cl{},k) &=& \{ \cdots, \shift_{-\cl{}-1},
\Refl_{-\cl{}+k-1}, \shift_{-1},
\Refl_{k-1}, \shift_{\cl{}-1}, \continue
&& \Refl_{\cl{}+k-1}, \shift_{2\cl{}-1}, \Refl_{2\cl{}+k-1}, \shift_{3\cl{}-1},
\Refl_{3\cl{}+k-1}, \cdots\}
\,.
\nnu
\eea
Using $\{1$, $\shift, \cdots, \shift_{\cl{}-1}\}$ as representatives of
these cosets we see that the quotient group $\Dn{\infty}/H(\cl{},k)$ is
isomorphic to the cyclic group $\Cn{\cl{}}$.

%\item[2021-07-27 Han]
To show that $H(\cl{},k)$ is not a normal subgroup: using \refeq{D_nConj} we have:
$\shift_i\,\Refl_k\shift_i^{-1} = \Refl_{k-2i}$. For $i \neq \cl{}$, generally
$\Refl_{k-2i} = \shift_{2i}\Refl_{k}$ is not an element of $H(\cl{},k)$.

%\item[2021-07-27 Han]
Let $\phi(\cl{})$ be a {\lattstate} that is invariant under the action
of subgroup $H(\cl{})$:
\bea
H(\cl{}) \phi(\cl{}) = \phi(\cl{}) \, ,
\eea
and $\phi(\cl{},k)$ be a {\lattstate} that is invariant under the action
of subgroup $H(\cl{},k)$:
\bea
H(\cl{},k) \phi(\cl{},k) = \phi(\cl{},k) \, .
\eea
Since $H(\cl{})$ is a normal subgroup of $\Dn{\infty}$, we have:
\bea
H(\cl{}) g \phi(\cl{})
&=& g H(\cl{}) g^{-1} g \phi(\cl{}) \continue
&=& g H(\cl{}) \phi(\cl{}) \continue
&=& g \phi(\cl{}) \, ,
\quad g \in \Dn{\infty} \, .
\eea
So $g \phi(\cl{})$ with $g \in \Dn{\infty}$ is also a {\lattstate} that
is invariant under $H(\cl{})$. For the {\lattstate} $\phi(\cl{},k)$ we
have:
\bea
g H(\cl{},k) g^{-1} g \phi(\cl{},k)
&=& g H(\cl{},k) \phi(\cl{},k) \continue
&=& g \phi(\cl{},k) \, ,
\quad g \in \Dn{\infty} \, .
\eea
Since $H(\cl{},k)$ is not a normal subgroup, $g H(\cl{},k) g^{-1}$ is a
conjugate subgroup of
$H(\cl{},k)$. So $g \phi(\cl{},k)$ with $g \in \Dn{\infty}$ is not
invariant under $H(\cl{},k)$, but invariant under a conjugate subgroup of
$H(\cl{},k)$.




                                      \jumpBack{exam:DinftyCosets}
\authorHL{2021-07-28}
    } %end \example{Symmetry groups  in physics:}{ \label{exmp:DinftyCosets}
%%%%%%%%%%%%%%%%%%%%%%%%%%%%%%%%%%%%%%%%%%%%%%%%%%%%%%%%%%%%%%%%%
