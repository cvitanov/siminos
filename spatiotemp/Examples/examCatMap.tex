% siminos/spatiotemp/chapter/examCatMap.tex called by catMap.tex
% $Author: predrag $ $Date: 2021-12-24 01:25:20 -0500 (Fri, 24 Dec 2021) $

% Predrag                                               23jan2018

\section{Examples}
\label{exam:catMap}

%%%%%%%%%%%%%%%%%%%%%%%%%%%%%%%%%%%%%%%%%%%%%%%%%%%%%%%%%%%%%%%%%%
\example{Projection operator decomposition of the cat map:}{
            \label{exam:ProjOpCatMap}
\index{projection operator}
Let's illustrate how the decomposition works for the \PV\rf{PerViv}
``two-configuration representation'' of the Arnol'd cat map by the
$[2\!\times\!2]$ matrix
\beq
{\bf A}
=\MatrixII{0}{1}
          {-1}{s}
\,.
\ee{examPVmap}
To interpret $\Ssym{n}$'s, consider the action of
the  % Newtonian cat
this map \refeq{OneCat} on a 2\dmn\ \statesp\ point
$(\ssp_{n-1},\ssp_{n})$,
\beq
 \left(\begin{array}{c}
 \ssp_{n}  \\
 \ssp_{n+1}
 \end{array} \right )=
 {\bf A} \left(\begin{array}{c}
 \ssp_{n-1}  \\
 \ssp_{n}
 \end{array} \right ) %\mbox{ mod } 1
 - \left(\begin{array}{c}
 0  \\
 \Ssym{n}
 \end{array} \right )
 \,.
\ee{eq:StateSpCatMap}
%Every point $(\ssp_{0},\ssp_{1})$  in the 2\dmn\ \statesp\ $\pS$ (drawn
%here as a unit square) defines a unique orbit.
In Percival and Vivaldi\rf{PerViv} this representation of cat map is
referred to as ``the two-configuration representation''.
As illustrated in
\reffig{fig:CatMapStatesp}, in one time step the area preserving
map $A'$ stretches the unit square into a parallelogram, and a
point $(\ssp_{0},\ssp_{1})$ within the initial unit square
in general lands outside it, in another unit square $\Ssym{n}$
steps away. As they shepherd such stray points back into the unit
torus, the integers $\Ssym{n}$ can be interpreted as ``winding
numbers''\rf{Keating91}, or ``stabilising impulses''\rf{PerViv}.
The $\Ssym{n}$ translations reshuffle the \statesp, thus partitioning it into
$|\A|$ regions $\pS_\Ssym{}$, $\Ssym{}\in\A$.

Associated with each root $\ExpaEig^{i}$
in \refeq{exam:catEigs} is the {\em projection operator}
%\refeq{2dEigVec}
% ${\PP}_i$,
\( %beq
{\PP}^{i} = \prod ({\bf A}-\ExpaEig^{j} \matId)
                 /(\ExpaEig^{i} -\ExpaEig^{j})
\, ,
\) %ee{ProjOp2d:3.42}
$j\not= i$,
\index{orthogonality relation}
\index{completeness relation}
\bea
\PP^{+} &=& \frac{1}{\surd{D}} ({\bf A} - \ExpaEig^{-1}\matId)
\,=\,\frac{1}{\surd{D}}\MatrixII{-\ExpaEig^{-1}}{1}
                         {-1}{\ExpaEig}
\label{ProjOp2d:8.2} \\
\PP^{-} &=& - \frac{1}{\surd{D}}({\bf A} -\ExpaEig\,\matId)
\,=\, \frac{1}{\surd{D}}\MatrixII{\ExpaEig}{-1}
                           {1}{-\ExpaEig^{-1}}
\,.
\label{ProjOp2d:8.3}
\eea
Matrices ${\PP}^{\pm}$ are orthonormal
%\( %beq
%{\PP}_i {\PP}_j = \delta_{ij} {\PP}_j \ , % \quad (\hbox{no sum on} \ j) \, ,
%\) %ee{ProjOp2d:3.44}
and complete.
%\( %beq
%\sum %_{i=1}^{r}
%     {\PP}_i = \matId \, .
%\) %ProjOp2d:3.45}
The dimension of the $i$th subspace is given by
\( %beq
d_i = \tr {\PP}_i \,;
\) %ee{ProjOp2d:3.46}
in case at hand both subspaces are 1-dimensional.
%
From the characteristic equation %\refeq{ProjOp2d:3.41}
% and the form of the projection operator \refeq{ProjOp2d:3.42}
it follows that ${\PP}^{\pm}$ satisfy the eigenvalue equation
\( %beq
{\bf A} \, {\PP}^{\pm} = \ExpaEig^{\pm} {\PP}^{\pm}
\,, %\qquad (\hbox{no sum on} \ i) \, .
\) %ee{ProjOp2d:3.47}
with every column a right
eigen\-vector, and every row a left eigen\-vector. Picking --for example-- the
first row/column we get the right and the left eigen\-vectors:
\bea
\{ \jEigvec[+],\jEigvec[-] \} &=& \left\{
    \frac{1}{\surd{D}}\VectorII{-\ExpaEig^{-1}}{-1}
    \,,
     \frac{1}{\surd{D}}\VectorII{\ExpaEig}{1} \right\}
    \continue
\{ \jEigvecT[+],\jEigvecT[-] \} &=& \left\{
   \frac{1}{\surd{D}}[-\ExpaEig^{-1},1]
    \,,
    \frac{1}{\surd{D}}[\ExpaEig,-1] \right\}
\,,
\label{PerViv:eigVecs}
\eea                                                    \toCB
with overall scale arbitrary.
    \PC{2017-10-02}{
    compare with \refeq{Adler98:leftEigVecs}
    }
The matrix is not symmetric, so
$\{\jEigvec[j]\}$ do not form an orthogonal basis. The left-right
eigen\-vector dot products $\jEigvecT[j]\,\cdot\,\jEigvec[k]$, however,
are orthogonal,
\[
  \jEigvecT[i] \cdot \jEigvec[j] = c_j\,\delta_{ij}
\,.
\]
What does this do to the partition \reffig{fig:CatMapStatesp}?
The origin is still the fixed point. A \statesp\ point in the new, dynamically
intrinsic right eigenvector \AW\rf{AdWei70} coordinate basis is
\[
\left(\begin{array}{c}
 \ssp_{n-1}  \\
 \ssp_{n}
 \end{array} \right )
 = (\PP^{+}+\PP^{-})
\left(\begin{array}{c}
 \ssp_{n-1}  \\
 \ssp_{n}
 \end{array} \right )
\]
\[
=
\frac{1}{\surd{D}}\MatrixII{-\ExpaEig^{-1}}{1}
                         {-1}{\ExpaEig}
\left(\begin{array}{c}
 \ssp_{n-1}  \\
 \ssp_{n}
 \end{array} \right )
+
\frac{1}{\surd{D}}\MatrixII{\ExpaEig}{-1}
                           {1}{-\ExpaEig^{-1}}
\left(\begin{array}{c}
 \ssp_{n-1}  \\
 \ssp_{n}
 \end{array} \right )
\]
\[
= \frac{1}{\surd{D}}
\left(\begin{array}{c}
 \ssp_{n}-\ExpaEig^{-1}\ssp_{n-1}  \\
 \ExpaEig\ssp_{n}-\ssp_{n-1}
 \end{array} \right )
+
\frac{1}{\surd{D}}
\left(\begin{array}{c}
 -\ssp_{n}+\ExpaEig\ssp_{n-1}  \\
 -\ExpaEig^{-1}\ssp_{n}+\ssp_{n-1}
 \end{array} \right )
\]
\[
  =
-(\ExpaEig\ssp_{n}-\ssp_{n-1})\frac{1}{\surd{D}}\VectorII{-\ExpaEig^{-1}}{-1}
+
(-\ExpaEig^{-1}\ssp_{n}+\ssp_{n-1})\frac{1}{\surd{D}}\VectorII{\ExpaEig}{1}
\]
\[
=(-\ExpaEig\ssp_{n}+\ssp_{n-1})\,\PP^{+}
+ (-\ExpaEig^{-1}\ssp_{n}+\ssp_{n-1})\,\PP^{-}
\,.
\]
The abscissa ($\ssp_{n-1}$ direction) is not affected, but the ordinate
($\ssp_{n}$ direction) is flipped and stretched/shrunk by factor $-\ExpaEig$,
$-\ExpaEig^{-1}$ respectively, preserving the vertical strip nature of the
partition \reffig{fig:CatMapStatesp}. In the \AW\ right eigenbasis,
${\bf A}$ acts by stretching the $\jEigvec[+]$ direction by $\ExpaEig$, and
shrinking the $\jEigvec[-]$ direction by  $\ExpaEig^{-1}$, without any rotation
of either direction.

%(Continued in \refexam{exmp:invman??}.)
    } %end \example{Projection operator decomposition in 2\dmn\
%%%%%%%%%%%%%%%%%%%%%%%%%%%%%%%%%%%%%%%%%%%%%%%%%%%%%%%%%%%%%%

%%%%%%%%%%%%%%%%%%%%%%%%%%%%%%%%%%%%%%%%%%%%%%%%%%%%%%%%%%%%%%%%%%%%%
\example{A linear cat map code.}{\label{exam:catLinSymDyn}
% from siminos/cats/catMap.tex called by GHJSC16
Eqs.~(\ref{exmPerViv2.1b},\ref{exmPerViv2.1a}) are the discrete-time Hamilton's
equations, which induce temporal evolution on the 2-torus
$(\ssp_{n},p_{n})$ {\em phase space}. For the problem at hand, it pays to
go from the Hamiltonian $(\ssp_{n},p_{n})$ phase space formulation to the
Newtonian (or Lagrangian) $(\ssp_{n-1},\ssp_{n})$ {\em state space}
formulation\rf{PerViv}, with $p_n$ replaced by
\(
p_n = (\ssp_{n} - \ssp_{n-1})/\Delta t \,.
\)
Eq.~(\ref{exmPerViv2.1a}) then takes the 3-term recurrence form (the discrete
time Laplacian $\Box$ formula for the second order time derivative
${d^2}/{dt^2}$, with the time step set to $\Delta t=1$),
\beq
\Box \, \ssp_n \equiv \ssp_{n+1} - 2\ssp_{n} + \ssp_{n-1} = P(\ssp_{n})  \qquad  \mod 1
\,,
\ee{PerViv2.2e}
\ie, Newton's Second Law: ``acceleration equals force.''
For a cat map, with force $P(x)$ linear in the displacement $x$, the
Newton's equation of motion \refeq{PerViv2.2e} takes form
\beq
(\Box  +2 - s)\,\ssp_{n} =-\Ssym{n}
%    \,,  \qquad
% \Box\,\ssp_{n}:= \ssp_{n-1}-2\ssp_{n} +\ssp_{n+1}
\,,
\ee{OneCat}
with $\mod 1$ enforced by $\Ssym{n}$'s, integers from  the alphabet
\beq
\A=\{\underline{1},0,\dots s\!-\!1\}
\,,
\ee{catAlphabet1}
necessary  to keep $\ssp_{n}$ for all times $t$ within the unit interval
$[0,1)$. The genesis of this alphabet is illustrated by
\reffig{fig:CatMapStatesp}.
We have introduced here the  symbol   $ \underline{|m_n|} $  to denote $m_n$
with the negative sign, \ie, `$\underline{1}$' stands for symbol `$-1$'.

%       \jumpBack{exam:catLinSymDyn}
    } % end \example{Linear code.}{exam:catLinSymDyn}
%%%%%%%%%%%%%%%%%%%%%%%%%%%%%%%%%%%%%%%%%%%%%%%%%%%%%%%%%%%%%%%%%%%%%

%%%%%%%%%%%%%%%%%%%%%%%%%%%%%%%%%%%%%%%%%%%%%%%%%%%%%%%%%%%%%%%%%
%  PC {2018-02-11}
\example{\FPoper\ for the Arnol'd cat map.}{\label{exam:FP_eigs_CatMap}
For a piecewise linear maps acting on a finite generating partition
the \FPoper\ takes the finite, transfer matrix form (see \refref{CBmeasure}).
\beq
{\bf L}_{ij}\,=\,\frac{|{\pS}_{i} \cap
f^{-1}({\pS}_{j})|}{|{\pS}_{i}|} \,, \quad \msr' = \msr {\bf L}
\ee{overl-int} % PC kept ChaosBook label here

The two rectangles and five sub-rectangle areas $|\pS_{j}|$
are given by inspection of \reffig{fig:PVAdlerWeissS}\,(a):
    \PC{2018-02-16}{to Han: PLEASE RECHECK}
\bea
|\pS_{A}| &=&  % {(\ExpaEig-1)}/{D} =
                 \ExpaEig/(\ExpaEig+1)              \,,\quad
|\pS_{B}|  =   % {(\ExpaEig-2)}/{D}  =
              1/(\ExpaEig+1)\,,\quad
    \continue
|\pS_{1}| &=&  {|\pS_{A}|}/{\ExpaEig}        \,,\qquad
|\pS_{2}|  =   {(\ExpaEig-1)}{|\pS_{B}|}/{\ExpaEig}  \,,\quad
|\pS_{3}|  =   {|\pS_{A}|}/{\ExpaEig}  \,,
    \continue
|\pS_{4}| &=&  {|\pS_{B}|}/{\ExpaEig}        \,,\qquad
|\pS_{5}|  =   {(\ExpaEig-2)}{|\pS_{A}|}/{\ExpaEig}
\,,
\label{CatMapAreas}
\eea
where $\ExpaEig$ and $D$ are given in \refeq{exam:catEigs}, and we are
considering the $s=3$ Arnol'd cat map case (the generalization to $s>3$.
is immediate) The areas are symplectic invariants, and thus the same in
any choice of cat-map coordinates.
    % \(
    % |\pS_{A}|+|\pS_{B}|=1  \,,\\
    % |\pS_{A{}^0\!A}| = |\pS_{A{}^1\!A}|
    %                  = |\pS_{A}|/\ExpaEig \,,\\
    % |\pS_{B{}^0\!B}| = |\pS_{B}|/\ExpaEig \,,\\
    % |\pS_{A{}^{\underline{1}}\!B}| =
    % |\pS_{B{}^1\!A}| = |\pS_{B}|(1-1/\ExpaEig) \,,\\
    % |\pS_{A{}^{\underline{1}}\!B}| = |\pS_{A}| - |\pS_{A{}^0\!A}| - |\pS_{A{}^1\!A}|
    %                                =(1-2/\ExpaEig)|\pS_{A}| \,,\\
    % |\pS_{A{}^{\underline{1}}\!B}| = |\pS_{B}| - |\pS_{B{}^0\!B}|
    %                                =(1-1/\ExpaEig)|\pS_{B}| \,,\\
    % \Rightarrow
    % |\pS_{B}|=\frac{\ExpaEig-2}{\ExpaEig-1}|\pS_{A}|
    %          % = (1- \frac{1}{\ExpaEig-1})|\pS_{A}|
    % \Rightarrow
    % (1+ \frac{\ExpaEig-2}{\ExpaEig-1})|\pS_{A}| = 1 \\
    % \Rightarrow\quad \hskip 5ex |\pS_{A}| = \frac{\ExpaEig-1}{2\ExpaEig-3} \,,\;\;
    % |\pS_{B}| = -\frac{\ExpaEig-2}{2\ExpaEig-3} \,.\\
    % \)
As in the Chaos\-Book example
{\em exam:FP\_eigs\_Ulam} (currently 19.1),  %%% FIX!
the \AW\ partitioned \PV\ cat map is an expanding
piecewise-linear map, so we can construct the associated transfer matrix %\FPoper\
explicitly,
by weighing the links of {\markGraph} \reffig{fig:PVAdlerWeissS}\,(a) by the
ratios of out-, in-\-rectangle areas $T_{kj}=
|\pS_{\Ssym{k}}|/|\pS_{\Ssym{j}}|$:
        \index{Perron-Frobenius!matrix}\index{Markov!matrix}
        \index{probability!matrix}\index{stochastic!matrix}
    \PC{2018-02-11}{the matrix is NOT CORRECT yet, FIX!}
\beq
\left[
\begin{array}{c}
 \phi_1' \\ \phi_2' \\ \phi_3' \\ \phi_4' \\ \phi_5' \\
\end{array}
\right]
=
{\bf T}\phi=
\frac{1}{\ExpaEig}
\left[
\begin{array}{ccccc}
 {1} & {\ExpaEig-2}{} & {1} & 0            &  0             \\
 0   & 0              & 0   & {\ExpaEig-1} &  1 \\
 {1} & {\ExpaEig-2}{} & {1} & 0            &  0             \\
 0   & 0              & 0   & {1} & {\ExpaEig-1}{}  \\
{1}  &{\ExpaEig-1}{} &\frac{\ExpaEig-2}{\ExpaEig-1} & {0} &  0 \\
\end{array}
\right]
\left[\begin{array}{c}
 \phi_1 \\ \phi_2 \\ \phi_3 \\ \phi_4 \\ \phi_5 \\
\end{array}\right]
\ee{catMapPFmat}
The probability for starting in initial
state $j$ is conserved,
\(
\sum_{{k}}L_{{k}{j}}=1
\,,
\)
as it should be.
Such non-negative matrix whose columns conserve
probability  is called {\em Markov}, {\em
probability} or {\em stochastic} matrix.
Thanks to the same expansion everywhere, and a finite \markGraph, the
Fredholm determinant is the characteristic polynomial of the transfer
matrix (currently Chaos\-Book Eq.~(18.13)) defined by the \markGraph\ of
\reffig{fig:Lect13p8}\,(c), expanded in non-intersecting loops
\(
t_A = T_{A{}^0\!A},
t_A' = T_{A{}^1\!A},
t_B = T_{B{}^0\!B},
t_{AB} = T_{A{}^{\underline{1}}\!B} T_{B{}^{1}\!A}
\,:
\)
\beq
\det (1-z{\bf T})
     =  1-z(t_A+t_A'+t_{B}) - z^2 t_{AB} + z^2 (t_A+t_A')t_{B}
     =  1-3\frac{z}{\ExpaEig} - (\ExpaEig -3)\frac{z^2}{\ExpaEig}
\,,
\label{CattyZeta}
\eeq
\beq
\det (1-z{\bf T})
     =  1-3\frac{z}{\ExpaEig} - (\ExpaEig -3)\frac{z^2}{\ExpaEig}
\,,
\label{CattyZeta1}
\eeq
in agreement with \refeq{CatMapDetTransMat}.
This counts the fixed point at the origin thrice (it lives
in the invariant subspace spanned by stable and unstable manifolds, the
border) so that has to be divided out.

Due to probability (unit area) conservation, ${\bf T}$ has a unit eigenvalue
$z=1=e^{\eigenvL_0}$,
with constant density eigen\-vector $\msr_0 =\msr_1$.


In the orbit-counting case one retrieves Isola's {$\zeta$}-function\rf{Isola90}
\refeq{Isola90-13}.

This simple explicit matrix representation of the  {\FPoper} is
a consequence of the piecewise linearity of the time-forward map,
and the restriction of the densities
$\msr$ to the space of  piecewise constant functions.
%~~(continued in \refexam{exam:Trans-matrices})
%                                           \jumpBack{exam:FP_eigs_Ulam}
} %end \example{Eigenvalues of the skew {\fullTent} \FPoper.}{
%%%%%%%%%%%%%%%%%%%%%%%%%%%%%%%%%%%%%%%%%%%%%%%%%%%%%%%%%%%%%%%%%


%%%%%%%%%%%%%%%%%%%%%%%%%%%%%%%%%%%%%%%%%%%%%%%%%%%%%%%%%%%%%%%%%%%%%
\example{Counting \templatt\ {\lattstate}s.}{\label{exam:tempCatCount}
% from \subsection{Counting \templatt\ {\lattstate}s (experimental)}
%      \label{s:tempCatCountTEMP}
% 2020-06-10 Predrag

The \templatt\ equation \refeq{catMapNewt} is
a linear {$2$nd-order inhomogeneous difference} equation
($3$-term recurrence relation) with constant coefficients
%\beq
%\ssp_{\zeit+1}  -  s \, \ssp_{\zeit} + \ssp_{\zeit-1}
%    =
%-\Ssym{\zeit}
%%\,.
%\ee{eq:CatMapNewton2}
that can be solved by standard methods\rf{Elaydi05} that
parallel the theory of linear differential equations.
    \PC{2020-06-10}{
    Comparing with \refeq{genFuncts:CatRec-s} we see that we need to
    solve a second-order inhomogeneous difference equation with a
    constant forcing term $2\,(s-2)$.
    }
Inserting a solution of form $\ssp_{\zeit}=\ExpaEig^\zeit$ into the
associated (\Ssym{\zeit}=0) homogenous {$2$nd-order difference equation}
\beq
\ssp_{\zeit+1} - {s}\,\ssp_{\zeit} + \ssp_{\zeit-1}= 0
\ee{diffEqs:CatCharEq}
yields the {characteristic equation}
\beq
\ExpaEig^{2} - {s}\ExpaEig + 1 = 0
\,,
\ee{diffEqs:StabMtlpr}
which, for $|s|>2$, has two real roots
% stability multipliers
$\{\ExpaEig\,,\;\ExpaEig^{-1}\}$,
\beq
\ExpaEig
=\frac{1}{2}(s+\sqrt{(s-2)(s+2)})
\,,
\ee{PCStabMtlpr}
and the so-called \emph{complementary} solution of form
\beq
\ssp_{c,\zeit}  = a_1\ExpaEig^\zeit+a_{-1}\ExpaEig^{-\zeit}
\,.
\label{PC(2.3.4)}
\eeq
% where constants $a_i$ can be determined by specifying
% $\{\ssp_{0},\ssp_{1}\}$.

A difference of any pair of solutions to the \templatt\
inhomogenous equation \refeq{catMapNewt}
%\beq
%\ssp_{\zeit+1} - {s}\,\ssp_{\zeit} + \ssp_{\zeit-1}= -\Ssym{\zeit}
%%\,,
%\ee{PC(2.4.4)}
is a solution of the homogenous difference equation
\refeq{diffEqs:CatCharEq}, so the general solution is a sum of the
{complementary} solution \refeq{PC(2.3.4)} and a \emph{particular}
solution $\ssp_{p}$,
\beq
\ssp_{\zeit} = \ssp_{c,\zeit} + \ssp_{p,\zeit}
\,.
\ee{PC(2.4.3)}
Eq.~\refeq{diffEqs:CatCharEq} is time-reversal invariant,
$\ssp_{\zeit} = \ssp_{-\zeit}$, so $a_1=a_{-1}=a$.
To determine the particular solution, assume that both the source
 $\Ssym{\zeit}=\Ssym{}$
and $\ssp_{p,\zeit}=\ssp_{p}$
 in \refeq{catMapNewt} are site-independent,
\beq
\ssp_{p}  -  s \,\ssp_{p} + \ssp_{p}
    = -\Ssym{}
\,,
\ee{eq:CatMapNewton5}
so
\(
%  2\,\ssp_{p}-s\,\ssp_{p}={M}
%  \quad \to \quad
  \ssp_{p} = \Ssym{}/(s-2)
\,.
\)
Hence the solution is
\beq
\ssp_{\zeit} = \ssp_{c,\zeit} + \ssp_{p,\zeit}
= a\left(\ExpaEig^{\zeit} + \ExpaEig^{-\zeit}\right) + {\Ssym{}}/(s-2)
\,,
\ee{Chen11:1stepDiffSolu}
with $a_i$ determined by fields at two lattice sites,
\[
\ssp_{0}= 2a + {\Ssym{}}/(s-2)
\,,\quad
\ssp_{1}= a\left(\ExpaEig + \ExpaEig^{-1}\right) + {\Ssym{}}/(s-2)
\,,\quad
\,.
\]
\tempLatt\ starts with $N_{0}=0$, and according to \refeq{catFundPar1},
$N_{1}=s-2$, so $a=1$, $\Ssym{}=-2(s-2)$, and the number
of temporal {\lattstate}s of period $\cl{}$ is
\beq
N_{\cl{}} =
    \ExpaEig^{\cl{}} + \ExpaEig^{-\cl{}} - 2
\,.
\ee{PC:1stepDiffSolu}
%       \jumpBack{exam:XXX}
    } % end \exampleXXX.}{exam:XXX}
%%%%%%%%%%%%%%%%%%%%%%%%%%%%%%%%%%%%%%%%%%%%%%%%%%%%%%%%%%%%%%%%%%%%%
