% siminos/cats/nonlin-v1/reviews/response1-v1.tex
% $Author: predrag $ $Date: 2020-11-18 00:46:59 -0500 (Wed, 18 Nov 2020) $

% Predrag                                       2020-11-16
% Predrag                                       2020-10-26
% started with siminos/cats/nonlin-v2/response-v1.tex
% Boris                                         2020-09-25

%%%%%%%%%%%%%%%%%%%%%%%%%%%%%%%%%%%%%%%%%%%%%%%%%%%%%%%%%%%%%%%%%%%%%%%%%%
                        \newif\ifboyscout\boyscouttrue      %% commented %%
                        \newif\ifsubmission\submissionfalse
                        \newif\ifhighlightedits\highlighteditstrue
% Toggle between draft and public versions:
% \boyscoutfalse\highlighteditsfalse    % public, hyperlinked
% \boyscoutfalse\highlighteditstrue     % for Nonlinearity referees
% \boyscoutfalse\submissiontrue         % for Nonlinearity
%%%%%%%%%%%%%%%%%%%%%%%%%%%%%%%%%%%%%%%%%%%%%%%%%%%%%%%%%%%%%%%%%%%%%%%%%%

\documentclass[12pt]{iopart}
% loads AMS amsgen, amsfonts, amsbsy, amssymb:
\usepackage{iopams} % to load AMS extension fonts msam and msbm
                     % the blackboard bold alphabet, extra maths symbols
                     % and extra definitions for bold Greek letters.
                     % do not use amsmath.sty

\pdfminorversion=4  % the very start the TeX file, so PDF or bitmap figures
                    % are PDF version 1.4 or lower

\usepackage{graphicx}              % is the recommended load line
\graphicspath{{../figs/}{../Fig/}}  %% directories with graphics files
\usepackage{color} % dvips allows for colors
\usepackage[colorlinks]{hyperref} %% hyperlinks
\usepackage{amsthm}
\usepackage{parskip}
\input{../biblatex}    % this makes references hyperlinked
\addbibresource{../../bibtex/siminos.bib}

% siminos/cats/defsCats.tex
% $Author: predrag $ $Date: 2018-03-26 13:14:38 -0400 (Mon, 26 Mar 2018) $

%%%%%%%%%%%%%% GHJSC16 specific %%%%%%%%%%%%%%%%%%%%%%%%%%%%
\newcommand{\conf}{\ensuremath{x}} %Configuration space coordinate
\newcommand{\Fu}{\tilde{u}}
%%%%%%%%%%%%%%%%%%%%%%%%%%%%%%%%%%%%%%%%%%%%%%%%%%%%%%%%%%%


\newcommand{\NBBpost}[2]{\item[#1 Burak] {#2}}
\newcommand{\PCpost}[2]{\item[#1 Predrag] {#2}}
\newcommand{\BGpost}[2]{\item[#1 Boris] {#2}}
\newcommand{\AKSpost}[2]{\item[#1 Adrien] {#2}}
\newcommand{\RJpost}[2]{\item[#1 Rana] {#2}}
\newcommand{\LHpost}[2]{\item[#1 Li Han] {#2}}
\newcommand{\MNGpost}[2]{\item[#1 Matt] {#2}}
\newcommand{\XDpost}[2]{\item[#1 Xiong] {#2}}

\ifboyscout %%%%%%%% DISPLAY COMMENTS IN THE TEXT %%%%%%%%%%%%%%%%%%%%
            %%%%%%%% turn on labeling of equations on margins %%%%%%%%
    % also search the text for lines starting with %%  to
    % locate various internal comments, recent edits etc.
    \typeout{============ COMMENTED =====}
  \newcommand{\PublicPrivate}[2]
    {\marginpar{\color{blue}$\Downarrow$\footnotesize PRIVATE}%
    {\color{blue}#2}%
    \marginpar{\color{blue}$\Uparrow$\footnotesize PRIVATE}}
  \newcommand{\PC}[2]{$\footnotemark\footnotetext{Predrag #1: #2}$}
  % \newcommand{\PC}[2]{\\{\color{red} [{Predrag #1: #2}]}\\}
  \newcommand{\PCedit}[1]{{\color{magenta}#1}}
  \newcommand{\BG}[2]{$\footnotemark\footnotetext{Boris #1: #2}$}
  \newcommand{\BGedit}[1]{{\color{red}#1}}
  \newcommand{\AKS}[2]{$\footnotemark\footnotetext{Adrien #1: #2}$}
  \newcommand{\AKSedit}[1]{{\color{green}#1}}
  \newcommand{\RJ}[2]{$\footnotemark\footnotetext{Rana #1: #2}$}
  \newcommand{\RJedit}[1]{{\color{blue}#1}}
  \newcommand{\MNG}[2]{$\footnotemark\footnotetext{Matt #1: #2}$}
  \newcommand{\MNGedit}[1]{{\color{red}#1}}
%  \newcommand{\BB}[2]{$\footnotemark\footnotetext{Burak #1: #2}$}
  \newcommand{\BBedit}[2]{{\color{red}#1}}
  \newcommand{\Xiong}[2]{$\footnotemark\footnotetext{XD #1: #2}$} %date, comment
  \newcommand{\Xiongedit}[1]{{\color{green}#1}}
  \newcommand{\Private}[1]{{\color{blue}#1}}
  \newcommand{\toCB}{\marginpar{\footnotesize 2CB}}  % to compare with ChaosBook
  \newcommand{\inCB}{\marginpar{\footnotesize now in CB}} % entered in ChaosBook
  \newcommand{\CBlibrary}[1]
             {\href{http://ChaosBook.org/library/#1.pdf} { (click here)}}
\else % drop comments
      % do not turn on labeling of equations on margins
  \typeout{============ UNCOMMENTED =====}
  \newcommand{\PublicPrivate}[2]{#1}
  \newcommand{\PC}[2]{}
  \newcommand{\PCedit}[1]{#1}
  \newcommand{\AKS}[2]{}
  \newcommand{\AKSedit}[1]{#1}
  \newcommand{\RJ}[2]{}
  \newcommand{\RJedit}[1]{#1}
%  \newcommand{\BB}[2]{}{}
  \newcommand{\BBedit}[1]{#1}
  \newcommand{\Xiong}[2]{}{} %date, comment
  \newcommand{\Xiongedit}[1]{#1}
  \newcommand{\Private}[1]{}
  \newcommand{\toCB}{}
  \newcommand{\inCB}{}
  \newcommand{\CBlibrary}[1]{}
\fi  %%%%%%%%%%%% END OF ON/OFF COMMENTS SWITCH %%%%%%%%%%%%%%%%%%%%

%%%%%%%%%%%%%%% EQUATIONS %%%%%%%%%%%%%%%%%%%%%%%%%%%%%%%
\newcommand{\beq}{\begin{equation}}
\newcommand{\continue}{\nonumber \\ }
\newcommand{\nnu}{\nonumber}
\newcommand{\eeq}{\end{equation}}
\newcommand{\ee}[1] {\label{#1} \end{equation}}
\newcommand{\bea}{\begin{eqnarray}}
\newcommand{\ceq}{\nonumber \\ & & }
\newcommand{\eea}{\end{eqnarray}}
\newcommand{\barr}{\begin{array}}
\newcommand{\earr}{\end{array}}

%%%%%%%%%%%%%%% REFERENCING EQUATIONS ETC, Nonlinearity style only %%%%%%%
\newcommand{\rf}     [1] {~\cite{#1}}
\newcommand{\refref} [1] {\cite{#1}}
\newcommand{\refRef} [1] {\cite{#1}}
\newcommand{\refrefs}[1] {\cite{#1}}
\newcommand{\refRefs}[1] {\cite{#1}}
\newcommand{\refeq}  [1] {(\ref{#1})}
            % in amstex, \eqref is predefined and better than \refeq
\newcommand{\refeqs} [2]{(\ref{#1}--\ref{#2})}
\newcommand{\reffig} [1] {figure~\ref{#1}}
\newcommand{\reffigs} [2] {figures~\ref{#1} and~\ref{#2}}
\newcommand{\refFig} [1] {Figure~\ref{#1}}
\newcommand{\refFigs} [2] {Figures~\ref{#1} and~\ref{#2}}
\newcommand{\reftab} [1] {table~\ref{#1}}
\newcommand{\refTab} [1] {Table~\ref{#1}}
\newcommand{\reftabs}[2] {tables~\ref{#1} and~\ref{#2}}
\newcommand{\refsect}[1] {sect.~\ref{#1}}
\newcommand{\refsects}[2] {sects.~\ref{#1} and \ref{#2}}
\newcommand{\refSect}[1] {Sect.~\ref{#1}}
\newcommand{\refSects}[2] {Sects.~\ref{#1} and \ref{#2}}
\newcommand{\refsecttosect}[2] {Sects.~\ref{#1} to~\ref{#2}}
\newcommand{\refchap}[1] {chapter~\ref{#1}}
\newcommand{\refappe}[1] {appendix~\ref{#1}}
\newcommand{\refappes}[2] {appendices~\ref{#1} and~\ref{#2}}
\newcommand{\refAppe}[1] {Appendix~\ref{#1}}
\newcommand{\refexam}[1] {example~\ref{#1}}
\newcommand{\refExam}[1] {Example~\ref{#1}}

\newcommand{\cl}[1]{{\ensuremath{|#1|}}}  % the length of a periodic orbit, Ronnie

%%%%%%%%%%%%%%% ChaosBook Abbreviations %%%%%%%%%%%%%%%%%%%%%%%%

\newcommand{\statesp}{state space}
\newcommand{\Statesp}{State space}
\newcommand{\stateDsp}{state-space}
\newcommand{\StateDsp}{State-space}
\newcommand{\fixedpnt}{fixed point}
\newcommand{\Fixedpnt}{fixed point}
\newcommand{\jacobian}{Jacobian}        % determinant
% \newcommand{\jacobianM}{fundamental matrix} % no known standard name?
% \newcommand{\jacobianMs}{fundamental matrices}  %
% \newcommand{\JacobianM}{Fundamental matrix} %
% \newcommand{\JacobianMs}{Fundamental matrices}  %
\newcommand{\jacobianM}{Jacobian matrix}  % back to Predrag's name 20oct2009
\newcommand{\jacobianMs}{Jacobian matrices}   % matrices
\newcommand{\JacobianM}{Jacobian matrix} %
\newcommand{\JacobianMs}{Jacobian matrices}  %
\newcommand{\FloquetM}{Floquet matrix} % specialized to periodic orb
\newcommand{\FloquetMs}{Floquet matrices}  %
% \newcommand{\stabmat}{matrix of variations}   % Arnold, says Vattay
\newcommand{\stabmat}{stability matrix}     % stability matrix, velocity gradients
\newcommand{\Stabmat}{Stability matrix}     % Stability matrix
\newcommand{\stabmats}{stability matrices}
\newcommand{\monodromyM}{monodromy matrix} % monodromy matrix, Poincare cut
\newcommand{\MonodromyM}{Monodromy matrix} % monodromy matrix, Poincare cut
\newcommand{\dzeta}{dyn\-am\-ic\-al zeta func\-tion}
\newcommand{\Dzeta}{Dyn\-am\-ic\-al zeta func\-tion}
\newcommand{\tzeta}{top\-o\-lo\-gi\-cal zeta func\-tion}
\newcommand{\Tzeta}{Top\-o\-lo\-gi\-cal zeta func\-tion}
%\newcommand{\tzeta}{Artin-Mazur zeta func\-tion} %alternative to topological
\newcommand{\Gt}{Gutz\-willer trace formula}
\newcommand{\Fd}{spec\-tral det\-er\-min\-ant}
%\newcommand{\fd}{spec\-tral det\-er\-min\-ant} %in many articles
\newcommand{\FD}{Spec\-tral det\-er\-min\-ant}
\newcommand{\cycForm}{cycle averaging formula}
\newcommand{\CycForm}{Cycle averaging formula}
\newcommand{\pdes}{partial differential equations}
\newcommand{\Pdes}{Partial differential equations}
\newcommand{\dof}{dof}         % Hamiltonian deegree of freedom
% \newcommand{\dof}{deegree of freedom}


%%%%%%%%%%%%%%% relative periodic orbits: %%%%%%%%%%%%%%%%%%%%%%%%%%%%
\newcommand{\po}{periodic orbit}
\newcommand{\Po}{Periodic orbit}
\newcommand{\rpo}{rela\-ti\-ve periodic orbit}
\newcommand{\Rpo}{Rela\-ti\-ve periodic orbit}
\newcommand{\ppo}{pre-periodic orbit}
\newcommand{\Ppo}{Pre-periodic orbit}
\newcommand{\eqv}{equi\-lib\-rium}
\newcommand{\Eqv}{Equi\-lib\-rium}
\newcommand{\eqva}{equi\-lib\-ria}
\newcommand{\Eqva}{Equi\-lib\-ria}
\newcommand{\reqv}{rela\-ti\-ve equi\-lib\-rium}
%   \newcommand{\reqv}{travelling wave}
\newcommand{\Reqv}{Rela\-ti\-ve equi\-lib\-rium}
%   \newcommand{\Reqv}{travelling wave}
\newcommand{\reqva}{rela\-ti\-ve equi\-lib\-ria}
\newcommand{\Reqva}{Rela\-ti\-ve equi\-lib\-ria}
\newcommand{\equilibrium}{equi\-lib\-rium}
\newcommand{\equilibria}{equi\-lib\-ria}
\newcommand{\Equilibria}{Equi\-lib\-ria}
% \newcommand{\equilibrium}{steady state}
% \newcommand{\equilibria}{steady states}
% \newcommand{\Equilibria}{Steady states}

%%%%%%%%%%%%%%% SECTIONS, SLICES %%%%%%%%%%%%%%%%%%%%%%%%%%%%%%%%%

\newcommand{\expct}    [1]{\langle {#1} \rangle}
\newcommand{\spaceAver}[1]{\langle {#1} \rangle}
%\newcommand{\expct}    [1]{\left\langle {#1} \right\rangle}
%\newcommand{\spaceAver}[1]{\left\langle {#1} \right\rangle}
\newcommand{\timeAver} [1]{\overline{#1}}
\newcommand{\norm}[1]{\left\Arrowvert \, #1 \, \right\Arrowvert}
\newcommand{\pS}{\ensuremath{{\cal M}}}          % symbol for state space
\newcommand{\ssp}{\ensuremath{x}}                % state space point
\newcommand{\Poincare}{Poincar\'e }
\newcommand{\PoincSec}{Poincar\'e section}
\newcommand{\equivariantsp}{equivariant {\statesp}} % full state space
\newcommand{\Equivariantsp}{Equivariant {\statesp}}
% \newcommand{\reducedsp}{orbit space}
% \newcommand{\Reducedsp}{Orbit space}
\newcommand{\reducedsp}{reduced state space}
\newcommand{\Reducedsp}{Reduced state space}
\newcommand{\fixedsp}{fixed-point subspace}
\newcommand{\Fixedsp}{Fixed-point subspace}
\newcommand{\mslices}{method of slices}
\newcommand{\Mslices}{Method of slices}
\newcommand{\mframes}{method of moving frames}
\newcommand{\Mframes}{Method of moving frames}
\newcommand{\templates}{templates} % {slice-fixing point} % {reference state}
\newcommand{\movframe}{moving frame}
\newcommand{\movFrame}{Moving frame}
\newcommand{\comovframe}{comoving frame}
\newcommand{\comovFrame}{Comoving frame}
\newcommand{\mconn}{method of \comovframe s}
\newcommand{\Mconn}{Method of \comovframe s}
\newcommand{\fFslice}{first Fourier mode slice}
\newcommand{\FFslice}{First Fourier mode slice}
\newcommand{\poincBord}{section border}
\newcommand{\PoincBord}{Section border}
% \newcommand{\poincBord}{\PoincSec\ border}
% \newcommand{\PoincBord}{\PoincSec\ border}
% \newcommand{\poincBord}{border of transversality}
\newcommand{\template}{template} % {slice-fixing point} % {reference state}
\newcommand{\pSRed}{\ensuremath{\hat{\cal M}}} % reduced state space Jan 2012
%\newcommand{\pSRed}{\ensuremath{\bar{\cal M}}} % reduced state space
\newcommand{\sspRed}{\ensuremath{\hat{\ssp}}}    % reduced state space point Jan 2012
% \newcommand{\sspRed}{\ensuremath{y}}    % reduced state space point, experiment
% \newcommand{\sspRed}{\ensuremath{\bar{x}}}    % reduced state space point
\newcommand{\csspRed}{\ensuremath{\hat{u}}}      % Symmetry reduced complex state space point
\newcommand{\velRed}{\ensuremath{\hat{\vel}}}    % ES reduced state space velocity Jan 2012
% \newcommand{\velRed}{\ensuremath{\bar{v}}}    % PC reduced state space velocity
% \newcommand{\velRed}{\ensuremath{u}}    % ES reduced state space velocity
\newcommand{\MvarRed}{\ensuremath{\hat{\Mvar}}}  %Reduced stability matrix
\newcommand{\velRel}{\ensuremath{c}}    % relative state or phase velocity
\newcommand{\phaseVel}{phase velocity}      % pipe slicing
\newcommand{\phaseVels}{phase velocities}   % pipe slicing
\newcommand{\PhaseVel}{Phase velocity}      % pipe slicing
\newcommand{\PhaseVels}{Phase velocities}   % pipe slicing

\newcommand{\slicep}{{\ensuremath{\sspRed'}}}   % slice-fixing point Jan 2012
% \newcommand{\slicep}{{\ensuremath{y'}}}   % slice-fixing point, experimental
% \newcommand{\slicep}{\ensuremath{\ssp'}}   % slice-fixing point
%\newcommand{\sliceTan}[1]{\ensuremath{t_{#1}(y')}}    % tangent at slice-fixing, experimental
\newcommand{\sliceTan}[1]{\ensuremath{t'_{#1}}}    % group orbit tangent at slice-fixing
\newcommand{\groupTan}{\ensuremath{t}}    % group orbit tangent

\newcommand{\zeit}{\ensuremath{t}}  %time variable Ashley
\newcommand{\sspSing}{\ensuremath{\ssp^\ast}} 	% inflection point
\newcommand{\sspRSing}{\ensuremath{\sspRed^\ast}} 	% inflection point, reduced space

%%%%%%%%%%%%%%% Group theory %%%%%%%%%%%%%%%%%%%%%%
%\newcommand{\Group}{\ensuremath{\Gamma}}    % Siminos Lie group
\newcommand{\Group}{\ensuremath{G}}         % Predrag Lie or discrete group
\newcommand{\LieEl}{\ensuremath{g}}  % Predrag group element
%\newcommand{\Lg}{\mathfrak{a}}             % Siminos Lie algebra generator
\newcommand{\Lg}{\ensuremath{\mathbf{T}}}   % Predrag Lie algebra generator
\newcommand{\gSpace}{\ensuremath{{\bf \phi}}}   % MA group rotation parameters
% \newcommand{\gSpace}{\ensuremath{{\bf \theta}}}   % PC group rotation parameters

%%%%%%%% Siminos macros %%%%%%%%%%%%%%%%%%%%%%%%%%%%%%
\newcommand{\Rls}[1]{\ensuremath{\mathbb{R}^{#1}}}
\newcommand{\ii}{\ensuremath{\mathrm{i}}} % sqrt{-1}
\newcommand{\Un}[1]{\ensuremath{\textrm{U}(#1)}}         % in DasBuch
\newcommand{\SUn}[1]{\ensuremath{\textrm{SU}(#1)}}         % in DasBuch
%\newcommand{\On}[1]{\ensuremath{\mathbf{O}(#1)}}
\newcommand{\On}[1]{\ensuremath{\textrm{O}(#1)}}
%\newcommand{\SOn}[1]{\ensuremath{\mathbf{SO}(#1)}} % in Siminos thesis
\newcommand{\SOn}[1]{\ensuremath{\textrm{SO}(#1)}}         % in DasBuch
\newcommand{\Spn}[1]{\ensuremath{\textrm{Sp}(#1)}}         % in DasBuch
%\newcommand{\Dn}[1]{\ensuremath{\mathbf{D}_{#1}}    % in Siminos thesis
\newcommand{\Dn}[1]{\ensuremath{\textrm{D}_{#1}}}              % in DasBuch
%\newcommand{\Zn}[1]{\ensuremath{\mathbf{Z}_{#1}}}    % in Siminos thesis
\newcommand{\Zn}[1]{\ensuremath{\textrm{C}_{#1}}}              % in DasBuch
%\newcommand{\Ztwo}{\ensuremath{\mathbf{Z}_2}}      % in Siminos thesis
\newcommand{\Ztwo}{\ensuremath{\textrm{C}_2}}                % in DasBuch
%\newcommand{\Refl}{\ensuremath{\kappa}}            % Siminos uses R for rotations.
\newcommand{\Refl}{\ensuremath{\sigma}}             % in DasBuch
%\newcommand{\Shift}{\ensuremath{\tau}}
\newcommand{\Rot}[1]{\ensuremath{C^{#1}}}           % in DasBuch, e.g. C^{1/3}
%\newcommand{\Rot}[1]{\ensuremath{R(#1)}}           % Siminos uses R for rotations.
%\newcommand{\Drot}{\ensuremath{\zeta}}
%\newcommand{\Lg}{\mathcal{G}}
%\newcommand{\stab}[1]{\ensuremath{\Sigma_{#1}}}
\newcommand{\stab}[1]{\ensuremath{G_{#1}}}
\newcommand{\shift}{\ensuremath{d}}
\newcommand{\Shift}{\ensuremath{\tau}}
\newcommand{\Fix}[1]{\ensuremath{\mathrm{Fix}\left(#1\right)}}

%%%%%%%%%%%%%% ks.tex specific %%%%%%%%%%%%%%%%%%%%%%%%%%%%
\newcommand{\KS}{Ku\-ra\-mo\-to-Siva\-shin\-sky}
\newcommand{\KSe}{Ku\-ra\-mo\-to-Siva\-shin\-sky equation}
\newcommand{\pCf}{plane Couette flow}
\newcommand{\PCf}{Plane Couette flow}
\newcommand{\dmn}{-dimensional}  %  experimental 220ct2009
%\newcommand{\dmn}{\ensuremath{d}}  %  n-dimensional
%\newcommand{\dmn}{\ensuremath{\!-\!d}}  %  n-dimensional
\newcommand{\expctE}{\ensuremath{E}}    % E space averaged
\newcommand{\tildeL}{\ensuremath{\tilde{L}}}
\newcommand{\EQV}[1]{\ensuremath{EQ_{#1}}} %experimental
% \newcommand{\EQV}[1]{\ensuremath{q_{#1}}} %ChaosBook
% \newcommand{\EQV}[1]{\ensuremath{E_{#1}}} %Ruslan
% E_0: u = 0 - trivial equilibrium
% E_1,E_2,E_3, for 1,2,3-wave equilibria
\newcommand{\REQV}[2]{\ensuremath{TW_{#1#2}}} % #1 is + or -
% TW_1^{+,-} for 1-wave traveling waves (positive and negative velocity).
\newcommand{\PO}[1]{\ensuremath{PO_{#1}}}
% PO_{period to 2-4 significant digits} - periodic orbits
\newcommand{\RPO}[1]{\ensuremath{\overline{rpo}_{#1}}} % Xiang experimental
%\newcommand{\RPO}[1]{\ensuremath{RPO_{#1}}}
% RPO_{period to 2-4 significant digits} - relative PO.  We use ^{+,-}
% to distinguish between members of a reflection-symmetric pair.
% \newcommand{\PPO}[1]{\ensuremath{PPO_{#1}}}
\newcommand{\PPO}[1]{\ensuremath{\overline{ppo}_{#1}}} % Xiang experimental
% Gibson likes:
\newcommand{\tEQ}{\ensuremath{{EQ}}}

















%%%%%%%%%%%% REMOVE THIS EVENTUALLY %%%%%%%%%%%
%
      %% all article-specific edits: \renewcommand, etc

\pagestyle{plain}
\renewcommand{\baselinestretch}{1.0}
\setlength{\parindent}{2em}

\begin{document}
                    \hfill November 15, 2020

\bigskip
\noindent
---------------------------------------------------------------------- \\
Referee 1 Report -- NON-104197 / Gutkin et al \\
---------------------------------------------------------------------- \\

\bigskip
\bigskip



\noindent
We cannot express strongly enough how very grateful we are to the referee
for the exceptional care taken in reading our manuscript, and many
improvements that referee's suggestions have led to.

In what follows, referee's remarks are followed by our responses, either
as indented ``{\em Response~~~} $\cdots$'' blocks of text, or as
\edit{blue text}. In the attached revised manuscript, all edits
are indicated by \edit{blue text}.
\bigskip\bigskip

The referee writes:
[$\cdots$], besides the customary small corrections, two substantial
problems must be properly addressed:

\begin{enumerate}
  \item
The meaning of some key constructs is obscured by poor notation and terminology.
  \item
Some key results should be formalised as theorems; some numerical results should
be distilled into conjectures.
\end{enumerate}

\section*{MAIN POINTS}

\begin{enumerate}
  \item
The definition and usage of the main objects of study is confused by
inappropriate or ambiguous notation and unsettled terminology
[$\cdots$].

On p.3, the quantity
 \(
\{\ssp_{z} \in  \mathbb{T}^{1},  z\in \integers^{\edit{2}}\}
 \)
is first a `lattice state' (shorthanded
$\{\ssp_{z}\}$), and
few lines below a `spatiotemporal solution', called \Xx.

As written, \Xx\ is rather a countable
collection of circles, indexed by $\integers^{2}$.

\begin{quote}
{\em Response~~~}
We feel that a reader is best served by a succession of definitions.
First we say that the state space of this
problem has a (local) field $\ssp_{z}$ at each integer lattice site
$z\in \integers^{\edit{2}}$ restricted to
a circle,
\(
\{\ssp_{z} \in  \mathbb{T}^{1}\}
 \).
Then we say that we are considering only those (global) lattice states
\Xx\ which -in $d$ dimensions- are solutions of a specific linear matrix equation,
\bea
 (-\Box + d(s-2))\,\ssp_{z} &=& \Ssym{z}
\,,
\label{LinearConn}
\eea
an equation related to
discrete screened Poisson equation\rf{Dorr70,HuCon96}.
\end{quote}


If its elements $\ssp_{z}$ are to be connected by \refeq{LinearConn}, then surely this
must be part of the symbolic definition. For the latter, the use a
Zermelo-type set notation is plainly inappropriate, since \Xx\ is not a
set (see below).

The quantity \Mm\ suffers from the same problem, compounded by an
excessive overloading of the symbol. Initially (bottom of p.3) \Mm\ is
referred to as a `block', then (p. 17) as
the `integer lattice',
    \begin{quote}
{\em Response~~~}
You are right, we have now changed
``As we now show,  \edit{the integer lattice}
\(
\Mm= \{\Ssym{z} \in \A \,,\; z\in \integers^2 \}
\)
can be used...'' to
``As we now show,
\(
\Mm= \{\Ssym{z} \in \A \,,\; z\in \integers^2 \}
\)
can be used...''
Were we to use term `integer lattice' (we do not, in this paper), it
would refer \emph{only} to the spatiotemporal $\integers^{d}$ lattice.
    \end{quote}
a `code',
and then again a `block'. On page 4 and beyond, the term `block' means a
finite symbolic array. These blocks appear as $\Mm_\R$, to become just
$b$ in one dimension (bottom of p.11).
    \begin{quote}
{\em Response~~~} We now explain after eq.~(19) that, as
for the cat map a symbol \brick\ $\Mm_{\R}$ is 1\dmn, in the cat-map section
it is convenient to replace \brick\ $\Mm_{\R}$
over domain \R\ by a \brick\ $b$ of length $\ell$.
    \end{quote}

The nature of the subscripts of \Mm\ fluctuates throughout the paper. In
one dimension $b$ is a block, but in two dimensions \R\ is not a block
but rather a domain.
    \begin{quote}
{\em Response~~~}
We are puzzled. We use term `\brick' only in reference to \emph{symbol}
\brick.
\R\ is a collection of contiguous lattice sites, mostly rectangular in
shape, but not always (see Figure~10). \R\ it is \emph{never} a `\brick'.
    \end{quote}
On p. 28, two lines after being reminded that the
subscript $\R_i$ denotes a restriction, the stranded reader will
encounter integer subscripts.
    \begin{quote}
{\em Response~~~}
When one considers many domains \R\ simultaneously, one has to either
distinguish domains by labels, like $\R_i$, or give them new names, like
$A_j$.
    \end{quote}




Let me indicate how to restore some order. \Xx\ and \Mm\ are \emph{functions}
$\Xx : \integers^{2}\to[0,1)$, and
$\Mm : \integers^{2}\to\A$.
The former is
defined recursively from an initial set of data (which, incidentally,
needs to be described)

\begin{quote}
{\em Response~~~}
No recursion here. \Xx\ is a solution of the linear matrix equation
\refeq{LinearConn},
a relative of Helmholtz equation sometimes called
discrete screened Poisson equation\rf{Dorr70,HuCon96}.
A given  \brick\ \Mm\ is \emph{admissible} if all
$\Xx_\Mm=\{\ssp_{z}\}$ are within the unit interval.

There is no ``initial set of data'', it's just a linear, matrix equation
which for given \Mm\ either has a solution
$\Xx : \integers^{2}\to[0,1)^{\ell_1\ell_2}$,
or not.

A physicist or an engineer would be confused
if we were to refer to -let's stay-
the Maxwell electromagnetic \emph{field}
as the Maxwell \emph{function}, so we would prefer stick to
the established disorder of physics terminology of referring to ``lattice states"
and ``symbol \brick s,'' rather than \emph{functions}
\Xx\ and \Mm.
\end{quote}

$\Mm : \integers^{2}\to\A$ is defined from \Xx\ via a linear
relation. A block $B$ ($b$ in one dimension) is also a function $B_\R : \R\to\A$,
whose finite and convex domain \R\ -the \emph{base} of the block- is inscribed
in a rectangle $\Pi_\R$ of minimal size. The latter is then embedded
canonically in $\integers^{2}$, to allow comparisons with the restriction
$\Mm|_\R$, using, say, Hamming's distance.

\begin{quote}
{\em Response~~~}
Sorry, we do not feel the suggested reformulation would improve the exposition.
And a Hamming's distance does not work for us, we use the L2 distance of
Figure~7.
\end{quote}

  \item
{[$\cdots$]} the authors should attempt to convert the main results into
theorems. If gaps of rigour obstruct this process, these should be
spelled out clearly.

\begin{quote}
{\em Response~~~}
We have now combined the single cat statements around eqs.~(27) to~(31)
into Theorem~3.1, and the spatiotemporal cat statements around eqs.~(42)
to~(45) into Theorem~4.1.

In our calculation of measures  $\mu(M_\R)$ everything is in principle
rigorous.
We have now attempted to define  more clearly what is our measure $\mu$
and what is  $\mu(M_\R)$. But we agree with the referee that in the $d=2$
case we fall short of presenting this in a mathematically
clean way.
\end{quote}

Moreover, given the advertised effectiveness of the computation of exact
probabilities, I find it disappointing that the numerical experiments did
not result in the formulation of conjectures.

\begin{quote}
{\em Response~~~}
we now say explicitly that on the basis of our numerics, the natural
measure $\Msr$ is conjectured to be ergodic.
\end{quote}

[$\cdots$]
if and where the authors have pushed their computations to the limit, and
in this respect some estimate of the complexity of the algorithms would
also be welcome.
\begin{quote}
{\em Response~~~}
sorry, we have no reliable estimate of the complexity.
\end{quote}

\end{enumerate}

\section*{DETAILS}

[$\cdots$] this paper gives the impression to cover the $d$-dimensional
case [$\cdots$].
\begin{quote}
{\em Response~~~}
$d=2$ suffices for the goals of this paper, so the general $d$ is now
mentioned only as a conjecture, the first paragraph of sect.~6.1. {\em
Discussion and future directions}.
% It's as much a `conjecture' as
% Helmholtz equation in $d$ dimensions is a `conjecture', but solving it we
% might run into further surprises, so OK.

We, however, need to make referee aware that
we have redefined the stretching factor $s$ from the initial
Nonlinearity submission  $(\Box + s - 2d) x_z = m_z$  to
\beq
(\Box + d(s - 2)) x_z = m_z
\ee{kittensEq}
Why?

We do not believe it would be wise to get into the larger picture in this
paper (whose focus is narrow - evidence for the conjectured \catlatt\
ergodicity), so the remarks that follow are just for the referee:

\catLatt's continuum relatives are
the inhomogeneous \emph{Helmoltz equation}\rf{DiHaHu01,Lick89,FetWal03}
\beq
   (\Box+k^2)\,\field(x)= -4\pi\rho(x)\,,\qquad x\in \reals^d
\,,
\label{CatMapContinuesPC}
\eeq
and, for
the ${\mu}^2=-k^2>0$ (imaginary $k$) case, the equation
\beq
   (\Box-{\mu}^2)\,\field(x)= -\rho(x)\,,\qquad x\in \reals^d
\,,
\label{sPe}
\eeq
known as  the
{\sPe}, Klein–Gordon or Yukawa equation\rf{Dorr70,GoVanLo96,HuCon96,HuRyCo98}.
Sitting between the two is
 the Poisson equation, the $k \to 0$ limit of the Helmholtz equation;
\beq
   \Box\,\field(x)= -4\pi\rho(x)\,,\qquad x\in \reals^d
\,.
\ee{PoissonEq}
For $\rho=0$, the equation is known as \emph{Laplace's equation}.

The $d$\dmn, purely hyperbolic ${\mu}^2>0$
{\catlatt} \refeq{kittensEq}
\beq
 (\Box - {\mu}^2\unit)_{zz'} \field_{z'} = \m_z
    \,, \qquad
  \field_{z} \in  \mathbb{T}^{1}
    \,, \quad
  m_{z} \in \A^{1}
    \,, \quad
  z\in \integers^{d}
\,,
\ee{GreenLinearConnPC}
that we study is a discretization of the inhomogeneous {\sPe}
\refeq{sPe},
where the Yukawa massive field mass parameter is related to the \catlatt\
\refeq{kittensEq} stretching parameter ${s}$ by
\beq
{\mu}^2=d(s-2)
\,.
\ee{catlattMass}
In this parametrization it is clear
that \catlatt\ is hyperbolic for $s>2$ in any dimension $d$.

For $|s|<2$ the limit is the Helmholtz equation
\refeq{CatMapContinuesPC}, with oscillatory eigenmodes.
The difference between the three cases is illustrated by
the $d=3$ dimensions' outgoing Green's functions for $|s|<2$,
$s=2$, and $s>2$, respectively:
\bea
g_+({x},x') &=& -\frac{e^{ik\vert{x}- x'\vert}}{4\pi\vert{x}- x'\vert}
\continue
g({x},x') &=&-\frac{1}{4\pi\vert{x}- x'\vert}
\continue
g({x},x') &=& -\frac{e^{-{\mu}\vert{x}- x'\vert}}{4\pi\vert{x}- x'\vert}
\,.
\label{GreenFunContinuesPC1}
\eea
\end{quote}

\begin{description}
  \item[p. 2, l.9~~]
The meaning of the symbols in the displayed formula is clarified only in the
following section. Symbols must be explained before, or immediately after, their first
appearance.
\edit{\qquad we have now removed the equation}
  \item[p. 3, l.16-17~~]
Avoid using juxtaposed subscripts, e.g., $\ssp_{nt}$ in place of
$\ssp_{n,t}$. They create notational schizophrenia, as in equations (3)
and (6).
\edit{\qquad implemented}
  \item[p. 3, formula (4)~~]
Explain here the rationale for the choice of the alphabet. At the moment
one has to wait until page 8.
\edit{\qquad now explained here, repeating the explanation from
p. 3, l.10 }
  \item[p. 3, last paragraph]
The second sentence is badly phrased. The fact that given \Xx, the block
\Mm\ is unique and admissible is a property of every symbolic dynamics.
You could be referring to the injectivity brought about by linearity. I
found the converse statement misleading, since not all \Mm s are admissible.
Please clarify.
\edit{\qquad now removed $\Mm\ \to \Xx$ part of the statement}
  \item[p. 4, l.9,12,33,49; p. 12, l.7]
Explain the meaning of the term `generic' in the present context. Is the
measure $\mu$ appearing at line 48 connected to this?
\edit{\qquad
The referee is right to question our use of `generic', it is the central
problem that motivates the numerical investigations of this paper.

The initial conditions for numerical simulations are chosen to be generic
with respect to the natural (Lebesgue) measure $\mu$ - this is what we
mean by a ``generic solution". We say that in the paper now.

The argument that frequencies of block appearances are given by the
measures of the corresponding cylinder sets is based on the ergodic
theorem,  but we do not prove  that  our natural (Lebesgue) measure is
uniquely ergodic. We only show that the system is fully hyperbolic. Our
claim on p.4 ``The $d = 2$ spatiotemporal cat is [$\cdots$]
\emph{ergodic} for $s > 2$'' is a conjecture. This we have not proven,
however we do have numerical support for the claim. The main  value of
our numerics is exactly this - we confirm that the time averages
(frequencies) converge to our analytically calculated measures of the sets.

We now explicitly state that the ergodicity of \catlatt\ is a conjecture.
     }
  \item[p. 5, l.26]
`d\R\ depends only on the shape of $\R$' The term `shape' is vague, and it
shouldn't be difficult to make it precise using, say, an embedding
rectangle. Is $|\R|$ fixed here?
\edit{\qquad got rid of the `shape' of $\R$ everywhere. Almost everywhere.}
  \item[p. 5, l.32] `small $\R$' should be `small $|\R|$.
\edit{\qquad implemented}
  \item[p. 6, l.25]
`length of the block'. A block does not have a length, unless you are in one
dimension, in which case this must be stated.
\edit{\qquad now stated that it is a \edit{1\dmn} \brick}
  \item[p. 9, table 1]
  The usefulness of displaying approximate numerical values is unclear to me,
and even to the authors themselves (p. 12).
\edit{\qquad as to the power of numerical results;
             once the junior collaborators saw that one could compute
             analytically a result of a long simulation, they were sold
             on the project. While p.~12 does display much bravado, this
             table has been a source of comfort to all 5 of us, as, as
             confessed above, we have not \emph{proven} ergodicity.
   }
  \item[p. 10, caption of fig.2]
The labels (a) to (i) do not appear in the figure. Also, the information
displayed on page 10 should not require an entire page.
\edit{\qquad
We are aware of the missing labels, and will try to insert them before
the paper goes into a publication stage. In our experience, figure~2
helps our audiences understand why measures associated with symbol blocks
differ and what they are, and is a gateway drug to cat map polygons (figure~3)
and spatiotemporal cat polytopes  (figure~4).
    }
  \item[p. 12, l.14]
The `unwieldy and unilluminating' computations may nonetheless be telling
an interesting story. Is there a more informative description of the
authors' findings?
\edit{\qquad We wish. We have been working on making it ``more informative"
    since spring of 2016, but the 2\dmn\ lattice problem turns out to be
    harder than what we have expected.}
  \item[p. 16, table 4]
Pity that you don't attempt any arithmetical justification. Is this a pure
exponential? Can you formulate a conjecture?
\edit{\qquad
Computing Table 2 was a Mathematica programming tour de force, but
ultimately we had to accept that $\tilde{N}_n$, the number of \emph{new}
pruned {\brick s} of length  $\cl{b}$, is a number theory puzzle that
we were not able to elucidate.
    }
  \item[p. 18, l.18]
Explain precisely what `simply connected' means for a subset of $\integers^{2}$.
\edit{\qquad
Changed to `rectangle' now, thanks for pointing this out.
    }
  \item[p. 37, l.17]
The verb `verify' suggests that the vanishing of the geometrical part of
the entropy is heuristic. If this is not the case, then replace it with
`illustrate'. In any case, fig C1 is rather uninformative; one would like
some information of the rate of convergence.
\edit{\qquad
We know that it must vanish (so changed to `illustrate' in the text).
Our numerics suggests a rate of
convergence, but we would do not feel confident enough in the results we have
to formulate a conjecture.
    }
  \item[p. 37, last formula] How fast is this limit approached?
\edit{\qquad
We do not know the rate of convergence here.
     }
\end{description}



\section*{GRAMMAR/TYPOS}

\begin{description}
  \item[p. 11, l.41-42]
Replace `$(x_0,p_0)$ phase space' with `phase space coordinates $(x_0,p_0)$'
\edit{\qquad implemented}
  \item[p. 17, l.32] [p. 8, eq (17)] If you write $\underline{1}$ to mean $-1$,
then $\underline{|m_z|}$ means $-\underline{|m_z|}$, not $-m_z$
as stated.

[For consistency, you should also write $\underline{\Box} + s\underline{4}$,
and 1\underline{d}imensional.]
\edit{\qquad
Ah-ah,  oh-oh. Symbols in symbolic dynamics are \underline{symbols}. We
can denote them \c{c}, $\check{s}, \check{z}, \cdots$ if it please the
reader, and indeed, in Figure~1. we represent them by colors. To impugn
that symbols {\color{red}red} or $\underline{1}$ imply that the
\emph{integer }$-1$ is to be denoted $\underline{1}$ is ... is ... we are
speechless.

But -a silver lining-  $\underline{|m_z|}$ is gone.
}

  \item[p. 18, l.50]
If `1 or `2 is even, the barycentre z is not a lattice point; either
round it to a lattice point, or use a different symbol.
\edit{\qquad you are right, now we have changed `even' to `odd'.}

  \item[p. 31, l.22] Shouldn't $g$ be $\gd$?
Also, `Green's function $\gd$' should be `the Green's function
$\gd$', but on p.34, l.9, `the Green's theorem' should be `Green's theorem'.
\edit{\qquad implemented}
\end{description}



% \newpage
\printbibliography[
heading=bibintoc,
title={References}
				  ] %, type=online]  % if not using default "Bibliography"

\end{document}
