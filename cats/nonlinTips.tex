% siminos/cats/nonlinTips.tex
% $Author: predrag $ $Date: 2019-10-31 17:06:14 -0400 (Thu, 31 Oct 2019) $

\section{Nonlinearity journal tips}
\label{sect:NonlinTips}

\HREF{https://mc04.manuscriptcentral.com/non} {Author help}

\subsection{Naming your files}

Please name all your files, both figures and text, as follows:
\begin{itemize}
\item Use only characters from the set a to z, A to Z, 0 to 9 and underscore (\_).
\item Do not use spaces or punctuation characters in file names.
\item Do not use any accented characters such as
\'a, \^e, \~n, \"o.
\item Include an extension to indicate the file type (e.g., \verb".tex", \verb".eps", \verb".txt", etc).
\item Use consistent upper and lower case in filenames and in your \LaTeX\ file.
If your \LaTeX\ file contains the line \verb"\includegraphics{fig1.eps}" the figure file must be called
\verb"fig1.eps" and not \verb"Fig1.eps" or \verb"fig1.EPS".  If you are on a Unix system, please ensure that
there are no pairs of figures whose names differ only in capitalization, such as \verb"fig_2a.eps" and \verb"fig_2A.eps",
as Windows systems will be unable to keep the two files in the same directory.
\end{itemize}

\subsection{Preparing your article}

%    \PC{2016-08-18}{
Footnotes should be avoided whenever possible and can often be included
in the text as phrases or sentences in parentheses. If required, they
should be used only for brief notes that do not fit conveniently into the
text. The use of displayed mathematics in footnotes should be avoided
wherever possible and no equations within a footnote should be numbered.
The standard \LaTeX\ macro \verb"\footnote" should be used.  Note that in
\verb"iopart.cls" the \verb"\footnote" command produces footnotes indexed
by a variety of different symbols, whereas in published articles we use
numbered footnotes.  This is not a problem: we will convert
symbol-indexed footnotes to numbered ones during the production process.
%    }

\subsection{The abstract}
The abstract should be self-contained---there should be no references to
figures, tables, equations, bibliographic references etc.

\subsection{Some matters of style}
It will help the readers if your article is written in a clear,
consistent and concise manner. During the production process
we will try to make sure that your work is presented to its
readers in the best possible way without sacrificing the individuality of
your writing.

\begin{enumerate}
\item We recommend using `-ize' spellings (diagonalize,
renormalization, minimization, etc) but there are some common
exceptions to this, for example: devise,
promise and advise.

Do not include `eq.', `equation' etc before an equation number or `ref.'\, `reference' etc before a reference number.
\end{enumerate}

\subsection{Two-line constructions}
For simple fractions in the text the solidus \verb"/", as in
$\lambda/2\pi$, should be used instead of \verb"\frac" or \verb"\over",
using parentheses where necessary to avoid ambiguity,
for example to distinguish between $1/(n-1)$ and $1/n-1$. Exceptions to
this are the proper fractions $\frac12$, $\frac13$, $\frac34$,
etc, which are better left in this form. In displayed equations
horizontal lines are preferable to solidi provided the equation is
kept within a height of two lines. A two-line solidus should be
avoided where possible; the construction $(\ldots)^{-1}$ should be
used instead. For example use:
\begin{equation*}
\frac{1}{M_{\rm a}}\left(\int^\infty_0{\rm d}
\omega\;\frac{|S_o|^2}{N}\right)^{-1}\qquad\mbox{instead of}\qquad
\frac{1}{M_{\rm a}}\biggl/\int^\infty_0{\rm d}
\omega\;\frac{|S_o|^2}{N}.
\end{equation*}

\subsection{Roman and italic in mathematics}
In mathematics mode there are some cases where it is preferable to use a
Roman font; for instance, a Roman d for a differential d, a Roman e for
an exponential e and a Roman i for the square root of $-1$. To
accommodate this and to simplify the  typing of equations,
\verb"iopart.cls" provides some extra definitions. \verb"\rmd",
\verb"\rme" and \verb"\rmi" now give Roman d, e and i respectively for
use in equations, e.g.\ $\rmi x\rme^{2x}\rmd x/\rmd y$ is obtained by
typing \verb"$\rmi x\rme^{2x}\rmd x/\rmd y$".

Certain other common mathematical functions, such as cos, sin, det and
ker, should appear in Roman type. Standard \LaTeX\ provides macros for
most of these functions
(in the cases above, \verb"\cos", \verb"\sin", \verb"\det" and \verb"\ker"
respectively); \verb"iopart.cls" also provides
additional definitions for $\Tr$, $\tr$ and
$\Or$ (\verb"\Tr", \verb"\tr" and \verb"\Or", respectively).

Subscripts and superscripts should be in Roman type if they are labels
rather than variables or characters that take values. For example in the
equation
\[
\epsilon_m=-g\mu_{\rm n}Bm
\]
$m$, the $z$ component of the nuclear spin, is italic because it can have
different values whereas n is Roman because it
is a label meaning nuclear ($\mu_{\rm n}$
is the nuclear magneton).

\subsection{Special characters for mathematics}
Bold italic characters can be used in our journals to signify vectors
(rather than using an upright bold or an over arrow). To obtain this
effect when using \verb"iopart.cls", use \verb"\bi{#1}" within maths
mode, e.g. $\bi{ABCdef}$. Similarly, in \verb"iopart.cls", if upright
bold characters are required in maths, use \verb"\mathbf{#1}" within
maths mode, e.g. $\mathbf{XYZabc}$. The calligraphic (script) uppercase
alphabet is obtained with \verb"\mathcal{AB}" or \verb"\cal{CD}"
($\mathcal{AB}\cal{CD}$).


The package \verb"iopams.sty" uses the definition \verb"\boldsymbol" in \verb"amsbsy.sty"
which allows individual non-alphabetical symbols and Greek letters to be
made bold within equations.
The bold Greek lowercase letters \ifiopams$\balpha \ldots\bomega$,\fi
are obtained with the commands
\verb"\balpha" \dots\ \verb"\bomega" (but note that
bold eta\ifiopams, $\bfeta$,\fi\ is \verb"\bfeta" rather than \verb"\beta")
and the capitals\ifiopams, $\bGamma\ldots\bOmega$,\fi\ with commands
\verb"\bGamma" \dots\
\verb"\bOmega". Bold versions of the following symbols are
predefined in \verb"iopams.sty":
bold partial\ifiopams, $\bpartial$,\fi\ \verb"\bpartial",
bold `ell'\ifiopams, $\bell$,\fi\  \verb"\bell",
bold imath\ifiopams, $\bimath$,\fi\  \verb"\bimath",
bold jmath\ifiopams, $\bjmath$,\fi\  \verb"\bjmath",
bold infinity\ifiopams, $\binfty$,\fi\ \verb"\binfty",
bold nabla\ifiopams, $\bnabla$,\fi\ \verb"\bnabla",
bold centred dot\ifiopams, $\bdot$,\fi\  \verb"\bdot". Other
characters are made bold using
\verb"\boldsymbol{\symbolname}".

\begin{table}
\caption{\label{math-tab2}Other macros defined in {\tt iopart.cls} for use in maths.}
\begin{tabular*}{\textwidth}{@{}l*{15}{@{\extracolsep{0pt plus
12pt}}l}}
\br
Macro&Result&Description\\
\mr
\verb"\fl"&&Start line of equation full left\\
\verb"\case{#1}{#2}"&$\case{\#1}{\#2}$&Text style fraction in display\\
\verb"\Tr"&$\Tr$&Roman Tr (Trace)\\
\verb"\tr"&$\tr$&Roman tr (trace)\\
\verb"\Or"&$\Or$&Roman O (of order of)\\
\verb"\lshad"&$\lshad$&Text size left shadow bracket\\
\verb"\rshad"&$\rshad$&Text size right shadow bracket\\
\br
\end{tabular*}
\end{table}

\subsection{Alignment of displayed equations}

The \verb"iopart.cls" class file aligns left and indents each line of a
display.  To make any line start at the left margin of the page, add
\verb"\fl" at start of the line (to indicate full left).

Using the \verb"eqnarray" environment equations will naturally be aligned
left and indented without the use of any ampersands for alignment, see
equations (\ref{eq1}) and (\ref{eq2})
\begin{eqnarray}
\alpha + \beta =\gamma^2, \label{eq1}\\
\alpha^2 + 2\gamma + \cos\theta = \delta. \label{eq2}
\end{eqnarray}

Where some secondary alignment is needed, for instance a second part of
an equation on a second line, a single ampersand is added at the point of
alignment in each line  (see  (\ref{eq3}) and (\ref{eq4})).
\begin{eqnarray}
\alpha &=2\gamma^2 + \cos\theta + \frac{XY \sin\theta}{X+ Y\cos\theta} \label{eq3}\\
 & = \delta\theta PQ \cos\gamma. \label{eq4}
\end{eqnarray}

Two points of alignment are possible using two ampersands for alignment
(see  (\ref{eq5}) and (\ref{eq6})).  Note in this case extra space
\verb"\qquad" is added before the second ampersand in the longest line
(the top one) to separate the condition from the equation.
\begin{eqnarray}
\alpha &=2\gamma^2 + \cos\theta + \frac{XY \sin\theta}{X+ Y\cos\theta}\qquad& \theta > 1 \label{eq5}\\
 & = \delta\theta PQ \cos\gamma &\theta \leq 1.\label{eq6}
\end{eqnarray}

For a long equation which has to be split over more than one line the
first line should start at the left margin, this is achieved by inserting
\verb"\fl" (full left) at the start of the line. The use of the alignment
parameter \verb"&" is not necessary unless some secondary alignment is
needed.
\begin{eqnarray}
\fl \alpha + 2\gamma^2 = \cos\theta + \frac{XY \sin\theta}{X+ Y\cos\theta} +  \frac{XY \sin\theta}{X- Y\cos\theta} +
+ \left(\frac{XY \sin\theta}{X+ Y\cos\theta}\right)^2 \nonumber\\
+  \left(\frac{XY \sin\theta}{X- Y\cos\theta}\right)^2.\label{eq7}
\end{eqnarray}

The plain \TeX\ command \verb"\eqalign" can be used within an
\verb"equation" environment to obtain a multiline equation with a single
centred number, for example
\begin{equation}
\eqalign{\alpha + \beta =\gamma^2 \cr
\alpha^2 + 2\gamma + \cos\theta = \delta.}
\end{equation}

\subsection{Miscellaneous}

Exponential expressions, especially those containing subscripts or
superscripts, are clearer if the notation $\exp(\ldots)$ is used, except for
simple examples. For instance $\exp[\rmi(kx-\omega t)]$ and $\exp(z^2)$ are
preferred to $\e^{\rmi(kx-\omega t)}$ and $\e^{z^2}$, but
$\e^x$
is acceptable.

The square root sign $\sqrt{\phantom{b}}$ should
only be used with relatively
simple expressions, e.g.\ $\sqrt2$ and $\sqrt{a^2+b^2}$;
in other cases the
power $1/2$ should be used; for example, $[(x^2+y^2)/xy(x-y)]^{1/2}$.

It is important to distinguish between $\ln = \log_\e$ and $\lg
=\log_{10}$. Braces, brackets and parentheses should be used in the
following order: $\{[(\;)]\}$.

Decimal fractions should always be preceded by a zero: for example 0.123
{\bf not} .123. For long numbers use thin spaces after every third
character away from the position of the decimal point, unless this leaves
a single separated character: e.g.\ $60\,000$, $0.123\,456\,78$ but 4321
and 0.7325.

Equations should be followed by a full stop (periods) when at the end
of a sentence.

\subsection{Equation numbering and layout in {\tt iopart.cls}}
\label{eqnum}

If the command \verb"\eqnobysec" is included in the preamble, equation
numbering by section is obtained, e.g.\ (2.1), (2.2), etc.
Refer to equations in the text using the
equation number in parentheses. It is not normally necessary to include
the word equation before the number; and abbreviations such as eqn or eq
should not be used. In \verb"iopart.cls", there are alternatives to the
standard \verb"\ref" command that you might find useful---see
\tref{abrefs}.

Sometimes it is useful to number equations as parts of the same
basic equation. This can be accomplished in \verb"iopart.cls" by inserting the
commands \verb"\numparts" before the equations concerned and
\verb"\endnumparts" when reverting to the normal sequential numbering.
For example using \verb"\numparts \begin{eqnarray}" ... \verb"\end{eqnarray} \endnumparts":

\numparts
\begin{eqnarray}
T_{11}&=(1+P_\e)I_{\uparrow\uparrow}-(1-P_\e)
I_{\uparrow\downarrow},\label{second}\\
T_{-1-1}&=(1+P_\e)I_{\downarrow\downarrow}-(1-P_\e)I_{\uparrow\downarrow},\\
S_{11}&=(3+P_\e)I_{\downarrow\uparrow}-(3-P_e)I_{\uparrow\uparrow},\\
S_{-1-1}&=(3+P_\e)I_{\uparrow\downarrow}-(3-P_\e)
I_{\downarrow\downarrow}.
\end{eqnarray}
\endnumparts

Equation labels within the \verb"\eqnarray" environment will be referenced
as subequations, e.g. (\ref{second}).

\subsection{Miscellaneous extra commands for displayed equations}
The \verb"\cases" command has been amended slightly in \verb"iopart.cls" to
increase the space between the equation and the condition.
\Eref{cases}
demonstrates simply the output from the \verb"\cases" command
\begin{equation}
\label{cases}
X=\cases{1&for $x \ge 0$\\
-1&for $x<0$\\}
\end{equation}
%The code used was:
%\small\begin{verbatim}
%\begin{equation}
%\label{cases}
%X=\cases{1&for $x \ge 0$\\
%-1&for $x<0$\\}
%\end{equation}
%\end{verbatim}
%\normalsize

To obtain text style fractions within displayed maths the command
\verb"\case{#1}{#2}" can be used instead
of the usual \verb"\frac{#1}{#2}" command or \verb"{#1 \over #2}".

When two or more short equations are on the same line they should be
separated by a `qquad space' (\verb"\qquad"), rather than
\verb"\quad" or any combination of \verb"\,", \verb"\>", \verb"\;"
and \verb"\ ".

\subsection{Preprint references}
Preprints may be referenced but if the article concerned has been
published in a peer-reviewed journal, that reference should take
precedence. If only a preprint reference can be given, it is helpful to
include the article title. Examples are:
\vskip6pt
\numrefs{1}
\item Neilson D and Choptuik M 2000 {\it Class. Quantum Grav.} {\bf 17}
        761 (arXiv:gr-qc/9812053)
\item Sundu H, Azizi K, S\"ung\"u J Y and Yinelek N 2013
        Properties of $D_{s2}^*(2573)$ charmed-strange tensor meson
        arXiv:1307.6058
\endnumrefs


\subsection{Cross-referencing\label{xrefs}}

\verb"label" may contain letters, numbers
or punctuation characters but must not contain spaces or commas. It is also
recommended that the underscore character \_{} is not used in cross
referencing.

Thus labels of the form \verb"eq:partial", \verb"fig:run1",
\verb"eq:dy'", etc, may be used. When several references occur together
in the text \verb"\cite" may be used with multiple labels with commas but
no spaces separating them; the output will be the numbers within a single
pair of square brackets with a comma and a thin space separating the
numbers. Thus \verb"\cite{label1,label2,label4}" would give [1,\,2,\,4].
Note that no attempt is made by the style file to sort the labels and no
shortening of groups of consecutive numbers is done. Authors should
therefore either try to use multiple labels in the correct order, or use
a package such as \verb"cite.sty" that reorders labels correctly.


\Table{\label{abrefs}
Alternatives to the normal references command {\tt $\backslash$ref}
available in {\tt iopart.cls}, and the text generated by them. Note it is
not normally necessary to include the word equation before an equation
number except where the number starts a sentence. The versions producing
an initial capital should only be used at the start of sentences.}
\br
Reference&Text produced\\
\mr
\verb"\eref{<label>}"&(\verb"<num>")\\
\verb"\Eref{<label>}"&Equation (\verb"<num>")\\
\verb"\fref{<label>}"&figure \verb"<num>"\\
\verb"\Fref{<label>}"&Figure \verb"<num>"\\
\verb"\sref{<label>}"&section \verb"<num>"\\
\verb"\Sref{<label>}"&Section \verb"<num>"\\
\verb"\tref{<label>}"&table \verb"<num>"\\
\verb"\Tref{<label>}"&Table \verb"<num>"\\
\br
\endTable

\subsection{Tables and table captions}
Tables are numbered serially and referred to in the text
by number (table 1, etc, {\bf not} tab. 1). Each table should have an
explanatory caption which should be as concise as possible. If a table
is divided into parts these should be labelled \pt(a), \pt(b),
\pt(c), etc but there should be only one caption for the whole
table, not separate ones for each part.

The standard form for a table in \verb"iopart.cls" is:
\small\begin{verbatim}
\begin{table}
\caption{\label{label}Table caption.}
\begin{indented}
\item[]\begin{tabular}{@{}llll}
\br
Head 1&Head 2&Head 3&Head 4\\
\mr
1.1&1.2&1.3&1.4\\
2.1&2.2&2.3&2.4\\
\br
\end{tabular}
\end{indented}
\end{table}
\end{verbatim}\normalsize

\begin{enumerate}
\item The caption comes before the table. It should have a period at
the end.

\item Tables are normally set in a smaller type than the text.
The normal style is for tables to be indented. This is accomplished
by using \verb"\begin{indented}" \dots\ \verb"\end{indented}"
and putting \verb"\item[]" before the start of the tabular environment.
Omit these
commands for any tables which will not fit on the page when indented.

\item The default is for columns to be aligned left and
adding \verb"@{}" omits the extra space before the first column.

\item Tables have only horizontal rules and no vertical ones. The rules at
the top and bottom are thicker than internal rules and are set with
\verb"\br" (bold rule).
The rule separating the headings from the entries is set with
\verb"\mr" (medium rule).  These are special \verb"iopart.cls" commands.

\item Numbers in columns should be aligned on the decimal point;
to help do this a control sequence \verb"\lineup" has been defined
in \verb"iopart.cls"
which sets \verb"\0" equal to a space the size of a digit, \verb"\m"
to be a space the width of a minus sign, and \verb"\-" to be a left
overlapping minus sign. \verb"\-" is for use in text mode while the other
two commands may be used in maths or text.
(\verb"\lineup" should only be used within a table
environment after the caption so that \verb"\-" has its normal meaning
elsewhere.) See table~\ref{tabone} for an example of a table where
\verb"\lineup" has been used.
\end{enumerate}

\begin{table}
\caption{\label{tabone}A simple example produced using the standard table commands
and $\backslash${\tt lineup} to assist in aligning columns on the
decimal point. The width of the
table and rules is set automatically by the
preamble.}

\begin{indented}
\lineup
\item[]\begin{tabular}{@{}*{7}{l}}
\br
$\0\0A$&$B$&$C$&\m$D$&\m$E$&$F$&$\0G$\cr
\mr
\0\023.5&60  &0.53&$-20.2$&$-0.22$ &\01.7&\014.5\cr
\0\039.7&\-60&0.74&$-51.9$&$-0.208$&47.2 &146\cr
\0123.7 &\00 &0.75&$-57.2$&\m---   &---  &---\cr
3241.56 &60  &0.60&$-48.1$&$-0.29$ &41   &\015\cr
\br
\end{tabular}
\end{indented}
\end{table}

\subsection{Simplified coding and extra features for tables}
The basic coding format can be simplified using extra commands provided in
the \verb"iopart" class file. The commands up to and including
the start of the tabular environment
can be replaced by
\small\begin{verbatim}
\Table{\label{label}Table caption}
\end{verbatim}\normalsize
and this also activates the definitions within \verb"\lineup".
The final three lines can also be reduced to \verb"\endTable" or
\verb"\endtab". Similarly for a table which does not fit on the page when indented
\verb"\fulltable{\label{label}caption}" \dots\ \verb"\endfulltable"
can be used. \LaTeX\ optional positional parameters can, if desired, be added after
\verb"\Table{\label{label}caption}" and \verb"\fulltable{\label{label}caption}".


\verb"\centre{#1}{#2}" can be used to centre a heading
\verb"#2" over \verb"#1"
columns and \verb"\crule{#1}" puts a rule across
\verb"#1" columns. A negative
space \verb"\ns" is usually useful to reduce the space between a centred
heading and a centred rule. \verb"\ns" should occur immediately after the
\verb"\\" of the row containing the centred heading (see code for
\tref{tabl3}). A small space can be
inserted between rows of the table
with \verb"\ms" and a half line space with \verb"\bs"
(both must follow a \verb"\\" but should not have a
\verb"\\" following them).

\Table{\label{tabl3}A table with headings spanning two columns and containing notes.
To improve the
visual effect a negative skip ($\backslash${\tt ns})
has been put in between the lines of the
headings. Commands set-up by $\backslash${\tt lineup} are used to aid
alignment in columns. $\backslash${\tt lineup} is defined within
the $\backslash${\tt Table} definition.}
\br
&&&\centre{2}{Separation energies}\\
\ns
&Thickness&&\crule{2}\\
Nucleus&(mg\,cm$^{-2}$)&Composition&$\gamma$, n (MeV)&$\gamma$, 2n (MeV)\\
\mr
$^{181}$Ta&$19.3\0\pm 0.1^{\rm a}$&Natural&7.6&14.2\\
$^{208}$Pb&$\03.8\0\pm 0.8^{\rm b}$&99\%\ enriched&7.4&14.1\\
$^{209}$Bi&$\02.86\pm 0.01^{\rm b}$&Natural&7.5&14.4\\
\br
\end{tabular}
\item[] $^{\rm a}$ Self-supporting.
\item[] $^{\rm b}$ Deposited over Al backing.
\end{indented}
\end{table}

Units should not normally be given within the body of a table but
given in brackets following the column heading; however, they can be
included in the caption for long column headings or complicated units.
Where possible tables should not be broken over pages.
If a table has related notes these should appear directly below the table
rather than at the bottom of the page. Notes can be designated with
footnote symbols (preferable when there are only a few notes) or
superscripted small roman letters. The notes are set to the same width as
the table and in normal tables follow after \verb"\end{tabular}", each
note preceded by \verb"\item[]". For a full width table \verb"\noindent"
should precede the note rather than \verb"\item[]". To simplify the coding
\verb"\tabnotes" can, if desired, replace \verb"\end{tabular}" and
\verb"\endtabnotes" replaces
\verb"\end{indented}\end{table}".

\subsection{Inclusion of graphics files\label{figinc}}

Below each figure should be a brief caption describing it and, if
necessary, interpreting the various lines and symbols on the figure.
As much lettering as possible should be removed from the figure itself
and included in the caption. If a figure has parts, these should be
labelled ($a$), ($b$), ($c$), etc and all parts should be described
within a single caption. \Tref{blobs} gives the definitions for describing
symbols and lines often used within figure captions (more symbols are
available when using the optional packages loading the AMS extension fonts).

\subsection{Supplementary Data}

Supplementary data
enhancements typically consist of video clips, animations or
data files, tables of extra information or extra figures. See
guidelines on supplementary data file formats,
`Author Guidelines' link at \verb"http://authors.iop.org".

Software, in the form of input scripts for mathematical packages (such as
Mathematica notebook files), or source code that can be interpreted or
compiled (such as Python scripts or Fortran or C programs), or executable
files, can sometimes be accepted as supplementary data, but the journal
may ask you for assurances about the software and distribute them from
the article web page only subject to a disclaimer.  Contact the journal
in the first instance if you want to submit software.

\begin{table}[t]
\caption{\label{blobs}Control sequences to describe lines and symbols in figure
captions.}
\begin{indented}
\item[]\begin{tabular}{@{}lllll}
\br
Control sequence&Output&&Control sequence&Output\\
\mr
\verb"\dotted"&\dotted        &&\verb"\opencircle"&\opencircle\\
\verb"\dashed"&\dashed        &&\verb"\opentriangle"&\opentriangle\\
\verb"\broken"&\broken&&\verb"\opentriangledown"&\opentriangledown\\
\verb"\longbroken"&\longbroken&&\verb"\fullsquare"&\fullsquare\\
\verb"\chain"&\chain          &&\verb"\opensquare"&\opensquare\\
\verb"\dashddot"&\dashddot    &&\verb"\fullcircle"&\fullcircle\\
\verb"\full"&\full            &&\verb"\opendiamond"&\opendiamond\\
\br
\end{tabular}
\end{indented}
\end{table}




\begin{table}
\caption{Macros defined within {\tt iopart.cls}
for use with figures and tables.}
\begin{indented}
\item[]\begin{tabular}{@{}l*{15}{l}}
\br
Macro name&Purpose\\
\mr
\verb"\Figures"&Heading for list of figure captions\\
\verb"\Figure{#1}"&Figure caption\\
\verb"\Tables"&Heading for tables and table captions\\
\verb"\Table{#1}"&Table caption\\
\verb"\fulltable{#1}"&Table caption for full width table\\
\verb"\endTable"&End of table created with \verb"\Table"\\
\verb"\endfulltab"&End of table created with \verb"\fulltable"\\
\verb"\endtab"&End of table\\
\verb"\br"&Bold rule for tables\\
\verb"\mr"&Medium rule for tables\\
\verb"\ns"&Small negative space for use in table\\
\verb"\centre{#1}{#2}"&Centre heading over columns\\
\verb"\crule{#1}"&Centre rule over columns\\
\verb"\lineup"&Set macros for alignment in columns\\
\verb"\m"&Space equal to width of minus sign\\
\verb"\-"&Left overhanging minus sign\\
\verb"\0"&Space equal to width of a digit\\
\br
\end{tabular}
\end{indented}
\end{table}
