% siminos/gudorf/thesis/chapter/selectionKS.tex
% $Author: predrag $ $Date: 2020-05-25 15:18:45 -0400 (Mon, 25 May 2020) $

%\subsection{Selection rules for \KS}

\subsection{Selection rules for real-valued Fourier coefficients}
%\label{sect:selectionKS}

%Possible figures
%Pictographic display of selection rules/constraints




Although the \spt\ \KSe\ is easier to write in terms of a complex
Fourier-Fourier basis, the symmetry invariant subspaces generated
by symmetry constraints is easier to describe in terms of \rv\ \Fcs.
The \rv\
{\spt} Fourier expansion can be written

This is the expansion for a general {\spt} solution. For each discrete
symmetry of the \spt\ \KSe\ there is a unique set of constraints or
``selection rules'' for the \spt\ \Fcs. These selection rules constitute
\textit{symmetry invariant subspaces} of solutions of the \spt\ \KSe. In
this section we commit to the description of the selection rules of the
\Fcs. For more discussion on the symmetries themselves we refer the
reader to \refsect{sect:KSsymm}.

The two discrete symmetries we will describe are spatial reflection
symmetry and \spt\ shift-reflection symmetry. The shift-reflection
symmetry is a special case of the broader symmetry group $\Dn{n}\times
\Cn{n}$ ($n=2$). Due to the uncommon appearance of solutions with $n>2$
and the relatively easy generalization of the $n=2$ shift-reflection
case, we shall only consider the $n=2$ symmetry group.

The general procedure for producing these selection rules is not very
complicated. Let $R$ represent an arbitrary symmetry operation. If a
solution is invariant under $R$ then it satisfies the \textit{invariance
condition} $Ru = u$ or equivalently $(R-1)u=0$. Substitution of the
expansion \refeq{e-RealFourier} produces a set of constraints that can
only be satisfied when a subset of \rv\ \Fcs\ are individually equal to
zero. To begin we start with spatial reflection symmetry, as it is almost
trivial. Solutions invariant under spatial reflection only admit
spatially antisymmetric basis functions. Therefore, the selection rules
for spatial reflection symmetry are
\beq \label{e-ReflRules}
\akj,\bkj = 0 \mbox{  for all  } k,j
\,.
\eeq
\Spt\ shift-reflection is the composition of two symmetry operations:
spatial reflection $\Refl_x$ and time translation by $\tau_{\period{}/2}$.
The action of this symmetry is as follows $\Refl_x \tau_{\period{}/2} u(\conf, \zeit) = -u(-\conf, \zeit + \period{}/2)$.
When directly applied to the \rv\ Fourier expansion \refeq{e-RealFourier}
and by virtue of trigonometric identities and the parity of $\sin$ and $\cos$ we have
\beq
\begin{split}
\Refl_x \tau_{\period{}/2}\,u(\xm, \tn) &= -\sum_{k,j}
                                \cos(\wavek (-\xm)) (\akj\cos(\freqj (\tn + \period{}/2)) + \bkj \sin(\freqj (\tn + \period{}/2))) \continue
                                &+ \sin(\wavek(-\xm)) (\ckj\cos(\freqj (\tn + \period{}/2)) + \dkj\sin(\freqj (\tn + \period{}/2))) \continue
                            &= \sum_{k,j} -\cos(\wavek \xm) (\akj \cos(\freqj \tn)\cos(\pi j) + \bkj \sin(\freqj \tn)\cos(\pi j)))\continue
                                &+ \sin(\wavek \xm) (\ckj\cos(\freqj \tn)\cos(\pi j) + \dkj \sin(\freqj \tn)\cos(\pi j)) \continue
                            &= \sum_{k,j} (-1)^{j+1} \cos(\wavek \xm) (\akj\cos(\freqj \tn) + \bkj\sin(\freqj \tn))\continue
                                &+ (-1)^{j} \sin(\wavek \xm) (\ckj \cos(\freqj \tn) + \dkj \sin(\freqj \tn)) \, ,
\end{split}
\ee{e-ShiftReflectBasis}
By combining this with the invariance condition
$(\Refl_x \tau_{\period{}/2} - 1)u = 0$,
we find that the selection rules for \spt\ shift-reflection are as follows
\bea \label{e-ShiftReflRules}
\akj,\bkj &=& 0 \, \mbox{ for }\, j \, \mbox{ even } \continue
\ckj,\dkj &=& 0 \, \mbox{ for }\, j \, \mbox{ odd }
\,.
\eea
