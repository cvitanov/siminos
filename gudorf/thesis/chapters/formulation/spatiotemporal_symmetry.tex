The symmetries of the \KSe\ in the \spt\ formulation include Galilean invariance and the
symmetry group $G \equiv \SOn{2}\times On{2}$. This \spt\ symmetry group provides a new perspective with which
to view orbits and their symmetries. This in turn provides a new tool which greatly benefits the computations to follow.
These benefits rely on an important distinction to be made; namely, a distinction must be made between \textit{equivariance}
and \textit{invariance} of orbits with respect to group operations.

For a subgroup $H \subset G$, an orbit is $H$-equivariant, if every element of the group orbit produced by $H$
\beq
\Manif_{} = {h \cdot \orbit | h \in H}
\eeq
are themselves orbits, that is, they are all solutions to \refeq{e-ks}. Invariance is used to describe
the isotropy subgroup (alternatively \textit{stabilizer group} or \textit{little group})
with respect to an orbit; that is if $K \subset H \subset G$ is the isotropy subgroup of \utensor\ if
\beq
k \cdot \utensor = \utensor, \quad \forall k \in K\,.;s
\ee{e-invariance}
Note that in this context, this is a very precise statement that the velocity field \dufield is \textit{exactly}
the same as it was prior to the symmetry operation.

For clarity, the statements: `an orbit has a discrete symmetry' and
`an orbit with discrete symmetry' is always a statement of invariance, not equivariance. First the group
orbits and isotropy groups considered here are defined. In order to


The group orbit for orbits without discrete symmetries are $G$-equivariant, their isotropy subgroups
are the trivial subgroup. In the discrete setting, the group orbit \dufield\ is not $G$. Because the field
is defined only at the collocation points, arbitrary rotations are essentially interpolations of the field
because they return the field at points in between the collocation points, which do not satisfy \refeq{e-ks}
numerically. Therefore, the group orbit of the discretized field \dufield is defined by
spatial reflection and discrete \spt\ rotations, i.e.
\beq
H_{\orbit} = \Cn{N} \times \Dn{M}
\ee{e-grouporbit}
$\Cn{N}$ is the cyclic group of order $N$ that accounts for discrete time rotations. $\Dn{M}$ is the dihedral group
generated by discrete spatial rotations and reflection about $\frac{\speriod{}}{2}$. The isotropy group in this case
is the trivial subgroup of $G$.

For brevity of the discussion that follows, let $\sigma$ represent spatial reflection about $x = \frac{\speriod{}}{2}$
such that
\beq
\sigma u(\tn, \xm) = -u(\tn, \speriod{} - \xm)
\ee{e-reflect}
and let $\tau$ represent half-cell ($\period{}/2$) translations in time
\beq
\tau u(\tn, \xm) = u(\tn + \frac{\period{}}{2}, \xm) \,.
\ee{e-tshift}
Finally, the \spt\ shift-reflection is defined by the composition of these two operations $s = \sigma \tau$ such that
\beq
\sigma\tau u(\tn, \xm) = -u(\tn + \frac{\period{}}{2}, \speriod{} - \xm) \,.
\ee{e-shiftreflect}
The effect of these transformations in the mode basis is determined by the trigonometric basis functions, written
in tensor notation
\begin{align*}
\sigma \utensor &=
\begin{bmatrix}
-\ajk & \cjk\\
-\bjk & \djk
\end{bmatrix}
&
\tau \utensor &=
\begin{bmatrix}
(-1)^{j}\ajk & (-1)^{j}\cjk\\
(-1)^{j}\bjk & (-1)^{j}\djk
\end{bmatrix}
&
\sigma\tau\utensor &=
\begin{bmatrix}
(-1)^{j+1}\ajk & (-1)^{j}\cjk\\
(-1)^{j+1}\bjk & (-1)^{j}\djk
\end{bmatrix}
& \numberthis %\label{e-modesymmetries}
\end{align*}

Hereafter, orbits invariant under spatial reflection and \spt\ shift-reflection
will be referred to as antisymmetric orbits and shift-reflection orbits, respectively.

The group orbit for both antisymmetric
and shift-reflection symmetric orbits is generated by half-cell shifts and reflections in space, \Dn{2}, and discrete time rotations, \Cn{N}.
\beq
H_{\orbit} = \Cn{N} \times \Dn{2} ,.
\ee{e-discgrouporbit}
For antisymmetric orbits,
the isotropy subgroup consists of only spatial reflections,
\beq
H_a = \{e, \sigma\}
\ee{e-anti_isotropy}
and similarly for shift-reflection, represented as $\sigma\tau$,
\beq
H_{\sigma\tau} = \{e, \sigma\tau\} \,.
\ee{e-sr_isotropy}

There are a number of important similarities between antisymmetric and shift-reflection orbits.
The group orbits for both are generated by \refeq{e-discgrouporbit}, the isotropy groups are both equivalent to \Zn{2} in their structure,
and each symmetry constrains half of the orbits' modes to be equal to zero (different subsets of modes).
A connection can be made by realizing that the isotropy groups are both subgroups of the Klein four-group, $K_4$,
comprised of the elements $\{e, \sigma, \tau, \sigma\tau\}$ which represent the identity, spatial reflection,
half-cell time translation, and \spt\ shift-reflection. Shift-reflection is also sometimes referred to as \textit{glide reflection}.
Shift-reflection orbits are the \spt\ equivalent of {\ppo}s described in the dynamical systems formulation, as can
be seen by making the connection to plane Couette flow; another system where shift-reflection presents itself. If constrained to two dimensions,
the correspondence is seen immediately between the shift-reflection defined here and defined in.% ref gibson/others
\bea \label{shiftreflect_comparison}
s_1 \circ w(z,x) &=& -w(\speriod{z} - z, x+\frac{\speriod{x}}{2}) \continue
 \sigma \tau \circ u(x,t) &=& = -u(\speriod{}-x, t+\frac{\period{}}{2})
\eea
Clearly, $(w(z, x), z, x)$ of plane Couette share a correspondence with $(u(x, t), x, t)$ of the \KSe, respectively.
The notion of `fundamental domains' and pre-periodicity is replaced with a \spt\ symmetry invariant subspace of the modes, defined by constraints which require
modes to vanish. These constraints shall hereafter be referred to as \textit{selection rules}.

There are three different methods that these invariant subspaces and their selection rules can be derived.
These three methods are: group theoretic derivation using projection operators and irreducible subspaces,
multiplication of the matrix representations of projection operators with the mode vector, and lastly
the application of the invariance condition \refeq{e-invariance}. The advantages and disadvantages for
each of these three methods are as follows. The `group theoretic' derivation is useful when the symmetry
group is simple; i.e. of low order, when the character table is small, or the number of irreducible representations
is small (are all equivalent conditions). When this is the case, the decomposition into components and their
substitution into any nonlinearities is manageable. When the order of the symmetry group is large it would likely not be
obvious which subspace nonlinear combinations of components belong to, especially in the presence of non-commuting
differential operators. The advantage of this calculation is a precise description of the invariant subspaces.
The second method, construction of the matrix representations of the projection operators allows other linear operators
to be projected into the invariant subspaces; something which will be done to the temporal Fourier transform shortly
hereafter. The third method, via the invariance condition \refeq{e-invariance} allows for quick computation
of the selection rules for usage in computations.

The symmetry groups in %\refrefs{SCD07, KNSks90}
are also representations of $K_4$, the difference is that now the group elements involve operations in both space and time.
The proceeding analysis is nearly identical in its derivation but the interpretation is necessarily different due to
the lack of dynamics.


Starting with the group theoretic calculations; starting with the character table \reftab{K4table} is
\refeq{e-K4operators}.
%%%%%%%%%%%%%%%%%%%%
\begin{table}[h!]
\caption{\label{K4table}
Because the direct product group is abelian we only have one dimensional
representations and as such the character table follows directly.
\newline    }
\centering
\begin{tabular}{|c|c|c|c|c|}
\quad & $e$ & $\sigma$ & $\tau$ & $\sigma \tau$ \\
\hline
$E$ & 1 & 1 & 1 & 1 \\
$\Gamma_1$ & 1 & 1 & -1 & -1 \\
$\Gamma_2$ & 1 & -1 & 1 & -1 \\
$\Gamma_3$ & 1 & -1 & -1 & 1 \\
\end{tabular}
\end{table}
The character table, \reftab{K4table}, leads
to the construction of four linear projection operators
\bea \label{e-K4operators}
P^{(++)} &=& \frac{1}{4}(1+\sigma+\tau+\sigma\tau) \continue
P^{(+-)} &=& \frac{1}{4}(1+\sigma-\tau-\sigma\tau) \continue
P^{(-+)} &=& \frac{1}{4}(1-\sigma+\tau-\sigma\tau) \continue
P^{(--)} &=& \frac{1}{4}(1-\sigma-\tau+\sigma\tau)
\,,
\eea
The solution space can be decomposed into the irreducible subspaces
$\bbU = \bbU^{(++)} \oplus \bbU^{(+-)} \oplus \bbU^{(-+)} \oplus \bbU^{(--)}$.
Applying the projection operators \refeq{e-K4operators} to the modes in tensor notation
and using the relations \refeq{e-modesymmetries} to evaluate \refeq{e-K4operators} yields the four irreducible
subspaces of modes
\begin{align*}
P^{(--)}\utensor = \utensor^{(--)}&=
\begin{bmatrix}
0 & 0 \\
a_{\sss{1,k}} & 0 \\
0 &  0 \\
\vdots & \vdots \\
b_{\sss{1,k}} & 0 \\
0 & 0 \\
\vdots & \vdots
\end{bmatrix}
&
P^{(-+)}\utensor = \utensor^{(-+)} &=
\begin{bmatrix}
a_{\sss{0,k}} & 0 \\
0 & 0 \\
a_{\sss{2,k}} &  0\\
\vdots & \vdots \\
0 & 0 \\
b_{\sss{2,k}} & 0 \\
\vdots & \vdots
\end{bmatrix}
&\\ \\
P^{(+-)}\utensor =  \utensor^{(+-)} &=
\begin{bmatrix}
0 & 0 \\
0 & c_{\sss{1,k}} \\
0 &  0\\
\vdots & \vdots \\
0 & d_{\sss{1, k}} \\
0 & 0\\
\vdots & \vdots
\end{bmatrix}
&
P^{(++)}\utensor = \utensor^{(++)}  &=
\begin{bmatrix}
0 & c_{\sss{0,k}} \\
0 & 0 \\
0 &  c_{\sss{2,k}} \\
\vdots & \vdots \\
0 & 0 \\
0 & d_{\sss{2, k}} \\
\vdots & \vdots
\end{bmatrix} \numberthis %x\label{e-modeprojections}
\end{align*}
These four subspaces are apparently based on being symmetric or antisymmetric with respect to
reflections about $\speriod{}/2$ and half-cell time translations of $\period{}/2$.
Before applying these subspaces the effect of exchanging the projection operators \refeq{e-K4operators}
and spatial differentiation operator is derived; as this is
directly relevant for the calculation of the nonlinear term in \refeq{e-ks}
\bea \label{K4projderiv}
D_{\conf} P^{(++)} &=& \frac{1}{4}D_{\conf}(1 +\sigma  + \tau + \sigma \tau) \continue
                 &=& \frac{1}{4}(1 -\sigma  + \tau- \sigma \tau)D_{\conf} \continue
                 &=& P^{(-+)}D_{\conf} \continue
D_{\conf} P^{(+-)} &=& \frac{1}{4}D_{\conf}(1 + \sigma  - \tau- \sigma \tau) \continue
                 &=& \frac{1}{4}(1 -\sigma  - \tau+ \sigma \tau)D_{\conf} \continue
                 &=& P^{(--)}D_{\conf} \continue
D_{\conf} P^{(-+)} &=& \frac{1}{4}D_{\conf}(1 -\sigma  + \tau - \sigma \tau) \continue
                 &=& \frac{1}{4}(1 +\sigma + \tau + \sigma \tau)D_{\conf} \continue
                 &=& P^{(++)}D_{\conf} \continue
D_{\conf} P^{(--)} &=& \frac{1}{4}D_{\conf}(1 -\sigma  - \tau + \sigma \tau) \continue
                 &=& \frac{1}{4}(1 +\sigma  - \tau - \sigma \tau)D_{\conf} \continue
                 &=& P^{(+-)}D_{\conf}\,.
\eea
The effect of exchanging the operators can be summarized by flipping the sign of the first $\pm$, pertaining
to the coefficient of the spatial reflection terms in \refeq{e-K4operators}. The components
of the nonlinear term in each subspace are found to be
\bea \label{e-K4nonlinear}
P^{(++)}(u\partial_x u) &=& u^{(\pm \pm)}\partial_{\conf}(u^{(\pm \pm)})\continue
P^{(+-)}(u\partial_x u) &=& u^{(\pm \pm)}\partial_{\conf}(u^{(\pm \mp)})\continue
P^{(-+)}(u\partial_x u) &=& u^{(\pm \pm)}\partial_{\conf}(u^{(\mp \pm)})\continue
P^{(--)}(u\partial_x u) &=& u^{(\pm \pm)}\partial_{\conf}(u^{(\mp \mp)})\,.
\eea
Proper usage of $\pm$ and $\mp$ in \refeq{e-K4nonlinear} generates only four unique terms per
projection; i.e. the values of plus-minuses outside the differential are paired to those within.
Using these relations \refeq{e-K4nonlinear} we can produce the projections
of the \KSe\ onto the different irreducible subspaces, noting that the projection operator
commutes with the linear terms such that
\bea \label{e-KSEprojections}
P^{(++)}\goveqn &=& u_{\zeit}^{(++)}+u_{\conf \conf}^{(++)}+u_{\conf \conf \conf \conf}^{(++)} \continue
           &\quad&+ \big[u^{(++)}\partial_{\conf}(u^{(++)}) + u^{(+-)}\partial_{\conf}(u^{(+-)}) \continue
           &\quad&\:+u^{(-+)}\partial_{\conf}(u^{(-+)}) + u^{(--)}\partial_{\conf}(u^{(--)})\big]  \continue
P^{(+-)}\goveqn  &=& u_{\zeit}^{(+-)}+u_{\conf \conf}^{(+-)}+u_{\conf \conf \conf \conf}^{(+-)}\continue
           &\quad&+\big[u^{(++)}\partial_{\conf}(u^{(+-)}) + u^{(+-)}\partial_{\conf}(u^{(++)}) \continue
           &\quad&\:+\,u^{(-+)}\partial_{\conf}(u^{(--)}) + u^{(--)}\partial_{\conf}(u^{(-+)})\big]  \continue
P^{(-+)}\goveqn  &=& u_{\zeit}^{(-+)}+u_{\conf \conf}^{(-+)}+u_{\conf \conf \conf \conf}^{(-+)}\continue
           &\quad&+\big[u^{(++)}\partial_{\conf}(u^{(-+)}) + u^{(+-)}\partial_{\conf}(u^{(--)}) \continue
           &\quad&\:+\,u^{(-+)}\partial_{\conf}(u^{(++)}) + u^{(--)}\partial_{\conf}(u^{(+-)})\big] \continue
P^{(--)}\goveqn  &=& u_{\zeit}^{(--)}+u_{\conf \conf}^{(--)}+u_{\conf \conf \conf \conf}^{(--)}\continue
           &\quad&+\big[u^{(++)}\partial_{\conf}(u^{(--)}) + u^{(+-)}\partial_{\conf}(u^{(-+)}) \continue
           &\quad&\:+\,u^{(-+)}\partial_{\conf}(u^{(+-)}) + u^{(--)}\partial_{\conf}(u^{(++)})\big]\,.
\eea
With this we can determine the \textit{symmetry invariant subspaces} or simply \textit{invariant subspaces}
by solving $P\goveqn=\mathbf{F}(P\statev)$ where $P=\sum_k P^{(k)}$ noting that the projection operator acts only
on the $\utensor$ component of \statev.
This is similar to the notion of \textit{flow invariant subspaces} except that there are no longer any dynamics,
and so the use of `flow' does not apply here. One method of deriving the invariant subspaces is to
simply assume a decomposition of $u$, then show that the sum of the relevant equations \refeq{e-KSEprojections}
commute with the projection operator.
For example, assume $u=u^{(++)}$; substitution into \refeq{e-KSEprojections} yields
\bea \label{e-KSplusplus}
P^{(++)}F(u^{(++)}) &=& u_{\zeit}^{(++)}+u_{\conf \conf}^{(++)}+u_{\conf \conf \conf \conf}^{(++)}
                +u^{(++)}\partial_{\conf}(u^{(++)}) \continue
F(P^{(++)}u) &=& u_{\zeit}^{(++)}+u_{\conf \conf}^{(++)}+u_{\conf \conf \conf \conf}^{(++)}
                +\frac{1}{2}\partial_x ((u^{(++)})^2) \,.
\eea
which are equivalent after applying the derivative in the second equation.
Therefore $\bbU^{(++)}$ constitutes an invariant subspace. This subspace corresponds to
antisymmetric equilibria.
The rest of the symmetry invariant subspaces follow from a
similar substitutions. To expedite the derivation process, note
that the equation for $P^{(++)}F$ contains
all of the symmetric terms $u^{\pm \pm}\partial_{\conf}(u^{\pm \pm})$
such that invariant subspaces must have non-empty intersection with $\bbU^{(++)}$.
Following a process of elimination we can show that the possible (nontrivial)
symmetry invariant subspaces are $\bbU^{(++)}$, $\bbU^{(++)}\oplus \bbU^{(--)}$,
$\bbU^{(++)}\oplus \bbU^{(+-)}$ and $\bbU^{(++)}\oplus \bbU^{(-+)}$. We can interpret
these subspaces by addition of the corresponding projection
operators \refeq{e-K4operators}
\bea \label{e-invariantoperators}
P^{0}\equiv P^{(++)} &=& \frac{1}{4}(1 +\sigma +\tau+ \sigma \tau) \continue
P^{\sigma}\equiv P^{(++)}+P^{(+-)} &=& \frac{1}{2}(1 + \sigma) \continue
P^{\tau}\equiv P^{(++)}+P^{(-+)} &=& \frac{1}{2}(1 + \tau) \continue
P^{\sigma \tau}\equiv P^{(++)}+P^{(--)} &=& \frac{1}{2}(1+\sigma \tau)
\,.
\eea
The invariant subspaces following trivially (with a slight abuse of the irreducible subspace notation)
\bea \label{e-invariantsubspaces}
\bbU_{0} &\equiv& \{\statev\:|\:  P_{0}\statev_{\utensor}=\statev_{\utensor}\} \continue
\bbU_{\sigma} &\equiv&  \{\statev\:|\:  P_{\sigma}\statev_{\utensor}=\statev_{\utensor}\} \continue
\bbU_{\tau} &\equiv&  \{\statev\:|\:  P_{\tau}\statev_{\utensor}=\statev_{\utensor}\} \continue
\bbU_{\sigma\tau} &\equiv&  \{\statev\:|\:  P_{\sigma\tau}\statev_{\utensor}=\statev_{\utensor}\}
\eea
where $\statev_{\utensor}$ are the modes of \refeq{e-statevector}.
$\bbU^{0}$ represents the subspace of antisymmetric equilibria,
$\bbU^{\sigma}$ the spatial reflection invariant subspace,
$\bbU^{\sigma\tau}$ the shift-reflection invariant subspace,
and lastly $\bbU^{\tau}$ with solutions `twice-repeating' solutions, i.e.
those symmetric with respect to $\period{}/2$. The continuous spatial translations are eliminated
from every subspace \textit{except} $\bbU^{(++)}\oplus \bbU^{(-+)}$. This can
be made explicit by acting on the projected modes \refeq{e-e-modeprojections} with the
generator of translations (to be defined in \refsect{sect:mapping}. Because of its unique properties,
the $\bbU^{(++)}\oplus \bbU^{(-+)}$ subspace will be discussed at the end of the section.

Naturally the definition of \goveqn\ in the shift-reflection and reflection invariant subspaces comes
from the substitution of $u^{(++)}+u^{(--)}$ and $u^{(++)}+u^{(+-)}$ into \goveqn\ however, this is not a
very useful representation of the equations; instead the equations are derived in terms of the components
of the mode tensor \refeq{e-modetensor} in \refsect{sect:mapping}.

The invariance condition \refeq{e-invariance} is used to derive
constraints on which modes are non-vanishing, referred to as \textit{selection rules}. These selection rules
correspond directly to the invariant subspaces produced by \refeq{e-invariantoperators}.
Therefore, using the invariance condition \refeq{e-invariance} is equivalent to applying the
corresponding projection operator \refeq{e-invariantoperator} and finding which modes survive, $P \utensor = \utensor$.
The invariance condition will be used to derive the selection rules in a functional form,
but the matrix representations of the projection operators \refeq{e-invariantoperators}
are useful for applying the selection rules to other linear operators. The focus is only on reflection
and shift-reflection invariance, such that the derivations are limited to the matrix representations
of $P_{\sigma}$ and $P_{\sigma\tau}$. All derivations will utilize the conventions on the notation
for identity matrices from \refeq{e-identitysize}.
For the real and imaginary components of a single spatial mode,
the spatial reflection operator (with reflection axis $x = \speriod{}/2$) is
\beq
\sigma^{''} =
\begin{bmatrix}
-1 & 0 \\
0 & 1 \\
\end{bmatrix}
\eeq
This changes the sign of the real component of the mode. Generalizing to a set of $\frac{M}{2}-1$ spatial modes (using
the arrangement convention described in \refsect{sect:spt_fourier} this becomes
\begingroup
\renewcommand*{\arraystretch}{1.5}
\beq
\sigma^{'} = \sigma^{''} \otimes \mathbb{I}_{k}
= \begin{bmatrix}
-\mathbb{I}_{k} & 0 \\
0 & \mathbb{I}_{k} \\
\end{bmatrix}
\ee{e-reflsinglespace}
Where $0$ represents the appropriately sized \textit{matrix} of zeros whose dimension in this case is $(\frac{M}{2}-1)^2$.
The \spt\ version of \refeq{e-reflsinglespace} is a diagonal matrix consisting of $N-1$ repeats of \refeq{e-reflsinglespace}
\bea\label{e-refloperator}
\sigma &=& \mathbb{I}_{(N-1)} \otimes \sigma^{'} \continue
        &=& \begin{bmatrix}
        -\mathbb{I}_{k} &&&& \\
        & \mathbb{I}_{k} &&& \\
        && -\mathbb{I}_{k} && \\
        &&& \mathbb{I}_{k} & \\
        &&&& \ddots
        \end{bmatrix} \,.
\eea
with \refeq{e-refloperator} the projection operator $P_{\sigma}$ can be defined
\bea\label{e-reflproj}
P_{\sigma} &=& \frac{1}{2}(\mathbb{I} + \sigma)\continue
        &=& \begin{bmatrix}
        0 &&&& \\
        & \mathbb{I}_{k} &&& \\
        && 0 && \\
        &&& \mathbb{I}_{k} & \\
        &&&& \ddots
        \end{bmatrix} \,.
\eea
Which can be seen clearly keeps the spatially antisymmetric components of the modes with respect to the
ordering \refeq{e-modevector}.

The shift-reflection projection operator requires both the temporal half-cell shift $\tau$ and the reflection operator \refeq{e-refloperator}.
Half-cell shift is equivalent to rotation by $\pi$ such that
\beq \label{e-tausingle}
\tau^{'} =
\begin{bmatrix}
\cos(j\pi) & \sin(j\pi) \\
-\sin(j\pi) & \cos(j\pi) \\
\end{bmatrix}
= \begin{bmatrix}
\quad 1 & & \\
& \quad (-1)^{j}\: \mathbb{I}_{j} & \\
&&(-1)^{j}\:\mathbb{I}_{j}\quad\\
\end{bmatrix}
\eeq
This specific form of \refeq{e-tausingle} is used to represent the fact that $j$ follows the tensor index convention
defined in \refeq{e-modeindices}. The \spt\ version of \refeq{e-srsingle}
is achieved via another Kronecker product.

\beq \label{e-tauoperator}
\tau = \tau^{'} \otimes \mathbb{I}_{(M-2)}
= \begin{bmatrix}
\quad 1 &&& \\
& \quad (-1)^{j}\: \mathbb{I}_{j} && \\
&&(-1)^{j}\:\mathbb{I}_{j} & \\
&&& {\ddots\quad} \\
\end{bmatrix}
\eeq
Using the Kronecker product identity $(A\otimes B)(C\otimes D) = (AC \otimes BD)$ (if the sizes are consistent) we have
\bea
P_{\sigma\tau} &=& \frac{1}{2}(\mathbb{I} + \sigma\tau) \continue
&=& \frac{1}{2}(\mathbb{I} + (\mathbb{I}_{(N-1)} \otimes \sigma^{'})(\tau^{'} \otimes \mathbb{I}_{(M-2)})) \continue
                &=& \frac{1}{2}(\mathbb{I} + (\tau^{'} \otimes \sigma^{'})))
\eea
where
\begin{flalign}
\label{e-sigmatauprod}
(\tau^{'} \otimes \sigma^{'}) &\equiv
\begin{bmatrix}
-\mathbb{I}_{k} &&&&&&&   \\
& \quad\mathbb{I}_{k} &&&&&& \\
&&(-\mathbb{I}_{k})^{j+1} &&&&&   \\
&&& (-\mathbb{I}_{k})^{j} &&&& \\
&&&& \ddots &&& \\
&&&&&(-\mathbb{I}_{k})^{j+1} &&  \\
&&&&&&(-\mathbb{I}_{k})^{j}&  \\
&&&&&&& \ddots
\end{bmatrix}
\end{flalign}
Substitution into the expression for $P_{\sigma\tau}$ yields a diagonal matrix where
the terms with $j+1$ survive with $j$ is odd, and the terms with $j$ survive when $j$ is even. This makes it so the diagonal
alternates between two identity matrix blocks and two zero blocks until, that is, $j$ `restarts'.
\begin{flalign}
\label{e-srproj}
P_{\sigma\tau} &\equiv
\begin{bmatrix}
0 &&&&&&&   \\
& \quad\mathbb{I}_{k} &&&&&& \\
&& \mathbb{I}_{k} &&&&&   \\
&&& 0 &&&& \\
&&&& 0 &&& \\
&&&&&\ddots &&  \\
&&&&&&\mathbb{I}_{k}&  \\
&&&&&&& \ddots
\end{bmatrix}
\,.
\end{flalign}
\endgroup

This concludes the derivation of the two projection operators found to be useful in this study.
These projections can be formulated in a more efficient manner where the matrices  are never explicitly constructed.

The explicit construction of the matrices \refeq{e-reflproj} and \refeq{e-srproj} is inefficient
and unnecessary as the projections can be written in a functional form as a set of constraints or
\textit{selection rules} as they are denoted here.

The selection rules pertaining to a particular symmetry are derived by substituting \refeq{e-rfft} into \refeq{e-invariance}.
The modes that can only satisfy the equation if they are identically zero will be referred to as \textit{vanishing modes}.

For spatial reflection symmetry the selection rules come immediately without any calculations;
only modes attached to $\sin(\wavek\xm)$ are allowed.
In other words the vanishing modes are coefficients to $\cos(\wavek\xm)$ in \refeq{e-rfft}
\beq \label{e-arules}
\ajk=\bjk\ = 0\quad\forall(j,k)
\,.
\eeq
The selection rules for equilibria follow trivially from this, simply get rid
of any dependence on time index $j$.
\bea \label{e-eqvrules}
\ajk=\bjk=\djk =& 0 \quad\forall(j, k) \continue
\cjk =& 0\quad\forall j \neq 0\continue
(j,k) &\widehat{=}&\{\{0\}\}\times \{\{k\}\}\continue
|(j,k)| &=& \frac{M}{2}  - 1 \quad\mbox{non-vanishing modes.}
\,.
\eea
Unfortunately the shift-reflection selection rules are not as simply derived.
The exact long form expression of the Fourier transform \refeq{e-rfft} is replaced with a simplified version
to make the derivation concise and clear.
Utilizing trigonometric identities and the parity of $\sin$ and $\cos$ we have

\begin{alignat}{3}
\sigma \tau\,u(\tn, \xm)&=-&&\sum_{k,j}&&\Big(\ajk\cos(\tilde{\omegaj}(\tn + \period{}/2))+\bjk\sin(\tilde{\omegaj}(\tn + \period{}/2))\Big)\cos(\wavek (\speriod{}-\xm)) \continue
                              &&&&&\Big(-\cjk\cos(\tilde{\omegaj}(\tn +\period{}/2))-\djk\sin(\tilde{\omegaj} (\tn + \period{}/2))\Big)\sin(\wavek(\speriod{}-\xm)) \continue
                         &=&&\sum_{k,j}&-&\cos(\pi j)\Big((\ajk \cos(\tilde{\omegaj} \tn)+\bjk \sin(\tilde{\omegaj} \tn))\Big)\cos(\wavek \xm)\continue
                        &&&&+&\cos(\pi j)\Big(-\cjk\cos(\tilde{\omegaj} \tn)-\djk \sin(\tilde{\omegaj} \tn)\Big)\sin(\wavek \xm) \continue
                         &=&&\sum_{k,j}&&(-1)^{j+1}\Big((\ajk\cos(\tilde{\omegaj} \tn)+\bjk\sin(\tilde{\omegaj} \tn))\Big)\cos(\wavek \xm)\continue
                         &&&&+&(-1)^{j}\Big(-\cjk \cos(\tilde{\omegaj} \tn)-\djk\sin(\tilde{\omegaj}\tn)\Big)\sin(\wavek \xm)\, ,
\end{alignat}
substitution into the invariance condition \refeq{e-invariance} yields the following selection rules
 for \spt\ shift-reflection
\bea \label{e-srrules}
&\ajk = \bjk = 0 \, \mbox{ for }\, j \, \mbox{ even } \continue
&\cjk = \djk = 0 \, \mbox{ for }\, j \, \mbox{ odd }
\,.
\eea
To benefit from these selection rules numerically, simply constraining the modes to be zero is not sufficient. The main benefit is
derived from the discarding of the vanishing modes which reduces the number of \cdof. The non-vanishing modes
must be represented in a contiguous fashion and so a rearrangement of the array is in order.
Much like the derivation of the selection rules themselves, this is straightforward for antisymmetric orbits; this
can also be described via the index notation of \refeq{e-modeindices}
\bea \label{e-antitensor}
\utensor_{\sigma} &\equiv&
\begin{bmatrix}
\cjk \\
\djk
\end{bmatrix}\,. \continue
(j,k) &\widehat{=}&\{\{0\}, \{j\}, \{j\}\} \times \{k\}\continue
|(j,k)| &=& (N-1) \times (\frac{M}{2} - 1) \quad\mbox{non-vanishing modes}\,.
\eea
Shift-reflection seems harder to handle at first but luckily there is a trick that
can be exploited to yield a coherent form. This is most easily seen by explicitly
applying the selection rules to \refeq{e-modetensor}
\beq \label{e-srtensor}
\utensor_{\sigma\tau} \equiv
\begin{bmatrix}
0 & c_{\sss{0,k}} \\
a_{\sss{1,k}} & 0 \\
0 &  c_{\sss{2,k}} \\
\vdots & \vdots \\
b_{\sss{1,k}} & 0 \\
0 & d_{\sss{2, k}} \\
\vdots & \vdots
\end{bmatrix}
\,.
\eeq
This staggered structure can be exploited by discarding the zeros and `shuffling' the modes together
\bea \label{e-srtensorordered}
\utensor_{\sigma\tau} &\equiv&
\begin{bmatrix}
c_{\sss{0, k}} \\
a_{\sss{1, k}} \\
c_{\sss{2, k}} \\
\vdots  \\
b_{\sss{1, k}} \\
d_{\sss{2, k}} \\
\vdots
\end{bmatrix} \continue
&\widehat{=}&\{\{0\}, \{j\}, \{j\}\} \times \{k\}\continue
|(j,k)| &=& (N-1) \times (\frac{M}{2} - 1) \quad\mbox{non-vanishing modes}
\,.
\eea
By formatting the modes in this way note that the temporal frequencies are arranged in a manner identical
to \refeq{e-modetensor} and \refeq{e-antitensor}. In other words, no extra work is required to define
time differentiation. Also note that the shift-reflection subspace has the same dimension as the antisymmetric
subspace which makes sense in the scope of the symmetry group. The down side is that the modes
need to be `unshuffled' before the inverse transform \irfft\ can be applied, that is, the zeros
present in \refeq{e-srtensor} need to be reinserted.

The selection rules are incorporated into the definition of the temporal
transforms to yield a `symmetry invariant' Fourier transform. By doing so, there is no possibility of forgetting to
apply the selection rules. It is also necessary to apply these selection rules to the matrix representations of linear operators. Discarding
the vanishing modes requires a modification of the projection operators \refeq{e-reflproj}, \refeq{e-srproj}. Note that
the projection operators are diagonal matrices and that the vanishing modes correspond to where the diagonal equals zero. Therefore,
the vanishing modes are properly discarded by removing these rows of the projection matrices.
Applying this to other linear operators is simply derived by matrix multiplication and so the explicit results are not derived.

Before proceeding to continuous symmetries, here is a summary on what has been covered for discrete symmetries.
First, the distinction between equivariance and invariance was made. By doing so, very specific definitions of group orbits and
isotropy groups were defined.
It was realized that the isotropy subgroups of interest were both subgroups of a larger group which is equivalent
to the Klein four group. Owing to the structure of this group, irreducible subspaces and their linear projection operators
were defined. Four symmetry invariant subspaces were then derived by acting on the \KSe\ with combinations of these projection operators.
Each of these subspaces has a very useful set of constraints denoted as selection rules (equivalent to projection onto the respective subspace).
The selection rules were defined in both a functional form and as matrices.
Finally, the rules were incorporated into the temporal transforms,
referring to these new transforms as symmetry invariant Fourier transforms. For orbits within
each invariant subspace the redefined Fourier transforms remain orthogonal transformations.

There are two continuous symmetries that are accounted for, Galilean invariance and spatial translation symmetry which
results in relative periodic orbits.
The first of these two symmetries, invariance under Galilean transformations, results in the following equivariance:
if \ufield is a solution, then $u(t, x - ct) - c$ is also a solution ($c$ being an arbitrary constant speed or `mean flow').
This is handled with a simple constraint which constrains the mean flow with respect to space
\beq
\spaceAver{u}(\zeit)
  \,=\, \int_0^{\speriod{}} d\conf \, u(\conf,\zeit) = 0
\,.
\ee{GalInv}
The practical method to enforce this is to simply discard the zeroth spatial modes, i.e. those with $k=0$.
This is a conventional choice but not necessarily the best choice; there is to be a longer discussion on this later,
but the general idea is that reducing the group orbit without purpose is not desirable. Finding orbits is the
goal, not finding specific members of group orbits. It is believed that reducing the dimension of the group orbit may
reduce the likelihood of finding that group orbit numerically. Regardless, the convention defined by \refeq{e-GalInv}
is enforced.

The last \spt\ symmetry to discuss is spatial translation symmetry which creates relative periodic orbits. To
make a distinction between uniform rotations, this symmetry will be referred to as \textit{spatial shift} symmetry.
A relative periodic orbit is defined here as an orbit whose field \dufield is only
periodic after accounting for a `spatial drift' or `shift'. In other words, relative
periodic orbits abide by the following relation
\beq
\ufield = g \circ u(t + \period{}, x)\,.
\eeq
Where $g$ represents a spatial shift defined by the \goveqn.
If this shift is not accounted for, the \ufield\ will be discontinuous in time.
To make the most out of the Fourier basis some intervention is required to generate
a truly periodic representation of the field.

Two ideas come to mind, either utilize a comoving reference frame or slice (quotient)
the continuous symmetry. The slicing method has a number of disadvantages and
so only a comoving frame is used.
These disadvantages and their derivation are relegated to the appendix.
The comoving frame and its spatial shift needs to be formulated in dimensionless spatial units
as opposed to an angular quantity. This is important because $\period{}$ and $\speriod{}$ and $s$ are
changing in the numerical optimization process and they are coupled in a non-trivial manner. First,
spatial rotations are defined in a manner practically identical to \refeq{e-srsingle}
The matrix representation of \SOn{2} group elements for a spatial shift by an
amount $s$, for a single time value $u(t',\xm-s)$ is
\beq \label{e-spacesingle}
\LieEl(s) \equiv
\begin{bmatrix}
\cos(\wavek s) & \sin(\wavek s)\\
-\sin(\wavek s) & \cos(\wavek s)
\end{bmatrix}\,.
\eeq
The index $k$ indicate that each element \refeq{e-spacesingle} represents
a diagonal matrix where $k$ takes values \refeq{e-jkindices}.
This matrix applied to a set of spatial modes defined at a specific instant in time, shifts them by $s$
in the positive $x$ direction.  The \spt\ version of this operator, that is, uniform spatial rotations
of the entirety of \dufield, is a block diagonal matrix with $N$ blocks; one for each value of $\tn$.
Both uniform spatial translations and comoving rotations
are applied in the spatial mode basis, as it makes more sense considering the parameterization in time
to follow. The uniform shift \refeq{e-spacesingle} generalizes to space-time via Kronecker product as
seen before in \refeq{e-tauoperator}
\beq \label{e-SOnspacetime}
\mathbf{\LieEl}(s)  = \mathbb{I}_{N} \otimes \LieEl(s)
\eeq
which, explicitly, equals
\beq \label{e-SOnspacetimematrep}
\mathbf{\LieEl}(s) \equiv
\begin{bmatrix}
\LieEl(s) & 0 & \cdots & 0 \\
0 & \LieEl(s) & \cdots & 0 \\
\vdots & \vdots & \ddots & \vdots \\
 0 & 0 & 0 & \LieEl(s)
\end{bmatrix}
\,.
\eeq
The general idea originates in L{\'o}pez \etal\rf{lop05rel} where the field is kept in a co-moving reference frame.
The co-moving reference frame transformation is a generalization of \refeq{e-SOnspacetimematrep} where the previously
uniform shift $s$ is replaced by a spatial shift linearly parameterized by time. That is, at every discrete
time $\tn$, the field is shifted by an amount $\cmshift$ via the symmetry operation $\LieEl(\cmshift)$.
For brevity let $s_n \equiv s(\tn) \equiv \cmshift$.
Using \refeq{e-SOnGroupElement} the matrix representation
of the co-moving frame transformation is as follows
\beq \label{e-comovingmatrix}
\mathbf{\LieEl}(\phi_n) \equiv
\begin{bmatrix}
\LieEl(\phi_{\scalebox{.4}{$N-1$}}) & 0 & \cdots & 0 \\
0 & \LieEl(\phi_{\scalebox{.4}{$N-2$}})  & \cdots & 0 \\
\vdots & \vdots & \ddots & \vdots \\
 0 & 0 & 0 & \LieEl(\phi_0)
\end{bmatrix}
\,.
\eeq
Note that the time is decreasing along the diagonal due to the conventions of the numerical implementation.
Using \refeq{e-comovingmatrix} The ansatz for relative periodic solutions can be written as the modification of the
\spt\ transform \refeq{e-rfft}. As a convention, the shift s is defined as the shift from the comoving frame to the physical
frame such that. The most succinct form of the comoving transformation is writing the expression in terms of operators for the
translations and Fourier transforms.
\beq \label{e-ansatzdufield}
\dufield = \IFFT_x (\mathbf{\LieEl}(\cmshift) \IFFT_t(\uvec))\,.
\eeq
Where the action of $\mathbf{\LieEl}(\cmshift)$ maps $\xm \to \xm - \cmshift$, a translation in the positive
spatial direction. Equivalently, this can be written as
the inverse Fourier transform of rotated spatial modes
\begin{alignat}{2}\label{e-comoving}
\mathbf{\LieEl}\;\dufield = \frac{2}{\sqrt{M}}&\sum_{k = 1}^{\frac{M}{2} - 1}&&\big[e_{k}(\tn)\cos(\wavek\cmshift)+f_k(\tn)\sin(\wavek \cmshift)\big]\cos(\wavek \xm) \continue
&&+&\big[-e_{k}(\tn)\sin(\wavek\cmshift)+f_k(\tn)\cos(\wavek\cmshift)\big]\sin(\wavek \xm) \continue
=\frac{2}{\sqrt{M}} &\sum_{k = 1}^{\frac{M}{2} - 1}&&e_{k}(\tn)\cos(\wavek (\xm - \cmshift))+f_k(\tn)\sin(\wavek (\xm - \cmshift))\,.
\end{alignat}
An important detail is that \refeq{e-comoving} reproduces the \dufield\ which solves \refeq{e-ks}, \textit{however}, numerically
the field is kept always kept in the comoving frame; such that symmetry operation is not built into the temporal Fourier transform
as it is for other symmetries. The reasoning for this choice presents itself when deriving the \goveqn\ in terms of the modes.

In conclusion, a \spt\ formulation requires a \spt\ description of symmetries. The same concepts from %
apply here and so the derivations look quite similar. The interpretation however is not similar due to the absence
of dynamics. Previously, the symmetries flow-invariant subspaces which are unstable to perturbations. Now, the \spt\ symmetries
manifest as selection rules; constraints which require subsets of modes to vanish, i.e. equal zero.
One application of this is the shift-reflection invariant subspace, which cannot be derived as a flow
invariant subspace, as the symmetry involves time. It \textit{is} possible to derive a shooting-method
type constraint $u(t)=\sigma u(t+\frac{\period{}}{2})$

but this type of equation is clearly highly dependent on time evolution. Therefore, the \spt\ formulation
Before moving on, one last comment is to be made about the subspace corresponding to $P_{\tau}$ of \refeq{e-invariantoperators} which was not mentioned so far.
This subspace contains the `twice-repeating' solutions of the \KSe.
Now, this might seem trivial at first, because every orbit is doubly-periodic by definition.
That is, any orbit has a representation which exists in this invariant subspace; created by simply taking
a tiling of size $[0, 2\period{}] \times [0, \speriod{}]$. Note that one must use this representation of the orbit otherwise
it would not be \textit{invariant} under the transformation. This subspace remains unutilized
currently; one hypothesis is that \spt\ symmetry groups of higher order (i.e. higher order cyclic subgroups) would permit
more interesting behavior than simply `twice-repeating'. One usage of the `twice-repeating' subspace is that it can
be used in conjunction with other symmetries to impose even stricter selection rules than \refeq{e-arules} and \refeq{e-srrules}.
Why? For example, two repeats of a shift reflection orbit is invariant under more combinations of symmetry actions: for example,
compositions of quarter-cell and three-quarter cell shifts with reflection leave the orbit invariant. Therefore, in order
to be invariant under more symmetries, more modes are constrained to zero. A preliminary investigation shows that
for twice-repeating shift-reflection orbits, the number of non-vanishing modes reduces by another factor of two. This
can be interpreted via \spt\ symbolic dynamics, however the discussion is not yet equipped to handle this, however, the
followed hypothesis is offered up as a potential explanation.
Part of the hypothesis is that the frequency with which \twots\ are shadowed is somehow inversely proportional to their size.
Therefore, a region of space-time shadowing two-repeats of an orbit is a stricter and more specific condition
than shadowing a single period. This manifests as a stricter constraint on the modes which in turn yields more selection rules.

In summary, the symmetries of the \KSe\ have been cast into a \spt\ form. From the discrete and continuous symmetries considered the following
categories are permissable: antisymmetric, shift-reflection, relative periodic, equilibrium and relative equilibrium. Discrete symmetries
can be utilized to produce selection rules, constraints on the modes, such that the \cdof\ can be reduced substantially. On the other hand,
orbits with continuous symmetry actually require an additional \cdof, namely, the spatial shift accumulated after one temporal period.

