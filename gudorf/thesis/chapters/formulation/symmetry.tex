The symmetries of the \KSe\ in the \spt\ formulation include Galilean invariance and the
symmetry group $G \equiv \SOn{2}\times On{2}$. This \spt\ symmetry group provides a new perspective with which
to view orbits and their symmetries. This in turn provides a new tool which greatly benefits the computations to follow.
These benefits rely on an important distinction to be made; namely, a distinction must be made between \textit{equivariance}
and \textit{invariance} of orbits with respect to group operations.

For a subgroup $H \subset G$, an orbit is $H$-equivariant, if every element of the group orbit produced by $H$
\beq
\Manif_{\orbit} = {h \cdot \orbit | h \in H}
\eeq
are themselves orbits, that is, they are all solutions to \refeq{e-ks}. Invariance is then the term
used to describe the isotropy subgroup (alternatively \textit{stabilizer group} or \textit{little group})
with respect to an orbit; that is if $K \subset H \subset G$ is the isotropy subgroup of \orbit\ if
\beq
k \cdot \orbit = \orbit, \quad \forall k \in K\,.
\ee{e-invariance}
Note that in this context, this is a very precise statement that the velocity field \dufield is \textit{exactly}
the same as it was prior to the symmetry operation.

For clarity, the statements: `an orbit has a discrete symmetry' and
`an orbit with discrete symmetry' is always a statement of invariance, not equivariance. First the group
orbits and isotropy groups considered here are defined. In order to


The group orbit for orbits without discrete symmetries are $G$-equivariant, their isotropy subgroups
are the trivial subgroup. In the discrete setting, the group orbit \dufield\ is not $G$. Because the field
is defined only at the collocation points, arbitrary rotations are essentially interpolations of the field
because they return the field at points in between the collocation points, which do not satisfy \refeq{e-ks}
numerically. Therefore, the group orbit of the discretized field \dufield is defined by
spatial reflection and discrete \spt\ rotations, i.e.
\beq
H_{\orbit} = \Cn{N} \times \Dn{M}
\ee{e-grouporbit}
$\Cn{N}$ is the cyclic group of order $N$ that accounts for discrete time rotations. $\Dn{M}$ is the dihedral group
generated by discrete spatial rotations and reflection about $\frac{\speriod{}}{2}$. The isotropy group in this case
is the trivial subgroup of $G$.

For brevity of the discussion that follows, let $\sigma$ represent spatial reflection about $x = \frac{\speriod{}}{2}$
such that
\beq
\sigma \circ u(\tn, \xm) = -u(\tn, \speriod{} - \xm)
\ee{e-reflect}
and let $\tau$ represent half-cell ($\period{}/2$) translations in time such that
\beq
\tau \circ u(\tn, \xm) = u(\tn + \frac{\period{}}{2}, \xm) \,.
\ee{e-tshift}
The \spt\ shift-reflection is also defined by the composition of these two operations $s = \sigma \tau$ such that
\beq
s \circ u(\tn, \xm) = -u(\tn + \frac{\period{}}{2}, \speriod{} - \xm) \,.
\ee{e-shiftreflect}

The next orbit invariances to be described are spatial reflection invariance and \spt\ shift-reflection invariance.
Hereafter, orbits with these symmetries are referred to as antisymmetric orbits and shift-reflection orbits, respectively.
Shift-reflection is also sometimes referred to as \textit{glide reflection}.
Shift-reflection orbits are the \spt\ equivalent of {\ppo}s described in the dynamical systems formulation. The `fundamental domains'
and pre-periodicity of the dynamical systems formulation is replaced with a \spt\ symmetry invariant subspace of the modes, defined
by the constraint imposed by the corresponding invariance condition.

The group orbit for both antisymmetric
and shift-reflection symmetric orbits is generated by half-cell shifts and reflections in space, \Dn{2}, and discrete time rotations, \Cn{N}.
\beq
H_{\orbit} = \Cn{N} \times \Dn{2} ,.
\ee{e-discgrouporbit}
For antisymmetric orbits,
the isotropy subgroup consists of only spatial reflections,
\beq
H_a = \{e, \sigma\}
\ee{e-anti_isotropy}
and similarly for shift-reflection, represented as $s$,
\beq
H_{s} = \{e, s\} \,.
\ee{e-sr_isotropy}

There are a number of important similarities between antisymmetric and shift-reflection orbits.
The group orbits for both are generated by \refeq{e-discgrouporbit}, the isotropy groups are both equivalent to \Zn{2} in their structure,
and each symmetry constrains half of the orbits' modes to be equal to zero (albeit different subsets).
A connection can be made by realizing that the isotropy groups are both subgroups of the Klein four-group, $K_4$,
comprised of the elements $\{e, \sigma, \tau, \sigma\tau\}$. Note that because shift-reflection
satisfies $s = \sigma\tau = \tau\sigma$, the choice is made to keep it written as a composition of $\sigma$ and $\tau$
for clarity and for the following comparison.
Namely, the connection to plane Couette flow; another system where shift-reflection presents itself. If constrained to two dimensions,
the correspondence is seen immediately between the shift-reflection defined here and defined in{}.% ref gibson/others
\bea \label{shiftreflect_comparison}
s_1 \circ w(z,x) &=& -w(\speriod{z} - z, x+\frac{\speriod{x}}{2}) \continue
 \sigma \tau \circ u(x,t) &=& = -u(\speriod{}-x, t+\frac{\period{}}{2})
\eea
Clearly, $(w(z, x), z, x)$ of plane Couette share a correspondence with $(u(x, t), x, t)$ of the \KSe, respectively.

The symmetry groups in %\refrefs{SCD07, KNSks90}
are also representations of $K_4$, the difference is that now the group elements involve operations in both space and time.
It's no surprise then, that the analysis that follows is nearly identical in derivation but different in interpretation.

We demonstrate some standard group theoretic calculations such as
looking at the character table \reftab{D1C2table} and projection operators
\refeq{e-D1C2operators}.

%%%%%%%%%%%%%%%%%%%%
\begin{table}[h!]
\caption{\label{D1C2table}
Because the direct product group is abelian we only have one dimensional
representations and as such the character table follows directly.
\newline    }
\centering
\begin{tabular}{|c|c|c|c|c|}
\quad & $e$ & $\sigma$ & $\tau$ & $\sigma \tau$ \\
\hline
$E$ & 1 & 1 & 1 & 1 \\
$\Gamma_1$ & 1 & 1 & -1 & -1 \\
$\Gamma_2$ & 1 & -1 & 1 & -1 \\
$\Gamma_3$ & 1 & -1 & -1 & 1 \\
\end{tabular}
\end{table}
The character table \reftab{D1C2table}, leads
to the construction of four linear projection operators
\bea \label{e-D1C2operators}
P^{++} &=& \frac{1}{4}(1 +\sigma + \tau + \sigma \tau) \continue
P^{+-} &=& \frac{1}{4}(1 +\sigma - \tau - \sigma \tau) \continue
P^{-+} &=& \frac{1}{4}(1 -\sigma  + \tau - \sigma \tau) \continue
P^{--} &=& \frac{1}{4}(1 -\sigma  - \tau + \sigma \tau)
\,,
\eea
The solution space can be decomposed into the irreducible subspaces produced
by these projection operators
$\bbU = \bbU^{++} \oplus \bbU^{+-} \oplus \bbU^{-+} \oplus \bbU^{--}$.
The projection operators directly correspond to the four different \spt\ parities; specifically
\bea \label{D1C2subspaces}
P^{-+}\dufield = u^{-+}(\conf,\zeit)  &=&  \frac{4}{\sqrt{NM}} \sum_{k = 1}^{\frac{M}{2} - 1} \Big[ \frac{a_{\sss{{0}{k}}}}{2} + \sum_{j=1}^{\frac{N}{2}-1}\ajk\cos(\omegaj \tn)\cos(\wavek \xm) \Big] \continue
P^{--}\dufield =  u^{--}(\conf,\zeit)  &=& \frac{4}{\sqrt{NM}} \sum_{k = 1}^{\frac{M}{2} - 1}  \bjk\sin(\omegaj \tn)\cos(\wavek \xm)
\continue
P^{++}\dufield = u^{++}(\conf,\zeit)  &=& \frac{4}{\sqrt{NM}} \sum_{k = 1}^{\frac{M}{2} - 1} \Big[ \frac{c_{\sss{{0}{k}}}}{2} + \sum_{j=1}^{\frac{N}{2}-1}\cjk\cos(\omegaj \tn)\sin(\wavek \xm)\Big]\continue
P^{+-}\dufield = u^{+-}(\conf,\zeit) &=& \frac{4}{\sqrt{NM}} \sum_{k = 1}^{\frac{M}{2} - 1} \djk \sin(\omegaj \tn)\sin(\wavek \xm)  \,.
\eea

It is useful to first derive the commutation relation between
the projection operators \refeq{e-D1C2operators} and the spatial differentiation operator.
\bea \label{D2C2projopderivx}
D_{\conf} P^{++} &=& \frac{1}{4}D_{\conf}(1 +\sigma  + \tau + \sigma \tau) \continue
                 &=& \frac{1}{4}(1 -\sigma  + \tau- \sigma \tau)D_{\conf} \continue
                 &=& P^{-+}D_{\conf} \continue
D_{\conf} P^{+-} &=& \frac{1}{4}D_{\conf}(1 + \sigma  - \tau- \sigma \tau) \continue
                 &=& \frac{1}{4}(1 -\sigma  - \tau+ \sigma \tau)D_{\conf} \continue
                 &=& P^{--}D_{\conf} \continue
D_{\conf} P^{-+} &=& \frac{1}{4}D_{\conf}(1 -\sigma  + \tau - \sigma \tau) \continue
                 &=& \frac{1}{4}(1 +\sigma + \tau + \sigma \tau)D_{\conf} \continue
                 &=& P^{++}D_{\conf} \continue
D_{\conf} P^{--} &=& \frac{1}{4}D_{\conf}(1 -\sigma  - \tau + \sigma \tau) \continue
                 &=& \frac{1}{4}(1 +\sigma  - \tau - \sigma \tau)D_{\conf} \continue
                 &=& P^{+-}D_{\conf}\,.
\eea

These relations are useful for the computation of the nonlinear term due to the lack
of commutivity. Note that the effect can be summarized by flipping the first $\pm$, pertaining
to the coefficient of the spatial reflection terms in \refeq{e-D1C2operators}. The
projections of the nonlinear term are
\bea \label{e-D1C2nonlinear}
P^{++}(u\partial_x u) &=& u^{\pm \pm}\partial_{\conf}(u^{\pm \pm})\continue
P^{+-}(u\partial_x u) &=& u^{\pm \pm}\partial_{\conf}(u^{\pm \mp})\continue
P^{-+}(u\partial_x u) &=& u^{\pm \pm}\partial_{\conf}(u^{\mp \pm})\continue
P^{--}(u\partial_x u) &=& u^{\pm \pm}\partial_{\conf}(u^{\mp \mp})\,.
\eea
Using these relations \refeq{e-D1C2nonlinear} we can produce the projections
of the \KSe\ onto the different irreducible subspaces, noting that the projection operator
commutes with the linear terms such that
\bea \label{e-KSEprojections}
P^{++}F(u) &=& u_{\zeit}^{++}+u_{\conf \conf}^{++}+u_{\conf \conf \conf \conf}^{++} \continue
           &\quad&+ \big[u^{++}\partial_{\conf}(u^{++}) + u^{+-}\partial_{\conf}(u^{+-}) \continue
           &\quad&+u^{-+}\partial_{\conf}(u^{-+}) + u^{--}\partial_{\conf}(u^{--})\big]  \continue
P^{+-}F(u) &=& u_{\zeit}^{+-}+u_{\conf \conf}^{+-}+u_{\conf \conf \conf \conf}^{+-}\continue
           &\quad&+\big[u^{++}\partial_{\conf}(u^{+-}) + u^{+-}\partial_{\conf}(u^{++}) \continue
           &\quad&+\,u^{-+}\partial_{\conf}(u^{--}) + u^{--}\partial_{\conf}(u^{-+})\big]  \continue
P^{-+}F(u) &=& u_{\zeit}^{-+}+u_{\conf \conf}^{-+}+u_{\conf \conf \conf \conf}^{-+}\continue
           &\quad&+\big[u^{++}\partial_{\conf}(u^{-+}) + u^{+-}\partial_{\conf}(u^{--}) \continue
           &\quad&+\,u^{-+}\partial_{\conf}(u^{++}) + u^{--}\partial_{\conf}(u^{+-})\big] \continue
P^{--}F(u) &=& u_{\zeit}^{--}+u_{\conf \conf}^{--}+u_{\conf \conf \conf \conf}^{--}\continue
           &\quad&+\big[u^{++}\partial_{\conf}(u^{--}) + u^{+-}\partial_{\conf}(u^{-+}) \continue
           &\quad&+\,u^{-+}\partial_{\conf}(u^{+-}) + u^{--}\partial_{\conf}(u^{++})\big]\,.
\eea
Solutions to \refeq{e-ks} satisfy $F = 0$ by definition so
by extension solutions must also satisfy $P^{\pm \pm}F=0$.
With this we can determine the combinations of projection operators whose equations
are ``self contained''. This is similar to the notion of \textit{flow invariant subspaces}
but because we do not have dynamics we can't really use this term. Instead,
these subspaces correspond to a constrained set of equations that solutions with
particular discrete symmetries must adhere to.
For example, assume that the only nonzero component $u$ is $u=u^{++}$.
Substitution of \refeq{e-KSEprojections} yields
\bea \label{e-KSplusplus}
P^{++}F(u^{++}) &=& u_{\zeit}^{++}+u_{\conf \conf}^{++}+u_{\conf \conf \conf \conf}^{++}
                +u^{++}\partial_{\conf}(u^{++}) \continue
P^{+-}F(u^{++}) &=& 0 \continue
P^{-+}F(u^{++}) &=& 0 \continue
P^{--}F(u^{++}) &=& 0 \,,
\eea
so $\bbU^{++}$ is
an invariant subspace. In fact,
this subspace
corresponds to equilibria solutions which
live on the $\period{}=0$ line. The meaning
of self contained in this example is that we
assumed that $u=u^{++}$ and the only nonzero part
of \refeq{e-KSplusplus} is the $P^{++}F(u^{++})$ component.
The rest of the symmetry invariant subspaces follow from a
similar substitutions. To expedite the derivation process, note
that the equation for $P^{++}F$ contains
all of the symmetric terms $u^{\pm \pm}\partial_{\conf}(u^{\pm \pm})$
such that invariant subspaces
must have non-empty intersection with $\bbU^{++}$.
Following a process of elimination we can show that the possible
symmetry invariant subspaces are $\bbU^{++}$, $\bbU^{++}\oplus \bbU^{--}$,
$\bbU^{++}\oplus \bbU^{+-}$ and $\bbU^{++}\oplus \bbU^{-+}$ and
of course the full space $\bbU$. There are no triplet subspaces
(comprised of three components) which can be shown using
the parity of the different subspaces. We can interpret
these subspaces by addition of the corresponding projection
operators \refeq{e-D1C2operators}
\bea \label{e-invariantoperators}
P_{0}\equiv P^{++} &=& \frac{1}{4}(1 +\sigma +\tau+ \sigma \tau) \continue
P_{\sigma}\equiv P^{++}+P^{+-} &=& \frac{1}{2}(1 + \sigma) \continue
P_{\tau}\equiv P^{++}+P^{-+} &=& \frac{1}{2}(1 + \tau) \continue
P_{\sigma \tau}\equiv P^{++}+P^{--} &=& \frac{1}{2}(1+\sigma \tau)
\,.
\eea
$\bbU^{++}$ represents the subspace of antisymmetric equilibria,
$\bbU^{++}\oplus \bbU^{+-}$ the spatial reflection invariant subspace,
$\bbU^{++}\oplus \bbU^{--}$ the shift-reflection invariant subspace,
and lastly $\bbU^{++}\oplus \bbU^{-+}$ which
contains solutions that are odd with respect to $\period{}/2$.
in time. Each subspace corresponds to a subgroup which is a representation of \Zn{2}.
The continuous spatial translations are eliminated from every subspace \textit{except}
$\bbU^{++}\oplus \bbU^{-+}$. Because every orbit is doubly-periodic by definition,
any orbit has a representation which exists in this subspace; created by simply taking
a tiling of size $[0, 2\period{}] \times [0, \speriod{}]$. The meaning behind this subspace
has not been explored yet; but it may be trivialized by the simplicity of the subgroup being examined.

As previously mentioned, the invariance condition \refeq{e-invariance} can be used to derive
constraints on which modes are non-vanishing, referred to as \textit{selection rules}.
The general procedure for producing these selection rules is straightforward, simply substitute
\refeq{e-rfft} and a specific symmetry operation into \refeq{e-invariance}. However,
for spatial reflection symmetry, it is much simpler to identify that only
which are antisymmetric with respect to $\speriod{}/2$ are allowed. This
implies that all terms with $\cos(\wavek\xm)$ must be identically equal to zero, such that the
precise \spt\ definition of antisymmetric orbits
\beq \label{e-reflrules}
\ajk, \bjk = 0 \quad \forall \quad k, j
\,.
\eeq
Note that equilibria trivially satisfy this as well.
Unfortunately the shift-reflection selection rules are not as simple.
The action of this symmetry is as follows $\sigma \tau u(x, t) = -u(\speriod{}-x, t + \period{}/2)$.
To avoid cluttering the derivation, the specifics of \refeq{e-rfft} that are unnecessary are not included simply
to avoid an overly cluttered derivation. Utilizing trigonometric identities and the parity of $\sin$ and $\cos$ we have
\bea \nonumber
\sigma \tau\,u(\tn, \xm) &=& -\sum_{k,j}\Bigg[\cos(\wavek (\speriod{}-\xm)) (\ajk\cos(\omegaj (\tn + \period{}/2)) + \bjk \sin(\omegaj (\tn + \period{}/2))) \continue
                         &\quad&+ \sin(\wavek(\speriod{}-\xm)) (\cjk\cos(\omegaj (\tn + \period{}/2)) + \djk\sin(\omegaj (\tn + \period{}/2)))\Bigg] \continue
                         &=& \sum_{k,j}\Bigg[ -\cos(\wavek \xm) (\ajk \cos(\omegaj \tn)\cos(\pi j) + \bjk \sin(\omegaj \tn)\cos(\pi j)))\continue
                         &\quad&+ \sin(\wavek \xm) (\cjk\cos(\omegaj \tn)\cos(\pi j) + \djk \sin(\omegaj \tn)\cos(\pi j))\Bigg] \continue
                         &=& \sum_{k,j}\Bigg[ (-1)^{j+1} \cos(\wavek \xm) (\ajk\cos(\omegaj \tn) + \bjk\sin(\omegaj \tn))\continue
                         &\quad&+ (-1)^{j} \sin(\wavek \xm) (\cjk \cos(\omegaj \tn) + \djk \sin(\omegaj \tn)) \Bigg]\, ,
\eea
substitution into the invariance condition \refeq{e-invariance} yields the following selection rules
we find that the selection rules for \spt\ shift-reflection are as follows
\bea \label{e-srrules}
\ajk,\bjk &=& 0 \, \mbox{ for }\, j \, \mbox{ even } \continue
\cjk,\djk &=& 0 \, \mbox{ for }\, j \, \mbox{ odd }
\,.
\eea

This concludes the discrete symmetries considered for individual orbits; however, this does
not exclude the possibility of creating orbits with more complex symmetries via the gluing process described in \refsect{sect:glue}.

For continuous symmetries there are two details that must be accounted for; orbits with relative periodicity and Galilean invariance.
The latter, invariance under Galilean transformations results in the following equivariance: if \ufield is a solution, then
$u(t, x - ct) - c$ is also a solution ($c$ being an arbitrary constant speed). The manner in which this symmetry is currently handled
is to constrain the modes with constant spatial modulation to zero. In other words, the mean flow condition is enforced
\beq
\spaceAver{u}(\zeit)
  \,=\, \int_0^{\speriod{}} d\conf \, u(\conf,\zeit) = 0
\,.
\ee{GalInv}
This is a conventional choice but not necessarily the best choice. This reduces
the \cdof\ but it may have a lingering effect later on when looking to glue orbits together which is
discussed in \refsect{sect:glue}.
The last consideration that we make regarding \spt\ symmetries concerns relative periodic orbits.
A relative periodic orbit is defined here as an orbit whose field \dufield is only
periodic after accounting for a `spatial drift' or `shift'. In other words, relative
periodic orbits abide by the following relation
\beq
\ufield = g \circ u(t + \period{}, x)\,.
\eeq
We need some way of accounting for this spatial shift and the corresponding
discontinuity in time be creating a truly periodic representation of the orbit.

Two ideas come to mind, either utilize a comoving reference frame or slice (quotient)
the continuous symmetry. Upon comparison there were a number of disadvantages to the
slicing method both numerically and theoretically. Only a comoving frame is ever used,
and so the derivation and argument against slicing is reserved for the appendix.

The real valued representation \refeq{e-rfft} that we employ complicates the derivations
slightly; the $\Un{1}$ generator of translations is diagonal but the \SOn{2} generator
is not. The comoving frame and its spatial shift is formulated in dimensionless spatial units
as opposed to an angular quantity. This is important because both $\period{}$ and $\speriod{}$ are
changing in the numerical optimization. The matrix representation of \SOn{2} 
group elements for a spatial shift by an amount $s$ for a single instant in time, $\tn$ more on this later are as follows 
\beq \label{e-SOnGroupElement}
\LieEl(s) \equiv
\begin{bmatrix}
\cos \wavek s & -\sin \wavek s\\
\sin \wavek s & \cos \wavek s
\end{bmatrix}\,.
\eeq
Due to the numerical arrangement of the array of modes, each matrix element in \refeq{e-SOnGroupElement} 
should be interpreted as a diagonal matrix such that the dimension of the entire matrix is $M-2 \times M-2$. 
This matrix applied to a set of spatial modes defined at a specific instant in time, shifts them by $s$
in the positive $x$ direction. The \spt\ version of this operator, that is, uniform spatial rotations 
of the entirety of \dufield, is a block diagonal matrix with $N$ blocks; one for each $\tn$. Both uniform spatial
translations and those corresponding to the change of reference frame are applied to the spatial modes, not the
spatiotemporal modes; mainly because this allows for a single, general formulation instead of a symmetry specific formulation.
For a uniform shift, this \spt\ operator can be generated by taking the Kronecker product of the
identity matrix with $N$ diagonal elements with \refeq{e-SOnGroupElement} such that
\beq \label{e-SOnspacetime}
\mathbf{\LieEl}(s)  = \mathbb{I}_{N} \otimes \LieEl(s)
\eeq
which, explicitly, equals
\beq \label{e-SOnspacetimematrep}
\mathbf{\LieEl}(s) \equiv
\begin{bmatrix}
\LieEl(s) & 0 & \cdots & 0 \\
0 & \LieEl(s) & \cdots & 0 \\
\vdots & \vdots & \ddots & \vdots \\
 0 & 0 & 0 & \LieEl(s)
\end{bmatrix}
\,.
\eeq
The bold face indicates that \refeq{e-SOnspacetimematrep} is the formulation of the \spt\ operator.
The co-moving reference frame is a generalization of \refeq{e-SOnspacetimematrep} where the previously
uniform shift $s$ is replaced by a spatial shift linearly parameterized by time. That is, at every discrete
time $\tn$, the field is shifted by an amount $\LieEl(\frac{s \tn}{\period{}})$. 
Using \refeq{e-SOnGroupElement} the matrix representation
of the co-moving frame transformation is as follows
\beq \label{e-comovingmatrix}
\LieEl(s) \equiv
\begin{bmatrix}
\LieEl(\frac{s t_{\scalebox{.4}{$N-1$}}}{\period{}}) & 0 & \cdots & 0 \\
0 & \LieEl(\frac{s t_{\scalebox{.4}{$N-2$}}}{\period{}}) & \cdots & 0 \\
\vdots & \vdots & \ddots & \vdots \\
 0 & 0 & 0 & \LieEl(\frac{s t_{0}}{\period{}})
\end{bmatrix}
\,.
\eeq
Transformations of the type \refeq{e-comovingmatrix} will be used in our ansatz for doubly periodic solutions of
the \KSe\ which are relatively periodic. An important numerical detail to highlight is that the reference frame transformation
can be applied in a matrix free manner, the operator \refeq{e-comovingmatrix} need not be constructed explicitly.

The general idea is akin to L{\'o}pez \etal\rf{lop05rel} where the approximate solution or initial condition is shifted
into the co-moving frame to first make the initial condition truly periodic and then the inverse shift
is kept track of so that the ``true" solution (the one in the initial frame) can be recomputed after
convergence. If I've understood everything this should allow for the following equations, note the very
similar rescaling by the inverse of the diagonal operator containing the second and fourth powers of the wavenumber.

Due to the way that the comoving transformation is formulated, the ansatz for relative periodic orbits takes the following form
This shall be referred to hereafter as the \emph{co-moving frame ansatz}. A key numerical detail is that \dufield is only ever stored
in the comoving reference frame. The way that the comoving frame is accounted for is by an additional linear term which results from
the time derivative of the ansatz. 


This can be written into \refeq{e-rfft} 
\beq \label{e-ansatz}
\begin{split}
u(\tn, \xm) = \sum_{k} \LieEl\bigg(\frac{\sigma \tn}{\period{}}\bigg) \sum_{j} & \ajk \cos(\omegaj \tn)\cos(\wavek \xm) + \bjk \sin(\omegaj \tn)\cos(\wavek \xm) \continue
                                &+\cjk \sin(\wavek \xm)\cos(\omegaj \tn) + \djk \sin(\wavek \xm)\sin(\omegaj \tn)\,.
\dufield&=&\frac{4}{\sqrt{NM}}\sum_{k = 1}^{\frac{M}{2} - 1} \Bigg[\frac{a_{\sss{{0}{k}}}}{2} + \sum_{j=1}^{\frac{N}{2}-1}\ajk\cos(\omegaj \tn)+ \bjk\sin(\omegaj \tn)\Bigg]\cos(\wavek \xm) \continue
&-&\frac{4}{\sqrt{NM}} \sum_{k = 1}^{\frac{M}{2} - 1} \Bigg[\frac{c_{\sss{{0}{k}}}}{2} + \sum_{j=1}^{\frac{N}{2}-1}\cjk\cos(\omegaj \tn)+ \djk\sin(\omegaj \tn)\Bigg]\sin(\wavek \xm) \,.
\end{split}
\eeq

The one trade-off of this method is increasing the dimensionality of the problem by one, this is because of the extra
parameter keeping track of the shift.

\bea \label{eqn:rpo_spacetime_reform}
G(\Fu, T, L) &\equiv& D_X^{-1}((D_t + S) \Fu + D_x F (F^{-1} \Fu)^2 ) - \Fu = 0
    \continue
D_X &\equiv& D_{xx} - D_{xxxx}
    \continue
S   &\equiv& diag(\frac{-i m \ell}{T})
\eea

where, the definition of the operator $S$ comes naturally from the definition of the shift required to transform from the
co-moving frame to the initial frame, $\hat{\Fu} = e^{\frac{-i m \ell_p t}{T}}\Fu$.
I'm cheating for the sake of simplicity in typing the equations because
I'm actually using the real-valued representation where the operator is actually an off diagonal \SOn{2} type matrix.

With the concept of a very small number of active spatiotemporal modes, usually due to discrete symmetries
I applied this in a very rough way to test the number of active modes in one of the shortest orbits (in time)
of the HKW computation cell, also known to channelflow users as "p19p02".

I first took $M=16$ points in time (16 snapshots) of the shortest periodic
orbit whose spatial discretization requires $N_x = 32 \times N_y = 49 \times N_z = 32 \times 3 = 150528 \equiv dim(\mathcal{M}_{xyz})$ physical
space variables to describe. The total number of variables in the spatiotemporal discretization is then $150528*16 = 2408448$.

To test the number of "active modes" spatiotemporally, I used real-valued FFTs in the spanwise and streamwise direction, and
the discrete cosine transform (which I think is equivalent to the Chebyshev coordinates but I could be drastically wrong) in the
wall normal direction. To describe the number of active modes I counted the number of spatiotemporal coefficients,
now $(Fourier \times Fourier \times Chebyshev) \times Fourier$ in three directions,
had a absolute value (not the square) that was greater than $10^{-14}$.

I claim that in the shortest periodic orbit 'p19p02' is
$N_{active} = 564477$. In other words, approximately seventy-five percent of the spatiotemporal information is redundant/not used.
Therefore, for a full spatiotemporal description using 16 points in time in this basis, the number of variables used
to describe it only increases by a factor of $3.74998007015$. This is a nontrivial increase in dimensionality,
but much less than the expected multiple of $16$.

For $M=32$ the reduction percentage increases, but so do the number of modes corresponding to time. The multiple for
$M=32$ was $7.26536591199$,  which is slightly less than two times the $M=16$ multiple. I believe that this is due
to extra modes in time which themselves are non-active.

I still need to test this on a case that has non-zero shift, but in order to exploit these methods I need a truly periodic
orbit not a fundamental domain, so the number of points in time would have to increase by two, and I am unclear whether
it would be as beneficial or even better.
I'm indebted to Roman because even though I've scoured the literature for a
matrix free representation of the product $\transp{J} x$, he pointed out that
Ravi did this, and after some thought I realized that if I can explicitly
form the \jacobianM, then reverse engineering what this matrix vector product
might be possible. The problem lies in the fact that Ravi used finite
differences to define derivatives so that the product with the transpose is
much more straightforward.

$G$, the group of actions $ g \in G $ on a
\statesp\ (reflections, translations, \etc) is a symmetry of the KS
flow \refeq{e-ks} if $g\,u_t = F(g\,u)$.
The \KSe\ is time translationally invariant, and space translationally invariant
on a periodic domain under
the 1-parameter group of
$\On{2}: \{\Shift_{\shift/\speriod{}},\Refl \}$.
If $u(\conf,\zeit)$ is a solution, then
$\Shift_{\shift/\speriod{}}\, u(\conf,\zeit) = u(x+\shift,t)$
is an equivalent solution for any shift
$-\speriod{}/2 < \shift \leq \speriod{}/2$,
as is the
reflection (`parity' or `inversion')
\beq
    \Refl \, u(x) = -u(-x)
\,.
\ee{SCD07:KSparity}
The translation operator action on the Fourier coefficients \refeq{eq:ksexp},
represented here by a complex valued vector
$a = \{a_k\in\mathbb{C}\,|\,k = 1, 2, \ldots\}$, is given by
\beq
  \Shift_{\shift/\speriod{}}\, a = \mathbf{g}(\shift) \, a \,,
  \label{eq:shiftFour}
\eeq
where $\mathbf{g}(\shift) = \mbox{diag}( e^{i q_k\, \shift} )$ is a complex
valued diagonal matrix, which amounts to the $k$-th mode complex plane
rotation by an angle $k\, \shift /\tildeL$.  The reflection acts on
the Fourier coefficients by complex conjugation,


  Let $\bbU$ be the space of
real-valued velocity fields periodic and square integrable
on the interval $\Omega = [-\speriod{}/2,\speriod{}/2]$,
\begin{align}
 \bbU  &= \{u \in \speriod{}^2(\Omega) \; | \; u(x) = u(x+\speriod{})\}  \,.
\end{align}
A continuous symmetry maps each state $u \in \bbU$
to a manifold of functions with identical dynamic behavior.

but otherwise the nonlinear terms in \refeq{SCD07:KSD1}
mix the two subspaces.

Any rational shift $ \Shift_{1/m}u(x)=u(x+\speriod{}/m)$ generates a discrete
cyclic subgroup $\Cn{m}$ of $\On{2}$, also a symmetry of \KSe.

While in general the bilinear term $(u^2)_x$  mixes the
irreducible subspaces of $\Dn{n}$, for $\Dn{2}$ there are
four subspaces invariant under the flow\rf{KNSks90}:
%\begin{itemize} %{romannum}

With the continuous
translational symmetry eliminated within each subspace, there are no
\reqva\ and \rpo s, and one
can focus on the \eqva\ and \po s only, as was done
for $\bbU^+$ in \refrefs{Christiansen97,LanThesis,lanCvit07}.
In the Fourier
representation, the
$u \in \bbU^+$
antisymmetry amounts to having purely imaginary
coefficients, since $a_{-k}= a^\ast_k = -a_k$.
The 1/2 cell-size shift $\Shift_{1/2}$
generated 2-element discrete subgroup
$\{1,\Shift_{1/2}\}$ is
of particular interest
because in the $\bbU^+$ subspace the translational invariance of the full system reduces to
invariance under discrete translation \refeq{KSshift} by half a
spatial period $\speriod{}/2$.

Each of the above dynamically invariant subspaces is unstable
under small perturbations, and generic solutions of \KSe\ belong to
the full space.
Nevertheless, since  all \eqva\ of the KS flow studied in \refref{SCD07}
lie in the $\bbU^+$ subspace, $\bbU^+$  plays important role for the global
geometry of the flow.
However, linear stability of these \eqva\ has
eigenvectors both in and outside of $\bbU^+$, and needs to be
computed in the full \statesp.
