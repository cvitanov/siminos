In light of all of these difficulties we believe that new, bold ideas are
required to resume forward progress. Specifically, we have begun a completely
{\spt} formulation of chaos which treats all continuous dimensions democratically.
The main idea is to discard the idea of a dynamical system completely; exponential
instabilities mean that conventional methods never could have worked. By converting to
a truly {\spt} formulation we have discarded dynamics and the inherent difficulties therein.
This allows us to quantify and characterize infinite space-time via shadowing
of fundamental {\spt} patterns.

The primary claim that we make is that in hindsight, describing turbulence
via an exponentially unstable dynamical equation never could have worked.
Conventional methods treat spatial dimensions
as finite and fixed and time as inherently infinite.
Our {\spt} formulation of chaos treats all continuous dimensions with translational
invariance democratically as $(1+D)$ different `times'.
The proposal is inspired by the Gutkin and Osipov\rf{GutOsi15}
modelling of chain of $N$ coupled particle by temporal evolution of a
lattice of $N$ coupled cat maps.

The alternative that we propose to describe infinite space-time chaos via
the shadowing of fundamental patterns which we refer to as ``tiles''.
These tiles are the minimal ``building blocks'' of turbulence; they are realized
as \twots\ which are global solutions with compact support.
Finding the tiles of turbulence is fundamentally easier than finding
\twots\ on larger domains due to the exponential growth in complexity of
solutions. In other words there are fewer important solutions on smaller
domains. This in turn implies that there can only be a small
number of fundamental tiles. This is what makes the problem tractable:
if we can collect the complete set of tiles then we have the ability to
construct every \twot\ according to our theory.

The lack of exponentially unstable dynamics has powerful and immediate effects.
Because there is no time integration, the problem of finding \twots\ is
now a variational one. The benefit of this is that there is no need to start
an initial guess on the attractor; the optimization process handles this
entirely. This allows us to find arbitrarily sized \twots but in fact
there is no need to. Our hypothesis is that we need only to find the
building blocks which shadow larger \twots\ and infinite space-time.

The {\spt} formulation allows a much easier categorization of what is
``fundamental'' by virtue of the frequency that patterns admit in the
collection of \twots. By identifying the most frequent patterns, we shall
clip these patterns out of the \twots\ they shadow and use them as initial
conditions to search for our tiles.

The notion of ``building blocks of turbulence'' is one of the reasons for
studying fluid flows in the first place. There is evidence that certain
physical processes are fundamental, but they have yet to be used in a
constructive manner. The {\spt} description is able to actually put
these ideas in practice.
The \spt\ completely avoids this by constructing larger \twots\
from the combination of smaller \twots.
The reason
why the search for the fundamental tiles is classified as ``easy'' is because
in the small domain size limit there just aren't that many \twots; the dynamics
is relatively simple.

The first key difference is that the governing equation
dictates the \spt\ domain size in an unsupervised
fashion. The results here are not
The only reason why $L$ was treated as fixed is due to the
inherent instability it includes when treated as a varying quantity.
This small detail, allowing the domain size $L$ to vary,
is not as trivial as it seems.
 This difficulty
is especially evident in the \KSe, whose spatial derivative terms
are of higher order than the first order time derivative, but also
there is a spatial derivative present in the nonlinear component.

Specifically, we propose to study the evolution of \KS\ on the $2$\dmn\
infinite {\spt}domain and develop a $2$\dmn\ symbolic dynamics for it:
the columns coding admissible time itineraries, and rows coding the
admissible spatial profiles. Our {\spt} method is the clear winner in
both a computational and theoretical sense. By converting to a tile based
shadowing description we have essentially removed the confounding notion
of an infinite number of infinitely complex {\twots} from the discussion.
Now we must put these ideas into practice.

The testing grounds for these ideas will be the \spt\ \KSe

\beq \label{e-ks}
u_t + u_{xx} + u_{xxxx} + u u_x = 0 \quad \mbox{where} \quad x\in[0,\speriod{}], t\in[0,\period{}]
\eeq
where $u = u(x, t)$ represents a \spt\ velocity field. This
equation has been used to model many different processes such as
the laminar flame front velocity of Bunsen burners.
While \refeq{e-ks} is much simpler than the {\spt} Navier-Stokes equation,
we would argue that the main benefit is the simplicity of
visualizing its two-dimensional space-time. This visualization
makes the arguments more understandable and compelling in addition
to making the tiles easier to identify.

The plans for our {\spt} formulation have been laid bare. The main concept is that the infinities
of turbulence can be described by {\spt} symbolic dynamics whose letters are fundamental {\spt}
patterns. Consequentially, we have created numerical methods which not only perform better than
conventional methods but also present incredible newfound capabilities. These newfound capabilities
include but are not limited to finding small \twots\ which shadow larger \twots\ but also constructing
larger \twots\ from smaller ones. These new and robust methods alone present a way forward for turbulence research, hence their is merit in a {\spt} formulation even though the theory has not
been fully fleshed out.

To test our \spt\ ideas we require three separate numerical methods: the first
should be able to find \twots\ of arbitrary domain size. The second needs to be able to
clip or extract tiles from these \twots. Lastly, we need a method of
gluing these tiles together. All three of these techniques require the ability
to solve the optimization problem $F(\Fu,\speriod{},\period{})=0$
on an arbitrarily sized doubly periodic domain.

The primary claim that we make is that in hindsight, describing turbulence
via an exponentially unstable dynamical equation never could have worked.
Conventional methods treat spatial dimensions
as finite and fixed and time as inherently infinite.
Our {\spt} formulation of chaos treats all continuous dimensions with translational
invariance democratically as $(1+D)$ different `times'.
The proposal is inspired by the Gutkin and Osipov\rf{GutOsi15}
modelling of chain of $N$ coupled particle by temporal evolution of a
lattice of $N$ coupled cat maps.

%\subsubsection{characterize and quantify all patterns}
The alternative that we propose to describe infinite space-time chaos via
the shadowing of fundamental patterns which we refer to as ``tiles''.
These tiles are the minimal ``building blocks'' of turbulence; they are realized
as \twots\ which are global solutions with compact support.
Finding the tiles of turbulence is fundamentally easier than finding
\twots\ on larger domains due to the exponential growth in complexity of
solutions. In other words there are fewer important solutions on smaller
domains. This in turn implies that there can only be a small
number of fundamental tiles. This is what makes the problem tractable:
if we can collect the complete set of tiles then we have the ability to
construct every \twot\ according to our theory.

%\subsubsection{no instabilities}
The lack of exponentially unstable dynamics has powerful and immediate effects.
Because there is no time integration, the problem of finding \twots\ is
now a variational one. The benefit of this is that there is no need to start
an initial guess on the attractor; the optimization process handles this
entirely. This allows us to find arbitrarily sized \twots but in fact
there is no need to. Our hypothesis is that we need only to find the
building blocks which shadow larger \twots\ and infinite space-time.

%\subsubsection{Large to small}
The {\spt} formulation allows a much easier categorization of what is
``fundamental'' by virtue of the frequency that patterns admit in the
collection of \twots. By identifying the most frequent patterns, we shall
clip these patterns out of the \twots\ they shadow and use them as initial
conditions to search for our tiles.
%\subsubsection{small to large}
The notion of ``building blocks of turbulence'' is one of the reasons for
studying fluid flows in the first place. There is evidence that certain
physical processes are fundamental, but they have yet to be used in a
constructive manner. The {\spt} description is able to actually put
these ideas in practice.
The \spt\ completely avoids this by constructing larger \twots\
from the combination of smaller \twots.
The reason
why the search for the fundamental tiles is classified as ``easy'' is because
in the small domain size limit there just aren't that many \twots; the dynamics
is relatively simple.

%\subsubsection{numerical benefits?}
The first key difference is that the governing equation
dictates the \spt\ domain size in an unsupervised
fashion. The results here are not
The only reason why $L$ was treated as fixed is due to the
inherent instability it includes when treated as a varying quantity.
This small detail, allowing the domain size $L$ to vary,
is not as trivial as it seems.
 This difficulty
is especially evident in the \KSe, whose spatial derivative terms
are of higher order than the first order time derivative, but also
there is a spatial derivative present in the nonlinear component.



Once the collection is deemed sufficient we proceed to visual inspection. In this manner
we determine the most frequent patterns and single them out as tile candidates. This is
done by literally clipping them out of the \twots\ that they shadow. Each clipping is
then treated as an initial guess for a fundamental tile which is itself a \twot. Therefore,
these represent initial conditions for the optimization method. It is not a guarantee
that every clipping converges to a \twot; therefore the number of attempts to find a tile
should continue until it does in fact converge. The number of convergence attempts is typically
proportional to how confident we are that the pattern being scrutinized is in fact a tile.

Once a collection of tiles is collected, we can construct new and reproduce known \twots.
This is completed with a method we refer to as ``gluing''. It is as straightforward as one
might infer: tiles are combined in a {\spt} array to form initial conditions used to find larger
\twots. Methods of gluing temporal sequences of \twots\ exist but never has the ability to
glue \twots\ spatiotemporally existed before.

With the implementation of the gluing method can begin to
probe the 2\dmn\ {\spt} symbolic dynamics
previously mentioned. A fully determined symbolic dynamics is sufficient
to describe infinite space-time completely.
We already have the two edges of this symbol plane - the $\speriod{}=22$ minimal
cell\rf{SCD07,lanCvit07} is sufficiently small that we can think of it as
a low-dimensional (``few-body'' in Gutkin and Klaus
Richter\rf{EPUR14,EDASRU14,EnUrRi15,EDUR15} condensed matter parlance)
dynamical system, the left-most column in the Gutkin and
Osipov\rf{GutOsi15} $2D$ symbolic dynamics {\spt} table (not a
1\dmn\ symbol sequence block), a column whose temporal symbolic dynamics
we will know, sooner or later. Michelson\rf{Mks86} has described the
bottom row. The remainder of the theory will be developed from the
bottom up, starting with small {\spt} blocks.



Therefore we offer a new {\spt} formulation of chaos which
treats all dimensions with continuous symmetries democratically as $(D+1)$ different `times'.
This equal treatment of space and time removes the requirement for
spatial dimensions to be finite or fixed. In other words,
the role of spatial periods is now identical to that of time periods
in the sense that they are
variables determined by the governing equations.
    \PCedit{ % 2020-05-07
`Tiling' in the title of this paper refers to our attempt to
systematically triangulate this set in terms of dynamically invariant
solutions (\eqva, \po s, $\ldots$), in a PDE representation and numerical
simulation algorithm independent way.
    }

Within this framework there
are no longer any dynamics; instead, solutions are {\spt} combinations of
$(D+1)$ invariant tori (to which we shall henceforth refer to simply as \emph{{\po}s}).

While there are an infinite number of {\po}s our theory only utilizes a small
number of very important ones. We denote these special orbits as {\fpo}s and claim
that they are the long sought after ``building blocks''
of turbulence; that is, every solution can be described as a collection of {\fpo}s.

The goal is to collect, enumerate and utilize all {\fpo}s.
The collection of {\fpo}s proceeds in the following manner:
identify the most frequently recurring patterns in a collection of {\po}s,
extract and use said patterns as initial guesses for {\fpo}s. Taking the
unique results from this search establishes a finite collection or library
of {\fpo}s.
All {\po}s can then be constructed from a complete collection of {\fpo}s; we
aim at least collecting the most important {\fpo}s for now.
