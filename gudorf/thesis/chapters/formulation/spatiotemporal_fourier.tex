The \spt\ theory developed here focuses solely on the \KSe, however, the first definition
provided will be a general one. For any governing equation, the application of this theory is
predicated upon the existence of periodic orbits (`orbit' will be
used for brevity, as everything is assumed to at least approximate a periodic orbit in this formulation).
Therefore, the first step is to define what an `orbit' is exactly. Let $\mathbf{f}$ represent a partial differential
equation (vector or scalar, going with boldface to indicate vector) with all terms
grouped on one side such that $\mathbf{f}(\statev)=0$; this will
be referred to as the \textit{governing equation}. The variable \statev represents the `state-vector',
which contains all relevant independent variables; it will be defined shortly. Most of the components of \statev\ correspond
to the field $u$, typically a velocity field (scalar or vector), is defined on a $d+1$-dimensional space-time configuration manifold $\mathcal{M}$.
Assuming translational invariance, $\mathcal{M}$ can be defined by the Cartesian product of intervals of real numbers
\bea
\mathcal{M} &=& [0, \period{}] \times \prod_{i}^{d} [0, \speriod{}_i] \continue
            &=& \{(t, \mathbf{x})=(t, x_1,...,x_d)\:|\: x_i \in [0, L_i],\: \forall i=\{1, ..., d\}\}\,.
\eea
Periodic orbits are then defined as differentiable fields $\mathbf{u}(t, \mathbf{x})$ which satisfy
periodic boundary conditions in time $\mathbf{u}(t, \mathbf{x}) = \mathbf{u}(t+T, \mathbf{x})$.
Spatial boundary conditions are not necessarily periodic, however, they are assumed to be compact after
quotienting the translational symmetries.
Therefore, a periodic orbit, or simply \textit{orbit} is
uniquely identified by the tuple $(\mathcal{M}, u(t, \mathbf{x}), f)$. In order these are: the configuration manifold $\mathcal{M}$,
the velocity field $\mathbf{u}(t, \mathbf{x})$ defined on $\mathcal{M}$, and the governing equation $f$ for which
$\mathbf{f}(\mathbf{u})=0$ and $\mathbf{u}(t, \mathbf{x}) = \mathbf{u}(t+T, \mathbf{x})$ are satisfied.

For the numerical implementation all variables need to
be stored in a single vector which shall be denoted as the \textit{state vector}. Providing a state
vector and specifying any symmetries is sufficient to define an orbit.
The vector representation of the orbits \refeq{e-sptorbit} in the Fourier basis is
\beq \label{e-statevector}
\statev \equiv [\uvec, \mathbf{p}]^{\top} \,.
\eeq
Parameters which
are constrained or non-applicable can simply be removed from the state vector \refeq{e-statevector}
With this in mind, the following terminology is introduced. The first, the configuration manifold $\mathcal{M}$ shall be referred
to as the \spt\ tile or simply \textit{tile} of an orbit. The dimensions of a tile, or \textit{tile dimensions} are simply the
values $(T, L_1, L_2, ...)$. The field of an orbit refers to $\mathbf{u}(t, \mathbf{x})$. The set of all periodic orbits
of an equation $f$ is defined as
\beq \label{e-sptorbit}
\mathbb{O}(\mathcal{M}, f, \mathbf{u}) \equiv \{\mathbf{u}(t, \mathbf{x})\: | \: \mathbf{f}(\statev)=0\quad\mbox{and}\quad \mathbf{u}(t, \mathbf{x})=\mathbf{u}(t+T, \mathbf{x})\quad \forall (t, \mathbf{x})\in \mathcal{M}\}
\eeq
The set of orbits \refeq{e-sptorbit} is defined in this manner because it is befitting of the object-oriented programming used
for this project, \texttt{orbithunter} on Github. In other words, the properties of the abstract set \refeq{e-sptorbit} are captured
by the `Orbit' computational class.

As this study only focuses on the \KSe\ and its orbits, \refeq{e-sptorbit} will simply be referenced by $\mathbb{O}$.

Motivated by the translational invariance, periodic boundary conditions, and smoothness,
a Fourier-Fourier basis was a natural choice for the numerical representation of \ufield.
In the words of \rf{boyd01}:
\begin{quote}
\textit{`The general rule is: Geometry chooses the basis set. The engineer never has to make a choice.' -- John P. Boyd}
\end{quote}

The first step towards such a description is to discretize \refeq{e-sptorbit}. Specifically,
the \twot\ defined by the tile $\mathcal{M}$ is discretized via the two-dimensional lattice (alternatively, grid or mesh)
of space-time points $\mathcal{M}_{nm} = (\tn, \xm)$ where
\bea \label{e-dtile}
\tn &=& \frac{n T}{N}, \quad n \in 0, 1, \dots, N-1 \continue
\xm &=& \frac{m L}{M}, \quad m \in 0, 1, \dots, M-1 \,,
\eea
which has discrete periods $N$ in time and $M$ in space. For brevity, discretized tiles will also
be referred to simply as tiles; it should be fairly obvious whether the context is continuous or discrete.
In the discrete setting, a `solution' is now a discretized field \dufield\ such that at every point on
the tile \dufield\ solves \refeq{e-ks} locally; or at least this would be the case if we remained in the physical
field basis. The Fourier basis, however, is an inherently
global description of space-time and \refeq{e-ks} as the field is expanded in terms global, periodic basis functions.
The tile now represents the set of \textit{interpolation points} or \textit{collocation points} where the residual function must
vanish; this is getting ahead of the current discussion, however, and so it is left to \refsect{sect:spatiotemporal_mapping}.



These points are commonly referred to as collocation points, and this type of formulation
can be classified as a \textit{collocation method} or \textit{pseudospectral method}.
The benefits of pseudospectral methods are numerous; most importantly they increase the accuracy of
our calculations while also minimizing the amount of computational memory required \rf{boyd01}.

The inherently infinitely dimensional equations are approximated
by a Galerkin truncation of these \spt\ Fourier modes.

Instead of using an overly verbose and repetitive verbiage for the coefficients of the Fourier basis function, i.e.
\textit{spatiotemporal Fourier modes} or \textit{spatiotemporal Fourier coefficients};they will simply be referred to as \textit{modes},
unless specifically mentioned otherwise. This avoids
excessive usage of `\textit{spatiotemporal}' and `\textit{Fourier}'; nearly everything from here on out is spatiotemporal. Likewise,
\textit{discrete \spt\ \rv\ Fourier transform} (it can get even worse) will be cut short to either \textit{Fourier transform} or a
reference to a computational function such as \texttt{rfft} from the SciPy numerical package.

The computational transforms are computed as
\bea
&\tilde{u}_k(\tn)& = \frac{1}{\sqrt{M}}\sum_{m = 0}^{M - 1} \dufield e^{-i(2\pi k m/M)}\continue
&\dufield& = \frac{1}{\sqrt{M}}\sum_{k = 1}^{M/2-1}\tilde{u}_k(\tn)  e^{i(2\pi k m/M)}
\eea
The convention that they utilize regarding the signs of the exponents are cause for confusion
upon reformulation in the real-valued basis. 

After the definition of tiles in place one would think that the definition of the Fourier
transforms would come very naturally, which is true until some numerical details are considered.
These details result in less than perfect expressions for the Fourier transforms, but
they are written such that evaluating the expressions on paper will yield the exact same result as that described in \refsect{sect:orbithunter}.
The first detail is that a real-valued (\SOn{2}) representation of the modes is required
(or at very least, greatly beneficial) for this formulation.
The first reason behind the choice of is numerical in nature; if all \cdof\ are complex numbers, then constraints
are needed to ensure the dimensions \tile\ are maintained as real numbers.
The second reason is that expanding in sines and cosines turns out to be very helpful
for orbits with discrete \spt\ symmetries, due to the parity of said functions.
For real valued input the negative frequency and positive frequency modes of the Fourier transform are related by conjugation relations.
Due to this conjugation relation and the previously described discretization,
the Fourier transform takes the form of a truncated Fourier series, whose summation
ranges over the frequencies $\wavek, \omegaj$
\bea \label{e-sptfreq}
\wavek &=& \frac{-2\pi k}{\speriod{}} \quad \mbox{where, }\: k \,\in 1, \dots \frac{M}{2}-1 \continue
\omegaj &=&  \frac{-2\pi j}{\speriod{}} \quad \mbox{where, }\: j \,\in 0, \dots \frac{N}{2}-1
\eea


It is helpful to derive the spatiotemporal transform via the
composition of one-dimensional spatial and temporal transforms. Again, it should be stressed that
in the derivations that follow \textit{everything is written to agree with output of the computational codes}.
This can lead to some expressions that look unsimplified, but the idea is that calculating the modes by hand would
be exactly equivalent. Let $\wavek = -\wavek$ and $\omegaj=-\omegaj$, the numerically implemented
spatial transform (and its inverse) for real valued input take the following form
\bea \label{e-spacerfft}
e_{k}(\tn) &=&  \frac{1}{\sqrt{M}}\sum_{m = 0}^{M - 1} \dufield  \cos(\wavek\xm) \continue
f_{k}(\tn) &=&  \frac{1}{\sqrt{M}}\sum_{m = 0}^{M - 1} \dufield  \sin(\wavek\xm)  \continue
\dufield &=& \frac{2}{\sqrt{M}} \sum_{k = 1}^{\frac{M}{2} - 1} e_{k}(\tn)\cos(\wavek\xm) + f_k(\tn)\sin(\wavek\xm)\,.
\eea
The $k\in \{0,\frac{M}{2}\}$ spatial modes are constrained to 0; hence why they are not included in \refeq{e-spatialrfft}.
The factor of two in the inverse transform expression is a result of keeping only the positive half of the spectrum.
The constraint on the zeroth mode is justified by invariance to Galilean transformations,
discussed in \refsect{sect:symmetry}.
The constraint on the Nyquist mode is a numerical choice; numerically, this mode is returned as the sum
of the corresponding negative and positive modes.
This, in combination with the conjugation relation, means that only the real component (or twice thereof, technically) is returned.
Because of this behavior, and the fact that the magnitude is typically small for sufficiently large discretizations,
this mode is excluded from calculations (i.e. discarded after both space and time transforms).

The spatiotemporal transform follows by taking the temporal transform of the time dependent spatial modes $e_k(\tn), f_k(\tn)$ of \refeq{e-spatialrfft}.
Due to the numerical implementation, time is technically parameterized as $\period{}-\tn$, as the first row corresponds to $t = T$ and
the last row, $t=0$. The expression \refeq{e-spatialrfft} is still correct,
but now that the time ordering matters, the time transform and its inverse take the form
\bea \label{e-timerfftdown}
\ajk &=& \frac{1}{\sqrt{N}}\sum_{n=0}^{N-1} e_k(\period{}-\tn) \cos(\omegaj\tn) \continue
\bjk &=& \frac{1}{\sqrt{N}}\sum_{n=0}^{N-1} e_k(\period{}-\tn) \sin(\omegaj\tn) \continue
\cjk &=& \frac{1}{\sqrt{N}}\sum_{n=0}^{N-1} f_k(\period{}-\tn) \cos(\omegaj\tn) \continue
\djk &=& \frac{1}{\sqrt{N}}\sum_{n=0}^{N-1} f_k(\period{}-\tn) \sin(\omegaj\tn) \continue
e_k(\period{}-\tn) &=& \frac{2}{\sqrt{N}}\Big(\frac{\azk}{2} + \sum_{j=1}^{\frac{N}{2}-1} \ajk\cos(\omegaj \tn) + \bjk\sin(\omegaj \tn)\Big) \continue
f_k(\period{}-\tn) &=& \frac{2}{\sqrt{N}}\Big(\frac{\czk}{2} +\sum_{j=1}^{\frac{N}{2}-1} \cjk\cos(\omegaj \tn) + \djk\sin(\tilde{} \tn)\Big) \;.
\eea

\begin{alignat}{2}
\label{e-rfftdown}
\ajk=\frac{1}{\sqrt{NM}}&\sum_{n=0}^{N-1} &\sum_{m = 0}^{M - 1}& u(\period{}-\tn, \xm)  \cos(\wavek \xm) \cos(\omegaj \tn) \continue
\bjk=\frac{1}{\sqrt{NM}}&\sum_{n=0}^{N-1} &\sum_{m = 0}^{M - 1}& u(\period{}-\tn, \xm)  \cos(\wavek \xm) \sin(\omegaj \tn) \continue
\cjk=\frac{1}{\sqrt{NM}}&\sum_{n=0}^{N-1} &\sum_{m = 0}^{M - 1}&u(\period{}-\tn, \xm) \sin(\wavek \xm) \cos(\omegaj \tn) \continue
\djk=\frac{1}{\sqrt{NM}}&\sum_{n=0}^{N-1} &\sum_{m = 0}^{M - 1}&u(\period{}-\tn, \xm) \sin(\wavek \xm) \sin(\omegaj \tn) \continue
u(T-\tn, \xm)=\frac{4}{\sqrt{NM}}&\sum_{k = 1}^{\frac{M}{2} - 1}&\Bigg[&\bigg(\frac{\azk}{2} + \sum_{j=1}^{\frac{N}{2}-1}\ajk\cos(\omegaj \tn)+\bjk\sin(\omegaj\tn)\bigg)\cos(\wavek \xm) \continue
&&+&\bigg(\frac{\czk}{2} + \sum_{j=1}^{\frac{N}{2}-1}\cjk\cos(\omegaj\tn)+\djk\sin(\omegaj\tn)\bigg)\sin(\wavek \xm)\Bigg] \,.
\end{alignat}
This is a technical detail but it is relevant when taking time derivatives; if
overlooked then the time derivative would be off by a negative sign.
Due to periodicity in time \refeq{e-timerfftdown} can be rewritten to yield the expressions for the increasing-time parameterization as
\bea \label{e-timerfft}
e_k(\tn) &=& \frac{2}{\sqrt{N}} \Big(\frac{\azk}{2} + \sum_{j=1}^{\frac{N}{2}-1}\ajk\cos(-\omegaj \tn) + \bjk\sin(-\omegaj \tn)\Big) \continue
f_k(\tn) &=& \frac{2}{\sqrt{N}} \Big(\frac{\czk}{2} + \sum_{j=1}^{\frac{N}{2}-1}\cjk\cos(-\omegaj \tn) + \djk\sin(-\omegaj \tn)\Big) \;.
\eea
The convention was chosen to use \refeq{e-timerfftdown}; this should only be viewed as a numerical parameterization of the field \ufield. This is not
trying to imply anything regarding time reversal. The result of this is that an extra negative sign is required
for time differentation.  Again the entire point of this is that \refeq{e-timerfftdown} more accurately represents the actual numerical calculations being
performed, but \refeq{e-timerfft} it is much more convenient to write the equation for \dufield, so it will be expressed in terms of the modes of $u(\period{}-\tn, \xm)$.
Finally, upon substitution of \refeq{e-timerfft} into \refeq{e-spacerfft} \spt\ transform and its inverse are derived.
\begin{alignat}{2}
\label{e-rfft}
\ajk=\frac{1}{\sqrt{NM}}&\sum_{n=0}^{N-1} &\sum_{m = 0}^{M - 1}& u(\tn, \xm)  \cos(\wavek \xm) \cos(\tilde{\omegaj} \tn) \continue
\bjk=\frac{1}{\sqrt{NM}}&\sum_{n=0}^{N-1} &\sum_{m = 0}^{M - 1}& u(\tn, \xm)  \cos(\wavek \xm) \sin(\tilde{\omegaj} \tn) \continue
\cjk=\frac{1}{\sqrt{NM}}&\sum_{n=0}^{N-1} &\sum_{m = 0}^{M - 1}&u(\tn, \xm) \sin(\wavek \xm) \cos(\tilde{\omegaj}\tn) \continue
\djk=\frac{1}{\sqrt{NM}}&\sum_{n=0}^{N-1} &\sum_{m = 0}^{M - 1}&u(\tn, \xm) \sin(\wavek \xm) \sin(\tilde{\omegaj} \tn) \continue
\dufield=\frac{4}{\sqrt{NM}}&\sum_{k = 1}^{\frac{M}{2} - 1}&\Bigg[&\bigg(\frac{\azk}{2} + \sum_{j=1}^{\frac{N}{2}-1}\ajk\cos(\tilde{\omegaj} \tn)+\bjk\sin(\tilde{\omegaj}\tn)\bigg)\cos(\wavek \xm) \continue
&&+&\bigg(\frac{\czk}{2} + \sum_{j=1}^{\frac{N}{2}-1}\cjk\cos(\tilde{\omegaj} \tn)+\djk\sin(\tilde{\omegaj}\tn)\bigg)\sin(\wavek \xm)\Bigg] \continue
&&\quad&\mbox{where, } \omegaj = -\omegaj\,.
\end{alignat}
Practically speaking the only difference between \refeq{e-rfft} and \refeq{e-rfftdown} that occurs is an extra
negative sign upon time differentiation.
The expressions \refeq{e-rfft} and \refeq{e-rfftdown} both yield the same Fourier modes; however \dufield is used in almost
all expressions and so the extra $-1$ factor must be accounted for moving forward, most importantly for time differentiation.
The spatiotemporal \rfft \refeq{e-rfft} stands as the equation for orbits of all symmetry types;
however, as shown in \refsect{sect:symmetry}, discrete symmetries impose constraints which determine the set of non-vanishing modes.
These constraints will be referred to as selection rules; their derivation is left to the discussion of
\spt\ symmetries \refsect{sect:symmetry}.
The explicit formulae \refeq{e-timerfft}, \refeq{e-spacerfft}, \refeq{e-rfft} explicitly and exactly state how
the discrete Fourier transforms are defined. The numerical implementation is a \textit{pseudospectral} implementation. The defining
quality of this implementation is that the computation of the nonlinear term of the \KSe\
is performed as an elementwise (alternatively, Schur or Hadamard) product of fields as opposed to a convolution of modes.
Computationally this provides benefits to accuracy and computation time
(especially if the discretization sizes are powers of two, which FFT algorithms can exploit).
Due to this, an explicit expression in terms of the modes is not required,
hence \refeq{e-rfft} will only really ever be used implicitly; as a convention
will be used to represent the forward and backward transforms as \FFT, \IFFT.
If a transform for a single dimension needs to be referenced, it shall be
done via a subscript, i.e. $\FFT_x$ refers to \refeq{e-spacerfft}.

Finally, we have arrived at a formula which defines \dufield in terms of \spt\ Fourier modes.
The expression \refeq{e-rfft} is very cumbersome, however, and so two other representations
are explained for usage in future derivations and computations.
The first of the two representations defined is a tensor representation. The tensor arranges
the four sets of modes of \refeq{e-rfft} into blocks based on frequencies
\beq \label{e-modetensorawk}
\utensor \equiv
\begin{bmatrix}
\azk & \czk \\
\ajk & \cjk\\
\bjk & \djk
\end{bmatrix}\,.
\eeq
The zeroth $j=0$ time modes are indicated separately because there are no analogous terms in $\bjk, \djk$; they
were discarded because the zeroth modes must be real valued. This is an important factor
to note for time differentiation however for simplicity
the awkwardness of the zeroth mode will be implicit in the mode tensor expression such that
\beq \label{e-modetensor}
\utensor \equiv
\begin{bmatrix}
\ajk & \cjk\\
\bjk & \djk
\end{bmatrix}\,.
\eeq
In the context of the tensor notation the sets of modes $\{\ajk, \bjk, \cjk, \djk\}$ will be referred to as \textit{mode blocks}.
The shape of the mode tensors
is fundamentally different when symmetries are considered; therefore, a notation which specifies the
shape of the mode tensor based on the $j,k$ indices is developed.
Defining
\bea \label{e-jkindices}
k &\in& \{1, ..., M/2-1\} \continue
j &\in& \{1, ..., N/2-1\}
\eea
While the indices of \refeq{e-modetensor} are never referenced directly, due to the lack
of uniqueness, the blocks' indices \refeq{e-blockindices} are by the following notation, which
generates the ordered pairs of indices for each block; the first term in the product will be
called the \textit{time indices} and the latter, \textit{space indices}
the ordered pairs specifying the elements of \refeq{e-modetensor} will be defined via set notation
\beq \label{e-modeindices}
\{(j, k)\}\:\widehat{=}\:\{\{\{0\}, \{j_{ac}\}, \{j_{bd}\}\} \times \{\{k_{ab}\},\{k_{cd}\}\}\} \,.
\eeq
where $\{j_{ac}\}=\{j_{bd}\}=\{j\}$ and $\{k_{ab}\},\{k_{cd}\}=\{k\}$ from \refeq{e-jkindices}.
This is simply an attempt at formalizing the fact that to uniquely specify an element in \refeq{e-modetensor}
the block in addition to the indices $j,k$ must be provided. Additionally, another important usage
of \refeq{e-modeindices} is that the dimensions of the
mode tensor can be easily deduced given the definitions
of the sets \refeq{e-jkindices}. For example, the dimensions of the mode tensor \refeq{e-modetensor} are
\bea \label{e-modeshape}
|\utensor| &=& |\{\{\{0\}, \{j_{ac}\}, \{j_{bd}\}\}| \times | \{\{k_{ab}\},\{k_{cd}\}\}\}|\continue
&=& (|\{0\}|+|\{j\}|+|\{j\}| )\times (|\{k\}|+|\{k\}|) \continue
&=& (1 + (N/2-1) + (N/2-1)) \times (M/2-1 + M/2-1)\continue
&=& (N-1) \times (M-2)
\eea
It is very important to get this ordering correct in order to correctly take derivatives.

For practical purposes the non-existent $\bjknum{0}{k}$ modes can be assumed to equal zero in expressions such as $\ajk+\bjk$.
The reason why this is being discussed at all is so that all equations reflect the numerical computations as accurately as possible,
without being overly verbose.

In anticipation of future derivations one final convention concerning the dimensionality of \refeq{e-modeindices} is introduced.
Specifically, for the derivation of linear operators there will be Kronecker products with identity matrices of various sizes
defined by the number of modes; something which varies depending on symmetry. The following
definitions allow for a general description without having to constantly specify symmetries
\bea \label{e-identitysize}
\mathbb{I}_{\frac{N}{2}-1} &\equiv& \mathbb{I}_j \continue
\mathbb{I}_{\frac{M}{2}-1} &\equiv& \mathbb{I}_k \continue
\mathbb{I}_x &\equiv& \mathbb{I}_{|\{(j, k)\}|_t}\continue
\mathbb{I}_t &\equiv& \mathbb{I}_{|\{(j, k)\}|_x}\,.
\eea
The last two identity matrices diagonals' have dimension equal to the cardinality of the set of space indices and time
indices. For \refeq{e-modeshape} this would imply $\mathbb{I}_t = N-1$ and $\mathbb{I}_x = M-2$

The second representation of the modes will be a vector representation.
This requires two decisions to be made; how to order
real and imaginary components with respect to one another and
how to order space and time indices relative to each other.
The conventions employed here are to separate the real and imaginary components such that they
have the general form (using dummy variables) $\uvec = [\re(z_i),\im(z_i)]$. The alternative
is to alternate between real and imaginary components
$\uvec = [\re(z_0),\im(z_0), \dots, \re(z_k),\im(z_k)]$.
Next is how to order the vector elements with respect to space and time, i.e. the indices $j, k$.
The convention here is to have the spatial index $k$ as the `inner' index.
That is, cycling through the vectors' elements will cycle through all values of the inner index $k$
for each increment of the outer index $j$. For those familiar with programming parlance,
this would be equivalent to nested `for-loops'.
To represent this visually this can informally be written as
\beq \label{e-modevector}
\uvec \equiv
\overbrace{\Big[\underbrace{\ajknum{0}{k}}_{\re(x)}\;\underbrace{\cjknum{0}{k}}_{\im(x)}\;\dots}^{\re(t)}\quad \overbrace{\underbrace{\bjknum{1}{k}}_{\re(x)}\;\underbrace{\djknum{1}{k}}_{\im(x)}\;\dots\,\Big]}^{\im(t)} \;.
\eeq
The labels $\re(x), \im(x), \re(t), \im(t)$ are simply a means of demonstrating the pattern with which indices are cycled through; no formal Mathematical statement is being made in \refeq{e-modevector} other than the mode ordering.


This vector representation is used almost exclusively in the implementation of Newton's method \refsect{sect:numerical}, which requires explicit
construction of the Jacobian. Naturally, in order to explicitly construct the Jacobian matrix, a matrix representation is required for each relevant linear operator.
Therefore as the final task of this section, the matrix representations of the \rfft\ and \irfft\ operators \FFT, \IFFT are derived. In practice
this is constructed by first constructing the matrix for a single instant in time or space, and then using a Kronecker product to extend it
to space-time. In either case, the matrix is constructed by taking the \rfft\ (\irfft) of each column of appropriately sized identity matrix; this
yields a Vandemonde matrix wherein the elements are powers
of the roots of unity. The matrix representation of the linear
operator $\FFT_x$ is
\beq \label{e-RFFTxmat0}
\begin{bmatrix}
1 & 1 & 1 & 1 &\dots & 1 \\
1 & r  & r^2 & r^3 & \cdots & r^{(M-1)} \\
1 & r^2 & r^4 & r^6 & \cdots & r^{2(M-1)} \\
\vdots & \vdots & \vdots &\vdots & \ddots & \vdots \\
 1 & r^{(\frac{M}{2})} & r^{2(\frac{M}{2})} & r^{3(\frac{M}{2})} & \cdots & r^{(\frac{M}{2})(M-1)}
\end{bmatrix}
\eeq
with $r = \cos(\frac{2\pi}{M}) + i\sin(\frac{2\pi}{M})$. The remaining steps are: to account for the constraints on $k \in \{0, \frac{M}{2}\}$, to make bring \refeq{e-RFFTxmat} to a real-valued form and to extend the definition to create an
operator which acts on the entirety of space-time.

The first of these steps, accounting for the constraints, is handled by discarding the first and last rows; these correspond to the constrained modes.
\beq \label{e-RFFTxvander}
\begin{bmatrix}
1 & r  & r^2 & r^3 & \cdots & r^{(M-1)} \\
1 & r^2 & r^4 & r^6 & \cdots & r^{2(M-1)} \\
\vdots & \vdots & \vdots &\vdots & \ddots & \vdots \\
 1 & r^{(\frac{M}{2}-1)} & r^{2(\frac{M}{2}-1)} & r^{3(\frac{M}{2}-1)} & \cdots & r^{(\frac{M}{2}-1)(M-1)}
\end{bmatrix}
\eeq
Substituting the identity $r^p = (\cos(\frac{2\pi}{M}) + i\sin(\frac{2\pi}{M}))^p = (\cos(\frac{2\pi p}{M})+i\sin(\frac{2\pi p}{M}))$ into
this previous expression allows us to easily separate the real and imaginary components; the spatial transform operator
is created by concatenating (alternatively: conjoining, appending) the imaginary components of this matrix to the bottom of the real component and multiplying
by a normalization factor
\begingroup
\renewcommand*{\arraystretch}{1.5}
\begin{flalign}
\label{e-RFFTxmat}
[\FFT_x] &\equiv \frac{1}{\sqrt{M}}
\begin{bmatrix}
1 & \cos(\frac{2\pi}{M})   & \cos(\frac{4\pi}{M}) & \cdots & \cos(\frac{\sml{2(M-1)}\pi}{M})  \\
1 & \cos(\frac{4\pi}{M}) & \cos(\frac{8\pi}{M})& \cdots & \cos(\frac{\sml{4(M-1)}\pi}{M}) \\
\vdots & \vdots & \vdots & \ddots & \vdots \\
 1 & \cos(\frac{\sml{2(\frac{M}{2}-1)}\pi}{M})& \cos(\frac{\sml{4(\frac{M}{2}-1)}\pi}{M})& \cdots & \cos(\frac{2\sml{(\frac{M}{2}-1)}\sml{(M-1)}\pi}{M}) \\
1 & \sin(\frac{2\pi}{M})   & \sin(\frac{4\pi}{M}) & \cdots & \sin(\frac{\sml{2(M-1)}\pi}{M})  \\
1 & \sin(\frac{4\pi}{M}) & \sin(\frac{8\pi}{M})& \cdots & \sin(\frac{\sml{4(M-1)}\pi}{M}) \\
\vdots & \vdots & \vdots & \ddots & \vdots \\
 1 & \sin(\frac{\sml{2(\frac{M}{2}-1)}\pi}{M})& \sin(\frac{\sml{4(\frac{M}{2}-1)}\pi}{M})& \cdots & \sin(\frac{2\sml{(\frac{M}{2}-1)}\sml{(M-1)}\pi}{M}) \\
\end{bmatrix}
\end{flalign}
\endgroup
The \xdFt\ in the real-valued basis is an orthogonal transformation and so the inverse of \refeq{e-RFFTxmat} can be derived by taking
the matrix transpose of \refeq{e-RFFTxmat} and accounting for the normalization factor in definition of the inverse transform \refeq{e-spacerfft}.
\beq
[\IFFT_x] = 2 [\FFT_x]^{\top}
\eeq
Recall that \refeq{e-RFFTxmat} was constructed such that it returns the spatial modes defined at a single instant in time $u(t', x_m)$.
The generalization to space time is therefore simply derived by taking
the Kronecker outer product of \refeq{e-RFFTxmat} with the identity matrix whose diagonal is the same dimension as the time discretization.
\beq
\mathbf{M[\FFT_x]} = \mathbb{I}_N \otimes [\FFT_x]
\eeq
and likewise for the matrix inverse.

The matrix representation for the temporal transforms follows from the previous exposition  \refeq{e-RFFTxmat}, the only
difference being the inclusion of the zeroth mode. The temporal transform matrix is then
\begingroup
\renewcommand*{\arraystretch}{1.5}
\begin{flalign}
\label{e-RFFTtmat}
[\FFT_t] &\equiv \frac{1}{\sqrt{N}}
\begin{bmatrix}
1 & 1 & 1 & \dots & 1 \\
1 & \cos(\frac{2\pi}{N})   & \cos(\frac{4\pi}{N}) & \cdots & \cos(\frac{\sml{2(N-1)}\pi}{N})  \\
1 & \cos(\frac{4\pi}{N}) & \cos(\frac{8\pi}{N})& \cdots & \cos(\frac{\sml{4(N-1)}\pi}{N}) \\
\vdots & \vdots & \vdots & \ddots & \vdots \\
 1 & \cos(\frac{\sml{2(\frac{N}{2}-1)}\pi}{N})& \cos(\frac{\sml{4(\frac{N}{2}-1)}\pi}{N})& \cdots & \cos(\frac{2\sml{(\frac{N}{2}-1)}\sml{(N-1)}\pi}{N}) \\
1 & \sin(\frac{2\pi}{N})   & \sin(\frac{4\pi}{N}) & \cdots & \sin(\frac{\sml{2(N-1)}\pi}{N})  \\
1 & \sin(\frac{4\pi}{N}) & \sin(\frac{8\pi}{N})& \cdots & \sin(\frac{\sml{4(N-1)}\pi}{N}) \\
\vdots & \vdots & \vdots & \ddots & \vdots \\
 1 & \sin(\frac{\sml{2(\frac{N}{2}-1)}\pi}{N})& \sin(\frac{\sml{4(\frac{N}{2}-1)}\pi}{N})& \cdots & \sin(\frac{2\sml{(\frac{N}{2}-1)}\sml{(N-1)}\pi}{N}) \\
\end{bmatrix}
\end{flalign}
\endgroup
and the inverse
\beq
[\IFFT_t] = 2 [\FFT_t]^{\top} \,.
\eeq
The Kronecker product appears except in reversed order; also, the identity matrix reflects the spatial dimension
\beq
\mathbf{M[\FFT_t]} = [\FFT_t] \otimes \mathbb{I}_{x}
\eeq
This concludes the definition of the spatial and temporal Fourier transform matrices. In order to get the
\spt\ version, simply take the product $M[\FFT_t]M[\FFT_x]$. However, these matrices are kept
separate for reasons pertaining to the formulation of \goveqn.

