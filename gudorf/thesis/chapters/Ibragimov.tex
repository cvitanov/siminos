% siminos/gudorf/thesis/chapter/Ibragimov.tex
% $Author: predrag $ $Date: 2020-05-25 15:18:45 -0400 (Mon, 25 May 2020) $

%\section{Formal Lagrangian adjoint derivation}
%\label{sect:Ibragimov}
% Matt 2018-05-07

Following sect.~2 {\em Quasi-self-adjoint equations} of
Ibragimov\rf{Ibragimov07b} (which \PCedit{does not} reference
\refrefs{Ibragimov06,Ibragimov07a}),
%I believe
we can write
the \emph{formal Lagrangian} of the \KSe\ to derive the {\spt}
adjoint equations in terms of the original {\spt} field
$u(\conf,\zeit)$, and then one is free to use whatever representation
suits the user in discretization; the cavaet is that numerically it seems
better to use a real valued representation Fourier representation for the
adjoint descent.
    \PC{2018-05-08} {Ibragimov\rf{Ibragimov07b} is included in
    \HREF{http://gammett.ugatu.su/images/archives_of_alga/archives\%204_1.pdf}
    {Archives of ALGA {\bf 4}}. }

Ibragimov notation, for the \KS\ case: the independent variable is
denoted by $\conf$. The dependent variable is $u$,
used together with its first-order partial derivative
   $u_{(1)}:$
  $$
  u_{(1)} = \{u^\alpha_i\}, \quad u^\alpha_i = D_i (u^\alpha),
  $$
and higher-order derivatives $u_{(2)}, \ldots, u_{(4)}\,,$ where
 $$
  u_{(2)} = \{u^\alpha_{ij}\}, \quad u^\alpha_{ij} = D_i D_j
  (u^\alpha),\ldots,
  $$
% up to $s$th-order  derivatives $u_{(s)}:$
  $$
  u_{(s)} = \{u^\alpha_{i_1\cdots i_s}\}, \quad u^\alpha_{i_1\cdots i_s}
  = D_{i_1} \cdots D_{i_s}(u^\alpha).
  $$
  Here $D_i$ is the total differentiation with respect to $x^i:$
 \begin{equation}
 \label{sa:cl.diff1}
  D_i = \frac{\partial}{\partial x^i} +
u^\alpha_{i}\frac{\partial}{\partial u^\alpha} +
u^\alpha_{ij}\frac{\partial}{\partial u^\alpha_j} +
 \cdots\,.
 \end{equation}


Using the definition for the \emph{formal Lagrangian} $\mathcal{L}$,
\beq
\mathcal{L} \equiv v \left[u_{\zeit} + u_{\conf \conf} + u_{\conf \conf \conf \conf}
                        + uu_{\conf} \right],
\ee{FormalLagrangian}
and then taking the functional derivative,
% (I tend to think of this as a total
%  derivative but that might be confusing notation),
\beq
\frac{\delta \mathcal{L}}{\delta u} = 0 \,.
\ee{variationalequality}
The surviving terms from this functional derivative are
\bea \label{LagrangianDeriv}
\frac{\partial \mathcal{L}}{\partial u}              &=& vu_{\conf} \continue
-\partial_{\zeit} \frac{\partial \mathcal{L}}{\partial u_{\zeit}} &=& -v_t \continue
-\partial_{\conf} \frac{\partial \mathcal{L}}{\partial u_{\conf}} &=& -vu_{\conf}-uv_{\conf} \continue
\partial_{\conf}^2 \frac{\partial \mathcal{L}}{\partial u_{\conf \conf}} &=& v_{\conf \conf} \continue
\partial_{\conf}^4 \frac{\partial \mathcal{L}}{\partial u_{\conf \conf \conf \conf}} &=& v_{\conf \conf \conf \conf},
\eea
where the sum of these terms equals \refeq{variationalequality} and hence
must be zero. The resultant adjoint equation is ($\pm
vu_{\conf}$ terms cancel),
\beq
-v_t + v_{\conf \conf} + v_{\conf \conf \conf \conf} - uv_{\conf} = 0
\,.
\ee{adjoint_equation}

If we take the adjoint variable to be the \KSe,
\beq
F \equiv v = - \left[u_{\zeit} + u_{\conf \conf} + u_{\conf \conf \conf \conf}
                        + uu_{\conf} \right]
\,,
\ee{adjointchoice}
we arrive at the equation which \emph{I claim} provides adjoint descent
direction without explicit construction of the gradients matrix $J$,
\beq
-J^{\dagger}F = (\partial_{\zeit} + \partial_{\conf}^2 + \partial_{\conf}^4)F
                    + u \partial_{\conf} F
\ee{adjointdescent}
(I believe the negative sign in \refeq{adjointchoice} is motivated so
that the functional derivative is strictly decreasing or zero).

Numerical evidence is suggestive as the real valued adjoint descent is
working better than before when I was trying to reverse engineer
$J^{\dagger}F$ in a matrix-free way.
