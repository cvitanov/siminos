% siminos/gudorf/thesis/chapter/KSsols.tex
% $Author: predrag $ $Date: 2020-05-25 15:18:45 -0400 (Mon, 25 May 2020) $

% \section{Exact solutions of \KS}
% \label{sect:KSsols}
%   TEMPORARY, ELIMINATE EVENTUALLY
%   was KSe.tex, copied from siminos/rpo_ks/current/

\subsection{\Eqva\ and \reqva}
\label{sec:stks}

\Eqva\  (or the steady solutions)
are the fixed profile time-invariant solutions,
\beq
 u(\conf,\zeit) = u_\stagn(x)
\,.
\ee{eqva}
Due to the spatial translational symmetry,
the \KS\ system also allows for
\reqva\ (traveling waves, rotating waves),
characterized by a fixed profile $u_\stagn(x)$
moving with constant speed $c$, {\ie}
\beq
 u(\conf,\zeit) =  u_\stagn(x-ct)
\,.
\ee{reqva}
Here suffix ${}_\stagn$ labels a particular invariant solution.
Because of the reflection symmetry \refeq{SCD07:KSparity},
the \reqva\ come in counter-traveling pairs
$u_\stagn(x-ct)$, $-u_\stagn(-x+ct)$.

The \reqv\ condition for the {\KS} PDE \refeq{e-ks} is the ODE
\beq
{\textstyle\frac{1}{2}}(u^2)_x-c\,u_x+u_{xx}+ u_{xxxx}=0
\ee{KSeqvCond}
which can be analyzed as a dynamical system in its own right.
Integrating once we get
\beq
{\textstyle\frac{1}{2}}u^2 - c u + u_x + u_{xxx}=\expctE
\,.
\label{eq:stdks}
\eeq
        \PC{2019-09-21}{
Is this an interesting form?
\[
{\textstyle\frac{1}{2}}(u-c/2)^2  + u_x + u_{xxx}=\expctE+{c^2}/{8}
\,.
\]

    }
This equation can be interpreted as a 3-dimen\-si\-on\-al dynamical system
with spatial coordinate $x$ playing the role of `time,'
and the integration constant \expctE\ can be interpreted as `energy,'
see \refsect{sect:KSenerg}.

For $\expctE>0$ there is rich $\expctE$-dependent dynamics,
with fractal sets of bounded solutions investigated in depth
by Michelson\rf{Mks86}.

From \refeq{SCD07:expan} we see that the origin $u(\conf,\zeit) = 0$
has Fourier modes as the linear stability eigenvectors.  The $|k|<\tildeL$
long wavelength perturbations of the flat-front {\eqv}
are linearly unstable, while all
$|k|> \tildeL$ short wavelength perturbations are strongly contractive.
The high $k$ eigenvalues, corresponding to rapid variations of
the flame front, decay so fast that the corresponding eigendirections
are physically irrelevant.
The most unstable mode, nearest to $|k|=\tildeL/\sqrt{2}$,
sets the scale of the mean wavelength $\sqrt{2}$
of the \KS\ `turbulent' dynamics,
see \reffig{f:ks_largeL}.

\MNG{2019-05-13}{\Spt\ symmetries are discussed in \refsect{sect:KSsymm}, I think it
is worthy to elaborate on the different classes of solutions but I don't think
\ppo{}s should be referred to as \rpo{} solutions with reflection}

\subsection{\Rpo s, symmetries and \po s} \label{sec:KSePO}

\KSe\ \refeq{e-ks} is time translationally invariant, and
space translationally invariant under the 1-$d$ Lie group of $\On{2}$
rotations: if $u(\conf,\zeit)$ is a solution, then $u(x+\shift,t)$ and
$-u(-x,t)$ are equivalent solutions for any $-\speriod{}/2 < \shift \leq
\speriod{}/2$.
As a result of invariance under $\Shift_{\shift/\speriod{}}$,
\KSe\ can have \rpo\ solutions
with a profile $u_p(x)$, period $\period{p}$, and a
nonzero shift $\shift_p$
\beq
  \Shift_{\shift_p/\speriod{}}u(x,\period{p}) =
  u(x+\shift_p,\period{p}) = u(x,0) = u_p(x)\,.
\label{KSrpos}
\eeq
{\Rpo s} \refeq{KSrpos} are periodic in
$v_p=\shift_p/\period{p}$ co-rotating frame,
%(see \reffig{f:MeanVelocityFrame}),
but in the stationary frame their
trajectories are quasiperiodic.  Due to the reflection symmetry
\refeq{SCD07:KSparity} of \KSe, every {\rpo} $u_p(x)$ with shift
$\shift_p$ has a symmetric partner $-u_p(-x)$ with shift $-\shift_p$.

Due to invariance under reflections, \KSe\ can also have
\rpo s {\em with reflection}, which are
characterized by a profile $u_p(x)$ and
period $\period{p}$
\beq
  \Refl u(x+\shift,\period{p}) =
  -u(-x-\shift,\period{p}) = u(x+\shift,0) = u_p(x)
  \,,
\label{KSpos}
\eeq
giving the family of equivalent solutions
parameterized by $\shift$
(as the choice of the reflection point is arbitrary,
the shift can take any value in $-\speriod{}/2 < \shift \leq \speriod{}/2$).

As $\shift$ is continuous in the interval $[-\speriod{}/2, \speriod{}/2]$,
the likelihood of a \rpo\ with $\shift_p = 0$ shift is zero,
unless an exact periodicity is enforced by a discrete symmetry,
such as the dihedral symmetries discussed above.
If the shift $\shift_p$ of a \rpo\ with period $\period{p}$ is such
that $\shift_p /\speriod{}$ is a rational number, then the orbit is
periodic with period $n\period{p}$.  The likelihood to find such \po s is
also zero.

However, due to the \KSe\ invariance under
the dihedral $\Dn{n}$ and cyclic $\Cn{n}$ subgroups, the following
types of \po s are possible:

{\bf (a)} The \po\ lies
within a subspace pointwise invariant under the action of
$\Dn{n}$ or $\Cn{n}$. For instance, for $\Dn{1}$ this is the
$\bbU^+$ antisymmetric subspace, $-u_p(-x) = u_p(x)$, and
$u(x,\period{p}) = u(x,0) = u_p(x)$. The periodic orbits
found in \refrefs{Christiansen97,lanCvit07} are
all in $\bbU^+$, as the dynamics is restricted to
antisymmetric subspace. For $\speriod{}=3.5014087480216975$ the dynamics in $\bbU^+$
is dominated by attracting (within the subspace)
heteroclinic connections and thus we have no periodic orbits
of this type, or in any other of the $\Dn{n}$--invariant
subspaces.

{\bf (b)} The \po\ satisfies
\beq
	 u(x,t+\period{p})=\LieEl u(\conf,\zeit)\,,
	\label{eq:POspattemp}
\eeq
for some group element $\LieEl\in \On{2}$ such that
$\LieEl^m=e$ for some integer $m$ so that the orbit repeats
after time $m \period{p}$ (see
\refref{golubitsky2002sp} for a general discussion of
conditions on the symmetry of \po s).
If an orbit is of reflection type \refeq{KSpos},
$\Refl\Shift_{\shift/\speriod{}} u(x,\period{p}) =
-u(-x-\shift,\period{p}) = u(x,0)$, then it is pre-periodic
to a \po\ with period $2\period{p}$. Indeed, since
$(\Refl\Shift_{\shift/\speriod{}})^2 = \Refl^2 = 1$, and the KS
solutions are time translation invariant, it follows from
\refeq{KSpos} that
\[
  u(x,2\period{p}) = \Refl\Shift_{\shift/\speriod{}} u(x,\period{p}) =
  (\Refl\Shift_{\shift/\speriod{}})^2 u(x,0) = u(x,0)\;.
\]
Thus any shift acquired during time $0$ to
$\period{p}$ is compensated by the opposite shift during
evolution from $\period{p}$ to $2 \period{p}$.
\Ppo s are a hallmark of any dynamical system
with a discrete symmetry, where they have a natural
interpretation as \po s in the fundamental
domain\rf{CvitaEckardt,DasBuch}.



    \PCedit{ % 2019-09-21
For any given \rpo\ a convenient visualization is offered by the {\em
mean velocity frame}, {\ie}, a reference frame that rotates with velocity
$v_p=\shift_p/\period{p}$. In the mean velocity frame a \rpo\ becomes a
\po. However, each {\rpo} has its own mean velocity frame and thus sets
of \rpo s are difficult to visualize simultaneously.

\section{\Ppo s} \label{ssec:po}

As discussed in \refSect{sec:KSePO}, a \rpo\ will be
periodic, \ie, $\shift_p = 0$, if it either {\bf (a)} lives
within the $\bbU^+$ antisymmetric subspace, $-u(-x,0) =
u(x,0)$, or {\bf (b)} returns to its reflection
or its discrete rotation after a period:
$u(x,t+\period{p})=\LieEl u(\conf,\zeit)$, $\LieEl^m=e$,
and is thus periodic with period $m\period{p}$.
    }
