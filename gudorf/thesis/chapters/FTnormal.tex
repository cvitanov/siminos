% siminos/gudorf/thesis/chapter/FTnormal.tex
% $Author: predrag $ $Date: 2020-05-25 15:18:45 -0400 (Mon, 25 May 2020) $

% called by
%           siminos/spatiotemp/chapter/spatiotemp.tex
%           siminos/tiles/GuBuCv17.tex

%\section{Appendix: Fourier transforms and normalization factors}
%\label{sect:FTnormal}
% Predrag                                           26 May 2016


\MNG{2016-06-17}{Burak Budanur wrote this.}
%
We have time-periodic solutions, namely two repeats of \ppo s,
however, in my intuition (which might be wrong) we don't
necessarily need to start simulations from those. Since the dynamics is
chaotic, after a finite time, say 20 Lyapunov times
($e^{-20} \sim O(10^{-9})$), correlations between initial condition and
final point will be so low that imposing periodicity in time will not
effect the outcome.

With this in mind, let's say that you solved the \KSe\ in spatially
periodic domain, with an initial condition on the strange attractor
(no transients), and obtained $u(\conf, \zeit)$, for $\conf \in [0, \speriod{})$
and $\zeit \in [0, T)$. You can find the spatial derivatives by
inverse Fourier transformations:
\bea
    u_{\conf}( \conf, \zeit) &=&
                \mathcal{F}^{-1} \left\{i q_k u_k \right\} \,, \quad
    u_{\conf \conf}( \conf, \zeit) =
                \mathcal{F}^{-1} \left\{(i q_k)^2 u_k \right\} \,, \quad
    \continue
    u_{\conf \conf \conf}( \conf, \zeit) &=&
                 \mathcal{F}^{-1} \left\{(i q_k)^3 u_k \right\} \,, \quad
    u_{\conf \conf \conf \conf}( \conf, \zeit) =
                 \mathcal{F}^{-1} \left\{(i q_k)^4 u_k \right\} \, .
    \label{e-Spectralderiv}
\eea
In fact, you must compute spatial derivatives as above (not by
approximating with finite-differences) because otherwise they will not
be as accurate numerically. Side note: depending on the implementation
you're using, Fourier transforms would need you to add a normalization
factor, usually a division by $N$ (number of modes).

Now that you have $u_{\conf}( \conf, \zeit)$ and its space derivatives,
you should take their value at $\conf = 0$ for $\zeit \in [0, T)$ as
your initial condition and input it to the space-integrator. Then you
can compare the outcome with the one you already have from
time-integration.
%}


\MNG{2016-07-13}{Burak Budanur wrote this section.}
In this section we go through the derivation of \refeq{e-FksX}
and state the correct normalizations for Fourier transforms.

Let us start from the following definition of the Fourier expansion of the
time-periodic function $u(t) = u(t + \period{})$:
\beq
    u(\zeit) = \sum_{k = -\infty}^{\infty}
    \Fu_k e^{i \omega_k \zeit} \, , \quad \mbox{where }
    \omega_k = 2 \pi k / \period{}
\,.
\label{e-Fseries}
\eeq

In order to find Fourier coefficients $\Fu_k$, we multiply the above equation
from the left by
$\frac{1}{\period{}} \int_0^{\period{}} d\zeit\ e^{-i \omega_m t}$,
on the RHS we get:
\beq
    \sum_{k = -\infty}^{\infty} \Fu_k \frac{1}{\period{}}
    \int_0^{\period{}} d\zeit\ e^{i (\omega_k - \omega_m) \zeit}
    =
    \sum_{k = -\infty}^{\infty} \Fu_k \frac{1}{\period{}}
    \int_0^{\period{}} d\zeit\ e^{i 2 \pi (k - m) \zeit / \period{}}
     \, .
\eeq
If $k \neq m$, then the integral above is integral of a periodic function
over one full period, hence $0$. If $k = m$, then it is the integral of $1$
from $0$ to $\period{}$, and we can replace the integral by $\period{} \delta_{km}$, which
picks out $\Fu_m$ from the sum. Hence we obtain the forward Fourier transform
of $u(\zeit)$ as
\beq
    \Fu_k = \frac{1}{T} \int_0^{\period{}} d\zeit\ u(\zeit)
            e^{- i \omega_k \zeit}
\eeq
We can approximate the above transformation by replacing the integral by a
Riemann sum $\int_0^T dt \rightarrow \sum_{n=0}^{N-1}\Delta t$, $\Delta t ={T}/{N} $,
hence we
obtain the discrete Fourier transform as
\bea
    \Fu_k &=& \frac{1}{N} \sum_{n=0}^{N-1} u(t_n) e^{-i \omega_k t_n} \,,
    \mbox{where } t_n = n T / N \continue
          &=& \frac{1}{N} \sum_{n=0}^{N-1} u(t_n) e^{-i 2 \pi n k / N}
          \, , \continue
          &=& \frac{1}{N} \mathcal{F} \{ u (t_n) \} \, ,
\label{e-FdscrApprx}
\eea
where $\mathcal{F} \{ . \}$ denotes the Fourier transformation in Matlab's
normalization convention. Consequently, if we take $2N+1$ terms from the
series \refeq{e-Fseries}, we obtain the inverse discrete Fourier transform as
\bea
    u(\zeit_n) &=& \sum_{k = - N/2}^{N/2} \Fu_k e^{i \omega_k \zeit_n}
                   \,, \quad
               \,=\, \sum_{k = - N/2}^{N/2} \Fu_k e^{i 2 \pi k n / N}
                   \,, \continue
               &=& N \mathcal{F}^{-1} \{\Fu_k \} \, ,
\eea
where $\mathcal{F}^{-1} \{ . \}$  is the inverse Fourier transform in the
Matlab's convention.

In Matlab it is probably is computationally preferable to carry out the
convolution in the fourth equation of \refeq{e-FksX} in time-domain as
\beq
    \sum_{m = -\infty}^{\infty} \Fu^{(0)}_{\ell - m} \Fu^{(1)}_m
        = \mathcal{F} \left\{ \mathcal{F}^{-1} \left\{\Fu^{(0)} \right\}
                              \mathcal{F}^{-1} \left\{\Fu^{(1)} \right\}
                      \right\} \, ,
\eeq
where  $\mathcal{F}$ and $\mathcal{F}^{-1}$ denote the Fourier transform
and its inverse, respectively. One should experiment with time-domain
sizes and truncation of the Fourier expansion.

One can insert the definition \refeq{e-Fseries} into \refeq{e-ksX} and then
multiply from left by the integral
$\frac{1}{\period{}} \int_0^{\period{}} d\zeit\ e^{-i \omega_m t}$
in order to confirm that the equation \refeq{e-FksX} is correct. But in order
to compute the nonlinear term pseudospectrally, we take the
Fourier transform of $u^{(0)} u^{(1)}$, that is
\bea
    \frac{1}{\period{}} \int_0^{\period{}} d\zeit\ e^{-i \omega_m t}
    u^{(0)} (t) u^{(1)} (t)
    &\approx& \frac{1}{N} \sum_{n = 0}^{N-1} u^{(0)}(t_n) u^{(1)}(t_n)
                                             e^{-i \omega_m t_n}
    \, , \continue
    &=& \frac{1}{N} \mathcal{F} \{ u^{(0)} u^{(1)} \} \, .
\eea
