

I believe I found the minimal hook and streak patterns (or at least
members of the same families) today. This was accomplished by using the
hybrid-methods that I used to trawl. Previously, I was only using
Gauss-Newton to converge subdomains as it worked but the hook required
adjoint descent before applying Gauss-Newton in order to converge.

Solution hook = \emph{rpo\_L13.07\_T10} was found by using a family
member of the shortest \ppo\ in time at $L=22$ whose energy was a local
maximum. This ended up being at $L=20$, but this turns out to not matter.
What matters in the end is that I use hybrid methods and not just
Gauss-Newton to try to converge the tiles; just like how I had done for
all of the {\twots} beforehand. The expedient results of
just using Gauss-Newton and finding ``defects" and ``gaps" had led me
astray.

Similar story to that of the hook; applying hybrid numerical methods
allowed me to find an \eqv\ solution whose spatial extent is
approximately $2\pi$, or $\pi$ in the fundamental domain of the
antisymmetric subspace $\bbU^+$.

Numerical continuation of the hook and gap solutions provide continuous
families of solutions of which it seems the Predrag is correct that the
``defect" type solution and ``hook" type solution are of the same family.
As a means of trying to identify some criterion by which to choose
members of said families I produced plots of members of the continuous
families adjacent to plots of the entire family's energy, spatiotemporal
averaged energy dissipation and production. There is some slight
numerical discrepancy but they should be equal in theory; It could be the
quadrature rule that I am using to calculate $<u_{\conf\conf}^2>_{L\period{}}$
and $<u_{\conf}^2>_{L\period{}}$, where $< \star >_{L\period{}}$ indicates
spatiotemporal average.