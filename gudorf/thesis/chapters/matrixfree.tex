% siminos/gudorf/thesis/chapter/matrixfree.tex
% $Author: predrag $ $Date: 2020-05-25 15:18:45 -0400 (Mon, 25 May 2020) $

%Possible figures
%Display of computation time vs. matrix and matrix-free methods?

An important piece that appears in our calculations but we have yet to cover
is how, given limits to computational memory, we calculate matrix vector
products such as $\frac{\partial F}{\partial \mathbf{z}}\delta \mathbf{z}$
and $\frac{\partial F}{\partial \mathbf{z}}^{\top}\delta\mathbf{z}$. These
quantities arise frequently in many numerical methods designed
for finding periodic solutions of a nonlinear PDE. Very often, it results from
the linearization that leads to the Newton equation\rf{Visw08,ChaKer12,WillisPipe17}.
In all of those examples, the matrix-vector product
involving the Jacobian matrix is approximated with a (typically first order)
finite-difference approximation\rf{KK04,Brown87}.
For equations of motion
\beq \label{e-eqnmotion}
u_t \equiv v(u(t))
\eeq
the time evolution rule is given by
\beq \label{e-timeevolution}
f^{t}(u_0) = u(t) = u_0 + \int_0^t d\tau v(u(\tau))\,.
\eeq
The Taylor expansion
\beq \label{e-tangentspace}
f^{t}(u+\delta u) \approx f^{t}(u) + \frac{\partial f^{t}(u)}{\partial u(0)}\delta u + ...
\eeq
where we define the Jacobian matrix as
\beq \label{e-Jacobian}
J^{t}(u_0) = \frac{\partial  f^{t}(u)}{\partial u(0)}\,.
\eeq
The finite difference approximation in question is derived
by simply rearrangement of \refeq{e-Jacobian}
\beq \label{e-finitediffJac}
\frac{f^{t}(u+\epsilon \delta u)-f^{t}(u)}{\epsilon} \approx J^{t}(u_0) \delta u
\eeq
where the replacement $\delta u \to \epsilon \delta \hat{u}$ where
$|\delta u| = 1$ is merely to abide by convention. The approximation
\refeq{e-finitediffJac} relates a finite difference
with the \emph{action} of the Jacobian on a vector. The implicit time dependence is accounted
for via time integration. One consequence of this is that the computational
effort required to evaluate this approximation can be quite high.
In our case, however, there are no dynamics, and hence no integration.
The lack of dynamics or time dependence means that in our situation the
matrix derived in a similar fashion is no longer a Jacobian matrix by our
definition as it does not transport tangent space perturbations. To avoid confusion
we shall refer to the analogous matrix of our system as the \emph{sensitivity matrix}
because it represents the response of \refeq{e-ks} to perturbations of system parameters.
Once again our \spt\ formulation is advantageous as the finite-difference approximation
\refeq{e-finitediffJac} can
be exchanged for a faster and more accurate spectral computation. Speed
and accuracy are usually inversely proporational which further reinforces
how potent the spectral calculation is.
%I feel like I'll need to include
%proof that it's faster and more accurate.
Our intent of this discussion is to very thoroughly describe what we are doing,
in part because the notion of approximating the Jacobian via finite-differences
is seemingly hardwired in certain disciplines.
In order to do keep the derivations
as concise and understandable as possible
we elect to use a ``direct matrix'' notation\rf{Chu09}.
That is to say, we rewrite \refeq{e-kssFb}
in terms of linear operators and pointwise multiplications
\beq \label{e-KSmatrixform}
(\mathcal{D}_t + \mathcal{D}_{xx} + \mathcal{D}_{xxxx}+\frac{\sigma}{T}\mathcal{D}_x) \Fu
+ \frac{1}{2} \mathcal{D}_{x} \mathcal{F}(\mathcal{F}^{-1}\Fu*\mathcal{F}^{-1}\Fu)
\eeq
where subscripts indicate differentiation.
This notation will simplify the expressions derived for the sensitivity
matrix and its adjoint. Namely, the columns of the sensitivity matrix
\beq \label{e-SensMatGeneral}
\frac{\partial F}{\partial z} =
\begin{bmatrix}
\frac{\partial F}{\partial \Fu}         &
\frac{\partial F}{\partial \speriod{}}  &
\frac{\partial F}{\partial \period{}}   &
\frac{\partial F}{\partial \sigma}
\end{bmatrix}\,.
\eeq
can be written as follows in the direct matrix notation
\bea \label{e-Partials}
\frac{\partial F}{\partial \Fu}
&=& (\mathcal{D}_t + \mathcal{D}_{xx}
+\mathcal{D}_{xxxx}+\frac{\sigma}{\period{}}\mathcal{D}_x)
+\mathcal{D}_{x} \mathcal{F}\cdot \text{Diag}(u)\cdot \mathcal{F}^{-1}
\continue
\frac{\partial F}{\partial \speriod{}}
&=& (\frac{-2}{\speriod{}}\mathcal{D}_{xx}
-\frac{4}{\speriod{}}\mathcal{D}_{xxxx}
-\frac{\sigma}{\period{}\speriod{}}\mathcal{D}_x)\Fu
-\frac{1}{2L} \mathcal{D}_{x}
\mathcal{F}(\mathcal{F}^{-1}\Fu)*\mathcal{F}^{-1}\Fu)
\continue
\frac{\partial F}{\partial \period{}}
&=& (-\frac{1}{\period{}}\mathcal{D}_{t}
-\frac{\sigma}{T^2}\mathcal{D}_x)\Fu
\continue
\frac{\partial F}{\partial \sigma}
&=&\frac{1}{\period{}}\mathcal{D}_x \Fu\,.
\eea
The adjoint of the first equation of \refeq{e-Partials} is
\beq \label{e-AdjSens}
\frac{\partial F}{\partial \Fu}^{\dagger}
= -\mathcal{D}_t + \mathcal{D}_{xx}
+\mathcal{D}_{xxxx}-\frac{\sigma}{\period{}}\mathcal{D}_x
-\mathcal{F}\cdot \text{Diag}(u)\cdot \mathcal{F}^{-1}\mathcal{D}_{x}
\eeq
The remaining adjoints (row vectors) are not shown because
the matrix free computation is the same as \refeq{e-Partials}
merely with an additional transposition.
In summary, to compute all relevant matrix vector products
we need matrix free versions of Fourier transforms, differentiation,
elementwise multiplication. The only difficulty
that arises are when \emph{symmetry specific} \xDft\ are desired.
As implied by the name, these are \xDft\ which remove redundant
\spt\ \Fcs\ which are constrained by symmetry.
For a discussion of these constraints see \refsect{sect:selectionKS}.
