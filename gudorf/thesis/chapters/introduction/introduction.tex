Chaotic or turbulent processes categorize one of the
few outstanding problems to be solved in classical physics.
While deterministic, the complexity of the problem
can be categorized by infrequent or lack of analytic results.
It is often necessary to rely on models which capture the
important quantitative properties and behavior of the underlying
process. These models often take form as nonlinear partial
differential equations which are hyperbolically unstable.
The systems
whose spatial correlations decay sufficiently fast, and the
attractor dimension and number of positive Lyapunov exponents
diverges with system size are said\rf{HNZks86,man90b,cross93}
to be extensive, `spatio-temporally chaotic' or `weakly
turbulent'.

The substantial computational results have all occurred on
very small computational domains, also known as minimal cells. % ref
The combination of minimal cells with periodic orbit theory
shed light on the geometry of the state space of the Navier-Stokes
equations. Very important solutions denoted as ``exact coherent structures'' (ECS)
are the crux of this work\rf{W01, WK04}.
The notion of ECS came about by considering fluid flow
as a traversal of an infinite dimensional state space, here after referred
to as the ``dynamical systems formulation''.
In this infinite dimensional state space, the time invariant sets
which are topological invariants; quantities that remain
unchanged regardless of the representation of the problem. In addition,
their cycle multipliers; eigenvalues from linear stability
analysis are also metric invariants\rf{DasBuch}.
It is their unstable and stable manifolds which shape the geometry of
state space and dictate the dynamics.

By their nature
\twots\ are global (infinite) solutions due to their space-time periodicity.
Our hypothesis is that infinite space-time can be described
as an infinite collection of shadowing of \twots. Going even further, it is
not only the infinite ergodic trajectory which is shadowed locally by \twots;
larger \twots\ can also be described by the shadowing of smaller \twots. \twoTs\
can be subdivided until the ``atoms'' of space-time are extracted.
We shall refer to these ``indivisible'' \twots\ as \textit{fundamental tiles}.
In this context, ``fundamental'' is a reference to the fundamental physical
behaviors that tiles (typically) represent. In addition to the decomposition
of \twots\ into tiles, the reverse is also true:
finding and enumerating all tiles enables the construction
of the general space-time solution.

This is an incredibly powerful
tool whose importance will be asserted frequently and is the basis for the
entire \spt\ theory. The same can be said of periodic orbits, so what gives?
The trick hidden behind these statements is that the number of fundamental
tiles is not only finite but incredibly small; as of now there are only three
fundamental tiles. To be more precise; these tiles exist in
continuous families. It is these families that contain
the continuous deformations tiles, which helps explain the differences
in appearance between shadowing events.

By definition shadowing is not the exact realization of a \twot; it is a ``fuzzy window'' which
represents a local region of space-time that is in the proximity of
the \twot\ in question (in some norm). As the size of the shadowed \twot\
increases, so does the accuracy of the shadowing region. that is, away from
the boundary, the shadowing becomes exponentially more accurate.
As a testing ground for this new approach, we work with the \KSe,
a partial differential equation whose space-time is two dimensional.
The reason for the use of this equation is not just the reduction
of computational complexity.

The two dimensional space-time of the \KSe\ is far easier to visualize than the four
dimensional space-time of the Navier-Stokes equations.
This visualization makes the arguments more
understandable and compelling. A foreseeable critique is
that the \KSe\ is too simple of a playground; the methods developed
in this setting would not work in the Navier-Stokes setting.
The efficacy and potency of the \spt\ formulation,
however, is a great indication that the results can generalize
to higher dimensions. This is a very attractive proposition; there have been many attempts to
find the fundamental structures necessary for the reproduction
of turbulence in the Navier-Stokes equations.
It is possible that the proper structures have already been determined but they
have not been utilized properly; that is, they have not been used in a \spt\
formulation.

% The period AND domain size will be variables; the difficulty therein.
What are the differences and advantages of this?
The first key difference is that the governing equation
dictates the \spt\ domain size in an unsupervised
fashion; the decision of what specific domain size
to study is no longer present in the discussion.
This is another manner in which
time and space are being treated as equals; the parameters $(L,T)$
that determine the size of the \spt\ domain are both allowed to vary.
The values of these parameters
are determined by the requirement that the equations
must be satisfied locally at every lattice site.
This small detail, allowing the domain size \speriod{} to vary,
is not as trivial as it seems. At present it has not been seen in
the dynamical systems literature. The variation of the period $T$ is
common, however. The likely culprit behind this different treatment
is likely a result of the equations themselves. This difficulty
is especially evident in the \KSe, whose spatial derivative terms
are of higher order than the first order time derivative, but also
there is a spatial derivative present in the nonlinear component.

Flows described by partial differential equations (PDEs) are
said to be infinite dimensional because if one writes them
down as a set of ordinary differential equations (ODEs), a set
of infinitely many ODEs is needed to represent the dynamics
of one PDE. Even though their {\statesp} is thus
$\infty$-dimensional, the long-\-time dynamics of viscous
flows, such as Navier-Stokes, and PDEs modeling them, such as
Kuramoto-Sivashinsky, exhibits, when dissipation is high and
the system spatial extent small, apparent `low-dimensional'
dynamical behaviors. For some of these the asymptotic
dynamics is known to be confined to a finite-\-dimensional
{\em inertial manifold}.

For large spatial extent the complexity of the spatial
motions also needs to be taken into account. The systems
whose spatial correlations decay sufficiently fast, and the
attractor dimension and number of positive Lyapunov exponents
diverges with system size are said\rf{HNZks86, man90b, cross93}
to be extensive, `spatio-temporally chaotic' or `weakly
turbulent.'

We propose to study the evolution of \KS\ on the $1$\dmn\ infinite
spatial domain and develop a $2$\dmn\ symbolic dynamics for it: the
columns coding admissible time itineraries, and rows coding the
admissible spatial profiles.
The claim is that when the laws of motion
have several commuting continuous symmetries (time-translation
invariance; space-translation invariance), all continuous symmetries
directions should be treated democratically, as $(1+D)$ different
`times'. The proposal is inspired by the Gutkin and Osipov\rf{GutOsi15}
modelling of chain of $N$ coupled particle by temporal evolution of a
lattice of $N$ coupled cat maps.

We already have the two edges of this symbol plane - the $\speriod{}=22$ minimal
cell\rf{SCD07,lanCvit07} is sufficiently small that we can think of it as
a low-dimensional (``few-body'' in Gutkin and Klaus
Richter\rf{EPUR14,EDASRU14,EnUrRi15,EDUR15} condensed matter parlance)
dynamical system, the left-most column in the Gutkin and
Osipov\rf{GutOsi15} $2D$ symbolic dynamics {\spt} table (not a
1\dmn\ symbol sequence block), a column whose temporal symbolic dynamics
we will know, sooner or later. Michelson\rf{Mks86} has described the
bottom row, which, for continuous time, is discussed here in
\refsect{sect:KSeqva}, in Lan's thesis\rf{LanThesis} and his thesis-work
article\rf{lanCvit07}, unfortunately only for the reflection-invariant
subspace. The full Michelson \statesp\ is 4\dmn\ (3-dimensional dynamics
plus a parameter), and, in addition to the translation equivariance, it
is reflection equivariant which makes it \On{2}
equivariant, so slicing together with a \Zn{2} symmetry reduction is
required.

We start in \refsect{sect:KStimeInt}  by reviewing the case considered in
all of the earlier, temporal-evolution literature, \KS\ on a spatial periodic
domain of fixed width $\speriod{}$, evolved for all times $t$.
In \refsect{sect:KSspaceInt} we consider the case of temporally periodic
domain of fixed period $\period{}$,  for all positions $x$,
and
in \refsect{sect:KStwots} we consider {\spt}ly periodic torus, see
\reffig{fig:spaceTime1}.

%unused example of t=0 line in KSe
In the time independent case, the \KSe\ on the $T=0$ line,
there is exponential growth in complexity
as space tends towards infinity\rf{Mks86}. Inclusion of time
only makes this worse; how then, can infinite space-time ever
be explained? This question is answered by the existence
of very important solutions defined on small computational domains
which we have named \textit{fundamental tiles}. This is interesting
because the dynamics on these small \spt\ domains are typically
very simple; a result of there being few invariant solutions.
It is these tiles which shall form the foundation for the
\spt\ theory

Specifically, we propose to study the evolution of \KS\ on the $2$\dmn\
infinite {\spt}domain and develop a $2$\dmn\ symbolic dynamics for it:
the columns coding admissible time itineraries, and rows coding the
admissible spatial profiles. Our {\spt} method is the clear winner in
both a computational and theoretical sense. By converting to a tile based
shadowing description we have essentially removed the confounding notion
of an infinite number of infinitely complex {\twots} from the discussion.
Now we must put these ideas into practice.

To test our \spt\ ideas we require three separate numerical methods: the first
should be able to find \twots\ of arbitrary domain size. The second needs to be able to
clip or extract tiles from these \twots. Lastly, we need a method of
gluing these tiles together. All three of these techniques require the ability
to solve the optimization problem $F(\utensor,\speriod{},\period{})=0$
on an arbitrarily sized doubly periodic domain.

Long wavelength perturbations are linearly unstable, while short
wavelength perturbations are strongly contractive. For example, an
equilibrium solution can be very unstable in 5 eigen-directions, but with
strong contraction in higher, stable eigen-directions. In particular, the
equilibrium $u(\conf,\zeit)=0$ has Fourier modes as linear stability
eigenvectors. The Fourier modes who satisfy $|k|<\tildeL$ are unstable.
The most unstable mode has $k$ closest to $\frac{\tildeL}{\sqrt{2}}$.

Truncation of higher Fourier modes can by justified through the following
analysis: If the initial $\utensor_k$ are small then the bilinear term $\utensor_m
\utensor_{k-m}$ can be neglected. Then the equations become decoupled linear
equations whose solutions are exponentials. There are then a finite
number of modes growing with time. These unstable modes excite the higher
modes through the bilinear term. However, higher modes are highly damped.
Therefore the intermediate wavelengths play an important role in
maintaining a dynamic, on-average equilibrium, but truncation can be
employed so long as all important modes are kept. A consequence of this
is that an infinite-dimensional problem has become finite dimensional.

The explorations into the geometry of its {\statesp}\rf{SCD07}, the
dimension of a possible inertial manifold\rf{DCTSCD14}, and many other
studies have been done on it as it proves itself an interesting case
study of chaotic dynamical systems in continuous variables. The steady
solutions were studied by Michelson\rf{Mks86} as well as Dong and
Lan\rf{DoLa14} in attempts to categorize the geometry and symbolic
dynamics of solutions on the $T=0$ line.

Dynamical \statesp\ representations of PDEs are $\infty$-dimensional, but
attractors of dissipative, strongly contracting flows such as \KS\ are
contained within finite-dimensional inertial manifolds%
\rf{constantin_integral_1989, infdymnon, temam90, Foias1988a,
Robinson1995} in non-trivial, nonlinear ways\rf{YaTaGiChRa08, TaGiCh11,
ginelli-2007-99, WoSa07, GiChLiPo12, DCTSCD14}. While the same has been
not been proven for \NS\ flows, by now many
experimental and theoretical explorations of fluid-dynamical attractors
also lend support to a dynamical vision of turbulence: within any finite spatial
and temporal window a turbulent flow shadows a member of a finite set
of \spt\ patterns.

In the past few decades, turbulent flows witnessed computational successes
by utilizing small computational domains also known as minimal cells.
These minimal cells were chosen to be large
enough to support turbulent behavior but also small enough to
remain tractable.
The question is how to characterize and classify these patterns.
In the past few decades computational successes were made
by studying turbulent flows on small computational domains, also known as minimal cells.
These minimal cells
were chosen to be large
enough to support turbulence but also small enough to
remain computationally tractable.
The main achievement of these cells
were the calculation of unstable periodic solutions of the
Navier-Stokes equations \rf{GHCW07, HGC08, N97}. %I understand this is redundant
of admissible patterns\rf{focusPOT}.

The challenge is to characterize and classify these patterns.
So far, they were
by studying turbulent flows on small computational domains, also known as minimal cells.
These minimal cells
were chosen to be large
enough to support turbulence but also small enough to
remain computationally tractable.
The main achievement
was the accurate calculation of unstable periodic solutions for
equations such as the Navier-Stokes equations. %I understand this is redundant
These periodic solutions also known as ``exact coherent structures'' (ECS) are
identifiable by shapes and patterns
which persist across time \rf{W01, WK04}.

The {\KSe} has been used to model many physical processes such as
the laminar flame front velocity of Bunsen burners.
The main benefit other than reduced computational complexity
is the ease with which 2{\dmn} {\spt}
solutions can be visualized, namely, as a 2{\dmn} color coded scalar
field. This visualization makes our arguments more understandable as well as compelling.
in addition to making {\fpo}s easier to identify.

For a subset of physicists and mathematicians who study idealized `fully
developed,' `homogenous' turbulence the generally accepted usage is that
the `turbulent' fluid is characterized by a range of scales and an energy
cascade describable by statistic assumptions\rf{frisch}. What
experimentalists, engineers, geophysicists, astrophysicists actually
observe looks nothing like a `fully developed turbulence' \rf{JT63, Kim87, Mckeon04}.
In the physically driven wall-bounded shear flows, the turbulence is dominated
by unstable \emph{coherent structures}, that is, localized recurrent
vortices, rolls, streaks and like. The statistical assumptions fail, and
a dynamical systems description from first principles is called
for\rf{Holmes96}.

If we ban the words `turbulence' and `{\spt} chaos' from our study of
small extent systems, the relevance of what we do to larger systems is
obscured. The exact unstable coherent structures we determine pertain not
only to the spatially small `chaotic' systems, but also the spatially
large `{\spt}ly chaotic' and the spatially very large `turbulent'
systems. So, for the lack of more precise nomenclature, we take the
liberty of using the terms `chaos,' `{\spt} chaos,' and `turbulence'
interchangeably.

Generally, most studies have
been spatiotemporal, with the time being a continuous parameter that
induces a dissipative semi-flow $t \geq 0$. In this case study, we pose
the problem of finding spatiotemporal invariant solutions by assuming
doubly periodic initial conditions, transforming to a Fourier-Fourier
basis, and then solving the truncated set of nonlinear algebraic
equations
