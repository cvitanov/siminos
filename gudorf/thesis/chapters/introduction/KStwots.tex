
Here we propose to describe all solutions $u(x,t)$ of \KSe\ \refeq{e-ks} as the
closure of the union of the set of all prime (non-repeating)
\twots\
\bea
u(x,t) &=& u(x,t+\period{})
       \;=\; u(x+\speriod{},t)
\continue
       &=&  u(x+\speriod{},t+\period{})
\,,\quad
(x,t)\in([0,\speriod{}),[0,\period{}))
\,.
    \label{e-ksTorus}
\eea

Consider a torus $(\speriod{},\period{})$ with {\spt}ly
doubly-periodic $u(x,t)=u(x+\speriod{},t+\period{})$,
$(x,t)\in([0,\speriod{}),[0,\period{}))$, see
\reffig{fig:spaceTime1}\,(c), and combine the discrete spatial
Fourier modes expansion \refeq{spatFT} with the discrete temporal Fourier
modes expansion \refeq{BBtemporFourier}.
Discretize $u(x_n,t_m)= u_{nm}$ over
$N$ points of a periodic spatial lattice $x_n = n \speriod{}/N$,
and
$M$ points of a periodic temporal lattice $t_m = m \period{}/M$,
%\beq
%    \utensor_{k,m}^{(i)} = \frac{1}{M} \sum_{n = 0}^{M-1}
%    \utensor^{(i)}_{k,n} e^{i \omega_n t_m}
%    \, , \quad
%    \utensor^{(i)}_{k,n} = \sum_{m=0}^{M-1}\utensor_{k,m}^{(i)}  e^{-i \omega_n t_m}
%    \, , \quad
%    \mbox{where }
%    \omega_n = 2 \pi n / \period{} \, .
%\eeq
%
\bea
\utensor_{kj} &=&
  \frac{1}{NM} \sum^{N-1}_{n=0} \sum^{M-1}_{m=0}
  u_{mn} \, e^{-i(\wavek \xm + \freqj \tn)}
    \,,\quad
\wavek = \frac{2 \pi k}{\speriod{}}
    \,,\;
\freqj = \frac{2 \pi j}{\period{}}
    \continue
u_{mn} &=&\sum_{k=0}^{M-1}\sum^{N-1}_{j=0}
   \utensor_{kj} \, e^{i(\wavek \xm + \freqj \tn)}
\,,
\label{spattempFT}
\eea
In its Fourier discretization, \KS\ PDE \refeq{e-Fks} is a set of $MN$
algebraic equations for the {\spt} Fourier coefficients $\utensor$,
\beq
\left[ i \freqj - ( \wavek^2 - \wavek^4 ) \right]\utensor_{kj}
+ i \frac{\wavek}{2} \!\sum_{k'=0}^{M-1} \sum^{M-1}_{j'=0}\!\!
\utensor_{k'j'} \utensor_{k-k',j-j'}
    = 0
\,.
\label{e-FksSpattemp}
\eeq
In other words, we have reduced the problem of either temporal or
spatial evolution on a
{\spt}ly periodic domain to the fixed point problem of determining an
invariant 2-torus, a fixed point in the $NM$\dmn\ \statesp.

\MNG{2019-05-13}{Elimination of space and time
translations effectively reduces the codimension of each solution by two, this is
undesirable when numerically solving the \refeq{e-pseudospectral} in a least-squares sense}
First, however, we must eliminate the temporal translation marginal eigenmode
by a {\PoincSec}, and the spatial translation marginal eigenmode by a {slice}.
Generically, they both can fixed by the first Fourier mode {\PoincSec} /
{slice}.
The Newton method then requires inversion of $1-J$, \ie,
$\det(1-J)$, where $J$ is the 2-torus \jacobianM.
    %
    \PC{2018-04-08}{see also \refsect{sect:KurSivSpTmpStab}.}

We can generalize the notion of average `energy' density
\refeq{SCD07ksEnergy} and apply it to any \twot\ $p$ by also
averaging over time
\beq
    \expctE_p=
  \frac{1}{\speriod{}\period{}}\!\oint d\pSpace\oint dt \frac{u^2}{2}
  \,,
  \label{twotEnergy}
\eeq
The Fourier space average spatial energy density \refeq{SCD07EFourier}
generalized to any {\spt} solution \refeq{spattempFT} is then:
\beq
\expctE
          =  \sum_{k=0}^{N-1}\sum^{M-1}_{n=0}
             {\textstyle\frac{1}{2}}|\utensor_{kn}|^2
\,.
\ee{twotEFourier}


\subsection{Spatiotemporal $u=0$ equilibrium}
\label{sect:KSu0equiST}

    \PC{2019-03-16}{
    Incorporate here the spatial evolution stability analysis of
    blog posts starting with {\bf 2016-09-02 Matt Stability of u=0 equilibria}.
   }
   %
Consider the $\utensor_{kn}=0$ \eqv\ of \refeq{e-FksSpattemp}. Its
linearization is

\bea
i \omega_n  &=&  q_k^2 - q_k^4
    \continue
i \frac{2\pi n}{\period{}}   &=&
\left(\frac{2\pi k}{\speriod{}}\right)^2 - \left(\frac{2\pi k}{\speriod{}}\right)^4
\,.
\label{e-FksSpattemp0}
\eea
The two special cases are the spatial strip of \refsect{sect:KStimeInt},
which yields the $2N$ real temporal stability exponents
$\Lyap_k=i\omega(k)$ for a fixed spatial strip of width $\speriod{}$, and the
temporal strip of \refsect{sect:KSspaceInt}, which yields the $4+8(M-1)$ complex
spatial stability exponents $i q^{\pm,\pm}(n)$ for a fixed temporal strip of period
$\period{}$, as the four solutions of the quartic equation \refeq{e-FksSpattemp}
for each $i \omega_n$, $n = 0, \pm 1, \pm 2, \pm M/2$ :
\beq
(i q_k)^4 + (i q_k)^2 + i \omega_n  =0
    \quad\Rightarrow\quad
i q^{\pm,\pm}(n) = \pm \sqrt{(-1\pm \sqrt{1 -4 i \omega_n})/2}
\ee{spatialStabExp}
    \MNG{2016-09-20}{
    I think I should have been looking for $i q_k$ not $q_k$? If so, it
    matches the stability exponents of \refeq{PCeqvaStblty} %\refeq{MNGeqvaStblty}
    }
Each solution appears twice, corresponding to  $(\omega_n,-\omega_n)$,
except for
\beq
q^{\pm,\pm}(0) = \mp i\sqrt{(-1\pm 1)/2} =
(0,0,-1,1)
\,.
\ee{spatialStabExp0}
There are two root magnitudes for each $n \neq 0$, independent of the
sign of $n$:
%\[%beq
%|q^{\pm,\pm}(n)|^2 = (-1\pm \sqrt{1 +4 i \omega_n})
%(-1\pm \sqrt{1 -4 i \omega_n})/4
%\]%ee{spatialStabExp01}
%
\beq
|2q^{\pm,\pm}(n)|^2 = 1+ \sqrt{1 +  16 \omega_n^2}
\pm \sqrt{1 +4 i \omega_n}
\pm \sqrt{1 -4 i \omega_n}
\,.
\ee{spatialStabExp1} 