The next benefit is not due to the inclusion of new methods but
rather the exclusion of old methods. Time integration and
recurrence functions based on pairwise distance have long been used in combination
to find initial conditions. % schatz/grigoriev, kerswell, viswanath.
In the high dimensional limit, both
of these components are time consuming. This is yet another
component of the dynamical systems formulation that gets worse
as spatial sizes increase. There are two detrimental factors
that contribute towards this. The number of dimensions must increase
in order to accurately resolve the domain. The other factor is that
the growth of complexity of solutions can reduce the number of recurrences
drastically. There isn't really a manner to deal with the increasing
number of computational variables other than to wait for improvements
in computing power and memory availability. As for the recurrences, the
typical solution for increasingly rare events is to compute in parallel when
possible. The exponential growth in complexity makes even this proposition
a daunting one.

% The italicized text captures my sentiment of this next paragraph.
The \spt\ completely avoids this by constructing larger \twots\
from the combination of smaller \twots. That is,
we locate the fundamental tiles, which are easy to find due to their small
domain size, and then build them up to create larger \twots. The only
search required is the search for the fundamental tiles. To stress
this even further \textit{one of the challenges of turbulence
computations has been eliminated}. The reason
why the search for the fundamental tiles is classified as ``easy'' is because
in the small domain size limit there just aren't that many \twots; the dynamics
is relatively simple.

% I'm trying to say that the arbitrary choice of norm is not as impactful in our case....
Recurrence functions also require the introduction of a norm,
typically chosen without taking the geometry of the state space into account.
Points that are close in this norm can be far apart in a dynamical sense (\ie, on opposite sides
of an unstable manifold). An arbitrary norm is also chosen in the \spt\
context but there are some subtle differences. For starters, the
norm introduced in the \spt\ formulation is not beholden to dynamics, as
there are no longer any dynamics to speak of.
Additionally, the norm in the \spt\ case measures the distance between \twots,
not just single state space points. This is not a statement of proof but rather
a suggestion that the underlying topology improves the reliability of
the chosen norm. Restated in a different manner, the \spt\ norm
takes both the magnitude and phase into account.

Conventional methods treat spatial dimensions
as finite and fixed; meanwhile, time is treated as infinite.
One interpretation is that this is natural due to the
human familiarity with finite space, especially in regards
to experimental setups.
This assumption is actually a very unnatural one
in the context of state space. The fundamental reason
for this is that it disregards
the translational invariance of the equations and while there
are implicit physical scales, choosing a specific domain size
to study is completely arbitrary. This notion
must be reconsidered going forward as it is a very strict constraint
on the space of solutions and on the study of turbulence in general.
Finite spatial dimensions of course have practical import, but
these specific constraints should only be imposed after the
study of infinite space-time, as they represent special
cases of the general equations. The \spt\ formulation handles
this properly by treating all continuous dimensions as equal
by respecting all translational symmetries.
What are the differences and advantages of this?
The first key difference is that the governing equation
dictates the \spt\ domain size in an unsupervised
fashion; the decision of what specific domain size
to study is no longer present in the discussion.
This is another manner in which
time and space are being treated as equals; the parameters $(L,T)$
that determine the size of the \spt\ domain are both allowed to vary.
The values of these parameters
are determined by the requirement that the equations
must be satisfied locally at every lattice site.
This small detail, allowing the domain size $L$ to vary,
is not as trivial as it seems. At present it has not been seen in
the dynamical systems literature. The variation of the period $T$ is
common, however. The likely culprit behind this different treatment
is likely a result of the equations themselves. This difficulty
is especially evident in the \KSe, whose spatial derivative terms
are of higher order than the first order time derivative, but also
there is a spatial derivative present in the nonlinear component.


    The conventional method to generate initial conditions
involves time integration and recurrence functions, the latter simply
calculates the pairwise distance between all points in a time integrated
series.%schatz/grigoriev, kerswell, viswanath.
In the high dimensional limit, both
of these components are time consuming. This is yet another
component of the dynamical systems formulation that gets worse
as spatial sizes increase. There are two detrimental factors
that contribute towards this. The number of dimensions must increase
in order to accurately resolve the domain. The other factor is that
the growth of complexity of solutions can reduce the number of recurrences
drastically. There isn't really a manner to deal with the increasing
number of computational variables other than to wait for improvements
in computing power and memory availability. As for the recurrences, the
typical solution for increasingly rare events is to compute in parallel when
possible. The exponential growth in complexity makes even this proposition
a daunting one.

The \spt\ completely avoids this by constructing larger \twots\
from the combination of smaller \twots. That is,
we locate the fundamental tiles, which are easy to find due to their small
domain size, and then build them up to create larger \twots. The only
search required is the search for the fundamental tiles. To stress
this even further \textit{one of the challenges of turbulence
computations has been eliminated}. The reason
why the search for the fundamental tiles is classified as ``easy'' is because
in the small domain size limit there just aren't that many \twots; the dynamics
is relatively simple.

Recurrence functions also require the introduction of a norm,
typically chosen without taking the geometry of the state space into account.
Points that are close in this norm can be far apart in a dynamical sense (\ie, on opposite sides
of an unstable manifold). An arbitrary norm is also chosen in the \spt\
context but there are some subtle differences. For starters, the
norm introduced in the \spt\ formulation is not beholden to dynamics, as
there are no longer any dynamics to speak of.
Additionally, the norm in the \spt\ case measures the distance between \twots,
not just single state space points. This is not a statement of proof but rather
a suggestion that the underlying topology improves the reliability of
the chosen norm. Restated in a different manner, the \spt\ norm
takes both the magnitude and phase into account.
Another numerical advantage is that the \spt\ formulation is able to find solutions
of the \KSe\ starting from modulated random noise. The specifics
of ``modulated random noise'' are described in the numerical methods section
but it can essentially be thought of as randomly assigning values to \spt\
Fourier modes. The ability to find solutions from this starting point
is a radical improvement over the conventional capabilities. This is of course
in conjunction with allowing the \spt\ domain to change. The reaction to these
changes individually has induced skepticism and disbelief; together they comprise
a completely unheard of force.

The \spt\ formulation also includes the improvement of a commonly
practiced numerical method known as pseudo-arclength continuation.
The general idea is to track a solution as a parameter is varied. In the
Navier-Stokes equations this is typically the Reynolds number.
The improvement is due to our common refrain: the lack of dynamical
instability and the topological constraint of \twots. There can
be more confidence that if the continuation fails it is due to the solution
not existing rather than not being able to converge due to dynamical instability.
