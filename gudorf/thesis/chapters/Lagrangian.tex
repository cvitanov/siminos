% siminos/gudorf/thesis/chapter/Lagrangian.tex
% $Author: predrag $ $Date: 2020-05-25 15:18:45 -0400 (Mon, 25 May 2020) $

\section{Generating functions; \templatt}
\label{s:GenFctn}
% PC 2019-09-05 moved from siminos/kittens/Lagrangian.tex to here
% \PC{2019-08-03} draft 1.1
% \HL{2019-07-30} draft 1.0

                                                      \toCB
Lagrangian systems are conservative dynamical systems which have a
variational formulation.
To understand
the relation between the discrete time Hamiltonian and Lagrangian
formulations, one needs to understand the discrete mapping generating
function, such as \refeq{GutOsi15-3.1:model}.

        \PC{2019-08-05}{
In preparing this summary we have found expositions of Lagrangian
dynamics for discrete time systems by MacKay, Meiss and
Percival\rf{MKMP84,meiss92}, and Li and Tomsovic\rf{LiTom17b} particulary
helpful.
    }
% Percival and Vivaldi state the Lagrangian variational
% principle in
% Sect.~6. {\em Codes, variational principle and the static model}:

    \PC{2018-02-16}{
We need a simple explanation for why the 2\dmn\ $1-A^n$ and the
linearization of the \po\ $2n$\dmn\ Hessian matrix give the same
multipliers.
    }

%\PCedit{2019-07-30 TEMPORARY:
Consider a cat map\rf{ArnAve} of form % \refeq{eq:HamEqMot}.
 \beq
 \left(\begin{array}{c}
   \coord_{\zeit+1}  \\
   p_{\zeit+1}
  \end{array} \right )=
  A \left(\begin{array}{c}
   \coord_{\zeit}  \\
   p_\zeit
  \end{array} \right )\quad \mod 1
\,,\qquad
A = \left (
\begin{array}{cc}
s-1 & 1 \\
s-2 & 1 \\
\end{array}
    \right )
\,,
\ee{eq:CatMapIntr}
with both $\coord_{\zeit}$ and $p_\zeit$ in the unit interval,
$A$ a linear, \statesp\ (area) preserving map of a 2-torus onto itself,
and
$s=\tr{A} > 2$ an integer.
%Rewrite  back to \refeq{eq:CatMapIntr}'s form,
Implement explicitly, as in \refeq{eq:StateSpCatMap}, the $\mod 1$
operation by introducing $m^\coord$ and $m^p$ winding numbers,
 \beq
 \left(\begin{array}{c}
   \coord_{\zeit+1}  \\
   p_{\zeit+1}
  \end{array} \right )=
A
   \left(\begin{array}{c}
   \coord_\zeit  \\
   p_\zeit
  \end{array} \right )
  -
   \left(\begin{array}{c}
   m^\coord_{\zeit+1}  \\
   m^p_{\zeit+1}
  \end{array} \right )
\,.
\ee{eq:HamEqMot}

This is a non-autonomous, time-forced Hamiltonian equation of
motion of form
(\ref{PerViv2.1b},\ref{PerViv2.1a}):
\bea
\coord_{\zeit+1}
  &=& \coord_{\zeit} + p_{\zeit+1} + (\Ssym{\zeit+1}^p - \Ssym{\zeit+1}^\coord)
\label{HL1dCatMap2a}\\
p_{\zeit+1}
  &=&  p_\zeit + (s-2) \coord_\zeit - \Ssym{\zeit+1}^p \,,
\label{HL1dCatMap2b}
\eea
with the force and the corresponding potential energy given by
\bea
P(\coord_{\zeit}) = -\frac{dV(\coord_{\zeit})}{d\coord_{\zeit}} = (s-2) \coord_\zeit - \Ssym{\zeit+1}^p \, ,
\label{HL1dCatMapForce}
\\
V(\coord_{\zeit})
= \frac{1}{2} (2-s)\,\coord_\zeit^2 + \Ssym{\zeit+1}^p \coord_\zeit
\,.
\label{HL1dCatMapPotential}
\eea
As always, the  Lagrangian, or, in the parlance of discrete
time dynamics, the \emph{generating function}
$\genF(\coord_{i},\coord_{i+1})$,
is given by the difference of the kinetic and potential energies,
where in the literature\rf{MacMei83,MKMP84,meiss92,BolTre10}
 there are different choices of the instant in
time at which $V(\coord)$ is be evaluated. We define the generating function as
\[
\genF(\coord_{\zeit},\coord_{\zeit+1})
= \frac{1}{2} p_{\zeit+1}^2 - V(\coord_{\zeit})
\,.
\]
Next one eliminates momenta in favor of velocities, using \refeq{HL1dCatMap2a}
\bea
\genF(\coord_{\zeit},\coord_{\zeit+1})
&=& \frac{1}{2} (\coord_{\zeit+1} - \coord_{\zeit} - \Ssym{\zeit+1}^p + \Ssym{\zeit+1}^\coord)^2 + \frac{s-2}{2} \coord_\zeit^2 - \Ssym{\zeit+1}^p \coord_\zeit
\continue
&=& \frac{1}{2}\coord_{\zeit+1}^2 + \frac{s-1}{2} \coord_{\zeit}^2 - \coord_{\zeit}\coord_{\zeit+1}
\continue
&& - \coord_{\zeit+1} \Ssym{\zeit+1}^p + \coord_{\zeit+1} \Ssym{\zeit+1}^\coord - \coord_{\zeit} \Ssym{\zeit+1}^\coord + \mathrm{constant}
\,.
\label{HLOneStepAction}
\eea
And this generating function satisfies \refeq{MKMP843.2}.

\bigskip

Consider a symplectic (``area preserving'') map acting on phase space
\[
\ssp_{\zeit+1}= M(\ssp_\zeit)\,,\qquad \ssp_\zeit=(\coord_\zeit,p_\zeit)
\]
that maps $\ssp_\zeit$ to $\ssp_{\zeit+1}$ while preserving the symplectic area.

A {\em path} is any set of successive  \emph{configuration space} points
\beq
\{\coord_i\} = \{\coord_\zeit,\coord_{\zeit+1},\cdots,\coord_{\zeit+k}\}
\,.
\ee{pathPC}
In a Lagrangian system each path of finite
length in the configuration space is assigned an
 \emph{action}.
    \PC{2019-08-04}{
repeat of text in catLagrang.tex
        }

To get the action of an orbit from time $\zeit_0$ to $\zeit_n$, we only
need to sum \refeq{HLOneStepAction} over intermediate time steps:
\bea
\action(\coord_{\zeit_0}, \coord_{\zeit_0+1}, \dots, \coord_{\zeit_n-1}, \coord_{\zeit_n}) =
\sum_{\zeit = \zeit_0}^{\zeit_n-1} \genF(\coord_{\zeit},\coord_{\zeit+1})
\,.
\label{1DCatAction}
\eea

 For example, in a
discrete-time one-degree-of-freedom Lagrangian system with the configuration
coordinate $\coord_i$ at the discrete time $i$, and \emph{generating function}
(``Lagrangian density'') $\genF(\coord_{i},\coord_{i+1})$, the action of path
$\{\coord_i\}$ is
\beq
\action_{\zeit,\zeit+k} \equiv \sum_{i=\zeit}^{\zeit+k-1}\genF(\coord_{i},\coord_{i+1})
\,,
\ee{MKMP84(3.5)}

For 1-dof systems,
the geometrical interpretation of the action
$\action_{\zeit,\zeit+k}$ is that $\genF(\coord_{\zeit},\coord_{\zeit+1})$ is, up to an overall
constant, the phase-space area below the $p_\zeit$ to $p_{\zeit+k}$
graph for the $(\coord_\zeit,\coord_{\coord+k})$ path in the
$(\coord,p)$ phase plane.
%                                \toExam{exam:MKMP84(3.4)}

%\hfill    \fastTrackExam{exam:MKMP84(3.4)}


    \PC{2016-11-11, 2018-09-26}{
What follows is (initially)
copied from Li and Tomsovic\rf{LiTom17b}, \emph{Exact relations between
homoclinic and \po\ actions in chaotic systems} arXiv source
file, then merged with the MacKay-Meiss-Percival action
principle \refrefs{MKMP84,meiss92}.
    }

Denoting the derivatives of the generating function $\genF(\coord,\coord')$ as
\bea
\genF_{1}(\coord,\coord') &=& \frac{\partial~}{\partial \coord} \genF(\coord,\coord')
\,,\quad
\genF_{2}(\coord,\coord')= \frac{\partial~}{\partial \coord'} \genF(\coord,\coord')
    \continue
\genF_{12}(\coord,\coord') &=& \genF_{21}(\coord,\coord')
               =\frac{\partial^2}{\partial \coord\partial \coord'} \genF(\coord,\coord')
\,,
\label{MKMP84(3.1a)}
\eea
the \emph{momenta} are given by\rf{MKMP84,meiss92}
\beq
p_{n}=-\genF_{1}(\coord_{n},\coord_{n+1})
\,,\quad
p_{n+1}=\genF_{2}(\coord_{n},\coord_{n+1})
\,.
\label{MKMP84(3.2)}
\eeq
%\begin{equation}\label{eq:Definition generating function}
%\begin{split}
%&p_{n}=-\partial \genF/\partial \coord_{n}\\
%&p_{n+1}=\partial \genF/\partial \coord_{n+1}.
%\end{split}
%\end{equation}
The twist condition
\beq
\partial p_{n+1}/\partial \coord_n \neq 0 \mbox{  for all } p_{n+1}, \coord_n
\,,
\ee{MKMP84(3.1)}
ensures that
\beq
\genF_{12}(\coord_{n},\coord_{n+1}) \neq 0
    \,.
\ee{MKMP84(3.3)}
We distinguish a {\em path} \refeq{pathPC}, which is any set of
successive points $\{\coord_n\}$ in the configuration space, from the
{\em orbit segment} $M^{k}(\ssp_n)$ from $\ssp_n$ to $\ssp_{n+k}$, a set
of successive  \emph{phase space} points
\beq
\{\ssp_i\} = \{\ssp_n,\ssp_{n+1},\cdots,\ssp_{n+k}\}
\,.
\ee{trajectoryPC}
that extremizes the {\em action} \refeq{MKMP84(3.5)},
with momenta given by \refeq{MKMP84(3.2)}. In other words, not only
$\coord_n$, but also $p_n$ have to align from phase space point to phase
space point\rf{PerViv},
\beq
\frac{\partial}{\partial \coord_{n}}
    \left( \genF(\coord_{n-1},\coord_{n}) + \genF(\coord_{n},\coord_{n+1}) \right)=0
    \,.
\ee{MKMP84(3.7)}
Any finite path for which the action is stationary with respect to variations
of the segment keeping the endpoints fixed, is called an orbit segment or
\emph{trajectory}\rf{DasBuch}.
%The orbit segment is a path that satisfies a variational principle, \ie, the
%stationarity of the action \refeq{MKMP84(3.5)} at each orbit point $\coord_{i}$
%between the end points $\coord_{n}$ and $\coord_{n+k}$.
Infinite paths for which each finite segment is an orbit segment are called
\emph{orbits}.


Given by Keating\rf{Keating91}, for the 1\dmn\ cat map
\refeq{eq:StateSpCatMap}, the action of a one-step orbit (which is the
generating function) from $(\ssp_\zeit, p_\zeit)$ to $(\ssp_{\zeit+1},
p_{\zeit+1})$ can be written as \refeq{eq:OneStepAction}. And the map
\refeq{eq:HamEqMot} can be generated using\rf{MKMP84,meiss92}:
\beq
p_\zeit = - \partial \genF(\ssp_{\zeit},\ssp_{\zeit-1}) / \partial \ssp_\zeit \, , \quad p_{\zeit+1} = \partial \genF(\ssp_{\zeit},\ssp_{\zeit-1}) / \partial \ssp_{\zeit+1}
\label{MKMP843.2}
\eeq
\HL{2019-08-01}
{\refeq{eq:OneStepAction} is given by Keating\rf{Keating91} but I cannot
find the derivation of this generating function in that paper and the
papers referred\rf{PerViv87b,Keating91a}. The following derivation of
generating function is from our blog.
}


Setting the first variation of the action $\delta \action$ to 0 we get:
\bea
\frac{\partial \action}{\partial \ssp_\zeit} =
\frac{\partial \genF(\ssp_{\zeit},\ssp_{\zeit+1})}{\partial \ssp_\zeit} + \frac{\partial \genF(\ssp_{\zeit-1},\ssp_{\zeit})}{\partial \ssp_\zeit} &=& 0 \\
\Rightarrow  - \ssp_{\zeit-1} + s \ssp_\zeit - \ssp_{\zeit+1}
&=& \Ssym{\zeit+1}^x - \Ssym{\zeit}^x + \Ssym{\zeit}^p
\,.
\label{ActVariantion}
\eea
This is the \PV\ 2-step difference equation of the cat map with $\Ssym{\zeit} = \Ssym{\zeit+1}^x - \Ssym{\zeit}^x + \Ssym{\zeit}^p$.

Using \refeqs{HLOneStepAction}{1DCatAction} we can compute the action of
any finite trajectory. For a trajectory $\dots \ssp_{\zeit-1}
\ssp_{\zeit} \ssp_{\zeit+1} \ssp_{\zeit+2} \dots$, the action can be
written as:
\beq
\action(\bi{x}) = \frac{1}{2} \transp{\bi{\ssp}} H \bi{\ssp} - \transp{\bi{\Ssym{}}} \bi{\ssp}
\,,
\ee{1DCatOrbitAction}
where $\bi{\ssp}$ and $\bi{\Ssym{}}$ are column vectors,
\bea
\bi{\ssp}
=
\left[
\begin{array}{c}
\vdots \\ \ssp_{\zeit-1} \\ \ssp_{\zeit} \\ \ssp_{\zeit+1} \\ \ssp_{\zeit+2} \\ \vdots
\end{array}
\right]
\,,
\quad
\bi{\Ssym{}}
=
\left[
\begin{array}{c}
\vdots \\ \Ssym{\zeit-1} \\ \Ssym{\zeit} \\ \Ssym{\zeit+1} \\ \Ssym{\zeit+2} \\ \vdots
\end{array}
\right]
\,,
\label{eq:VectorFieldSourceNonPeriodic}
\eea
and the Hessian $H$ is a Toeplitz matrix
\bea
H
=
\left[\begin{array}{ccccccccc}
\ddots & \ddots & \ddots & \ddots & \ddots & \ddots & \ddots & \ddots & \ddots \\
\ddots &  s & -1 & 0 & 0 &\dots & 0 & 0 & \ddots \\
\ddots & -1 &  s & -1 & 0 &\dots &0&0 & \ddots\\
\ddots & 0 & -1 &  s & -1 &\dots &0 & 0 & \ddots\\
\ddots & \vdots & \vdots &\ddots & \ddots & \ddots &\vdots &\vdots & \ddots\\
\ddots & 0 & 0 & \dots &\dots &\dots  & s & -1 & \ddots \\
\ddots & 0 & 0 & \dots &  \dots &\dots&-1 &  s & \ddots \\
\ddots & \ddots & \ddots & \ddots &  \ddots &\ddots & \ddots &  \ddots & \ddots
        \end{array} \right]
 \, .
\label{eq:HessianMatrixNonPeriodic}
\eea
For an orbit with finite length, we need to know the {\bcs}
to find the action at boundaries. Note that the action computed in this
way will not have the constant terms in \refeq{HLOneStepAction}. The
matrix $H$ has same effect as $(s-\Box -2)$ where the $\Box$ is the
discrete one\dmn\ Laplacian defined in \refeq{PerViv2.2}.

\bigskip
============ the rest: TEMPORARY TEXT ===========
% Predrag 2019-07-30
% a block of text extracted from siminos/spatiotemp/chapter/Hill.tex
% remove \HLpost{2018-09-26}{ ... }  from there when done here

\subsection{Temporary: Cat map in the Lagrangian formulation}

Rewrite \refeq{eq:StateSpCatMap} back to \refeq{eq:CatMapIntr}'s form,
and let $m^\coord$ and $m^p$ be the winding numbers, we can get
    \PC{2019-08-04}{
\PV\rf{PerViv} (3.1) uses only $m^p$, no need for this confusing
additional $m^\coord$, for their
Hamiltonian (2.1), with no specialization to the \PV\ cat map.
                    }
 \beq
 \left(\begin{array}{c}
   \coord_{\zeit+1}  \\
   p_{\zeit+1}
  \end{array} \right )=
   \left (
\begin{array}{cc}
s-1 & 1 \\
s-2 & 1 \\
\end{array}
    \right )
   \left(\begin{array}{c}
   \coord_\zeit  \\
   p_\zeit
  \end{array} \right )
  -
   \left(\begin{array}{c}
   m^\coord_{\zeit+1}  \\
   m^p_{\zeit+1}
  \end{array} \right )
\,.
\ee{eq:HamEqMotion}
The action of the system in this one-step motion is\rf{Keating91}
%(see \refappe{s:GenFctn})
    \PC{2019-08-04}{
By MacKay, Meiss and Percival\rf{MKMP84,meiss92} convention (3.2),
and Li and Tomsovic\rf{LiTom17b} convention (9) we
should always have $\genF(\coord_{\zeit},\coord_{\zeit+1})$  .
Unfortunately Keating\rf{Keating91} definition (3) corresponds to
$\genF(\coord_{\zeit},\coord_{\zeit-1})$, but we do not take that one.
                    }
\beq
\genF(\coord_{\zeit},\coord_{\zeit-1})
= \frac{1}{2}[ (s-1) \coord_{\zeit-1}^2
               - 2 \coord_{\zeit-1} (\coord_{\zeit} + m_{\zeit}^\coord)
         + (\coord_{\zeit} + m_{\zeit}^\coord)^2 - 2 m_{\zeit}^p \coord_{\zeit}]
\,.
\ee{eq:OneStepAction}
The action of a longer orbit is the sum of the one-step actions at each
time step. The Lagrangian equations of motion are obtained by
demanding that the first variation of the action
vanishes:
\bea
\frac{\partial \genF(\coord_{\zeit+1},\coord_\zeit)}{\partial \coord_\zeit} + \frac{\partial \genF(\coord_{\zeit},\coord_{\zeit-1})}{\partial \coord_\zeit} &=& 0 \\
 - \coord_{\zeit-1} + s \coord_\zeit - \coord_{\zeit+1}
&=& m_{\zeit+1}^\coord - m_{\zeit}^\coord + m_{\zeit}^p
= m_t
\,,
\label{eq:ActVar}
\eea
which gives us the {\sPe} \refeq{eq:CatMapNewt} with
$m_t=m_{\zeit+1}^\coord - m_{\zeit}^\coord + m_{\zeit}^p$.
If the orbit has periodic {\bcs} with period $\cl{}$,
$\coord_{\zeit} = \coord_{\zeit+n}$, the action of the periodic orbit can be
written as \refeq{pAction},
% \beq
%\action(\bi{x}) = \frac{1}{2} \bi{x}^T H_n \bi{x} - \bi{m}^T \bi{x}
%\,,
%\ee{eq:PeriodicOrbitAction}
where the $n\!\times\!n$ matrix $H_n$ is given by
\refeq{Hessian},
and
\bea
\bi{x}
=
\left[
\begin{array}{c}
\ssp_1 \\ \ssp_2 \\ \ssp_3 \\ \vdots \\ \ssp_\cl{}
\end{array}
\right]
\, ,
\quad
\bi{m}
=
\left[
\begin{array}{c}
m_1 \\ m_2 \\ m_3 \\ \vdots \\ m_\cl{}
\end{array}
\right]
\, .
\label{eq:VectorFieldSource}
\eea
$H_{\cl{}}$ is called the Hessian matrix of period $\cl{}$.
The element of matrix $H_{\cl{}}$ is
$(H_{\cl{}})_{ij} = \partial^2 \genF(\bi{x}) / \partial \ssp_i \partial \ssp_j$.
Letting the first derivative of action \refeq{pAction} be 0,
we can see that a periodic point of cat map with period $\cl{}$ satisfies:
\bea
\left[\begin{array}{cccccc}
s & -1 & 0 & \dots & -1\\
-1 & s & -1 & \dots & 0 \\
0 & -1 & s & \dots & 0 \\
\vdots & \vdots & \vdots & \ddots & \vdots \\
-1 & 0 & 0 & \dots & s
        \end{array} \right]
\left[
\begin{array}{c}
\ssp_1 \\ \ssp_2 \\ \ssp_3 \\ \vdots \\ \ssp_\cl{}
\end{array}
\right]
=
\left[
\begin{array}{c}
m_1 \\ m_2 \\ m_3 \\ \vdots \\ m_\cl{}
\end{array}
\right] \, ,
\left[
\begin{array}{c}
m_1 \\ m_2 \\ m_3 \\ \vdots \\ m_\cl{}
\end{array}
\right]
 \in \integers^\cl{} \, ,
\label{eq:LargrangianPerOrbits}
\eea

    \PC{2019-05-27}{We have more detailed discussion in the blog, for example
    \refeq{HL1dCatMap3} in {\em spatiotemp/chapter/Hill.tex} temporarily included
    in this paper draft. See also {\em spatiotemp/chapter/examCatMap.tex} text:
    \emph{generating function} \refeq{MKMP84(3.6)HL1}
    \index{generating function}
    This generating function is the discrete time Lagrangian for a  particle moving
    in potential $V(x)$.
    }
\beq
\genF(\ssp_{\zeit+1},\ssp_\zeit)
= \frac{1}{2}[ (s-1) \ssp_\zeit^2 - 2 \ssp_\zeit (\ssp_{\zeit+1} + m_{\zeit+1}^x)
         + (\ssp_{\zeit+1} + m_{\zeit+1}^x)^2 - 2 m_{\zeit+1}^p \ssp_{\zeit+1}]
\,.
\ee{eq:LagrangianOneStepAction1}
	\HL{2019-06-10}{
	\refeq{eq:OneStepAction} is already given by Keating
\rf{Keating91}. Do we want to add our procedure here? I got the
Lagrangian \refeq{HL1dCatMap4} which is different from
\refeq{eq:LagrangianOneStepAction1} only by a constant.
	}
	\HL{2019-06-12}{
	The generating function of a 2\dmn\ \catlatt\
    \refeq{GutOsi15-3.1:model} in given by Gutkin and Osipov\rf{GutOsi15}.
	}

% \HLpost{2018-09-26}{
Consider cat map of form \refeq{eq:CatMapIntr}.
\beq
L(q_{\zeit},q_{\zeit+1}) =
\frac{1}{2}(q_{\zeit+1} - q_{\zeit})^2 - V(q_{\zeit})
    \,,\qquad
P(q) = - \frac{dV(q)}{dq}
\,,
\ee{MKMP84(3.6)HL1}

The problem with formulation \refeq{HL1dCatMapPotential} is that
the potential energy contribution is
defined asymmetrically in \refeq{MKMP84(3.6)HL1}.
We should really follow Bolotin and Treschev\rf{BolTre10} eq.~(2.5),
and define a symmetric generating function
\beq
L(q_{\zeit},q_{\zeit+1}) =
\frac{1}{2}  (q_{\zeit+1} - q_{\zeit})^2
    - \frac{1}{2} [V(q_{\zeit}) +V(q_{\zeit+1})]
\,,
\ee{MKMP84(3.6)}
The first
variation \refeq{MacMei83(7)} of the action vanishes,
\bea
0 &=& L_2(q_{\zeit+1}, q_\zeit) + L_1(q_{\zeit}, q_{\zeit-1})
                            \label{1dCatMap5}\\
&=& q_\zeit -q_{\zeit+1} + (s-2)q_\zeit - \Ssym{\zeit+1}^p + q_\zeit - q_{\zeit-1} \continue
&=& - q_{\zeit+1} + s q_\zeit -q_{\zeit-1} - \Ssym{\zeit+1}^p
\,,
\nnu
\eea
hence
\beq
q_{\zeit+1} - s q_\zeit + q_{\zeit-1} = - \Ssym{\zeit+1}^p
\,.
\label{1dCatMap5a}
\eeq
Defining $\Ssym{\zeit} = - \Ssym{\zeit+1}^p$,
we recover the {\sPe} \refeq{eq:CatMapNewt}.

Alternatively,
Han's generating function (1-step Lagrangian density) is:
\bea
L(q_{n+1}, q_n)
&=& \frac{1}{2} \left[ p_{n+1}(q_{n+1}, q_n) \right]^2 - V(q_n)
\label{HL1dCatMap4a}\\
&=& \frac{1}{2} (q_{n+1} - q_n +m_{n+1}^q - m_{n+1}^p)^2 + \frac{s-2}{2} q_n^2 -m^p_{n+1} q_n \, . \continue
\nnu
\eea
The action is the sum over the Lagrangian density over the orbit. The first
variation \refeq{MacMei83(7)} of the action vanishes,
\bea
0 &=& L_2(q_{n+1}, q_n) + L_1(q_{n}, q_{n-1})
    \continue
&=& q_n -q_{n+1} + m_{n+1}^p - m_{n+1}^q
\label{1dCatMap5b}\\
&& + (s-2)q_n -m^p_{n+1} + q_n - q_{n-1} + m_n^q - m_n^p \continue
&=& - q_{n+1} + s q_n -q_{n-1} - (m_{n+1}^q - m_n^q + m_n^p)
\,,
\nnu
\eea
hence
\beq
- q_{n+1} + s q_n -q_{n-1} = m_{n+1}^q - m_n^q + m_n^p
\,.
\label{1dCatMap5c}
\eeq
Letting $m_n = m_{n+1}^q - m_n^q + m_n^p$, we recover the Lagrangian
formulation \refeq{eq:CatMapNewt},
    \PCedit{
except for the wrong sign for $m_n$.
    }
Now we see why $m_n$'s are called `sources'.
