% siminos/gudorf/thesis/chapter/orbithunter.tex
% $Author: predrag $ $Date: 2020-05-25 15:18:45 -0400 (Mon, 25 May 2020) $


\chapter{Tori hunter}
\label{chap:torihunter}

% Docstring and description of python package.
%\MNGpost{2020-2-2}{
\begin{description}
\item[Introduction]
In order to expand the user base and influence of our \spt\ formulation
I converted my research code into a Python package that can be distributed
as an open source project on Github. The original code fell into the classic
pitfall of being only well known to the author. The newer version of the code
is incredibly user friendly and tries to model itself after very popular
python packages

The following package is a result of the complete rewrite of my
research code to maximize user friendliness and computational speed.
Because this is a complete refactoring of my research code there
are likely still a few bugs. It also should be noted that this is
a completely independent endeavor of somebody with a Physics background;
there might be some details or choices in the code that are suboptimal.
This is why this code will be available as an open source github repository.

Main benefits of using this package. Plugging in your own Torus class.
It has all been vectorized and so it's pretty fast. It's incredibly user-friendly.
It has many different numerical methods available, although there are
a number that can be used only if constraints are included as they
require square linear systems (such as SciPy's implementation of GMRES).

The other main goal of this project was to make the code modular; obviously
the code has been written around the \KSe\ but it can be tailored to
encompass any equation and any dimensional torus. The main
requirements to interface with the current package are the
following:

A function which returns a class instance in terms of real-valued spectral basis; it doesn't
necessarily have to be Fourier, but the function is required to be called this for compatibility sake.
This might be changed upon release just to prevent confusion; it's annoying but it would only
take the renaming of functions. Note that for a system with $D+1$ dimensional space-time, this
must take a physical field to $D$ dimensional spectral basis, even though it acts on the entire
$D+1$ dimensional torus. This was done because it is natural for those who simulate fluid turbulence;
most codes are spectral in nature and so these transforms are already implemented.

A function which returns a class instance in terms of real-valued \spt\ spectral basis. The requirements
follow the same as the spatial transform.

A function which calculates the equations in the \spt\ \Fcs\

\end{description}
