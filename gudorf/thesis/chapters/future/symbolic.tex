% siminos/gudorf/thesis/chapter/symbolic.tex
% $Author: predrag $ $Date: 2020-10-24 01:45:26 -0400 (Sat, 24 Oct 2020) $

\section{Symbolic dynamics: a glossary}
\label{s-SymbDynGloss}

%%%%%% ChaosBook convention start %%%%%%%%%%%%%%%%%%%%%
\renewcommand{\statesp}{state space}
\renewcommand{\Statesp}{State space}
\renewcommand{\stateDsp}{state-space}
\renewcommand{\StateDsp}{State-space}

Analysis of a low\dmn\ chaotic dynamical system typically
starts\rf{DasBuch} with establishing that a flow is locally stretching, globally
folding. The flow is then reduced to a discrete time return map by appropriate
Poincar\'e sections. Its state space is partitioned, the partitions labeled by an
alphabet, and the qualitatively distinct solutions classified by their temporal
symbol sequences. Thus our analysis of the cat map and the {\catlatt} requires
recalling and generalising a few standard symbolic dynamics notions.

%\noindent
{\bf Partitions, alphabets.}
A division of {\statesp} $\pS$ into a disjoint union of distinct regions
$\pS_A,\pS_B,\ldots,\pS_Z$ constitutes a {\em
partition}. Label each region by a symbol $\Ssym{}$ from an
$N$-letter  {\em alphabet}
$\A=\{A,B,C,\cdots,Z\}$, where $N=\cl{\A}$ is
the number of such regions. Alternatively, one can distinguish different
regions by coloring them, with colors serving as the ``letters'' of the
alphabet.
% missing in kittens:  , as in \reffigs{fig:SingleCatPartit}{fig:AKScloseActSp}.
For notational convenience, in alphabets we sometimes denote negative integer
$\Ssym{}$ by underlining it, as in
\(
\A = \{ -{2}, -{1}, 0, 1\}
   = \{ \underline{2}, \underline{1}, 0, 1\}
\,.
\)


%\noindent
{\bf Itineraries.}
For a dynamical system evolving in time,
every {\statesp} point $\xInit \in \pS$ has the {\em future itinerary},
an infinite sequence of symbols
$\Sfuture(\xInit)=\Ssym{1}\Ssym{2}\Ssym{3}\cdots$ which indicates the
temporal order in which the regions shall be visited. Given a trajectory
$\ssp_1,\ssp_2,\ssp_3,\cdots$ of the initial point $\xInit$ generated
by a time-evolution law
\( %beq
   \ssp_{n+1}=f(\ssp_n)
    % \,, \quad \ssp_0=\xInit
\,,
\) %ee{CMx-iterated}
the itinerary is given by the symbol sequence
\beq
   \Ssym{n} = \Ssym{} \qquad \mbox{if\ } \qquad  \ssp_n \in \pS_{\Ssym{}}
 \,.
\ee{CMsymbol-def}
The {\em past itinerary} $\Spast(\xInit)=\cdots\Ssym{-2}
\Ssym{-1}\Ssym{0} $ describes the order in which the regions were visited
up to arriving to the point $\xInit$. Each point $\xInit$ thus has
associated with it the bi-infinite itinerary
\beq
\Sbiinf(\xInit) % = (\Ssym{k})_{k\in \integers}
        = \Spast.\Sfuture  =
 \biinf{\Ssym{-2}\Ssym{-1}\Ssym{0}}{\Ssym{1}\Ssym{2} \Ssym{3}}
\,,
\ee{CMbiifs}
or simply `itinerary', if we chose not to use the decimal point
to indicate the present,
\beq
   \{\Ssym{\zeit}\} = \cdots\Ssym{-2}\Ssym{-1}\Ssym{0}\Ssym{1}\Ssym{2} \Ssym{3}\cdots
\ee{Itinerary}


%\noindent
{\bf Shifts.}
A forward iteration of temporal dynamics $x\rightarrow x' = f(x)$ shifts
the entire itinerary to the left through the `decimal point'. This
operation, denoted by the shift operator \shift{},
\beq
   \shift{}(\biinf{\Ssym{-2}\Ssym{-1}\Ssym{0}}{\Ssym{1}\Ssym{2} \Ssym{3}})
     =  \biinf{ \Ssym{-2}\Ssym{-1}\Ssym{0}\Ssym{1}}{ \Ssym{2} \Ssym{3}}
\,,
\ee{CMshift-s}
demotes the current partition label $\Ssym{1}$ from the future $\Sfuture$
to the past $\Spast$.
The inverse shift $\shift{}^{-1}$ shifts the entire itinerary one step
to the right.

The set of all itineraries that can be formed from the letters of the
alphabet $\A$ is called the {\em full shift}
\beq
% \A^\integers 2017-08-05 Predrag dropped this notation
\hat{\AdmItnr} = \{ (\Ssym{k})
              : \Ssym{k} \in \A \quad \mbox{for all} \quad k \in  \integers \}
\,.
\ee{CMFullSh}

The itinerary is infinite for any trapped (non-escaping or \nws\ orbit) orbit
(such as an orbit that stays on a chaotic
repeller), and infinitely repeating for a periodic orbit $p$ of period \cl{p}.
A map $f$ is said to be a \emph{horseshoe} if its restriction to the \nws\ is
hyperbolic and topologically conjugate to the full $\A$-shift.

%\noindent
{\bf Lattices.}
Consider a $d$\dmn\ hypercubic lattice infinite in extent, with each site
labeled by $d$ integers $z\in \integers^{d}$. Assign to each site $z$ a
letter \Ssym{z}\ from a finite alphabet $\A$. A particular fixed set
of letters  \Ssym{z}\ corresponds to a particular lattice state
\(
\Mm= \{\Ssym{z}\} % \in \A \,,\; z\in \integers^d \}
\,.
\)
%infinite in extent along all directions.
In other words, a $d$\dmn\ lattice requires a {$d$\dmn\ code}
\(
% \{\m_{z}\}
\Mm = \{\m_{n_1 n_2 \cdots n_d}\}
%\,,
\)
for a complete specification of the corresponding state $\Xx$.
In the lattice case, the {\em full shift} is the set of all $d$\dmn\
symbol \brick s that can be formed from the letters of the alphabet $\A$
\beq
% \A^{\integers^d}   2017-08-05 Predrag dropped this notation
\hat{\AdmItnr} = \{ \{\Ssym{z}\} % (\Ssym{z}) %_{z\in\integers^d}
              : \Ssym{z} \in \A \quad \mbox{for all} \quad z \in  \integers^d\}
\,.
\ee{LatticeFullSh}

%\noindent
{\bf Commuting discrete translations.}
%{\bf .}
%%%%%%%%%%%%%%%%%%%%%%%%%%%%%%%%%%%%%%%%%%%%%%%%%%%%%%%%%%%%%%%%%%%%%%%%
%    \PC{2016-01-12} {
% in the spirit of \refRefs{PetCorBol07}:
For an autonomous dynamical system, the evolution law $f$ is of the same form for
all times. If $f$ is also of the same form at every lattice site, the group of
lattice translations (sometimes called multidimensional shifts), acting along
$j$th lattice direction by shift $\shift{j}$, is a spatial symmetry that commutes
with the temporal evolution. A temporal mapping $f$ that satisfies
$f\circ\shift{j}=\shift{j}\circ{f}$ along the $d\!-\!1$ spatial lattice directions
is said to be {\em shift invariant}, with the associated symmetry of dynamics
given by the $d$\dmn\ group of discrete {\spt} translations.

\bigskip

Assign to each site $z$ a
letter \Ssym{z}\ from the alphabet $\A$. A particular fixed set
of letters  \Ssym{z}\ corresponds to a particular lattice symbol array
\(
\Mm= \{\Ssym{z}\} % \in \A \,,\; z\in \integers^d \}
 = \{\Ssym{n_1 n_2 \cdots n_d}\}
\,,
\)
which yields a complete specification of the corresponding state $\Xx$.
In the lattice case, the {\em full shift} is the set of all $d$\dmn\
symbol arrays that can be formed from the letters of the alphabet $\A$

as in \refeq{LatticeFullSh}

A $d$\dmn\ {\spt} field
\(
\Xx=\{\ssp_{z}\}
\)
is determined by the corresponding {\em $d$\dmn} {\spt}
symbol array
\(
\Mm=\{\Ssym{z}\}
\,.
\)
Consider next a finite \brick\ of symbols $\Mm_{\R}\subset\Mm$,
over a finite rectangular $[\speriod{1}\!\times\!\speriod{2}\!\times\cdots\times\!\speriod{d}]$
lattice region $\R\subset \integers^d$.
In particular, let $\Mm_{p}$ over a finite rectangular
$[\speriod{1}\!\times\!\speriod{2}\!\times\cdots\times\!\speriod{d}]$ lattice region be the
$[\speriod{1}\!\times\!\speriod{2}\!\times\cdots\times\!\speriod{d}]$ $d$-periodic \brick\ of
\Mm\ whose repeats tile $\integers^d$.

%\noindent
{\bf {\Brick s}.} In the case of temporal dynamics, a finite itinerary
\\
$\Mm_{\R}={\Ssym{k+1}\Ssym{k+2}\cdots\Ssym{k+\speriod{}}}$ of symbols from
$\A$ is called a {\em \brick} of length $\speriod{}=\cl{\R}$. More generally, let
$\R\subset\integers^d$  be a
$[\speriod{1}\!\times\!\speriod{2}\!\times\!\cdots\speriod{d}]$ rectangular lattice region,
$\speriod{k}\geq1$,
whose lower left corner is the $n=(n_{1}n_{2}\cdots{n_{d}})$ lattice site
\beq
  \R = \R_{n}^{[\speriod{1}\!\times\!\speriod{2}\!\times\!\cdots\speriod{d}]}
  =\{(n_1+j_1,\cdots n_d+j_d) \mid 0\leq j_k\leq \speriod{k}-1\}
\,.
\ee{dDimRect}
The associated finite {\brick} of symbols $\Ssym{z}\in\A$ restricted to  \R,
\(
\Mm_{\R}=\{\Ssym{z}| z\in \R \} \subset \Mm
\)
is called the \brick\ $\Mm_{\R}$ of volume
$\cl{\R} = \speriod{1}\speriod{2}\cdots\speriod{d}$. For example, for a 2\dmn\ lattice
a
$\R = [3\!\times\!2]$ \brick\ is of form
\beq
\Mm_{\R}=\left[\begin{array}{c}
\Ssym{12}\ \Ssym{22}\ \Ssym{32}\\
\Ssym{11}\ \Ssym{21}\ \Ssym{31}
\end{array}\right]
\ee{3times2brick}
and volume (in this case, an area) equals $3\times 2 = 6$.
In our convention, the first index is `space', increasing from left to right,
and the second index is `time', increasing from bottom up.

%\noindent
{\bf Cylinder sets.}
While a particular {\admissible} infinite symbol array
\(
\Mm= \{\Ssym{z}\} % \in \A \,,\; z\in \integers^d \}
\)
defines a point $\Xx$ (a unique lattice state) in the \statesp,
the \emph{cylinder set}
$\pS_{\Mm_{\R}}$,
% $ \pS_{\R}$,
corresponds to the totality  of
\statesp\ points $\Xx$ that share the same given finite {\brick} $\Mm_{\R}$
symbolic representation over the region \R. For example, in $d=1$ case
\beq
\pS_{\Mm_{\R}} =
    \{\cdots a_{-2} a_{-1}\,.\,
   \Ssym{1}\Ssym{2}\cdots \Ssym{\speriod{}}
   a_{\speriod{}+1}a_{\speriod{}+2}\cdots\}
\,,
\ee{finiteBlock}
with the symbols  $a_{j}$ outside of the {\brick}
$\Mm_{\R}=[\Ssym{1}\Ssym{2}\cdots \Ssym{\speriod{}}]$
unspecified.
\index{block!finite sequence}
\index{cylinder!set}

%\noindent
{\bf \Po s, \dtors.}
A {\statesp} point $\ssp_z\in\Xx$ is {\spt}ly
{\em periodic}, $\ssp_z=\ssp_{z+\speriod{}}$, if its spacetime orbit returns to it
after a finite lattice shift
\(
\speriod{}= (\speriod{1},\speriod{2},\cdots,\speriod{d})
\)
over region $\R$ defined in \refeq{dDimRect}.
The infinity of repeats of the corresponding {\brick} $\Mm_{\R}$ then tiles the lattice.
For a {\spt}ly {periodic} state $\Xx$, a {\em prime} {\brick}
$\Mm_{p}$ (or $p$) is a smallest such \brick\
\(
\speriod{p}= (\speriod{1},\speriod{2},\cdots,\speriod{d})
\)
that cannot itself be tiled by repeats of a shorter {\brick}.

The periodic tiling of the lattice by the infinitely many repeats of a prime
{\brick} is denoted by a bar: $\cycle{\Mm}_{p}$. We shall omit the bar whenever
it is clear from the context that the state is periodic.
    \PC{2019-01-19}{eliminate \prune{ \Ssym{-m+1}\cdots \Ssym{0}} and
    \rctngl{ \Ssym{-m+1}\cdots \Ssym{0}},
    notation in favor a single convention}
    \PC{2018-11-07}{
    Generalize to \dtors.
    }


In $d=1$ dimensions, prime {\brick} is called a {\em prime} cycle $p$, or a
single traversal of the orbit; its label is a {\brick} of $\cl{p}$ symbols that
cannot be written as a repeat of a shorter {\brick}.
Each {\em periodic point}
\(
      \ssp_{ \Ssym{1} \Ssym{2} \cdots \Ssym{\cl{p}}}
\)
is then labeled by the starting symbol $\Ssym{1}$, followed by
the next $(\cl{p}-1)$ steps of its future itinerary.
The set of periodic points $\pS_p$ that belong to a given periodic orbit
form a {\em cycle}
\beq
p =  \cycle{ \Ssym{1} \Ssym{2} \cdots \Ssym{\cl{p}}}
  = \{
      \ssp_{ \Ssym{1} \Ssym{2}\cdots \Ssym{\cl{p}}},
      \ssp_{ \Ssym{2} \cdots \Ssym{\cl{p}} \Ssym{1}},
    \cdots,
      \ssp_{ \Ssym{\cl{p}} \Ssym{1}\cdots \Ssym{\cl{p}-1}}
     \}
\,.
\ee{PeriodCyc}

More generally, a {\statesp} point is {\em {\spt}ly periodic} if
it belongs to an \dtor, \ie, its symbolic representation is a \brick\
over region $\R$ defined by \refeq{dDimRect},
\beq
  \Mm_{p} = \Mm_{\R}
  \,,\qquad
  \R = \R_{0}^{[\speriod{1}\!\times\!\speriod{2}\!\times\cdots\times\!\speriod{d}]}
\,,
\ee{dTorus}
that
tiles the lattice state  $\Mm$ periodically, with period $\speriod{j}$ in the
$j$th lattice direction.


%\noindent
{\bf Generating partitions.}
A temporal partition is called {\em generating} if every bi-infinite itinerary
corresponds to a distinct point in {\statesp}.
In practice almost any
generating partition of interest is infinite.
Even when the dynamics assigns a unique infinite itinerary
$\biinf{\Ssym{-2}\Ssym{-1}\Ssym{0}}{\Ssym{1}\Ssym{2} \Ssym{3}}$ to each
distinct orbit, there generically exist full shift itineraries
\refeq{CMFullSh} which are not realized as orbits; such sequences are
called {\em \inadmissible}, and we say that the symbolic dynamics is {\em
pruned}.

%\noindent
{\bf Dynamical partitions.}
If the symbols outside of given temporal {\brick} $b$ remain unspecified, the
set of all {\admissible} {\brick s} of length $\cl{b}$ yield a dynamically
generated partition of the \statesp, $\pS = \cup_b \pS_b$.

%\noindent
{\bf Subshifts.}
A dynamical system $(\pS,f)$ given by a mapping $f : \pS \to \pS$
together with a {partition} $\A$ induces {\em topological dynamics}
$(\AdmItnr,\shift{})$, where the {\em subshift}
\beq
\AdmItnr = \{  (\Ssym{k})_{k\in \integers} \}
\,,
\ee{subshift}
is the set of all {\em \admissible} itineraries, and
$ \shift{} \,:\, \AdmItnr \to \AdmItnr $ is the shift
operator \refeq{CMshift-s}.
The designation `subshift' comes from the fact that
$\AdmItnr$  % \subset \hat{\AdmItnr}$
% \A^\integers 2017-08-05 Predrag dropped this notation
 is the subset of the full shift.

%%%%%%%%%%%%%%%%%%%%%%%%%%%%%%%%%%%%%%%%%%%%%%%%%%%%%%%%%%%%%%%%%%%%%%%%
%    \PC{2016-10-11} { inspired by {Ban} \etal\rf{BHLL11}
Let $\hat{\AdmItnr}$ be the full lattice shift  \refeq{CMFullSh}, \ie,
the set of all possible lattice state $\Mm$ labelings by the alphabet
$\A$, and $\hat{\AdmItnr}(\Mm_{\R})$ is
the set of such {\brick s} over a region {\R}. The principal task
in developing the symbolic dynamics of a dynamical system is to determine
$\AdmItnr$, the set of all \emph{{\admissible}} itineraries/lattice states,
\ie, all states that can be realized by the given system.

%\noindent
{\bf Pruning, grammars, recoding.}
If certain states are {\inadmissible}, the alphabet must be supplemented by a
{\em grammar},
a set of pruning rules.
Suppose that
the grammar can be stated as a finite number of pruning rules, each
forbidding a {\brick} of finite size,
\beq
 {\cal G} = \left\{
        b_1, b_2, \cdots b_k
        \right\}
\,,
\ee{grammar}
where a {\em pruned {\brick}} $b$ is an array of symbols defined over a
finite $\R$ lattice region of size
$[\speriod{1}\!\times\!\speriod{2}\!\times\cdots\times\speriod{d}]$. In
this case we can construct a finite Markov partition by replacing finite
size \brick s of the original partition by letters of a new alphabet. In
the case of a 1\dmn, the temporal lattice, if the longest forbidden {\brick}
is of length $L+1$, we say that the symbolic dynamics is Markov, a shift
of finite type with {$L$-step memory}.

%\noindent
{\bf Subshifts of finite type.}
A {topological dynamical system} $(\AdmItnr,\shift{})$ for which all
{\admissible} states $\Mm$ are generated by recursive application
of the finite set of pruning rules \refeq{grammar}
%of the  finite transition matrix
%\beq
%\AdmItnr = \left\{ (\Ssym{k})_{k\in \integers}
%           \,:\,
%                 T_{\Ssym{k} \Ssym{k+1}} = 1
%        \quad \mbox{for all $k$} \right\}
%\ee{AdmItnr}
is called a subshift of {\em finite type}.

                                            \toCB % to kneading.tex
If a map can be topologically conjugated to a linear map, the symbolic
dynamics of the linear map offers a dramatically simplified description
of all {\admissible} solutions of the original flow, with the temporal
symbolic dynamics and the state space dynamics related by linear recoding
formulas. For example, if a map of an interval, such as a parabola, can
be conjugated to a piecewise linear map, the kneading theory\rf{MilThu88}
classifies \emph{all} of its {\admissible} orbits.

%%%%%% ChaosBook convention,  BORIS conventions start %%%%%%%%%%%%%%%%%%%%%
\renewcommand{\statesp}{phase space}
\renewcommand{\Statesp}{Phase space}
\renewcommand{\stateDsp}{phase-space}
\renewcommand{\StateDsp}{Phase-space}
%%%%%% BORIS convention end   %%%%%%%%%%%%%%%

% siminos/gudorf/thesis/chapter/SpatTempSymbDyn.tex
% $Author: predrag $ $Date: 2020-10-24 01:45:26 -0400 (Sat, 24 Oct 2020) $

% called by
%           siminos/spatiotemp/chapter/spatiotemp.tex
%           siminos/tiles/GuBuCv17.tex

%\section{Spatiotemporal symbolic dynamics}
%\label{sect:SpatTempSymbDyn}

\renewcommand{\shift}{\ensuremath{\sigma}}
\renewcommand{\Xx}{\ensuremath{\mathsf{U}}}

%\PCpost{2018-07-27, 2018-08-08}{ {\bf spacetime indices convention}
The square lattice discretization $u_z$ of a spacetime field $u(\ssp,\tau)$ is
obtained by
specifying its values $u_{mn}= u(\ssp_m, \zeit_n)$ on lattice points $z=(m,n)
\in \integers^2$. Examples are diffusive coupled map
lattices\rf{Kaneko83,Kaneko84} and Gutkin \etal\ \catlatt\rf{GutOsi15,GHJSC16,CL18}.
In this paper the first index will refer to configuration space, and the second to time.
In the Fourier representation $\Fu_{kj}$, the first index will refer to the spatial
Fourier mode, and the second to the frequency.

There are two lattices at play here: (i) the spacetime discretization
\refeq{spattempFT}, and (ii) the symbolic dynamics discretization. What follows refers
to as yet unattained latter.

{\bf Lattices.}
Consider a $2$\dmn\ square lattice infinite in extent, with each site
labeled by $2$ integers $z=(m,n)\in \integers^{2}$. Assign to each site $z$ a
letter \Ssym{z}\ from a finite alphabet $\A$. A particular fixed set
of letters  \Ssym{z}\ corresponds to a particular lattice state
\(
\Mm= \{\Ssym{z}\} % \in \A \,,\; z\in \integers^d \}
\,.
\)
%infinite in extent along all directions.
In other words, a $2$\dmn\ lattice requires a {$d$\dmn\ code}
\(
% \{\m_{z}\}
\Mm = \{\m_{n_1 n_2}\}
%\,,
\)
for a complete specification of the corresponding state $\Xx$.
The {\em full shift} is the set of all $2$\dmn\
symbol \brick s that can be formed from the letters of the alphabet $\A$
\beq
% \A^{\integers^d}   2017-08-05 Predrag dropped this notation
\hat{\AdmItnr} = \{ \{\Ssym{z}\} % (\Ssym{z}) %_{z\in\integers^d}
              : \Ssym{z} \in \A \quad \mbox{for all} \quad z \in  \integers^2\}
\,.
\ee{LatticeFullSh}

%\noindent
{\bf Multidimensional shifts.}
%%%%%%%%%%%%%%%%%%%%%%%%%%%%%%%%%%%%%%%%%%%%%%%%%%%%%%%%%%%%%%%%%%%%%%%%
%    \PC{2016-01-12} {
% in the spirit of \refRefs{PetCorBol07}:
For an autonomous dynamical system, the evolution law $f$ is of the same form for all times.
If $f$ is also of the same form at every lattice site, the
group of lattice translations, acting along the spatial lattice direction by
shift \shift{}, is a spatial symmetry that commutes with the temporal
evolution. A temporal mapping $f$ that satisfies
$f\circ\shift{j}=\shift{}\circ{f}$ along the spatial lattice
direction is said to be {\em shift invariant}, with the associated
symmetry of dynamics given by the $d$\dmn\ group of discrete
{\spt} translations.

%\noindent
{\bf {\Brick s}.} Let
$\R_{z}\subset\integers^2$  be a finite
$[\ell_1\!\times\!\ell_2]$ rectangular lattice region,
$\ell_k\geq1$,
whose lower left corner is the $z=(n_{1}n_{2})$ lattice site
\beq
  \R = \R_{n}^{[\ell_1\!\times\!\ell_2]}
  =\{(n_1+j_1,\cdots n_2+j_2) \mid 0\leq j_k\leq \ell_k-1\}
\,.
\ee{dDimRect}
The associated finite {\brick} of symbols $\Ssym{z}\in\A$ restricted to  \R,
\(
\Mm_{\R}=\{\Ssym{z}| z\in \R \} \subset \Mm
\)
is called the \brick\ $\Mm_{\R}$ of area
$\cl{\R} = \ell_1 \ell_2$. For example,
a
$\R = [3\times 2]$ \brick\ is of form
\beq
\Mm=\left[\begin{array}{c}
\Ssym{12} \Ssym{22} \Ssym{32}\\
\Ssym{11} \Ssym{21} \Ssym{31}
\end{array}\right]
\ee{3times2brick}
and volume (in this case, an area) equals $3\times 2 = 6$.
In our convention, the first index is `space', increasing from left to right,
and the second index is `time', increasing from bottom up.

%\noindent
{\bf Cylinder sets.}
While a particular admissible infinite symbol array
\(
\Mm= \{\Ssym{z}\} % \in \A \,,\; z\in \integers^d \}
\)
defines a point $\Xx$ (a unique lattice state) in the \statesp,
the \emph{cylinder set}
$\pS_{\R}$,
corresponding to the totality  of
\statesp\ points $\Xx$ that share the same given finite {\brick} $\Mm_{\R}$
symbolic representation over the region \R. For example, in $d=1$ case
\beq
\pS_{\R} =
    \{\cdots a_{-2} a_{-1}\,.\,
   \Ssym{1}\Ssym{2}\cdots \Ssym{\ell}
   a_{\ell+1}a_{\ell+2}\cdots\}
\,,
\ee{finiteBlock}
with the symbols outside of the block unspecified.

%\noindent
{\bf \twoTs.}
A {\statesp} point is {\em {\spt}ly periodic} if
it belongs to a \twot, \ie, its symbolic representation is a \brick\
over region $\R$ defined by \refeq{dDimRect},
\beq
  \Mm_{p} = \Mm_{\R}
  \,,\qquad
  \R = \R_{0}^{[\speriod{}\!\times\!\period{}]}
\,,
\ee{dTorus}
that
tiles the lattice state  $\Mm$ periodically, with period $\speriod{}$ in the
spatial lattice direction, and period $\period{}$ in the
time lattice direction.


%\noindent
{\bf Subshifts.}
%%%%%%%%%%%%%%%%%%%%%%%%%%%%%%%%%%%%%%%%%%%%%%%%%%%%%%%%%%%%%%%%%%%%%%%%
%    \PC{2016-10-11} { inspired by {Ban} \etal\rf{BHLL11}
Let $\hat{\AdmItnr}$ be the full lattice shift
\refeq{LatticeFullSh}, \ie, %{CMFullSh}
the set of all possible lattice state $\Mm$ labelings by the alphabet
$\A$, and $\hat{\AdmItnr}(\Mm_{\R})$ is
the set of such {\brick s} over a region {\R}. The principal task
in developing the symbolic dynamics of a dynamical system is to determine
$\AdmItnr$, the set of all \emph{admissible} itineraries/lattice states,
\ie, all states that can be realized by the given system.

%\noindent
{\bf Pruning, grammars, recoding.}
If certain states are inadmissible, the alphabet must be supplemented by a
{\em grammar},
a set of pruning rules.
Suppose that
the grammar can be stated as a finite number of pruning rules, each
forbidding a {\brick} of finite size,
\beq
 {\cal G} = \left\{
        b_1, b_2, \cdots b_k
        \right\}
\,,
\ee{grammar}
where a {\em pruned {\brick}} $b$ is an array of symbols defined over a
finite $\R$ lattice region of size $[ \speriod{1}\!\times\!\speriod{2}]$. In
this case we can construct a finite Markov partition by replacing finite
size \brick s of the original partition by letters of a new alphabet. In
the case of a 1\dmn, the temporal lattice, if the longest forbidden {\brick}
is of length $\speriod{}+1$, we say that the symbolic dynamics is Markov, a shift
of finite type with {$\speriod{}$-step memory}.

Let
\(
\Xx=\{u_{z} \in  \mathbb{T}^{1},\, z\in \integers^2\}
\)
be a {\spt}ly infinite  solution of \KSe\ \refeq{spattempFT},
and let
\(
\Mm= \{\Ssym{z} \in \A \,,\; z\in \integers^2 \}
\)
be its symbolic representation. By the presumed connection between \Xx\ and
\Mm, the corresponding symbolic dynamics {\brick} \Mm\ is unique and
admissible, \ie, \Mm\ defines the unique {\spt} state \Xx\ and
vice-versa.

Assume now that only partial information is available, and we know
only a finite \brick\ of symbols $\Mm_{\R}\subset\Mm$,
over a finite lattice region $\R\subset \integers^2$. What information
about the local {\spt} pattern
%$\ssp_{z},z\in\R$
\(
\Xx_{\R}=\{\ssp_{z} \in  \mathbb{T}^{1},\, z\in\R\}
\)
does this give us?
To be specific, let \R\  be a   rectangular $[\ell_1\!\times\!\ell_2]$
region (see \refeq{dDimRect} for the definition),
and let
\(
\Mm_{\R} % =\{\Ssym{z}\in \Mm | z\in \R \}
\)
be the  $[\ell_1\!\times\!\ell_2]$
\brick\ of \Mm\ symbols.

%%%%%%%%%%%%%%%%%%%%%%%%%%%%%%%%%%%%%%%%%%%%%%%%%%%%%%%%%%
%This next part assumes that the concepts of symbolic dynamics
%have been adequately explained

This formalism is necessary for us to interpret results but
it cannot inform us of the grammar (rules which dictate admissible combinations)
of the symbolic dynamics. To uncover the grammar of the
symbolic dynamics we can combine tile solutions numerically
and find which combinations are inadmissible. The manner
with which we combine such tiles is described in \refsect{sect:tiles}.
This can proceed in an automated manner, taking all possible
\spt\ combinations of tiles. This portion has not been
completed just yet and there are some key details
that must be worked out before we can proceed.
The main problem is in regards to false negatives. Because
the gluing procedure of \refsect{sect:glue} is a numerical
effort, utmost care must be taken in combining solutions.
If the tiles are not combined properly, the combination
of tiles may be falsely deemed an inadmissible combination
which would give us the incorrect grammar or set
of pruning rules. We will provide a number
of examples of what can go wrong but the list
is far from exhaustive. The first example of how a false negative may
occur is that if the wrong members of the continuous families
are combined then the combination may not converge even
though the corresponding symbolic block is admissible.
Another example is that when the tiles are being combined
there must be some numerical procedure for how to smooth
discontinuities where the boundaries of the tiles are conjoined.
This typically isn't a huge issue but it definitely has a larger effect
on smaller combinations. One must also take into consideration
the initial conditions for the extent of the \spt\ domain $[T\!\times\!L]$.
For this we approximate the \spt\ domain as simply being the
arithmetic average of the dimensions but this may be too crude
as it does not take into account how the coupling between tiles
affects the domain size. Our last example
is a debate between whether to
glue progressively larger symbolic blocks together (\ie, the ``tiles''
are replaced by larger and larger numerically converged \twots\ ).
Alternatively, we can start each attempt by gluing the
indivisible elements, the tiles. These two methods both
take combinations of numerically converged \twots. The
difference lies in the size of the continuous families of solutions.
Anecdotal evidence seems to suggest that
the tiles exist on a smaller interval of spatial
domain sizes $L \pm \epsilon$ when compared to
larger \twots.

The goal of the previous discussion is to enumerate
and describe the ``best'' method to produce the grammar
of the \spt\ symbolic dynamics. Once the details are
in place, the combinations would be tested in an automated
fashion which would hopefully produce a consistent
set of pruning rules which could be used to define
the inadmissible symbolic blocks of the \spt\
symbolic dynamics.

\renewcommand{\shift}{\ensuremath{d}}
\renewcommand{\Xx}{\ensuremath{\mathsf{X}}}


% siminos/gudorf/thesis/chapter/symbolic.tex
% $Author: predrag $ $Date: 2020-10-24 01:45:26 -0400 (Sat, 24 Oct 2020) $

\section{Symbolic dynamics: a glossary}
\label{s-SymbDynGloss}

%%%%%% ChaosBook convention start %%%%%%%%%%%%%%%%%%%%%
\renewcommand{\statesp}{state space}
\renewcommand{\Statesp}{State space}
\renewcommand{\stateDsp}{state-space}
\renewcommand{\StateDsp}{State-space}

Analysis of a low\dmn\ chaotic dynamical system typically
starts\rf{DasBuch} with establishing that a flow is locally stretching, globally
folding. The flow is then reduced to a discrete time return map by appropriate
Poincar\'e sections. Its state space is partitioned, the partitions labeled by an
alphabet, and the qualitatively distinct solutions classified by their temporal
symbol sequences. Thus our analysis of the cat map and the {\catlatt} requires
recalling and generalising a few standard symbolic dynamics notions.

%\noindent
{\bf Partitions, alphabets.}
A division of {\statesp} $\pS$ into a disjoint union of distinct regions
$\pS_A,\pS_B,\ldots,\pS_Z$ constitutes a {\em
partition}. Label each region by a symbol $\Ssym{}$ from an
$N$-letter  {\em alphabet}
$\A=\{A,B,C,\cdots,Z\}$, where $N=\cl{\A}$ is
the number of such regions. Alternatively, one can distinguish different
regions by coloring them, with colors serving as the ``letters'' of the
alphabet.
% missing in kittens:  , as in \reffigs{fig:SingleCatPartit}{fig:AKScloseActSp}.
For notational convenience, in alphabets we sometimes denote negative integer
$\Ssym{}$ by underlining it, as in
\(
\A = \{ -{2}, -{1}, 0, 1\}
   = \{ \underline{2}, \underline{1}, 0, 1\}
\,.
\)


%\noindent
{\bf Itineraries.}
For a dynamical system evolving in time,
every {\statesp} point $\xInit \in \pS$ has the {\em future itinerary},
an infinite sequence of symbols
$\Sfuture(\xInit)=\Ssym{1}\Ssym{2}\Ssym{3}\cdots$ which indicates the
temporal order in which the regions shall be visited. Given a trajectory
$\ssp_1,\ssp_2,\ssp_3,\cdots$ of the initial point $\xInit$ generated
by a time-evolution law
\( %beq
   \ssp_{n+1}=f(\ssp_n)
    % \,, \quad \ssp_0=\xInit
\,,
\) %ee{CMx-iterated}
the itinerary is given by the symbol sequence
\beq
   \Ssym{n} = \Ssym{} \qquad \mbox{if\ } \qquad  \ssp_n \in \pS_{\Ssym{}}
 \,.
\ee{CMsymbol-def}
The {\em past itinerary} $\Spast(\xInit)=\cdots\Ssym{-2}
\Ssym{-1}\Ssym{0} $ describes the order in which the regions were visited
up to arriving to the point $\xInit$. Each point $\xInit$ thus has
associated with it the bi-infinite itinerary
\beq
\Sbiinf(\xInit) % = (\Ssym{k})_{k\in \integers}
        = \Spast.\Sfuture  =
 \biinf{\Ssym{-2}\Ssym{-1}\Ssym{0}}{\Ssym{1}\Ssym{2} \Ssym{3}}
\,,
\ee{CMbiifs}
or simply `itinerary', if we chose not to use the decimal point
to indicate the present,
\beq
   \{\Ssym{\zeit}\} = \cdots\Ssym{-2}\Ssym{-1}\Ssym{0}\Ssym{1}\Ssym{2} \Ssym{3}\cdots
\ee{Itinerary}


%\noindent
{\bf Shifts.}
A forward iteration of temporal dynamics $x\rightarrow x' = f(x)$ shifts
the entire itinerary to the left through the `decimal point'. This
operation, denoted by the shift operator \shift{},
\beq
   \shift{}(\biinf{\Ssym{-2}\Ssym{-1}\Ssym{0}}{\Ssym{1}\Ssym{2} \Ssym{3}})
     =  \biinf{ \Ssym{-2}\Ssym{-1}\Ssym{0}\Ssym{1}}{ \Ssym{2} \Ssym{3}}
\,,
\ee{CMshift-s}
demotes the current partition label $\Ssym{1}$ from the future $\Sfuture$
to the past $\Spast$.
The inverse shift $\shift{}^{-1}$ shifts the entire itinerary one step
to the right.

The set of all itineraries that can be formed from the letters of the
alphabet $\A$ is called the {\em full shift}
\beq
% \A^\integers 2017-08-05 Predrag dropped this notation
\hat{\AdmItnr} = \{ (\Ssym{k})
              : \Ssym{k} \in \A \quad \mbox{for all} \quad k \in  \integers \}
\,.
\ee{CMFullSh}

The itinerary is infinite for any trapped (non-escaping or \nws\ orbit) orbit
(such as an orbit that stays on a chaotic
repeller), and infinitely repeating for a periodic orbit $p$ of period \cl{p}.
A map $f$ is said to be a \emph{horseshoe} if its restriction to the \nws\ is
hyperbolic and topologically conjugate to the full $\A$-shift.

%\noindent
{\bf Lattices.}
Consider a $d$\dmn\ hypercubic lattice infinite in extent, with each site
labeled by $d$ integers $z\in \integers^{d}$. Assign to each site $z$ a
letter \Ssym{z}\ from a finite alphabet $\A$. A particular fixed set
of letters  \Ssym{z}\ corresponds to a particular lattice state
\(
\Mm= \{\Ssym{z}\} % \in \A \,,\; z\in \integers^d \}
\,.
\)
%infinite in extent along all directions.
In other words, a $d$\dmn\ lattice requires a {$d$\dmn\ code}
\(
% \{\m_{z}\}
\Mm = \{\m_{n_1 n_2 \cdots n_d}\}
%\,,
\)
for a complete specification of the corresponding state $\Xx$.
In the lattice case, the {\em full shift} is the set of all $d$\dmn\
symbol \brick s that can be formed from the letters of the alphabet $\A$
\beq
% \A^{\integers^d}   2017-08-05 Predrag dropped this notation
\hat{\AdmItnr} = \{ \{\Ssym{z}\} % (\Ssym{z}) %_{z\in\integers^d}
              : \Ssym{z} \in \A \quad \mbox{for all} \quad z \in  \integers^d\}
\,.
\ee{LatticeFullSh}

%\noindent
{\bf Commuting discrete translations.}
%{\bf .}
%%%%%%%%%%%%%%%%%%%%%%%%%%%%%%%%%%%%%%%%%%%%%%%%%%%%%%%%%%%%%%%%%%%%%%%%
%    \PC{2016-01-12} {
% in the spirit of \refRefs{PetCorBol07}:
For an autonomous dynamical system, the evolution law $f$ is of the same form for
all times. If $f$ is also of the same form at every lattice site, the group of
lattice translations (sometimes called multidimensional shifts), acting along
$j$th lattice direction by shift $\shift{j}$, is a spatial symmetry that commutes
with the temporal evolution. A temporal mapping $f$ that satisfies
$f\circ\shift{j}=\shift{j}\circ{f}$ along the $d\!-\!1$ spatial lattice directions
is said to be {\em shift invariant}, with the associated symmetry of dynamics
given by the $d$\dmn\ group of discrete {\spt} translations.

\bigskip

Assign to each site $z$ a
letter \Ssym{z}\ from the alphabet $\A$. A particular fixed set
of letters  \Ssym{z}\ corresponds to a particular lattice symbol array
\(
\Mm= \{\Ssym{z}\} % \in \A \,,\; z\in \integers^d \}
 = \{\Ssym{n_1 n_2 \cdots n_d}\}
\,,
\)
which yields a complete specification of the corresponding state $\Xx$.
In the lattice case, the {\em full shift} is the set of all $d$\dmn\
symbol arrays that can be formed from the letters of the alphabet $\A$

as in \refeq{LatticeFullSh}

A $d$\dmn\ {\spt} field
\(
\Xx=\{\ssp_{z}\}
\)
is determined by the corresponding {\em $d$\dmn} {\spt}
symbol array
\(
\Mm=\{\Ssym{z}\}
\,.
\)
Consider next a finite \brick\ of symbols $\Mm_{\R}\subset\Mm$,
over a finite rectangular $[\speriod{1}\!\times\!\speriod{2}\!\times\cdots\times\!\speriod{d}]$
lattice region $\R\subset \integers^d$.
In particular, let $\Mm_{p}$ over a finite rectangular
$[\speriod{1}\!\times\!\speriod{2}\!\times\cdots\times\!\speriod{d}]$ lattice region be the
$[\speriod{1}\!\times\!\speriod{2}\!\times\cdots\times\!\speriod{d}]$ $d$-periodic \brick\ of
\Mm\ whose repeats tile $\integers^d$.

%\noindent
{\bf {\Brick s}.} In the case of temporal dynamics, a finite itinerary
\\
$\Mm_{\R}={\Ssym{k+1}\Ssym{k+2}\cdots\Ssym{k+\speriod{}}}$ of symbols from
$\A$ is called a {\em \brick} of length $\speriod{}=\cl{\R}$. More generally, let
$\R\subset\integers^d$  be a
$[\speriod{1}\!\times\!\speriod{2}\!\times\!\cdots\speriod{d}]$ rectangular lattice region,
$\speriod{k}\geq1$,
whose lower left corner is the $n=(n_{1}n_{2}\cdots{n_{d}})$ lattice site
\beq
  \R = \R_{n}^{[\speriod{1}\!\times\!\speriod{2}\!\times\!\cdots\speriod{d}]}
  =\{(n_1+j_1,\cdots n_d+j_d) \mid 0\leq j_k\leq \speriod{k}-1\}
\,.
\ee{dDimRect}
The associated finite {\brick} of symbols $\Ssym{z}\in\A$ restricted to  \R,
\(
\Mm_{\R}=\{\Ssym{z}| z\in \R \} \subset \Mm
\)
is called the \brick\ $\Mm_{\R}$ of volume
$\cl{\R} = \speriod{1}\speriod{2}\cdots\speriod{d}$. For example, for a 2\dmn\ lattice
a
$\R = [3\!\times\!2]$ \brick\ is of form
\beq
\Mm_{\R}=\left[\begin{array}{c}
\Ssym{12}\ \Ssym{22}\ \Ssym{32}\\
\Ssym{11}\ \Ssym{21}\ \Ssym{31}
\end{array}\right]
\ee{3times2brick}
and volume (in this case, an area) equals $3\times 2 = 6$.
In our convention, the first index is `space', increasing from left to right,
and the second index is `time', increasing from bottom up.

%\noindent
{\bf Cylinder sets.}
While a particular {\admissible} infinite symbol array
\(
\Mm= \{\Ssym{z}\} % \in \A \,,\; z\in \integers^d \}
\)
defines a point $\Xx$ (a unique lattice state) in the \statesp,
the \emph{cylinder set}
$\pS_{\Mm_{\R}}$,
% $ \pS_{\R}$,
corresponds to the totality  of
\statesp\ points $\Xx$ that share the same given finite {\brick} $\Mm_{\R}$
symbolic representation over the region \R. For example, in $d=1$ case
\beq
\pS_{\Mm_{\R}} =
    \{\cdots a_{-2} a_{-1}\,.\,
   \Ssym{1}\Ssym{2}\cdots \Ssym{\speriod{}}
   a_{\speriod{}+1}a_{\speriod{}+2}\cdots\}
\,,
\ee{finiteBlock}
with the symbols  $a_{j}$ outside of the {\brick}
$\Mm_{\R}=[\Ssym{1}\Ssym{2}\cdots \Ssym{\speriod{}}]$
unspecified.
\index{block!finite sequence}
\index{cylinder!set}

%\noindent
{\bf \Po s, \dtors.}
A {\statesp} point $\ssp_z\in\Xx$ is {\spt}ly
{\em periodic}, $\ssp_z=\ssp_{z+\speriod{}}$, if its spacetime orbit returns to it
after a finite lattice shift
\(
\speriod{}= (\speriod{1},\speriod{2},\cdots,\speriod{d})
\)
over region $\R$ defined in \refeq{dDimRect}.
The infinity of repeats of the corresponding {\brick} $\Mm_{\R}$ then tiles the lattice.
For a {\spt}ly {periodic} state $\Xx$, a {\em prime} {\brick}
$\Mm_{p}$ (or $p$) is a smallest such \brick\
\(
\speriod{p}= (\speriod{1},\speriod{2},\cdots,\speriod{d})
\)
that cannot itself be tiled by repeats of a shorter {\brick}.

The periodic tiling of the lattice by the infinitely many repeats of a prime
{\brick} is denoted by a bar: $\cycle{\Mm}_{p}$. We shall omit the bar whenever
it is clear from the context that the state is periodic.
    \PC{2019-01-19}{eliminate \prune{ \Ssym{-m+1}\cdots \Ssym{0}} and
    \rctngl{ \Ssym{-m+1}\cdots \Ssym{0}},
    notation in favor a single convention}
    \PC{2018-11-07}{
    Generalize to \dtors.
    }


In $d=1$ dimensions, prime {\brick} is called a {\em prime} cycle $p$, or a
single traversal of the orbit; its label is a {\brick} of $\cl{p}$ symbols that
cannot be written as a repeat of a shorter {\brick}.
Each {\em periodic point}
\(
      \ssp_{ \Ssym{1} \Ssym{2} \cdots \Ssym{\cl{p}}}
\)
is then labeled by the starting symbol $\Ssym{1}$, followed by
the next $(\cl{p}-1)$ steps of its future itinerary.
The set of periodic points $\pS_p$ that belong to a given periodic orbit
form a {\em cycle}
\beq
p =  \cycle{ \Ssym{1} \Ssym{2} \cdots \Ssym{\cl{p}}}
  = \{
      \ssp_{ \Ssym{1} \Ssym{2}\cdots \Ssym{\cl{p}}},
      \ssp_{ \Ssym{2} \cdots \Ssym{\cl{p}} \Ssym{1}},
    \cdots,
      \ssp_{ \Ssym{\cl{p}} \Ssym{1}\cdots \Ssym{\cl{p}-1}}
     \}
\,.
\ee{PeriodCyc}

More generally, a {\statesp} point is {\em {\spt}ly periodic} if
it belongs to an \dtor, \ie, its symbolic representation is a \brick\
over region $\R$ defined by \refeq{dDimRect},
\beq
  \Mm_{p} = \Mm_{\R}
  \,,\qquad
  \R = \R_{0}^{[\speriod{1}\!\times\!\speriod{2}\!\times\cdots\times\!\speriod{d}]}
\,,
\ee{dTorus}
that
tiles the lattice state  $\Mm$ periodically, with period $\speriod{j}$ in the
$j$th lattice direction.


%\noindent
{\bf Generating partitions.}
A temporal partition is called {\em generating} if every bi-infinite itinerary
corresponds to a distinct point in {\statesp}.
In practice almost any
generating partition of interest is infinite.
Even when the dynamics assigns a unique infinite itinerary
$\biinf{\Ssym{-2}\Ssym{-1}\Ssym{0}}{\Ssym{1}\Ssym{2} \Ssym{3}}$ to each
distinct orbit, there generically exist full shift itineraries
\refeq{CMFullSh} which are not realized as orbits; such sequences are
called {\em \inadmissible}, and we say that the symbolic dynamics is {\em
pruned}.

%\noindent
{\bf Dynamical partitions.}
If the symbols outside of given temporal {\brick} $b$ remain unspecified, the
set of all {\admissible} {\brick s} of length $\cl{b}$ yield a dynamically
generated partition of the \statesp, $\pS = \cup_b \pS_b$.

%\noindent
{\bf Subshifts.}
A dynamical system $(\pS,f)$ given by a mapping $f : \pS \to \pS$
together with a {partition} $\A$ induces {\em topological dynamics}
$(\AdmItnr,\shift{})$, where the {\em subshift}
\beq
\AdmItnr = \{  (\Ssym{k})_{k\in \integers} \}
\,,
\ee{subshift}
is the set of all {\em \admissible} itineraries, and
$ \shift{} \,:\, \AdmItnr \to \AdmItnr $ is the shift
operator \refeq{CMshift-s}.
The designation `subshift' comes from the fact that
$\AdmItnr$  % \subset \hat{\AdmItnr}$
% \A^\integers 2017-08-05 Predrag dropped this notation
 is the subset of the full shift.

%%%%%%%%%%%%%%%%%%%%%%%%%%%%%%%%%%%%%%%%%%%%%%%%%%%%%%%%%%%%%%%%%%%%%%%%
%    \PC{2016-10-11} { inspired by {Ban} \etal\rf{BHLL11}
Let $\hat{\AdmItnr}$ be the full lattice shift  \refeq{CMFullSh}, \ie,
the set of all possible lattice state $\Mm$ labelings by the alphabet
$\A$, and $\hat{\AdmItnr}(\Mm_{\R})$ is
the set of such {\brick s} over a region {\R}. The principal task
in developing the symbolic dynamics of a dynamical system is to determine
$\AdmItnr$, the set of all \emph{{\admissible}} itineraries/lattice states,
\ie, all states that can be realized by the given system.

%\noindent
{\bf Pruning, grammars, recoding.}
If certain states are {\inadmissible}, the alphabet must be supplemented by a
{\em grammar},
a set of pruning rules.
Suppose that
the grammar can be stated as a finite number of pruning rules, each
forbidding a {\brick} of finite size,
\beq
 {\cal G} = \left\{
        b_1, b_2, \cdots b_k
        \right\}
\,,
\ee{grammar}
where a {\em pruned {\brick}} $b$ is an array of symbols defined over a
finite $\R$ lattice region of size
$[\speriod{1}\!\times\!\speriod{2}\!\times\cdots\times\speriod{d}]$. In
this case we can construct a finite Markov partition by replacing finite
size \brick s of the original partition by letters of a new alphabet. In
the case of a 1\dmn, the temporal lattice, if the longest forbidden {\brick}
is of length $L+1$, we say that the symbolic dynamics is Markov, a shift
of finite type with {$L$-step memory}.

%\noindent
{\bf Subshifts of finite type.}
A {topological dynamical system} $(\AdmItnr,\shift{})$ for which all
{\admissible} states $\Mm$ are generated by recursive application
of the finite set of pruning rules \refeq{grammar}
%of the  finite transition matrix
%\beq
%\AdmItnr = \left\{ (\Ssym{k})_{k\in \integers}
%           \,:\,
%                 T_{\Ssym{k} \Ssym{k+1}} = 1
%        \quad \mbox{for all $k$} \right\}
%\ee{AdmItnr}
is called a subshift of {\em finite type}.

                                            \toCB % to kneading.tex
If a map can be topologically conjugated to a linear map, the symbolic
dynamics of the linear map offers a dramatically simplified description
of all {\admissible} solutions of the original flow, with the temporal
symbolic dynamics and the state space dynamics related by linear recoding
formulas. For example, if a map of an interval, such as a parabola, can
be conjugated to a piecewise linear map, the kneading theory\rf{MilThu88}
classifies \emph{all} of its {\admissible} orbits.

%%%%%% ChaosBook convention,  BORIS conventions start %%%%%%%%%%%%%%%%%%%%%
\renewcommand{\statesp}{phase space}
\renewcommand{\Statesp}{Phase space}
\renewcommand{\stateDsp}{phase-space}
\renewcommand{\StateDsp}{Phase-space}
%%%%%% BORIS convention end   %%%%%%%%%%%%%%%


