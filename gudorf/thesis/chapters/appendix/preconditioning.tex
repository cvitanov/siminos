
One aspect that must be taken into consideration whenever
a linear system is being solved numerically is whether or
not the results are accurate. One quantitative measure
of this is the condition number of the corresponding matrix.
The condition number is calculated by taking the ratio
of the largest and smallest singular values computed
by singular value decomposition. Broadly speaking,
the condition number indicates how sensitive
the system is to error or perturbation. If the condition number
large the system is said to be ``ill-conditioned'' which in
turn greatly affects the accuracy of solutions. For chaotic
initial value problems the magnitude of the condition
number depends on the maximal Lyapunov exponent\rf{WaHuBl14}.
For our \spt\ problem, we do not have a quantitative measure
of this but the numerical problem is almost always ill-conditioned.
Therefore, accounting for this is very important if we want
to trust our computations. This can be accomplished by applying
what is known as \emph{preconditioning}. Deciding on a specific
\emph{preconditioner} to use is a dark art, but some general
guidelines are that the preconditioner should not be expensive
to compute and it should approximately equal the inverse of the
ill-conditioned linear operator.

This motivates preconditioning in the context of iterative methods
and solving linear systems, but how does it apply to descent methods?
The dimensionality of our problem is large, but that does not imply
that all dimensions are equally important. To gain insight on this
we will apply what we known about the \KSe\ as a dynamical system
and then extend these ideas to our \spt\ formulation. The linear
spatial derivative terms in the \KSe\ have direct effect on the spectrum
of Fourier coefficients. Namely, the dissipation due to the fourth
order term prevents shocks (discontinuities)
and small wavelength oscillations. This ensures that the solutions
are relatively smooth, the implication being that the spectra
of \Fcs\ converges as the wavenumbers increase. In other words,
the magnitudes of the \spt\ Fourier modes vary greatly.

We demonstrate this with the following comparison. Let us consider
two \spt\ Fourier modes, whose {\spt} wavenumbers $k_1,k_2$ and $j_1,j_2$
differ greatly, $k_1 << k_2 $ and $j_1 << j_2$.
Due to the higher order spatial derivatives the
linear term of \refeq{e-kssFb} is dominant when the wavenumbers become large.
As a result, the \spt\ \Fcs\ with large wavenumbers are approximately
magnified by a factor of $|q_m^2-q_m^4| ~ m^4$.
This magnification is large enough (technically its even larger due to
the matrix-vector product) such that the components
of the descent direction \refeq{e-descentdirection} are all of the same
order of magnitude. This completely betrays the convergent behavior
of the spectrum. Therefore in order to match the adjoint descent direction
to our expectations we apply preconditioning. In this context, the
preconditioning can be viewed geometrically; it rescales the
components of the adjoint descent direction to better reflect
the spectra of \Fcs.
As mentioned before, preconditioners should be cheap and approximate the
inverse of whatever transformation is being applied.
As previously argued the ill-conditioned nature of the \spt\ \KSe\
is the linear term. By applying a transformation which approximately
equals the inverse of the linear term we can correctly scale the
related components of both the adjoint descent direction and the
solutions that result by iterative methods. Specifically let
$P$ denote the linear transformation defined by
\beq \label{e-preconditioner}
P = \text{diag}\Big( \frac{1}{|\omegaj|+\wavek^2+\wavek^4}\Big)\,.
\eeq
In addition we have found from experience that rescaling the magnitude of
the changes in torus parameters $(\speriod{},\period{})$ is helpful.
Depending on the circumstances the following preconditioning may or may
not be applied
\bea \label{e-preconditionerTL}
P\cdot \delta \period{} &=& \frac{\delta \period{}}{\period{}} \continue
P\cdot \delta \speriod{} &=& \frac{\delta \speriod{}}{\speriod{}^4}\,.
\eea
The preconditioning given by \refeq{e-preconditionerTL} is mainly an
attempt to imitate the effect of \refeq{e-preconditioner} for the
parameters $(\speriod{},\period{})$. The main motivation behind this
rescaling is experiential in nature. A common behavior for initial
conditions which are very much non-solutions (cost functional has very
large magnitude) and initial conditions on small \spt\ domains the
partial derivatives with respect to $(\speriod{},\period{})$ tend to be
large in magnitude. Without rescaling this often leads to dramatic
changes in spatial domain size or convergence to equilibrium solutions.
One way of interpreting this is that the trivial manner that can reduce
the magnitude of the cost functional is to increase the values of
$(\speriod{},\period{})$ when the linear term and nonlinear term are out
of proportion.


By disabling changes to the period and length of the system this is lowered by an
order of magnitude, however, I believe this still implies that the system is highly ill-conditioned
even when trying to find fixed points of the spatiotemporal mapping on a fixed spatiotemporal domain.
Xiong suggested that perhaps I can use the singular values to produce a preconditioner that might help,
I'm looking into this as well as some other papers\rf{Meza95}