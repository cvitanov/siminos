
Firstly, we must generate symmetry reduced spatiotemporal initial conditions, this, plainly, is done by
using a symmetry reduced time integrator such as Burak's \texttt{ksETDRK4red.m} to generate a spatiotemporal
discretization that will be used for the hunt.

Next, I rewrite the direct-matrix equations in a symmetry reduced form. I find it easiest to first
look at the symmetry reduced velocity equations for \KSe\ and then figure out what is necessary to take
the leap into turning them into a spatiotemporal mapping such as \refeq{e-FksSpattemp}.

Taking equations (8),(9) from \refref{BudCvi14}, the symmetry reduced flow takes the form
\bea
    \hat{v}(\hat{a}) &=&
                v(\hat{a}) - \dot{\theta}(\hat{a})t(\hat{a}) \,, \quad
    \dot{\theta}(\hat{a}) =
                \frac{\langle v(\hat{a}) | t'(\hat{a}) \rangle}{\langle t(\hat{a}) | t'(\hat{a}) \rangle} t(\hat{a})    \,, \quad
    \label{e-symred}
\eea

Now, in order to put this symmetry reduced flow
into a spatiotemporal setting, where the entire orbit is then treated as a vector in the domain the spatiotemporal mapping,
we must make note that for the $N-by-M$ discretizations that we are going to use in the fixed point finder
we need to expand these formulae to account for the fact that we need $M$ copies of the template vector $t(\hat{a})$ for every
point in time; likewise in order to evaluate the group tangent of at for the $2NM$
dimensional vector (2 because of Fourier)
we need to take a Kronecker product between the identity and the \SOn{2} generator in order to have a
$[2NM\!\times\!by-2NM]$ matrix that
will produce the correct group tangent after multiplication with the spatiotemporal vector.

Now once we have all of these things in place we can produce an equation similar to \refeq{e-FksSpattempMat} for the
symmetry reduced flow. Substituting $v$ from \refeq{e-MNGkstimeDM},
(the transparency) the symmetry reduced flow can be written as
\beq
\hat{v}(\hat{a})= \frac{\partial a}{\partial \tau} = f(a),
\eeq
where $\tau$ is the in-slice time.

Then applying an operator $F_t$ that takes the temporal Fourier transform around the orbit, and rewriting the time-derivative
by exploiting the Fourier transform we are left with
\beq
W_{\tau}*\hat{u} - F_{\tau} \cdot f(\hat{a}) = 0
\eeq
Now, for the record, I write the equation in its complete
direct-matrix, symmetry reduced form.
\bea
0 &=&
W_{\tau}*\hat{u}
- F_t ( Q_1 \cdot \Fu - Q_2 \cdot F \cdot ((F^{-1}\Fu) \star (F^{-1}\Fu))
    \ceq
- \frac{\langle Q_1 \cdot \Fu - Q_2 \cdot F \cdot ((F^{-1}\Fu) \star (F^{-1}\Fu)) | t'(\hat{a}) \rangle}
       {\langle t(\hat{a}) | t'(\hat{a}) \rangle} t(\hat{a}))
\eea
where $\Fu$ is now the vector quantity that completely describes the spatiotemporal symmetry reduced orbit in question.

Note: the definitions of the operators before taking Kronecker products are from \refeq{e-MNGwoperator}, \refeq{e-MNGq1operator}, \refeq{e-MNGq2operator}.

Finding some better numerical results with the alternate definitions for
\refeq{e-symred} which are, ($x_i$ implying real part of $i$th Fourier mode,
$y_i$ is the imaginary part, where $i = 1, ..., N/2-1$

\bea
    \frac{\partial \hat{a}}{\partial \tau} &=&
                \hat{x}_1 v(\hat{a}) - \dot{y}_1(\hat{a}) t(\hat{a}) \,, \quad
    \frac{\partial \theta(\hat{a})}{\partial \tau} = \dot{y}_1(\hat{a})
                   \,, \quad
    \label{e-symred2}
\eea

I believe that the stability matrix with take the following form,
\beq
\hat{A} = diag(x_1)\cdot A - diag(\dot{y}_1(\hat{a}))\cdot T
\eeq


Finished implementing the newer version of symmetry reduction variational method
\refeq{e-symred2}.
Still having issues as initial condition for the value of the cost-functional, which
is the best indication I have that I am still somehow messing up the symmetry reduced equations;
the only other contributor could be the approximate loop tangent, which I realized today
should also be symmetry reduced, i.e. it doesn't make sense to compare the in-slice velocity
to an \MNGedit{approximations of element in the full {\statesp} tangent bundle}.

After being frustrated for a while I finally realized the dumb mistake I have been making
when it comes to symmetry reduction. I have been trying to compute the symmetry reduced
velocity at every point of the discretized symmetry reduced {\rpo}s all at once. This was
the wrong choice because the equations rely on inner products of vectors specific to each point
along the discretization (namely the full \statesp\ velocity and the group tangents at
the slice template point and each time discretization point).

I don't know why I didn't realize this before (I guess I'm learning?) but something clicked
when I noticed that the wrong elements were being set to zero in my previous formulation, in other
words the in-slice symmetry reduced velocity wasn't tangent to the slice.
I tested out the formulae I have been using for a single point along the discretization in time and everything resolved.

In more detail, previously I was using a template point that was the same dimension as the entire discretized
orbit, i.e. for an $N-by-M$ space-by-time discretization of an \rpo\ I was using $M$ copies
of the same template vector expecting it to work but this is fundamentally flawed because
when I use the equations \refeq{e-symred} all this does is mix all of the information of
the orbit together in a jumbled fashion that makes absolutely no sense.

If I can't figure out a way then there is the back-up
plan of performing operations one at a time (as opposed to using
linear operators that act on the entire discretized orbit at once)
by computing the symmetry reduced velocity of each \statesp\ point and each symmetry reduced
stability matrix one by one and then compiling them respectively into the vector, matrix that I need:
A vector containing the in-slice velocities at every point along the orbit and a block
diagonal matrix whose blocks are comprised of in-slice stability matrices evaluated as the
respective point along the orbit.

After I posted yesterday I almost immediately figured out the issue with the current formulation and why it won't work. It's a matter
of the ordering of the variables; the way I have the code currently written the variables are ordered in a spatiotemporal vector
that cycles through the time first and then space. I.e. if $k$ and $\ell$ were indices that take values from $0, \cdots, N-1$
and $0, \cdots, M-1$ respectively representing the the space and time discretization, for every unit change to $k$, the index
$\ell$ would have cycled through $M$ values.

Now, this isn't important it's just a clerical part of coding but the problem I am now facing is that this specific ordering
is going to make it much harder to finish the symmetry reduction variational code. I've tried to work things out
in a number of ways but they are require some very unintuitive measures that I feel would be best to avoid. So, for now,
I am going to rewrite everything such that the ordering of the \statesp\ vector cycles through spatial indices first, in other
words switching to a different standard.

With this in mind, I believe I have a way to rewrite \refeq{e-symred} such that it will serve my purposes of symmetry reduction.
Like I've said before the problem I had from a lack of practice was that I was trying to symmetry reduce the entire orbit at
once not taking into consideration that the group tangent direction (amongst other things) is a time-dependent quantity.

Here is how I believe I can fix things once I change the ordering of the spatiotemporal vector i.e. the orbit in vector form.
It basically relies on computing $M$ copies of \refeq{e-symred} while keeping all of the information at different times separate.
Starting with,
\bea \nonumber
    \hat{v}(\hat{a}) &=&
                v(\hat{a}) - \dot{\theta}(\hat{a})t(\hat{a}) \,, \quad
    \dot{\theta}(\hat{a}) =
                \frac{\langle v(\hat{a}) | t'(\hat{a}) \rangle}{\langle t(\hat{a}) | t'(\hat{a}) \rangle} t(\hat{a})    \,, \quad
\eea

Let $v$ now be matrix whose rows (instead of columns as to reshape the matrix into a vector properly)
are the full \statesp\ velocity evaluated along the time discretization of a symmetry reduced
\rpo. Likewise, define $\hat{a}$ as a matrix whose rows are the current (in fictitious time) representation of the symmetry
reduced \statesp\ points that comprise the \rpo. Then, we can change the inner product notation to a matrix-vector product notation
by virtue of the group tangent template vector being constant. For example, the velocity will now be concatenated in a matrix
in the following manner.

\beq
v(\hat{a}) \rightarrow \mathbf{V} \equiv [v(\hat{a}(\tau_0)) | v(\hat{a}(\tau_1)) | \cdots | v(\hat{a}(\tau_{M-1})) ]
\eeq

following this notation the equation for evaluating the symmetry reduced velocity at every point along our discretization
will be

\bea \nonumber
    \hat{\mathbf{V}}^T(\hat{a}) &=&
                \mathbf{V}^T(\hat{a}) - \dot{\theta}(\hat{a}) \mathbf{t}^T(\hat{a}) \,, \quad
    \dot{\mathbf{\theta}}(\hat{a}) =
                (\mathbf{V}^T(\hat{a}) \cdot t'(\hat{a})*/ \mathbf{t}^T(\hat{a}) \cdot t'(\hat{a})) \mathbf{t}^T(\hat{a})  \,, \quad
\eea

Now the onto the stability matrix; as opposed to the symmetry reduced velocity where
I can just concatenate vectors into matrix and perform element-wise operations and matrix-vector products, with the stability
matrices I needed to rewrite definitions of all of the matrices involved in the direct-matrix derivation which I am
still in the process of. Some of them are easy, where the effect of the reordering merely results in the opposite order
kronecker product.

Wrote code to produce more general initial conditions for symmetry reduced variational methods.
Alright. I have mostly figured out the issues. I guess I was confused on how to use the in-slice time symmetry reduced
velocity in conjunction with the direct-matrix formulation before. I was attempting to use equations \refeq{e-symred}
in place
of something similar to \refeq{e-symred2} and this was a tragic mistake that took me a week to figure out.

The following are what I believe to be the correct implementation of symmetry reduction for the variational
method when applied to a {\rpo} that has been generated with a symmetry reduced integrator. In other words,
when the initial condition is a spatiotemporal discretization embedded in the slice

The new equation I am using for calculation of the symmetry reduced velocities (w.r.t. in-slice time) in the
direct-matrix formulation is,
\beq \label{e-KSsymDMvel}
\hat{v} = (diag(x_1(t)) \otimes I_{N-2})\cdot v - (diag(\dot{y}_1(t)) \otimes I_{N-2}) \cdot (I_M \otimes T) \cdot \Fu
\eeq

This is relatively simple to explain when using \refeq{e-symred2} as a reference. $diag(a_1(t))$ is a diagonal
matrix that when multiplied with the full \statesp\ velocity vector produces a vector whose entries
are equal to $x_1(t_i)*v(t_i)$ where $x_1$ is the real component of the first Fourier mode. likewise for $diag(\dot{y}_1(t))$
except $\dot{y}_1$ is the velocity of the imaginary component of the first Fourier mode. the $\otimes$ symbol here
denotes a Kronecker product (outer product) whose purpose is to copy the values $x_1$ and $y_1$ as to create a diagonal
matrix whose dimension equals the dimension of the velocity vector $v$. i.e. I need an outer product with a $N-2$ identity
matrix to ensure that I have $N-2$ copies of $x_1$ and $y_1$ so that each of the $N-2$ Fourier modes gets multiplied by
the correct factor.

To simplify the notation and make it more apparent how to derive the stability matrix in direct-matrix notation, I rewrite
$X_1 = diag(x_1(t)) \otimes I_{N-2}$ and $Y_1 = diag(\dot{y}_1(t)) \otimes I_{N-2}$.

The equation for the stability matrix for the entire orbit is therefore.

\beq \label{e-KSsymDMstb}
\frac{\partial \hat{v}}{\partial \Fu} = \hat{A} = X_1 \cdot A - Y_1 \cdot T
\eeq

I believe it might be that I have treated $X_1$ and $Y_1$ as constant matrices as opposed to functions of $\Fu$. This
is the only place that I find room for errors so far.

Small progress on symmetry reduced variational methods front. With some small changes to how I implemented
\label{e-KSsymDM0}. and playing around
with different discretizations of \RPO{16.3} I was able to get a final result whose value
of the cost functional is $F^2 \approx 10^{-12}$. This is four orders of magnitude of improvement
from the initial value of $F^2 \approx 10^{-8}$ but it's still not good enough.

The new formulations of \refeq{e-KSsymDMvel} and \refeq{e-KSsymDMstb} are written in the following ways, to more
explicitly represent the dependence of the matrices $X$ and $Y$ on the underlying quantities, namely the
orbit vector (vector representing the time discretizations of the spatial Fourier series)
and the full {\statesp} velocity vector.

\beq \label{e-KSsymDMvelnew}
\hat{v} = (X \cdot \Fu) \ast v - (\dot{Y} \cdot v) \ast (I_M \otimes T) \cdot \Fu
\eeq

with this (more) correct direct-matrix velocity equation we can then derive the corresponding stability matrix,
using the rules of calculus described in \refref{Chu09}

\beq \label{e-KSsymDMstbnew}
\frac{\partial \hat{v}}{\partial \Fu} = \hat{A} = diag(v)\cdot X + diag(X \cdot \Fu )\cdot A - diag(T \cdot \Fu)\cdot (Y \cdot A) - diag(Y \cdot v)\cdot T
\eeq

The new formulation helped me find an error in how I was deriving the stability matrix as there was dependence on
both the orbit vector $\Fu$ and velocity vector $v$ in the matrices $X_1$ and $Y_1$ in \refeq{e-KSsymDMstb} that
I was not accounting for.


Made some changes that make the problem seem more natural, namely, I included some terms such that everything
is in terms of linear operators acting on vectors of spatiotemporal Fourier coefficients as opposed
to a weird mixture I had before. The upside is that everything is defined spatiotemporally, but it might
look a little odd at first glance because there are certain terms which only have one type of Fourier transform operator being applied.

I also elected to change the temporal Fourier transform to a RFFT based description as opposed to a FFT
based description (FFT produces both sides of the spectrum, negative and positive Fourier modes based on index
where RFFT just produces half of the spectrum). I avoided this earlier because it was easier at first but
now that I have some experience this RFFT based description is definitely desired as it minimizes the number
of variables in the system. (It is possible, of course to take a FFT and then keep half of the terms but
this would be hard to implement given my direct-matrix implementation). The problem isn't completely
symmetric in terms of space and time however, as the Galilean invariance eliminates the need to keep
track of the terms corresponding to zeroth and Nyquist $(N/2)$ modes of the spatial spectrum; The easiest
way to describe this is that the discretization in spatiotemporal Fourier space is $N-2 by M$ where $N$
is the spatial discretization and $M$ is the temporal discretization.


The mapping now takes the form,
\beq \label{e-symredmapping}
W\cdot \Fu -F_t\cdot((X \cdot F_t^{-1} \Fu) \ast v - (\dot{Y} \cdot v) \ast (I_M \otimes T) \cdot F_t^{-1} \cdot \Fu) = 0
\eeq
where, $\Fu$ denotes a vector of spatiotemporal Fourier coefficients.

The symmetry reduced stability matrix needed for Newton's method then takes the form
\bea
M &=& W - F_t [diag(X \cdot F_t^{-1} \cdot \Fu)\cdot A - diag(v)\cdot(X\cdot F_t^{-1})
    \ceq
+ diag(Y\cdot v) \cdot (T \cdot F_t^{-1})
+ diag(T\cdot F_t^{-1} \cdot \Fu) \cdot (Y\cdot A)]
\eea
This is in addition to the partial derivatives with respect to in-slice time
and domain size that have yet to be elucidated.

I have been thinking for a while when this works how I will begin to apply it to glue different orbits together,
but finding {\rpo}s in a symmetry reduced space and {\ppo}s in full {\statesp} can't really be stuck together
in my mind so I think the correct way would be to find the {\rpo}s in reduced {\statesp}, reproduce
them in full {\statesp} through numerical integration and then try to glue them together. This wouldn't
warrant a further attempt to find another spatiotemporal fixed point however because the \rpo\ in full
{\statesp} through one prime period would not be periodic in time, but it would be periodic in space.
There a couple of thoughts I have had about this, one being that perhaps maybe when looking at the
problem spatiotemporally this means that rpo's must be maintained within the interior of a spatiotemporal
domain in order to ensure the tiling is a fixed point, this might point towards pruning rules of the underlying
spatiotemporal dynamics but it is kind of a cheap trick and isn't really what I had imagined, which was just
to treat {\ppo}s and {\rpo}s on the same footing in a symmetry reduced space; even though it isn't meaningful
(or possible?) to bring a \ppo\ to a slice, but I was hoping to be able to get a point where the different types
of periodic orbits would be building blocks treated as equal in a spatiotemporal setting.

In the spatial system, with periodic boundary conditions in time, i.e. a
slightly different problem than Dong and Lan\rf{DoLa14}, I am curious as
to whether it would be easier to develop symbolic dynamics there, as I
believe all spatial ``periodic orbits" are either temporal equilibria,
{\ppo}s in time or {\rpo}s in time, due to the spatial boundary
conditions imposed when looking at the time system. My main idea is that
because the existence of these different types of ``orbits" in the
spatial system depend on the spatial size of the system, which is
ostensibly the parameter controlling the onset of turbulence, perhaps it
would be easier to designate symbolic dynamics because there is somewhat
clear hierarchy as to when the corresponding objects appear; equilibria
appear first, then {\ppo}s and {\rpo}s.

I was having some weird issues when using two real-valued fft's (half-spectrum), so I have also
reverted to using a half-spectrum fft in space, and a full-spetrum fft in time. I tried rewriting
this portion of the code in probably five different ways but I reverted it due to the weird issues,
which were the spatiotemporal Fourier coefficients had the property that for $\Fu_kl$,
$Re[\Fu_kl] = 0$ for $l= even$, $Im[\Fu_kl] = 0$ for $l=odd$. I don't know why this is but I've reverted
back to the previous definitions, modified for different ordering.

The residual by using a least-squares solver on a 32 by 32 space-by-time parameterization of \RPO{16.3}
can currently be reduced to $10^{-9}$. Most of this reduction is obtained with the first few Newton steps,
where after the fourth step or so the residual is reduced to $10^{-8}$. Even after the application of a total
of 50 Newton steps the residual only reduces minimally (to the level previous stated). The slice condition
of the phase of the first Fourier mode to be zero is not being maintained, and I believe this is why the
residual does not fall to within machine precision. In my experience for variables that are constrained to
zero (i.e. the first Fourier mode phase) it is best to leave them out of the calculation of the least-squares
solver, like what is done for the zeroth spatial mode terms (which equal zero due to Galilean invariance), such
that no erroneous changes can be made, i.e. if they aren't a part of the calculation due to their values equaling
zero, they really should not be carried along by the numerical method, it should be implied that they are always zero,
and not kept track of.

The first attempt to improve the algorithm was simply to include constraints that make sure the Newton steps
are orthogonal to the directions of time and spatial equivariance by requiring the corresponding
inner products to be zero. This was done by including two additional rows to the previously over-determined
system that correspond to $\frac{\partial \Fu}{\partial \conf}$ and $\frac{\partial \Fu}{\partial \zeit}$
such that the following relations hold $<\frac{\partial \Fu}{\partial \conf},\delta \Fu >=0$, $<\frac{\partial \Fu}{\partial \zeit},\delta \Fu >=0$
I believe this is what was meant by Burak and PC in last week's meeting when they mentioned the hyperplane formed
by time and spatial equivariance. This offered no improvement over the least-squares solver solving the under-determined
system (i.e. the system without constraints).

The second constraint that I tried was to hold certain spatiotemporal Fourier coefficients fixed, by fixing their
real and imaginary components separately to yield two constraints. This also offered no improvements to the convergence,
but I need to see if whether this is due to the particular choice of Fourier coefficient still.

Another tactic that I tried to employ was to remove the symmetry reduced variables (the spatiotemporal spectrum
corresponding to the symmetry reduced first spatial modes) from the computation of Newton step corrections
using the least squares solver. Because the slice condition is to hold these components to be equal to zero,
I removed the corresponding rows and columns from the spatiotemporal \jacobianM\ to see whether the least-squares
solver was using them as extra degrees of freedom used to minimize the residual of the mapping, $F^{2}$.

What I noticed is that by removing them from the system, the residual \emph{increases}, in other words the convergence
to a spatiotemporal fixed point is worse. While it is hard for me to currently quantify, the extra degrees of freedom
seem to give extra leeway to the least-squares solver, which includes "corrections" to the symmetry reduced variables
of order $10^{-7}$ even though the computation of the mapping has zero magnitude as it should. In other words,
through computation of the spatiotemporal fixed point using Newton's method in conjunction with a least-squares solver,
a non-zero phase is picked up that, although relatively small, breaks the slice constraint.

With these changes, the in-slice spatiotemporal representation of \RPO{16.3} in
conjunction with the constraints to be orthogonal to the directions of time and space
equivariance brought the Newton residual to within machine precision in two steps,
when using an 32-by-32 space-by-time discretization (spatiotemporal Fourier modes)
of the \rpo.

The first would be to not have the spatiotemporal system be Fourier-Fourier
in spacetime but keep the time direction in physical space. This would enable
me to modify the spatiotemporal fixed point equation for {\ppo}s to be reduced
to solving for the corresponding spatiotemporal fixed point using only
prime periods if I used the finite difference scheme in conjunction with the
reflection symmetry operator to evaluate the time derivative.

The modified equation would be,
\beq
D \cdot u + Q_1 \cdot u + Q_2 \cdot FFT_x \cdot (IFFT_x \cdot u)^2 = 0,
\eeq
Where the only difference is that the spatiotemporal vector would now
represent $u_k(t_m)$ instead of $\Fu_{k,l}$.

The second option would be to remove the extraneous (zero valued) variables
from the system of equations, which would require a bit of effort as
I would have to modify all of the operators I am using in direct-matrix
notation.

The easier of the two by-far is the finite difference method in time so that
is what I am going to run through today.
The finite difference scheme with the reflection symmetry operation isn't
working too well for me so I am going to move on to the second method
which uses spatiotemporal Fourier coefficients and then discards the variables
that are zero.

I guess it makes sense that half of the spatiotemporal variables are discarded
because only half of them are truly unique due to the prime period being
the irreducible representation of the full orbit, and the prime period is
exactly half of the full orbit. A curious manifestation of the reflection
symmetry that I didn't realize.


Uploaded the beginnings of the python translations of the initial condition generation codes in order
to enable andy and myself to work completely in Python. The idea is to have a script that depending
on the user's choice will integrate in time, either in a symmetry reduced
subspace or not, with a random or user provided initial condition, integrate out the transient parts,
set up a Poincar\'e section and find a close recurrence mainly by itself, prepare the initial condition
by removing higher frequency Fourier mode components and saving the file so that it may be used
to test the spatiotemporal code. It's taking a while because what plan for is an amalgamation of
approximately five different codes in one. The main goal is to put ease-of-use at the forefront
to the user by changing function arguments rather than changing the code manually.

The definitions that I am using for the following equations are as such,
$Y$ denotes a matrix that picks out the first imaginary spatial mode component,
such that a matrix-vector product $Y \cdot v$ is equivalent to the inner product
$< v | t_p >$ where $t_p$ is the group tangent template vector, and $v$ is the
velocity. Note that the construction of $Y$ is completely dependent on the fact
that the inner product only has one non-zero term (namely, the imaginary component
of the first mode) due to the specific nature of the group tangent template $t_p$.

Likewise, $X$ is a matrix that will pick out the first real component, such that
the expression $X \cdot F^{-1}_t \cdot \Fu$ is equivalent to the inner product
$<t | t_p>$. Again, this is predicated on the group tangent template having such
a specific form. The general idea is that the group tangent row vectors $<t|$
are coordinate dependent, so in order to calculate them, we need to be in a
Fourier-Time representation, and not Fourier-Fourier.

With these definitions in mind, the dynamical time representation of the symmetry reduced
equations takes the form

\beq \label{MNGsymred_dyn_t}
\hat{v} = v - \frac{Y \cdot v}{X \cdot F^{-1}_t \Fu} .* (T \cdot F_t^{-1} \cdot \Fu)
\eeq

One can see that substitution of the spatiotemporal
terms with their "equivalent" inner product notations makes this equation equal to
\HREF{Chaosbook.org} equations $(13.7)$ and $(13.8)$. I put "equivalent" in quotes
because the inner products in the equations just listed are coordinate dependent
quantities, while this notation calculates everything at once.

The spatiotemporal mapping (applying Fourier transform of time of the symmetry reduced flow field)
the takes the form,

\beq
W \cdot \Fu - F_t \cdot( v - \frac{Y \cdot v}{X \cdot F^{-1}_t \Fu} .* (T \cdot F_t^{-1} \cdot \Fu)) = 0
\eeq

The symmetry reduced stability matrix then takes the form,

\beq
\hat{A} = A - diag(\frac{Y \cdot v}{X \cdot F_t^{-1}\cdot \Fu})\cdot T\cdot F_t^{-1}
- diag(T \cdot F_t^{-1}\cdot \Fu) \cdot \frac{diag(X \cdot F_t^{-1}\cdot \Fu)\cdot(Y\cdot A) -diag(Y \cdot v)\cdot X \cdot F_t^{-1}}{(X \cdot F_t^{-1}\cdot \Fu)^2}
\eeq

Again with the substitutions $X \cdot F^{-1}_t \cdot \Fu \equiv <t | t_p>$ and $Y \cdot v \equiv < v | t_p >$ the
equation is reminiscent of the corresponding equation in  \HREF{Chaosbook.org} equation $(13.24)$.

The full spatiotemporal stability matrix then takes the form,

\beq
\frac{\partial G}{\partial \Fu} = W + F_t \cdot \hat{A}
\eeq 

where, $s$ is the parameterization variable, here defined with a subscript $n$
to indicate that the parameterization need not be uniform around the initial guess
loop (although this is what I work worth as it makes things much easier to program).
$t$ in this instance stands for the real dynamical time of the orbit.

If we substitute the definition for the in-slice time, this equation takes the form

\beq
\lambda_n = \frac{\Delta \hat(t)_n (x_1)_n }{\Delta s_n}
\eeq

Going through the almost identical derivation for the system of equations there
is one place where I believe they differ. Normally, there is the substitution

\beq
\delta \tn = \frac{\partial \lambda_n}{\partial \tau} \Delta s_n \delta \tau
\eeq

Normally this is easily generalized in order to produce a uniform rescaling of the period
around the orbit,
but if the coordinate $x_1$ is involved I believe that this should take the form

\beq \label{e-inslicevarmeth}
\delta \tn = (\frac{\partial \lambda_n}{\partial \tau}(x_1)_n + \frac{\partial (x_1)_n}{\partial \tau} \lambda_n)\Delta s_n \delta \tau
\eeq

As the in-slice time rescaling is coordinate dependent, this formula seems to beg for a description
of the general equation that has a coordinate dependent rescaling. This is kind of what I have been stuck on
as this is quite different from what I am used to.

For a orbit that is described by $M$ points in time we would need ${\lambda_i}$ where $i = 0, ... , M-1$.
I'm currently working on implementing this into my current code but it's gotten the best of me so far.

The way I am attempting to solve the problem I am having with in-slice time is as such,
for $M$ discretized points representing an in-slice time description of a {\rpo} I am introducing
$M$ different time rescaling quantities $\lambda_m$. In this way the original variational
method equation changes from
\beq
\begin{bmatrix} D - \lambda diag(A_0, ... , A_{M-1}), & -v \end{bmatrix}
       \left[ \begin{array}{c} \delta \tilde{x} \\ \delta \lambda \end{array} \right] =
    \delta \tau \left[ \begin{array}{c} \lambda v - \tilde{v} \end{array} \right],
\eeq
to
\bea
&&\begin{bmatrix} D - diag(\lambda_0 * A_0, ... , \lambda_{M-1} * A_{M-1}), & -diag(v_0, ..., v_{M-1}) \end{bmatrix}  \left[ \begin{array}{c} \delta \tilde{x} \\ \delta \lambda_{m}
            \end{array} \right]
\continue
&&=
    \delta \tau \left[ \begin{array}{c} diag(\lambda_m) v - \tilde{v} \end{array} \right],
\eea
In this way the correction mentioned in \refeq{e-inslicevarmeth} is not being resolved but I want to try
to see if more flexibility in the rescaling of in-slice time with respect to fictitious time evolution
is sufficient before getting into an equation I derived; it's also a simpler step whose changes could be
carried over to the full in-slice time description I probably will need.

Wrote some new code for symmetry reduced spatiotemporal fixed point finding; currently
it seems a little strange to me because I only know how to quotient the spatial translation
symmetry in spatial Fourier space as opposed to spatiotemporal (double) Fourier space.

The way that this I am attempting this, are with the equations, in direct-matrix notation:

Using the definition of the reduced velocity in ti
me, \refeq{e-KSsymDMvelnew}, restated here
for reference,
\beq
\hat{v} = (X \cdot \utensor) \ast v - (\dot{Y} \cdot v) \ast (I_M \otimes T) \cdot \utensor,
\eeq
We then find the equivalent spatiotemporal system by Fourier transforming in time by means
of application of $F_t$ linear operator denoting the transform and moving everything to one
side. Here, the $\utensor$ stands for a description of the initial solution that is only Fourier in space.
\beq
W\cdot F_t \cdot \utensor -F_t\cdot((X \cdot \utensor) \ast v - (\dot{Y} \cdot v) \ast (I_M \otimes T) \cdot \utensor) = 0
\eeq
The stability matrix resulting from this symmetry reduced mapping is therefore,

where $\hat{A}$ is in reference to the equation previously defined \refeq{e-KSsymDMstbnew}.

\beq
J = W\cdot F_t - F_t\cdot \hat{A}
\eeq