
I messed around with a calculation but just came to the same conclusion I had previously reached, namely if the adjoint variable $v$
is chosen to be the \KSe\ then the variational derivative of the formal Lagrangian in the Ibragimov sense corresponds to the adjoint
descent direction. This is perhaps why it works so well for me. \ie the variational derivative obeys the equation

\bea
\frac{\delta L(u,v)}{\delta u} &=& -v_t + v_{xx} + v_{xxxx} - uv_x \continue
\frac{\delta L(u,-F(u))}{\delta u} &=& F_t -F_{xx} -F_{xxxx} + u F_x \equiv -J^{\dagger} F
\eea

Following sect.~2 {\em Quasi-self-adjoint equations} of
Ibragimov\rf{Ibragimov07b} (which \PCedit{does not} reference
\refrefs{Ibragimov06,Ibragimov07a}),
%I believe
we can write
the \emph{formal Lagrangian} of the \KSe\ to derive the {\spt}
adjoint equations in terms of the original {\spt} field
$u(x,t)$, and then one is free to use whatever representation
suits the user in discretization; the cavaet is that numerically it seems
better to use a real valued representation Fourier representation for the
adjoint descent.
    \PC{2018-05-08} {Ibragimov\rf{Ibragimov07b} is included in
    \HREF{http://gammett.ugatu.su/images/archives_of_alga/archives\%204_1.pdf}
    {Archives of ALGA {\bf 4}}. }

Ibragimov notation, for the \KS\ case: the independent variable is
denoted by $x$. The dependent variable is $u$,
used together with its first-order partial derivative
   $u_{(1)}:$
  $$
  u_{(1)} = \{u^\alpha_i\}, \quad u^\alpha_i = D_i (u^\alpha),
  $$
and higher-order derivatives $u_{(2)}, \ldots, u_{(4)}\,,$ where
 $$
  u_{(2)} = \{u^\alpha_{ij}\}, \quad u^\alpha_{ij} = D_i D_j
  (u^\alpha),\ldots,
  $$
% up to $s$th-order  derivatives $u_{(s)}:$
  $$
  u_{(s)} = \{u^\alpha_{i_1\cdots i_s}\}, \quad u^\alpha_{i_1\cdots i_s}
  = D_{i_1} \cdots D_{i_s}(u^\alpha).
  $$
  Here $D_i$ is the total differentiation with respect to $x^i:$
 \begin{equation}
 \label{sa:cl.diff1}
  D_i = \frac{\partial}{\partial x^i} +
u^\alpha_{i}\frac{\partial}{\partial u^\alpha} +
u^\alpha_{ij}\frac{\partial}{\partial u^\alpha_j} +
 \cdots\,.
 \end{equation}


Using the definition for the \emph{formal Lagrangian} $\mathcal{L}$,
\beq
\mathcal{L} \equiv v \left[u_{t} + u_{x x} + u_{x x x x}
                        + uu_{x} \right],
\ee{FormalLagrangian}
and then taking the functional derivative,
% (I tend to think of this as a total
%  derivative but that might be confusing notation),
\beq
\frac{\delta \mathcal{L}}{\delta u} = 0 \,.
\ee{variationalequality}
The surviving terms from this functional derivative are
\bea \label{LagrangianDeriv}
\frac{\partial \mathcal{L}}{\partial u}              &=& vu_{x} \continue
-\partial_{t} \frac{\partial \mathcal{L}}{\partial u_{t}} &=& -v_t \continue
-\partial_{x} \frac{\partial \mathcal{L}}{\partial u_{x}} &=& -vu_{x}-uv_{x} \continue
\partial_{x}^2 \frac{\partial \mathcal{L}}{\partial u_{x x}} &=& v_{x x} \continue
\partial_{x}^4 \frac{\partial \mathcal{L}}{\partial u_{x x x x}} &=& v_{x x x x},
\eea
where the sum of these terms equals \refeq{variationalequality} and hence
must be zero. The resultant adjoint equation is ($\pm
vu_{x}$ terms cancel),
\beq
-v_t + v_{x x} + v_{x x x x} - uv_{x} = 0
\,.
\ee{adjoint_equation}

If we take the adjoint variable to be the \KSe,
\beq
F \equiv v = - \left[u_{t} + u_{x x} + u_{x x x x}
                        + uu_{x} \right]
\,,
\ee{adjointchoice}
we arrive at the equation which \emph{I claim} provides adjoint descent
direction without explicit construction of the gradients matrix $J$,
\beq
-J^{\dagger}F = (\partial_{t} + \partial_{x}^2 + \partial_{x}^4)F
                    + u \partial_{x} F
\ee{adjointdescent}
(I believe the negative sign in \refeq{adjointchoice} is motivated so
that the functional derivative is strictly decreasing or zero).

Numerical evidence is suggestive as the real valued adjoint descent is
working better than before when I was trying to reverse engineer
$J^{\dagger}F$ in a matrix-free way.





I messed around with a calculation but just came to the same conclusion I had previously reached, namely if the adjoint variable $v$
is chosen to be the \KSe\ then the variational derivative of the formal Lagrangian in the Ibragimov sense corresponds to the adjoint
descent direction. This is perhaps why it works so well for me. \ie the variational derivative obeys the equation

\bea
\frac{\delta L(u,v)}{\delta u} &=& -v_t + v_{xx} + v_{xxxx} - uv_x \continue
\frac{\delta L(u,-F(u))}{\delta u} &=& F_t -F_{xx} -F_{xxxx} + u F_x \equiv -J^{\dagger} F
\eea
