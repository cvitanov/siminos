
This is implemented by introducing the preconditioning matrix $M$ such that the new adjoint direction
is given by  $\partial_\tau \Fu = -M\transp{J}(\Fu) F(\Fu)$

The other advancement is described in (its a secret Russian paper that only few may read apparently)
\rf{Nesterov83} that proves that by including a 'momentum' term in the iterative process, then
the descent process can be sped up by including a contribution from the previous fictitious time
direction. The algorithm that this manifests in is given by

\bea \label{eqn:Nesterov_factor}
\mu_0 &=& 1 \continue
\mu_k &=& \frac{1 + \sqrt{1+4*\mu_k^2}}{2} \continue
\delta \Fu_{k+1} &=& \delta \Fu_k + \frac{\mu_k-1}{\mu_k +1} \delta \Fu_{k-1}
\eea

If the residual increases then the momentum is restarted at $\mu = 1$ and the iterative
procedure begins again.

Now specifically, I haven't found an initial condition that the full hybrid process
in conjunction with Newton works yet. So the next steps are to implement Newton-Krylov
hookstep methods (finally getting to it after everything else) and clean up the initial
condition generation code (which I started today).

If both of these changes does nothing then the last resort will be to pass close recurrence
initial conditions to a Newton-descent in time in order to give one of the tangent spaces the
correct shape before starting the spatiotemporal hybrid descent.