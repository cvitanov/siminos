% siminos/gudorf/thesis/chapter/LieGroupAnalysis.tex
% $Author: predrag $ $Date: 2020-10-24 01:45:26 -0400 (Sat, 24 Oct 2020) $

    \PC{2019-09-11}{
I am thinking of moving everything from \refeq{e-formallagrangian} to
\refeq{e-conservedvector}
into a Chapter of your thesis, remarking here briefly that Ibragimov
methods seem to have not worked for you?
    }
To explore this we will introduce what is sometimes
referred to as the
\textit{formal Lagrangian}
\refrefs{BorSch11,WeiWan19,Ibragimov11,Ibragimov10,Ibragimov07b,Ibragimov18,KraMaj15}
\beq \label{e-formallagrangian}
\mathcal{L} = v G(u(x,t),\dots,u_{(n)}(x,t))\,.
\eeq
This construction introduces the adjoint variable
$v$, which is quitessentially a Lagrange multiplier
as the form of \refeq{e-formallagrangian} implies.
Although relatively simplistic in its construction, the
corresponding Euler-Lagrange equations of the
formal Lagrangian \refeq{e-formallagrangian} provide
us with the \KSe\ \refeq{e-ks}
and its adjoint \refeq{e-adjointeqn}.
The Euler-Lagrange equations can be produced
by taking the variational derivative \refeq{e-variationalderivative}
of \refeq{e-formallagrangian}
with respect to either $u$ and $v$ separately.
\beq \label{e-variationalderivative}
\frac{\delta }{\delta u} = \frac{\partial}{\partial u} + \sum_{n=1}^{\infty} (-1)^s D_{i_1} \dots D_{i_n} \frac{\partial}{\partial u_{i_1\dots i_n}}
\eeq
which is the operator that
produces the Euler-Lagrange equations of the
corresponding Lagrangian\rf{Ibragimov10}.

The recovery of the original equation is self evident as
$\frac{\delta \mathcal{L}}{\delta v}=\frac{\partial \mathcal{L}}{\partial v}=G$.
For the adjoint equation we have
\beq \label{e-adjointeqn}
\frac{\delta \mathcal{L}}{\delta u}
= G^*(x,t,u,v,v_t,v_x,v_{xx},v_{xxxx}) =  - v_t + v_{xx} + v_{xxxx} - u v_x
\eeq
both \refeq{e-adjointeqn} and \refeq{e-ks} create
what is known as the \textit{optimality system}
of equations\rf{Gunzburger02}, show in \refeq{e-optimalitysystem} for completeness.
\bea \label{e-optimalitysystem}
\frac{\partial \mathcal{L}}{\partial v} &\equiv& G =  u_t + u_{xx} + u_{xxxx} + u u_x \continue
\frac{\partial \mathcal{L}}{\partial u} &\equiv& G^* =  -v_t + v_{xx} + v_{xxxx} - uv_x
\eea

The adjoint variable $v$ is to eventually be replaced by a function of
$u$ and its derivatives such that the system of equations
\refeq{e-optimalitysystem} is solved whenever $u$ is a solution. The main
method to determine the explicit form of this substitution is to find the
form of $v$ which transforms the adjoint equation into the original
equation. If the substitution $v=u$ satisfies this requirement, then the
equation is said to be \textit{self adjoint}. Specifically, the operator
resultant from linearization of \refeq{e-ks} is the actual object with
this property. Other than self-adjointness there are also the notions of
quasi-self adjoint equations and nonlinearly adjoint equations
\refrefs{WeiWan19, Ibragimov10,Ibragimov11,Ibragimov18}. The
substitutions that define quasi self-adjoint equations and nonlinearly
self-adjoint equations are functions $v = \phi(u)$ and
$v=\phi(x^i,u,u^{(n)})$ that transform \refeq{e-adjointeqn} into
\refeq{e-ks} (the term $u^{(n)}$ represents all derivatives up to order
$n$).
Upon substitution,
the self-adjoint condition can be written explicitly
as
\beq \label{e-selfadjoint}
G^*(x,t,u,\phi(x,t,u,u^{(n)})) = \lambda G(x,t,u)\,,
\eeq
For the adjoint
equation \refeq{e-adjointeqn}
the only solution that guarantees the adjoint equation \refeq{e-adjointeqn}
is satisfied when $u$ is a solution is $v=c$ where $c$ is a constant.
This is determined by making the substitution $v=\phi(x^i,u,u^{(n)})$
and solving the set of equations similar to the determining
equations which produced the generators of the Lie
algebra of infinitesimal symmetries. Specifically, the result
of matching terms in \refeq{e-selfadjoint}
can be summarized by two of the resulting equations,
\bea
\phi_u &=& \lambda \continue
\phi_u &=& -\lambda \,,
\eea
which in turn implies that $\lambda$ is equal to zero.
This might seem like a red flag, the \twots\ of the \KSe\
are critical points of \refeq{e-formallagrangian} and this
seems to indicate that is not possible; but $v=G(u)$
is actually a representation of the trivial solution when $u$
is a solution because $G(u)=0$, so there is no contradiction.

There is an another  whose
Euler-Lagrange equations are the same.
Following the analysis of Burgers' equation and
the prescription of\rf{IbragiKols04}
we find an alternative form of the Lagrangian for the \KSe\ to be
\beq
\mathcal{L} = \frac{1}{2}(v u_t - u v_t) -u_x v_x + u_{xx}v_xx + \frac{u}{3}(vu_x - uv_x)
\eeq
There are a number of mathematical and
numerical techniques available
to our doubly periodic variational problem
and not the
equivalent dynamical system. For instance, we can utilize
Lie group analysis
to derive continuous symmetries pertaining to our Lagrangian.
y Noether's
Theorem these symmetries imbue the corresponding
Euler-Lagrange equations with conservation laws\rf{Noether15}.
On the numerical front we can utilize \spt\ two dimensional
spectral methods (collocation methods, technically).

Pertaining
to our system of equations, \refeq{e-ks} and \refeq{e-adjointeqn}.
There are a number of different paths we can take for this
type of analysis but for now we present the investigation of
whether or not there are any non-trivial conservation laws
involving the Euler-Lagrange equations resulting from
the Lagrangian density \refeq{e-formallagrangian}.
As a note we could write \refeq{e-formallagrangian}
in an alternate manner, and then search for variational
symmetries\rf{liede}. These would be imparted
onto \refeq{e-ks} and \refeq{e-adjointeqn}
because they are the corresponding Euler-Lagrange equations.

We already know the equations with which we intend
to work so we may begin after introducing some notation.
In order to derive conserved quantities using the
machinery of\rf{Ibragimov07a} we first need
to find the vector fields that span the Lie
algebra of infinitesimal symmetries of our
equations. Our description
follows Theorem 2.36 from\rf{liede}
in terms of notation.

%%%%%%%%%


To begin, take an arbitrary
 vector field defined on the space $X \times U$
which contains the independent variables $x^i$ and
dependent variables $u^{\alpha}$ such that
\beq \label{e-generalvectorfield}
\mathbf{v} = \xi^i(x^i,u^{\alpha}) \frac{\partial}{\partial x^i}
+ \phi_{\alpha}(x^i,u^{\alpha})\frac{\partial}{\partial u^{\alpha}}\,,
\eeq
where summation is implied by repeated indices.
To create a vector field applicable to our equations,
we need to ``prolong''
\refeq{e-generalvectorfield} or perform a ``jet prolongation''
to the jet space of the same order as our equations,$n$.
Informally this just means extending the definition \refeq{e-generalvectorfield}
to the same order as the equations being studied.
The general formula for the prolongation to
the $n$th jet bundle is
\beq
\mathbf{\text{pr}}^{(n)}v = v + \phi_{\alpha}^J \frac{\partial}{\partial u_J^{\alpha}}\,,
\eeq
where the coefficients $\phi_{\alpha}^J$ are
given by
\beq \label{e-vectorcoeff}
\phi_{\alpha}^J(x,u^{(n)} = D_J (\phi_{\alpha}-\xi^i u_i^{\alpha}) + \xi^{i} u_{J,i}^{\alpha}.
\eeq

We can now begin to apply this
to our equation of interest, the \KSe, \refeq{e-ks}.
Starting with the prolongation of
the general vector field, We need the fourth
prolongation which seems like a lot of work
(there is a coefficient for every combination of
partial derivatives, and each higher order
coefficient becomes more involved due to increased
numbers of differentiation operations). Luckily,
we already know that we are going to apply
the vector field to the \KSe\ such that instead
of calculating all $2+4+8+16$ jet prolongation
coefficients (all combinations of
$t,x$ derivatives of order one, two, three and four)
we only need the coefficients which accompany
vectors $\frac{\partial}{\partial u_J}$ which
appear in the \KSe. Namely, $\{\phi^J\} = \{\phi^t,\phi^x,\phi^{xx},\phi^{xxxx} \}$,
which in turn creates the vector field
specific to the \KSe
\bea \label{e-prolong4}
\mathbf{\text{pr}}\,v^{(4)} &=& \epsilon(x,t,u)\frac{\partial}{\partial x}
                            +\tau(x,t,u)\frac{\partial}{\partial t}
                            +\phi(x,t,u)\frac{\partial}{\partial u}
                            +\phi^t(x,t,u^{(1)})\frac{\partial}{\partial u_t} \continue
                            &+&\phi^x(x,t,u^{(1)})\frac{\partial}{\partial u_x}
                            +\phi^{xx}(x,t,u^{(2)})\frac{\partial}{\partial u_{xx}}
                            +\phi^{xxxx}(x,t,u^{(4)})\frac{\partial}{\partial u_{xxxx}}\,.
\eea

All other higher order terms will annihilate when
acting on the \KSe. Note that the higher the ``order'' of the
coefficient, the higher the order of the jet bundle that
the coefficients depend on. This can
be seen by the definition of the coefficients
\refeq{e-vectorcoeff}, the higher the ``order'' of
the coefficients, the more derivatives are taken to define
it.
Now that we have the general
form of the vector field we can begin to derive
the \textit{infinitesimal generators} which span
the Lie algebra. To accomplish this, we will derive
the \textit{determining equations} which are produced
by applying \refeq{e-generalvectorfield} to the system of differential
equations and equating to zero, that is
\beq  \label{e-phicoefficients} % label added by \PC{2019-09-11}, could be wrong
\mathbf{\text{pr}}\,v^{(4)}(G(u^{(\alpha)}(x,t),u_{(1)}^{(\alpha)}(x,t),\dots,u_{(n)}^{(\alpha)}(x,t))) = 0
\,.
\eeq
Performing this operation yields
\beq \label{e-KSEcoeff}
\phi^t + \phi^{xx} + \phi^{xxxx} + u \phi^x + \phi u_x = 0
\,.
\eeq

We finally are forced to derive the coefficients $\phi^J$
and to include as many details as possible
we will write the exact formulas needed to derive them
as well as the long form expressions that they are equal to.

\bea
\phi^t &=& D_t (\phi(x,t,u) -\epsilon(x,t,u) u_x -\tau(x,t,u) u_t) + \tau(x,t,u) u_tt + \epsilon(x,t,u) u_xt \continue
\phi^x &=& D_x (\phi(x,t,u) -\epsilon(x,t,u) u_x -\tau(x,t,u) u_t) + \tau(x,t,u) u_tx + \epsilon(x,t,u) u_xx  \continue
\phi^{xx} &=& D_{x}^2 (\phi(x,t,u) -\epsilon(x,t,u) u_x -\tau(x,t,u) u_t) + \tau(x,t,u) u_{txx} + \epsilon(x,t,u) u_{xxx} \continue
\phi^{xxxx} &=& D_{x}^4 (\phi(x,t,u) -\epsilon(x,t,u) u_x -\tau(x,t,u) u_t) + \tau(x,t,u) u_{txxx} + \epsilon(x,t,u) u_{xxxx}
\eea

where $D_t$ and $D_x$ represent \textit{total differentiation} operators,

\beq
D_i = \frac{\partial}{\partial x^i} + u_i \frac{\partial}{\partial u} + u_{ii} \frac{\partial}{\partial u_i} + \dots
\eeq

the long form expressions from each of these are
\beq
\phi ^t=u_t^2 \left(-\tau _u\right)-\tau _t u_t-u_t u_x \epsilon _u-\epsilon _t u_x+u_t \phi _u+\phi _t
\eeq

\beq
\phi ^x=-u_t \tau _u u_x-u_t \tau _x+u_x^2 \left(-\epsilon _u\right)-u_x \epsilon _x+u_x \phi _u+\phi _x
\eeq

\bea
\phi ^{\text{xx}}&=&-u_t u_x^2 \tau _{\text{uu}}-2 u_t u_x \tau _{\text{xu}}-u_t \tau _u u_{\text{xx}}\continue
&-&u_t \tau _{\text{xx}}+u_x^3 \left(-\epsilon _{\text{uu}}\right)+u_x^2 \phi _{\text{uu}}\continue
&-&2 \tau _u u_x u_{\text{xt}}-2 u_{\text{xt}} \tau _x-2 u_x^2 \epsilon _{\text{xu}}\continue
&+&2 u_x \phi _{\text{xu}}-3 u_x u_{\text{xx}} \epsilon _u-u_x \epsilon _{\text{xx}}\continue
&-&2 u_{\text{xx}} \epsilon _x+u_{\text{xx}} \phi _u+\phi _{\text{xx}}
\eea

\bea
\phi ^{\text{xxxx}}&=&-4 u_t u_x u_{\text{xxx}} \tau _{\text{uu}}-3 u_t u_{\text{xx}}^2 \tau _{\text{uu}}-6 u_t u_x^2 u_{\text{xx}} \tau _{\text{uuu}}-u_t u_x^4 \tau _{\text{uuuu}} \continue
&-&12 u_t u_x u_{\text{xx}} \tau _{\text{xuu}}-4 u_t u_x^3 \tau _{\text{xuuu}}-6 u_t u_x^2 \tau _{\text{xxuu}}-4 u_t u_x \tau _{\text{xxxu}}-4 u_t u_{\text{xxx}} \tau _{\text{xu}}\continue
&-&6 u_t u_{\text{xx}} \tau _{\text{xxu}}-u_t \tau _u u_{\text{xxxx}}-u_t \tau _{\text{xxxx}}-12 u_x u_{\text{xt}} u_{\text{xx}} \tau _{\text{uu}} \continue
&-&15 u_x u_{\text{xx}}^2 \epsilon _{\text{uu}}-6 u_x^2 u_{\text{xxt}} \tau _{\text{uu}}-10 u_x^2 u_{\text{xxx}} \epsilon _{\text{uu}}+4 u_x u_{\text{xxx}} \phi _{\text{uu}} \continue
&+&3 u_{\text{xx}}^2 \phi _{\text{uu}}-4 u_x^3 u_{\text{xt}} \tau _{\text{uuu}}-10 u_x^3 u_{\text{xx}} \epsilon _{\text{uuu}}+6 u_x^2 u_{\text{xx}} \phi _{\text{uuu}}\continue
&+&u_x^5 \left(-\epsilon _{\text{uuuu}}\right)+u_x^4 \phi _{\text{uuuu}}-12 u_x^2 u_{\text{xt}} \tau _{\text{xuu}}-12 u_x u_{\text{xt}} \tau _{\text{xxu}}\continue
&-&12 u_x u_{\text{xxt}} \tau _{\text{xu}}-16 u_x u_{\text{xxx}} \epsilon _{\text{xu}}-24 u_x^2 u_{\text{xx}} \epsilon _{\text{xuu}}+12 u_x u_{\text{xx}} \phi _{\text{xuu}}\continue
&-&4 u_x^4 \epsilon _{\text{xuuu}}+4 u_x^3 \phi _{\text{xuuu}}-18 u_x u_{\text{xx}} \epsilon _{\text{xxu}}-6 u_x^3 \epsilon _{\text{xxuu}}+6 u_x^2 \phi _{\text{xxuu}}\continue
&-&4 \tau _u u_x u_{\text{xxxt}}-4 u_{\text{xxxt}} \tau _x-4 u_x^2 \epsilon _{\text{xxxu}}+4 u_x \phi _{\text{xxxu}}-5 u_x u_{\text{xxxx}} \epsilon _u-u_x \epsilon _{\text{xxxx}}\continue
&-&4 u_{\text{xxxx}} \epsilon _x-12 u_{\text{xt}} u_{\text{xx}} \tau _{\text{xu}}-4 \tau _u u_{\text{xt}} u_{\text{xxx}}-4 u_{\text{xt}} \tau _{\text{xxx}}\continue
&-&12 u_{\text{xx}}^2 \epsilon _{\text{xu}}+4 u_{\text{xxx}} \phi _{\text{xu}}-6 \tau _u u_{\text{xx}} u_{\text{xxt}}-6 u_{\text{xxt}} \tau _{\text{xx}}\continue
&+&6 u_{\text{xx}} \phi _{\text{xxu}}-10 u_{\text{xx}} u_{\text{xxx}} \epsilon _u-6 u_{\text{xxx}} \epsilon _{\text{xx}}-4 u_{\text{xx}} \epsilon _{\text{xxx}}\continue
&+&u_{\text{xxxx}} \phi _u+\phi _{\text{xxxx}}
\eea

upon substitution into \refeq{e-KSEcoeff} we can separate the terms by coefficients
of monomials which gives us the determining equations as previously mentioned
\beq
\begin{array}{c}
 \phi _t+\phi _{\text{xx}}+\phi _{\text{xxxx}}=0 \\
 -4 \tau _x=0 \\
 -6 \tau _{\text{xx}}=0 \\
 -2 \tau _x-4 \tau _{\text{xxx}}=0 \\
 -4 \epsilon _x+\tau _t+\tau _{\text{xx}}+\tau _{\text{xxxx}}=0 \\
 4 \phi _{\text{xu}}-6 \epsilon _{\text{xx}}=0 \\
 -4 \tau _u=0 \\
 4 \tau _{\text{xu}}=0 \\
 -2 \epsilon _x-4 \epsilon _{\text{xxx}}+\tau _t+\tau _{\text{xx}}+\tau _{\text{xxxx}}+6 \phi _{\text{xxu}}=0 \\
 -6 \tau _u=0 \\
 -12 \tau _{\text{xu}}=0 \\
 6 \tau _{\text{xxu}}=0 \\
 4 \tau _{\text{xu}}-10 \epsilon _u=0 \\
 -12 \epsilon _{\text{xu}}+6 \tau _{\text{xxu}}+3 \phi _{\text{uu}}=0 \\
 3 \tau _{\text{uu}}=0 \\
 3 \tau _{\text{uu}}=0 \\
 \phi -\epsilon _t-\epsilon _{\text{xx}}-\epsilon _{\text{xxxx}}+2 \phi _{\text{xu}}+4 \phi _{\text{xxxu}}=0 \\
 -4 \tau _u=0 \\
 -12 \tau _{\text{xu}}=0 \\
 -2 \tau _u-12 \tau _{\text{xxu}}=0 \\
 -4 \epsilon _u+2 \tau _{\text{xu}}+4 \tau _{\text{xxxu}}=0 \\
 4 \phi _{\text{uu}}-16 \epsilon _{\text{xu}}=0 \\
 4 \tau _{\text{uu}}=0 \\
 -2 \epsilon _u-18 \epsilon _{\text{xxu}}+2 \tau _{\text{xu}}+4 \tau _{\text{xxxu}}+12 \phi _{\text{xuu}}=0 \\
 -12 \tau _{\text{uu}}=0 \\
 12 \tau _{\text{xuu}}=0 \\
 4 \tau _{\text{uu}}=0 \\
 12 \tau _{\text{xuu}}-15 \epsilon _{\text{uu}}=0 \\
 -2 \epsilon _{\text{xu}}-4 \epsilon _{\text{xxxu}}+\phi _{\text{uu}}+6 \phi _{\text{xxuu}}=0 \\
 -6 \tau _{\text{uu}}=0 \\
 -12 \tau _{\text{xuu}}=0 \\
\end{array}
\eeq

\beq \nonumber
\begin{array}{c}
 \tau _{\text{uu}}+6 \tau _{\text{xxuu}}=0 \\
 -10 \epsilon _{\text{uu}}=0 \\
 -24 \epsilon _{\text{xuu}}+\tau _{\text{uu}}+6 \tau _{\text{xxuu}}+6 \phi _{\text{uuu}}=0 \\
 6 \tau _{\text{uuu}}=0 \\
 6 \tau _{\text{uuu}}=0 \\
 -\epsilon _{\text{uu}}-6 \epsilon _{\text{xxuu}}+4 \phi _{\text{xuuu}}=0 \\
 -4 \tau _{\text{uuu}}=0 \\
 4 \tau _{\text{xuuu}}=0 \\
 4 \tau _{\text{xuuu}}-10 \epsilon _{\text{uuu}}=0 \\
 \phi _{\text{uuuu}}-4 \epsilon _{\text{xuuu}}=0 \\
 \tau _{\text{uuuu}}=0 \\
 \tau _{\text{uuuu}}=0 \\
 -\epsilon _{\text{uuuu}}=0 \\
 \phi _x=0 \\
 \tau _x=0 \\
 \tau _x=0 \\
 -\epsilon _x+\tau _t+\tau _{\text{xx}}+\tau _{\text{xxxx}}=0 \\
 4 \tau _{\text{xu}}=0 \\
 6 \tau _{\text{xxu}}=0 \\
 3 \tau _{\text{uu}}=0 \\
 2 \tau _{\text{xu}}+4 \tau _{\text{xxxu}}=0 \\
 4 \tau _{\text{uu}}=0 \\
 12 \tau _{\text{xuu}}=0 \\
 \tau _{\text{uu}}+6 \tau _{\text{xxuu}}=0 \\
 6 \tau _{\text{uuu}}=0 \\
 4 \tau _{\text{xuuu}}=0 \\
 \tau _{\text{uuuu}}=0 \\
 \tau _x=0 \\
\end{array}
\eeq

While initially intimidating, these equations can be solved by noticing
the lower order equations such as $\tau_x = \tau_u = 0$ which means
that $\tau$ can only be a function of $t$. Following this reasoning
we find that in fact

\bea
\tau(x,t,u) &=& \tau = c_1 \continue
\epsilon(x,t,u)&=&\epsilon(t)= c_3 t + c_1 \continue
\phi(x,t,u)&=& \phi = c_3
\,,
\eea
such that the Lie algebra of infinitesimal symmetries
is spanned by
\bea \label{e-ksegenerators}
v_1 &=& \partial_x \continue
v_2 &=& \partial_t \continue
v_3 &=& t \partial_x + \partial_u
\,,
\eea
which are the generators of space and time translations, and Galilean
transformations. This is not surprising, as these symmetries have been
previously described\rf{BudCvi14}. The reason why this calculation was pursued
in the first place was to see if there were any ``hidden'' continuous symmetries
afforded by a \spt\ formulation that were not present when
the problem was viewed as a dynamical system.
This is true for \textit{discrete symmetries}, but unfortunately not so
for continuous symmetries.
To carry the calculation through to finality
we need to know the prolongations of \refeq{e-ksegenerators}
and their extensions
to the adjoint variables present in \refeq{e-adjointeqn}, as
the Lie algebra needs to be extended to account for
both Euler-Lagrange equations.

The prolongations of \refeq{e-ksegenerators} result in
\bea \label{e-ksegeneratorsprolong}
\mathbf{\text{pr}}\,v_1 &=& y_1 = \partial_x \continue
\mathbf{\text{pr}}\,v_2 &=& y_2 = \partial_t \continue
\mathbf{\text{pr}}\,v_3 &=& y_3 = \partial_x + \partial_u - u_x \partial_{u_t}
\,.
\eea

We can now derive the extended versions of \refeq{e-ksegeneratorsprolong}
such that we can apply them to the formal Lagrangian
\refeq{e-formallagrangian}. Once again we deploy the machinery of
Ibragimov to extend \refeq{e-ksegeneratorsprolong} to the adjoint
variables. Unfortunately it seems that the symmetries were too simple to
actually have extensions to the adjoint variables, but we can still go
forward with the conservation law calculations regardless. Both
Ibragimov\rf{Ibragimov07a} and Olver\rf{liede} work through the
proof that there is a conserved vector (as Ibragimov names it) such that
its divergence provides a conservation law (technically infinite number
of conservation laws because they are equations involving PDE solutions).
The components of the conserved vector (one for each independent
variable) are given by
\bea \label{e-conservedcomponent}
C^i &=& \xi^i \mathcal{L} + W^{\alpha}[\frac{\partial \mathcal{L}}{\partial u^{\alpha}_i}-D_j \frac{\partial \mathcal{L}}{\partial u^{\alpha}_{ij}}
    +D_j D_k \frac{\partial \mathcal{L}}{\partial u^{\alpha}_{ijk}} - +D_j D_k D_l \frac{\partial \mathcal{L}}{\partial u^{\alpha}_{ijkl}}] \continue
    &+& D_j(W^{\alpha})[\frac{\partial \mathcal{L}}{\partial u^{\alpha}_{ij}}-D_k \frac{\partial \mathcal{L}}{\partial u^{\alpha}_{ijk}}
    +D_k D_l \frac{\partial \mathcal{L}}{\partial u^{\alpha}_{ijkl}}] \continue
    &+& D_j D_k (W^{\alpha})[\frac{\partial \mathcal{L}}{\partial u^{\alpha}_{ijk}}-D_l \frac{\partial \mathcal{L}}{\partial u^{\alpha}_{ijkl}}
+D_k D_j \frac{\partial \mathcal{L}}{\partial u^{\alpha}_{ikj}} - +D_k D_j D_l \frac{\partial \mathcal{L}}{\partial u^{\alpha}_{ikjl}}]\continue
&+& D_j D_k D_l (W^{\alpha})[\frac{\partial \mathcal{L}}{\partial u^{\alpha}_{ijkl}}]\,,
\eea
where the equation has been extended to include
all possible non-zero terms in the context of the \KSe, $W^{\alpha}$ is shorthand
for $\phi^{\alpha} + \xi^i u_i^{\alpha}$. Applying this to our generators
yields one unique conservation law which we shall now detail.

For the Galilean transformation generator $\mathbf{\text{pr}}\,v_3 = t \partial_x + \partial_u -u_x \partial_{u_t}$
the components equal
\bea
C^x &=& t*\mathcal{L} + W[\frac{\partial \mathcal{L}}{\partial u_x}-D_x\frac{\partial \mathcal{L}}{\partial u_{xx}}-D_x^3\frac{\partial \mathcal{L}}{\partial u_{xxxx}}] \continue
    &+& D_x(W)[\frac{\partial \mathcal{L}}{\partial u_{xx}}+D_x^2\frac{\partial \mathcal{L}}{\partial u_{xxxx}}]\continue
    &+& D_x^2(W)[-D_x\frac{\partial \mathcal{L}}{\partial u_{xxxx}}]\continue
    &+& D_x^3(W)[\frac{\partial \mathcal{L}}{\partial u_{xxxx}}]\continue
C^t &=& 0*\mathcal{L} + (1-t u_x)[\frac{\partial \mathcal{L}}{\partial u_t}]
\,.
\eea
Both expressions simplify to
    \PC{2019-09-11}{
not sure about $v_{t t}$ term...
    }
\MNG{2019-09-12}{Accidentally wrote down the divergence $D_t C^t$ instead of $C^t$}
\bea
C^x &=& t(u_tv+u_xv_x+u_xv_{xxx}+u_{xxx}v_x+u_{xx}v_{xx})+uv-v_x-v_{xxx}
    \continue
C^t &=& (1-t u_x)v
\,,
\eea
such that the conservation law is given by the divergence
\bea \label{e-conservedvector}
D_x (C^x) + D_t (C^t) &=& 0 \continue
                      &=& v_t - v u_x - u_x v_t t + uv_x +vu_x - v_{xx} - v_{xxxx}\ceq
                       +\,t(v_x(u_t+u_{xx}+u_{xxxx})+u_x(v_{xx}+v_{xxxx})+2D_x(u_{xx}v_{xx}))\continue
                      &=& t(v_x(u_t+u_{xx}+u_{xxxx})+u_x(-v_t+v_{xx}+v_{xxxx})+2D_x(u_{xx}v_{xx}))
                      \continue
                      &=& 2D_x(u_{xx}v_{xx})
\,.
\eea

This analysis is only relevant if there are non-trivial
solutions to the
adjoint equation \refeq{e-adjointeqn},
because otherwise one does not know how to evaluate
\refeq{e-conservedvector}. As we claimed previously,
the only solutions to \refeq{e-adjointeqn} are constant in nature;
implying that the conserved vector \refeq{e-conservedvector}
is unfortunately a trivial conservation law.
There are a number of reasons why we believe
this analysis fails to yield anything useful, the
most obvious being that the Lie algebra of infinitesimal
symmetries is too simple.
    \PC{2019-09-11}{
I am thinking of moving everything from \refeq{e-formallagrangian} to
%\refeq{e-conservedvector}
here into a Chapter of your thesis, remarking here briefly that Ibragimov
methods seem to have not worked for you?
    }
