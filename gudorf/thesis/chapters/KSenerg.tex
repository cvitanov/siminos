% siminos/gudorf/thesis/chapter/KSenerg.tex
% $Author: predrag $ $Date: 2020-05-25 15:18:45 -0400 (Mon, 25 May 2020) $

% called by
%           siminos/spatiotemp/chapter/spatiotemp.tex
%           siminos/tiles/GuBuCv17.tex

%\section{Energy transfer rates}
%\label{sect:KSenerg}
% Predrag copied from siminos/rpo_ks/current/           2019-05-15


In physical settings where the observation times are much
longer than the dynamical `turnover' and Lyapunov times
(statistical mechanics, quantum physics, turbulence) periodic
orbit theory\rf{DasBuch} provides highly accurate predictions
of measurable long-time averages such as the dissipation and
the turbulent drag\rf{GHCW07}. Physical predictions have to
be independent of a particular choice of ODE representation
of the PDE under consideration and, most importantly,
invariant under all symmetries of the dynamics. In this
section we discuss a set of such physical observables for the
1-$d$ \KS\ invariant under reflections and translations. They
offer a representation of dynamics in which the symmetries
are explicitly quotiented out.

\PublicPrivate{}{
\PCedit{ % 2019-05-18 to be merged
The time-dependent $L^2$ norm of $u$,
\beq
    \expctE(t)=
  \Lint{\pSpace} \frac{u^2}{2}
  \,,
  \label{SCD07ksEnergy}
\eeq
has a physical interpretation\rf{ksgreene88} as the average energy density.
The energy density is intrinsic to the flow, invariant under translations and
reflections, and independent of the particular ODE basis set chosen to
represent the PDE.
In the Fourier space the
energy density \refeq{SCD07ksEnergy} is a diagonalized quadratic norm,
\beq
\expctE(t)
          =  \sum_{k=-\infty}^{\infty} E_k(t)
\,,\qquad
E_k (t)=
    {\textstyle\frac{1}{2}}|\Fu_k(t)|^2
\,.
\ee{SCD07EFourier}
}
} %end \PublicPrivate

The {space average} of a function $\obser = \obser(\pSpace,t) = \obser(u(\conf,\zeit))$  on
the interval $\speriod{}$,
\beq
    \expct{\obser} = \Lint{\pSpace}\, \obser(\pSpace,t)
    \,,
    \label{rpo:spac_ave}
\eeq
is in general time dependent.
Its mean value is given by the {time average}
\beq
\timeAver{\obser}
    =
\lim_{t\rightarrow \infty} \frac{1}{t} \int_0^t \! d\tau \, \expct{\obser}
    =
\lim_{t\rightarrow \infty} \frac{1}{t} \int_0^t \!
    \Lint{\tau}  d\pSpace\, \obser(\pSpace,\tau)
    \,.
\label{rpo:tim_ave}
\eeq
The mean value of $\obser = \obser(u_\stagn) \equiv \obser_\stagn$ evaluated on $q$
\eqv\ or {\reqv} $u(\pSpace,t) = u_\stagn(\pSpace-ct)$ is
\beq
\timeAver{\obser}_\stagn = \expct{\obser}_\stagn = \obser_\stagn\,.
\label{rpo:u-eqv} \eeq Evaluation of the infinite time average
\refeq{rpo:tim_ave} on a function of a \po\ or \rpo\
$u_p(\pSpace,t)=u_p(\pSpace,t+\period{p})$ requires only a single
$\period{p}$ traversal,
\beq
  \timeAver{\obser}_p = \frac{1}{{\period{p}}}
    \int_0^{\period{p}} \! d\tau \, \expct{\obser}
\,.
\label{rpo:u-cyc}
\eeq

Equation \refeq{e-ks} can be written as
\beq
    u_t=- V_x
        \,,\qquad
    V(\conf,\zeit)={\textstyle\frac{1}{2}}u^2+u_{x} + u_{xxx}
    \,.
\ee{ksPotent}
If $u$ is `flame-front velocity' then \expctE, defined in
\refeq{eq:stdks}, can be interpreted as the mean energy
density. So, even though \KS\ is a phenomenological
small-amplitude equation, the time-dependent $L^2$ norm
of $u$,
\beq
    \expctE=
  \Lint{\pSpace}
  V(\conf,\zeit)=
  \Lint{\pSpace} \frac{u^2}{2}
  \,,
  \label{ksEnergy}
\eeq
has a physical interpretation\rf{ksgreene88} as the average `energy'
density of the flame front. This analogy to the mean kinetic energy
density for the Navier-Stokes motivates what follows.

The energy \refeq{ksEnergy} is intrinsic to the flow,
independent of the particular ODE basis set chosen to
represent the PDE. However, as the Fourier amplitudes are
eigenvectors of the translation operator, in the Fourier
space the energy is a diagonalized quadratic norm,
\beq
\expctE
          =  \sum_{k=-\infty}^{\infty} E_k
\,,\qquad
E_k =
    {\textstyle\frac{1}{2}}|a_k|^2
\,,
\ee{EFourier}
and explicitly invariant term by term
under translations
\refeq{eq:shiftFour}
and reflections \refeq{SCD07:KSparity}.

Take time derivative of the energy density \refeq{ksEnergy},
substitute \refeq{e-ks} and integrate by parts. Total derivatives vanish
by the spatial periodicity on the $\speriod{}$ domain:
\bea
   \dot{\expctE} &=&
     \expct{u_t \, u}
         = - \expct{\left({u^2}/{2} + u_{x} + u_{xxx}\right)_x u }
    \continue
    &=&
\expct{ u_x \, {u^2}/{2} + u_{x}{}^2 + u_x \, u_{xxx}}
    \,.
\label{rpo:ksErate}
\eea
The first term in \refeq{rpo:ksErate} vanishes by
integration by parts,
\(
3 \expct{ u_x \, u^2}= \expct{(u^3)_x} = 0
\,,
\)
and integrating the third term by parts yet again
one gets\rf{ksgreene88} that the energy variation
\beq
   \dot{\expctE} = P - D
                \,,\qquad
      P =  \expct{u_{x}{}^2}
                \,,\quad
      D =  \expct{u_{xx}{}^2}
\ee{EnRate}
balances the power $P$ pumped in by anti-diffusion $u_{xx}$
against the energy dissipation rate $D$
by hyper-viscosity $u_{xxxx}$
in the \KSe\ \refeq{e-ks}.

The time averaged energy density  $\timeAver{E}$
computed on a typical orbit goes to a constant, so
the expectation values \refeq{rpo:EtimAve} of drive and dissipation
exactly balance each out:
\beq
    \timeAver{\dot{E}}  =
    \lim_{t\rightarrow \infty}
        \frac{1}{t} \int_0^t d\tau \, \dot{\expctE}
=
      \timeAver{P} - \timeAver{D}
= 0
    \,.
\ee{rpo:EtimAve}
In particular, the \eqva\
and \reqva\ fall onto the diagonal in  \PCedit{reffig{f:drivedrag}},
    \PC{2019-12-06}{
    Reference `f:drivedrag' undefined
    }
and so do time averages computed on \po s and \rpo s:
\beq
\timeAver{E}_p =
\frac{1}{{\period{p}}} \int_0^{\period{p}}d\tau \, E(\tau)
    \,,\qquad
\timeAver{P}_p =
\frac{1}{{\period{p}}} \int_0^{\period{p}} d\tau \, P(\tau)
    =
      \timeAver{D}_p
    \,.
\label{poE}
\eeq
In the Fourier basis \refeq{EFourier} the conservation of energy on average
takes form
\beq
0 = \sum_{k=-\infty}^{\infty} ( q_k^2 - q_k^4 )\,
    \timeAver{E}_k
\,,\qquad
E_k(t) =  {\textstyle\frac{1}{2}} |a_k(t)|^2
\,.
\ee{EFourier1}
The large $k$ convergence of this series is insensitive to the
system size $\speriod{}$; $\timeAver{E_k}$ have to decrease much faster than
$q_k^{-4}$.
Deviation of $E_k$ from this bound for small $k$ determines the active modes.
For \eqva\ the $\speriod{}$-independent bound
    on $E$ is given by Michaelson\rf{Mks86}.
The best current bound\rf{GiacoOtto05,bronski2005} on the long-time limit
of $E$
as a function of the system size $\speriod{}$ scales as
$E \propto \speriod{}^2$.
