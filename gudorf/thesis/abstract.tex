% siminos/gudorf/thesis/chapter/abstract.tex
% $Author: predrag $ $Date: 2020-10-24 01:45:26 -0400 (Sat, 24 Oct 2020) $

% called by
%           siminos/spatiotemp/chapter/spatiotemp.tex
%           siminos/tiles/GuBuCv17.tex

Advances in experimental imaging, computational methods, and
dynamical systems theory reveal that the unstable recurrent flows
observed in moderate Reynolds number turbulent flows result from
close passes to unstable invariant solutions of Navier-Stokes
equations. In past decade hundreds of such solutions been computed
for a variety of flow geometries, always confined to small
computational domains (minimal cells). While the setting is classical,
such classical field-theory advances offer new semi-classical
approaches to quantum field theory and many-body problems.

The Gutkin and Osipov on many-particle quantum chaos
(in particular, the {\catlatt} lattice models) suggests a path to
determining such solutions on spatially infinite domains. Flows of
interest (pipe, channel flows) often come equipped with $D$
continuous spatial symmetries. If the theory is recast as a
$(D+1)$-dimensional space-time theory, the space-time
translationally recurrent invariant solutions are $(D+1)$-tori (and
not the $1$-dimensional periodic orbits of the traditional periodic
orbit theory). {\catLatt} lattice models suggest that symbolic
dynamics should likewise be $(D+1)$-dimensional (rather than a
single temporal string of symbols), and that the corresponding zeta
functions should be sums over tori, rather than $1$-dimensional
periodic orbits.
