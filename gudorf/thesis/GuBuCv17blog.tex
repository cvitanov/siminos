% siminos/spatiotemp/chapter/GuBuCv17blog.tex
% $Author: predrag $ $Date: 2020-10-24 01:45:26 -0400 (Sat, 24 Oct 2020) $

\section{Tiles' GuBuCv17 clippings and notes}
\label{sect:GuBuCv17blog}

Move good text not used in \refref{GuBuCv17} to this file, for possible
reuse later.

\begin{description}
\item[2016-11-05 Predrag]
A theory of turbulence that has done away with \emph{dynamics}?
We rest our case.

\item[2019-03-19 Predrag] Dropped this:
\\
In what follows
we shall state results of all calculations either in units of the
`dimensionless system size' $\tildeL$, or the system size $\speriod{} = 2 \pi
\tildeL$.

Due to the hyperviscous damping $u_{xxxx}$, long time solutions of \KSe\
are smooth, $a_k$ drop off fast with $k$, and truncations of
\refeq{SCD07:expan} to $16 \leq N \leq 128$ terms yield accurate
solutions for system sizes considered here (see
\refappe{sec:fourierRLD}).

For the case investigated here, the
\statesp\ representation dimension $d \sim 10^2$ is set by
requiring that the exact invariant solutions that we compute
are accurate to $\sim 10^{-5}$.

\end{description}

\subsection{GuBuCv17 to do's}
\label{sect:GuBuCv17ToDo}
Internal discussions of \refref{GuBuCv17} edits.

\begin{description}

\item[2019-03-17 Predrag to Matt]
My main problem in writing this up is that I see nothing in
the blog that formulates the variational methods that you use,
in a mathematically clear and presentable form.
Perhaps there is some text from\\
\texttt{siminos/gudorf/thesisProposal/proposal.tex}\\
that you can use to start writing up variational justification for
your numerical codes, section~\ref{sect:variational} {\em Variational methods}.

\item[2019-03-17 Predrag to Matt]
Please write up {\em tile extraction} and {\em glueing} in the style of a
SIADS article.


\item[2019-03-17 Predrag to Matt]
Should any of
Appendix \ref{sect:FTnormal}~{\em Fourier transform normalization factors}
be incorporated into {\bf GuBuCv17}\rf{GuBuCv17}?

\item[2019-04-10 Matt writing]
To begin \texttt{variational.tex} I included two equivalent formulations
of the variational problem; the first is written in a more concise manner
while the second is written in a more explicit manner. The longer of the
two is commented out. The more explicit description uses dummy variables
(Lagrange multipliers) which replace parameters $(\speriod{},\period{})$
as independent variables.

I'm including explanations of the numerical algorithms but I don't think
I should present them in their style for algorithms, because we didn't
invent them just applied them in a unique way. If desired I think the
easiest way of including them per SIADS style guide is to use the
algorithm package they suggest: \texttt{algpseudocode} and
\texttt{algorithmic} are the package names.

I feel conflicted as to whether to define the gradient matrix using a new
letter or the ``mathematician way". e.g. $A(x)$ or $DG(x)$. Also, I
started using $\mathbf{z}$ to represent {\statesp} vectors. I'm not a fan
of using $z$ but I don't want to confuse people by using $u$,$x$, etc.

I need to get better at writing or stop being OCD over how sentences are written.

\item[2019-04-16 Matt update]
In an effort to make the chapters and \texttt{GuBuCv17.tex} more modular,
I've split apart some of the chapters into smaller, more manageable pieces.
For example, \texttt{variational.tex} was covering too many topics to be
reflected by the file name and \texttt{numerics.tex} predominately covered discrete
lagrangian systems and Noether's theorem. The algorithms (matrix free adjoint descent,
matrix free GMRES and Gauss-Newton) have yet to be discussed in excruciating detail.
This is my fault, in hindsight I've done a poor job with recording what I do and
how I do it. I'm going to get better at this.

For the time being, until it is deemed unnecessary or unintelligent, I am going
to break the chapters into the files \texttt{adjointdescent.tex} and \texttt{iterativemethods.tex}.
I'm going to change the discourse so that instead of requiring the current order,
namely, \texttt{variational.tex}->\texttt{adjointdescent.tex}->\texttt{iterativemethods.tex}
the pieces will be written as to be independent of one another.

In order to get specific, I needed to include the \KSe\ written in the Fourier-Fourier basis; I put this in \texttt{sFb.tex}

\item[2019-04-17 MNG update]
Realized that in order to get specific with the numerical methods I need to
include both an exposition on the {\spt} Fourier modes  as well
as the matrix-free computations. The latter really stresses the improvements
over the finite-difference approximation of the Jacobian that requires
time integration ubiquitous in plane-couette and pipe numerics.
Expanding on \texttt{adjointdescent.tex} and \texttt{iterativemethods.tex}.
Again, the main stratagem is to make the separate \texttt{.tex} files
as independent as possible to avoid ``long distance references''.

\item[2019-04-18 MNG]
Heavy edits to \texttt{tiles.tex}
Added section on preconditioning \texttt{preconditioning.tex}
Formatting edits to \texttt{matrixfree.tex} can be ignored.
\\
Added details in \texttt{iterativemethods.tex} regarding GMRES
and SciPy wrapper for LAPACK solver GELSD

\item[2019-04-23 MNG]
Converting indices to abide by the conventions: physical space
indices $u(x_m,t_n)$, and \spt\ Fourier space indices
$\Fu_{kj}$.
\\{\bf 2018-05-09 PC} can do. Also, remember that
$u(x_m,\zeit_n)$ implies that everywhere the ordering is
$(\speriod{},\period{})$, and not $(\period{},\speriod{})$.

Luca Dieci asked (borderline pleaded) to abide by the
mathematics convention that $n$ is the index for discrete
time. I'm avoiding $\ell$ and $\tau_t$ due to the unnecessary
confusion with domain size $\speriod{}$ and period $\period{}$.
\\{\bf 2018-05-09 PC} Agreed. $\tau_t$ we usually control by macro
$\setminus zeit$, so currently
$\zeit_n$.

\item[2019-04-24 MNG]
Discussion of how I foresee paper(s) playing out in \texttt{blogMNG.tex}
by considering subject matter, narratives, and paper length.
Perhaps unsurprisingly I lean towards structuring a paper similar
to my thesis.

I'm unsure how to approach \spt\ symmetries in a practical manner.
Projection operators which produces symmetry invariant
subspaces are nice and complements the selection rules for different
symmetries nicely. Specifically it provides the reason for why
the selection rules exist and motivates the use of symmetry
constrained Fourier transforms. The only issue I have with this
is that the results of the formal derivation are not really used
beyong that.
I think this is likely a case of ``It-is-trivial-now-that-I-know-it' syndrome.
Perhaps it would be
sufficient to say that the selection rules constitute these subspaces
without the formalism?

\item[2019-04-29 MNG]
Rewrite of \texttt{KSsymm.tex} after double checking the
derivations.
Going to rewrite \texttt{sFb.tex}, I'm paying for the expedient
manner in which is was written; in other words just use a single
Fourier basis as opposed to a real basis and a complex basis, Matt.

\item[2019-04-30 MNG]
Rewrites to describe the \spt\ \KSe\ only in terms of \rv\
\Fcs\ for consistency. The index notation gets a little rough
but the pseudospectral form of the equation is nice enough.

Tried to find the most concise description of how I handle
\rpo s using mean velocity frame (time dependent rotation
transformation).

\item[2019-05-02 MNG]
Is it necessary to recap all of the results in \refsect{sect:KStimeInt}
in this paper? Other than the spatial integration calculation the results
described in \refrefs{DasBuch,SCD07}. I'm unsure how to connect the
{\spt} calculations to results pertaining to the dynamical system
formulation, e.g. temporal stability and energy budget.

Moved \texttt{SpatTempSymbDyb.tex} to after \texttt{tiles.tex} such that
it proceeds from finding tiles to using tiles.

The bulk of each section is complete; perhaps need to add some more
detail to \texttt{glue.tex} and \texttt{tiles.tex} but mostly need to
work on picking, producing, and inserting figures.

Going to list suggestions for figures at the top of each section in
commented text.

\item[2019-05-02 MNG]
Added tile figures: Extraction and converged results in \texttt{tiles.tex}.

Modifying scripts to produce figures of general numerical convergence
(initial condition to final converged \twot), produce figures
demonstrating step-by-step gluing for repeated gluing,
and produce figures for the ``frankenstein'' plots (combining tiles
to produce \twots). Basically just producing more figures.

\item[2019-05-11 PC]
moved Ibragimov to \texttt{gudorf/thesis/thesis.tex} until we find it useful.

\item[2019-05-13 MNG]
\begin{itemize}
%\item Reduced Whitespace on figures
\item Added spatial gluing figures
\item Added description of gluing procedure
\end{itemize}

\item[2019-05-13 PC]
Figures are looking great, and in my talks people seem to ``get'' tile
extraction and gluing, so they are very important. A few notes, before you
produce the next versions:
\begin{itemize}
%  \item
%never include labels like (a) in this example
%into a figure file - that is job for LaTeX and such labels can change
%from time to time, for the same figure.
%  \item
%If at all possible for your scrips: Never create a white border around a
%figure (see the example above),
%
%\fbox{\includegraphics[width=.4\textwidth]{ks_ux_largeL}}
%
%that wastes valuable real estate, always
%clip it tightly up to the first/last non-white pixel of the figure.
  \item
I think you should label all $u$ color bars in multiples of 1, or
or 0.5 if that is really needed, not different units in every plot.
  \item
Once you have improved a given figure,
% like the \texttt{MNG\_hook\_initial.png},
keep the same name rather than renaming it
(they are often shared between different articles, presentations and blogs)
\end{itemize}

\item[2019-07-05 PC] dropped from trawl.tex: ``
In both formulations there is no guarantee of convergence
but it is clearly better to take less time regardless of convergence.

In our formulation, convergence can not be guaranteed either, but the
resources committed to the initial guesses generation are negligible.
''

\bea \label{e-discretedefsOld}
\wavek &=& 2\pi\frac{k}{\speriod{}}\,,\qquad k=1,\cdots,M/2-1 \continue
\freqj &=& 2\pi\frac{j}{\period{}}\,,\qquad j=0,\cdots,N/2-1 \continue
\xm &=& \frac{m}{M}\speriod{}\,,\qquad m=0,\cdots,M-1 \continue
\tn &=& \frac{n}{N} \period{}\,,\qquad n=0,\cdots,N-1\,.
\eea

\item[2019-08-21 MNG]
Moved discussion of recurrence plots and multiple shooting from trawl.tex
to variational.tex

It seemed more coherent to first describe the disadvantages
of the IVP to motivate the variational problem. I'm going
to refer to what I do as ``solving a variational problem'' as
opposed to boundary value problem because it insinuates
(at least to me) that we're solving a Dirichlet BC in 1 + 1 dimensions
problem.

General narrative of variational.tex
\begin{itemize}
\item Exponential instability bad
\item Variational formulation good
\item How to solve variational problem (general description of optimization)
\item Losses from variational formulation (notion of dynamics, stability, bifurcation analysis).
\item How to recoup from these losses (adjoint sensitivity, Lagrangian, Hill's formula)
\end{itemize}

It's currently a hot mess.


\item[2019-09-20 MNG]
Input references to topological defects and motifs in complex networks. Renamed
the ``defect tile'' to the ``merger tile'' but also made the connection
that similar patterns in crystals are referred to as ``edge dislocations''.

Just clean up and rewriting \texttt{tiles.tex} mainly; it's almost in shape.

\item[2018-05-09 PC] Dropped:
The following definitions will be devoid of symmetry considerations
such that the equations represent the general case.

For $\tildeL<1$ the only \eqv\ of the
system is the globally attracting constant solution
$u(\conf,\zeit)=0$, denoted $\EQV{0}$ from now on. With increasing
system size $\speriod{}$ the system undergoes a series of
bifurcations. The resulting \eqva\ and \reqva\ are described
in the classical papers of Kevrekidis, Nicolaenko and
Scovel\rf{KNSks90}, and Greene and Kim\rf{ksgreene88},
among others. The relevant bifurcations up to the
system size investigated here are summarized in
\reffig{fig:SCD07ksBifDiag}: at $\tildeL=22/2\pi = 3.5014\cdots$,
the {\eqva} are the constant solution \EQV{0},
the  \eqv\ \EQV{1} called GLMRT by Greene and
Kim\rf{laquey74,ksgreene88},
the $2$- and $3$-cell states
\EQV{2} and \EQV{3}, and the pairs of \reqva\ \REQV{\pm}{1},
\REQV{\pm}{2}.
All \eqva\ are in the antisymmetric subspace $\bbU^+$, while
\EQV{2} is also invariant under $\Dn{2}$ and \EQV{3} under $\Dn{3}$.

Due to the translational invariance of {\KSe},
they form invariant circles
in the full \statesp.
In the $\bbU^+$ subspace considered here,
they correspond to $2n$ points, each shifted by $\speriod{}/2n$.
For a sufficiently small $\speriod{}$
the number of {\eqva} is small and
concentrated on the low wave-number end of the Fourier spectrum.

    %\PC{2018-05-04} {
dropped this: \Group, the group of actions $ g \in G $
    on a \statesp\ (reflections, translations, \etc) is a spatial symmetry of
    a given system if $g u_\zeit = F(g\,u)$.
%     u_\zeit + u_{\conf \conf} + u_{\conf \conf \conf \conf} + u u_\conf = 0
%    \refeq{e-ks}

An instructive example is offered by the dynamics for
the  $(\speriod{},\period{})=(22,\period{})$  system
that \refref{SCD07} specializes to.
The size of this
small system is $\sim 2.5$ mean wavelengths
($\tildeL/\sqrt{2}= 2.4758\ldots$),
and the competition between states with wavenumbers 2 and 3.

The two zero Lyapunov exponents are due to the time and
space translational symmetries of the \KSe.

For large system size, as the one shown in \reffig{f:ks_largeL}, it is
hard to imagine a scenario under which attractive periodic states (as
shown in \refref{FSTks86}, they do exist) would have significantly large
immediate basins of attraction.

\item[2019-10-17 MNG]: Merged symmetry discussions.
\texttt{KSsymmMNG1} was deleted because seems to be an old discussion predating
the \spt\ symmetry group discussion as it still mentions
equivariance. The focus should only be on invariance under symmetry operations,
as invariance gives rise to the the practical application
of the symemtry discussion which is constraints on the \spt\
\Fcs.
\texttt{KSsymmMNG} was deleted because it is just an older version
of \texttt{KSsymm}.
\texttt{KSsymmPC} uses different notation and says things better than I do
so I'll have to figure out how to merge it in.

\item[2019-10-25 PC]  dropped from \emph{variational.tex}:

Linear stability analysis has been used in bifurcation
analysis of describe the existence and bifurcations of solutions
as well as the geometry of state spaces corresponding to different
flows \refrefs{GHCW07,lanCvit07,W97}.

Commonly time variational integrators preserve symplectic structure

\item[2019-09-05 MNG] Dropped from \emph{variational.tex}:
{multishooting optimization of cost functional
because it doesn't jive with \spt\ methods (based on integration)}

{Adjoint sensitivity and Hill's formula sections
when I figure them out or they seem useful}:

Section on adjoint sensitivity
The \spt\ reformulation of a dynamical problem also
requires a reformulation of its linear stability
analysis.

Nevertheless, we still have the notions of tangent spaces and derivatives
so the natural replacement is the notion of sensitivity. In the context
of finite element (finite difference) representation, instead of computing
a derivative and transporting it around a periodic orbit, it instead
computes the derivative of the temporal average of the quantity with
respect to whichever parameter is desired\rf{Blonigan17,LaShMe18,Wang13}.
Because there is no transport,
one need not worry about the exponential instability present.
Essentially sensitivity is to stability as boundary value problem is
to initial value problem in this context. Because the \spt\ boundary
problem is defined on a compact domain on which the scalar field does
not diverge, dynamical observables are bounded; they do not experience
numerical overflow (underflow) associated with unstable (stable) manifolds.
%%%%%%%% Hill's formula.
\beq
S = \int_{\mathcal{M}} \mathcal{L}(u,v,u_x,v_x,u_t,v_t,u_xx,v_xx) dx dt
\eeq
such that the matrix of second variations, or Hessian, of this action functional
is defined as
\beq
H = \nabla \nabla^{\top} S
\eeq
such that the derivatives are taken with respect to the infinite dimensional
scalar fields $u,v,\dots,$ such that the Hessian matrix is infinite dimensional
prior to discretization of the scalar fields.
The resultant discrete Lagrangian system and subsequent Hessian should be
the Hessian of Hill's formula, I believe. If one is trying to derive
Hamilton's action principle as a result of discretization (\ie, finite differences)
as in \refref{KraMaj15} then one must take care to define \spt\ differentiation operators
in a manner consistent with an action principle. A large amount of
the derivation of the discrete action principle and discrete Noether's theorem
of\rf{KraMaj15} relates to using a finite element discretization in physical
space. I am unsure how these ideas extend to a Fourier basis; I currently
am assuming that as long as the differentiation operators, and hence the derivatives
(jet bundle) is properly defined then everything should work out.
When two total derivatives of the Lagrangian density are taken, one arrives
at the following matrix representation of the Hessian. Keep in mind that
we have ordered the variables in terms of the
order of the corresponding derivatives $(u,v,u_t,v_t,u_x,v_x,u_xx,v_xx)$.
\beq
\begin{bmatrix}
-v_x(t,x)/3 & u_x(t,x)/3 & 0 & -1/2 &v(t,x)/3 & -2u(t,x)/3 & 0 &0 \\
u_x(t,x)/3& 0 & 1/2 & 0 & u/3 & 0& 0& 0\\
0    &1/2 &0 &0 &0 &0 &0 &0 \\
-1/2 &0 &0 &0 &0 &0 &0 &0 \\
v(t,x)/3  &u(t,x)/3 &0 &0& 0& &-1 &0 &0 \\
-2u(t,x)/3 &0 &0 &0 &-1 &0 &0 &0 \\
0    &0  &0 &0 &0 &0 &0 &1 \\
0    &0  &0 &0 &0 &0 &1 &0 \\
\end{bmatrix} \,.
\ee{NotHessiantake2}
This is an infinite dimensional matrix, but upon discretization each block
will represent a diagonal matrix whose diagonal contains the scalar
field values of the corresponding spacetime coordinates. For instance,
$u_x/3 \equiv \frac{1}{3} u_x(x,t) \to \frac{1}{3} u_x(t_n, x_m)$. Because
each of the blocks are diagonal, \ie, $1 \equiv \mathcal{I}^{N*M}$, the
determinant expansion is long but not impossible to decipher. Note
the presence of the adjoint variables $v,v_x$. There is freedom in
the choice of what these variables should be, because they are non-physical.


\end{description}


\bigskip
Note to Predrag - send this paper to
