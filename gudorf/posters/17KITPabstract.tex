% siminos/gudorf/presentations/17KITPabstract.tex     pdflatex 17KITPabstract
% $Author: predrag $ $Date: 2016-12-16 11:31:34 -0500 (Fri, 16 Dec 2016) $

% Poster for the KITP, UC Santa Barbara  conference Jan 2017

\documentclass[11pt]{article}
\usepackage{calc}
\usepackage{color}
\usepackage{amsfonts}
\usepackage{latexsym}
\usepackage{placeins}
\ifx\pdftexversion\undefined
  \usepackage[dvips]{graphicx}
\else
  \usepackage[pdftex]{graphicx}
\fi
\usepackage{amssymb}
\usepackage{authblk}
\usepackage{amsmath}
\usepackage[cp1250]{inputenc}
\usepackage[OT4]{fontenc}

\addtolength{\voffset}{-3.5cm} \addtolength{\textheight}{4cm}

\renewcommand\Authfont{\scshape\small}
\renewcommand\Affilfont{\itshape\small}
\setlength{\affilsep}{1em}

\newcommand{\smalllineskip}{\baselineskip=15pt}
\newcommand{\keywords}[1]{{\footnotesize\hspace{0.68cm}{\textit{Keywords}: }#1\par
  \vskip.7\baselineskip}}
\renewenvironment{abstract}[0]{\small\rm
        \begin{center}ABSTRACT
        \\ \vspace{8pt}
        \begin{minipage}{5.2in}\smalllineskip
        \hspace{1pc}}{\end{minipage}\end{center}\vspace{-1pt}}
\newcommand{\emailaddress}[1]{\newline{\sf#1}}

\let\LaTeXtitle\title
\renewcommand{\title}[1]{\LaTeXtitle{\large\textsf{\textbf{#1}}}}

%%%TITLE
\title{Tiling space-time: Spatiotemporal dynamics of the Kuramoto-Sivashinsky
	   equation}
\date{}

%%AFFILIATIONS
\author[1]{Matthew Gudorf}
\author[1]{Predrag Cvitanovi\'{c}}
\affil[1]{School of Physics and Center for Nonlinear Science,
		  Georgia Inst. of Technology,
		  Atlanta, GA  30332, USA \emailaddress{matthew.gudorf@gatech.edu}}


%%DOCUMENT
\begin{document}
\maketitle

%%PLEASE PUT YOUR ABSTRACT HERE
\begin{abstract}
Insights into dynamics of the 1D Kuramoto-Sivashinsky equation are often
helpful in developing intuition about turbulence in physical, 3D
Navier-Stokes fluid flows~\cite{Holmes96}. In both settings there is by now
strong evidence that the turbulent dynamics are shaped and organized by
invariant solutions (equilibria, traveling waves, periodic orbits) and their
invariant manifolds~\cite{science04}. Among them, relative periodic orbits
computed on spatially periodic domains, periodic in both in space and time,
play important role in shaping structures that evolve in time while drifting
in space~\cite{SCD07}. One can view these invariant solutions as space-time
rectangles that tile the spatiotemporal state space, infinite in both the
time and the spatial directions. To demonstrate that in turbulence the space
and time evolution can be thought of on the same footing, we evolve here in
configuration space the Kuramoto-Sivashinsky states initiated on temporally
periodic initial conditions. It turns out that such solutions are highly
unstable, so new robust variational methods~\cite{lanVar1} will need to be
developed in order to systematically locate and compute orbits (invariant
2-tori) doubly-periodic in both space and time.

\end{abstract}
%%THE END OF ABSTRACT
% \newpage

\begin{thebibliography}{99}
\small
\bibitem{Holmes96} P. Holmes, J.L. Lumley and G. Berkooz (1996)
Turbulence, Coherent Structures, Dynamical Systems and Symmetry,
Cambridge Univ. Press.

\bibitem{science04}
B. Hof {\em et al.} (2004)
Experimental observation of nonlinear traveling waves in turbulent pipe flow,
 Science 305, 1594-1598

%
%\bibitem{Christiansen97}
%F. Christiansen, P.Cvitanovi\'c and V. Putkaradze (1997)
%Spatiotemporal chaos in terms of unstable recurrent patterns,
%Nonlinearity, 10, 55-70

\bibitem{SCD07}
  P. Cvitanovi{\'c},  R.L. Davidchack and E. Siminos (2014)
  On the state space geometry of the Kuramoto-Sivashinsky flow in a periodic domain,
  SIAM J. Appl. Dyn. Syst. 9, 1-33.

\bibitem{lanVar1}
 Y. Lan and P. Cvitanovi\'c (2004)
 Variational method for finding periodic orbits in a general flow,
 Phys. Rev. E 69, 016217
\end{thebibliography}
\end{document}
