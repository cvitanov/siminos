% siminos/gudorf/thesisProposal/abstract.tex
% $Author: predrag $ $Date: 2019-01-14 17:05:18 -0500 (Mon, 14 Jan 2019) $

Turbulence is one of the oldest unsolved problems of classical
physics. The study of turbulence is intimately
tied to the study of chaotic partial differential equations, a discipline which
has made significant advances as computing power has become more readily available.
A number of these advances are the result of combining numerical simulations with older
theories; for instance, in the study of fluid dynamics, numerical
simulations paired with periodic orbit theory and dynamical systems theory
have over the past two decades
contributed greatly to the understanding of turbulent and chaotic behavior.
% \rf{GHCW07,W02,VC08,WFSBC15}.
The next logical step of extending these methods to systems with larger physical scales has proved to be
very challenging. To address these challenges, this work offers a new \spt\ formulation
to provide a new perspective which can possibly circumvent these difficulties.
The main goal of this reformulation is to provide a qualitative and
quantitative description of the infinite space-time behavior
of the \KSe{}.
\PC{2019-01-14}{basically, the rule is never any references in the abstract}
