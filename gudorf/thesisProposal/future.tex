% siminos/gudorf/thesisProposal/proposal.tex
% $Author: predrag $ $Date: 2019-01-14 17:05:18 -0500 (Mon, 14 Jan 2019) $
\newpage
\section{Future work}
\label{section:future}


\subsection{Spatiotemporal symbolic dynamics}
\label{subsection:symbolic}
The next major step of this project is to complete the alphabet and grammar of the \spt\ symbolic dynamics of the \KSe\,, if possible.
In other words, the goal is to discover all continuous families of tiles as well as the rules that govern the admissibility of tile combinations.
Finding a minimal set of tiles seems possible; there is a lower bound on domain size $\tilde{L} = \frac{L}{2\pi} = 1$ below which the laminar solution $u=0$ is a global attractor\rf{lanCvit07}.
As $L$ increases, unstable solutions are born from sequences of bifurcations\rf{KNSks90,AGHks89}. It is unknown whether there exists
an upper bound for the spatial domain size of tile solutions, but it does not seem unreasonable to conjecture that patterns start
to repeat after the domain size exceeds the largest spatial correlation length.
The less reasonable goal seems to be enumerating the grammar of the symbolic dynamics that determines admissible patterns.
All hope is not lost, however, as the grammar might be able to be procedurally generated with a complete set of tiles.
The procedure would consist of attempting to
numerically converge initial conditions built from all symbolic combinations of tiles.
If the combination is not found then the corresponding \spt\ symbol shape would be inadmissible.
If every \twot\ can be decomposed into a collection of tiles then quantitative behavior on infinite \spt\ domains
would only depend on the frequency of \spt\ tiles as opposed to the frequency of arbitrarily large
\spt\ \twot\ solutions.

\subsection{Quasi 2D Kolmogorov flow}
In an effort to increase the impact of the techniques developed here we intend to apply the same numerical
methods on a more computationally challenging system; the quasi (depth averaged) 2-D Kolmogorov flow with periodic boundary conditions
\rf{ChanKers13}. This work will be performed concurrently with the \KSe\ research. Essentially,  the generalization of the \spt\ formulation
will be tested. This is an intermediate step between the \spt\ \KSe\ and three dimensional fluid flows. It is currently unknown whether
it will be computationally viable without first finding even better numerical methods because of the even greater number of computational
variables.
