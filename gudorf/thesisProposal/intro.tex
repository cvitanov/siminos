% siminos/gudorf/thesisProposal/intro.tex
% $Author: predrag $ $Date: 2019-01-14 17:05:18 -0500 (Mon, 14 Jan 2019) $
\newpage
\section{Introduction}
\label{section:Introduction}

\subsection{Literature review}
\label{subsection:lit_review}
Nonlinear dynamics, chaotic partial differential equations,
and other related disciplines greatly benefit from accessability to computing power.
Solutions almost always need to be determined numerically, as the equations usually
lack analytic solutions.

Some great advances were made by combining custom numerical codes
with the ideas from periodic orbit theory and dynamical systems theory.
Two commonly used codes for incompressible shear flows, \texttt{openpipeflow}\rf{openpipeflow} and
\texttt{Channelflow}\rf{channelflow}. These libraries are used by many to study incompressible
shear flows with either cylindrical pipe or plane-Couette geometry.

With these codes, periodic orbits and
equilibria of the respective equations can be found numerically\rf{W01,WK04}.
The number and impact of the studies that utilize these computational codes
is impressive to say the least; however, their returns are diminishing
over time and the next step forward is not obvious. This difficulty and ambiguity
is the motivating force behind this body of work.

This study will be dedicated to the \KSe{}, a partial differential equation (PDE)
that describes reaction-diffusion phenomena.
A common interpretation is that it models the velocity field
of a circular laminar flame front, e.g. the flames of a bunsen burner.
Previous studies of the \KSe{} have covered a wide range of topics, including
but not limited to: the geometry of the \KSe\,'s
state-space\rf{SCD07}, its unstable recurrent patterns\rf{lanCvit07},
the dimension of an inertial manifold\rf{DCTSCD14},
the existence and bifurcations of steady solutions\rf{Mks86,DoLa14},
families of solutions and bifurcation analysis\rf{KNSks90}, and
continuous symmetry reduction\rf{BudCvi14}.

Spatiotemporal methods are not a new idea, they have been used in studies of
the complex Ginzburg-Landau equation\rf{lop05rel,Lopez2015},
binary fluid convection\rf{KnoMoor90}, and plane-Poiseuille flow\rf{SoiMei91}.
A \spt\ approach has also been stated as a possible approach by Brown and
Kevrekidis\rf{BrKevr96}.
In addition to these examples, Wang \etal\rf{WGBGQ13} not only use a spatiotemporal formulation
to solve equations using Lagrange multipliers, but also claim that their
method outperforms other conventional methods in terms of scalability.

The main purpose of this work is to develop a spatiotemporal theory which can discover
quantitative information of the \KSe\ on an infinite spatiotemporal domain. The aforementioned
studies\rf{lop05rel,Lopez2015,KnoMoor90,SoiMei91,BrKevr96} use spatiotemporal methods to find
solutions, but revert back to conventional methods of analysis after solutions
are found. The goal of this work is to develop a purely spatiotemporal theory with new and unique
numerical methods and tools of analysis.

\newpage
\subsection{Problem statement}
\label{subsection:summary}
The nondimensionalized form of the spatiotemporal \KSe\ is defined by
\beq
u_t(x,t) + u_{\conf \conf}(x,t) + u_{\conf \conf \conf \conf}(x,t)
         + \frac{1}{2}(u(x,t)^2)_{\conf} = 0
\,,
\ee{KS}
where $u(x,t)$ is a scalar field defined on a two dimensional spatiotemporal domain. The general
behavior of this PDE can be crudely described by the following process: long-wavelength instabilities
pump energy into the system, which is transferred to short-wavelength scales via nonlinear
coupling. The energy is then dissipated by the ``hyper''-diffusion (fourth derivative) term.


This body of work will begin with a proper definition of the ``spatiotemporal formulation''.
The next step is to develop methods for finding \spt\ solutions and then generate a collection of
said solutions.
Once the solution space is believed to be adequately sampled, the search for
a set of small \spt\ solutions (``tiles'') begins. These tiles will serve as the
building blocks for a two-dimensional \spt\ symbolic dynamics.
After a sufficient set of tiles is found, the set of rules that determine the admissibility
of solutions must be generated. The success of the \spt\ formulation can and should be
measured on both computational and theoretical grounds, but the metrics for success have
yet to be determined.
