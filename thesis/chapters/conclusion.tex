%  conclusion.tex
% $Author: predrag $ $Date: 2009-08-20 12:41:28 -0400 (Thu, 20 Aug 2009) $

\section{\Statesp\ geometry of spatially extended systems}


This thesis contribution to the dynamical system's approach
to spatially extended systems is to provide a framework for
elucidating state space geometry in the presence of
continuous symmetries. The presence of symmetry enriches
\statesp\ structure and profoundly influences dynamical
behavior. A a striking example is provided by the robust
homoclinic (or heteroclinic) connections in \KS\ flow
(discussed in \refchap{chap:kseStSp}) that provide a
recurrence mechanism by connecting neighborhoods of saddles
along a homoclinic (or heteroclinic) loop and organizing a
group of {\rpo s} around them.

The  \statesp\ structure remains unclear until points related
by continuous symmetry are identified and the dynamics is visualized in
reduced \statesp. Once this reduction procedure was carried out for \KS\ flow
we were able to identify (in \refchap{chap:kseRedStSp}) the
``stretching and folding'' of the unstable manifold of a \reqv\ as
the mechanism responsible for organizing a different group of \rpo s. Moreover
we were able to demonstrate that \rpo s  follow the unstable manifold of \REQV{\pm}{1}
for a while until carried over to the unstable manifold of \EQV{2} therefore
revealing the interplay of unstable manifolds of different objects, living
in subspaces with different symmetry, in shaping the geometry of the attractor.

The understanding of the geometry of \KSe\ for $L=22$ is by
no means complete. The obvious next step is to identify
suitable Poincar\'e sections for the study of unstable
manifolds and the \rpo s clustered around them. Contrary to
the \CLe\ example of \refchap{chap:lasers} where a global
section was found and the dynamics was described as a first
return map to the section, in the case of \KS\ equation we
will need more than one sections. Each section will be used
to capture the dynamics of the unstable manifold of a
(relative) equilibrium until it starts folding back to
itself. Parameterizing the intersection of a manifold with
the \Poincare\ section by Euclidean length along it, a
forward map from section to section will be constructed and
convolution of those maps will result in a return map. This
approach meshes very well with the construction of a Markov
partition of the dynamics, if such a partition is within
reach. A potential obstacle is that unstable manifolds of
objects of interest for \KS\ dynamics are often
high-dimensional, \eg $4$-dimensional for \REQV{\pm}{1}, and
their visualization and parametrization is a non trivial
task. Nevertheless, since the separation of the leading
eigenvalues is large, we expect that the continuation of the
strongly unstable eigenspace will play the dominant role.
Furthermore, we still need to investigate the role played by
trajectories originating in the $\eigExp[1,2]$ eigenspace of
\EQV{1} that are not in the antisymmetric subspace and are
therefore expected to play a role in organizing the
relative periodic orbits.


\section{Symmetry reduction}

For this geometric understanding to be possible we had to develop a
a symmetry reduction procedure for our specific needs
and with the following
constraints in mind:
1) the method must work efficiently in high-dimensional \statesp, 2) reduction
can be local but the local pieces have to be joined together in
a way that the global geometry of the attractor is elucidated.

For visualization purposes the method of moving frames is efficient in providing
symbolic expressions for invariants up to moderate dimension. When the
representation of the symmetry group is a direct sum of irreducible representations,
as usually is the case with discretizations of PDEs, we can define a moving
frame in one irreducible subspace and construct invariants for the rest of
the irreducible subspaces, as necessary for visualization.
The invariants thus obtained are singular but the singularities can be removed,
or merely moved away from regions of dynamical interest.
Then solutions computed in the equivariant variables can be visualized
in the invariant basis without any discontinuities introduced.

This leads us to the next step, which is reduction using the
geometrical interpretation of moving frames as a group
operation that brings points back to a local {\csection} of
group orbits. This is a linear operation for any given point
and can be implemented efficiently even for high-dimensional
discretizations of PDEs. The crucial step is to avoid
transformation singularities by restricting attention to
local, group-invariant Poincar\'e sections that do not
contain any points on which the transformations become
singular.

As noted in the introduction, a method of symmetry reduction
for PDEs has been presented by Rowley and
Marsden\rf{rowley_reconstruction_2000}, that allows one to
integrate a PDE defined in the reduced space along with a
reconstruction equation to recover the dynamics of the
original problem. This procedure identifies the reduced space
$\Manif/\Group$ with a subset of $\Manif$, called a slice, in
the same spirit we identified the reduced space with a
cross-section.
    \PC{rewrite this sentence}
The reconstruction equation is guaranteed to
work locally, in the neighborhood of the initial condition
but can fail globally. In \refref{rowley_reconstruction_2000}
choosing a new slice is proposed as a method to overcome this
difficulty and the different slices are to be treated as
local coordinate charts on $\Manif/\Group$. Yet, this can
obscure the study of global aspects of dynamics. It will be
interesting to investigate how this difficulty is connected
to the singularities present in the moving frame method and
whether the insight gained here can help one avoid
singularities while still identifying the reduced space with
a single slice.


    \PC{
perhaps mention as future generalizations: invariant tori -
``larger'' symmetries?
    }
