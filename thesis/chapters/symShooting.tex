% section: Numerically locating relative equilibria and periodic orbits

\subsection{Multipoint shooting for \rpo s}
\label{sec:symShoot}

Assume that we have an initial guess for a \rpo\ of period
$T_p$ and phase shift $\phi_p$.
Let the guess be given as $N$ initial conditions $x_i,\, i=1\ldots N$
for each segment of the \rpo, along with the flight times $T_i$, such that $\sum_i^N T_i = T_p$, and the
phase shift $\phi_p$. For the true \rpo\ we have
\bea
	f^{\tilde{T}_i}(\tilde{x}_i) & = & \tilde{x}_{i+1}\,,\  i=1,\ldots N-1,\continue
	R(\tilde{\phi}_p)f^{\tilde{T}_N}(\tilde{x}_N) & = & \tilde{x}_{1}\,.
	\label{eq:rpoCond0}
\eea

Assuming that our guess is in the linear neighborh of the
\rpo\ we can Taylor expand \refeq{eq:rpoCond0} around the guess
to linear order in the small quantities $\Delta
x_i=\tilde{x}_i-x_i,\, \Delta T_i=\tilde{T}_i-T_i,\, \Delta
\phi_p=\tilde{\phi}_p-\phi_p$ to get
\scriptsize
\bea
	J^{T_i}(x_i)\Delta x_i + v_{T_i}\Delta T_i -\Delta x_{i+1}& = & x_{i+1}-f^{T_i}(x_i)\continue
	R(\phi_p)J^{T_N}(x_N)\Delta x_N + R(\phi_p)v_{T_N}\Delta T_N +\Lg R(\phi_p)f^{T_N}(x_N)\Delta \phi -\Delta x_1 & = & x_{1}-R(\phi_p)f^{T_N}(x_N)\,,
	\label{eq:rpoCond}
\eea
\normalsize
where $v_{T_i}$ denotes $v$ evaluated at ${f^{T_i}\left(x_i\right)}$, $\Lg$ denotes the generator of the group and the matrix 
\beq
	J^{t}(x_i)=\frac{\partial f^t(x_i)}{\partial x}
\eeq
is evaluated by integrating, along with the flow, the equation
\beq
    	\dot{J}^t=A J^t \, ,
 	\label{eq:Jdef}
\eeq
with initial condition $J^0=\mathbf{1}$.
Interpreting $\Delta x_i,\, \Delta T_i,\, \Delta \phi_p$ as corrections to our guess solution we iteratively improve our approximation
of $\tilde{x}_p$.
To exclude variations along the two unit eigendirections of $R(\tilde{\phi}_p)J^{\tilde{T}_p}(\tilde{x})$ 
we impose the conditions
\bea
	v(x_i)\cdot\Delta x_i  &=& 0\,, \label{eq:transpV}\\
	\left(\Lg x_N\right) \cdot \Delta x_N &=& 0\,. \label{eq:transpLie}
\eea

Conditions \refeq{eq:transpV} ensures that the correction will be transverse to the eigendirection associated
with time translational invariance, while condition \refeq{eq:transpV} prohibits correction along the direction
of infinitesimal group action. In matrix form we have the system
\scriptsize
\begin{eqnarray*}
  \lefteqn{   \left( \begin{array}{ccccc|ccccc|c}
        \mathbf{J}^{T_1}(x_1) 	&\hspace{-4pt}-\mathbf{1}	& 					&			& 						&v_{T_1}	&			&			&			&			&\\
				&\hspace{-4pt}\ddots	&\hspace{-4pt}-\mathbf{1}			&			& 						&		&\hspace{-10pt}\ddots	&			&			&			&\\	
				&			&\hspace{-4pt}\mathbf{J}^{T_i}(x_i)	& \hspace{-4pt}-\mathbf{1}	& 						& 		&			& \hspace{-10pt}v_{T_i}	&			&			&\\
				&			&					&\hspace{-4pt}\ddots	&\hspace{-4pt}-\mathbf{1}				&		&			&			&\hspace{-10pt}\ddots	&			&\\
			-\mathbf{1}	&			&					&			&\hspace{-4pt}R(\phi_p)\mathbf{J}^{T_N}(x_N)	&		&			&			&			&\hspace{-10pt}R(\phi_p)v_{T_N}	&\hspace{-4pt}\Lg R(\phi_p)f^{T_N}(x_N)\\ \hline
	v(x_1)			&			&					&			&						&		&			&			&			&			&\\
				&\hspace{-4pt}\ddots	&					&			&						&		&			&			&			&			&\\
				&			&\hspace{-4pt}v(x_i)			&			&						&		&			&			&			&			&\\
				&			&					&\hspace{-4pt}\ddots	&						&		&			&			&			&			&\\
				&			&					&			&	\hspace{-4pt}v(x_N)			&		&			&			&			&			&\\ \hline
				&			&					&			&	\hspace{-4pt}\Lg x_N 			& 		&			&			&			&			&
     \end{array}\right)
     \left(\begin{array}{c}
        \Delta x_1 \\
	\vdots\\
	\Delta x_i \\
	\vdots\\
	\Delta x_N \\
        \Delta T_1 \\
	\vdots	\\
	\Delta T_i \\
	\vdots	\\
	\Delta T_N \\	
	\Delta \phi
     \end{array}\right)}\\%end lefteqn
     & & 
     \hspace{300pt}=\left(\begin{array}{c}
	x_2-f^{T_1}(x_1)\\
	\vdots\\	
        x_{i+1}-f^{T_i}(x_i) \\
	\vdots\\
	x_{1}-R(\phi_p)f^{T_N}(x_N)\\
       	0    \\
	\\
	\\
	\vdots\\
	\\
	\\
	0
     \end{array}\right)\,.
     \label{eq:NewtonScheme}
\end{eqnarray*}
\normalsize
Solving this linear system and iterating the procedure we refine the initial guess to desired accuracy,
provided the initial guess is sufficiently close to the true solution. If we are in the linear neighborh
of the solution we are guarranted to find it. In practice we can increase the chances of convergence 
by taking more segments along the orbit. The limit of $N\rightarrow\infty$ leads one to consider
variational methods, see for example \refrefs{bl,lanVar1}.

By setting $N=1$ in \refeq{eq:rpoCond} we observe that we have to interprete $R(\phi_p)J^{T_p}(x_p)$
as the fundamental matrix for relative periodic orbits, with eigenvalues playing the role
of Floquet multipliers.

