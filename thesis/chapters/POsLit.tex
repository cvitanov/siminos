%from ChaosBook.org \Chapter{flows}{27aug2008}{Go with the flow}
% $Author: predrag $ $Date: 2009-03-03 01:03:03 -0500 (Tue, 03 Mar 2009) $

% \section{Literature survey}

\paragraph{Stability ordering:}
The stability ordering was introduced by
Dahlqvist and Russberg\rf{DR91} in a study of chaotic dynamics for the
$(x^2y^2)^{1/a}$ potential.
%in the quantum case.
The presentation here runs along the lines of
Dettmann and Morriss\rf{DM97} for the Lorentz gas
which is hyperbolic but the symbolic dynamics is highly pruned, and
Dettmann and Cvitanovi\'c\rf{DC98} for a family of intermittent maps.
In
%all of
the
%above
applications discussed in the above papers, the stability ordering
yields a considerable improvement over the topological length
ordering. In quantum chaos applications cycle expansion cancelations
are affected by the phases of pseudocycles (their actions), hence
{\em period ordering} rather than stability is frequently employed.


\paragraph{Surveys of rigorous theory:}\label{rem:rig-z}
We recommend the references listed in \refrem{s-GuideLit}
for an introduction to
the mathematical literature on this subject.
For a physicist, Driebe's monograph\rf{Driebe99}
might be the most accessible
introduction into mathematics discussed briefly in this chapter.
There are a number of reviews of the mathematical approach to
\dzeta s and \Fd s, with pointers to the original references,
such as \refrefs{Viv-1,Poll-1}. An alternative approach to spectral
properties of the \FPoper\ is given in \refref{Via-1}.

Ergodic theory, as presented by Sinai\rf{er-sinai} and others,
tempts one to describe the densities on which the \evOper\ acts
in terms of either
integrable or square-integrable functions. For our
purposes, as we have already seen,
this space is not suitable.
An introduction to ergodic theory is given by
Sinai, Kornfeld and Fomin\rf{Korn}; more advanced
old-fashioned  presentations are Walters\rf{Walt82}
and Denker, Grillenberger and Sigmund\rf{DGS76}; and a more formal
one is given by Peterson\rf{Peterson}.
        %\MAP{  Warwick Tucker
W.~Tucker\rf{tucker1,tucker2,MIViana} has proven rigorously via interval
     arithmetic that the Lorentz attractor is strange for
     the original parameters, and has a long stable periodic orbit for
     the slightly different parameters.
