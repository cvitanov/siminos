% Discussion of applications for the work, speculation on the future
% $Author: siminos $ - $Date: 2009-04-06 09:18:32 -0400 (Mon, 06 Apr 2009) $ - $Id: loma.tex 846 2008-08-13 19:40:49Z predrag $




\section{What are the cycles good for?}

Up to this point we have concentrated in the geometry of the
\statesp\ and haven't addressed the second constituent of the
dynamicist's view of turbulence, the natural measure. The
periodic and relative periodic orbits found for \KS\ equation
form a skeleton of the dynamics in a geometrical sense but
also, through trace formulas and spectral
determinants\rf{DasBuch}, provide a means to accurately
evaluate the spectra of evolution operators and evaluate the
asymptotic values of observables.

Quoting \refref{DasBuch}, ``the strategy is 1) count, 2) weigh,
3) add up.'' The weights are given by the stability of the
cycles, we can use trace formulas to add them up (in our case
the continuous symmetry version in \refref{Cvi07}) but we have to
start from the beginning and complete step number one. Counting
means that we are able to organize and label all cycles up to a
given length, establishing a hierarchy that will then be
exploited in highly convergent trace formulas or spectral
determinants. The need to organize the periodic and relative
periodic orbits found for \KS\ equation is underlined in
\refsect{s-StOrdKS}, where we try to utilize a set 
of $20,000$ periodic and {\rpo s} computed by Davidchack\rf{Davidchack_priv},
by means of periodic orbit theory, but without any understanding
of their organization.

%% To stress out the need to organize the cycles we present our
%attempt to use an ordering criterion  that has been successfully
%used in calculations with periodic orbits, namely stability
%ordering. Stability ordering was introduced by Dahlqvist and
%Russberg\rf{DR91} in a study of chaotic dynamics for the
%$(x^2y^2)^{1/a}$ potential.

% \section{Invariant tori}
% 
% A natural question is what remains of this 
