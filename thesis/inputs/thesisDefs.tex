% thesisDefs.tex
% Macros and such that are specific to the thesis
% $Author: jhalcrow $ $Date: 2007-09-19 16:30:51 -0400 (Wed, 19 Sep 2007) $


%%%%%%%%%%%%%%%%%%%%%% Weblinks in PDF %%%%%%%%%%%%%%%%%%%
\newcommand{\weblink}[1]{\href{http://#1}{{\tt #1}}}
\newcommand{\arXiv}[1]{\href{http://arXiv.org/abs/#1}{{\tt #1}}}

%%%%%%%%%%%%%%%%%%%%%% COMMENTS %%%%%%%%%%%%%%%%%%%
 
% \renewcommand{\topfraction}{1}
% \renewcommand{\bottomfraction}{1}
% \renewcommand{\textfraction}{0}

\newcommand{\rf}     [1] {~\cite{#1}}
\newcommand{\refref} [1] {ref.~\cite{#1}}
\newcommand{\refRef} [1] {Ref.~\cite{#1}}
\newcommand{\refrefs}[1] {refs.~\cite{#1}}
\newcommand{\refRefs}[1] {Refs.~\cite{#1}}
\newcommand{\refeq}  [1] {(\ref{#1})}
\newcommand{\refeqs} [2]{(\ref{#1}--\ref{#2})}
\newcommand{\eqn}[1]{eqn.\ {\ref{#1}}}
\newcommand{\Eqn}[1]{Eqn.\ {\ref{#1}}}
\newcommand{\refpage}[1] {p.~\pageref{#1}}
\newcommand{\reffig} [1] {figure~\ref{#1}}
\newcommand{\reffigs} [2] {figures~\ref{#1} and~\ref{#2}}
\newcommand{\refFig} [1] {Figure~\ref{#1}}
\newcommand{\refFigs} [2] {Figures~\ref{#1} and~\ref{#2}}
\newcommand{\reftab} [1] {table~\ref{#1}}
\newcommand{\refTab} [1] {Table~\ref{#1}}
\newcommand{\reftabs}[2] {tables~\ref{#1} and~\ref{#2}}
\newcommand{\refsect}[1] {\S\,\ref{#1}}
\newcommand{\refsects}[2] {\S\,\ref{#1} and \S\,\ref{#2}}
\newcommand{\refSect}[1] {\S\,\ref{#1}}
\newcommand{\refappe}[1] {appendix~\ref{#1}}
\newcommand{\refappes}[2] {appendices~\ref{#1} and \ref{#2}}
\newcommand{\refAppe}[1] {Appendix~\ref{#1}}
\newcommand{\refexam}[1] {example~\ref{#1}}
\newcommand{\refExam}[1] {Example~\ref{#1}}


%%%%%%%%%%%%%%% EQUATIONS %%%%%%%%%%%%%%%%%%%%%%%%%%%%%%%
\newcommand{\beq}{\begin{equation}}
\newcommand{\continue}{\nonumber \\ }
\newcommand{\nnu}{\nonumber}
\newcommand{\eeq}{\end{equation}}
\newcommand{\ee}[1] {\label{#1} \end{equation}}
\newcommand{\ceq}{\nonumber \\ & & }
\newcommand{\bea}{\begin{eqnarray}}
\newcommand{\eea}{\end{eqnarray}}
\newcommand{\barr}{\begin{array}}
\newcommand{\earr}{\end{array}}
% Unfortunately these don't work. Latex doesn't recognize the \end{align}.
%\newcommand{\bal}{\begin{align}}
%\newcommand{\eal}{\end{align}}

%%%%%%%%%%%%    PREDRAG'S FAVORITE MACROS %%%%%%%%%%%%%

\def\ie{{\em i.e.\ }}
\newcommand{\NS}{Navier-Stokes}
\newcommand{\NSe}{Navier-Stokes equation}
\newcommand{\KS}{Kuramoto-Sivashinsky}
\newcommand{\KSe}{Kuramoto-Sivashinsky equation}
% \newcommand{\KS}{KS}
% \newcommand{\KSe}{KS equation}
\newcommand{\Reynolds}{\textit{Re}}  % Reynolds number
% \newcommand\Rey{\mbox{\textit{Re}}}  % Reynolds number ELIMINATE?
\newcommand{\pCf}{plane Couette flow}
\newcommand{\PCf}{Plane Couette flow}
\newcommand{\ubranch}{upper-branch}
\newcommand{\Ubranch}{Upper-branch}
\newcommand{\lbranch}{lower-branch}
\newcommand{\Lbranch}{Lower-branch}

%%%% dynamical systems nomenclature:
\newcommand{\eqb}{equilibrium}
\newcommand{\Eqb}{equilibrium}
\newcommand{\eqba}{equilibria}
\newcommand{\Eqba}{Equilibria}
\newcommand{\eqpoint}{equilibrium point}
\newcommand{\Eqpoint}{Equilibrium point}
\newcommand{\reqb}{relative equilibrium}
\newcommand{\Reqb}{Relative equilibrium}
\newcommand{\reqba}{relative equilibria}
\newcommand{\Reqba}{Relative equilibria}
%\newcommand{\reqv}{equivariant equilibrium}
%\newcommand{\Reqv}{Equivariant equilibrium}
%\newcommand{\reqva}{equivariant equilibria}
%\newcommand{\Reqva}{Equivariant equilibria}
%\newcommand{\equilibrium}{equilibrium}
%\newcommand{\equilibria}{equilibria}
%\newcommand{\Equilibria}{Equilibria}

%%%% fluid dynamics nomenclature:
% \newcommand{\eqv}{steady state}
% \newcommand{\Eqv}{Steady state}
% \newcommand{\eqva}{steady states}
% \newcommand{\Eqva}{Steady states}
% \newcommand{\reqv}{traveling wave}
% \newcommand{\Reqv}{Traveling wave}
% \newcommand{\reqva}{traveling waves}
% \newcommand{\Reqva}{Traveling waves}
% \newcommand{\equilibrium}{steady state}
% \newcommand{\equilibria}{steady states}
% \newcommand{\Equilibria}{Steady states}

\newcommand{\po}{periodic orbit}
\newcommand{\Po}{Periodic orbit}
\newcommand{\rpo}{relative periodic orbit}
%   \newcommand{\rpo}{equivariant periodic orbit}
\newcommand{\Rpo}{Relative periodic orbit}
%   \newcommand{\Rpo}{Equivariant periodic orbit}
\newcommand{\UPO}{unstable periodic orbit}


\newcommand{\cohStr}{coherent state}
\newcommand{\recurrStr}{recurrent coherent state}
\newcommand{\RecurrStr}{Recurrent coherent state}

\newcommand{\statesp}{state-\-space}
\newcommand{\Statesp}{State-\-space}
\newcommand{\nameit}{E}         % equilibrium label
\newcommand{\SIS}{non-wondering set}
\newcommand{\descent}{Newton descent}
\newcommand{\Descent}{Newton Descent}
\newcommand{\stabmat}{stability matrix}     % stability matrix
\newcommand{\Stabmat}{Stability matrix}     % Stability matrix
\newcommand{\jacobianM}{fundamental matrix}     % standard name
\newcommand{\jacobianMs}{fundamental matrices}  %
\newcommand{\JacobianM}{Fundamental matrix}     %
\newcommand{\JacobianMs}{Fundamental matrices}  %
%\newcommand{\jacobian}{Jacobian}                % determinant
%\newcommand{\jacobianM}{Jacobian matrix}        % matrix
%\newcommand{\jacobianMs}{Jacobian matrices}     % matrices

\newcommand{\ew}{eigen\-value}
\newcommand{\ev}{eigen\-vector}
\newcommand{\ef}{eigen\-function}


%%%%%%%%%%%%%%%%%%%%% JH Math Shortcuts %%%%%%%%%%%%%%%%%%%%%%%%%%%%%%%%%
\newcommand{\ident}{\mathbf{1}}
\newcommand{\vect}{\mathbf}
\newcommand{\matr}{\mathit}

%%%%%%%%%%%%%%%%%%%%%%%%%%%%%%%%%%%%%%%%%%%%%%%%%%%%%%%%%%%
% JFG favorite macros

\newcommand{\bu}{\ensuremath{{\bf u}}}
\newcommand{\bv}{\ensuremath{{\bf v}}}
\newcommand{\bff}{\ensuremath{{\bf f}}}
\newcommand{\dbu}{\delta {\bf u}}
\newcommand{\dbv}{\delta {\bf v}}
\newcommand{\hbu}{\hat{{\bf u}}}
\newcommand{\hbv}{\hat{{\bf v}}}
\newcommand{\hu}{\hat{u}}
\newcommand{\hv}{\hat{v}}
\newcommand{\hw}{\hat{w}}
\newcommand{\be}{{\bf e}}
\newcommand{\bx}{{\bf x}}
\newcommand{\ex}{{\hat{\bf x}}} % unit vectors
\newcommand{\ey}{{\hat{\bf y}}}
\newcommand{\ez}{{\hat{\bf z}}}
\newcommand{\Om}{\Omega}    % JFG mantra

\newcommand{\bPhi}{{\bf \Phi}}
\newcommand{\bphi}{{\bf \phi}}
\newcommand{\bhphi}{{\bf \hat{\phi}}}
\newcommand{\bU}{{\bf U}}
\newcommand{\bW}{{\bf W}}
\newcommand{\lapl}{\nabla^2}
\newcommand{\tEQ}{\ensuremath{{\text{EQ}}}}
\newcommand{\tNB}{\ensuremath{{\text{NB}}}}
\newcommand{\tLB}{\ensuremath{{\text{LB}}}}
\newcommand{\tUB}{\ensuremath{{\text{UB}}}}
\newcommand{\tLM}{\ensuremath{{\text{LM}}}}
\newcommand{\tNS}{\ensuremath{{\text{NS}}}}
\newcommand{\tCFD}{\ensuremath{{\text{CFD}}}}
% \newcommand{\tODE}{\text{ODE}}    % JFG school of decoration
\newcommand{\tODE}{}        % PC: ODE is only one we use
\newcommand{\stagn}{\ensuremath{\text{\tiny EQ}}}% JFG equilib/stagnation point
\newcommand{\uEQ}{\ensuremath{\bu_{\text{\tiny EQ}}}}
\newcommand{\vEQ}{\ensuremath{\bv_{\text{\tiny EQ}}}}
\newcommand{\uLM}{\ensuremath{\bu_{\text{\tiny LM}}}}
\newcommand{\uLB}{\ensuremath{\bu_{\text{\tiny LB}}}}
\newcommand{\uNB}{\ensuremath{\bu_{\text{\tiny NB}}}}
\newcommand{\uUB}{\ensuremath{\bu_{\text{\tiny UB}}}}
\newcommand{\vLM}{\ensuremath{\bv_{\text{\tiny LM}}}}
\newcommand{\vLB}{\ensuremath{\bv_{\text{\tiny LB}}}}
\newcommand{\vNB}{\ensuremath{\bv_{\text{\tiny NB}}}}
\newcommand{\vUB}{\ensuremath{\bv_{\text{\tiny UB}}}}
\newcommand{\bbR}{\mathbb{R}}
\newcommand{\bbU}{\mathbb{U}}
\newcommand{\bbUsymm}{\bbU_{S}}
\newcommand{\half}{\frac{1}{2}}
\newcommand{\pd}[2]{\frac{\partial #1}{\partial #2}}
\newcommand{\Norm}[1]{\|{#1}\|}
\newcommand{\grad}{\boldsymbol{\nabla}}

\newcommand{\evOper}{evolution oper\-ator}
\newcommand{\EvOper}{Evolution oper\-ator}
\newcommand{\Fd}{spec\-tral det\-er\-min\-ant}
\newcommand{\fd}{spec\-tral det\-er\-min\-ant}
\newcommand{\FD}{Spec\-tral det\-er\-min\-ant}

%%%%%multiletter symbols
\newcommand\Real{\mbox{Re}} % cf plain TeX's \Re, not Reynolds number
\newcommand\Imag{\mbox{Im}} % cf plain TeX's \Im

%%%%%%%%%%%%%%% Sundry symbols within math eviron.: %%%%%%%%%%%%
\newcommand\flow[2]{{f^{#1}(#2)}}
\newcommand\timeflow{{f^t}}
\newcommand{\reals}{\mathbb{R}}
\newcommand{\PoincS}{{\cal P}}     % symbol for Poincare section
\newcommand{\PoincM}{{P}}      % symbol for Poincare map
\newcommand{\PoincC}{{U}}      % symbol for Poincare constraint function
\newcommand{\pde}{\partial}
\newcommand{\jMP}{{\bf \hat{J}}}   % jacobiam matrix, Poincare return
\newcommand{\monodromy}{{\bf J}}   % monodromy matrix, full Poincare cut
                   % Fredholm det jacobian weight:
\newcommand{\oneMinJ}[1]
           {\left|\det\!\left(\matId-\monodromy_p^{#1}\right)\right|}

\newcommand{\dmn}{\ensuremath{\!-\!d}}             %  n-dimensional

\newcommand{\obser}{a}      % an observable from state space to R^n
\newcommand{\Obser}{A}      % time integral of an observable
\newcommand{\expct}    [1]{\left\langle {#1} \right\rangle}
\newcommand{\spaceAver}[1]{\left\langle {#1} \right\rangle}
\newcommand{\timeAver} [1]{\overline{#1}}
\newcommand{\Lop}{{\cal L}}    % evolution operator
\renewcommand\Im{{\rm Im\,}}
\renewcommand\Re{{\rm Re\,}}
\renewcommand{\det}{\mbox{\rm det}\,}
\newcommand{\Det}{\mbox{\rm Det}\,}
\newcommand{\tr}{\mbox{\rm tr}\,}
\newcommand{\Tr}{\mbox{\rm tr}\,}

\newcommand{\Refl}{\ensuremath{\mathbf{R}}}
\newcommand{\Shift}{\ensuremath{\mathbf{S}}}
\newcommand{\shift}{\ensuremath{\ell}}
\newcommand\period[1]{{T_{#1}}}         %continuous cycle period
\newcommand{\cl}[1]{{n_{#1}}}   % discrete length of a cycle, Predrag
\newcommand{\pS}{{\cal M}}          % symbol for state space
% \newcommand\pSpace{x}     % phase space x=(q,p) coordinate
\newcommand{\ssp}{a}            % state space point
\newcommand\velField[1]{{F(#1)}}    % Gibson statespace velocity field

\newcommand\xInit{{a_0}}        %initial x
\newcommand{\deltaX}{{\delta a}}                %trajectory displacement

\newcommand{\costFct}{cost function}    % functional to minimize
\newcommand{\costF}{F^2}        % cost function,
\newcommand{\Loop}{L}
\newcommand{\pVeloc}{v}         % phase-space velocity
\newcommand{\lSpace}{\tilde{x}}     % a point on a loop
\newcommand{\lVeloc}{\tilde{v}}     % loop tangent
\newcommand{\damp}{\Delta\tau}      % descrete fictitous time step
\newcommand{\prpgtr}[1]{\delta\negthinspace\left( {#1} \right)}
\newcommand{\matId}{{\bf 1}}       % matrix identity
\newcommand{\inertM}{{\mathcal M}}          % inertial manifold
\newcommand{\Mvar}{\ensuremath{A}}  % stability matrix
% \newcommand{\derF}[1]{\ensuremath{A(#1)}}   % Predrag stability matrix
\newcommand{\derF}[1]{{DF |_{#1}}}        % Gibson stability matrix
\newcommand{\jMps}{\ensuremath{J}}   % fundamental matrix, phase space
% \newcommand{\jMps}{\ensuremath{{\bf J}}}  % bold fundamental matrix phase space
\newcommand{\derf}[2]{\ensuremath{{J}^{#1}(#2)}}    % Predrag fundamental matrix
% \newcommand{\derf}[2]{\ensuremath{{\bf J}^{#1}(#2)}}  % Predrag bold fundamental matrix
% \newcommand{\derf}[2]{{Df^{#1}|_{#2}}}   % Gibson fundamental matrix
\newcommand{\ExpaEig}{\Lambda}
\newcommand{\PredragsGarlic}{e}
\newcommand\Lyap{\lambda}                       %Lyapunov exponent
\newcommand{\eigenvL}{{s}}
\newcommand{\eigenvG}{{m}}         % compact group eigenvalues

%%       optional parameter comes in [\ldots], for example
%%       \newcommand\eigRe[1][ ]{\ensuremath{\mu_{#1}}}
%%       no subscript: \eigRe\
%%       with subscript j: \eigRe[j]
%%
%%      Guckenheimer-Holmes:  lambda = alpha + i beta
%%      Hirsch-Smale:         lambda = a     + i b
%%      Boyce-di Prima:       lambda = mu    + i nu
%%      Gibson:        lambda = mu    + i omega (best of the bunch!)
%
% Re eigen-exponent superscripting
% Getting into the ChaosBookie groove... awesome!
% The groove is groovy when the macros reduce typing...

\newcommand{\eigExp}[1][]{
\ifthenelse{\equal{#1}{}}{\ensuremath{\lambda}}{\ensuremath{\lambda^{(#1)}}}
                        }
\newcommand{\eigRe}[1][]{
\ifthenelse{\equal{#1}{}}{\ensuremath{\mu}}{\ensuremath{\mu^{(#1)}}}
                        }
\newcommand{\eigIm}[1][]{
  \ifthenelse{\equal{#1}{}}{\ensuremath{\omega}}{\ensuremath{\omega^{(#1)}}}
            }

\newcommand{\tny}[1]{{\text{\tiny {#1}}}}

\newcommand{\jEigvec}[1][]{\ensuremath{{\bf v}^{(#1)}}} % jacobiam eigenvector

% Guck & Holmes use $W^s$, $W^u$ for stable, unstable manifolds.
% usage: \Wmnfld{u,(n)}{NB} unstable manifold of NB's nth eigenvalue.

\newcommand{\Wmnfld}[2]{%
\ifthenelse{\equal{#2}{}}{\ensuremath{W_{#1}}\!}
                         {\ensuremath{W^{#1}_{\text{\tiny #2}}}\!} %Negative space is screwing up spacing in text
                        }
