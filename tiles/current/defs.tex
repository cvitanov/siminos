% siminos/tiles/current/defs.tex
% SIAM Journal on Applied Dynamical Systems SIADS style
% $Author: mgudorf3 $ $Date: 2020-10-22 10:19:52 -0400 (Thu, 22 Oct 2020) $

%%%%%%%%%%%%% SETUP          %%%%%%%%%%%%%%%%%%%
    \usepackage{amssymb}
    \usepackage{amsmath}
    \usepackage{amsfonts}
    \usepackage{graphicx}
    \usepackage{subfigure}
    \usepackage{dsfont}
    \usepackage{mathrsfs}
    \usepackage{color} % dvips allows for colors

    \bibliographystyle{siamplain} % in 2007 used {hsiam}
    \graphicspath{{figs/}{../../figs/}}  %% directories with graphics files

% Predrag commented out - option clash?             2019-03-12
%    \usepackage[bookmarks,colorlinks]{hyperref}
    \hypersetup{
   pdfauthor={M. N. Gudorf and P. Cvitanovic},
   pdfkeywords=Kuramoto-Sivashinsky chaos,
   pdftitle=Spatiotemporal tiling of Kuramoto-Sivashinsky flow
                }


%%%%%%%%%%%%%%%%%%%%%% COMMENTS %%%%%%%%%%%%%%%%%%%
    \ifdraft    % display comments in text
\newcommand{\Preliminary}[1]{\color{blue}#1\color{black}}
\newcommand{\PublicPrivate}[2]{{\color{green}#2}}% PRIVATE
\newcommand{\PC}[2]{$\footnotemark\footnotetext{Predrag #1: #2}$} %date, comment
\newcommand{\PCedit}[1]{{\color{red}#1}}
\newcommand{\MNG}[2]{$\footnotemark\footnotetext{Matt #1: #2}$} %date, comment
\newcommand{\MNGedit}[1]{{\color{magenta}#1}}
%\newcommand{\file}[1]{$\footnotemark\footnotetext{{\bf file} #1}$}
\def\mycomment #1#2 {\noindent \textbf{\underline{#1}}: \emph{#2}}
\newcommand{\toCB}{\marginpar{\footnotesize 2CB}}  % to compare with ChaosBook
\newcommand{\inCB}{\marginpar{\footnotesize now in CB}} % entered in ChaosBook
    \else   % drop comments
\newcommand{\Preliminary}[1]{}
\newcommand{\PublicPrivate}[2]{#1} % PUBLIC
\newcommand{\PC}[2]{}
\newcommand{\PCedit}[1]{#1}
\newcommand{\MNG}[2]{}
\newcommand{\MNGedit}[1]{#1}
%\newcommand{\file}[1]{}
\def\mycomment #1#2 {}
\newcommand{\toCB}{}
\newcommand{\inCB}{}
    \fi
%%%%%%%%%%%%%%%%%%%%%% COMMENTS END %%%%%%%%%%%%%%%

  \newcommand{\colorcomm}[2]{#1}
  %%%%%%%%%%%%%%%%%%%%%% Weblinks in PDF %%%%%%%%%%%%%%%%%%%
  % keep homepages flexible, if hyperlinked:
  %\newcommand{\weblink}[1]{{\tt #1}}
  %% RESTORE \newcommand{\weblink}[1]{\href{http://#1}{{\tt #1}}}
  %\newcommand{\arXiv}[1]{arXiv.org:#1}
  %% RESTORE \newcommand{\arXiv}[1]{\href{http://arXiv.org/abs/#1}{{\tt #1}}}
  %\newcommand{\wwwcb}[1]{{\tt ChaosBook.org#1}}
  %\newcommand{\HREF}[2]{{#2}}
  %% RESTORE \newcommand{\HREF}[2]{{\href{#1}{#2}}}
  \newcommand{\wwwcb}[1]{{\tt \href{http://ChaosBook.org#1}
         {ChaosBook.org#1}}}
  \newcommand{\wwwQFT}[1]{
         {\tt \href{http://ChaosBook.org/FieldTheory#1}
         {ChaosBook.org/\-Field\-Theory#1}}}
  \newcommand{\wwwnsQFT}[1]{
         {\tt \href{http://ChaosBook.org/FieldTheory#1}
         {ChaosBook.org/\-Field\-Theory#1}}}
  \newcommand{\weblink}[1]{{\tt \href{http://#1}{#1}}}
  \newcommand{\HREF}[2]{{\href{#1}{#2}}}
  \newcommand{\mpArc}[1]{
         {\tt \href{http://www.ma.utexas.edu/mp_arc-bin/mpa?yn=#1}
         {\goodbreak mp\_arc~#1}}}
  \newcommand{\arXiv}[1]{
         {\tt \href{http://arXiv.org/abs/#1}{\goodbreak #1}}}


%%%%%%%%%%%% MACROS, spatiotemporal %%%%%%%%%%
\newcommand{\conf}{\ensuremath{x}}      % Configuration space coordinate
\newcommand{\Fu}{\hat{\mathbf{u}}}
\newcommand{\FFT}{\ensuremath{\mathcal{F}}}
\newcommand{\IFFT}{\ensuremath{\mathcal{F}^{-1}}}
\newcommand{\dof}{degree of freedom} % ChaosBook Hamiltonian deegree of freedom
\newcommand{\dofs}{degrees of freedom}
% \newcommand{\Dofs}{Degrees of freedom}
\newcommand{\cdof}{computational degree of freedom}  % ChaosBook
\newcommand{\cdofs}{computational degrees of freedom}
\newcommand{\Cdof}{Computational degree of freedom}
\newcommand{\Cdofs}{Computational degrees of freedom}

\newcommand{\abs}[1]{\lvert#1\rvert}    % Absolute value
\newcommand{\twot}{doubly-periodic orbit} % Predrag, to distinguish from a PO
\newcommand{\twots}{doubly-periodic orbits}
\newcommand{\twoT}{Doubly-periodic orbit}
\newcommand{\twoTs}{Doubly-periodic orbits}
%\newcommand{\twot}{invariant 2-torus} % Predrag, to distinguish from a PO
%\newcommand{\twots}{invariant 2-tori}
%\newcommand{\twoT}{Invariant 2-torus}
%\newcommand{\twoTs}{Invariant 2-tori}
\newcommand{\PV}{Percival-Vivaldi}
\newcommand{\AW}{Adler-Weiss}
\newcommand{\catlatt}{spatiotemporal cat}     % Predrag, experimental
%\newcommand{\catlatt}{spatiotemporal cat map}     % Predrag, experimental
\newcommand{\catLatt}{Spatiotemporal cat}     % Predrag, experimental
%\newcommand{\catLatt}{Spatiotemporal cat map}     % Predrag, experimental
%\newcommand{\block}[1]{\ensuremath{#1}} % PC 07sep2008: conflict with beamer
\newcommand{\brick}{symbolic block}
\newcommand{\Brick}{Symbolic block}
\newcommand{\AdmItnr}{\Sigma}      % set of admissible itineraries
\newcommand{\spt}{spatiotemporal}
\newcommand{\Spt}{Spatiotemporal}
\newcommand{\fpo}{fundamental orbit}
\newcommand{\Fpo}{Fundamental orbit}
\newcommand{\tile}{spatiotemporal tile}
\newcommand{\po}{periodic orbit}
\newcommand{\Po}{Periodic orbit}
\newcommand{\rpo}{relative periodic orbit}
\newcommand{\Rpo}{Relative periodic orbit}
\newcommand{\ppo}{pre-periodic orbit}
\newcommand{\Ppo}{Pre-periodic orbit}
\newcommand{\cl}[1]{{\ensuremath{|#1|}}}  % the length of a periodic orbit, Ronnie
\newcommand{\velgradmat}{velocity gradients matrix}
\newcommand{\Poincare}{Poincar\'e }
\newcommand{\PoincSec}{Poincar\'e section}
\newcommand{\PoincMap}{return map} %\Poincare\ return map
\newcommand{\slice}{slice}
\newcommand{\Slice}{Slice}
\newcommand{\symbolic}{symbolic represention}
\newcommand{\rv}{real valued}
\newcommand{\Rv}{Real valued}
\newcommand{\Fcs}{Fourier coefficients}
\newcommand{\extent}{spatiotemporal domain size}
\newcommand{\wiggle}{spatiotemporal wiggle}
\newcommand{\defect}{spatiotemporal defect}
\newcommand{\streak}{spatiotemporal streak}


\newcommand{\eqv}{equilibrium}
\newcommand{\Eqv}{equilibrium}
\newcommand{\eqva}{equilibria}
\newcommand{\Eqva}{Equilibria}
\newcommand{\reqv}{relative equilibrium}
\newcommand{\Reqv}{Relative equilibrium}
\newcommand{\reqva}{relative equilibria}
\newcommand{\Reqva}{Relative equilibria}

\newcommand{\KS}{Ku\-ra\-mo\-to-Siva\-shin\-sky}
\newcommand{\KSe}{Ku\-ra\-mo\-to-Siva\-shin\-sky equation}
\newcommand{\KSf}{Ku\-ra\-mo\-to-Siva\-shin\-sky flow}
\newcommand{\NS}{Navier-Stokes}
\newcommand{\NSe}{Navier-Stokes equation}
\newcommand{\pCf}{plane Couette flow}
\newcommand{\PCf}{Plane Couette flow}

%%%%%% Boris definitions
\newcommand{\Aa}{\ensuremath{\bar{\A}}}
\newcommand{\A}{\ensuremath{\mathcal{A}}}       % alphabet
\newcommand{\Ai}{\ensuremath{\mathcal{A}_0}}    % alphabet inner
\newcommand{\Ae}{\ensuremath{\mathcal{A}_1}}    % alphabet outer
\newcommand{\R}{\ensuremath{\mathcal{R}}}
%\newcommand{\m}{\ensuremath{\mathsf{m}}}     % Boris
\newcommand{\m}{\ensuremath{m}}     % Predrag experimental 2016-11-08
\newcommand{\Mm}{\ensuremath{\mathsf{M}}}
\newcommand{\Xx}{\ensuremath{\mathsf{X}}}
\newcommand{\Zz}{\ensuremath{\mathbb{Z}^2}}
\newcommand{\ZLT}{\mathbb{Z}^2_{\scriptscriptstyle\mathrm{LT}}}
%\newcommand{\q}{\ensuremath{\mathsf{m}}}     % Boris
\newcommand{\q}{\ensuremath{q}}     % Predrag experimental 2016-11-08
\newcommand{\D}{\mathcal{D}}
\newcommand{\gd}{\mathsf{g}}
\newcommand{\gp}{\mathsf{g}^0}

\newcommand{\tildeL}{\ensuremath{\tilde{L}}}
\newcommand{\Lint}[1]{\frac{1}{L}\!\oint d#1\,}

%%%%%%%%%%%% MACROS, variational %%%%%%%%%%
% Predrag   defCrete.tex             4mar2003
% Predrag   loopDefs.tex            10jul2003
\newcommand{\descent}{Newton descent}
\newcommand{\Descent}{Newton Descent}
\newcommand{\CostFct}{Cost function}    % functional to minimize
\newcommand{\costFct}{cost function}    % functional to minimize
\newcommand{\costF}{F^2}        % cost function,
\newcommand{\Loop}{L}
\newcommand{\lSpace}{\tilde{x}}     % a point on a loop
\newcommand{\lVeloc}{\tilde{v}}     % loop tangent
\newcommand{\damp}{\Delta\tau}      % descrete fictitous time step
% \newcommand{\pSpaceDer}[1]{x^{(#1)}}
% \newcommand{\lSpaceDer}[1]{\tilde{x}^{(#1)}}


\newcommand{\EQV}[1]{\ensuremath{\mathrm{E}_{#1}}}
% E_0: u = 0 - trivial equilibrium
% E_1,E_2,E_3, for 1,2,3-wave equilibria
\newcommand{\REQV}[2]{\ensuremath{\mathrm{TW}_{#1#2}}} % #1 is + or -
% TW_1^{+,-} for 1-wave traveling waves (positive and negative velocity).
\newcommand{\PO}[1]{\ensuremath{\mathrm{PO}_{#1}}}
% PO_{period to 2-4 significant digits} - periodic orbits
\newcommand{\RPO}[1]{\ensuremath{\mathrm{RPO}_{#1}}}
% RPO_{period to 2-4 significant digits} - relative PO.  We also can use ^{+,-}
% here if necessary to distinguish between members of a reflection-symmetric
% pair.

\newcommand{\expctE}{\ensuremath{E}}    % E space averaged

%%%%%%%%%%%%    CROSS REFERENCING, SIADS conventions  %%%%%%%%%%%%%%%%%

\newcommand{\rf}      [1] {~\cite{#1}}
\newcommand{\refref}  [1] {\cite{#1}}
\newcommand{\refRef}  [1] {\cite{#1}}
\newcommand{\refrefs} [1] {\cite{#1}}
\newcommand{\refRefs} [1] {\cite{#1}}
\newcommand{\refeq}   [1] {(\ref{#1})}
\newcommand{\refeqs}  [2] {(\ref{#1}--\ref{#2})}
\newcommand{\refpage} [1] {page~\pageref{#1}}
    % Phys Rev style: Figure to start a sentence, else Fig.
\newcommand{\reffig}  [1] {Figure~\ref{#1}}
\newcommand{\reffigs} [2] {Figures~\ref{#1} and~\ref{#2}}
\newcommand{\refFig}  [1] {Figure~\ref{#1}}
\newcommand{\refFigs}  [2] {Figures~\ref{#1} and~\ref{#2}}
\newcommand{\refFigToFig}  [2] {Figures~\ref{#1} -~\ref{#2}}
\newcommand{\reftab}  [1] {Table~\ref{#1}}
\newcommand{\refTab}  [1] {Table~\ref{#1}}
\newcommand{\reftabs} [2] {Tables~\ref{#1} and~\ref{#2}}
\newcommand{\refsect} [1] {Sect.~\ref{#1}}
\newcommand{\refsects}[2] {Sects.~\ref{#1}--\ref{#2}}
\newcommand{\refSect} [1] {Sect.~\ref{#1}}
\newcommand{\refSects}[2] {Sects.~\ref{#1}--\ref{#2}}
\newcommand{\refappe} [1] {Appendix~\ref{#1}}
\newcommand{\refappes}[2] {Appendices~\ref{#1}--\ref{#2}}
\newcommand{\refAppe} [1] {Appendix~\ref{#1}}

%%%%%%%%%%%%%%% EQUATIONS, STANDARD %%%%%%%%%%%%%%%%%%%%%%%%%%%%%%%

\newcommand{\beq}{\begin{equation}}
\newcommand{\eeq}{\end{equation}}
\newcommand{\ee}[1]{\label{#1} \end{equation}}
\newcommand{\bea}{\begin{eqnarray}}
\newcommand{\ceq}{\nonumber \\ & & }
\newcommand{\continue}{\nonumber \\ }
\newcommand{\nnu}{\nonumber}
\newcommand{\eea}{\end{eqnarray}}

%%%%%%%%%%%%%%  Abbreviations %%%%%%%%%%%%%%%%%%%%%%%%%%%%%%%%%%%%%%%%

\newcommand{\etc}{{\em etc.}}       % etcetera in italics
\newcommand{\ie}{{that is}}     % use Latin or English?  Decide later.
\newcommand{\cf}{{\em cf.}}
\newcommand{\etal}{{\em et al.}}              % etcetera in italics

\newcommand{\jacobianM}{fundamental matrix} % standard name
\newcommand{\jacobianMs}{fundamental matrices}  %
\newcommand{\JacobianM}{Fundamental matrix} %
\newcommand{\JacobianMs}{Fundamental matrices}  %
\newcommand{\stabmat}{stability matrix}     % stability matrix
\newcommand{\Stabmat}{Stability matrix}     % Stability matrix
\newcommand{\statesp}{state space}
\newcommand{\Statesp}{State space}
\newcommand{\dmn}{-dim\-en\-sion\-al}
\newcommand{\pdes}{partial differential equations}
\newcommand{\Pdes}{Partial differential equations}

%%%%%%%%%%%%%%% Sundry symbols within math eviron.: %%%%%%%%%%%%

\newcommand{\reals}{\mathbb{R}}
\newcommand{\complex}{\mathbb{C}}
\newcommand{\integers}{\mathbb{Z}}
\newcommand{\rationals}{\mathbb{Q}}
\newcommand{\naturals}{\mathbb{N}}
\newcommand{\ii}{\ensuremath{\mathrm{i}}} % sqrt{-1}
\newcommand{\half}{{\scriptstyle{\frac{1}{2}}}}
\newcommand{\zeit}{\ensuremath{t}}  %time variable Ashley
\newcommand{\pde}{\partial}
\newcommand{\obser}{a}      % an observable from phase space to R^n
\newcommand{\Obser}{A}      % time integral of an observable
\renewcommand\Im{\ensuremath{{\rm Im\,}}}
\renewcommand\Re{\ensuremath{{\rm Re\,}}}
\renewcommand{\det}{\mbox{\rm det}\,}
\newcommand{\expct}    [1]{\langle {#1} \rangle}
\newcommand{\spaceAver}[1]{\langle {#1} \rangle}
%\newcommand{\expct}    [1]{\left\langle {#1} \right\rangle}
%\newcommand{\spaceAver}[1]{\left\langle {#1} \right\rangle}
\newcommand{\timeAver} [1]{\overline{#1}}
\newcommand{\matId}{\ensuremath{\mathbf 1}}       % matrix identity
\newcommand{\Ssym}[1]{{\ensuremath{s_{#1}}}}
\newcommand{\pS}{\ensuremath{{\cal M}}}          % symbol for state space
\newcommand{\ssp}{\ensuremath{x}}                % state space point
\newcommand{\bu}{\ensuremath{{\bf u}}}
\newcommand{\bx}{\ensuremath{{\bf \conf}}}
\newcommand{\bv}{\ensuremath{{\bf \vel}}}

\newcommand{\Un}[1]{\ensuremath{\textrm{U}(#1)}}         % in DasBuch
\newcommand{\SUn}[1]{\ensuremath{\textrm{SU}(#1)}}         % in DasBuch
%\newcommand{\On}[1]{\ensuremath{\mathbf{O}(#1)}}
\newcommand{\On}[1]{\ensuremath{\textrm{O}(#1)}}
%\newcommand{\SOn}[1]{\ensuremath{\mathbf{SO}(#1)}} % in Siminos thesis
\newcommand{\SOn}[1]{\ensuremath{\textrm{SO}(#1)}}         % in DasBuch
\newcommand{\Spn}[1]{\ensuremath{\textrm{Sp}(#1)}}         % in DasBuch
%\newcommand{\Dn}[1]{\ensuremath{\mathbf{D}_{#1}}    % in Siminos thesis
\newcommand{\Dn}[1]{\ensuremath{\textrm{D}_{#1}}}              % in DasBuch
\newcommand{\Cn}[1]{\ensuremath{\textrm{C}_{#1}}}
\newcommand{\Zn}[1]{\ensuremath{\textrm{Z}_{#1}}}   % C_n in DasBuch
%\newcommand{\Zn}[1]{\ensuremath{\mathbf{Z}_{#1}}}    % in Siminos thesis
%\newcommand{\Zn}[1]{\ensuremath{\textrm{C}_{#1}}}  % until 2018-05-02
%\newcommand{\Ztwo}{\ensuremath{\mathbf{Z}_2}}      % in Siminos thesis
\newcommand{\Ztwo}{\ensuremath{\textrm{Z}_2}}       % in DasBuch
%\newcommand{\Ztwo}{\ensuremath{\textrm{C}_2}}      % until 2018-05-02
\newcommand{\Refl}{\ensuremath{R}}
%\newcommand{\Refl}{\ensuremath{\kappa}}            % Siminos uses R for rotations.
%\newcommand{\Refl}{\ensuremath{\sigma}}             % in DasBuch
%\newcommand{\Shift}{\ensuremath{\tau}}
\newcommand{\Rot}[1]{\ensuremath{C^{#1}}}           % in DasBuch, e.g. C^{1/3}
%\newcommand{\Rot}[1]{\ensuremath{R(#1)}}           % Siminos uses R for rotations.
%\newcommand{\Drot}{\ensuremath{\zeta}}
%\newcommand{\Lg}{\mathcal{G}}
%\newcommand{\stab}[1]{\ensuremath{\Sigma_{#1}}}
\newcommand{\stab}[1]{\ensuremath{G_{#1}}}
%\newcommand{\shift}{\ensuremath{d}}
\newcommand{\shift}{\ensuremath{\ell}}
\newcommand{\Shift}{\ensuremath{\tau}}
\newcommand{\Fix}[1]{\ensuremath{\mathrm{Fix}\left(#1\right)}}

\newcommand{\pVeloc}{\ensuremath{v}}    % phase-space velocity
\newcommand{\Mvar}{\ensuremath{A}}  % matrix of variations
\newcommand{\jMps}{{J}}    % jacobiam matrix, full phase space
                   % Fredholm det jacobian weight:

\newcommand{\jEigvec}[1]{\ensuremath{{\mathbf e}^{(#1)}}}   % jacobiam eigenvector
\newcommand{\ExpaEig}{\ensuremath{\Lambda}}
\newcommand{\translGen}{\ensuremath{\mathcal{L}}}

\newcommand{\dual}[1]{{#1}^\dag}
% \newcommand{\dual}[1]{{#1}^\ast}
% \newcommand{\transp}[1]{\bar{#1}}
% \newcommand{\transp}[1]{{#1}{}^T}
\newcommand{\transp}[1]{{#1}{}^\top}
	% without large brackets:
\newcommand{\braket}[2]
		   {\langle{#1}\vphantom{#2}|\vphantom{#1}{#2}\rangle}
\newcommand{\bra}[1]{\langle{#1}\vphantom{ }|}
\newcommand{\ket}[1]{|\vphantom{}{#1}\rangle}
	% with large brackets:
%\newcommand{\bra}[1]{\left\langle{#1}\vphantom{ }\right|}
%\newcommand{\ket}[1]{\left|\vphantom{}{#1}\right\rangle}
%\newcommand{\braket}[2]{\left\langle{#1}
%                        \vphantom{#2}\right|\left.\vphantom{#1}
%                        {#2}\right\rangle}

%%   optional parameter comes in [\ldots], for example
%%   \newcommand\eigRe[1][ ]{\ensuremath{\mu_{#1}}}
%%   no subscript: \eigRe\
%%   with subscript j: \eigRe[j]
%%
\newcommand{\eigExp}[1][ ]{\ensuremath{\lambda_{#1}}}   % complex eigenexponent
%%  Guckenheimer&Holmes:  lambda = alpha + i beta
%%  Hirsch-Smale:         lambda = a     + i b
%%  Boyce-di Prima:       lambda = mu    + i nu
\newcommand{\eigRe}[1][ ]{\ensuremath{\mu_{#1}}}    % Re eigenexponent
\newcommand{\eigIm}[1][ ]{\ensuremath{\nu_{#1}}}    % Im eigenexponent

%%%%%%%%%%%%%%%%%%%%%%%%%%%%%%%%%%%%%%%%%%%%%%%%%%%%%%%%%%%
% Lopez favorite macros, REMOVE MOST EVENTUALLY
\newcommand{\sss}{\scriptscriptstyle}	
%\newcommand{\e}{\mathrm{e}}
%\newcommand{\mmbf}[1]{\mbox{\boldmath ${#1}$}}
%\newcommand{\Arot}{\theta}
%\newcommand{\spacetrans}{\sigma}
%\newcommand{\timetrans}{\tau}
%\newcommand{\lsym}{l}
%\newcommand{\ce}{c_{\sss{1}}}
%\newcommand{\co}{c_{\sss{2}}}
%\newcommand{\Lx}{L_{\sss x}}
\newcommand{\omegaj}{\omega_{\sss j}}
\newcommand{\freqj}{\omega_{\sss j}}
\newcommand{\wavek}{q_{\sss k}}
\newcommand{\am}[1]{a_{\sss{{#1}}}}
\newcommand{\akj}{\hat{a}_{\sss{kj}}}
\newcommand{\bkj}{\hat{b}_{\sss{kj}}}
\newcommand{\ckj}{\hat{c}_{\sss{kj}}}
\newcommand{\dkj}{\hat{d}_{\sss{kj}}}
\newcommand{\ukj}{\hat{u}_{\sss{kj}}}
\newcommand{\xm}{x_{\sss m}}
\newcommand{\tn}{t_{\sss n}}
\newcommand{\xDft}{discrete Fourier transform}

%%%%%%%%%%%%%%% Group theory %%%%%%%%%%%%%%%%%%%%%%

%\newcommand{\Group}{\ensuremath{\Gamma}}    % Siminos Lie group
\newcommand{\Group}{\ensuremath{G}}         % Predrag Lie or discrete group
\newcommand{\LieEl}{\ensuremath{g}}  % Predrag group element
%\newcommand{\Lg}{\mathfrak{a}}             % Siminos Lie algebra generator
\newcommand{\Lg}{\ensuremath{\mathbf{T}}}   % Predrag Lie algebra generator
\newcommand{\gSpace}{\ensuremath{{\bf \phi}}}   % MA group rotation parameters
% \newcommand{\gSpace}{\ensuremath{{\bf \theta}}}   % PC group rotation parameters
\newcommand{\trHalf}[1]{\tau_{#1}}    % 1/2 cell translation

%%%%%%%%%% flows: %%%%%%%%%%%%%%%%%%%%%%%%%%%%

\newcommand\flow[2]{{f^{#1}(#2)}}
\newcommand{\vel}{\ensuremath{v}}   % state space velocity
\newcommand\pSpace{x}       % phase space x=(q,p) coordinate
\newcommand\stagn{q}        %equilibrium/stagnation point suffix
\newcommand{\bbU}{\mathbb{U}}
\newcommand{\bbUsymm}{\bbU_{c}}

%%%%%%%%%% periods: %%%%%%%%%%%%%%%%%%%%%%%%%%%%

\newcommand\period[1]{\ensuremath{T_{#1}}}     %continuous cycle period
\newcommand\speriod[1]{{\ensuremath{\tilde{L}_{#1}}}}  %continuous spatial period
\newcommand\Lyap{\lambda}           %Lyapunov exponent

%%%%%%%%%%%%%%%%%%%%%%%%%%%%%%%%%%%%%%%%%%%%%%%%%%%%
