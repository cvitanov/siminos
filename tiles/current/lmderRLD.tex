% lmderRLD.tex
% $Author: predrag $ $Date: 2020-05-07 18:16:42 -0400 (Thu, 07 May 2020) $

\section{Searches for \po s}
\label{sec:lmderRLD}

In order to find periodic orbits, a system of nonlinear
algebraic equations needs to be solved.

Our approach differs from those used previously in that we do
not XXX.  Being an optimization solver,
the algorithm has no problem with XXX. In what
follows we give a detailed description of the algorithm and
the search strategy which we have used to find a large number
of \po s defined in \refeq{XXX}.

When searching for \rpo s of truncated \KS\ equation
\refeq{eq:KS}, we need to solve the system of $N-2$ equations
\beq
  {\bf g}(\shift)f^\period{}(a) - a = 0\,,
\ee{eq:system}
with $N$ unknowns $(a, \speriod{}, \period{}, \shift)$, where $f^t$
is the flow map of the \KSe.

We have tried two different implementations of XXX.
The emphasis was on the simplicity of the implementations, so, even
though both implementations worked equally well, each of them had
its own minor drawbacks.

In the first implementation, we XXX.

Let $(\hat{a}, \hat{\speriod{}}, \hat{\period{}}, \hat{\shift})$ be the starting guess for a \rpo\
obtained through XXX (see (XXX)).

With the
number of shooting stages equal to $m$, the system in
\refeq{eq:system} is rewritten as follows
\begin{eqnarray}\label{eq:MultShoot}
 F^{(1)}&\!=\!& f^\tau(a^{(1)}) - a^{(2)} = 0\,,\nonumber\\
 F^{(2)}&\!=\!& f^\tau(a^{(2)}) - a^{(3)} = 0\,,\nonumber\\
 && \cdots \\
 F^{(m-1)}&\!=\!& f^\tau(a^{(m-1)}) - a^{(m)} = 0\,,\nonumber\\
 F^{(m)}&\!=\!& {\bf g}(\shift)f^{\tau'}(a^{(m)}) - a^{(1)} = 0\,,\nonumber
\end{eqnarray}
where $\tau = \lfloor n/m \rfloor h$ ($\lfloor x \rfloor$ is the nearest
integer smaller than $x$),
$\tau' = nh - (m-1)\tau$, and $a^{(j)} = f^{(j-1)\tau}(a)$,
$j = 1, \ldots , m$.
With the \jacobianM\ of \refeq{eq:MultShoot} written as
\beq
  J = \left(\begin{array}{cccc}\!\!
   \displaystyle \frac{\partial F^{(j)}}{\partial a^{(k)}} &
   \displaystyle \frac{\partial F^{(j)}}{\partial \speriod{}} &
   \displaystyle \frac{\partial F^{(j)}}{\partial \period{}} &
   \displaystyle \frac{\partial F^{(j)}}{\partial \shift}\!\!
  \end{array}\right),\quad j,k = 1,\ldots,m\,,
\eeq
the partial derivatives with respect to $a^{(k)}$ can be calculated
using the solution of \refeq{eq:KStan}.  The partial derivatives
with respect to $T$ are given by
\beq
  \frac{\partial F^{(j)}}{\partial \period{}} =
  \left\{\begin{array}{ll}
    \frac{\partial f^\tau(a^{(j)})}{\partial \tau}
    \frac{\partial \tau}{\partial T} = v(f^\tau(a^{(j)}))
    \lfloor n/m \rfloor/n\,, & j = 1,\ldots, m-1\\[.5ex]
    {\bf g}(\shift) v(f^{\tau'}(a^{(j)}))
    (1 - \frac{m-1}{n} \lfloor n/m \rfloor ), & j = m\,.
  \end{array}\right.
\eeq
Partial derivatives $\partial F^{(j)}/\partial \shift$
are all equal to zero except for $j = m$, where it is given by
\beq
  \frac{\partial F^{(m)}}{\partial \shift} =
  \frac{d{\bf g}}{d\shift}f^{\tau'}(a^{(m)}) =
  \mbox{diag}(i q_k e^{i q_k\, \shift} )f^{\tau'}(a^{(m)})\,.
\eeq
The following search strategy was adopted: XXX
