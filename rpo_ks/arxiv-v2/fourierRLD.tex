% fourierRLD.tex
% $Author: siminos $ $Date: 2009-04-10 14:47:20 -0400 (Fri, 10 Apr 2009) $

\section{Integrating \KSe\ numerically} \label{sec:fourierRLD}

The \KSe\ in terms of Fourier modes:
\beq
  \hat{u}_k = {\cal F}[u]_k = \frac{1}{L}\int_0^L u(x,t) e^{-i q_kx}dx\,,
  \qquad u(x,t) = {\cal F}^{-1}[\hat{u}] =
  \sum_{k\in{\mathbb Z}} \hat{u}_k e^{i q_k x}
\eeq
is given by
\beq
  \dot{\hat{u}}_k = \left(q_k^2-q_k^4\right) \hat{u}_k -
  \frac{i q_k}{2}{\cal F}[({\cal F}^{-1}[\hat{u}])^2]_k\,.
\eeq
Since $u$ is real, the Fourier modes are related by $\hat{u}_{-k} =
\hat{u}^\ast_k$.

The above system is truncated as follows: The Fourier transform
${\cal F}$ is replaced by its discrete equivalent
\beq
  a_k = {\cal F}_N[u]_k = \sum_{n = 0}^{N-1} u(x_n)
  e^{-i q_k x_n}\,,\qquad u(x_n) = {\cal F}_N^{-1}[a]_n
  = \frac{1}{N}\sum_{k = 0}^{N-1} a_k e^{i q_k x_n}\,,
\eeq
where $x_n = 2\pi\tildeL/N$ and $a_{N-k} = a^\ast_k$.  Since $a_0
= 0$ due to Galilean invariance and setting $a_{N/2} = 0$ (assuming
$N$ is even), the number of independent variables in the truncated
system is $N-2$:
\beq
  \dot{a}_k = \pVeloc_k(a) = \left(q_k^2-q_k^4\right)a_k -
  \frac{i q_k}{2}{\cal F}_N[({\cal F}_N^{-1}[a])^2]_k\,,
\ee{eq:KS}
where $k = 1,\ldots,N/2-1$, although in the Fourier transform we need
to use $a_k$ over the full range of $k$ values from 0 to $N-1$.
As $a_k \in \mathbb{C}$, \refeq{eq:KS} represents a
system of ordinary differential equations in ${\mathbb R}^{N-2}$.

The discrete Fourier transform ${\cal F}_N$ can be computed by FFT.
In Fortran and C, the FFTW library \refref{Frigo05} can be used.

In order to find the \jacobianM\ of the solution, or compute Lyapunov
exponents of the \KS\ flow, one needs to solve the equation for a
displacement vector $b$ in the tangent space: \beq
  \dot{b} = \frac{\partial \pVeloc(a)}{\partial a} b\,.
\eeq
Since ${\cal F}_N$ is a linear operator, it is easy to show that
\beq
  \dot{b_k} = \left(q_k^2-q_k^4\right)b_k -
  iq_k{\cal F}_N[{\cal F}_N^{-1}[a]\otimes {\cal F}_N^{-1}[b]]_k\,,
\ee{eq:KStan}
where $\otimes$ indicates componentwise product of two
vectors, \ie, $a\otimes b = \diag(a)\,b = \diag(b)\,a$.  This equation
needs to be solved simultaneously with \refeq{eq:KS}.

Equations \refeq{eq:KS} and \refeq{eq:KStan} were solved using the
exponential time differencing 4th-order Runge-Kutta method
(ETDRK4)\rf{cox02jcomp,ks05com}.

\section{Determining stability properties of equilibria,
traveling waves, and \rpo s} \label{sec:stability}

Let $f^t$ be the flow map of the \KSe, \ie\ $f^t(a) = a(t)$ is the
solution of \refeq{eq:KS} with initial condition $a(0) = a$.
The stability properties of the solution $f^t(a)$ are
determined by the fundamental matrix $J(a,t)$ consisting of partial
derivatives of $f^t(a)$ with respect to $a$.  Since $a$ and
$f^t$ are complex valued vectors, the real valued matrix
$J(a,t)$ contains partial derivatives evaluated separately with
respect to the real and imaginary parts of $a$, that is
\beq
  J(a,t) = \frac{\partial f^t(a)}{\partial a}
  = \left(\begin{array}{cccc}
  \frac{\partial f^t_{R,1}}{\partial a_{R,1}} &
  \frac{\partial f^t_{R,1}}{\partial a_{I,1}} &
  \frac{\partial f^t_{R,1}}{\partial a_{R,2}} & \\[1ex]
  \frac{\partial f^t_{I,1}}{\partial a_{R,1}} &
  \frac{\partial f^t_{I,1}}{\partial a_{I,1}} &
  \frac{\partial f^t_{I,1}}{\partial a_{R,2}} & \cdots \\[1ex]
  \frac{\partial f^t_{R,2}}{\partial a_{R,1}} &
  \frac{\partial f^t_{R,2}}{\partial a_{I,1}} &
  \frac{\partial f^t_{R,2}}{\partial a_{R,2}} & \\
  & \vdots & & \ddots \end{array}\right)
\label{eq:FundMat}\eeq
where $a_k = a_{R,k} + i a_{I,k}$ and $f^t_k = f^t_{R,k} + i f^t_{I,k}$.
The partial derivatives $\frac{\partial f^t}{\partial a_{R,j}}$
and $\frac{\partial f^t}{\partial a_{I,j}}$ are determined
by solving \refeq{eq:KStan} with initial conditions
$b_k(0) = b_{N-k}(0) = 1 + 0i$ and $b_k(0) = -b_{N-k}(0) = 0 + 1i$,
respectively, for $k = j$ and $b_k(0) = 0$ otherwise.

The stability of a \po\ with period $\period{p}$ is determined by the location
of eivenvalues of $J(a_p,\period{p})$ with respect to the unit circle in the
complex plane.

Because of the translation invariance, the stability of a \rpo\ is
determined by the eigenvalues of the matrix
$\mathbf{g}(\shift_p)\,J(a_p,\period{p})$, where $\mathbf{g}(\shift)$
is the action of the translation operator introduced in
\refeq{eq:shiftFour}, which in real valued representation takes the form
of a block diagonal matrix with the $2\times 2$ blocks
\[
  \left( \begin{array}{cc}
   \cos q_k\shift  & \sin q_k\shift \\
   -\sin q_k\shift & \cos q_k\shift
  \end{array}\right),\ \ k=1,2,\ldots, N/2-1\,.
\]

For an equilibrium solution $a_q$, $f^t(a_q) = a_q$ and so
the fundamental matrix $J(a_q,t)$ can be expressed in terms of the
time independent stability matrix $A(a_q)$ as follows
\[  J(a_q,t) = e^{A(a_q)t}, \]
where
\beq
  A(a_q) = \left.\frac{\partial v}{\partial a}\right|_{a=a_q}.
\label{eq:StabMat}\eeq
Using the real valued representation of \refeq{eq:FundMat},
the partial derivatives of $v(a)$ with respect to the real and imaginary
parts of $a$ are given by
\bea
    \frac{\partial v_k}{\partial a_{R,j}} & = &
    \left(q_k^2 - q_k^4\right)\delta_{kj}
    - iq_k {\cal F}_N[{\cal F}_N^{-1}[a]\otimes {\cal F}_N^{-1}[b_R^{(j)}]]_k\,,
\continue
    \frac{\partial v_k}{\partial a_{I,j}} & = &
    \left(q_k^2 - q_k^4\right)i\delta_{kj}
    - iq_k {\cal F}_N[{\cal F}_N^{-1}[a]\otimes {\cal F}_N^{-1}[b_I^{(j)}]]_k\,,
\label{eq:StabMatElem}
\eea
where $b_R^{(j)}$ and $b_I^{(j)}$ are complex valued vectors such that
$b_{R,k}^{(j)} = b_{R,N-k}^{(j)} = 1 + 0i$ and
$b_{I,k}^{(j)} = -b_{I,N-k}^{(j)} = 0 + 1i$ for $k = j$ and
$b_{R,k}^{(j)} = b_{I,k}^{(j)} = 0$ otherwise.
In terms of $a_{R,k}$ and $a_{I,k}$ we have
\bea
    \frac{\partial v_{R,k}}{\partial a_{R,j}} & = &
    \left(q_k^2 - q_k^4\right)\delta_{kj}
    + q_k (a_{I,k+j} + a_{I,k-j})\,,
\continue
    \frac{\partial v_{R,k}}{\partial a_{I,j}} & = &
    - q_k (a_{R,k+j} - a_{R,k-j})\,,
\label{expanMvar}\\
    \frac{\partial v_{I,k}}{\partial a_{R,j}} & = &
    - q_k (a_{R,k+j} + a_{R,k-j})\,,
\continue
    \frac{\partial v_{I,k}}{\partial a_{I,j}} & = &
    \left(q_k^2 - q_k^4\right)\delta_{kj}
    - q_k (a_{I,k+j} - a_{I,k-j})\,,
\nnu
\eea
where $\delta_{kj}$ is Kronecker delta.

The stability of equilibria is characterized by the sign
of the real part of the eigenvalues of $A(a_q)$.
The stability of a \reqv\ is detemined in the co-moving reference frame, so
the fundamental matrix takes the form $\mathbf{g}(c_q t)\,J(a_q,t)$.  The
stability matrix of a \reqv\ is thus equal to $A(a_q) + c_q \translGen$
where $\translGen = iq_k\delta_{kj}$ is the Lie algebra translation
generator, which in the real-space representation takes the form
$ \translGen = \diag(0, q_1, 0, q_2, \ldots)$.
