% summary.tex
% $Author: siminos $ $Date: 2009-07-28 16:26:34 -0400 (Tue, 28 Jul 2009) $

\section{Summary}
\label{sect:rpo-sum}

In this paper we study the \KS\ flow as a staging ground for
testing dynamical systems approaches to
moderate Reynolds number turbulence in full-fledged
({\em not} a few-modes model),
infinite-dimensional \statesp\ PDE settings\rf{Holmes96},
and present a detailed geometrical portrait of dynamics in the
{\KS} \statesp\ for the $L=22$ system size, the smallest
system size for which this system empirically exhibits
`sustained turbulence.'

Compared to the earlier work
\rf{Christiansen97,LanThesis,lanCvit07,lop05rel},
the main advances here are the new insights in
the role that continuous symmetries,
discrete symmetries,
low-dimensional unstable manifolds of \eqva,
and the connections between \eqva\ play in organizing the flow.
The key new feature of the translationally invariant KS
on a periodic domain
are the attendant continuous families of
\reqva\ (traveling waves) and \rpo s.
We have now understood the preponderance of solutions of
relative type, and lost fear of them:
a large number of unstable \rpo s and \po s has been determined
here numerically.

Visualization of infinite-dimensional
\statesp\ flows, especially in presence of continuous symmetries,
is not straightforward.
At first glance, turbulent dynamics visualized in the \statesp\ appears
hopelessly complex, but under a detailed examination it is
much less so than feared: for strongly dissipative flows (KS, Navier-Stokes)
it is pieced together from low dimensional
local unstable manifolds connected by fast transient interludes.
In this paper we offer two low-dimensional visualizations of such
flows: (1) projections onto 2- or 3-dimensional,
PDE representation independent
dynamically invariant frames, and
(2) projections onto
the physical, symmetry invariant but time-dependent
energy transfer rates.

\Rpo s require a reformulation of the periodic orbit
theory\rf{Cvi07}, as well as a rethinking of the dynamical
systems approaches to constructing symbolic dynamics,
outstanding problems that we hope to address in near future%
\rf{SCD09b,SiminosThesis}.
What we have learned from the $L=22$ system  is that many of
these \rpo s appear organized by the unstable manifold of
$\EQV{2}$, closely following the homoclinic loop formed
between $\EQV{2}$ and $\Shift_{1/4}\EQV{2}$.

In the spirit of the parallel studies of boundary shear flows\rf{HaKiWa95},
the {\KS} $L=22$ system size was chosen as the smallest
system size for which KS empirically exhibits
`sustained turbulence.'
This is convenient both for
the analysis of the \statesp\ geometry, and for the numerical reasons,
 but the price is high - much of the
observed dynamics is specific to this unphysical, externally
imposed periodicity. What needs to be
understood is the nature of \eqv\ and \rpo\ solutions in the
$L \to \infty$ limit, and the structure of the $L = \infty$ periodic orbit
theory.

In summary, {\KS} (and \pCf, see \refref{GHCW07})  \eqva, \reqva, \po s and
\rpo s embody Hopf's vision\rf{hopf48}: together they form the
 repertoire of recurrent spatio-temporal
patterns explored by turbulent dynamics.
