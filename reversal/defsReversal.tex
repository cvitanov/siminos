% siminos/reversal/defsReversal.tex
% $Author: predrag $ $Date: 2021-12-12 22:57:47 -0500 (Sun, 12 Dec 2021) $

% started with % siminos/kittens/defsKittens.tex

%%%%%%%%%%%%%% CL18 specific %%%%%%%%%%%%%%%%%%%%%%%%%%%%
\newcommand{\fundPip}{fundamental parallelepiped}
\newcommand{\FundPip}{Fundamental parallelepiped}

\newcommand{\SLn}[2]{\ensuremath{\textrm{SL}_{#1}(#2)}}
\newcommand{\class}[1]{\ensuremath{{\cal C}_{#1}}}
\newcommand{\PP}{P}    % projection operator,          Predrag  2019-10-16
\newcommand{\lattice}{\ensuremath{\mathcal{L}}}
    % PC 2021-08-05: from https://tex.stackexchange.com/questions/86569/creating-uniformly-sized-boxes-around-text
\newcommand{\sitebox}[1]{%
        {% open a group for a local setting
   \setlength{\fboxsep}{-2\fboxrule}% the rule will be inside the box boundary
   \fbox{\hspace{1.2pt}\strut$#1$\hspace{1.2pt}}% print the box, with some padding at the left and right
        }% close the group
                        } %%%%%%%%%%%%%%%%%%%%%%%%%%%%%


%%%%%% Referencing equations etc, Nonlinearity style only %%%%%%%
\newcommand{\conf}{\ensuremath{x}} %Configuration space coordinate
\newcommand{\Fu}{\tilde{u}}
\newcommand{\twot}{periodic orbit} %Predrag experiment 2020-03-18
\newcommand{\twots}{periodic orbits}
\newcommand{\twoT}{Periodic orbit}
\newcommand{\twoTs}{Periodic orbits}
%\newcommand{\twot}{invariant 2-torus}
%\newcommand{\twots}{invariant 2-tori}
%\newcommand{\twoT}{Invariant 2-torus}
%\newcommand{\twoTs}{Invariant 2-tori}
\newcommand{\dtor}{invariant $d$-torus} % Predrag, to distinguish from a PO
\newcommand{\dtors}{invariant $d$-tori}
\newcommand{\dTor}{Invariant $d$-torus}
\newcommand{\dTors}{Invariant $d$-tori}
\newcommand{\lattstate}{lattice state}
\newcommand{\Lattstate}{Lattice state}
\newcommand{\orbit}{orbit} % Predrag 2021-08-10: retired ChaosBook {prime cycle}
\newcommand{\Orbit}{Orbit}
\newcommand{\templatt}{temporal cat}     % Predrag, experimental
\newcommand{\tempLatt}{Temporal cat}     % Predrag, experimental
\newcommand{\spt}{spatiotemporal}
\newcommand{\Spt}{Spatiotemporal}
\newcommand{\catlatt}{spatiotemporal cat}     % Predrag, experimental
\newcommand{\catLatt}{Spatiotemporal cat}     % Predrag, experimental
\newcommand{\Henon}{H\'enon}
\newcommand{\HenonMap}{H\'enon map}
\newcommand{\henlatt}{temporal H{\'e}non}     % Predrag, experimental
\newcommand{\Henlatt}{Temporal H{\'e}non}     % Predrag, experimental
\newcommand{\henSTlatt}{{\spt} H{\'e}non}     % Predrag, experimental
\newcommand{\HenSTlatt}{{\Spt} H{\'e}non}     % Predrag, experimental
\newcommand{\bcs}{bc's}
\newcommand{\PV}{Percival-Vivaldi}
\newcommand{\AW}{Adler-Weiss}
\newcommand{\GO}{Gutkin-Osipov}
\newcommand{\sPe}{screened Poisson equation}
\newcommand{\SPe}{Screened Poisson equation}
\newcommand{\zetatop}{\ensuremath{1/\zeta_{\mbox{\footnotesize top}} }}
\newcommand{\msr}{\ensuremath{\rho}}                % measure
%\newcommand{\Msr}{{f(}}                 % Boris frequency AKA measure
\newcommand{\Msr}{{\mu}}                % Predrag, experimental: measure
\newcommand{\dMsr}{{d\mu}}              % measure infinitesimal
\newcommand{\SRB}{{\rho_0}}             % natural measure
\newcommand{\coord}{q}        % configuration space q coordinate, dual to p
\newcommand{\jMps}{\ensuremath{J}}   % jacobian matrix, phase space/state space
% \newcommand{\jMps}{\ensuremath{{\bf J}}}  % bold fundamental matrix phase space
\newcommand{\monodromy}{\ensuremath{M}}   % monodromy matrix, full Poincare cut
\newcommand{\jMorb}{{\ensuremath{\mathcal{\jMps}}}}
\newcommand{\jMat}{\mathbb{\jMps}}

%\newcommand{\ssp}{\ensuremath{x}}               % state space point
\newcommand{\ssp}{\ensuremath{\phi}}             % kittens lattice site field
\newcommand{\cssp}{\ensuremath{\tilde{\phi}}}    % Complex state space point
\newcommand{\source}{{J}}
\newcommand{\sourceFT}{{\tilde{J}}}
\newcommand{\derSource}{\frac{d~}{d\source}}
\newcommand{\derSourceFT}{\frac{d~}{d\sourceFT}}


%%%%%% Boris definitions
\newcommand{\Mm}{\ensuremath{\mathsf{M}}}
%\newcommand{\Xx}{\ensuremath{\mathsf{X}}}      % Boris
\newcommand{\Xx}{\ensuremath{\mathsf{\Phi}}}      % kittens lattice field
\newcommand{\xx}{\ensuremath{\mathsf{\boldsymbol{\phi}}}} %{\bphi}}}      % kittens lattice field
\newcommand{\Aa}{\ensuremath{\bar{\A}}}
\newcommand{\A}{\ensuremath{\mathcal{A}}}       % alphabet
\newcommand{\Ai}{\ensuremath{\mathcal{A}_0}}    % alphabet interior (was inner)
\newcommand{\Ae}{\ensuremath{\mathcal{A}_1}}    % alphabet exterior (was outer)
\newcommand{\B}{\mathcal{B}}
\newcommand{\D}{\mathcal{D}}
\newcommand{\R}{\ensuremath{\mathcal{R}}}
\newcommand{\Pol}{\mathcal{P}}                  % Boris
\newcommand{\gd}{\mathsf{g}}
% Dirichlet Green's function
\newcommand{\gp}{\mathsf{g}^0}
% Periodic Green's function
%\newcommand{\m}{\ensuremath{\mathsf{m}}}     % Boris
\newcommand{\m}{\ensuremath{m}}     % Predrag experimental 2016-11-08
%\newcommand{\q}{\ensuremath{\mathsf{m}}}     % Boris
\newcommand{\q}{\ensuremath{q}}     % Predrag experimental 2016-11-08
\newcommand{\reals}{\mathbb{R}}
\newcommand{\naturals}{\mathbb{N}}
\newcommand{\integers}{\ensuremath{\mathbb{Z}}}
\newcommand{\Zz}{\ensuremath{\mathbb{Z}^2}}
%\newcommand{\Det}{\mbox{\rm Det}\,}
%\newcommand{\ZNT}{\mathbb{Z}^2_{\mbox{\scriptscriptstyle{NT}}}}
\newcommand{\ZLT}{\mathbb{Z}^2_{\scriptscriptstyle\mathrm{LT}}}
\newcommand{\MmR}{\Mm_\R}
\newcommand{\length}{\mathcal{D}}
\newcommand{\deltaX}{\ensuremath{{\delta x}}}       %trajectory displacement
%! Undefined control sequence.
%   \example
\newcommand{\pde}{\partial}
\newcommand{\vel}{\ensuremath{v}}   % state space velocity
\newcommand{\Idg}{\ensuremath{\mathbf{1}}}
\newcommand{\Cn}[1]{\ensuremath{\textrm{C}_{#1}}}
\newcommand{\jacobianOrb}{orbit Jacobian matrix}
\newcommand{\JacobianOrb}{Orbit Jacobian matrix}
\newcommand{\jacobianOrbs}{orbit Jacobian matrices}
\newcommand{\JacobianOrbs}{Orbit Jacobian matrices}
\newcommand{\HillDet}{Hill determinant}
\newcommand{\ecs}{exact coherent structure}
\newcommand{\Ecs}{Exact coherent structure}
%%%%%%%%%%%%%%%%%%%%%%%%%%%%%%%%%%%%%%%%%%%%%%%%%%%%%%%%%%%%%%%%%

\newcommand{\NBBpost}[2]{\item[#1 Burak] {#2}}
\newcommand{\PCpost}[2]{\item[#1 Predrag] {#2}}
\newcommand{\BGpost}[2]{\item[#1 Boris] {#2}}
\newcommand{\AKSpost}[2]{\item[#1 Adrien] {#2}}
\newcommand{\RJpost}[2]{\item[#1 Rana] {#2}}
\newcommand{\LHpost}[2]{\item[#1 Li Han] {#2}}
\newcommand{\MNGpost}[2]{\item[#1 Matt] {#2}}
\newcommand{\XDpost}[2]{\item[#1 Xiong] {#2}}
\newcommand{\HLpost}[2]{\item[#1 Han] {#2}}

\ifboyscout %%%%%%%% DISPLAY COMMENTS IN THE TEXT %%%%%%%%%%%%%%%%%%%%
            %%%%%%%% turn on labeling of equations on margins %%%%%%%%
    % also search the text for lines starting with %%  to
    % locate various internal comments, recent edits etc.
    \typeout{============ COMMENTED =====}
  \newcommand{\PublicPrivate}[2]
    {\marginpar{\color{blue}$\Downarrow$\footnotesize PRIVATE}%
    {\color{blue}#2}%
    \marginpar{\color{blue}$\Uparrow$\footnotesize PRIVATE}}
  \newcommand{\PC}[2]% {$\footnotemark\footnotetext{Predrag #1: #2}$}
                     {\begin{quote}\PCedit{[#1 Predrag] \small #2}\end{quote}}
  \newcommand{\PCedit}[1]{{\color{magenta}#1}}
  \newcommand{\HL}[2]%{$\footnotemark\footnotetext{Han: #1}$}
                     {\begin{quote}\HLedit{[#1 Han] \small #2}\end{quote}}
  \newcommand{\HLedit}[1]{{\color{blue}#1}}
  \newcommand{\BG}[2]% {$\footnotemark\footnotetext{Boris #1: #2}$}
                     {\begin{quote}\BGedit{[#1 Boris] \small #2}\end{quote}}
  \newcommand{\BGedit}[1]{{\color{red}#1}}
  \newcommand{\AKS}[2]{$\footnotemark\footnotetext{Adrien #1: #2}$}
  \newcommand{\AKSedit}[1]{{\color{green}#1}}
  \newcommand{\RJ}[2]%{$\footnotemark\footnotetext{Rana #1: #2}$}
                     {\begin{quote}\RJedit{[#1 Rana] \small #2}\end{quote}}
  \newcommand{\RJedit}[1]{{\color{green}#1}}
  \newcommand{\MNG}[2]{$\footnotemark\footnotetext{Matt #1: #2}$}
  \newcommand{\MNGedit}[1]{{\color{red}#1}}
  \newcommand{\LH}[2]%{$\footnotemark\footnotetext{Rana #1: #2}$}
                     {\begin{quote}\LHedit{[#1 Li] \small #2}\end{quote}}
  \newcommand{\LHedit}[1]{{\color{green}#1}}
  \newcommand{\Private}[1]{{\color{blue}#1}}
  \newcommand{\toCB}{\marginpar{\footnotesize 2CB}}  % to compare with ChaosBook
  \newcommand{\inCB}{\marginpar{\footnotesize now in CB}} % entered in ChaosBook
  \newcommand{\CBlibrary}[1]
             {\href{http://ChaosBook.org/library/#1.pdf} { (click here)}}
\else % drop comments
      % do not turn on labeling of equations on margins
  \typeout{============ UNCOMMENTED =====}
  \newcommand{\PublicPrivate}[2]{#1}
  \newcommand{\PC}[2]{}{}
  \newcommand{\PCedit}[1]{#1}
  \newcommand{\HL}[2]{}
  \newcommand{\HLedit}[1]{#1}
  \newcommand{\BG}[2]{}{}
  \newcommand{\BGedit}[1]{#1}
  \newcommand{\AKS}[2]{}{}
  \newcommand{\AKSedit}[1]{#1}
  \newcommand{\RJ}[2]{}{}
  \newcommand{\RJedit}[1]{#1}
  \newcommand{\LH}[2]{}{}
  \newcommand{\LHedit}[1]{#1}
  \newcommand{\Private}[1]{}
  \newcommand{\toCB}{}
  \newcommand{\inCB}{}
  \newcommand{\CBlibrary}[1]{}
\fi  %%%%%%%%%%%% END OF ON/OFF COMMENTS SWITCH %%%%%%%%%%%%%%%%%%%%

% online PDF \toChaosBook{}{} section pointer       % Predrag 2019-12-07
%                        %#1 = PDF file section, #2 = section title
\newcommand{\toChaosBook}[2]
{\HREF{http://ChaosBook.org/chapters/ChaosBook.pdf\##1}{ChaosBook\, #2}}

%%%%%%%%%%%%%%%%%%%%%%% set up submission, hyperlinked %%%%%%%%%%%
\ifsubmission % keep homepage flexible:
    \newcommand{\wwwcb}[1]{{\tt ChaosBook.org#1}}
    \newcommand{\weblink}[1]{{\tt #1}}
    \newcommand{\HREF}[2]{{#2}}
    \newcommand{\arXiv}[1]{ {\tt arXiv:#1}}
\else  %% hyperlinked pdf
    \newcommand{\wwwcb}[1]{
                  {\tt \href{http://ChaosBook.org#1}
              {ChaosBook.org#1}}}
    \newcommand{\weblink}[1]{{\tt \href{http://#1}{#1}}}
    \newcommand{\HREF}[2]{{\href{#1}{#2}}}
    \newcommand{\arXiv}[1]{
              {\tt \href{http://arXiv.org/abs/#1}{arXiv:#1}}}
\fi

%%%%%%%%%%%%%%%%%%%%%% QUOTATIONS %%%%%%%%%%%%%%%%%%%%%%%%%%%%%%%%%%%%%%
%
%  the learned/witty quotes at the chapter and section headings
%
\newsavebox{\bartName}
\newcommand{\bauthor}[1]{\sbox{\bartName}{\parbox{\textwidth}{\vspace*{0.8ex}
       %\hspace*{\fill}
       \hspace{2em}---\small\noindent #1}}}
\newenvironment{bartlett}{\hfill\begin{minipage}[t]{0.65\textwidth}\small}%
{\hspace*{\fill}\nolinebreak[1]\usebox{\bartName}\vspace*{1ex}\end{minipage}}
%

%%%%%%%%%%%%%%% EQUATIONS %%%%%%%%%%%%%%%%%%%%%%%%%%%%%%%
\newcommand{\beq}{\begin{equation}}
\newcommand{\continue}{\nonumber \\ }
\newcommand{\nnu}{\nonumber}
\newcommand{\eeq}{\end{equation}}
\newcommand{\ee}[1] {\label{#1} \end{equation}}
\newcommand{\bea}{\begin{eqnarray}}
\newcommand{\ceq}{\nonumber \\ & & }
\newcommand{\eea}{\end{eqnarray}}
\newcommand{\barr}{\begin{array}}
\newcommand{\earr}{\end{array}}

%%%%%%%%%%%%%%% REFERENCING EQUATIONS ETC, Nonlinearity style only %%%%%%%
\newcommand{\rf}     [1] {~\cite{#1}}
\newcommand{\refref} [1] {\cite{#1}}
\newcommand{\refRef} [1] {\cite{#1}}
\newcommand{\refrefs}[1] {\cite{#1}}
\newcommand{\refRefs}[1] {\cite{#1}}
\newcommand{\refeq}  [1] {(\ref{#1})}
            % in amstex, \eqref is predefined and better than \refeq
\newcommand{\refeqs} [2]{(\ref{#1}--\ref{#2})}
\newcommand{\reffig} [1] {figure~\ref{#1}}
\newcommand{\reffigs} [2] {figures~\ref{#1} and~\ref{#2}}
\newcommand{\refFig} [1] {Figure~\ref{#1}}
\newcommand{\refFigs} [2] {Figures~\ref{#1} and~\ref{#2}}
\newcommand{\reftab} [1] {table~\ref{#1}}
\newcommand{\refTab} [1] {Table~\ref{#1}}
\newcommand{\reftabs}[2] {tables~\ref{#1} and~\ref{#2}}
\newcommand{\refsect}[1] {section~\ref{#1}}
\newcommand{\refsects}[2] {sections~\ref{#1} and \ref{#2}}
\newcommand{\refSect}[1] {Section~\ref{#1}}
\newcommand{\refSects}[2] {Sections~\ref{#1} and \ref{#2}}
\newcommand{\refsecttosect}[2] {Sections~\ref{#1} to~\ref{#2}}
\newcommand{\refSecttosect}[2] {Sections~\ref{#1} to~\ref{#2}}
\newcommand{\refappe}[1] {\ref{#1}} % Nonlinearity only
\newcommand{\refAppe}[1] {\ref{#1}} % Nonlinearity only
\newcommand{\refpage}[1] {page~\pageref{#1}}
\newcommand{\refExam}[1] {Example~\ref{#1}}

%%%%%%%%%% periods: %%%%%%%%%%%%%%%%%%%%%%%%%%%%
\newcommand{\period}[1]{{\ensuremath{T_{#1}}}}         %continuous cycle period
\newcommand{\cl}[1]{{\ensuremath{n_{#1}}}}   % discrete length of a cycle, Predrag
%\newcommand{\cl}[1]{|#1|}  % the length of a periodic orbit, Ronnie
\newcommand{\speriod}[1]{{\ensuremath{L_{#1}}}}  %spatial period
\newcommand{\tilt}[1]{{\ensuremath{S_{#1}}}}  %relative shift
\newcommand{\LTS}[3]{{\ensuremath{[\speriod{#1}\!\times\!\period{#2}]_\tilt{#3}}}}
\newcommand{\BravCell}[3]{{\ensuremath{[#1\!\times\!#2]_{#3}}}}

%%%%%%%%%%%%%%% ChaosBook Abbreviations %%%%%%%%%%%%%%%%%%%%%%%%

\newcommand{\statesp}{state space}
\newcommand{\Statesp}{State space}
\newcommand{\stateDsp}{state-space}
\newcommand{\StateDsp}{State-space}
%%%%%% BORIS convention start %%%%%%%%%%%%%%%%%%%%%
%\renewcommand{\statesp}{phase space}
%\renewcommand{\Statesp}{Phase space}
%\renewcommand{\stateDsp}{phase-space}
%\renewcommand{\StateDsp}{Phase-space}
%%%%%% BORIS convention end   %%%%%%%%%%%%%%%
\newcommand{\fixedpnt}{fixed point}
\newcommand{\Fixedpnt}{fixed point}
\newcommand{\jacobian}{Jacobian}        % determinant
% \newcommand{\jacobianM}{fundamental matrix} % no known standard name?
% \newcommand{\jacobianMs}{fundamental matrices}  %
% \newcommand{\JacobianM}{Fundamental matrix} %
% \newcommand{\JacobianMs}{Fundamental matrices}  %
\newcommand{\jacobianM}{Jacobian matrix}  % back to Predrag's name 20oct2009
\newcommand{\jacobianMs}{Jacobian matrices}   % matrices
\newcommand{\JacobianM}{Jacobian matrix} %
\newcommand{\JacobianMs}{Jacobian matrices}  %
\newcommand{\FloquetM}{Floquet matrix} % specialized to periodic orb
\newcommand{\FloquetMs}{Floquet matrices}  %
% \newcommand{\stabmat}{matrix of variations}   % Arnold, says Vattay
\newcommand{\stabmat}{stability matrix}     % stability matrix, velocity gradients
\newcommand{\Stabmat}{Stability matrix}     % Stability matrix
\newcommand{\stabmats}{stability matrices}
\newcommand{\monodromyM}{monodromy matrix} % monodromy matrix, Poincare cut
\newcommand{\MonodromyM}{Monodromy matrix} % monodromy matrix, Poincare cut
\newcommand{\FPoper}{Perron-Frobenius oper\-ator} % Pesin's ordering
\newcommand{\FP}{Perron-Frobenius}
\newcommand{\dzeta}{dyn\-am\-ic\-al zeta func\-tion}
\newcommand{\Dzeta}{Dyn\-am\-ic\-al zeta func\-tion}
\newcommand{\tzeta}{top\-o\-lo\-gi\-cal zeta func\-tion}
\newcommand{\Tzeta}{Top\-o\-lo\-gi\-cal zeta func\-tion}
%\newcommand{\tzeta}{Artin-Mazur zeta func\-tion} %alternative to topological
\newcommand{\Gt}{Gutz\-willer trace formula}
\newcommand{\Fd}{spec\-tral det\-er\-min\-ant}
%\newcommand{\fd}{spec\-tral det\-er\-min\-ant} %in many articles
\newcommand{\FD}{Spec\-tral det\-er\-min\-ant}
\newcommand{\cycForm}{cycle averaging formula}
\newcommand{\CycForm}{Cycle averaging formula}
\newcommand{\pdes}{partial differential equations}
\newcommand{\Pdes}{Partial differential equations}
\newcommand{\dof}{dof}         % Hamiltonian deegree of freedom
% \newcommand{\dof}{deegree of freedom}
\newcommand{\nws}{non--wandering set}
\newcommand{\NWS}{\ensuremath{\Omega}}     % symbol for the non--wandering set
\newcommand{\MarkGraph}{Transition graph} % following Yorke
\newcommand{\markGraph}{transition graph} % following Yorke
% \newcommand{\MarkGraph}{Markov graph}
\newcommand{\admissible}{admissible}
\newcommand{\Admissible}{Admissible}
\newcommand{\inadmissible}{inadmissible}
% \newcommand{\obser}{\ensuremath{a}} % CBook: an observable from statespace to R^n
\newcommand{\obser}{\ensuremath{A}} % Boris: an observable from phase space to R^n

\newcommand\map{f}                  % other people like \phi's etc
\newcommand{\ExpaEig}{\ensuremath{\Lambda}}
\newcommand{\Lyap}{\ensuremath{\lambda}}            %Lyapunov exponent
\newcommand{\jEigvec}[1][]{\ensuremath{{\bf e}^{(#1)}}} % right jacobiam eigenvector
\newcommand{\oneMinJ}[1]
           {\left|\det\!\left(\matId-\monodromy_p^{#1}\right)\right|}
\newcommand{\Lop}{\ensuremath{\mathcal{L}}}       % evolution operator
\newcommand{\Det}{\mbox{\rm Det}\,}
\newcommand{\mod}{\mbox{\rm mod}\,}
\newcommand\xInit{{\ssp_0}}        %initial x
\newcommand{\cycle}[1]{{\ensuremath{\overline{#1}}}}
\newcommand{\block}[1]{\ensuremath{#1}} % PC 07sep2008: conflict with beamer
\newcommand{\brick}{block}
\newcommand{\Brick}{Block}
\newcommand{\prune}[1]{\ensuremath{\_{#1}\_}}        % fits into math env.
\newcommand{\biinf}[2]{\ensuremath{\cdots#1.#2\cdots}}
\newcommand{\rctngl}[2]{\ensuremath{[#1.#2]}}
\newcommand{\Spast}{\ensuremath{S^\textrm{-}}}       % past itenerary
\newcommand{\Sfuture}{\ensuremath{S^\textrm{\scriptsize +}}} % future itenerary
\newcommand{\Sbiinf}{\ensuremath{S}}             % biinf. itenerary
\newcommand{\Ssym}[1]{{\ensuremath{m_{#1}}}}    % Boris
% \newcommand{\Ssym}[1]{{\ensuremath{s_{#1}}}}  % ChaosBook
\newcommand{\AdmItnr}{\Sigma}      % set of admissible itineraries
% \newcommand{\action}{W}                 % PC 07jan2018
  \newcommand{\action}{S}               % Boris
%\newcommand{\genF}{F}                  % Li and Tomsovic generating function
%\newcommand{\genF}{F}                  % Li and Tomsovic generating function
\newcommand{\genF}{L}                   % McKay generating function
\newcommand{\hopMat}{\mathbf{\sigma}} % Dec 2012 experimental
\newcommand{\hop}{\sigma} % Dec 2012 experimental
\newcommand{\shiftOp}{shift operator}  % was \stepOp
\newcommand{\ShiftOp}{Shift operator}
\newcommand{\Laplacian}{\square}

%%%%%%%%%%%%%%% Lorentz gas section %%%%%%%%%%%%%%%%%%%%%%%%%%%%%%%%
\newcommand\flow[2]{{f^{#1}(#2)}}
\newcommand\hflow[2]{{\hat{f}^{#1}(#2)}}
\newcommand\hx{\hat \ssp}

%%%%%%%%%%%%%%% relative periodic orbits: %%%%%%%%%%%%%%%%%%%%%%%%%%%%
\newcommand{\po}{periodic orbit}
\newcommand{\Po}{Periodic orbit}
\newcommand{\rpo}{rela\-ti\-ve periodic orbit}
\newcommand{\Rpo}{Rela\-ti\-ve periodic orbit}
\newcommand{\ppo}{pre-periodic orbit}
\newcommand{\Ppo}{Pre-periodic orbit}
\newcommand{\eqv}{equi\-lib\-rium}
\newcommand{\Eqv}{Equi\-lib\-rium}
\newcommand{\eqva}{equi\-lib\-ria}
\newcommand{\Eqva}{Equi\-lib\-ria}
\newcommand{\reqv}{rela\-ti\-ve equi\-lib\-rium}
%   \newcommand{\reqv}{travelling wave}
\newcommand{\Reqv}{Rela\-ti\-ve equi\-lib\-rium}
%   \newcommand{\Reqv}{travelling wave}
\newcommand{\reqva}{rela\-ti\-ve equi\-lib\-ria}
\newcommand{\Reqva}{Rela\-ti\-ve equi\-lib\-ria}
\newcommand{\equilibrium}{equi\-lib\-rium}
\newcommand{\equilibria}{equi\-lib\-ria}
\newcommand{\Equilibria}{Equi\-lib\-ria}
% \newcommand{\equilibrium}{steady state}
% \newcommand{\equilibria}{steady states}
% \newcommand{\Equilibria}{Steady states}

\newcommand{\sym}{{s}}
\newcommand{\nsym}{{n_s}}
\newcommand{\asym}{{a}}
\newcommand{\nasym}{{n_a}}
\newcommand{\symf}{{\tilde s}}
\newcommand{\nsymf}{n_{\tilde s}}

%%%%%%%%%%%%%%% SECTIONS, SLICES %%%%%%%%%%%%%%%%%%%%%%%%%%%%%%%%%

\newcommand{\expct}    [1]{\langle {#1} \rangle}
\newcommand{\spaceAver}[1]{\langle {#1} \rangle}
%\newcommand{\expct}    [1]{\left\langle {#1} \right\rangle}
%\newcommand{\spaceAver}[1]{\left\langle {#1} \right\rangle}
\newcommand{\timeAver} [1]{\overline{#1}}
\newcommand{\norm}[1]{\left\Arrowvert \, #1 \, \right\Arrowvert}
\newcommand{\pS}{\ensuremath{\mathcal{M}}}          % symbol for state space
\newcommand{\Poincare}{Poincar\'e }
\newcommand{\PoincSec}{Poincar\'e section}
\newcommand{\PoincMap}{return map} %\Poincare\ return map
% \newcommand{\equivariantsp}{equivariant {\statesp}} % full state space
% \newcommand{\Equivariantsp}{Equivariant {\statesp}}
% \newcommand{\reducedsp}{orbit space}
% \newcommand{\Reducedsp}{Orbit space}
\newcommand{\reducedsp}{reduced state space}
\newcommand{\Reducedsp}{Reduced state space}
\newcommand{\fixedsp}{fixed-point subspace}
\newcommand{\Fixedsp}{Fixed-point subspace}
\newcommand{\mslices}{method of slices}
\newcommand{\Mslices}{Method of slices}
\newcommand{\mframes}{method of moving frames}
\newcommand{\Mframes}{Method of moving frames}
\newcommand{\templates}{templates} % {slice-fixing point} % {reference state}
\newcommand{\movframe}{moving frame}
\newcommand{\movFrame}{Moving frame}
\newcommand{\comovframe}{comoving frame}
\newcommand{\comovFrame}{Comoving frame}
\newcommand{\mconn}{method of \comovframe s}
\newcommand{\Mconn}{Method of \comovframe s}
\newcommand{\fFslice}{first Fourier mode slice}
\newcommand{\FFslice}{First Fourier mode slice}
\newcommand{\poincBord}{section border}
\newcommand{\PoincBord}{Section border}
% \newcommand{\poincBord}{\PoincSec\ border}
% \newcommand{\PoincBord}{\PoincSec\ border}
% \newcommand{\poincBord}{border of transversality}
\newcommand{\template}{template} % {slice-fixing point} % {reference state}
\newcommand{\pSRed}{\ensuremath{\hat\mathcal{M}}} % reduced state space Jan 2012
%\newcommand{\pSRed}{\ensuremath{\bar\mathcal{M}}} % reduced state space
\newcommand{\sspRed}{\ensuremath{\hat{\ssp}}}    % reduced state space point Jan 2012
% \newcommand{\sspRed}{\ensuremath{y}}    % reduced state space point, experiment
% \newcommand{\sspRed}{\ensuremath{\bar{x}}}    % reduced state space point
\newcommand{\csspRed}{\ensuremath{\hat{u}}}      % Symmetry reduced complex state space point
\newcommand{\velRed}{\ensuremath{\hat{\vel}}}    % ES reduced state space velocity Jan 2012
% \newcommand{\velRed}{\ensuremath{\bar{v}}}    % PC reduced state space velocity
% \newcommand{\velRed}{\ensuremath{u}}    % ES reduced state space velocity
\newcommand{\MvarRed}{\ensuremath{\hat{\Mvar}}}  %Reduced stability matrix
\newcommand{\velRel}{\ensuremath{c}}    % relative state or phase velocity
\newcommand{\phaseVel}{phase velocity}      % pipe slicing
\newcommand{\phaseVels}{phase velocities}   % pipe slicing
\newcommand{\PhaseVel}{Phase velocity}      % pipe slicing
\newcommand{\PhaseVels}{Phase velocities}   % pipe slicing

\newcommand{\slicep}{{\ensuremath{\sspRed'}}}   % slice-fixing point Jan 2012
% \newcommand{\slicep}{{\ensuremath{y'}}}   % slice-fixing point, experimental
% \newcommand{\slicep}{\ensuremath{\ssp'}}   % slice-fixing point
%\newcommand{\sliceTan}[1]{\ensuremath{t_{#1}(y')}}    % tangent at slice-fixing, experimental
\newcommand{\sliceTan}[1]{\ensuremath{t'_{#1}}}    % group orbit tangent at slice-fixing
\newcommand{\groupTan}{\ensuremath{t}}    % group orbit tangent

\newcommand{\zeit}{\ensuremath{t}}  %time variable Ashley
\newcommand{\sspSing}{\ensuremath{\ssp^\ast}} 	% inflection point
\newcommand{\sspRSing}{\ensuremath{\sspRed^\ast}} 	% inflection point, reduced space

\newcommand{\braket}[2]
		   {\langle{#1}\vphantom{#2}|\vphantom{#1}{#2}\rangle}
\newcommand{\bra}[1]{\langle{#1}\vphantom{ }|}
\newcommand{\ket}[1]{|\vphantom{}{#1}\rangle}
\newcommand{\dual}[1]{{#1}^\dag}
\newcommand{\transp}[1]{{#1}{}^\top}

%%%%%%%%%%%%%%% Group theory %%%%%%%%%%%%%%%%%%%%%%
%\newcommand{\Group}{\ensuremath{\Gamma}}    % Siminos Lie group
\newcommand{\Group}{\ensuremath{G}}         % Predrag Lie or discrete group
\newcommand{\LieEl}{\ensuremath{g}}  % Predrag group element
%\newcommand{\Lg}{\mathfrak{a}}             % Siminos Lie algebra generator
\newcommand{\Lg}{\ensuremath{\mathbf{T}}}   % Predrag Lie algebra generator
\newcommand{\gSpace}{\ensuremath{{\bf \phi}}}   % MA group rotation parameters
% \newcommand{\gSpace}{\ensuremath{{\bf \theta}}}   % PC group rotation parameters
\newcommand {\id}{{\ \hbox{{\rm 1}\kern-.62em\hbox{\rm 1}}}} % Roberto
\newcommand{\matId}{\ensuremath{{\bf 1}}}       % matrix identity
\newcommand{\unit}{{\bf 1}}                     %% in lattFT.tex %%
                                                %experiment with {\bf 1\!\!\!1}


%%%%%%%% Siminos macros %%%%%%%%%%%%%%%%%%%%%%%%%%%%%%
\newcommand{\Rls}[1]{\ensuremath{\mathbb{R}^{#1}}}
\newcommand{\ii}{\ensuremath{\mathrm{i}}} % sqrt{-1}
\newcommand{\Un}[1]{\ensuremath{\textrm{U}(#1)}}         % in DasBuch
\newcommand{\SUn}[1]{\ensuremath{\textrm{SU}(#1)}}         % in DasBuch
\newcommand{\On}[1]{\ensuremath{\textrm{O}(#1)}}
\newcommand{\SOn}[1]{\ensuremath{\textrm{SO}(#1)}}         % in DasBuch
\newcommand{\Spn}[1]{\ensuremath{\textrm{Sp}(#1)}}         % in DasBuch
\newcommand{\Dn}[1]{\ensuremath{\textrm{D}_{#1}}}              % in DasBuch
\newcommand{\Refl}{\ensuremath{\sigma}}            % in DasBuch
\newcommand{\stab}[1]{\ensuremath{G_{#1}}}
\newcommand{\shift}{\ensuremath{r}}
\newcommand{\Fix}[1]{\ensuremath{\mathrm{Fix}\left(#1\right)}}

%%%%%%%%%%%%%% ks.tex specific %%%%%%%%%%%%%%%%%%%%%%%%%%%%
\newcommand{\KS}{Ku\-ra\-mo\-to-Siva\-shin\-sky}
\newcommand{\KSe}{Ku\-ra\-mo\-to-Siva\-shin\-sky equation}
\newcommand{\NS}{Navier-Stokes}
\newcommand{\NSe}{Navier-Stokes equation}
\newcommand{\pCf}{plane Couette flow}
\newcommand{\PCf}{Plane Couette flow}
\newcommand{\dmn}{-dimensional}  %  experimental 220ct2009
%\newcommand{\dmn}{\ensuremath{d}}  %  n-dimensional
%\newcommand{\dmn}{\ensuremath{\!-\!d}}  %  n-dimensional
\newcommand{\expctE}{\ensuremath{E}}    % E space averaged
\newcommand{\tildeL}{\ensuremath{\tilde{L}}}
\newcommand{\EQV}[1]{\ensuremath{EQ_{#1}}} %experimental
% \newcommand{\EQV}[1]{\ensuremath{q_{#1}}} %ChaosBook
% \newcommand{\EQV}[1]{\ensuremath{E_{#1}}} %Ruslan
% E_0: u = 0 - trivial equilibrium
% E_1,E_2,E_3, for 1,2,3-wave equilibria
\newcommand{\REQV}[2]{\ensuremath{TW_{#1#2}}} % #1 is + or -
% TW_1^{+,-} for 1-wave traveling waves (positive and negative velocity).
\newcommand{\PO}[1]{\ensuremath{PO_{#1}}}
% PO_{period to 2-4 significant digits} - periodic orbits
\newcommand{\RPO}[1]{\ensuremath{\overline{rpo}_{#1}}} % Xiang experimental
%\newcommand{\RPO}[1]{\ensuremath{RPO_{#1}}}
% RPO_{period to 2-4 significant digits} - relative PO.  We use ^{+,-}
% to distinguish between members of a reflection-symmetric pair.
% \newcommand{\PPO}[1]{\ensuremath{PPO_{#1}}}
\newcommand{\PPO}[1]{\ensuremath{\overline{ppo}_{#1}}} % Xiang experimental
% Gibson likes:
\newcommand{\tEQ}{\ensuremath{{EQ}}}

%%%%%%%%%%%%%%  Abbreviations %%%%%%%%%%%%%%%%%%%%%%%%%%%%%%%%%%%%%%%%
%%% APS (American Physiology Society, it seems) style:
%%%     Latin or foreign words or phrases should be roman, not italic.
%%%     Insert a `hard' space after full points
%%%                                         that do not end sentences.

\newcommand{\etc}{{etc.}}       % APS
%\newcommand{\etal}{{\em et al.}}    % etal in italics, APS too
\newcommand{\ie}{{i.e.}}        % APS
\newcommand{\cf}{{\em cf.\ }}     % APS
\newcommand{\eg}{{e.g.\ }}        % APS, OUP, hard space '\eg\ NextWord'
% \newcommand{\etc}{{\em etc.}}     % etcetera in italics
% \newcommand{\ie}{{that is}}       % use Latin or English?  Decide later.
% \newcommand{\cf}{{cf.}}
% \newcommand{\eg}{{\it e.g.,\ }}   % Wirzba 2sep2001


%%%%%%%%%%%% REMOVE THIS EVENTUALLY %%%%%%%%%%%
%
