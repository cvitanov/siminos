% siminos/presentations/DS19/DS19.tex       latexpdf DS19; biber DS19
% $Author: predrag $ $Date: 2019-05-23 19:22:05 -0400 (Thu, 23 May 2019) $

% enter EDITS also in ../presentations/spatiotemp/spatiotemp.tex


%  started with .../Harvard19/Harvard19.tex                 2019-03-31
%  started with siminos/presentations/GTMAP18/GTMAP18.tex   2018-04-13
%  started with siminos/presentations/GTmath18/GTmath18.tex 2018-02-24
%  started with siminos/presentations/KITP17/UCSB17.tex     2017-01-26
%  started with siminos/presentations/Israel16/Israel16.tex 2016-08-17
%  started with siminos/presentations/GTmap16/GTmap16.tex   2016-08-17
%  started with talks/predrag/NBI16/NBI16.tex               2016-04-25
%  started with talks/predrag/RoySoc16/RoySoc16.tex         2016-04-25

\input ../../inputs/layoutBeamer
\usepackage[font=scriptsize, labelfont=bf]{caption}
\usepackage[
    backend=biber,  %bibtex,
    sorting=nyt,
    %refsection=chapter,
    %citereset=chapter,
    style=numeric, %alphabetic, % %style=authoryear,
    natbib=true,
    style=phys, % aps
    biblabel= brackets, % superscript, %
    articletitle=false, % true,  % false, % aps
    %chaptertitle=true,  % aip;  % false, % aps
    pageranges = true , % aip: the full range
             % = false, % aps: only the first page being printed
    sortlocale=en_US,
    firstinits=true,
    url=false, %true,  %
    doi=false, %true,
    eprint=false
]{biblatex}
\addbibresource{../../bibtex/siminos.bib}
\setbeamerfont{footnote}{size=\tiny}
\input{../../inputs/def} % no edits, always from dasbuch/book/inputs
\input{../../inputs/defsBeamer}
\renewcommand{\Ssym}[1]{{\ensuremath{m_{#1}}}}    % Boris
% \newcommand{\Ssym}[1]{{\ensuremath{s_{#1}}}}  % ChaosBook

% temporary fix, will not be needed in later MixTeX version         2018-11-02
% https://tex.stackexchange.com/questions/426088/texlive-pretest-2018-beamer-and-subfig-collide
\makeatletter
\let\@@magyar@captionfix\relax
\makeatother

\begin{document}


\title{
{a spatiotemporal theory of}
{\huge turbulence}
    \\
{in terms of exact coherent structures}
}
\author{P. Cvitanovi\'c}
\author[Cvitanovi\'c]
{
  \textcolor{green!50!black}{
  {Predrag~Cvitanovi\'c
  and
  Matt Gudorf
  }	%\inst{1}
  }
}
\institute
{
%  \inst{1}%
\HREF{meetings.siam.org/sess/dsp_programsess.cfm?SESSIONCODE=66488}
{SIAM - DS19 Dynamical Systems}
 }
\date{May 21, 2019}

\begin{frame}
  \titlepage
\end{frame}


\section[what this work is about]
 {what this work is about}

\begin{frame}{overview}
\begin{enumerate}
              \item {\Large
what this talk is about
                  }\textcolor{gray}{\small
              \item
turbulence in large domains
              \item
space is time
              \item
bye bye, dynamics
                    }
            \end{enumerate}
\end{frame}

\begin{frame}{how do clouds solve PDEs?}

\vfill

do clouds \textcolor{red}{integrate} Navier-Stokes equations?

\begin{center}
\centerline{\textcolor{red}{\Huge ?}}
%\end{center}
%for weather prediction, we store sets of weather sequences
%\bigskip\bigskip

%\begin{center}
\begin{minipage}[t]{\textwidth}
	\begin{center}
%\vspace{2ex}
\centerline{
\raisebox{-4.0ex}[5.5ex][4.5ex]
		 {\includegraphics[height=12ex]{Hopf-a}}
~~~ $\Longrightarrow$ ~~ {other swirls} ~~ $\Longrightarrow$ ~~~
	\raisebox{-4.0ex}[5.5ex][4.5ex]
		 {\includegraphics[height=12ex]{Hopf-b}}
          }
	\end{center}
\end{minipage}
\end{center}

are clouds Navier-Stokes supercomputers in the sky?

\end{frame}

\begin{frame}{part 1}
\begin{enumerate}
              \item {\Large
turbulence in large domains
                  }\textcolor{gray}{\small
              \item
space is time
              \item
spacetime
              \item
bye bye, dynamics
                    }
            \end{enumerate}
\end{frame}


\begin{frame}{goal : enumerate the building blocks of turbulence}
\begin{block}{Navier-Stokes equations} % (1822)}
\[
\dfrac{\partial \bv}{\partial t} + (\bv \cdot \nabla) \bv
	\,=\,
\frac{1}{R} \nabla ^2 \bv
-\nabla p
+ \mathbf{f}
    \,,\qquad
\nabla \cdot \bv = 0,
\]
\end{block}

\hfill{\small
velocity field  $\bv \in \mathbb{R}^3$
;
pressure field $p$
;
driving force $\mathbf{f}$
        }

\medskip

\begin{block}{describe turbulence}
starting from the equations (no statistical assumptions)
\end{block}

\bigskip

% large Reynolds number $R$:
\hfill {\Large\textcolor{red}{}}

\end{frame}

\begin{frame}{challenge : experiments are amazing}
\begin{center}
\includegraphics[width=0.7\textwidth]{mullin_puff2200} %pipe}
\end{center}
T. Mullin lab
\begin{center}
\bigskip
\includegraphics[width=0.7\textwidth]{deLHof09fig6} %pipe}
\end{center}
B. Hof lab
\end{frame}

%\begin{frame}{pipe theory and numerics}
%	\begin{columns}[t]
%	\column{.55\textwidth}
%amazing experiments! \\ amazing numerics! \\ beautiful visualizations !
%
%\bigskip\bigskip
%
%%relative periodic orbits,
%``Exact Coherent Structures'' :
%\\ numerical Navier-Stokes
%
%\medskip
%isosurfaces and cross sections \\ of the streamwise velocity
%
%\medskip
%
%\textcolor{red}{red} (\textcolor{blue}{blue}) streaks
%\\ are \textcolor{red}{faster} (\textcolor{blue}{slower}) \\ than the base flow
%
%\bigskip
%
%{{\tiny Ritter et al., Phys. Rev. Fluids (2018)}}
%
%	\column{.45\textwidth}
%\begin{center}
%  \includegraphics[width=1.0\textwidth,clip=true]
%                    {RZSEA18Fig3}
%\end{center}
%	\end{columns}
%\end{frame}

\section[dynamics in $\infty$ dimensions]
{dynamics in $\infty$ dimensions}

\begin{frame}{can simulate {\Huge large} computational domains}
\begin{center}
\includegraphics[width=1.0\textwidth]{AviHof13fig4}
\end{center}
pipe flow close to onset of turbulence
\footnote{M.~Avila and B.~Hof, {Phys. Rev. \bf E 87} (2013)}

\bigskip

but we have \textcolor{red}{\Huge hit a wall} :

\hfill exact coherent structures are too unstable to compute
\end{frame}

\begin{frame}{goal : we can do 3D turbulence, but for this talk}
\begin{block}{Navier-Stokes equations} % (1822)}
\[
\dfrac{\partial \bv}{\partial t} + (\bv \cdot \nabla) \bv
	\,=\,
\frac{1}{R} \nabla ^2 \bv + (\cdots)
% -\nabla p + \mathbf{f}
%    \,,\qquad
%\nabla \cdot \bv = 0,
\]
\end{block}

\hfill{\small velocity field  $\bv(\bx;t) \in \mathbb{R}^3$}

\medskip

\begin{block}{not helpful for developing intuition}
we cannot visualize 3D velocity field at every 3D spatial point
\end{block}

\bigskip

% large Reynolds number $R$:
\hfill {\Large\textcolor{red}{look instead at 1D `flame fronts'}}
\end{frame}

\begin{frame}{(1+1) spacetime dimensional ``Navier-Stokes''}
% conceptually not ready yet to explore \\ %the inertial manifold of
% $(1+3)$\dmn\ turbulence - start instead with

% \bigskip

\begin{block}{\KS\ $(1+1)$\dmn\ PDE}
\[
  u_t + u \triangledown u \,=\,
    {\color{red}-}\triangledown^2 u {\color{red}-\triangledown^4 u}
    \,,\qquad   x \in \mathbb{R}
    \,,
\]
\end{block}

\bigskip

describes spatially extended systems such as
\begin{itemize}
 \item flame fronts in combustion
 \item reaction-diffusion systems
 \item \ldots
\end{itemize}
\end{frame}

\begin{frame}{an example : \KS\ on a large domain}
\begin{center}
  \includegraphics[width=.6\textwidth]{MNG_uu500b500}
%  \includegraphics[width=0.6\textwidth] %,height=0.5\textheight,clip=true]
%  {ks_largeL_cbar_200} %{ksevol-fig} %{ks_largeL_cbar}
\end{center}

{\footnotesize
[horizontal] space $\ssp \in [0,L]$
\qquad
{[up]} time evolution
}

\begin{itemize}
\item turbulent behavior
\item simpler physical, mathematical and computational setting than Navier-Stokes
\end{itemize}
\end{frame}

\begin{frame}{compact space, infinite time cylinder}
\begin{center}
\includegraphics[width=0.9\textwidth]{cylinderTime}
\end{center}
% \hfill \color{red}{(impossible without xxx)}
so far : Navier-Stokes on compact spatial domains, all times
\end{frame}

\begin{frame}{compact space, infinite time} % \KS}
\begin{block}{\KS\ equation}
\[
  u_t \,=\,
    -({\color{red}+}\triangledown^2 +{\color{red}\triangledown^4}) u
    - u \triangledown u
    \,,\qquad   x \in [-L/2,L/2]
    \,,
\]
\end{block}

\bigskip

\begin{block}{in terms of discrete spatial Fourier modes}
$N$ ordinary differential equations (ODEs) in time
\[
\dot{\Fu}_k(\zeit) = ( q_k^2 - q_k^4 )\, \Fu_k(\zeit)
- i \frac{q_k}{2} \!\sum_{k'=0}^{N-1} \!\!\Fu_{k'}(\zeit) \Fu_{k-k'}(\zeit)
\,.
%\label{e-Fks}
\]
\end{block}
\end{frame}


\subsection{types of solutions}
\begin{frame}{part 2}
\begin{enumerate}
              \item
    \textcolor{gray}{\small
turbulence in large domains
        }
              \item
    {\Large
space is time
    }\textcolor{gray}{\small
              \item
spacetime
              \item
bye bye, dynamics
                    }
            \end{enumerate}
\end{frame}

\begin{frame}{yes, but}
\begin{center}
{\huge is space time?}
\end{center}
\end{frame}

\begin{frame}{can do : compact time, infinite space cylinder}
\begin{center}
\includegraphics[width=0.9\textwidth]{cylinderSpace}
\end{center}
% \hfill \color{red}{(impossible without xxx)}
\end{frame}

\begin{frame}{compact time, infinite space}
\begin{block}{ \KS\ $u(\conf, \zeit)$ as four fields, \\
1st order in spatial derivatives}
\bea
    u_\zeit &=&  - u u_\conf
    -u_{\conf \conf}-u_{\conf \conf \conf \conf}\,,
    % \label{e-ks}
\continue
    u_0 &\equiv& u \, , \quad
    u_1 \equiv u_{\conf} \, , \quad
    u_2 \equiv u_{\conf \conf} \, , \quad
    u_3 \equiv u_{\conf \conf \conf}
                        \nonumber
\eea
\end{block}

\begin{block}{evolve four 1st order PDEs $u_j(\zeit, \conf)$ in $\conf$,}
periodic boundary condition in time
              $u(\conf, \zeit) = u(\conf, \zeit + \period{})$
\bea
    u_{0,\conf} &=& u_1 \,,\quad
    u_{1,\conf}  =  u_2 \,,\quad
    u_{2,\conf}  =  u_3 \continue
    u_{3,\conf} &=& - u_{0,\zeit} - u_2 - u_0 u_1
                        \nonumber
\eea
\end{block}

\bigskip

initial
$u_0(\conf_0, \zeit)$,
$u_1( \conf_0, \zeit)$,
$u_2( \conf_0, \zeit)$,
$u_3( \conf_0, \zeit)$
    \\
specified for  $\zeit \in [0, \period{})$, at a fixed space point $\conf_0$
\end{frame}

\begin{frame}{a solution integrated in either time or space}
\begin{center}
  \begin{minipage}[height=.40\textheight]{.35\textwidth}
    \centering %\small{\texttt{(left)}}
    \includegraphics[width=\textwidth,height=.60\textheight]{MNGcomp32xint22}
  \end{minipage}
~~~~~~~~~
  \begin{minipage}[height=.40\textheight]{.35\textwidth}
    \centering %\small{\texttt{(right)}}
    \includegraphics[width=\textwidth,height=.60\textheight]{MNGcomp64xint22}
  \end{minipage}
\end{center}
    {\color{red}old} : time evolution~~~~~~~~~~~~~~~~~~~
    {\color{red}new} : space evolution
    \\
    $x=[0,L]$ %22]$,
       initial condition ~~~~~~~~~ time periodic line $t = [0,T]$
  %2\,T_{\PPO{10.2}})$
  %\label{fig:MNGcompxint2}

\vfill\hfill        Gudorf 2016
\end{frame}

\begin{frame}{but integrations are uncontrollably unstable}

\begin{center}
{\huge neither} time {\huge nor} space integration {\huge works} \\
for large domains
\end{center}

\bigskip \bigskip

we have \textcolor{red}{\Huge hit a wall} :

\hfill exact coherent structures are too unstable to compute

\vfill
\color{red}{rethink the formulation!}
\end{frame}

\begin{frame}{part 3}
\begin{enumerate}
              \item
    \textcolor{gray}{\small
turbulence in large domains
              \item
space is time
    }
              \item {\Large
spacetime
    }\textcolor{gray}{\small
              \item
bye bye, dynamics
                    }
            \end{enumerate}
\end{frame}

\begin{frame}
    \frametitle{\KS\ on a large spacetime domain}
\begin{block}{the same small tile recurs often in a turbulent pattern}
\includegraphics[width=.48\textwidth]{MNG_uu500b500co}
\includegraphics[width=.48\textwidth]{cutouts}
\end{block}
goal : define, enumerate nearly recurrent tiles
\end{frame}

\begin{frame}{use spatiotemporally compact solutions as spacetime `tiles'}
\begin{center}
\includegraphics[width=0.9\textwidth]{torusSpTime}
\end{center}
this `exact coherent structure'\\
\textcolor{red}{shadows} a small patch of spacetime solution $u( \conf, \zeit)$
% \hfill \color{red}{(impossible without xxx)}
\end{frame}

\begin{frame}{every compact solution is a fixed point on a discrete lattice}
discretize $u_{nm} = u(\conf_n,\zeit_m)$ over
$N M$ points of spatiotemporal periodic lattice $\conf_n = n L/N$,
 $\zeit_m = m \period{}/M$, Fourier transform :
%\beq
%    \Fu_{k,m}^{(i)} = \frac{1}{M} \sum_{\ell = 0}^{M-1}
%    \Fu^{(i)}_{k,\ell} e^{i \omega_\ell \zeit_m}
%    \, , \quad
%    \Fu^{(i)}_{k,\ell} = \sum_{m=0}^{M-1}\Fu_{k,m}^{(i)}  e^{-i \omega_\ell \zeit_m}
%    \, , \quad
%    \mbox{where }
%    \omega_\ell = 2 \pi \ell / \period{} \, .
%\eeq
%
\[
\Fu_{k\ell} \,=\,
  \frac{1}{NM} \sum^{N-1}_{n=0} \sum^{M-1}_{m=0}
  u_{nm} \, e^{-i(q_k\conf_n + \omega_\ell \zeit_m)}
    \,,\quad
q_k = \frac{2 \pi k}{L}
    \,,\;
\omega_\ell = \frac{2 \pi \ell}{\period{}}
% \label{spattempFT}
\]
\KS\ is no more a PDE, \\
but an algebraic $[N\!\times\!M]$\dmn\ fixed point problem \\
of determining a solution to
\[
\left[- i \omega_\ell - ( q_k^2 - q_k^4 ) \right]\Fu_{k\ell}
+ i \frac{q_k}{2} \!\sum_{k'=0}^{N-1} \sum^{M-1}_{m'=0}\!\!
\Fu_{k'm'} \Fu_{k-k',m-m'}
    =
0
%\,.
%\label{e-FksSpattemp}
\]
\end{frame}

\begin{frame}{every calculation is a spatiotemporal lattice calculation}
field is discretized as
$\Fu_{k\ell}$ values  \\ over
$N M$ points of a periodic lattice

%\medskip

\begin{center}
\includegraphics[width=0.9\textwidth]{torusSpTime}
\end{center}
% \hfill \color{red}{(impossible without xxx)}
\end{frame}


\begin{frame}{there is no more time evolution}
solution to \KS\ is now given as
\begin{block}{condition that}
at each lattice point $k\ell$ \\
the tangent field at $\Fu_{k\ell}$
\end{block}
satisfies the equations of motion
\[
\left[- i \omega_\ell - ( q_k^2 - q_k^4 ) \right]\Fu_{k\ell}
+ i \frac{q_k}{2} \!\sum_{k'=0}^{N-1} \sum^{M-1}_{m'=0}\!\!
\Fu_{k'm'} \Fu_{k-k',m-m'}
    =
0
%\,.
%\label{e-FksSpattemp}
\]

\bigskip

this is a \textcolor{red}{local} tangent field constraint on a \textcolor{red}{global} solution

\bigskip

robust : no exponential instabilities as there are no finite time / space integrations

\end{frame}

\begin{frame}{part 4}
\begin{enumerate}
              \item
    \textcolor{gray}{\small
turbulence in large domains
              \item
space is time
              \item
spacetime    }
              \item {\Large
spacetime computations
    }\textcolor{gray}{\small
              \item
bye bye, dynamics
                    }
            \end{enumerate}
\end{frame}

\begin{frame}{how do clouds solve PDEs?}
clouds do not \textcolor{red}{\Huge NOT} {integrate} Navier-Stokes equations

\bigskip\bigskip

\begin{center}
\begin{minipage}[t]{\textwidth}
	\begin{center}
\centerline{
\raisebox{-4.0ex}[5.5ex][4.5ex]
		 {\includegraphics[height=12ex]{Hopf-a}}
~~~ $\Longrightarrow$ ~~ {other swirls} ~~ $\Longrightarrow$ ~~~
	\raisebox{-4.0ex}[5.5ex][4.5ex]
		 {\includegraphics[height=12ex]{Hopf-b}}
          }
	\end{center}
\end{minipage}
\end{center}

do clouds satisfy Navier-Stokes equations?

\bigskip

{\Large yes!}

\centerline{
\textcolor{blue}{they satisfy them \textcolor{red}{\large locally}, everywhere and at all times}
}
\end{frame}

\begin{frame}{the equations imposed as local constraints}
\begin{block}{\KSe}
\[
F(u) = u_t + u_{\conf \conf} + u_{\conf \conf \conf \conf} + u u_{\conf} = 0
\]
\end{block}
\bigskip\bigskip
for example, minimize over the entire 2-torus
\begin{block}{cost function}
\[
G \equiv \frac{1}{2} |F(u)|^2_{L \times T}
%\ee{costfunctional}
\]
\end{block}
\vfill\hfill\textcolor{red}{\Huge need your help !}
\end{frame}

\begin{frame}{Adjoint descent}
cost function
\[
  G = \frac{1}{2} \mathbf{F}^{\top}\mathbf{F}
  \,.
\]
introduce fictitious time ($\tau$) flow by differentiation of cost function.
\[
  \partial_{\tau}G = (J^{\top}\mathbf{F})^{\top}(\partial_{\tau}\mathbf{x})
\]
  ``adjoint descent'' method defined by chosing\footfullcite{Faraz15}
\[
  \partial_{\tau}\mathbf{x} = -(J^{\top}\mathbf{F})
\]

\end{frame}

\begin{frame}{initial guess generation ?}
 \textcolor{blue}{the time scale} : the shortest
`turnover' scale (?)

\bigskip

\textcolor{blue}{the spatial scale} :
$\bar{L}=2\pi\sqrt{2}$, the  most unstable spatial wavelength of the \KS

\bigskip

%%%%%%%%%%%%%%%%%%%%%%%%%%%%%%%%%%%%%%%%%%%%%%%%%%%%%%%%%%%%%%%%
%\label{fig:MNG_spacetime_smoothed} from siminos/spatiotemp/blogMNG.tex
\begin{minipage}[height=.32\textheight]{.30\textwidth}
\includegraphics[width=\textwidth,height=.32\textheight]{MNG_T100L44_init}
\end{minipage}

\medskip
initial : spatial $\bar{L}$-modulated random guess
%%%%%%%%%%%%%%%%%%%%%%%%%%%%%%%%%%%%%%%%%%%%%%%%%%%%%%%%%%%%%%%%

% \vfill\hfill        Gudorf 2018
\end{frame}

\begin{frame}{KS \twot\ found variationally}
%%%%%%%%%%%%%%%%%%%%%%%%%%%%%%%%%%%%%%%%%%%%%%%%%%%%%%%%%%%%%%%%
%\label{fig:MNG_spacetime_smoothed} from siminos/spatiotemp/blogMNG.tex
\begin{minipage}[height=.72\textheight]{.40\textwidth}
\centering \small{\texttt{(left)}}
\includegraphics[width=\textwidth,height=.63\textheight]{MNG_T100L44_init}
\end{minipage}
\begin{minipage}[height=.72\textheight]{.40\textwidth}
\centering \small{\texttt{(right)}}
\includegraphics[width=\textwidth,height=.63\textheight]{MNG_T100L44_final}
\end{minipage}

\vfill
(left) initial : $\bar{L}=2\pi\sqrt{2}$ spatially modulated ``noisy'' guess

(right) adjoint descent : converged \twot\
% \\
% spatial and time periods
% $L=24.07$, $T=31.86$
%%%%%%%%%%%%%%%%%%%%%%%%%%%%%%%%%%%%%%%%%%%%%%%%%%%%%%%%%%%%%%%%

\vfill\hfill        Gudorf 2018
\end{frame}

\begin{frame}{initial guesses, embedded in ergodic sea?}
\begin{block}{Historically, }
guesses are extracted from close recurrences \\
observed in long turbulent simulations
\end{block}
\bigskip\bigskip
            \begin{enumerate}
              \item
inefficient, find only the shortest, least unstable orbits%
\footfullcite{pchaot}$^,$\footfullcite{GHCW07}
              \item
cannot integrate that far in time
            \end{enumerate}

\vfill\hfill\textcolor{red}{\huge need spatiotemporal guesses}
\end{frame}

\begin{frame}{guesses extracted from large spacetime domains}
\begin{minipage}[height=.45\textwidth]{.45\textwidth}
\centering %\small{\texttt{(a)}}
\includegraphics[width=\textwidth,height=.45\textheight]{MNGadjdescent500b500init}
\end{minipage}
\begin{minipage}[height=.45\textwidth]{.45\textwidth}
\centering %\small{\texttt{(b)}}
\includegraphics[width=\textwidth,height=.45\textheight]{MNGadjdescent500b500fin}
\end{minipage}
%\caption{ \label{fig:MNGlarge_adjointdescent}

(left) random initial state on
%\hfill
$(L,T)=(500,500)$

(right) adjoint descent $\to$ typical \KS\ state

\vfill\hfill
finite windows are starting guesses for \twots
\end{frame}

\begin{frame}{embarrassment of riches}
\begin{center}
{\huge what do do?}
\end{center}

\vfill

{\Large Matt Gudorf}

\medskip

\hfill has 1\,000's of such \twots

\vfill

{\scriptsize see Matt's poster, Session PP2  \hfill Wednesday, May 22}
\end{frame}


\begin{frame}{part 5}
\begin{enumerate}
              \item
    \textcolor{gray}{\small
turbulence in large domains
              \item
space is time
              \item
spacetime
    }
              \item {\Large
fundamental tiles
    }\textcolor{gray}{\small
              \item
bye bye, dynamics
                    }
            \end{enumerate}
\end{frame}


\begin{frame}{building blocks of turbulence}

how do we \textcolor{red}{recognize} a cloud?

\bigskip
\begin{center}
\centerline{\textcolor{red}{\Huge WATCH}}
%\end{center}
%for weather prediction, we store sets of weather sequences
%\bigskip\bigskip

%\begin{center}
\begin{minipage}[t]{\textwidth}
	\begin{center}
%\vspace{2ex}
\centerline{
\raisebox{-4.0ex}[5.5ex][4.5ex]
		 {\includegraphics[height=12ex]{Hopf-a}}
~~~ $\Longrightarrow$ ~~ {other swirls} ~~ $\Longrightarrow$ ~~~
	\raisebox{-4.0ex}[5.5ex][4.5ex]
		 {\includegraphics[height=12ex]{Hopf-b}}
          }
	\end{center}
\end{minipage}
\end{center}

\bigskip

{\Large by recurrent shapes!}

\vfill

\centerline{
\textcolor{blue}{so, construct an \textcolor{red}{\Large alphabet} of possible shapes}
}
\end{frame}

\begin{frame}{extracting a fundamental tile}
\begin{minipage}[height=.60\textheight]{.24\textheight}
\centering % \small{\texttt{(a)n}}
\includegraphics[width=.3\textheight,height=.55\textheight]{MNG_gapfull}
\end{minipage}
    $\quad\Rightarrow\quad$
\begin{minipage}[height=.60\textheight]{.24\textheight}
\centering % \small{\texttt{(b)}}
\includegraphics[width=.3\textheight,height=.25\textheight]{MNG_gapsub1}
\end{minipage}
    $\quad\Rightarrow\quad$
\begin{minipage}[height=.60\textheight]{.18\textheight}
\centering             % \small{\texttt{(c)}}
\includegraphics[width=0.20\textheight,height=.21\textheight]{ksstFigs/MNG_gap}
\end{minipage}
%    \quad\Rightarrow\quad
%\begin{minipage}[height=.60\textheight]{.12\textheight}
%\centering % \small{\texttt{(d)}}
%\includegraphics[width=0.17\textheight,height=.21\textheight]{MNG_gap_final}
%\end{minipage}
%\label{fig:MNG_catseyefigs}

1) \twot\ %full\_L26.7\_T54.
    \\
2) \twot\ computed from initial guess cut out from 1)
    \\
3) ``gap" \twot, % po\_L17.3\_T15.3  %\reffig{fig:MNG_ppo_subdomains},
     cut out from 2)
%     \\
%4) the ``gap"  \emph{prime}  \twot\ fundamental domain
\end{frame}

\begin{frame}%[allowframebreaks]
  \frametitle{a (too) simple set of prime (rubber) tiles}
  \begin{block} {an alphabet of \KS\ fundamental tiles}
\begin{minipage}[height=.60\textheight]{.25\textheight}
\centering             % \small{\texttt{(c)}}
\includegraphics[width=0.26\textheight,height=.27\textheight]{ksstFigs/MNG_defect}
\end{minipage} \qquad
  % \includegraphics[width=.2\textwidth]{MNG_hook}
  % \includegraphics[width=.2\textwidth]{MNG_half}
\begin{minipage}[height=.60\textheight]{.25\textheight}
\centering     %{\footnotesize\texttt{\quad\quad Gap}}
\includegraphics[width=0.26\textheight,height=.27\textheight]{ksstFigs/MNG_gap}
\end{minipage} \qquad
\begin{minipage}[height=.60\textheight]{.25\textheight}
\centering             % \small{\texttt{(c)}}
\includegraphics[width=0.25\textheight,height=0.25\textheight]{ksstFigs/MNG_streak}
\end{minipage}
  \end{block}
\vfill
utilize also discrete symmetries : \\
spatial reflection, spatiotemporal shift-reflect,
$\cdots$
\end{frame}

\begin{frame}{spacetime tiled by a small tile}
  \includegraphics[width=0.7\textwidth] %,height=0.5\textheight,clip=true]
  {MNG_tiling_ppo_L55p83_T24}
  \vfill
tiling by % shift-reflect
\twot\ \\ $(L,T)=(55.83,24)$
\end{frame}

\begin{frame}{any particular tiling looks nothing like \\ turbulent \KS!}
  \includegraphics[width=0.7\textwidth] %,height=0.5\textheight,clip=true]
  {MNG_large_ergodic}

{\footnotesize
[horizontal] space $\ssp \in [-L/2,L/2]$
\qquad
{[up]} time evolution
}
\end{frame}

\begin{frame}{part 5}
\begin{enumerate}
              \item
    \textcolor{gray}{\small
turbulence in large domains
              \item
space is time
              \item
spacetime
              \item
fundamental tiles
    }
              \item {\Large
gluing tiles
    }\textcolor{gray}{\small
              \item
bye bye, dynamics
                    }
            \end{enumerate}
\end{frame}


\begin{frame}%[allowframebreaks]
  \frametitle{a qualitative tiling guess}
  \begin{block} {a guess tiling and the resulting solution}
  \qquad
\begin{minipage}[height=.80\textheight]{.25\textheight}
\centering             % \small{\texttt{(c)}}
  \qquad\qquad
\includegraphics[width=0.25\textheight,height=.36\textheight]{MNG_ppo_frankenstein}
\end{minipage} \quad\quad
\begin{minipage}[height=.80\textheight]{.25\textheight}
\centering      {\footnotesize\texttt{\quad\quad 2-torus}}
\includegraphics[width=0.35\textheight,height=.45\textheight]{MNG_ppo_L30_T44}
\end{minipage} \qquad\qquad
  \end{block}
\end{frame}

\begin{frame}{turbulence.zip : each solution has a unique symbolic name}
  \begin{block} {symbolic dynamics is 2\dmn!}
  \begin{center}
  \includegraphics[width=.25\textwidth,height=.4\textheight]{MNG_ppo_frankenstein}
~~~\includegraphics[width=.25\textwidth,height=.4\textheight]{MNG_approxsymbdyn}
~~~\includegraphics[width=.25\textwidth,height=.4\textheight]{MNG_approxsymbdyntrinary}
%  \includegraphics[width=.3\textwidth,height=.3\textheight]{MNG_ppo_L30_T44}
  \end{center}
  \end{block}
\begin{itemize}
 \item each symbol indicates a corresponding spatiotemporal tile
 \item these are ``rubber'' tiles
\end{itemize}
\end{frame}

%%%%%%%%%%%%%%%%%%%%%%%%%%%%%%%%%%%%%%%%%%%%%%%%%%%%%%%%%%%%%%%%
\begin{frame}{part 5}
\begin{enumerate}
              \item
    \textcolor{gray}{\small
turbulence in large domains
              \item
space is time
    }
              \item
    {\Large
bye bye, dynamics
    }
            \end{enumerate}
\end{frame}

\begin{frame}{in future there will be no future}
\begin{center}
{\huge goodbye}
\end{center}

\vfill

to long time and/or space integrators

\medskip

\hfill they never worked and could never work
\end{frame}

\begin{frame}{life outside of time}
\textcolor{red}{the trouble}:

forward time-integration codes too unstable

\bigskip
\bigskip

\textcolor{blue}{multishooting inspiration}:
 replace a guess that a  \textcolor{blue}{point} is on the periodic
orbit by a guess of the \textcolor{blue}{entire orbit}.

\bigskip

$\to$

\bigskip

spatio-temporally periodic solutions of classical field theories
can be found by \textcolor{blue}{variational methods}
\end{frame}

\begin{frame}{the equations solved as global optimization problems}
\begin{block}{impose the equations as local constraints}
\[
F(u) = u_t + u_{\conf \conf} + u_{\conf \conf \conf \conf} + u u_{\conf} = 0
\]
\end{block}
\bigskip\bigskip
minimize globally
\begin{block}{perhaps using cost function}
\[
G \equiv \frac{1}{2} |F(u)|^2_{L \times T}
%\ee{costfunctional}
\]
\end{block}
\end{frame}

\begin{frame}{can computers}

\vfill

{\Huge
do this ?
                  }

\vfill

\end{frame}

\begin{frame}{machine learning will be needed}
``reservoir computing'' example\footfullcite{PHGLO18}

\bigskip

 \begin{columns}
 \column{0.35\textwidth}
~~~~~~\includegraphics[width=1\textwidth]{PHGLO18fig2}
 \column{0.65\textwidth}
 \begin{itemize}
 \item[(a)] data: \\ \KS\ simulation
 \item[(b)] reservoir computing prediction
 \item[(c)] two subtracted agree to \\
            $\sim 5$ Lyapunov times
 \end{itemize}
 \end{columns}
\vfill {\Large Q : \textcolor{red}{how would you learn this data?}}
\end{frame}

\begin{frame}{the answer is}

\vfill

{\Huge
scalability
                  }

\vfill

%\hfill in the spirit of this workshop
\end{frame}

\begin{frame}{compute locally, adjust globally}
\begin{block}{Navier-Stokes codes}
\begin{itemize}
 \item
T. Schneider : developing a matrix-free variational Navier-Stokes code,
machine learning initial guesses
 \item
D. Lasagna and A. Sharma  : developing variational adjoint solvers to
find periodic orbits with long periods
 \item
Q. Wang : parallelizing {\color{red}spatiotemporal}
computation is FLOPs intensive, but more robust than
integration forward in time
\end{itemize}
\end{block}

\vfill\hfill
it's rocket science%
\footfullcite{WGBGQ13}$^,$%
\footfullcite{Schneider19}$^,$%
\footfullcite{LaShMe18}
%\footnote{{\tiny Q. Wang \etal,}\HREF{https://doi.org/10.1063/1.4819390}
%{\tiny\em Towards scalable parallel-in-time turbulent flow simulations},
%{\tiny Physics of Fluids (2013)}}
\end{frame}

\begin{frame}{
towards scalable parallel-in-time turbulent flow simulations
}
\begin{block}{future :}%
processor speed $\to$ limit

\medskip

number of cores $\to 10^6 \to \cdots$

\medskip
\end{block}

\emph{Q. Wang% et al (2013)
    \,\footfullcite{WGBGQ13}%$^,$\footfullcite{Ihara66}
    :} % Gomez, Blonigan, Gregory and Qian 2013 :}

next-generation : spacetime parallel
simulations, \\
on discretized $4D$ spacetime
computational domains, \\
with each computing core handling a spacetime lattice cell

\bigskip

compared to time-evolution solvers:
significantly higher level of concurrency, reduction the ratio of
inter-core communication to floating point operations

\bigskip

$\qquad\qquad\Rightarrow$ a path towards exascale DNS of turbulent flows
\end{frame}

\begin{frame}{construct hierarchically larger and largerspatiotemporal patterns }
  \begin{block} {2D symbolic encoding $\Rightarrow$ admissible solutions}
  \begin{center}
\includegraphics[width=.25\textwidth,height=.38\textheight]{MNG_approxsymbdyntrinary}
~~~\includegraphics[width=.25\textwidth,height=.38\textheight]{MNG_ppo_frankenstein}
~~~\includegraphics[width=0.35\textheight,height=.45\textheight]{MNG_ppo_L30_T44}
  \end{center}
  \end{block}
\begin{itemize}
 \item each symbol indicates a minimal spatiotemporal tile
 \item glue them in all admissible ways
\end{itemize}
\end{frame}


\begin{frame}{take home : clouds do not integrate PDEs}

%and do they care what PST hour is it?
%\\
%and at what longitude are they?
%\\
do clouds integrate Navier-Stokes equations?

\begin{center}
\centerline{\textcolor{red}{\Huge NO!}}
%\end{center}
%for weather prediction, we store sets of weather sequences
%\bigskip\bigskip

%\begin{center}
\begin{minipage}[t]{\textwidth}
	\begin{center}
%\vspace{2ex}
\centerline{
\raisebox{-4.0ex}[5.5ex][4.5ex]
		 {\includegraphics[height=12ex]{Hopf-a}}
~~~ $\Longrightarrow$ ~~ {other swirls} ~~ $\Longrightarrow$ ~~~
	\raisebox{-4.0ex}[5.5ex][4.5ex]
		 {\includegraphics[height=12ex]{Hopf-b}}
          }
	\end{center}
\end{minipage}
\end{center}

at any spacetime point Navier-Stokes equations describe the local tangent space

\bigskip

\centerline{
\textcolor{blue}{they satisfy them \textcolor{red}{\large locally}, everywhere and at all times}
}
\end{frame}


\begin{frame}{summary}
\begin{enumerate}
              \item
study turbulence in infinite spatiatemporal domains
              \item
theory : classify all spatiotemporal tilings
              \item
numerics : future is spatiotemporal
\end{enumerate}

\vfill

there is no more time

\medskip

there is only enumeration of spacetime solutions
\end{frame}

\begin{frame}{spatiotemporally infinite \catlatt}
%\begin{center}
%\hfill\includegraphics[width=0.55\textwidth]{spatiotempCat}
\hfill\includegraphics[width=0.55\textwidth]{DawnBishopCats}
%\end{center}
\end{frame}

\begin{frame}{\po s generalize to $d$-tori}

\begin{block}{1 time, 0 space dimensions}
a {\statesp} point is {\em periodic} if its orbit returns to it
after a finite time \period{} ;
\\
such orbit tiles the time axis
by infinitely many repeats
\end{block}

\bigskip

\begin{block}{1 time, $d$-1 space dimensions}
 a {\statesp} point is {\em spatiotemporally periodic} if
it belongs to \\ an invariant $d$-torus ${\R}$ ;
\\
such torus tiles the spacetime
by infinitely many repeats
%,\\
%\ie, a \brick\ $\Mm_{\R}$ that
%tiles the lattice state  $\Mm$, \\
%with period $\ell_j$ in $j$th lattice direction
\end{block}
\end{frame}


%\begin{frame}{XXX}
%\end{frame}


\end{document}
