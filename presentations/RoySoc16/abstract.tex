
Periodic orbits theory of turbulent flows

    Predrag Cvitanović
    Center for Nonlinear Science, School of Physics
    Georgia Institute of Technology
    Atlanta, GA 30332-0430, USA

Partial differential equations are in principle infinite-dimensional
dynamical systems. However, recent studies offer strong numerical
evidence that the turbulent solutions of spatially
extended dissipative systems evolve within a manifold spanned by a
finite number of `entangled' modes, dynamically isolated from
the residual set of isolated, transient degrees of freedom. The initial
studies, based on numerical simulations of long ergodic trajectories,
yield no intuition about the geometry of such attractors. That is
attained by studying the hierarchies of unstable periodic orbits, invariant
solutions which, together with their Floquet vectors, provide an
effective description  of both the local hyperbolicity and the global
geometry of an attractor embedded in a high-dimensional state space.

Dynamical systems with translational or rotational symmetry arise
frequently in studies of spatially extended physical systems, such as
Navier-Stokes flows on periodic domains, with each fluid state having an
infinite number of equivalent solutions obtained from it by a translation
or a rotation. This multitude of equivalent solutions tends to obscure
the dynamics of turbulence, and the crucial step in the analysis of such
a system is symmetry reduction. We offer several implementations of
`method of slices' applicable to very high-dimensional problems and show
that after application of the method, hitherto unseen global structures,
for example to pipe flow, relative periodic orbits and their unstable
manifolds are uncovered.

Whether the periodic orbit theory of computing expectation values of
measurable observables is applicable to such high-dimensional flows
remains an open question.
