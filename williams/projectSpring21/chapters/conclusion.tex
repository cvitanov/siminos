\svnkwsave{$RepoFile: spatiotemp/conclusion.tex $}
\svnidlong {$HeadURL: svn://zero.physics.gatech.edu/siminos/spatiotemp/conclusion.tex $}
{$LastChangedDate: 2021-01-22 16:12:00 -0500 (Fri, 22 Jan 2021) $}
{$LastChangedRevision: 6161 $} {$LastChangedBy: predrag $}
\svnid{$Id: conclusion.tex 6161 2017-12-07 21:13:06Z predrag $}

In this project, we have investigated the effectiveness and accuracy in
finding spatially periodic equilibrium solutions in the one\dmn\ \KSe\
using a finite differencing method employed by Michelson. A periodic
orbit with parameters $c = 1.266$ and $s = 1.369$ was found with an
independently written python script that agrees with one solution listed
in Michelson\rf{Mks86}. The current method only has the ability to search through
a range of parameter values, returning a plot of the $(x,y)$ curve
generated.
Although this iterative process can find solutions to the \KSe, the
likelihood of locating a spatially periodic equilibrium is dependent on
the range and stepsize taken for the parameter sweep and a thorough
search can be highly computationally intensive.

The second half of this project explored the derivation of linear operators on the \KSe\ for the quasi Newton-correction method. The temporal and spatial parameters $\omega$ and $\nu$, respectively, are used in the initial form of the \KSe, ultimately arriving at expressions for the parameters in terms of the solution itself.

Neither of the two methods explored in this project involved the use of spectral methods through Fourier transforms. The processes outlined here are alternatives for finding spatially periodic equilibrium in the \KS\ system.

\subsection{Future work}

As described in \refChap{chap:KSmichelson}, the finite differencing search process for solutions to the \KSe\ is entirely qualitative and requires visual inspection of each curve produced by each unique set of parameters $c$ and $s$. To automate this process, the python script can be modified to sweep across an interval with a modest stepsize. When the search process encounters a set of parameters that produces a significant number of periodic cycles, say four or more repeated crest and peak patterns in the $(x,y)$ plot, a finer search in the neighborhood of the parameter set will be carried out. Furthermore, the automated process can then be modified to search for temporally periodic solutions in the \KS\ system.

For the Yang method discussed in \refChap{chap:KSyang}, the procedure for deriving the linear operators from the expressions for the parameters will need to be completed before applying the quasi Newton-correction method.
